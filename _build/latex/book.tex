%% Generated by Sphinx.
\def\sphinxdocclass{jupyterBook}
\documentclass[letterpaper,10pt,english]{jupyterBook}
\ifdefined\pdfpxdimen
   \let\sphinxpxdimen\pdfpxdimen\else\newdimen\sphinxpxdimen
\fi \sphinxpxdimen=.75bp\relax
\ifdefined\pdfimageresolution
    \pdfimageresolution= \numexpr \dimexpr1in\relax/\sphinxpxdimen\relax
\fi
%% let collapsible pdf bookmarks panel have high depth per default
\PassOptionsToPackage{bookmarksdepth=5}{hyperref}
%% turn off hyperref patch of \index as sphinx.xdy xindy module takes care of
%% suitable \hyperpage mark-up, working around hyperref-xindy incompatibility
\PassOptionsToPackage{hyperindex=false}{hyperref}
%% memoir class requires extra handling
\makeatletter\@ifclassloaded{memoir}
{\ifdefined\memhyperindexfalse\memhyperindexfalse\fi}{}\makeatother

\PassOptionsToPackage{warn}{textcomp}

\catcode`^^^^00a0\active\protected\def^^^^00a0{\leavevmode\nobreak\ }
\usepackage{cmap}
\usepackage{fontspec}
\defaultfontfeatures[\rmfamily,\sffamily,\ttfamily]{}
\usepackage{amsmath,amssymb,amstext}
\usepackage{polyglossia}
\setmainlanguage{english}



\setmainfont{FreeSerif}[
  Extension      = .otf,
  UprightFont    = *,
  ItalicFont     = *Italic,
  BoldFont       = *Bold,
  BoldItalicFont = *BoldItalic
]
\setsansfont{FreeSans}[
  Extension      = .otf,
  UprightFont    = *,
  ItalicFont     = *Oblique,
  BoldFont       = *Bold,
  BoldItalicFont = *BoldOblique,
]
\setmonofont{FreeMono}[
  Extension      = .otf,
  UprightFont    = *,
  ItalicFont     = *Oblique,
  BoldFont       = *Bold,
  BoldItalicFont = *BoldOblique,
]



\usepackage[Bjarne]{fncychap}
\usepackage[,numfigreset=1,mathnumfig]{sphinx}

\fvset{fontsize=\small}
\usepackage{geometry}


% Include hyperref last.
\usepackage{hyperref}
% Fix anchor placement for figures with captions.
\usepackage{hypcap}% it must be loaded after hyperref.
% Set up styles of URL: it should be placed after hyperref.
\urlstyle{same}


\usepackage{sphinxmessages}



        % Start of preamble defined in sphinx-jupyterbook-latex %
         \usepackage[Latin,Greek]{ucharclasses}
        \usepackage{unicode-math}
        % fixing title of the toc
        \addto\captionsenglish{\renewcommand{\contentsname}{Contents}}
        \hypersetup{
            pdfencoding=auto,
            psdextra
        }
        % End of preamble defined in sphinx-jupyterbook-latex %
        

\title{Python in Plain Terms}
\date{Feb 14, 2024}
\release{}
\author{Yusuf Danisman}
\newcommand{\sphinxlogo}{\vbox{}}
\renewcommand{\releasename}{}
\makeindex
\begin{document}

\pagestyle{empty}
\sphinxmaketitle
\pagestyle{plain}
\sphinxtableofcontents
\pagestyle{normal}
\phantomsection\label{\detokenize{ppt::doc}}


\sphinxAtStartPar
\sphinxincludegraphics{{meryem2}.png}

\sphinxAtStartPar
Many people often ask me how to learn Python. They come from different backgrounds, ages, and education levels, with most having no coding experience. They’re seeking a resource that’s easy to understand without getting into technical jargon. That’s the purpose of this online book. It explains Python concepts using simple language so that anyone, regardless of their field or expertise, can grasp it. After using this book, you’ll be able to understand and write medium\sphinxhyphen{}sized Python code.


\bigskip\hrule\bigskip


\sphinxAtStartPar
There is a separate section on how to use Google Colaboratory to write and execute Python code. However, the code in this book can be used in any major Python editor. It is easy to use Google Colab because you do not need to install any program on your computer. It is also possible to write your code on a tablet and store it in your Google Drive. Also, Google Colab makes it possible to share your code and work on code as a group.


\bigskip\hrule\bigskip


\sphinxAtStartPar
Each chapter of this book covers a Python subject along with optional items. You can skip the parts that might be too detailed for you. However, it is essential to write and execute all codes in this book yourself. Whenever you see code in this book, try to write it in your code editor and execute it. If possible, make some small changes to check your understanding. You can copy/paste the code, but for beginners, it is better to write the code yourself.


\bigskip\hrule\bigskip


\sphinxAtStartPar
Additionally, the book includes debugging, output, and coding questions tailored for different proficiency levels.
\begin{itemize}
\item {} 
\sphinxAtStartPar
Debugging: Identify errors within a given code.
\begin{itemize}
\item {} 
\sphinxAtStartPar
These questions help you understand the syntax better and have an idea about the most common mistakes.

\item {} 
\sphinxAtStartPar
In these questions, first try to find mistakes, then think about how to solve them.

\item {} 
\sphinxAtStartPar
It is strongly recommended to write the correct code and execute it just to be sure it works well.

\end{itemize}

\item {} 
\sphinxAtStartPar
Output: Determine the output produced by a given code.
\begin{itemize}
\item {} 
\sphinxAtStartPar
By solving these kinds of questions, you will be able to understand each small detail of a code.

\item {} 
\sphinxAtStartPar
The best way to solve these questions is to work on paper and not look at the solution until being sure about your answer.

\item {} 
\sphinxAtStartPar
Also, it is a great practice to try to change this code and write it in a different, and if possible, better or shorter way.

\end{itemize}

\item {} 
\sphinxAtStartPar
Coding: Write code for specific tasks.
\begin{itemize}
\item {} 
\sphinxAtStartPar
This is the most important part to practice your coding skills with various types of questions.

\item {} 
\sphinxAtStartPar
The key point here is not looking at the solutions until having a solution, which might be even a partial one.

\item {} 
\sphinxAtStartPar
Please try to solve the questions in these sections by using the information only in the preceding chapters.

\item {} 
\sphinxAtStartPar
Keep in mind that there might be more than one solution for such kind of questions. The solutions in this book might not be the best ones because simple solutions have been chosen for this textbook.

\end{itemize}

\end{itemize}


\bigskip\hrule\bigskip


\sphinxAtStartPar
This book will be updated very often, and a project chapter will be added soon.

\sphinxAtStartPar
This book is prepared by using Jupyter Book. Check out \sphinxhref{https://jupyterbook.org}{the Jupyter Book documentation} for more information.

\sphinxAtStartPar
\sphinxstylestrong{Yusuf Danisman}, the author, is a faculty member in the Mathematics and Computer Science Department at Queensborough Community College, CUNY. He obtained his PhD in Mathematics from The Ohio State University and worked at the University of Oklahoma before joining CUNY. He teaches Mathematics, Statistics, Computer Science, and Data Science courses. Dr. Danisman is currently researching Financial Machine Learning and Geometric Deep Learning. He is also a steering committee member of the \sphinxhref{https://nebigdatahub.org/}{Northeast Big Data Innovtaion Hub}.

\sphinxstepscope


\chapter{Acknowledgments}
\label{\detokenize{acknowledge:acknowledgments}}\label{\detokenize{acknowledge::doc}}
\sphinxAtStartPar
For Meryem and Sumeyra,

\sphinxAtStartPar
I am deeply grateful to \sphinxstylestrong{Hamza Ali Danisman} for his careful review and helpful comments during the preparation of this textbook.
I also want to thank the QCC STEM program and its director, \sphinxstylestrong{Anna Lee}, for her technical support during the preparation of this book.

\sphinxstepscope


\chapter{Python Introduction}
\label{\detokenize{python:python-introduction}}\label{\detokenize{python::doc}}\begin{itemize}
\item {} 
\sphinxAtStartPar
\sphinxhref{https://www.python.org/}{Python} is one of the most commonly used programming languages.

\item {} 
\sphinxAtStartPar
It was created by Dutch programmer \sphinxhref{https://en.wikipedia.org/wiki/Guido\_van\_Rossum}{Guido van Rossum}
in the late 1980s.

\item {} 
\sphinxAtStartPar
He named it Python inspired by the BBC show \sphinxhref{https://en.wikipedia.org/wiki/Monty\_Python\%27s\_Flying\_Circus}{Monty Python’s Flying Circus}.

\end{itemize}


\section{Advantages}
\label{\detokenize{python:advantages}}\begin{enumerate}
\sphinxsetlistlabels{\arabic}{enumi}{enumii}{}{.}%
\item {} 
\sphinxAtStartPar
Beginner\sphinxhyphen{}friendly: Pyhton’s syntax is simple and easy to read and write.
\begin{itemize}
\item {} 
\sphinxAtStartPar
Example: Utilizing indentation (spaces) instead of parentheses results in clean code.

\item {} 
\sphinxAtStartPar
The following two code checks whether the integer 15 is an even or odd number.

\item {} 
\sphinxAtStartPar
Even if you do not understand what these two pieces of code are doing, you can see that Python has a simpler syntax.

\end{itemize}

\sphinxAtStartPar
\sphinxincludegraphics{{java_vs_python}.jpg}
\begin{itemize}
\item {} 
\sphinxAtStartPar
Example: No need to declare a variable and its type before using it.
\begin{itemize}
\item {} 
\sphinxAtStartPar
In Java, you need to specify that age is an integer, name is a character (similar to a string in Python), game\_over is a boolean, and weight is a float. There is no such requirement in Python because Python can understand the type from the value.

\end{itemize}

\end{itemize}

\end{enumerate}


\begin{savenotes}\sphinxattablestart
\centering
\begin{tabulary}{\linewidth}[t]{|T|T|}
\hline
\sphinxstyletheadfamily 
\sphinxAtStartPar
Python
&\sphinxstyletheadfamily 
\sphinxAtStartPar
Java
\\
\hline
\sphinxAtStartPar
age=25
&
\sphinxAtStartPar
int age =25;
\\
\hline
\sphinxAtStartPar
name = ‘mike’
&
\sphinxAtStartPar
char name ‘mike’;
\\
\hline
\sphinxAtStartPar
game\_over=True
&
\sphinxAtStartPar
boolean game\_over=true;
\\
\hline
\sphinxAtStartPar
weight=120.75
&
\sphinxAtStartPar
float weight=120.75f;
\\
\hline
\end{tabulary}
\par
\sphinxattableend\end{savenotes}


\bigskip\hrule\bigskip

\begin{enumerate}
\sphinxsetlistlabels{\arabic}{enumi}{enumii}{}{.}%
\setcounter{enumi}{1}
\item {} 
\sphinxAtStartPar
Free and open source: Python is freely usable and distributable, including commercial purposes.
\begin{itemize}
\item {} 
\sphinxAtStartPar
You can freely download and install it to your computer.

\item {} 
\sphinxAtStartPar
Online editors are also available for use without any installation.

\item {} 
\sphinxAtStartPar
Python packages are developed by major companies and shared for everyone’s use.
\begin{itemize}
\item {} 
\sphinxAtStartPar
Tensorflow was developed by Google.

\item {} 
\sphinxAtStartPar
Pytorch was developed by Facebook.

\end{itemize}

\end{itemize}

\end{enumerate}


\bigskip\hrule\bigskip

\begin{enumerate}
\sphinxsetlistlabels{\arabic}{enumi}{enumii}{}{.}%
\setcounter{enumi}{2}
\item {} 
\sphinxAtStartPar
General\sphinxhyphen{}purpose programming language: You can write code for different purposes.
\begin{itemize}
\item {} 
\sphinxAtStartPar
Write game code

\item {} 
\sphinxAtStartPar
Develop programs for stores

\item {} 
\sphinxAtStartPar
Perform visualizations

\end{itemize}

\end{enumerate}


\bigskip\hrule\bigskip

\begin{enumerate}
\sphinxsetlistlabels{\arabic}{enumi}{enumii}{}{.}%
\setcounter{enumi}{3}
\item {} 
\sphinxAtStartPar
Rich Libraries: Python has rich collection of libraries encompassing of tools for various fields.
\begin{itemize}
\item {} 
\sphinxAtStartPar
Numpy: for scientific computing

\item {} 
\sphinxAtStartPar
Statistics: offers statistics tools

\item {} 
\sphinxAtStartPar
Pandas: for data wrangling and analysis

\item {} 
\sphinxAtStartPar
Matplotlib: for visualization

\item {} 
\sphinxAtStartPar
Keras: for constructing neural network models

\item {} 
\sphinxAtStartPar
Django: for develop websites

\item {} 
\sphinxAtStartPar
Flask: for online applications

\end{itemize}

\end{enumerate}


\bigskip\hrule\bigskip

\begin{enumerate}
\sphinxsetlistlabels{\arabic}{enumi}{enumii}{}{.}%
\setcounter{enumi}{4}
\item {} 
\sphinxAtStartPar
Object oriented: Build on the concept of objects.

\end{enumerate}
\begin{itemize}
\item {} 
\sphinxAtStartPar
Utilizes simple and reusable parts like blueprints

\item {} 
\sphinxAtStartPar
Short code may encompass many hidden functionalities

\end{itemize}


\bigskip\hrule\bigskip

\begin{enumerate}
\sphinxsetlistlabels{\arabic}{enumi}{enumii}{}{.}%
\setcounter{enumi}{5}
\item {} 
\sphinxAtStartPar
Community Support: Python has a vast and active community that can provide assistance.
\begin{itemize}
\item {} 
\sphinxAtStartPar
You can find answers to your rquestions on platfoms like \sphinxhref{https://stackoverflow.com/}{stackoverflow}.

\item {} 
\sphinxAtStartPar
Extensive resources available on platforms like \sphinxhref{https://linkedin.com/}{linkedin}

\end{itemize}

\end{enumerate}


\bigskip\hrule\bigskip

\begin{enumerate}
\sphinxsetlistlabels{\arabic}{enumi}{enumii}{}{.}%
\setcounter{enumi}{6}
\item {} 
\sphinxAtStartPar
Portability: Python code can be run on other platforms with little to no modifications.
\begin{itemize}
\item {} 
\sphinxAtStartPar
Works on virtual platforms

\item {} 
\sphinxAtStartPar
Windows, Unix, Linux, macOS

\end{itemize}

\end{enumerate}


\bigskip\hrule\bigskip

\begin{enumerate}
\sphinxsetlistlabels{\arabic}{enumi}{enumii}{}{.}%
\setcounter{enumi}{7}
\item {} 
\sphinxAtStartPar
Python code can be combined with components written in other languages like C++, Java.

\end{enumerate}


\bigskip\hrule\bigskip

\begin{enumerate}
\sphinxsetlistlabels{\arabic}{enumi}{enumii}{}{.}%
\setcounter{enumi}{8}
\item {} 
\sphinxAtStartPar
Python can be seen as a combination of general\sphinxhyphen{}purpose languages (such as C++ and Java) and domain\sphinxhyphen{}specific languages (like Matlab).

\end{enumerate}


\section{Disadvantages}
\label{\detokenize{python:disadvantages}}\begin{enumerate}
\sphinxsetlistlabels{\arabic}{enumi}{enumii}{}{.}%
\item {} 
\sphinxAtStartPar
Python code is visible to anyone using the application, allowing for code to be copied or modified.

\item {} 
\sphinxAtStartPar
The execution speed of Python is slower compared to languages like C++.
This is mainly due to the use of an interpreter instead of a compiler.
\begin{itemize}
\item {} 
\sphinxAtStartPar
The interpreter translates Python code into machine code, enabling the computer to understand and execute the code

\item {} 
\sphinxAtStartPar
While a compiler translates the entire source code in a single run, an interpreter processes the source code line by line.

\item {} 
\sphinxAtStartPar
To understand the difference between interpreters and compilers, you can watch the following video.

\end{itemize}

\end{enumerate}

\begin{sphinxuseclass}{cell}\begin{sphinxVerbatimInput}

\begin{sphinxuseclass}{cell_input}
\begin{sphinxVerbatim}[commandchars=\\\{\}]
\PYG{k+kn}{from} \PYG{n+nn}{IPython}\PYG{n+nn}{.}\PYG{n+nn}{lib}\PYG{n+nn}{.}\PYG{n+nn}{display} \PYG{k+kn}{import} \PYG{n}{YouTubeVideo}
\PYG{n}{YouTubeVideo}\PYG{p}{(}\PYG{l+s+s1}{\PYGZsq{}}\PYG{l+s+s1}{\PYGZus{}C5AHaS1mOA}\PYG{l+s+s1}{\PYGZsq{}}\PYG{p}{,} \PYG{n}{width}\PYG{o}{=}\PYG{l+m+mi}{500}\PYG{p}{,} \PYG{n}{height}\PYG{o}{=}\PYG{l+m+mi}{300}\PYG{p}{)}
\end{sphinxVerbatim}

\end{sphinxuseclass}\end{sphinxVerbatimInput}
\begin{sphinxVerbatimOutput}

\begin{sphinxuseclass}{cell_output}
\noindent\sphinxincludegraphics{{6f5611fde69020a2ea542c29b3ee52b330c3725656337f65d385e12964c2d308}.jpg}

\end{sphinxuseclass}\end{sphinxVerbatimOutput}

\end{sphinxuseclass}

\section{Installing Python}
\label{\detokenize{python:installing-python}}\begin{itemize}
\item {} 
\sphinxAtStartPar
It is easier to install Python with Anaconda

\item {} 
\sphinxAtStartPar
\sphinxhref{https://www.anaconda.com/}{Anaconda} is a free and open\sphinxhyphen{}source  distribution of the Python.

\item {} 
\sphinxAtStartPar
It installs Python.

\item {} 
\sphinxAtStartPar
It simplifies package management and deployment by providing a comprehensive collection of tools and libraries pre\sphinxhyphen{}installed.
\begin{itemize}
\item {} 
\sphinxAtStartPar
It comes with over 250 packages automatically installed.

\item {} 
\sphinxAtStartPar
It includes popular Python packages like NumPy, Pandas, Matplotlib, SciPy.

\end{itemize}

\end{itemize}


\section{Integrated development environment (IDE)}
\label{\detokenize{python:integrated-development-environment-ide}}\begin{itemize}
\item {} 
\sphinxAtStartPar
IDE provides tools to write codes in a better, easier, and faster way.

\item {} 
\sphinxAtStartPar
IDE includes:
\begin{itemize}
\item {} 
\sphinxAtStartPar
Code editor

\item {} 
\sphinxAtStartPar
Debugger

\item {} 
\sphinxAtStartPar
Code highlighting

\item {} 
\sphinxAtStartPar
Auto completion

\item {} 
\sphinxAtStartPar
Project Management

\end{itemize}

\item {} 
\sphinxAtStartPar
Examples :
\begin{itemize}
\item {} 
\sphinxAtStartPar
\sphinxhref{https://www.spyder-ide.org/}{Spyder}: It comes with Anaconda.
\begin{itemize}
\item {} 
\sphinxAtStartPar
It is free and open source.

\item {} 
\sphinxAtStartPar
Designed by and for scientists, engineers and data analysts.

\end{itemize}

\item {} 
\sphinxAtStartPar
\sphinxhref{https://www.jetbrains.com/pycharm/}{Pycharm}: It is developed by Jet Brains which is professional and very advanced.
\begin{itemize}
\item {} 
\sphinxAtStartPar
Its community edition is free.

\item {} 
\sphinxAtStartPar
Professional edition is free for students and educators.

\end{itemize}

\end{itemize}

\end{itemize}


\section{Jupyter Notebook}
\label{\detokenize{python:jupyter-notebook}}\begin{itemize}
\item {} 
\sphinxAtStartPar
\sphinxhref{https://jupyter.org/}{Jupyter Notebook} allows writing or running code in a web browser.

\item {} 
\sphinxAtStartPar
It does not require internet access.

\item {} 
\sphinxAtStartPar
It is free and comes with Anaconda.

\end{itemize}


\section{Google Colab}
\label{\detokenize{python:google-colab}}\begin{itemize}
\item {} 
\sphinxAtStartPar
Short for \sphinxhref{https://colab.research.google.com/?utm\_source=scs-index}{Google Colaboratory}.

\item {} 
\sphinxAtStartPar
An online tool for writing and running code.

\item {} 
\sphinxAtStartPar
A cloud\sphinxhyphen{}based computational environment or notebook.

\item {} 
\sphinxAtStartPar
Recommended for beginners since there’s no need to install any software.

\item {} 
\sphinxAtStartPar
Files are automatically saved to Google Drive.

\item {} 
\sphinxAtStartPar
As it is cloud\sphinxhyphen{}based, it requires an internet connection.

\item {} 
\sphinxAtStartPar
Colab notebooks are Jupyter notebooks that are hosted by Colab.

\end{itemize}


\section{Applications}
\label{\detokenize{python:applications}}\begin{itemize}
\item {} 
\sphinxAtStartPar
The following examples demonstrate what can be accomplished with Python.

\item {} 
\sphinxAtStartPar
The aim of this section is to provide you with an idea of Python’s capabilities.

\item {} 
\sphinxAtStartPar
You’re not expected to comprehend the code at this stage.

\end{itemize}


\subsection{Import Stock Data}
\label{\detokenize{python:import-stock-data}}\begin{itemize}
\item {} 
\sphinxAtStartPar
You can import historical stock data from Yahoo Finance.

\item {} 
\sphinxAtStartPar
The following represents data for Apple Stocks.

\end{itemize}

\begin{sphinxuseclass}{cell}\begin{sphinxVerbatimInput}

\begin{sphinxuseclass}{cell_input}
\begin{sphinxVerbatim}[commandchars=\\\{\}]
\PYG{k+kn}{import} \PYG{n+nn}{yfinance}
\PYG{n}{df} \PYG{o}{=} \PYG{n}{yfinance}\PYG{o}{.}\PYG{n}{Ticker}\PYG{p}{(}\PYG{l+s+s1}{\PYGZsq{}}\PYG{l+s+s1}{AAPL}\PYG{l+s+s1}{\PYGZsq{}}\PYG{p}{)}\PYG{o}{.}\PYG{n}{history}\PYG{p}{(}\PYG{p}{)}
\PYG{n}{df}\PYG{o}{.}\PYG{n}{head}\PYG{p}{(}\PYG{p}{)}\PYG{o}{.}\PYG{n}{round}\PYG{p}{(}\PYG{l+m+mi}{2}\PYG{p}{)}
\end{sphinxVerbatim}

\end{sphinxuseclass}\end{sphinxVerbatimInput}
\begin{sphinxVerbatimOutput}

\begin{sphinxuseclass}{cell_output}
\begin{sphinxVerbatim}[commandchars=\\\{\}]
                             Open    High     Low   Close    Volume  \PYGZbs{}
Date                                                                  
2024\PYGZhy{}01\PYGZhy{}03 00:00:00\PYGZhy{}05:00  184.22  185.88  183.43  184.25  58414500   
2024\PYGZhy{}01\PYGZhy{}04 00:00:00\PYGZhy{}05:00  182.15  183.09  180.88  181.91  71983600   
2024\PYGZhy{}01\PYGZhy{}05 00:00:00\PYGZhy{}05:00  181.99  182.76  180.17  181.18  62303300   
2024\PYGZhy{}01\PYGZhy{}08 00:00:00\PYGZhy{}05:00  182.09  185.60  181.50  185.56  59144500   
2024\PYGZhy{}01\PYGZhy{}09 00:00:00\PYGZhy{}05:00  183.92  185.15  182.73  185.14  42841800   

                           Dividends  Stock Splits  
Date                                                
2024\PYGZhy{}01\PYGZhy{}03 00:00:00\PYGZhy{}05:00        0.0           0.0  
2024\PYGZhy{}01\PYGZhy{}04 00:00:00\PYGZhy{}05:00        0.0           0.0  
2024\PYGZhy{}01\PYGZhy{}05 00:00:00\PYGZhy{}05:00        0.0           0.0  
2024\PYGZhy{}01\PYGZhy{}08 00:00:00\PYGZhy{}05:00        0.0           0.0  
2024\PYGZhy{}01\PYGZhy{}09 00:00:00\PYGZhy{}05:00        0.0           0.0  
\end{sphinxVerbatim}

\end{sphinxuseclass}\end{sphinxVerbatimOutput}

\end{sphinxuseclass}

\subsection{Import Data from Wikipedia}
\label{\detokenize{python:import-data-from-wikipedia}}\begin{itemize}
\item {} 
\sphinxAtStartPar
You can import tables from Wikipedia websites.

\item {} 
\sphinxAtStartPar
The following is the table of SP500 companies in the website \sphinxurl{https://en.wikipedia.org/wiki/List\_of\_S\%26P\_500\_companies}

\end{itemize}

\begin{sphinxuseclass}{cell}\begin{sphinxVerbatimInput}

\begin{sphinxuseclass}{cell_input}
\begin{sphinxVerbatim}[commandchars=\\\{\}]
\PYG{k+kn}{import} \PYG{n+nn}{pandas} \PYG{k}{as} \PYG{n+nn}{pd}
\PYG{n}{pd}\PYG{o}{.}\PYG{n}{read\PYGZus{}html}\PYG{p}{(}\PYG{l+s+s1}{\PYGZsq{}}\PYG{l+s+s1}{https://en.wikipedia.org/wiki/List\PYGZus{}of\PYGZus{}S}\PYG{l+s+s1}{\PYGZpc{}}\PYG{l+s+s1}{26P\PYGZus{}500\PYGZus{}companies}\PYG{l+s+s1}{\PYGZsq{}}\PYG{p}{)}\PYG{p}{[}\PYG{l+m+mi}{0}\PYG{p}{]}\PYG{o}{.}\PYG{n}{head}\PYG{p}{(}\PYG{p}{)}
\end{sphinxVerbatim}

\end{sphinxuseclass}\end{sphinxVerbatimInput}
\begin{sphinxVerbatimOutput}

\begin{sphinxuseclass}{cell_output}
\begin{sphinxVerbatim}[commandchars=\\\{\}]
  Symbol     Security             GICS Sector               GICS Sub\PYGZhy{}Industry  \PYGZbs{}
0    MMM           3M             Industrials        Industrial Conglomerates   
1    AOS  A. O. Smith             Industrials               Building Products   
2    ABT       Abbott             Health Care           Health Care Equipment   
3   ABBV       AbbVie             Health Care                   Biotechnology   
4    ACN    Accenture  Information Technology  IT Consulting \PYGZam{} Other Services   

     Headquarters Location  Date added      CIK      Founded  
0    Saint Paul, Minnesota  1957\PYGZhy{}03\PYGZhy{}04    66740         1902  
1     Milwaukee, Wisconsin  2017\PYGZhy{}07\PYGZhy{}26    91142         1916  
2  North Chicago, Illinois  1957\PYGZhy{}03\PYGZhy{}04     1800         1888  
3  North Chicago, Illinois  2012\PYGZhy{}12\PYGZhy{}31  1551152  2013 (1888)  
4          Dublin, Ireland  2011\PYGZhy{}07\PYGZhy{}06  1467373         1989  
\end{sphinxVerbatim}

\end{sphinxuseclass}\end{sphinxVerbatimOutput}

\end{sphinxuseclass}

\subsection{Scatter Plot}
\label{\detokenize{python:scatter-plot}}
\begin{sphinxuseclass}{cell}\begin{sphinxVerbatimInput}

\begin{sphinxuseclass}{cell_input}
\begin{sphinxVerbatim}[commandchars=\\\{\}]
\PYG{k+kn}{import} \PYG{n+nn}{matplotlib}\PYG{n+nn}{.}\PYG{n+nn}{pyplot} \PYG{k}{as} \PYG{n+nn}{plt}
\PYG{n}{plt}\PYG{o}{.}\PYG{n}{figure}\PYG{p}{(}\PYG{n}{figsize}\PYG{o}{=}\PYG{p}{(}\PYG{l+m+mi}{10}\PYG{p}{,}\PYG{l+m+mi}{5}\PYG{p}{)}\PYG{p}{)}
\PYG{n}{plt}\PYG{o}{.}\PYG{n}{scatter}\PYG{p}{(}\PYG{n}{df}\PYG{o}{.}\PYG{n}{index}\PYG{p}{,} \PYG{n}{df}\PYG{o}{.}\PYG{n}{Close}\PYG{p}{,} \PYG{n}{color}\PYG{o}{=}\PYG{l+s+s1}{\PYGZsq{}}\PYG{l+s+s1}{r}\PYG{l+s+s1}{\PYGZsq{}}\PYG{p}{)}
\PYG{n}{plt}\PYG{o}{.}\PYG{n}{xticks}\PYG{p}{(}\PYG{n}{rotation}\PYG{o}{=}\PYG{l+m+mi}{30}\PYG{p}{)}\PYG{p}{;}
\end{sphinxVerbatim}

\end{sphinxuseclass}\end{sphinxVerbatimInput}
\begin{sphinxVerbatimOutput}

\begin{sphinxuseclass}{cell_output}
\noindent\sphinxincludegraphics{{9e5480fad71a6d1bab1e059be025604c8890c72a57588015dea36ddc1f22bfed}.png}

\end{sphinxuseclass}\end{sphinxVerbatimOutput}

\end{sphinxuseclass}

\subsection{Line Plot}
\label{\detokenize{python:line-plot}}
\begin{sphinxuseclass}{cell}\begin{sphinxVerbatimInput}

\begin{sphinxuseclass}{cell_input}
\begin{sphinxVerbatim}[commandchars=\\\{\}]
\PYG{n}{plt}\PYG{o}{.}\PYG{n}{figure}\PYG{p}{(}\PYG{n}{figsize}\PYG{o}{=}\PYG{p}{(}\PYG{l+m+mi}{10}\PYG{p}{,}\PYG{l+m+mi}{5}\PYG{p}{)}\PYG{p}{)}
\PYG{n}{plt}\PYG{o}{.}\PYG{n}{plot}\PYG{p}{(}\PYG{n}{df}\PYG{o}{.}\PYG{n}{index}\PYG{p}{,} \PYG{n}{df}\PYG{o}{.}\PYG{n}{Close}\PYG{p}{,} \PYG{n}{color}\PYG{o}{=}\PYG{l+s+s1}{\PYGZsq{}}\PYG{l+s+s1}{r}\PYG{l+s+s1}{\PYGZsq{}}\PYG{p}{)}
\PYG{n}{plt}\PYG{o}{.}\PYG{n}{xticks}\PYG{p}{(}\PYG{n}{rotation}\PYG{o}{=}\PYG{l+m+mi}{30}\PYG{p}{)}\PYG{p}{;}
\end{sphinxVerbatim}

\end{sphinxuseclass}\end{sphinxVerbatimInput}
\begin{sphinxVerbatimOutput}

\begin{sphinxuseclass}{cell_output}
\noindent\sphinxincludegraphics{{21dd0222044b7067e7194d4b788b23c7a27fae230fc05fbd96aa58244906ce7c}.png}

\end{sphinxuseclass}\end{sphinxVerbatimOutput}

\end{sphinxuseclass}

\subsection{Histogram}
\label{\detokenize{python:histogram}}
\begin{sphinxuseclass}{cell}\begin{sphinxVerbatimInput}

\begin{sphinxuseclass}{cell_input}
\begin{sphinxVerbatim}[commandchars=\\\{\}]
\PYG{n}{plt}\PYG{o}{.}\PYG{n}{figure}\PYG{p}{(}\PYG{n}{figsize}\PYG{o}{=}\PYG{p}{(}\PYG{l+m+mi}{10}\PYG{p}{,}\PYG{l+m+mi}{3}\PYG{p}{)}\PYG{p}{)}
\PYG{n}{plt}\PYG{o}{.}\PYG{n}{hist}\PYG{p}{(}\PYG{n}{df}\PYG{o}{.}\PYG{n}{Close}\PYG{p}{,} \PYG{n}{bins}\PYG{o}{=}\PYG{l+m+mi}{20}\PYG{p}{,} \PYG{n}{color}\PYG{o}{=}\PYG{l+s+s1}{\PYGZsq{}}\PYG{l+s+s1}{orange}\PYG{l+s+s1}{\PYGZsq{}}\PYG{p}{,} \PYG{n}{orientation}\PYG{o}{=}\PYG{l+s+s1}{\PYGZsq{}}\PYG{l+s+s1}{horizontal}\PYG{l+s+s1}{\PYGZsq{}}\PYG{p}{)}\PYG{p}{;}
\end{sphinxVerbatim}

\end{sphinxuseclass}\end{sphinxVerbatimInput}
\begin{sphinxVerbatimOutput}

\begin{sphinxuseclass}{cell_output}
\noindent\sphinxincludegraphics{{9c59cc8609ff70af327bb5f8653ced058a6ea74f05638d92971efd3b9ea63e8b}.png}

\end{sphinxuseclass}\end{sphinxVerbatimOutput}

\end{sphinxuseclass}

\subsection{Pie Chart}
\label{\detokenize{python:pie-chart}}
\begin{sphinxuseclass}{cell}\begin{sphinxVerbatimInput}

\begin{sphinxuseclass}{cell_input}
\begin{sphinxVerbatim}[commandchars=\\\{\}]
\PYG{n}{number} \PYG{o}{=} \PYG{p}{[}\PYG{l+m+mi}{53}\PYG{p}{,} \PYG{l+m+mi}{122}\PYG{p}{,} \PYG{l+m+mi}{96}\PYG{p}{,} \PYG{l+m+mi}{239}\PYG{p}{]}
\PYG{n}{color\PYGZus{}list} \PYG{o}{=} \PYG{p}{[}\PYG{l+s+s1}{\PYGZsq{}}\PYG{l+s+s1}{y}\PYG{l+s+s1}{\PYGZsq{}}\PYG{p}{,} \PYG{l+s+s1}{\PYGZsq{}}\PYG{l+s+s1}{purple}\PYG{l+s+s1}{\PYGZsq{}}\PYG{p}{,} \PYG{l+s+s1}{\PYGZsq{}}\PYG{l+s+s1}{g}\PYG{l+s+s1}{\PYGZsq{}}\PYG{p}{,} \PYG{l+s+s1}{\PYGZsq{}}\PYG{l+s+s1}{r}\PYG{l+s+s1}{\PYGZsq{}}\PYG{p}{]}
\PYG{n}{coins}  \PYG{o}{=} \PYG{p}{[}\PYG{l+s+s1}{\PYGZsq{}}\PYG{l+s+s1}{Penny}\PYG{l+s+s1}{\PYGZsq{}}\PYG{p}{,} \PYG{l+s+s1}{\PYGZsq{}}\PYG{l+s+s1}{Nickel}\PYG{l+s+s1}{\PYGZsq{}}\PYG{p}{,} \PYG{l+s+s1}{\PYGZsq{}}\PYG{l+s+s1}{Dime}\PYG{l+s+s1}{\PYGZsq{}}\PYG{p}{,} \PYG{l+s+s1}{\PYGZsq{}}\PYG{l+s+s1}{Quarter}\PYG{l+s+s1}{\PYGZsq{}}\PYG{p}{]}
\PYG{n}{plt}\PYG{o}{.}\PYG{n}{pie}\PYG{p}{(}\PYG{n}{number}\PYG{p}{,} \PYG{n}{colors} \PYG{o}{=} \PYG{n}{color\PYGZus{}list}\PYG{p}{,} \PYG{n}{autopct}\PYG{o}{=}\PYG{l+s+s1}{\PYGZsq{}}\PYG{l+s+si}{\PYGZpc{}1.1f}\PYG{l+s+si}{\PYGZpc{}\PYGZpc{}}\PYG{l+s+s1}{\PYGZsq{}}\PYG{p}{,}  \PYG{n}{labels} \PYG{o}{=} \PYG{n}{coins}\PYG{p}{,} \PYG{n}{radius}\PYG{o}{=}\PYG{l+m+mf}{0.75}\PYG{p}{)}\PYG{p}{;}
\end{sphinxVerbatim}

\end{sphinxuseclass}\end{sphinxVerbatimInput}
\begin{sphinxVerbatimOutput}

\begin{sphinxuseclass}{cell_output}
\noindent\sphinxincludegraphics{{848cc178ee70b69285dcbcdb0886ec6928735aec21d144fcaa77b04b89078d9b}.png}

\end{sphinxuseclass}\end{sphinxVerbatimOutput}

\end{sphinxuseclass}

\subsection{Multiple Plots}
\label{\detokenize{python:multiple-plots}}
\begin{sphinxuseclass}{cell}\begin{sphinxVerbatimInput}

\begin{sphinxuseclass}{cell_input}
\begin{sphinxVerbatim}[commandchars=\\\{\}]
\PYG{n}{plt}\PYG{o}{.}\PYG{n}{figure}\PYG{p}{(}\PYG{n}{figsize} \PYG{o}{=} \PYG{p}{(}\PYG{l+m+mi}{20}\PYG{p}{,} \PYG{l+m+mi}{10}\PYG{p}{)}\PYG{p}{)}
\PYG{n}{color\PYGZus{}set} \PYG{o}{=} \PYG{p}{[}\PYG{l+s+s1}{\PYGZsq{}}\PYG{l+s+s1}{r\PYGZhy{}\PYGZhy{}}\PYG{l+s+s1}{\PYGZsq{}}\PYG{p}{,} \PYG{l+s+s1}{\PYGZsq{}}\PYG{l+s+s1}{g\PYGZhy{}\PYGZhy{}}\PYG{l+s+s1}{\PYGZsq{}}\PYG{p}{,} \PYG{l+s+s1}{\PYGZsq{}}\PYG{l+s+s1}{b\PYGZhy{}\PYGZhy{}}\PYG{l+s+s1}{\PYGZsq{}}\PYG{p}{,} \PYG{l+s+s1}{\PYGZsq{}}\PYG{l+s+s1}{o\PYGZhy{}\PYGZhy{}}\PYG{l+s+s1}{\PYGZsq{}}\PYG{p}{]}
\PYG{k}{for} \PYG{n}{i} \PYG{o+ow}{in} \PYG{n+nb}{range}\PYG{p}{(}\PYG{l+m+mi}{1}\PYG{p}{,}\PYG{l+m+mi}{5}\PYG{p}{)}\PYG{p}{:}
    \PYG{n}{plt}\PYG{o}{.}\PYG{n}{subplot}\PYG{p}{(}\PYG{l+m+mi}{2}\PYG{p}{,} \PYG{l+m+mi}{2}\PYG{p}{,} \PYG{n}{i}\PYG{p}{)}
    \PYG{n}{plt}\PYG{o}{.}\PYG{n}{plot}\PYG{p}{(}\PYG{n}{df}\PYG{o}{.}\PYG{n}{iloc}\PYG{p}{[}\PYG{p}{:}\PYG{p}{,}\PYG{n}{i}\PYG{o}{\PYGZhy{}}\PYG{l+m+mi}{1}\PYG{p}{]}\PYG{p}{,} \PYG{n}{color\PYGZus{}set}\PYG{p}{[}\PYG{n}{i}\PYG{o}{\PYGZhy{}}\PYG{l+m+mi}{1}\PYG{p}{]}\PYG{p}{)}\PYG{p}{;}
    \PYG{n}{plt}\PYG{o}{.}\PYG{n}{ylabel}\PYG{p}{(}\PYG{n}{df}\PYG{o}{.}\PYG{n}{columns}\PYG{p}{[}\PYG{n}{i}\PYG{o}{\PYGZhy{}}\PYG{l+m+mi}{1}\PYG{p}{]}\PYG{p}{,}\PYG{n}{fontsize}\PYG{o}{=}\PYG{l+m+mi}{15}\PYG{p}{)}
    \PYG{n}{plt}\PYG{o}{.}\PYG{n}{xticks}\PYG{p}{(}\PYG{n}{rotation}\PYG{o}{=}\PYG{l+m+mi}{30}\PYG{p}{)}
    \PYG{n}{plt}\PYG{o}{.}\PYG{n}{grid}\PYG{p}{(}\PYG{p}{)}
\end{sphinxVerbatim}

\end{sphinxuseclass}\end{sphinxVerbatimInput}
\begin{sphinxVerbatimOutput}

\begin{sphinxuseclass}{cell_output}
\noindent\sphinxincludegraphics{{7e453578ec92de1ebc5a95e01a895ba81ea1b1294905998cea59a23f8bccc858}.png}

\end{sphinxuseclass}\end{sphinxVerbatimOutput}

\end{sphinxuseclass}
\sphinxstepscope


\chapter{Google Colab}
\label{\detokenize{colab:google-colab}}\label{\detokenize{colab::doc}}
\sphinxAtStartPar
Google Colab (Colaboratory) is a cloud\sphinxhyphen{}based notebook for writing and executing Python code.
\begin{itemize}
\item {} 
\sphinxAtStartPar
You need to sign into your Google account to execute code and save your notebook.

\item {} 
\sphinxAtStartPar
There’s no requirement to install any software on your computer.

\item {} 
\sphinxAtStartPar
Sharing your notebooks is easy, and Colab files are automatically saved to Google Drive.

\end{itemize}


\section{Google Drive}
\label{\detokenize{colab:google-drive}}
\sphinxAtStartPar
Google Drive is used for storing files in the cloud.
\begin{itemize}
\item {} 
\sphinxAtStartPar
You can manage, modify, and share your files.

\item {} 
\sphinxAtStartPar
By default, you have 15 GB of free storage, but there are also options for paid storage, such as 100 GB, 200 GB, and so on.

\item {} 
\sphinxAtStartPar
Within your Google or Gmail account, you can access Google Drive and other applications by clicking on the Google Apps Icon, which has nine dots and resembles a square.

\item {} 
\sphinxAtStartPar
By clicking on \sphinxcode{\sphinxupquote{+New}} in the upper left corner of your Google Drive (located underneath the Drive icon), you can:
\begin{itemize}
\item {} 
\sphinxAtStartPar
Upload a folder or file to Drive

\item {} 
\sphinxAtStartPar
Create Google Docs, Google Sheets, Google Slides, Google Forms, etc.

\item {} 
\sphinxAtStartPar
When you click on the \sphinxcode{\sphinxupquote{More Actions}} icon (three vertical dots on the right), you will see options to modify your folder.

\item {} 
\sphinxAtStartPar
You can move, download, share, rename, or delete your files and folders.

\end{itemize}

\end{itemize}

\sphinxAtStartPar
\sphinxincludegraphics{{four_drive2}.png}


\section{Installing Colab to your Google Drive}
\label{\detokenize{colab:installing-colab-to-your-google-drive}}
\sphinxAtStartPar
You can easily create Colab notebooks on your Google Drive by installing Colab onto your Google Drive (not onto your computer).
\begin{itemize}
\item {} 
\sphinxAtStartPar
Follow the following steps for installation:
\begin{enumerate}
\sphinxsetlistlabels{\arabic}{enumi}{enumii}{}{.}%
\item {} 
\sphinxAtStartPar
Click on \sphinxcode{\sphinxupquote{+New}} in the upper left corner of your Google Drive (located underneath the Drive icon).

\item {} 
\sphinxAtStartPar
Choose the \sphinxcode{\sphinxupquote{More}} option at the end.

\item {} 
\sphinxAtStartPar
In the new menu, if you see the Google Colaboratory icon, it means you already have Colab installed on your drive.

\item {} 
\sphinxAtStartPar
If there is no Google Colaboratory icon, then click on \sphinxcode{\sphinxupquote{Connect More Apps.}}

\item {} 
\sphinxAtStartPar
Search for Google Colaboratory.

\item {} 
\sphinxAtStartPar
Click on Google Colaboratory in the search results and install Colab on your drive.

\end{enumerate}

\end{itemize}

\sphinxAtStartPar
\sphinxincludegraphics{{four_colab_install2}.png}


\section{Create a new Colab notebook}
\label{\detokenize{colab:create-a-new-colab-notebook}}
\sphinxAtStartPar
Follow these steps:
\begin{enumerate}
\sphinxsetlistlabels{\arabic}{enumi}{enumii}{}{.}%
\item {} 
\sphinxAtStartPar
Click on \sphinxcode{\sphinxupquote{+New}} in the upper left corner of your Google Drive (located underneath the Drive icon).

\item {} 
\sphinxAtStartPar
Choose the \sphinxcode{\sphinxupquote{More}} option at the end.

\item {} 
\sphinxAtStartPar
In the new menu, click on \sphinxcode{\sphinxupquote{Google Colaboratory}}.

\end{enumerate}

\sphinxAtStartPar
Additionally, instead of using \sphinxcode{\sphinxupquote{+New,}} you can right\sphinxhyphen{}click to access its contents.


\section{Rename a Colab File}
\label{\detokenize{colab:rename-a-colab-file}}\begin{itemize}
\item {} 
\sphinxAtStartPar
There are two ways to rename a Colab notebook:
\begin{itemize}
\item {} 
\sphinxAtStartPar
Use the \sphinxcode{\sphinxupquote{More Actions}} icon (three vertical dots on the right).

\item {} 
\sphinxAtStartPar
Open the Colab notebook (the default name is Untitled0) and click on its name in the upper left corner, next to the Google Drive icon.

\end{itemize}

\end{itemize}

\sphinxAtStartPar
\sphinxincludegraphics{{rename2}.png}


\section{Share a Colab File}
\label{\detokenize{colab:share-a-colab-file}}
\sphinxAtStartPar
To access the share options, you can do one of the following:
\begin{enumerate}
\sphinxsetlistlabels{\arabic}{enumi}{enumii}{}{.}%
\item {} 
\sphinxAtStartPar
Use the \sphinxcode{\sphinxupquote{More Actions}} icon (three vertical dots on the right).

\item {} 
\sphinxAtStartPar
Open the Colab notebook and click on \sphinxcode{\sphinxupquote{Share}} in the upper right corner.

\end{enumerate}


\bigskip\hrule\bigskip


\sphinxAtStartPar
There are two ways to share a Colab notebook:
\begin{enumerate}
\sphinxsetlistlabels{\arabic}{enumi}{enumii}{}{.}%
\item {} 
\sphinxAtStartPar
Enter the Gmail address of the person you want to share the notebook with.

\item {} 
\sphinxAtStartPar
Copy the link and share it, but change the last part from \sphinxcode{\sphinxupquote{Restricted}} to \sphinxcode{\sphinxupquote{Anyone with the link.}}

\end{enumerate}


\bigskip\hrule\bigskip


\sphinxAtStartPar
You have the following options to assign the person:
\begin{itemize}
\item {} 
\sphinxAtStartPar
Editor: Can change everything in the document.

\item {} 
\sphinxAtStartPar
Commenter: Can write comments.

\item {} 
\sphinxAtStartPar
Viewer: Can download a copy of the file but cannot modify your original file

\end{itemize}

\sphinxAtStartPar
\sphinxincludegraphics{{share_four3}.png}


\section{More Tools}
\label{\detokenize{colab:more-tools}}

\subsection{Theme}
\label{\detokenize{colab:theme}}
\sphinxAtStartPar
There are three theme options:
\begin{itemize}
\item {} 
\sphinxAtStartPar
light (white background)

\item {} 
\sphinxAtStartPar
dark (black background)

\item {} 
\sphinxAtStartPar
adaptive (color depends on time)

\end{itemize}

\sphinxAtStartPar
Follow these steps on the toolbar to change the theme:  \sphinxcode{\sphinxupquote{Tools}} –> \sphinxcode{\sphinxupquote{Settings}} –> \sphinxcode{\sphinxupquote{Theme}}


\subsection{Power Level, and Kitty, Corgi, Crab}
\label{\detokenize{colab:power-level-and-kitty-corgi-crab}}
\sphinxAtStartPar
These are some fun tools to make the notebook more fancy:
\begin{itemize}
\item {} 
\sphinxAtStartPar
Go to \sphinxcode{\sphinxupquote{Tools}} –> \sphinxcode{\sphinxupquote{Settings}} –> \sphinxcode{\sphinxupquote{Miscellaneous}}

\item {} 
\sphinxAtStartPar
You can adjust the power level to add sparks when you write or execute code.

\item {} 
\sphinxAtStartPar
You can add walking Kitty, Corgi, and Crab characters to your notebook.

\item {} 
\sphinxAtStartPar
\sphinxincludegraphics{{theme_kitty2}.png}

\end{itemize}


\section{Cells in Colab}
\label{\detokenize{colab:cells-in-colab}}
\sphinxAtStartPar
These are rectangular\sphinxhyphen{}shaped boxes that come in two types:
\begin{itemize}
\item {} 
\sphinxAtStartPar
Text Cell: Used for writing explanatory text.

\item {} 
\sphinxAtStartPar
Code Cell: Used for writing and executing Python code.

\end{itemize}


\bigskip\hrule\bigskip

\begin{itemize}
\item {} 
\sphinxAtStartPar
By default, a new notebook contains one code cell.

\item {} 
\sphinxAtStartPar
You can add, delete, and move cells up or down.

\item {} 
\sphinxAtStartPar
A Colab notebook is essentially a list of cells.

\end{itemize}


\subsection{Adding Cells}
\label{\detokenize{colab:adding-cells}}
\sphinxAtStartPar
You can add a text or code cell by using the \sphinxcode{\sphinxupquote{+Code}} and \sphinxcode{\sphinxupquote{+Text}} buttons on the toolbar (upper left corner).
\begin{itemize}
\item {} 
\sphinxAtStartPar
You can use these buttons to add a cell below.

\end{itemize}

\sphinxAtStartPar
When you move the cursor to the upper (lower) middle of a cell, you will also see \sphinxcode{\sphinxupquote{+Code}} and \sphinxcode{\sphinxupquote{+Text}} buttons.
\begin{itemize}
\item {} 
\sphinxAtStartPar
You can use them to add a cell above (below).

\end{itemize}

\sphinxAtStartPar
\sphinxincludegraphics{{cell_three2}.png}


\subsection{Delete, Move and Comment}
\label{\detokenize{colab:delete-move-and-comment}}
\sphinxAtStartPar
When you click on a cell, a menu will appear in the upper right corner of the cell. Using this menu:
\begin{itemize}
\item {} 
\sphinxAtStartPar
You can delete the cell.

\item {} 
\sphinxAtStartPar
You can move the cell up or down.

\item {} 
\sphinxAtStartPar
You can add a comment to the cell.

\end{itemize}


\subsection{Short cuts}
\label{\detokenize{colab:short-cuts}}\begin{itemize}
\item {} 
\sphinxAtStartPar
\sphinxcode{\sphinxupquote{A}}: Add a code cell above

\item {} 
\sphinxAtStartPar
\sphinxcode{\sphinxupquote{B}}: Add a code cell below

\item {} 
\sphinxAtStartPar
\sphinxcode{\sphinxupquote{shift}}+\sphinxcode{\sphinxupquote{Enter}}: Execute a code cell

\end{itemize}

\sphinxstepscope


\section{Text Cell}
\label{\detokenize{colab_text:text-cell}}\label{\detokenize{colab_text::doc}}
\sphinxAtStartPar
You can write explanatory text.
\begin{itemize}
\item {} 
\sphinxAtStartPar
By using tools covered in this section, you can compose well\sphinxhyphen{}organized texts.

\item {} 
\sphinxAtStartPar
To edit a text cell, double\sphinxhyphen{}click on it.

\item {} 
\sphinxAtStartPar
Click outside the text cell to exit the edit mode.

\item {} 
\sphinxAtStartPar
Text cells in Colab use Markdown language syntax.

\item {} 
\sphinxAtStartPar
When you start writing in a text cell, you will see another box on the right that shows the output.

\item {} 
\sphinxAtStartPar
Text cells do not have an arrow on the left for execution.

\item {} 
\sphinxAtStartPar
You do not need to memorize HTML code in this section; you can copy/paste it if needed.

\end{itemize}


\subsection{Heading}
\label{\detokenize{colab_text:heading}}\begin{itemize}
\item {} 
\sphinxAtStartPar
Start the line with a hashtag (\#) followed by a space.

\item {} 
\sphinxAtStartPar
To make the header smaller, add more hashtags (\#).

\end{itemize}

\begin{sphinxVerbatim}[commandchars=\\\{\}]
\PYG{g+gh}{\PYGZsh{} Level\PYGZhy{}1 heading}
\PYG{g+gu}{\PYGZsh{}\PYGZsh{} Level\PYGZhy{}2 heading}
\PYG{g+gu}{\PYGZsh{}\PYGZsh{}\PYGZsh{} Level\PYGZhy{}3 heading}
\PYG{g+gu}{\PYGZsh{}\PYGZsh{}\PYGZsh{}\PYGZsh{} Level\PYGZhy{}4 heading}
\PYG{g+gu}{\PYGZsh{}\PYGZsh{}\PYGZsh{}\PYGZsh{}\PYGZsh{} Level\PYGZhy{}5 heading}
\PYG{g+gu}{\PYGZsh{}\PYGZsh{}\PYGZsh{}\PYGZsh{}\PYGZsh{}\PYGZsh{} Level\PYGZhy{}6 heading}
\end{sphinxVerbatim}
\begin{itemize}
\item {} 
\sphinxAtStartPar
The table of contents is located on the left side of Colab, marked by an icon displaying three rows of a dot and line.

\item {} 
\sphinxAtStartPar
All headings appear in the table of contents according to their level.

\item {} 
\sphinxAtStartPar
You can also use the \sphinxcode{\sphinxupquote{tT}} icon on the text cell toolbar to create a text heading.

\end{itemize}


\subsection{Center}
\label{\detokenize{colab_text:center}}\begin{itemize}
\item {} 
\sphinxAtStartPar
You can use the following HTML code to center text: \sphinxcode{\sphinxupquote{<center> text </center>}}.

\item {} 
\sphinxAtStartPar
Output:

\end{itemize}




\subsection{Bold}
\label{\detokenize{colab_text:bold}}\begin{itemize}
\item {} 
\sphinxAtStartPar
Use \sphinxcode{\sphinxupquote{**}} (two asterisks) or \sphinxcode{\sphinxupquote{\_\_}} (two underscores) on both sides of the text to apply bold formatting.

\item {} 
\sphinxAtStartPar
Alternatively, you can utilize the \sphinxcode{\sphinxupquote{B}} icon on the text cell toolbar.

\item {} 
\sphinxAtStartPar
The output of \sphinxcode{\sphinxupquote{How **are** you?}} is
\begin{itemize}
\item {} 
\sphinxAtStartPar
How \sphinxstylestrong{are} you?

\end{itemize}

\end{itemize}


\subsection{Italic}
\label{\detokenize{colab_text:italic}}\begin{itemize}
\item {} 
\sphinxAtStartPar
Use \sphinxcode{\sphinxupquote{*}} (one asterisk) or \sphinxcode{\sphinxupquote{\_}} (one underscore) on both sides of the text to apply italic formatting.

\item {} 
\sphinxAtStartPar
Alternatively, you can utilize the \sphinxcode{\sphinxupquote{I}} icon on the text cell toolbar.

\item {} 
\sphinxAtStartPar
The output of \sphinxcode{\sphinxupquote{How *are* you?}} is
\begin{itemize}
\item {} 
\sphinxAtStartPar
How \sphinxstyleemphasis{are} you?

\end{itemize}

\end{itemize}


\subsection{Line Breaks}
\label{\detokenize{colab_text:line-breaks}}\begin{itemize}
\item {} 
\sphinxAtStartPar
Add 2 blank spaces or \sphinxcode{\sphinxupquote{\textbackslash{}}} at the end of the line to start a new line.

\end{itemize}


\subsection{Color}
\label{\detokenize{colab_text:color}}\begin{itemize}
\item {} 
\sphinxAtStartPar
You can use the following HTML code to change the color of text:
\sphinxcode{\sphinxupquote{ <font color=red> text </font>}}.

\item {} 
\sphinxAtStartPar
Output:
text

\end{itemize}


\subsection{Size}
\label{\detokenize{colab_text:size}}\begin{itemize}
\item {} 
\sphinxAtStartPar
You can use the following HTML code to change the size of text:\\
\sphinxcode{\sphinxupquote{ <font size=25> text </font>}}.

\item {} 
\sphinxAtStartPar
Output:

\end{itemize}

\sphinxAtStartPar
text


\subsection{Size and Color}
\label{\detokenize{colab_text:size-and-color}}\begin{itemize}
\item {} 
\sphinxAtStartPar
You can use the following HTML code to change the color and size of text:\\
\sphinxcode{\sphinxupquote{ <font size=25, color=red> text </font>}}.

\item {} 
\sphinxAtStartPar
Output:

\end{itemize}

\sphinxAtStartPar
text


\subsection{List}
\label{\detokenize{colab_text:list}}\begin{itemize}
\item {} 
\sphinxAtStartPar
Use a dash (\sphinxhyphen{}), asterisk (*), or plus sign (+) followed by a space for an unordered (bullet) list.

\item {} 
\sphinxAtStartPar
Use numbers followed by a period (.) and a space for a numbered (ordered) list.

\item {} 
\sphinxAtStartPar
Indent to create a sublist.

\end{itemize}

\sphinxAtStartPar
The output of

\begin{sphinxVerbatim}[commandchars=\\\{\}]
\PYG{k}{\PYGZhy{}}\PYG{+w}{ }Africa
\PYG{k}{\PYGZhy{}}\PYG{+w}{ }Europe
\end{sphinxVerbatim}

\sphinxAtStartPar
is
\begin{itemize}
\item {} 
\sphinxAtStartPar
Africa

\item {} 
\sphinxAtStartPar
Europe

\end{itemize}


\bigskip\hrule\bigskip


\sphinxAtStartPar
The output of

\begin{sphinxVerbatim}[commandchars=\\\{\}]
\PYG{k}{\PYGZhy{}}\PYG{+w}{ }Africa
\PYG{+w}{    }\PYG{k}{1.} Nigeria
\PYG{+w}{    }\PYG{k}{2.} Kenya
\PYG{k}{\PYGZhy{}}\PYG{+w}{ }Europe
\PYG{+w}{    }\PYG{k}{1.} Germany
\PYG{+w}{    }\PYG{k}{2.} Italy
\PYG{+w}{    }\PYG{k}{3.} Spain
\end{sphinxVerbatim}

\sphinxAtStartPar
is
\begin{itemize}
\item {} 
\sphinxAtStartPar
Africa
\begin{enumerate}
\sphinxsetlistlabels{\arabic}{enumi}{enumii}{}{.}%
\item {} 
\sphinxAtStartPar
Nigeria

\item {} 
\sphinxAtStartPar
Kenya

\end{enumerate}

\item {} 
\sphinxAtStartPar
Europe
\begin{enumerate}
\sphinxsetlistlabels{\arabic}{enumi}{enumii}{}{.}%
\item {} 
\sphinxAtStartPar
Germany

\item {} 
\sphinxAtStartPar
Italy

\item {} 
\sphinxAtStartPar
Spain

\end{enumerate}

\end{itemize}


\bigskip\hrule\bigskip


\sphinxAtStartPar
The output of

\begin{sphinxVerbatim}[commandchars=\\\{\}]
\PYG{k}{\PYGZhy{}}\PYG{+w}{ }Africa
\PYG{+w}{    }\PYG{k}{1.} Nigeria
\PYG{+w}{    }\PYG{k}{2.} Kenya
\PYG{k}{\PYGZhy{}}\PYG{+w}{ }Europe
\PYG{+w}{    }\PYG{k}{1.} Germany
\PYG{+w}{        }\PYG{k}{\PYGZhy{}}\PYG{+w}{ }Munich
\PYG{+w}{        }\PYG{k}{\PYGZhy{}}\PYG{+w}{ }Dordmund
\PYG{+w}{        }\PYG{k}{\PYGZhy{}}\PYG{+w}{ }Bonn
\PYG{+w}{    }\PYG{k}{2.} Italy
\PYG{+w}{    }\PYG{k}{3.} Spain
\end{sphinxVerbatim}

\sphinxAtStartPar
is
\begin{itemize}
\item {} 
\sphinxAtStartPar
Africa
\begin{enumerate}
\sphinxsetlistlabels{\arabic}{enumi}{enumii}{}{.}%
\item {} 
\sphinxAtStartPar
Nigeria

\item {} 
\sphinxAtStartPar
Kenya

\end{enumerate}

\item {} 
\sphinxAtStartPar
Europe
\begin{enumerate}
\sphinxsetlistlabels{\arabic}{enumi}{enumii}{}{.}%
\item {} 
\sphinxAtStartPar
Germany
\begin{itemize}
\item {} 
\sphinxAtStartPar
Munich

\item {} 
\sphinxAtStartPar
Dordmund

\item {} 
\sphinxAtStartPar
Bonn

\end{itemize}

\item {} 
\sphinxAtStartPar
Italy

\item {} 
\sphinxAtStartPar
Spain

\end{enumerate}

\end{itemize}
\begin{itemize}
\item {} 
\sphinxAtStartPar
Instead of dash (\sphinxhyphen{}) you can also use asterisk (*).

\end{itemize}

\sphinxAtStartPar
The output of

\begin{sphinxVerbatim}[commandchars=\\\{\}]
\PYG{k}{*}\PYG{+w}{ }Africa
\PYG{k}{*}\PYG{+w}{ }Europe
\end{sphinxVerbatim}

\sphinxAtStartPar
is
\begin{itemize}
\item {} 
\sphinxAtStartPar
Africa

\item {} 
\sphinxAtStartPar
Europe

\end{itemize}


\bigskip\hrule\bigskip



\subsection{Highlight}
\label{\detokenize{colab_text:highlight}}\begin{itemize}
\item {} 
\sphinxAtStartPar
Enclose the text with one backtick on each side (`text`) to format it.

\item {} 
\sphinxAtStartPar
In a light Google Colab theme, the highlight might not be very visible.

\item {} 
\sphinxAtStartPar
For example, adding one backtick to each side of How are you? will display it as \sphinxcode{\sphinxupquote{How are you?}}.

\end{itemize}


\subsection{Horizontal Line}
\label{\detokenize{colab_text:horizontal-line}}\begin{itemize}
\item {} 
\sphinxAtStartPar
Use three asterisks (***) or underscores (\_\_\_) on a new line.

\end{itemize}


\bigskip\hrule\bigskip

\begin{itemize}
\item {} 
\sphinxAtStartPar
In light mode, the horizontal line created by three asterisks (***) or underscores (\_\_\_) might not be very noticeable.

\end{itemize}


\subsection{Blockquote}
\label{\detokenize{colab_text:blockquote}}\begin{itemize}
\item {} 
\sphinxAtStartPar
To create a blockquote, start a line with a greater than symbol (>).

\item {} 
\sphinxAtStartPar
Use additional greater than symbols for nested blockquotes.

\end{itemize}

\sphinxAtStartPar
The output of

\begin{sphinxVerbatim}[commandchars=\\\{\}]
\PYG{k}{\PYGZgt{} }\PYG{g+ge}{aaa  }
\end{sphinxVerbatim}

\sphinxAtStartPar
is
\begin{quote}

\sphinxAtStartPar
aaa
\end{quote}


\bigskip\hrule\bigskip


\sphinxAtStartPar
The output of

\begin{sphinxVerbatim}[commandchars=\\\{\}]
\PYG{k}{\PYGZgt{} }\PYG{g+ge}{aaa \PYGZbs{}}
\PYG{k}{\PYGZgt{} }\PYG{g+ge}{bbb}
\end{sphinxVerbatim}

\sphinxAtStartPar
is
\begin{quote}

\sphinxAtStartPar
aaa\\
bbb
\end{quote}


\bigskip\hrule\bigskip


\sphinxAtStartPar
The output of

\begin{sphinxVerbatim}[commandchars=\\\{\}]
\PYG{k}{\PYGZgt{} }\PYG{g+ge}{aaa }
\PYGZgt{}\PYGZgt{} bbb
\end{sphinxVerbatim}

\sphinxAtStartPar
is
\begin{quote}

\sphinxAtStartPar
aaa
\begin{quote}

\sphinxAtStartPar
bbb
\end{quote}
\end{quote}


\bigskip\hrule\bigskip


\sphinxAtStartPar
The output of

\begin{sphinxVerbatim}[commandchars=\\\{\}]
\PYG{k}{\PYGZgt{} }\PYG{g+ge}{aaa  }
\PYG{k}{\PYGZgt{} }\PYG{g+ge}{bbb}
\PYGZgt{}\PYGZgt{} ccc
\PYGZgt{}\PYGZgt{}\PYGZgt{}ddd
\end{sphinxVerbatim}

\sphinxAtStartPar
is
\begin{quote}

\sphinxAtStartPar
aaa\\
bbb
\begin{quote}

\sphinxAtStartPar
ccc
\begin{quote}

\sphinxAtStartPar
ddd
\end{quote}
\end{quote}
\end{quote}


\subsection{Link}
\label{\detokenize{colab_text:link}}\begin{itemize}
\item {} 
\sphinxAtStartPar
Use the following Markdown syntax to create a link to a website: \sphinxcode{\sphinxupquote{ {[}title{]}(URL)}}

\item {} 
\sphinxAtStartPar
The output of \sphinxcode{\sphinxupquote{ {[}Python{]}(https://www.python.org)}}
is \sphinxhref{https://www.python.org}{Python}

\end{itemize}


\subsection{Table}
\label{\detokenize{colab_text:table}}\begin{itemize}
\item {} 
\sphinxAtStartPar
Use | (pipe) and \sphinxhyphen{} (dash) to create a table structure.

\end{itemize}

\sphinxAtStartPar
The output of

\begin{sphinxVerbatim}[commandchars=\\\{\}]
|Gender	| Age | State|  
| \PYGZhy{}\PYGZhy{}\PYGZhy{} | \PYGZhy{}\PYGZhy{}\PYGZhy{} | \PYGZhy{}\PYGZhy{}\PYGZhy{} |  
| Male | 25 | NY |  
| Male | 35 | FL |.
| Female | 20 | CA | 
\end{sphinxVerbatim}

\sphinxAtStartPar
is


\begin{savenotes}\sphinxattablestart
\centering
\begin{tabulary}{\linewidth}[t]{|T|T|T|}
\hline
\sphinxstyletheadfamily 
\sphinxAtStartPar
Gender
&\sphinxstyletheadfamily 
\sphinxAtStartPar
Age
&\sphinxstyletheadfamily 
\sphinxAtStartPar
State
\\
\hline
\sphinxAtStartPar
Male
&
\sphinxAtStartPar
25
&
\sphinxAtStartPar
NY
\\
\hline
\sphinxAtStartPar
Male
&
\sphinxAtStartPar
35
&
\sphinxAtStartPar
FL
\\
\hline
\sphinxAtStartPar
Female
&
\sphinxAtStartPar
20
&
\sphinxAtStartPar
CA
\\
\hline
\end{tabulary}
\par
\sphinxattableend\end{savenotes}


\subsection{Python code}
\label{\detokenize{colab_text:python-code}}\begin{itemize}
\item {} 
\sphinxAtStartPar
Use three backticks followed by the Python keyword to write Python code within a text cell, mimicking the appearance of a code cell.”

\end{itemize}

\sphinxAtStartPar
The output of

\sphinxAtStartPar
``` python\\
for i in range(6):\\
   if i>2:\\
        print(i)\\
  else:\\
        print(True)\\
```

\sphinxAtStartPar
is

\begin{sphinxVerbatim}[commandchars=\\\{\}]
\PYG{k}{for} \PYG{n}{i} \PYG{o+ow}{in} \PYG{n+nb}{range}\PYG{p}{(}\PYG{l+m+mi}{6}\PYG{p}{)}\PYG{p}{:}
  \PYG{k}{if} \PYG{n}{i}\PYG{o}{\PYGZgt{}}\PYG{l+m+mi}{2}\PYG{p}{:}
    \PYG{n+nb}{print}\PYG{p}{(}\PYG{n}{i}\PYG{p}{)}
  \PYG{k}{else}\PYG{p}{:}
    \PYG{n+nb}{print}\PYG{p}{(}\PYG{k+kc}{True}\PYG{p}{)}
\end{sphinxVerbatim}


\subsection{Mathematical Equations}
\label{\detokenize{colab_text:mathematical-equations}}\begin{itemize}
\item {} 
\sphinxAtStartPar
A text cell can execute Latex code.

\item {} 
\sphinxAtStartPar
If you re not familiar with Latex, feel free to skip this part.

\item {} 
\sphinxAtStartPar
The output of \sphinxcode{\sphinxupquote{\$ \textbackslash{}int\_1\textasciicircum{}5x\textasciicircum{}5\textbackslash{},dx\$}} is \(\int_1^5x^5\,dx\)

\end{itemize}


\subsection{Add an Image}
\label{\detokenize{colab_text:add-an-image}}\begin{enumerate}
\sphinxsetlistlabels{\arabic}{enumi}{enumii}{}{.}%
\item {} 
\sphinxAtStartPar
To import an image from your computer to a Colab notebook:
\begin{itemize}
\item {} 
\sphinxAtStartPar
Double\sphinxhyphen{}click on the text cell to edit it.

\item {} 
\sphinxAtStartPar
Click on the image icon in the text cell’s toolbar.

\item {} 
\sphinxAtStartPar
Select the image you want to upload.

\item {} 
\sphinxAtStartPar
You might see a lengthy and odd\sphinxhyphen{}looking text, but you can ignore it.

\item {} 
\sphinxAtStartPar
Click outside the text cell to exit the edit mode.

\end{itemize}

\item {} 
\sphinxAtStartPar
To import an image from an URL use the following markdown syntax:
\sphinxcode{\sphinxupquote{!{[}{]}(URL)}}

\end{enumerate}
\begin{itemize}
\item {} 
\sphinxAtStartPar
The output of \sphinxcode{\sphinxupquote{!{[}{]}(https://www.python.org/static/community\_logos/python\sphinxhyphen{}logo\sphinxhyphen{}master\sphinxhyphen{}v3\sphinxhyphen{}TM.png)}}
is
\sphinxincludegraphics{{python-powered-h-100x130}.png}

\end{itemize}
\begin{enumerate}
\sphinxsetlistlabels{\arabic}{enumi}{enumii}{}{.}%
\setcounter{enumi}{2}
\item {} 
\sphinxAtStartPar
If you remove the \sphinxcode{\sphinxupquote{!}} from the code above, it will create a link to the website of the image instead of displaying the image directly in the notebook.

\end{enumerate}
\begin{itemize}
\item {} 
\sphinxAtStartPar
The output of \sphinxcode{\sphinxupquote{{[}python logo{]}(https://www.python.org/static/community\_logos/python\sphinxhyphen{}powered\sphinxhyphen{}h\sphinxhyphen{}100x130.png)}}
is
\sphinxhref{https://www.python.org/static/community\_logos/python-powered-h-100x130.png}{python logo}

\end{itemize}

\sphinxstepscope


\subsection{Text Cell Questions}
\label{\detokenize{colab_text_questions:text-cell-questions}}\label{\detokenize{colab_text_questions::doc}}
\sphinxAtStartPar
Please type the passages into text cells following the provided format.


\subsubsection{Question}
\label{\detokenize{colab_text_questions:question}}
\sphinxAtStartPar
\sphinxincludegraphics{{colab_text_exer1}.png}

\begin{sphinxadmonition}{note}{Solution}

\sphinxAtStartPar
\sphinxincludegraphics{{colab_text_exer1_sol}.png}
\end{sphinxadmonition}


\subsubsection{Question}
\label{\detokenize{colab_text_questions:id1}}
\sphinxAtStartPar
\sphinxincludegraphics{{colab_text_exer2}.png}

\begin{sphinxadmonition}{note}{Solution}

\sphinxAtStartPar
\sphinxincludegraphics{{colab_text_exer2_sol}.png}
\end{sphinxadmonition}

\sphinxstepscope


\section{Code Cell}
\label{\detokenize{colab_code:code-cell}}\label{\detokenize{colab_code::doc}}
\sphinxAtStartPar
You can write and execute Python code in code cells.
\begin{itemize}
\item {} 
\sphinxAtStartPar
An arrow on the left of a code cell allows you to run the code.

\item {} 
\sphinxAtStartPar
Additionally, you can use the shortcut \sphinxcode{\sphinxupquote{Shift+Enter}} to execute the code.

\item {} 
\sphinxAtStartPar
A number in a square bracket on the left of executed code cells represents the execution order of that code cell.

\item {} 
\sphinxAtStartPar
Warning: The order of a code cell in the notebook and its execution order can differ if you run another code cell above it

\end{itemize}


\subsection{print() function}
\label{\detokenize{colab_code:print-function}}\begin{itemize}
\item {} 
\sphinxAtStartPar
Displays output on the screen.

\item {} 
\sphinxAtStartPar
In Colab notebooks, only the variable value in the last line is automatically displayed.

\item {} 
\sphinxAtStartPar
To display any value in another line, you need to use the print() function.

\end{itemize}

\begin{sphinxuseclass}{cell}\begin{sphinxVerbatimInput}

\begin{sphinxuseclass}{cell_input}
\begin{sphinxVerbatim}[commandchars=\\\{\}]
\PYG{c+c1}{\PYGZsh{} displays 5}
\PYG{n+nb}{print}\PYG{p}{(}\PYG{l+m+mi}{5}\PYG{p}{)}
\end{sphinxVerbatim}

\end{sphinxuseclass}\end{sphinxVerbatimInput}
\begin{sphinxVerbatimOutput}

\begin{sphinxuseclass}{cell_output}
\begin{sphinxVerbatim}[commandchars=\\\{\}]
5
\end{sphinxVerbatim}

\end{sphinxuseclass}\end{sphinxVerbatimOutput}

\end{sphinxuseclass}
\begin{sphinxuseclass}{cell}\begin{sphinxVerbatimInput}

\begin{sphinxuseclass}{cell_input}
\begin{sphinxVerbatim}[commandchars=\\\{\}]
\PYG{c+c1}{\PYGZsh{} notebook only displays the value in the last line of the code cell}
\PYG{l+m+mi}{5}\PYG{o}{\PYGZhy{}}\PYG{l+m+mi}{1}  \PYG{c+c1}{\PYGZsh{} not displayed}
\PYG{l+m+mi}{3}\PYG{o}{+}\PYG{l+m+mi}{7}  \PYG{c+c1}{\PYGZsh{} displayed}
\end{sphinxVerbatim}

\end{sphinxuseclass}\end{sphinxVerbatimInput}
\begin{sphinxVerbatimOutput}

\begin{sphinxuseclass}{cell_output}
\begin{sphinxVerbatim}[commandchars=\\\{\}]
10
\end{sphinxVerbatim}

\end{sphinxuseclass}\end{sphinxVerbatimOutput}

\end{sphinxuseclass}
\begin{sphinxuseclass}{cell}\begin{sphinxVerbatimInput}

\begin{sphinxuseclass}{cell_input}
\begin{sphinxVerbatim}[commandchars=\\\{\}]
\PYG{c+c1}{\PYGZsh{} print() displays the outputs}
\PYG{n+nb}{print}\PYG{p}{(}\PYG{l+m+mi}{5}\PYG{o}{\PYGZhy{}}\PYG{l+m+mi}{1}\PYG{p}{)}  \PYG{c+c1}{\PYGZsh{}  displayed}
\PYG{n+nb}{print}\PYG{p}{(}\PYG{l+m+mi}{3}\PYG{o}{+}\PYG{l+m+mi}{7}\PYG{p}{)}  \PYG{c+c1}{\PYGZsh{} displayed}
\end{sphinxVerbatim}

\end{sphinxuseclass}\end{sphinxVerbatimInput}
\begin{sphinxVerbatimOutput}

\begin{sphinxuseclass}{cell_output}
\begin{sphinxVerbatim}[commandchars=\\\{\}]
4
10
\end{sphinxVerbatim}

\end{sphinxuseclass}\end{sphinxVerbatimOutput}

\end{sphinxuseclass}

\subsection{Numbers}
\label{\detokenize{colab_code:numbers}}\begin{itemize}
\item {} 
\sphinxAtStartPar
You can use integers (…,\sphinxhyphen{}3,\sphinxhyphen{}2,\sphinxhyphen{}1,0,1,2,3,….) and decimal numbers called floats in a code cell.

\end{itemize}


\subsubsection{Algebraic Operations}
\label{\detokenize{colab_code:algebraic-operations}}\begin{itemize}
\item {} 
\sphinxAtStartPar
You can perform algebraic operations in code cells.

\item {} 
\sphinxAtStartPar
The multiplication operator is \sphinxcode{\sphinxupquote{*}} (asterisk).

\item {} 
\sphinxAtStartPar
The division operator is \sphinxcode{\sphinxupquote{/}} (forward slash).

\item {} 
\sphinxAtStartPar
The exponent operator is \sphinxcode{\sphinxupquote{**}} (double asterisk).

\end{itemize}

\begin{sphinxuseclass}{cell}\begin{sphinxVerbatimInput}

\begin{sphinxuseclass}{cell_input}
\begin{sphinxVerbatim}[commandchars=\\\{\}]
\PYG{n+nb}{print}\PYG{p}{(}\PYG{l+m+mi}{5}\PYG{o}{+}\PYG{l+m+mi}{9}\PYG{p}{)}
\end{sphinxVerbatim}

\end{sphinxuseclass}\end{sphinxVerbatimInput}
\begin{sphinxVerbatimOutput}

\begin{sphinxuseclass}{cell_output}
\begin{sphinxVerbatim}[commandchars=\\\{\}]
14
\end{sphinxVerbatim}

\end{sphinxuseclass}\end{sphinxVerbatimOutput}

\end{sphinxuseclass}
\begin{sphinxuseclass}{cell}\begin{sphinxVerbatimInput}

\begin{sphinxuseclass}{cell_input}
\begin{sphinxVerbatim}[commandchars=\\\{\}]
\PYG{n+nb}{print}\PYG{p}{(}\PYG{l+m+mf}{10.4}\PYG{o}{\PYGZhy{}}\PYG{l+m+mf}{4.5}\PYG{p}{)}
\end{sphinxVerbatim}

\end{sphinxuseclass}\end{sphinxVerbatimInput}
\begin{sphinxVerbatimOutput}

\begin{sphinxuseclass}{cell_output}
\begin{sphinxVerbatim}[commandchars=\\\{\}]
5.9
\end{sphinxVerbatim}

\end{sphinxuseclass}\end{sphinxVerbatimOutput}

\end{sphinxuseclass}
\begin{sphinxuseclass}{cell}\begin{sphinxVerbatimInput}

\begin{sphinxuseclass}{cell_input}
\begin{sphinxVerbatim}[commandchars=\\\{\}]
\PYG{n+nb}{print}\PYG{p}{(}\PYG{l+m+mi}{5}\PYG{o}{*}\PYG{l+m+mi}{6}\PYG{p}{)}
\end{sphinxVerbatim}

\end{sphinxuseclass}\end{sphinxVerbatimInput}
\begin{sphinxVerbatimOutput}

\begin{sphinxuseclass}{cell_output}
\begin{sphinxVerbatim}[commandchars=\\\{\}]
30
\end{sphinxVerbatim}

\end{sphinxuseclass}\end{sphinxVerbatimOutput}

\end{sphinxuseclass}
\begin{sphinxuseclass}{cell}\begin{sphinxVerbatimInput}

\begin{sphinxuseclass}{cell_input}
\begin{sphinxVerbatim}[commandchars=\\\{\}]
\PYG{n+nb}{print}\PYG{p}{(}\PYG{l+m+mi}{10}\PYG{o}{/}\PYG{l+m+mi}{2}\PYG{p}{)}
\end{sphinxVerbatim}

\end{sphinxuseclass}\end{sphinxVerbatimInput}
\begin{sphinxVerbatimOutput}

\begin{sphinxuseclass}{cell_output}
\begin{sphinxVerbatim}[commandchars=\\\{\}]
5.0
\end{sphinxVerbatim}

\end{sphinxuseclass}\end{sphinxVerbatimOutput}

\end{sphinxuseclass}
\begin{sphinxuseclass}{cell}\begin{sphinxVerbatimInput}

\begin{sphinxuseclass}{cell_input}
\begin{sphinxVerbatim}[commandchars=\\\{\}]
\PYG{n+nb}{print}\PYG{p}{(}\PYG{l+m+mi}{2}\PYG{o}{*}\PYG{o}{*}\PYG{l+m+mi}{3}\PYG{p}{)}
\end{sphinxVerbatim}

\end{sphinxuseclass}\end{sphinxVerbatimInput}
\begin{sphinxVerbatimOutput}

\begin{sphinxuseclass}{cell_output}
\begin{sphinxVerbatim}[commandchars=\\\{\}]
8
\end{sphinxVerbatim}

\end{sphinxuseclass}\end{sphinxVerbatimOutput}

\end{sphinxuseclass}

\subsection{Strings}
\label{\detokenize{colab_code:strings}}\begin{itemize}
\item {} 
\sphinxAtStartPar
Strings are sequences of characters.

\item {} 
\sphinxAtStartPar
They can be written inside single quotes, double quotes, triple single quotes, or triple double quotes.

\end{itemize}

\begin{sphinxuseclass}{cell}\begin{sphinxVerbatimInput}

\begin{sphinxuseclass}{cell_input}
\begin{sphinxVerbatim}[commandchars=\\\{\}]
\PYG{n+nb}{print}\PYG{p}{(}\PYG{l+s+s1}{\PYGZsq{}}\PYG{l+s+s1}{Hello}\PYG{l+s+s1}{\PYGZsq{}}\PYG{p}{)}
\end{sphinxVerbatim}

\end{sphinxuseclass}\end{sphinxVerbatimInput}
\begin{sphinxVerbatimOutput}

\begin{sphinxuseclass}{cell_output}
\begin{sphinxVerbatim}[commandchars=\\\{\}]
Hello
\end{sphinxVerbatim}

\end{sphinxuseclass}\end{sphinxVerbatimOutput}

\end{sphinxuseclass}
\begin{sphinxuseclass}{cell}\begin{sphinxVerbatimInput}

\begin{sphinxuseclass}{cell_input}
\begin{sphinxVerbatim}[commandchars=\\\{\}]
\PYG{n+nb}{print}\PYG{p}{(}\PYG{l+s+s2}{\PYGZdq{}}\PYG{l+s+s2}{Hello}\PYG{l+s+s2}{\PYGZdq{}}\PYG{p}{)}
\end{sphinxVerbatim}

\end{sphinxuseclass}\end{sphinxVerbatimInput}
\begin{sphinxVerbatimOutput}

\begin{sphinxuseclass}{cell_output}
\begin{sphinxVerbatim}[commandchars=\\\{\}]
Hello
\end{sphinxVerbatim}

\end{sphinxuseclass}\end{sphinxVerbatimOutput}

\end{sphinxuseclass}
\begin{sphinxuseclass}{cell}\begin{sphinxVerbatimInput}

\begin{sphinxuseclass}{cell_input}
\begin{sphinxVerbatim}[commandchars=\\\{\}]
\PYG{n+nb}{print}\PYG{p}{(}\PYG{l+s+s1}{\PYGZsq{}\PYGZsq{}\PYGZsq{}}\PYG{l+s+s1}{Hello}\PYG{l+s+s1}{\PYGZsq{}\PYGZsq{}\PYGZsq{}}\PYG{p}{)}
\end{sphinxVerbatim}

\end{sphinxuseclass}\end{sphinxVerbatimInput}
\begin{sphinxVerbatimOutput}

\begin{sphinxuseclass}{cell_output}
\begin{sphinxVerbatim}[commandchars=\\\{\}]
Hello
\end{sphinxVerbatim}

\end{sphinxuseclass}\end{sphinxVerbatimOutput}

\end{sphinxuseclass}
\begin{sphinxuseclass}{cell}\begin{sphinxVerbatimInput}

\begin{sphinxuseclass}{cell_input}
\begin{sphinxVerbatim}[commandchars=\\\{\}]
\PYG{n+nb}{print}\PYG{p}{(}\PYG{l+s+s2}{\PYGZdq{}\PYGZdq{}\PYGZdq{}}\PYG{l+s+s2}{Hello}\PYG{l+s+s2}{\PYGZdq{}\PYGZdq{}\PYGZdq{}}\PYG{p}{)}
\end{sphinxVerbatim}

\end{sphinxuseclass}\end{sphinxVerbatimInput}
\begin{sphinxVerbatimOutput}

\begin{sphinxuseclass}{cell_output}
\begin{sphinxVerbatim}[commandchars=\\\{\}]
Hello
\end{sphinxVerbatim}

\end{sphinxuseclass}\end{sphinxVerbatimOutput}

\end{sphinxuseclass}
\begin{sphinxuseclass}{cell}\begin{sphinxVerbatimInput}

\begin{sphinxuseclass}{cell_input}
\begin{sphinxVerbatim}[commandchars=\\\{\}]
\PYG{l+s+s1}{\PYGZsq{}}\PYG{l+s+s1}{3}\PYG{l+s+s1}{\PYGZsq{}}   \PYG{c+c1}{\PYGZsh{} string \PYGZsq{}3 \PYGZsq{}is not a number, it is only a character}
\end{sphinxVerbatim}

\end{sphinxuseclass}\end{sphinxVerbatimInput}
\begin{sphinxVerbatimOutput}

\begin{sphinxuseclass}{cell_output}
\begin{sphinxVerbatim}[commandchars=\\\{\}]
\PYGZsq{}3\PYGZsq{}
\end{sphinxVerbatim}

\end{sphinxuseclass}\end{sphinxVerbatimOutput}

\end{sphinxuseclass}
\begin{sphinxVerbatim}[commandchars=\\\{\}]
\PYG{c+c1}{\PYGZsh{} ERROR: you are trying to add a string (character) \PYGZsq{}3\PYGZsq{} and a number (5) }
\PYG{l+s+s1}{\PYGZsq{}}\PYG{l+s+s1}{3}\PYG{l+s+s1}{\PYGZsq{}}\PYG{o}{+}\PYG{l+m+mi}{5}   
\end{sphinxVerbatim}


\subsubsection{String Operations}
\label{\detokenize{colab_code:string-operations}}\begin{itemize}
\item {} 
\sphinxAtStartPar
\sphinxcode{\sphinxupquote{String + String}}:  combines two strings

\item {} 
\sphinxAtStartPar
\sphinxcode{\sphinxupquote{String * Integer}} or \sphinxcode{\sphinxupquote{Integer * String}}: makes copies of the string \sphinxcode{\sphinxupquote{Integer}} many times.

\end{itemize}

\begin{sphinxuseclass}{cell}\begin{sphinxVerbatimInput}

\begin{sphinxuseclass}{cell_input}
\begin{sphinxVerbatim}[commandchars=\\\{\}]
\PYG{c+c1}{\PYGZsh{} concatenation: combining two strings}
\PYG{n+nb}{print}\PYG{p}{(}\PYG{l+s+s1}{\PYGZsq{}}\PYG{l+s+s1}{Hello}\PYG{l+s+s1}{\PYGZsq{}} \PYG{o}{+} \PYG{l+s+s1}{\PYGZsq{}}\PYG{l+s+s1}{World}\PYG{l+s+s1}{\PYGZsq{}}\PYG{p}{)}
\end{sphinxVerbatim}

\end{sphinxuseclass}\end{sphinxVerbatimInput}
\begin{sphinxVerbatimOutput}

\begin{sphinxuseclass}{cell_output}
\begin{sphinxVerbatim}[commandchars=\\\{\}]
HelloWorld
\end{sphinxVerbatim}

\end{sphinxuseclass}\end{sphinxVerbatimOutput}

\end{sphinxuseclass}
\begin{sphinxuseclass}{cell}\begin{sphinxVerbatimInput}

\begin{sphinxuseclass}{cell_input}
\begin{sphinxVerbatim}[commandchars=\\\{\}]
\PYG{c+c1}{\PYGZsh{} repetition: \PYGZsh{} 10 Ms}
\PYG{n+nb}{print}\PYG{p}{(}\PYG{l+s+s1}{\PYGZsq{}}\PYG{l+s+s1}{M}\PYG{l+s+s1}{\PYGZsq{}}\PYG{o}{*}\PYG{l+m+mi}{10}\PYG{p}{)}
\end{sphinxVerbatim}

\end{sphinxuseclass}\end{sphinxVerbatimInput}
\begin{sphinxVerbatimOutput}

\begin{sphinxuseclass}{cell_output}
\begin{sphinxVerbatim}[commandchars=\\\{\}]
MMMMMMMMMM
\end{sphinxVerbatim}

\end{sphinxuseclass}\end{sphinxVerbatimOutput}

\end{sphinxuseclass}

\subsection{Booleans}
\label{\detokenize{colab_code:booleans}}\begin{itemize}
\item {} 
\sphinxAtStartPar
There are only two boolean values: \sphinxcode{\sphinxupquote{True}} and \sphinxcode{\sphinxupquote{False}}.

\item {} 
\sphinxAtStartPar
They indicate whether a condition is valid or not.

\item {} 
\sphinxAtStartPar
Booleans will be covered in Conditionals chapter with details.

\end{itemize}


\subsection{Comments}
\label{\detokenize{colab_code:comments}}\begin{itemize}
\item {} 
\sphinxAtStartPar
Use \sphinxcode{\sphinxupquote{\#}}  to add notes to your program, explaining the purpose of the code.

\item {} 
\sphinxAtStartPar
Comments are not executed but serve as explanations.

\item {} 
\sphinxAtStartPar
Additionally, you can place comments to the right of the code.

\item {} 
\sphinxAtStartPar
Comments make the code easier to read and understand.

\item {} 
\sphinxAtStartPar
Comments help team members understand each other’s code.

\end{itemize}

\begin{sphinxuseclass}{cell}\begin{sphinxVerbatimInput}

\begin{sphinxuseclass}{cell_input}
\begin{sphinxVerbatim}[commandchars=\\\{\}]
\PYG{c+c1}{\PYGZsh{} this is a comment: addition}
\PYG{n+nb}{print}\PYG{p}{(}\PYG{l+m+mi}{2}\PYG{o}{+}\PYG{l+m+mi}{8}\PYG{p}{)}
\end{sphinxVerbatim}

\end{sphinxuseclass}\end{sphinxVerbatimInput}
\begin{sphinxVerbatimOutput}

\begin{sphinxuseclass}{cell_output}
\begin{sphinxVerbatim}[commandchars=\\\{\}]
10
\end{sphinxVerbatim}

\end{sphinxuseclass}\end{sphinxVerbatimOutput}

\end{sphinxuseclass}
\begin{sphinxuseclass}{cell}\begin{sphinxVerbatimInput}

\begin{sphinxuseclass}{cell_input}
\begin{sphinxVerbatim}[commandchars=\\\{\}]
\PYG{n+nb}{print}\PYG{p}{(}\PYG{l+m+mi}{2}\PYG{o}{+}\PYG{l+m+mi}{8}\PYG{p}{)}  \PYG{c+c1}{\PYGZsh{} addition}
\end{sphinxVerbatim}

\end{sphinxuseclass}\end{sphinxVerbatimInput}
\begin{sphinxVerbatimOutput}

\begin{sphinxuseclass}{cell_output}
\begin{sphinxVerbatim}[commandchars=\\\{\}]
10
\end{sphinxVerbatim}

\end{sphinxuseclass}\end{sphinxVerbatimOutput}

\end{sphinxuseclass}

\subsection{Built\sphinxhyphen{}in functions}
\label{\detokenize{colab_code:built-in-functions}}\begin{itemize}
\item {} 
\sphinxAtStartPar
There are functions readily available for use in a notebook.

\item {} 
\sphinxAtStartPar
There is no need to import these built\sphinxhyphen{}in functions from external sources.

\item {} 
\sphinxAtStartPar
You can find the list of built\sphinxhyphen{}in functions in the following \sphinxhref{https://docs.python.org/3/library/functions.html}{link}.

\end{itemize}

\sphinxAtStartPar
\sphinxincludegraphics{{built_in_func}.png}

\begin{sphinxuseclass}{cell}\begin{sphinxVerbatimInput}

\begin{sphinxuseclass}{cell_input}
\begin{sphinxVerbatim}[commandchars=\\\{\}]
\PYG{c+c1}{\PYGZsh{} absolute value function}
\PYG{n+nb}{print}\PYG{p}{(}\PYG{n+nb}{abs}\PYG{p}{(}\PYG{o}{\PYGZhy{}}\PYG{l+m+mi}{7}\PYG{p}{)}\PYG{p}{)}
\end{sphinxVerbatim}

\end{sphinxuseclass}\end{sphinxVerbatimInput}
\begin{sphinxVerbatimOutput}

\begin{sphinxuseclass}{cell_output}
\begin{sphinxVerbatim}[commandchars=\\\{\}]
7
\end{sphinxVerbatim}

\end{sphinxuseclass}\end{sphinxVerbatimOutput}

\end{sphinxuseclass}
\begin{sphinxuseclass}{cell}\begin{sphinxVerbatimInput}

\begin{sphinxuseclass}{cell_input}
\begin{sphinxVerbatim}[commandchars=\\\{\}]
\PYG{c+c1}{\PYGZsh{} maximum of a list of numbers}
\PYG{n+nb}{print}\PYG{p}{(}\PYG{n+nb}{max}\PYG{p}{(}\PYG{l+m+mi}{3}\PYG{p}{,}\PYG{o}{\PYGZhy{}}\PYG{l+m+mi}{4}\PYG{p}{,}\PYG{l+m+mi}{0}\PYG{p}{,}\PYG{l+m+mi}{9}\PYG{p}{,}\PYG{l+m+mi}{2}\PYG{p}{)}\PYG{p}{)}
\end{sphinxVerbatim}

\end{sphinxuseclass}\end{sphinxVerbatimInput}
\begin{sphinxVerbatimOutput}

\begin{sphinxuseclass}{cell_output}
\begin{sphinxVerbatim}[commandchars=\\\{\}]
9
\end{sphinxVerbatim}

\end{sphinxuseclass}\end{sphinxVerbatimOutput}

\end{sphinxuseclass}
\begin{sphinxuseclass}{cell}\begin{sphinxVerbatimInput}

\begin{sphinxuseclass}{cell_input}
\begin{sphinxVerbatim}[commandchars=\\\{\}]
\PYG{c+c1}{\PYGZsh{} minimum of a list of numbers}
\PYG{n+nb}{print}\PYG{p}{(}\PYG{n+nb}{min}\PYG{p}{(}\PYG{l+m+mi}{3}\PYG{p}{,}\PYG{o}{\PYGZhy{}}\PYG{l+m+mi}{4}\PYG{p}{,}\PYG{l+m+mi}{0}\PYG{p}{,}\PYG{l+m+mi}{9}\PYG{p}{,}\PYG{l+m+mi}{2}\PYG{p}{)}\PYG{p}{)}
\end{sphinxVerbatim}

\end{sphinxuseclass}\end{sphinxVerbatimInput}
\begin{sphinxVerbatimOutput}

\begin{sphinxuseclass}{cell_output}
\begin{sphinxVerbatim}[commandchars=\\\{\}]
\PYGZhy{}4
\end{sphinxVerbatim}

\end{sphinxuseclass}\end{sphinxVerbatimOutput}

\end{sphinxuseclass}
\begin{sphinxuseclass}{cell}\begin{sphinxVerbatimInput}

\begin{sphinxuseclass}{cell_input}
\begin{sphinxVerbatim}[commandchars=\\\{\}]
\PYG{c+c1}{\PYGZsh{} sum of a list of numbers in a square bracket}
\PYG{n+nb}{print}\PYG{p}{(}\PYG{n+nb}{sum}\PYG{p}{(}\PYG{p}{[}\PYG{l+m+mi}{3}\PYG{p}{,}\PYG{l+m+mi}{5}\PYG{p}{,}\PYG{l+m+mi}{2}\PYG{p}{]}\PYG{p}{)}\PYG{p}{)}
\end{sphinxVerbatim}

\end{sphinxuseclass}\end{sphinxVerbatimInput}
\begin{sphinxVerbatimOutput}

\begin{sphinxuseclass}{cell_output}
\begin{sphinxVerbatim}[commandchars=\\\{\}]
10
\end{sphinxVerbatim}

\end{sphinxuseclass}\end{sphinxVerbatimOutput}

\end{sphinxuseclass}
\begin{sphinxuseclass}{cell}\begin{sphinxVerbatimInput}

\begin{sphinxuseclass}{cell_input}
\begin{sphinxVerbatim}[commandchars=\\\{\}]
\PYG{c+c1}{\PYGZsh{} rounding to the nearest hundreth}
\PYG{n+nb}{print}\PYG{p}{(}\PYG{n+nb}{round}\PYG{p}{(}\PYG{l+m+mf}{3.4678}\PYG{p}{,} \PYG{l+m+mi}{2}\PYG{p}{)}\PYG{p}{)}
\end{sphinxVerbatim}

\end{sphinxuseclass}\end{sphinxVerbatimInput}
\begin{sphinxVerbatimOutput}

\begin{sphinxuseclass}{cell_output}
\begin{sphinxVerbatim}[commandchars=\\\{\}]
3.47
\end{sphinxVerbatim}

\end{sphinxuseclass}\end{sphinxVerbatimOutput}

\end{sphinxuseclass}
\begin{sphinxuseclass}{cell}\begin{sphinxVerbatimInput}

\begin{sphinxuseclass}{cell_input}
\begin{sphinxVerbatim}[commandchars=\\\{\}]
\PYG{c+c1}{\PYGZsh{} notebook only displays the value in the last line of the code cell}
\PYG{l+m+mi}{5}\PYG{o}{\PYGZhy{}}\PYG{l+m+mi}{1}  \PYG{c+c1}{\PYGZsh{} not displayed}
\PYG{l+m+mi}{3}\PYG{o}{+}\PYG{l+m+mi}{7}  \PYG{c+c1}{\PYGZsh{} displayed}
\end{sphinxVerbatim}

\end{sphinxuseclass}\end{sphinxVerbatimInput}
\begin{sphinxVerbatimOutput}

\begin{sphinxuseclass}{cell_output}
\begin{sphinxVerbatim}[commandchars=\\\{\}]
10
\end{sphinxVerbatim}

\end{sphinxuseclass}\end{sphinxVerbatimOutput}

\end{sphinxuseclass}
\begin{sphinxuseclass}{cell}\begin{sphinxVerbatimInput}

\begin{sphinxuseclass}{cell_input}
\begin{sphinxVerbatim}[commandchars=\\\{\}]
\PYG{c+c1}{\PYGZsh{} print() displays the outputs}
\PYG{n+nb}{print}\PYG{p}{(}\PYG{l+m+mi}{5}\PYG{o}{\PYGZhy{}}\PYG{l+m+mi}{1}\PYG{p}{)}  \PYG{c+c1}{\PYGZsh{}  displayed}
\PYG{n+nb}{print}\PYG{p}{(}\PYG{l+m+mi}{3}\PYG{o}{+}\PYG{l+m+mi}{7}\PYG{p}{)}  \PYG{c+c1}{\PYGZsh{} displayed}
\end{sphinxVerbatim}

\end{sphinxuseclass}\end{sphinxVerbatimInput}
\begin{sphinxVerbatimOutput}

\begin{sphinxuseclass}{cell_output}
\begin{sphinxVerbatim}[commandchars=\\\{\}]
4
10
\end{sphinxVerbatim}

\end{sphinxuseclass}\end{sphinxVerbatimOutput}

\end{sphinxuseclass}

\subsection{Indentation}
\label{\detokenize{colab_code:indentation}}\begin{itemize}
\item {} 
\sphinxAtStartPar
Indentation refers to the spaces at the beginning of a code line.

\item {} 
\sphinxAtStartPar
Python uses indentation to indicate a block of code.

\item {} 
\sphinxAtStartPar
Do not use indentation unless you need it.

\item {} 
\sphinxAtStartPar
Indentation refers to the spaces at the beginning of a code line.

\item {} 
\sphinxAtStartPar
Python uses indentation to group the block of code.

\item {} 
\sphinxAtStartPar
Only use indentation when necessary.

\end{itemize}

\begin{sphinxVerbatim}[commandchars=\\\{\}]
\PYG{c+c1}{\PYGZsh{} ERROR: Indentation error}
\PYG{c+c1}{\PYGZsh{} There should not be space at the beginning}
\PYG{n+nb}{print}\PYG{p}{(}\PYG{l+m+mi}{3}\PYG{o}{+}\PYG{l+m+mi}{1}\PYG{p}{)}
 \PYG{n+nb}{print}\PYG{p}{(}\PYG{l+m+mi}{5}\PYG{o}{+}\PYG{l+m+mi}{6}\PYG{p}{)}  
\end{sphinxVerbatim}


\subsection{Import Modules/Packages/Libraries}
\label{\detokenize{colab_code:import-modules-packages-libraries}}\begin{itemize}
\item {} 
\sphinxAtStartPar
Only very commonly used, simple functions are available in a notebook by default.

\item {} 
\sphinxAtStartPar
Python has a vast set of modules, packages and libraries where one can find numerous functions.

\item {} 
\sphinxAtStartPar
For instance, the \sphinxcode{\sphinxupquote{mean()}} function is not available by default in a notebook.

\end{itemize}

\begin{sphinxVerbatim}[commandchars=\\\{\}]
\PYG{c+c1}{\PYGZsh{} ERROR: \PYGZsq{}mean\PYGZsq{} is not defined}
\PYG{n}{mean}\PYG{p}{(}\PYG{l+m+mi}{4}\PYG{p}{,}\PYG{l+m+mi}{8}\PYG{p}{)}
\end{sphinxVerbatim}
\begin{itemize}
\item {} 
\sphinxAtStartPar
The mean function can be imported from the \sphinxcode{\sphinxupquote{statistics}} module or \sphinxcode{\sphinxupquote{numpy}}library.

\item {} 
\sphinxAtStartPar
You can import a package and call any function from it using the \sphinxcode{\sphinxupquote{package.method}}.

\item {} 
\sphinxAtStartPar
You can import only the function(s) you need from the package using the \sphinxcode{\sphinxupquote{from PACKAGE import METHOD}}.

\item {} 
\sphinxAtStartPar
Additionally, you can import a package and rename it for shorter usage by employing the \sphinxcode{\sphinxupquote{import PACKAGE as ABBREVIATION}}.

\end{itemize}

\begin{sphinxuseclass}{cell}\begin{sphinxVerbatimInput}

\begin{sphinxuseclass}{cell_input}
\begin{sphinxVerbatim}[commandchars=\\\{\}]
\PYG{k+kn}{import} \PYG{n+nn}{statistics}        \PYG{c+c1}{\PYGZsh{} import whole module}
\PYG{n}{statistics}\PYG{o}{.}\PYG{n}{mean}\PYG{p}{(}\PYG{p}{[}\PYG{l+m+mi}{4}\PYG{p}{,}\PYG{l+m+mi}{8}\PYG{p}{]}\PYG{p}{)}   \PYG{c+c1}{\PYGZsh{} call mean method from statistics modu;e}
\end{sphinxVerbatim}

\end{sphinxuseclass}\end{sphinxVerbatimInput}
\begin{sphinxVerbatimOutput}

\begin{sphinxuseclass}{cell_output}
\begin{sphinxVerbatim}[commandchars=\\\{\}]
6
\end{sphinxVerbatim}

\end{sphinxuseclass}\end{sphinxVerbatimOutput}

\end{sphinxuseclass}
\begin{sphinxuseclass}{cell}\begin{sphinxVerbatimInput}

\begin{sphinxuseclass}{cell_input}
\begin{sphinxVerbatim}[commandchars=\\\{\}]
\PYG{k+kn}{from} \PYG{n+nn}{statistics} \PYG{k+kn}{import} \PYG{n}{mean}        \PYG{c+c1}{\PYGZsh{} import only mean method}
\PYG{n}{mean}\PYG{p}{(}\PYG{p}{[}\PYG{l+m+mi}{4}\PYG{p}{,}\PYG{l+m+mi}{8}\PYG{p}{]}\PYG{p}{)}                        \PYG{c+c1}{\PYGZsh{} call mean method }
\end{sphinxVerbatim}

\end{sphinxuseclass}\end{sphinxVerbatimInput}
\begin{sphinxVerbatimOutput}

\begin{sphinxuseclass}{cell_output}
\begin{sphinxVerbatim}[commandchars=\\\{\}]
6
\end{sphinxVerbatim}

\end{sphinxuseclass}\end{sphinxVerbatimOutput}

\end{sphinxuseclass}
\begin{sphinxuseclass}{cell}\begin{sphinxVerbatimInput}

\begin{sphinxuseclass}{cell_input}
\begin{sphinxVerbatim}[commandchars=\\\{\}]
\PYG{k+kn}{import} \PYG{n+nn}{statistics} \PYG{k}{as} \PYG{n+nn}{st}        \PYG{c+c1}{\PYGZsh{} import whole statistics module as st}
\PYG{n}{st}\PYG{o}{.}\PYG{n}{mean}\PYG{p}{(}\PYG{p}{[}\PYG{l+m+mi}{4}\PYG{p}{,}\PYG{l+m+mi}{8}\PYG{p}{]}\PYG{p}{)}                 \PYG{c+c1}{\PYGZsh{} call mean method from st module}
\end{sphinxVerbatim}

\end{sphinxuseclass}\end{sphinxVerbatimInput}
\begin{sphinxVerbatimOutput}

\begin{sphinxuseclass}{cell_output}
\begin{sphinxVerbatim}[commandchars=\\\{\}]
6
\end{sphinxVerbatim}

\end{sphinxuseclass}\end{sphinxVerbatimOutput}

\end{sphinxuseclass}
\begin{sphinxuseclass}{cell}\begin{sphinxVerbatimInput}

\begin{sphinxuseclass}{cell_input}
\begin{sphinxVerbatim}[commandchars=\\\{\}]
\PYG{k+kn}{import} \PYG{n+nn}{numpy} \PYG{k}{as} \PYG{n+nn}{np}        \PYG{c+c1}{\PYGZsh{} import numpy package as np}
\PYG{n}{np}\PYG{o}{.}\PYG{n}{mean}\PYG{p}{(}\PYG{p}{[}\PYG{l+m+mi}{4}\PYG{p}{,}\PYG{l+m+mi}{8}\PYG{p}{]}\PYG{p}{)}            \PYG{c+c1}{\PYGZsh{} call mean method from np package}
\end{sphinxVerbatim}

\end{sphinxuseclass}\end{sphinxVerbatimInput}
\begin{sphinxVerbatimOutput}

\begin{sphinxuseclass}{cell_output}
\begin{sphinxVerbatim}[commandchars=\\\{\}]
6.0
\end{sphinxVerbatim}

\end{sphinxuseclass}\end{sphinxVerbatimOutput}

\end{sphinxuseclass}\begin{itemize}
\item {} 
\sphinxAtStartPar
You can call mathematical constants and functions from the \sphinxcode{\sphinxupquote{math}} module.

\item {} 
\sphinxAtStartPar
Functions in modules are also referred to as methods.

\item {} 
\sphinxAtStartPar
The built\sphinxhyphen{}in \sphinxcode{\sphinxupquote{dir()}} and \sphinxcode{\sphinxupquote{help()}} functions display the lists of constants (attributes) and functions (methods).

\end{itemize}

\begin{sphinxVerbatim}[commandchars=\\\{\}]
\PYG{c+c1}{\PYGZsh{} displays all functions and constants in math module}
\PYG{k+kn}{import} \PYG{n+nn}{math}
\PYG{n+nb}{dir}\PYG{p}{(}\PYG{n}{math}\PYG{p}{)}
\end{sphinxVerbatim}

\begin{sphinxVerbatim}[commandchars=\\\{\}]
\PYG{c+c1}{\PYGZsh{} displays all functions and constants in math module with explanation}
\PYG{k+kn}{import} \PYG{n+nn}{math}
\PYG{n}{help}\PYG{p}{(}\PYG{n}{math}\PYG{p}{)}
\end{sphinxVerbatim}

\begin{sphinxuseclass}{cell}\begin{sphinxVerbatimInput}

\begin{sphinxuseclass}{cell_input}
\begin{sphinxVerbatim}[commandchars=\\\{\}]
\PYG{k+kn}{import} \PYG{n+nn}{math}
\PYG{n+nb}{dir}\PYG{p}{(}\PYG{n}{math}\PYG{p}{)}
\end{sphinxVerbatim}

\end{sphinxuseclass}\end{sphinxVerbatimInput}
\begin{sphinxVerbatimOutput}

\begin{sphinxuseclass}{cell_output}
\begin{sphinxVerbatim}[commandchars=\\\{\}]
[\PYGZsq{}\PYGZus{}\PYGZus{}doc\PYGZus{}\PYGZus{}\PYGZsq{},
 \PYGZsq{}\PYGZus{}\PYGZus{}file\PYGZus{}\PYGZus{}\PYGZsq{},
 \PYGZsq{}\PYGZus{}\PYGZus{}loader\PYGZus{}\PYGZus{}\PYGZsq{},
 \PYGZsq{}\PYGZus{}\PYGZus{}name\PYGZus{}\PYGZus{}\PYGZsq{},
 \PYGZsq{}\PYGZus{}\PYGZus{}package\PYGZus{}\PYGZus{}\PYGZsq{},
 \PYGZsq{}\PYGZus{}\PYGZus{}spec\PYGZus{}\PYGZus{}\PYGZsq{},
 \PYGZsq{}acos\PYGZsq{},
 \PYGZsq{}acosh\PYGZsq{},
 \PYGZsq{}asin\PYGZsq{},
 \PYGZsq{}asinh\PYGZsq{},
 \PYGZsq{}atan\PYGZsq{},
 \PYGZsq{}atan2\PYGZsq{},
 \PYGZsq{}atanh\PYGZsq{},
 \PYGZsq{}cbrt\PYGZsq{},
 \PYGZsq{}ceil\PYGZsq{},
 \PYGZsq{}comb\PYGZsq{},
 \PYGZsq{}copysign\PYGZsq{},
 \PYGZsq{}cos\PYGZsq{},
 \PYGZsq{}cosh\PYGZsq{},
 \PYGZsq{}degrees\PYGZsq{},
 \PYGZsq{}dist\PYGZsq{},
 \PYGZsq{}e\PYGZsq{},
 \PYGZsq{}erf\PYGZsq{},
 \PYGZsq{}erfc\PYGZsq{},
 \PYGZsq{}exp\PYGZsq{},
 \PYGZsq{}exp2\PYGZsq{},
 \PYGZsq{}expm1\PYGZsq{},
 \PYGZsq{}fabs\PYGZsq{},
 \PYGZsq{}factorial\PYGZsq{},
 \PYGZsq{}floor\PYGZsq{},
 \PYGZsq{}fmod\PYGZsq{},
 \PYGZsq{}frexp\PYGZsq{},
 \PYGZsq{}fsum\PYGZsq{},
 \PYGZsq{}gamma\PYGZsq{},
 \PYGZsq{}gcd\PYGZsq{},
 \PYGZsq{}hypot\PYGZsq{},
 \PYGZsq{}inf\PYGZsq{},
 \PYGZsq{}isclose\PYGZsq{},
 \PYGZsq{}isfinite\PYGZsq{},
 \PYGZsq{}isinf\PYGZsq{},
 \PYGZsq{}isnan\PYGZsq{},
 \PYGZsq{}isqrt\PYGZsq{},
 \PYGZsq{}lcm\PYGZsq{},
 \PYGZsq{}ldexp\PYGZsq{},
 \PYGZsq{}lgamma\PYGZsq{},
 \PYGZsq{}log\PYGZsq{},
 \PYGZsq{}log10\PYGZsq{},
 \PYGZsq{}log1p\PYGZsq{},
 \PYGZsq{}log2\PYGZsq{},
 \PYGZsq{}modf\PYGZsq{},
 \PYGZsq{}nan\PYGZsq{},
 \PYGZsq{}nextafter\PYGZsq{},
 \PYGZsq{}perm\PYGZsq{},
 \PYGZsq{}pi\PYGZsq{},
 \PYGZsq{}pow\PYGZsq{},
 \PYGZsq{}prod\PYGZsq{},
 \PYGZsq{}radians\PYGZsq{},
 \PYGZsq{}remainder\PYGZsq{},
 \PYGZsq{}sin\PYGZsq{},
 \PYGZsq{}sinh\PYGZsq{},
 \PYGZsq{}sqrt\PYGZsq{},
 \PYGZsq{}tan\PYGZsq{},
 \PYGZsq{}tanh\PYGZsq{},
 \PYGZsq{}tau\PYGZsq{},
 \PYGZsq{}trunc\PYGZsq{},
 \PYGZsq{}ulp\PYGZsq{}]
\end{sphinxVerbatim}

\end{sphinxuseclass}\end{sphinxVerbatimOutput}

\end{sphinxuseclass}
\begin{sphinxuseclass}{cell}\begin{sphinxVerbatimInput}

\begin{sphinxuseclass}{cell_input}
\begin{sphinxVerbatim}[commandchars=\\\{\}]
\PYG{n+nb}{print}\PYG{p}{(}\PYG{n}{math}\PYG{o}{.}\PYG{n}{pi}\PYG{p}{)}
\end{sphinxVerbatim}

\end{sphinxuseclass}\end{sphinxVerbatimInput}
\begin{sphinxVerbatimOutput}

\begin{sphinxuseclass}{cell_output}
\begin{sphinxVerbatim}[commandchars=\\\{\}]
3.141592653589793
\end{sphinxVerbatim}

\end{sphinxuseclass}\end{sphinxVerbatimOutput}

\end{sphinxuseclass}
\begin{sphinxuseclass}{cell}\begin{sphinxVerbatimInput}

\begin{sphinxuseclass}{cell_input}
\begin{sphinxVerbatim}[commandchars=\\\{\}]
\PYG{c+c1}{\PYGZsh{} ln(10)}
\PYG{n+nb}{print}\PYG{p}{(}\PYG{n}{math}\PYG{o}{.}\PYG{n}{log}\PYG{p}{(}\PYG{l+m+mi}{10}\PYG{p}{)}\PYG{p}{)}   \PYG{c+c1}{\PYGZsh{} natural log 10}
\end{sphinxVerbatim}

\end{sphinxuseclass}\end{sphinxVerbatimInput}
\begin{sphinxVerbatimOutput}

\begin{sphinxuseclass}{cell_output}
\begin{sphinxVerbatim}[commandchars=\\\{\}]
2.302585092994046
\end{sphinxVerbatim}

\end{sphinxuseclass}\end{sphinxVerbatimOutput}

\end{sphinxuseclass}\begin{itemize}
\item {} 
\sphinxAtStartPar
The \sphinxcode{\sphinxupquote{datetime}} module contains methods that can be used for working with time and date

\end{itemize}

\begin{sphinxuseclass}{cell}\begin{sphinxVerbatimInput}

\begin{sphinxuseclass}{cell_input}
\begin{sphinxVerbatim}[commandchars=\\\{\}]
\PYG{k+kn}{import} \PYG{n+nn}{datetime}
\PYG{n+nb}{print}\PYG{p}{(}\PYG{n}{datetime}\PYG{o}{.}\PYG{n}{datetime}\PYG{o}{.}\PYG{n}{now}\PYG{p}{(}\PYG{p}{)}\PYG{p}{)}    \PYG{c+c1}{\PYGZsh{} date and time now}
\end{sphinxVerbatim}

\end{sphinxuseclass}\end{sphinxVerbatimInput}
\begin{sphinxVerbatimOutput}

\begin{sphinxuseclass}{cell_output}
\begin{sphinxVerbatim}[commandchars=\\\{\}]
2024\PYGZhy{}02\PYGZhy{}02 23:50:34.368893
\end{sphinxVerbatim}

\end{sphinxuseclass}\end{sphinxVerbatimOutput}

\end{sphinxuseclass}\begin{itemize}
\item {} 
\sphinxAtStartPar
You can use the \sphinxcode{\sphinxupquote{calendar}} module to display a monthly calendar.

\end{itemize}

\begin{sphinxuseclass}{cell}\begin{sphinxVerbatimInput}

\begin{sphinxuseclass}{cell_input}
\begin{sphinxVerbatim}[commandchars=\\\{\}]
\PYG{k+kn}{import} \PYG{n+nn}{calendar}                   \PYG{c+c1}{\PYGZsh{} import calender module}
\PYG{n+nb}{print}\PYG{p}{(}\PYG{n}{calendar}\PYG{o}{.}\PYG{n}{month}\PYG{p}{(}\PYG{l+m+mi}{1900}\PYG{p}{,} \PYG{l+m+mi}{1}\PYG{p}{)}\PYG{p}{)}    \PYG{c+c1}{\PYGZsh{} display Jan month of 1900}
\end{sphinxVerbatim}

\end{sphinxuseclass}\end{sphinxVerbatimInput}
\begin{sphinxVerbatimOutput}

\begin{sphinxuseclass}{cell_output}
\begin{sphinxVerbatim}[commandchars=\\\{\}]
    January 1900
Mo Tu We Th Fr Sa Su
 1  2  3  4  5  6  7
 8  9 10 11 12 13 14
15 16 17 18 19 20 21
22 23 24 25 26 27 28
29 30 31
\end{sphinxVerbatim}

\end{sphinxuseclass}\end{sphinxVerbatimOutput}

\end{sphinxuseclass}\begin{itemize}
\item {} 
\sphinxAtStartPar
You can use the \sphinxcode{\sphinxupquote{random}}module to generate random numbers or make random choices.

\end{itemize}

\begin{sphinxuseclass}{cell}\begin{sphinxVerbatimInput}

\begin{sphinxuseclass}{cell_input}
\begin{sphinxVerbatim}[commandchars=\\\{\}]
\PYG{k+kn}{import} \PYG{n+nn}{random}
\PYG{n}{random}\PYG{o}{.}\PYG{n}{randint}\PYG{p}{(}\PYG{l+m+mi}{1}\PYG{p}{,}\PYG{l+m+mi}{10}\PYG{p}{)}  \PYG{c+c1}{\PYGZsh{} choose a random integer between 1 and 10}
\end{sphinxVerbatim}

\end{sphinxuseclass}\end{sphinxVerbatimInput}
\begin{sphinxVerbatimOutput}

\begin{sphinxuseclass}{cell_output}
\begin{sphinxVerbatim}[commandchars=\\\{\}]
3
\end{sphinxVerbatim}

\end{sphinxuseclass}\end{sphinxVerbatimOutput}

\end{sphinxuseclass}
\begin{sphinxuseclass}{cell}\begin{sphinxVerbatimInput}

\begin{sphinxuseclass}{cell_input}
\begin{sphinxVerbatim}[commandchars=\\\{\}]
\PYG{n}{random}\PYG{o}{.}\PYG{n}{choice}\PYG{p}{(}\PYG{p}{[}\PYG{l+s+s1}{\PYGZsq{}}\PYG{l+s+s1}{a}\PYG{l+s+s1}{\PYGZsq{}}\PYG{p}{,}\PYG{l+s+s1}{\PYGZsq{}}\PYG{l+s+s1}{b}\PYG{l+s+s1}{\PYGZsq{}}\PYG{p}{,}\PYG{l+s+s1}{\PYGZsq{}}\PYG{l+s+s1}{c}\PYG{l+s+s1}{\PYGZsq{}}\PYG{p}{]}\PYG{p}{)}  \PYG{c+c1}{\PYGZsh{} choose a random element from a list}
\end{sphinxVerbatim}

\end{sphinxuseclass}\end{sphinxVerbatimInput}
\begin{sphinxVerbatimOutput}

\begin{sphinxuseclass}{cell_output}
\begin{sphinxVerbatim}[commandchars=\\\{\}]
\PYGZsq{}c\PYGZsq{}
\end{sphinxVerbatim}

\end{sphinxuseclass}\end{sphinxVerbatimOutput}

\end{sphinxuseclass}

\subsection{Magic Methods}
\label{\detokenize{colab_code:magic-methods}}\begin{itemize}
\item {} 
\sphinxAtStartPar
The methods in \sphinxcode{\sphinxupquote{dir(math)}} enclosed by double underscores \sphinxcode{\sphinxupquote{\_\_}} are referred to as magic functions.

\item {} 
\sphinxAtStartPar
These functions operate in the backend and are not intended for direct user use.

\item {} 
\sphinxAtStartPar
While it is still possible to use them, there is usually a more user\sphinxhyphen{}friendly alternative.

\item {} 
\sphinxAtStartPar
For example, if you examine \sphinxcode{\sphinxupquote{dir(int)}}, you will find the \sphinxcode{\sphinxupquote{\_\_add\_\_}} method, which handles the addition operation internally, corresponding to the use of the + operator.

\end{itemize}

\begin{sphinxuseclass}{cell}\begin{sphinxVerbatimInput}

\begin{sphinxuseclass}{cell_input}
\begin{sphinxVerbatim}[commandchars=\\\{\}]
\PYG{n}{number} \PYG{o}{=} \PYG{l+m+mi}{3}
\PYG{n+nb}{print}\PYG{p}{(}\PYG{n}{number} \PYG{o}{+} \PYG{l+m+mi}{5}\PYG{p}{)}
\PYG{n+nb}{print}\PYG{p}{(}\PYG{n}{number}\PYG{o}{.}\PYG{n+nf+fm}{\PYGZus{}\PYGZus{}add\PYGZus{}\PYGZus{}}\PYG{p}{(}\PYG{l+m+mi}{5}\PYG{p}{)}\PYG{p}{)}  \PYG{c+c1}{\PYGZsh{} same as 3+5}
\end{sphinxVerbatim}

\end{sphinxuseclass}\end{sphinxVerbatimInput}
\begin{sphinxVerbatimOutput}

\begin{sphinxuseclass}{cell_output}
\begin{sphinxVerbatim}[commandchars=\\\{\}]
8
8
\end{sphinxVerbatim}

\end{sphinxuseclass}\end{sphinxVerbatimOutput}

\end{sphinxuseclass}

\subsection{Images}
\label{\detokenize{colab_code:images}}\begin{itemize}
\item {} 
\sphinxAtStartPar
You can import images from a website using the IPython library.

\item {} 
\sphinxAtStartPar
Adjust the size of the image by utilizing width and height parameters.

\end{itemize}

\begin{sphinxuseclass}{cell}\begin{sphinxVerbatimInput}

\begin{sphinxuseclass}{cell_input}
\begin{sphinxVerbatim}[commandchars=\\\{\}]
\PYG{k+kn}{from} \PYG{n+nn}{IPython}\PYG{n+nn}{.}\PYG{n+nn}{core}\PYG{n+nn}{.}\PYG{n+nn}{display} \PYG{k+kn}{import} \PYG{n}{Image}
\PYG{n}{Image}\PYG{p}{(}\PYG{l+s+s1}{\PYGZsq{}}\PYG{l+s+s1}{https://www.python.org/static/community\PYGZus{}logos/python\PYGZhy{}powered\PYGZhy{}h\PYGZhy{}100x130.png}\PYG{l+s+s1}{\PYGZsq{}}\PYG{p}{,}\PYG{n}{width}\PYG{o}{=}\PYG{l+m+mi}{100}\PYG{p}{,}\PYG{n}{height}\PYG{o}{=}\PYG{l+m+mi}{100}\PYG{p}{)}
\end{sphinxVerbatim}

\end{sphinxuseclass}\end{sphinxVerbatimInput}
\begin{sphinxVerbatimOutput}

\begin{sphinxuseclass}{cell_output}
\noindent\sphinxincludegraphics{{0e440193a7b074e7f072edfa1a2e85af0d9b6b26f50f388d88da73df50627c92}.png}

\end{sphinxuseclass}\end{sphinxVerbatimOutput}

\end{sphinxuseclass}

\subsection{Youtube Video}
\label{\detokenize{colab_code:youtube-video}}\begin{itemize}
\item {} 
\sphinxAtStartPar
You can import YouTube videos using the unique ID number.

\item {} 
\sphinxAtStartPar
The ID number of a YouTube video can be found in the video’s URL directly after the = sign

\item {} 
\sphinxAtStartPar
ID number of  \sphinxurl{https://www.youtube.com/watch?v=rNgswRZ2C1Y} is \sphinxcode{\sphinxupquote{rNgswRZ2C1Y}}

\end{itemize}

\begin{sphinxuseclass}{cell}\begin{sphinxVerbatimInput}

\begin{sphinxuseclass}{cell_input}
\begin{sphinxVerbatim}[commandchars=\\\{\}]
\PYG{k+kn}{from} \PYG{n+nn}{IPython}\PYG{n+nn}{.}\PYG{n+nn}{lib}\PYG{n+nn}{.}\PYG{n+nn}{display} \PYG{k+kn}{import} \PYG{n}{YouTubeVideo}
\PYG{n}{YouTubeVideo}\PYG{p}{(}\PYG{l+s+s1}{\PYGZsq{}}\PYG{l+s+s1}{rNgswRZ2C1Y}\PYG{l+s+s1}{\PYGZsq{}}\PYG{p}{,} \PYG{n}{width}\PYG{o}{=}\PYG{l+m+mi}{800}\PYG{p}{,} \PYG{n}{height}\PYG{o}{=}\PYG{l+m+mi}{300}\PYG{p}{)}
\end{sphinxVerbatim}

\end{sphinxuseclass}\end{sphinxVerbatimInput}
\begin{sphinxVerbatimOutput}

\begin{sphinxuseclass}{cell_output}
\noindent\sphinxincludegraphics{{9cf905f5ab14830304419eb16036bfccc85532693ca5924d632f4031ab0d5d05}.jpg}

\end{sphinxuseclass}\end{sphinxVerbatimOutput}

\end{sphinxuseclass}
\sphinxstepscope


\subsection{Code Cell Questions}
\label{\detokenize{colab_code_questions:code-cell-questions}}\label{\detokenize{colab_code_questions::doc}}

\subsubsection{Questions}
\label{\detokenize{colab_code_questions:questions}}
\sphinxAtStartPar
Perform each of the following algebraic operations in a separate code cell:
\begin{itemize}
\item {} 
\sphinxAtStartPar
20 \sphinxhyphen{} 3

\item {} 
\sphinxAtStartPar
72 + 34

\item {} 
\sphinxAtStartPar
\(46 \div 4\)

\item {} 
\sphinxAtStartPar
\(4 \times 12\)

\item {} 
\sphinxAtStartPar
\(5^3\)

\end{itemize}

\sphinxAtStartPar
\sphinxstylestrong{Solution}


\subsubsection{Question}
\label{\detokenize{colab_code_questions:question}}
\sphinxAtStartPar
Find the difference between the maximum and minumum of the following numbers:  \(65, 23, 1, 2, 3, 90, 99, 12, 34, 67\) using the built\sphinxhyphen{}in  \sphinxstyleemphasis{max()} and \sphinxstyleemphasis{min()} functions.

\sphinxAtStartPar
\sphinxstylestrong{Solution}


\subsubsection{Question}
\label{\detokenize{colab_code_questions:id1}}
\sphinxAtStartPar
Print the string ‘Python’ five times using string repetition.”

\sphinxAtStartPar
\sphinxstylestrong{Solution}


\subsubsection{Question}
\label{\detokenize{colab_code_questions:id2}}
\sphinxAtStartPar
Print the mean, median, mode, and standard deviation of the following list of numbers: \([12, 3, 12, 9, 7, 3, 25, 6]\), using the statistics module.
\begin{itemize}
\item {} 
\sphinxAtStartPar
You can refer to the available functions in the statistics module using \sphinxstyleemphasis{dir(statistics)} or \sphinxstyleemphasis{help(statistics)}.

\end{itemize}

\sphinxAtStartPar
\sphinxstylestrong{Solution}


\subsubsection{Question}
\label{\detokenize{colab_code_questions:id3}}
\sphinxAtStartPar
Print the square root of 75 using the \sphinxstyleemphasis{math}, \sphinxstyleemphasis{statistics} and \sphinxstyleemphasis{numpy} modules.

\sphinxAtStartPar
\sphinxstylestrong{Solution}

\sphinxstepscope


\section{Colab Exercises}
\label{\detokenize{colab_exercises:colab-exercises}}\label{\detokenize{colab_exercises::doc}}
\sphinxAtStartPar
Please follow the steps below:
\begin{enumerate}
\sphinxsetlistlabels{\arabic}{enumi}{enumii}{}{.}%
\item {} 
\sphinxAtStartPar
Create a folder in your Google Drive and rename it as \sphinxcode{\sphinxupquote{python\_labs}}.

\item {} 
\sphinxAtStartPar
Within this folder, create a Colab notebook and give it the name \sphinxcode{\sphinxupquote{firstname\_lastname\_lab\_colab}} using your own name.

\item {} 
\sphinxAtStartPar
Insert a \sphinxcode{\sphinxupquote{text}} cell, enter \sphinxstyleemphasis{Biography}, and set it as a heading with level\sphinxhyphen{}1.

\item {} 
\sphinxAtStartPar
Enter \sphinxstyleemphasis{Albert Einstein} to the same text cell, and move it to become the first cell in the notebook.

\item {} 
\sphinxAtStartPar
Below that, insert another \sphinxcode{\sphinxupquote{text}} cell, enter \sphinxstyleemphasis{Education}, and set it as a heading with level\sphinxhyphen{}2.

\item {} 
\sphinxAtStartPar
Include the following two items in this text cell as a bullet list, with degrees and years as sublists.
\begin{itemize}
\item {} 
\sphinxAtStartPar
Federal polytechnic school (Dipl., 1900)

\item {} 
\sphinxAtStartPar
University of Zurich (PhD, 1905)

\end{itemize}

\item {} 
\sphinxAtStartPar
Below that, insert another \sphinxcode{\sphinxupquote{text}} cell, enter \sphinxstyleemphasis{Awards}, and set it as a heading with level\sphinxhyphen{}2.

\item {} 
\sphinxAtStartPar
Include the following  items in this text cell as a numbered list.
\begin{itemize}
\item {} 
\sphinxAtStartPar
Barnard Medal (1920)

\item {} 
\sphinxAtStartPar
Nobel Prize in Physics (1921)

\item {} 
\sphinxAtStartPar
Matteucci Medal (1921)

\item {} 
\sphinxAtStartPar
ForMemRS (1921)

\item {} 
\sphinxAtStartPar
Copley Medal (1925)

\item {} 
\sphinxAtStartPar
Gold Medal of the Royal Astronomical Society (1926)

\item {} 
\sphinxAtStartPar
Max Planck Medal (1929)

\item {} 
\sphinxAtStartPar
Member of the National Academy of Sciences (1942)

\item {} 
\sphinxAtStartPar
Time Person of the Century (1999)

\end{itemize}

\item {} 
\sphinxAtStartPar
After the second item above, insert a horizontal line.

\item {} 
\sphinxAtStartPar
Change the color of the third item to red.

\item {} 
\sphinxAtStartPar
Emphasize the fourth item by making it bold.

\item {} 
\sphinxAtStartPar
Increase the font size of the seventh item to 20.

\item {} 
\sphinxAtStartPar
Italicize the eighth item.

\item {} 
\sphinxAtStartPar
Highlight the last one.

\item {} 
\sphinxAtStartPar
Create a new text cell and construct a table with two columns and nine rows, including \sphinxstyleemphasis{Award} and \sphinxstyleemphasis{Year} as the column headers.

\item {} 
\sphinxAtStartPar
Insert four code cells, perform one algebraic operation in each, and include a meaningful comment for clarity in each cell.

\item {} 
\sphinxAtStartPar
Insert a code cell and print ‘Tesla’ 10 times in a single line.

\item {} 
\sphinxAtStartPar
Insert a code cell and embed a YouTube video.

\item {} 
\sphinxAtStartPar
Insert a code cell and embed a picture from a website.

\item {} 
\sphinxAtStartPar
Insert a code cell and calculate the median of the numbers: \(9, 34, 13, 90, 5\).

\end{enumerate}

\sphinxAtStartPar
\sphinxstylestrong{Hint:} The output of the text cells should be as follows:


\section{Biography}
\label{\detokenize{colab_exercises:biography}}
\sphinxAtStartPar
Albert Einstein


\subsection{Education}
\label{\detokenize{colab_exercises:education}}\begin{itemize}
\item {} 
\sphinxAtStartPar
Federal polytechnic school
\begin{itemize}
\item {} 
\sphinxAtStartPar
Dipl.

\item {} 
\sphinxAtStartPar
1900

\end{itemize}

\item {} 
\sphinxAtStartPar
University of Zurich
\begin{itemize}
\item {} 
\sphinxAtStartPar
PhD

\item {} 
\sphinxAtStartPar
1905

\end{itemize}

\end{itemize}


\subsection{Awards}
\label{\detokenize{colab_exercises:awards}}\begin{enumerate}
\sphinxsetlistlabels{\arabic}{enumi}{enumii}{}{.}%
\item {} 
\sphinxAtStartPar
Barnard Medal (1920)

\item {} 
\sphinxAtStartPar
Nobel Prize in Physics (1921)

\end{enumerate}


\bigskip\hrule\bigskip

\begin{enumerate}
\sphinxsetlistlabels{\arabic}{enumi}{enumii}{}{.}%
\setcounter{enumi}{2}
\item {} 
\sphinxAtStartPar
 Matteucci Medal (1921) 

\item {} 
\sphinxAtStartPar
\sphinxstylestrong{ForMemRS (1921)}

\item {} 
\sphinxAtStartPar
Copley Medal (1925)

\item {} 
\sphinxAtStartPar
Gold Medal of the Royal Astronomical Society (1926)

\item {} 
\sphinxAtStartPar
  Max Planck Medal (1929)

\item {} 
\sphinxAtStartPar
\sphinxstyleemphasis{Member of the National Academy of Sciences (1942)}

\item {} 
\sphinxAtStartPar
\sphinxcode{\sphinxupquote{Time Person of the Century (1999)}}

\end{enumerate}


\begin{savenotes}\sphinxattablestart
\centering
\begin{tabulary}{\linewidth}[t]{|T|T|}
\hline
\sphinxstyletheadfamily 
\sphinxAtStartPar
Award
&\sphinxstyletheadfamily 
\sphinxAtStartPar
Year
\\
\hline
\sphinxAtStartPar
Barnard Medal
&
\sphinxAtStartPar
1920
\\
\hline
\sphinxAtStartPar
Nobel Prize in Physics
&
\sphinxAtStartPar
1921
\\
\hline
\sphinxAtStartPar
Matteucci Medal
&
\sphinxAtStartPar
1921
\\
\hline
\sphinxAtStartPar
ForMemRS
&
\sphinxAtStartPar
1921
\\
\hline
\sphinxAtStartPar
Copley Medal
&
\sphinxAtStartPar
1925
\\
\hline
\sphinxAtStartPar
Gold Medal of the Royal Astronomical Society
&
\sphinxAtStartPar
1926
\\
\hline
\sphinxAtStartPar
Max Planck Medal
&
\sphinxAtStartPar
1929
\\
\hline
\sphinxAtStartPar
Member of the National Academy of Sciences
&
\sphinxAtStartPar
1942
\\
\hline
\sphinxAtStartPar
Time Person of the Century
&
\sphinxAtStartPar
1999
\\
\hline
\end{tabulary}
\par
\sphinxattableend\end{savenotes}

\sphinxstepscope


\chapter{Chp\sphinxhyphen{}1: Variables}
\label{\detokenize{variables:chp-1-variables}}\label{\detokenize{variables::doc}}\begin{itemize}
\item {} 
\sphinxAtStartPar
Learning Objectives
\begin{itemize}
\item {} 
\sphinxAtStartPar
..

\item {} 
\sphinxAtStartPar
..

\end{itemize}

\end{itemize}

\sphinxAtStartPar
A variable is a name that refers to a value, such as a number (e.g., 25) or a name (e.g., ‘Michael’).
\begin{itemize}
\item {} 
\sphinxAtStartPar
It functions like a container for storing data values.

\item {} 
\sphinxAtStartPar
Variables allow you to store and manipulate data in your programs.

\end{itemize}
\begin{itemize}
\item {} 
\sphinxAtStartPar
To create a variable:
\begin{enumerate}
\sphinxsetlistlabels{\arabic}{enumi}{enumii}{}{.}%
\item {} 
\sphinxAtStartPar
Choose a name.

\item {} 
\sphinxAtStartPar
Assign a value to it using the \sphinxcode{\sphinxupquote{=}} operator.

\end{enumerate}

\end{itemize}
\begin{itemize}
\item {} 
\sphinxAtStartPar
Example: \sphinxcode{\sphinxupquote{age = 25}}
\begin{itemize}
\item {} 
\sphinxAtStartPar
The variable’s name is \sphinxstyleemphasis{age}.

\item {} 
\sphinxAtStartPar
The value of the \sphinxstyleemphasis{age} variable is 25.

\end{itemize}

\end{itemize}


\section{Purpose}
\label{\detokenize{variables:purpose}}\begin{itemize}
\item {} 
\sphinxAtStartPar
Why do we need variables?

\end{itemize}

\sphinxAtStartPar
For the code below:

\begin{sphinxVerbatim}[commandchars=\\\{\}]
\PYG{n+nb}{print}\PYG{p}{(}\PYG{l+m+mi}{40}\PYG{p}{)}
\PYG{n+nb}{print}\PYG{p}{(}\PYG{l+m+mi}{40}\PYG{o}{+}\PYG{l+m+mi}{1}\PYG{p}{)}
\PYG{n+nb}{print}\PYG{p}{(}\PYG{l+m+mi}{40}\PYG{o}{*}\PYG{o}{*}\PYG{l+m+mi}{2}\PYG{p}{)}
\PYG{n+nb}{print}\PYG{p}{(}\PYG{l+m+mi}{40}\PYG{o}{*}\PYG{l+m+mi}{7}\PYG{p}{)}
\PYG{n+nb}{print}\PYG{p}{(}\PYG{l+m+mi}{40}\PYG{o}{\PYGZhy{}}\PYG{l+m+mi}{13}\PYG{p}{)}
\end{sphinxVerbatim}
\begin{itemize}
\item {} 
\sphinxAtStartPar
How many changes do you need to make to get a similar output for 20 instead of 40?
\begin{itemize}
\item {} 
\sphinxAtStartPar
Answer: 5

\end{itemize}

\end{itemize}

\begin{sphinxVerbatim}[commandchars=\\\{\}]
\PYG{n+nb}{print}\PYG{p}{(}\PYG{l+m+mi}{20}\PYG{p}{)}       \PYG{c+c1}{\PYGZsh{} 1st change}
\PYG{n+nb}{print}\PYG{p}{(}\PYG{l+m+mi}{20}\PYG{o}{+}\PYG{l+m+mi}{1}\PYG{p}{)}     \PYG{c+c1}{\PYGZsh{} 2nd change}
\PYG{n+nb}{print}\PYG{p}{(}\PYG{l+m+mi}{20}\PYG{o}{*}\PYG{o}{*}\PYG{l+m+mi}{2}\PYG{p}{)}    \PYG{c+c1}{\PYGZsh{} 3rd change}
\PYG{n+nb}{print}\PYG{p}{(}\PYG{l+m+mi}{20}\PYG{o}{*}\PYG{l+m+mi}{7}\PYG{p}{)}     \PYG{c+c1}{\PYGZsh{} 4th change}
\PYG{n+nb}{print}\PYG{p}{(}\PYG{l+m+mi}{20}\PYG{o}{\PYGZhy{}}\PYG{l+m+mi}{13}\PYG{p}{)}    \PYG{c+c1}{\PYGZsh{} 5th change}
\end{sphinxVerbatim}
\begin{itemize}
\item {} 
\sphinxAtStartPar
In a lengthy program, you might need to make numerous changes, which can be time\sphinxhyphen{}consuming.

\item {} 
\sphinxAtStartPar
There is a risk of forgetting to update some of them, leading to errors or incorrect output.

\end{itemize}
\begin{itemize}
\item {} 
\sphinxAtStartPar
Let’s now write the same code, but this time using the variable \sphinxstyleemphasis{number}.

\end{itemize}

\begin{sphinxVerbatim}[commandchars=\\\{\}]
\PYG{n}{number} \PYG{o}{=} \PYG{l+m+mi}{40}
\PYG{n+nb}{print}\PYG{p}{(}\PYG{n}{number}\PYG{p}{)}
\PYG{n+nb}{print}\PYG{p}{(}\PYG{n}{number}\PYG{o}{+}\PYG{l+m+mi}{1}\PYG{p}{)}
\PYG{n+nb}{print}\PYG{p}{(}\PYG{n}{number}\PYG{o}{*}\PYG{o}{*}\PYG{l+m+mi}{2}\PYG{p}{)}
\PYG{n+nb}{print}\PYG{p}{(}\PYG{n}{number}\PYG{o}{*}\PYG{l+m+mi}{7}\PYG{p}{)}
\PYG{n+nb}{print}\PYG{p}{(}\PYG{n}{number}\PYG{o}{\PYGZhy{}}\PYG{l+m+mi}{13}\PYG{p}{)}
\end{sphinxVerbatim}
\begin{itemize}
\item {} 
\sphinxAtStartPar
If you use the variable \sphinxcode{\sphinxupquote{number}}, then to obtain the output for 20, you only need to make one change, as follows:

\end{itemize}

\begin{sphinxVerbatim}[commandchars=\\\{\}]
\PYG{n}{number} \PYG{o}{=} \PYG{l+m+mi}{20}          \PYG{c+c1}{\PYGZsh{} 1st change}
\PYG{n+nb}{print}\PYG{p}{(}\PYG{n}{number}\PYG{p}{)}
\PYG{n+nb}{print}\PYG{p}{(}\PYG{n}{number}\PYG{o}{+}\PYG{l+m+mi}{1}\PYG{p}{)}
\PYG{n+nb}{print}\PYG{p}{(}\PYG{n}{number}\PYG{o}{*}\PYG{o}{*}\PYG{l+m+mi}{2}\PYG{p}{)}
\PYG{n+nb}{print}\PYG{p}{(}\PYG{n}{number}\PYG{o}{*}\PYG{l+m+mi}{7}\PYG{p}{)}
\PYG{n+nb}{print}\PYG{p}{(}\PYG{n}{number}\PYG{o}{\PYGZhy{}}\PYG{l+m+mi}{13}\PYG{p}{)}
\end{sphinxVerbatim}


\section{id()}
\label{\detokenize{variables:id}}
\sphinxAtStartPar
The \sphinxcode{\sphinxupquote{id()}} function is a built\sphinxhyphen{}in function that returns the memory location of a variable.
\begin{itemize}
\item {} 
\sphinxAtStartPar
It can also be considered as the ID number of the variable.

\end{itemize}

\begin{sphinxuseclass}{cell}\begin{sphinxVerbatimInput}

\begin{sphinxuseclass}{cell_input}
\begin{sphinxVerbatim}[commandchars=\\\{\}]
\PYG{n}{age} \PYG{o}{=} \PYG{l+m+mi}{25}
\PYG{n+nb}{print}\PYG{p}{(}\PYG{n+nb}{id}\PYG{p}{(}\PYG{n}{age}\PYG{p}{)}\PYG{p}{)}
\end{sphinxVerbatim}

\end{sphinxuseclass}\end{sphinxVerbatimInput}
\begin{sphinxVerbatimOutput}

\begin{sphinxuseclass}{cell_output}
\begin{sphinxVerbatim}[commandchars=\\\{\}]
4300337000
\end{sphinxVerbatim}

\end{sphinxuseclass}\end{sphinxVerbatimOutput}

\end{sphinxuseclass}
\begin{sphinxuseclass}{cell}\begin{sphinxVerbatimInput}

\begin{sphinxuseclass}{cell_input}
\begin{sphinxVerbatim}[commandchars=\\\{\}]
\PYG{n}{state} \PYG{o}{=} \PYG{l+s+s1}{\PYGZsq{}}\PYG{l+s+s1}{Florida}\PYG{l+s+s1}{\PYGZsq{}}
\PYG{n+nb}{print}\PYG{p}{(}\PYG{n+nb}{id}\PYG{p}{(}\PYG{n}{state}\PYG{p}{)}\PYG{p}{)}
\end{sphinxVerbatim}

\end{sphinxuseclass}\end{sphinxVerbatimInput}
\begin{sphinxVerbatimOutput}

\begin{sphinxuseclass}{cell_output}
\begin{sphinxVerbatim}[commandchars=\\\{\}]
4356362800
\end{sphinxVerbatim}

\end{sphinxuseclass}\end{sphinxVerbatimOutput}

\end{sphinxuseclass}

\section{Values and Types}
\label{\detokenize{variables:values-and-types}}
\sphinxAtStartPar
In Python, the most commonly used basic data types are integers, floats, strings, and booleans.
\begin{enumerate}
\sphinxsetlistlabels{\arabic}{enumi}{enumii}{}{.}%
\item {} 
\sphinxAtStartPar
Integers
\begin{itemize}
\item {} 
\sphinxAtStartPar
These are numbers without a decimal point, such as \(-3, -2, -1, 0, 1, 2, 3, \dots\)

\item {} 
\sphinxAtStartPar
Their class is called \sphinxstyleemphasis{int}.

\item {} 
\sphinxAtStartPar
Examples: 5, 9, 123.

\end{itemize}

\item {} 
\sphinxAtStartPar
Floats
\begin{itemize}
\item {} 
\sphinxAtStartPar
These are decimal numbers, including those with a decimal point, such as 4.0.

\item {} 
\sphinxAtStartPar
Their class is called \sphinxstyleemphasis{float}.

\item {} 
\sphinxAtStartPar
Examples: \(4.89, 67.98, 5.0\)

\end{itemize}

\item {} 
\sphinxAtStartPar
Strings
\begin{itemize}
\item {} 
\sphinxAtStartPar
These are ordered sequences of characters.

\item {} 
\sphinxAtStartPar
They are enclosed in quotes.

\item {} 
\sphinxAtStartPar
Their class is called \sphinxstyleemphasis{str}.

\item {} 
\sphinxAtStartPar
Examples: ‘USA’, ‘Michael’, ‘3’, ‘5.67’, ‘True’

\end{itemize}

\item {} 
\sphinxAtStartPar
Booleans
\begin{itemize}
\item {} 
\sphinxAtStartPar
There are only two boolean values: \sphinxcode{\sphinxupquote{True}} and \sphinxcode{\sphinxupquote{False}}.

\item {} 
\sphinxAtStartPar
They represent the truth values of boolean expressions.

\item {} 
\sphinxAtStartPar
They are keywords in Python and cannot be used as variable names.

\item {} 
\sphinxAtStartPar
Their class is called \sphinxstyleemphasis{bool}.

\item {} 
\sphinxAtStartPar
The boolean \sphinxcode{\sphinxupquote{True}} and the string \sphinxcode{\sphinxupquote{'True'}} are different.

\item {} 
\sphinxAtStartPar
The boolean \sphinxcode{\sphinxupquote{False}} and the string \sphinxcode{\sphinxupquote{'False'}} are different.

\end{itemize}

\end{enumerate}

\sphinxAtStartPar
In Python, there’s no need to explicitly specify the type of a variable.
\begin{itemize}
\item {} 
\sphinxAtStartPar
The type is determined automatically based on the assigned value.

\end{itemize}


\section{type()}
\label{\detokenize{variables:type}}
\sphinxAtStartPar
The \sphinxcode{\sphinxupquote{type()}} function is a built\sphinxhyphen{}in function that returns the data type of a variable or value.

\begin{sphinxuseclass}{cell}\begin{sphinxVerbatimInput}

\begin{sphinxuseclass}{cell_input}
\begin{sphinxVerbatim}[commandchars=\\\{\}]
\PYG{c+c1}{\PYGZsh{} type of  2}
\PYG{n+nb}{print}\PYG{p}{(}\PYG{n+nb}{type}\PYG{p}{(}\PYG{l+m+mi}{2}\PYG{p}{)}\PYG{p}{)}
\end{sphinxVerbatim}

\end{sphinxuseclass}\end{sphinxVerbatimInput}
\begin{sphinxVerbatimOutput}

\begin{sphinxuseclass}{cell_output}
\begin{sphinxVerbatim}[commandchars=\\\{\}]
\PYGZlt{}class \PYGZsq{}int\PYGZsq{}\PYGZgt{}
\end{sphinxVerbatim}

\end{sphinxuseclass}\end{sphinxVerbatimOutput}

\end{sphinxuseclass}
\begin{sphinxuseclass}{cell}\begin{sphinxVerbatimInput}

\begin{sphinxuseclass}{cell_input}
\begin{sphinxVerbatim}[commandchars=\\\{\}]
\PYG{c+c1}{\PYGZsh{} type of  2.97}
\PYG{n+nb}{print}\PYG{p}{(}\PYG{n+nb}{type}\PYG{p}{(}\PYG{l+m+mf}{2.97}\PYG{p}{)}\PYG{p}{)}
\end{sphinxVerbatim}

\end{sphinxuseclass}\end{sphinxVerbatimInput}
\begin{sphinxVerbatimOutput}

\begin{sphinxuseclass}{cell_output}
\begin{sphinxVerbatim}[commandchars=\\\{\}]
\PYGZlt{}class \PYGZsq{}float\PYGZsq{}\PYGZgt{}
\end{sphinxVerbatim}

\end{sphinxuseclass}\end{sphinxVerbatimOutput}

\end{sphinxuseclass}
\begin{sphinxuseclass}{cell}\begin{sphinxVerbatimInput}

\begin{sphinxuseclass}{cell_input}
\begin{sphinxVerbatim}[commandchars=\\\{\}]
\PYG{c+c1}{\PYGZsh{} type of 2.0}
\PYG{n+nb}{print}\PYG{p}{(}\PYG{n+nb}{type}\PYG{p}{(}\PYG{l+m+mf}{2.0}\PYG{p}{)}\PYG{p}{)}
\end{sphinxVerbatim}

\end{sphinxuseclass}\end{sphinxVerbatimInput}
\begin{sphinxVerbatimOutput}

\begin{sphinxuseclass}{cell_output}
\begin{sphinxVerbatim}[commandchars=\\\{\}]
\PYGZlt{}class \PYGZsq{}float\PYGZsq{}\PYGZgt{}
\end{sphinxVerbatim}

\end{sphinxuseclass}\end{sphinxVerbatimOutput}

\end{sphinxuseclass}
\begin{sphinxuseclass}{cell}\begin{sphinxVerbatimInput}

\begin{sphinxuseclass}{cell_input}
\begin{sphinxVerbatim}[commandchars=\\\{\}]
\PYG{c+c1}{\PYGZsh{} type of \PYGZsq{}Florida\PYGZsq{}}
\PYG{n+nb}{print}\PYG{p}{(}\PYG{n+nb}{type}\PYG{p}{(}\PYG{l+s+s1}{\PYGZsq{}}\PYG{l+s+s1}{Florida}\PYG{l+s+s1}{\PYGZsq{}}\PYG{p}{)}\PYG{p}{)}
\end{sphinxVerbatim}

\end{sphinxuseclass}\end{sphinxVerbatimInput}
\begin{sphinxVerbatimOutput}

\begin{sphinxuseclass}{cell_output}
\begin{sphinxVerbatim}[commandchars=\\\{\}]
\PYGZlt{}class \PYGZsq{}str\PYGZsq{}\PYGZgt{}
\end{sphinxVerbatim}

\end{sphinxuseclass}\end{sphinxVerbatimOutput}

\end{sphinxuseclass}
\begin{sphinxuseclass}{cell}\begin{sphinxVerbatimInput}

\begin{sphinxuseclass}{cell_input}
\begin{sphinxVerbatim}[commandchars=\\\{\}]
\PYG{c+c1}{\PYGZsh{} type of boolean True}
\PYG{n+nb}{print}\PYG{p}{(}\PYG{n+nb}{type}\PYG{p}{(}\PYG{k+kc}{True}\PYG{p}{)}\PYG{p}{)}
\end{sphinxVerbatim}

\end{sphinxuseclass}\end{sphinxVerbatimInput}
\begin{sphinxVerbatimOutput}

\begin{sphinxuseclass}{cell_output}
\begin{sphinxVerbatim}[commandchars=\\\{\}]
\PYGZlt{}class \PYGZsq{}bool\PYGZsq{}\PYGZgt{}
\end{sphinxVerbatim}

\end{sphinxuseclass}\end{sphinxVerbatimOutput}

\end{sphinxuseclass}
\begin{sphinxuseclass}{cell}\begin{sphinxVerbatimInput}

\begin{sphinxuseclass}{cell_input}
\begin{sphinxVerbatim}[commandchars=\\\{\}]
\PYG{c+c1}{\PYGZsh{} type of boolean False}
\PYG{n+nb}{print}\PYG{p}{(}\PYG{n+nb}{type}\PYG{p}{(}\PYG{k+kc}{False}\PYG{p}{)}\PYG{p}{)}
\end{sphinxVerbatim}

\end{sphinxuseclass}\end{sphinxVerbatimInput}
\begin{sphinxVerbatimOutput}

\begin{sphinxuseclass}{cell_output}
\begin{sphinxVerbatim}[commandchars=\\\{\}]
\PYGZlt{}class \PYGZsq{}bool\PYGZsq{}\PYGZgt{}
\end{sphinxVerbatim}

\end{sphinxuseclass}\end{sphinxVerbatimOutput}

\end{sphinxuseclass}
\begin{sphinxuseclass}{cell}\begin{sphinxVerbatimInput}

\begin{sphinxuseclass}{cell_input}
\begin{sphinxVerbatim}[commandchars=\\\{\}]
\PYG{c+c1}{\PYGZsh{} type of string \PYGZsq{}True\PYGZsq{}}
\PYG{n+nb}{print}\PYG{p}{(}\PYG{n+nb}{type}\PYG{p}{(}\PYG{l+s+s1}{\PYGZsq{}}\PYG{l+s+s1}{True}\PYG{l+s+s1}{\PYGZsq{}}\PYG{p}{)}\PYG{p}{)}
\end{sphinxVerbatim}

\end{sphinxuseclass}\end{sphinxVerbatimInput}
\begin{sphinxVerbatimOutput}

\begin{sphinxuseclass}{cell_output}
\begin{sphinxVerbatim}[commandchars=\\\{\}]
\PYGZlt{}class \PYGZsq{}str\PYGZsq{}\PYGZgt{}
\end{sphinxVerbatim}

\end{sphinxuseclass}\end{sphinxVerbatimOutput}

\end{sphinxuseclass}
\begin{sphinxuseclass}{cell}\begin{sphinxVerbatimInput}

\begin{sphinxuseclass}{cell_input}
\begin{sphinxVerbatim}[commandchars=\\\{\}]
\PYG{c+c1}{\PYGZsh{} type of  string \PYGZsq{}2\PYGZsq{}}
\PYG{n+nb}{print}\PYG{p}{(}\PYG{n+nb}{type}\PYG{p}{(}\PYG{l+s+s1}{\PYGZsq{}}\PYG{l+s+s1}{2}\PYG{l+s+s1}{\PYGZsq{}}\PYG{p}{)}\PYG{p}{)}
\end{sphinxVerbatim}

\end{sphinxuseclass}\end{sphinxVerbatimInput}
\begin{sphinxVerbatimOutput}

\begin{sphinxuseclass}{cell_output}
\begin{sphinxVerbatim}[commandchars=\\\{\}]
\PYGZlt{}class \PYGZsq{}str\PYGZsq{}\PYGZgt{}
\end{sphinxVerbatim}

\end{sphinxuseclass}\end{sphinxVerbatimOutput}

\end{sphinxuseclass}
\sphinxAtStartPar
Warning:
\begin{itemize}
\item {} 
\sphinxAtStartPar
\sphinxcode{\sphinxupquote{2}} is an integer, making it a number

\item {} 
\sphinxAtStartPar
\sphinxcode{\sphinxupquote{'2'}}  is a string; it represents the character 2 and is not considered a number.

\end{itemize}


\section{Variable names}
\label{\detokenize{variables:variable-names}}
\sphinxAtStartPar
Variable names can contain lowercase and uppercase characters, digits, and underscores with the following properties:
\begin{itemize}
\item {} 
\sphinxAtStartPar
Variable names are case\sphinxhyphen{}sensitive.

\item {} 
\sphinxAtStartPar
A variable name cannot start with a digit.

\item {} 
\sphinxAtStartPar
The underscore character \sphinxcode{\sphinxupquote{\_}} can be used in a name as a substitute for a space.

\item {} 
\sphinxAtStartPar
Variable names cannot be any of the keywords (reserved words) listed below:

\end{itemize}

\begin{sphinxuseclass}{cell}\begin{sphinxVerbatimInput}

\begin{sphinxuseclass}{cell_input}
\begin{sphinxVerbatim}[commandchars=\\\{\}]
\PYG{c+c1}{\PYGZsh{} keywords in Python}
\PYG{n}{help}\PYG{p}{(}\PYG{l+s+s1}{\PYGZsq{}}\PYG{l+s+s1}{keywords}\PYG{l+s+s1}{\PYGZsq{}}\PYG{p}{)}
\end{sphinxVerbatim}

\end{sphinxuseclass}\end{sphinxVerbatimInput}
\begin{sphinxVerbatimOutput}

\begin{sphinxuseclass}{cell_output}
\begin{sphinxVerbatim}[commandchars=\\\{\}]
Here is a list of the Python keywords.  Enter any keyword to get more help.

False               class               from                or
None                continue            global              pass
True                def                 if                  raise
and                 del                 import              return
as                  elif                in                  try
assert              else                is                  while
async               except              lambda              with
await               finally             nonlocal            yield
break               for                 not                 
\end{sphinxVerbatim}

\end{sphinxuseclass}\end{sphinxVerbatimOutput}

\end{sphinxuseclass}
\begin{sphinxuseclass}{cell}\begin{sphinxVerbatimInput}

\begin{sphinxuseclass}{cell_input}
\begin{sphinxVerbatim}[commandchars=\\\{\}]
\PYG{c+c1}{\PYGZsh{} case sensitive}
\PYG{n}{name} \PYG{o}{=} \PYG{l+s+s1}{\PYGZsq{}}\PYG{l+s+s1}{Jim}\PYG{l+s+s1}{\PYGZsq{}}
\PYG{n}{Name} \PYG{o}{=} \PYG{l+s+s1}{\PYGZsq{}}\PYG{l+s+s1}{Joe}\PYG{l+s+s1}{\PYGZsq{}}
\end{sphinxVerbatim}

\end{sphinxuseclass}\end{sphinxVerbatimInput}

\end{sphinxuseclass}\begin{itemize}
\item {} 
\sphinxAtStartPar
The value of the \sphinxcode{\sphinxupquote{Name}} variable is different from that of the \sphinxcode{\sphinxupquote{name}} variable.

\end{itemize}

\begin{sphinxuseclass}{cell}\begin{sphinxVerbatimInput}

\begin{sphinxuseclass}{cell_input}
\begin{sphinxVerbatim}[commandchars=\\\{\}]
\PYG{n+nb}{print}\PYG{p}{(}\PYG{n}{name}\PYG{p}{)}
\end{sphinxVerbatim}

\end{sphinxuseclass}\end{sphinxVerbatimInput}
\begin{sphinxVerbatimOutput}

\begin{sphinxuseclass}{cell_output}
\begin{sphinxVerbatim}[commandchars=\\\{\}]
Jim
\end{sphinxVerbatim}

\end{sphinxuseclass}\end{sphinxVerbatimOutput}

\end{sphinxuseclass}
\begin{sphinxuseclass}{cell}\begin{sphinxVerbatimInput}

\begin{sphinxuseclass}{cell_input}
\begin{sphinxVerbatim}[commandchars=\\\{\}]
\PYG{n+nb}{print}\PYG{p}{(}\PYG{n}{Name}\PYG{p}{)}
\end{sphinxVerbatim}

\end{sphinxuseclass}\end{sphinxVerbatimInput}
\begin{sphinxVerbatimOutput}

\begin{sphinxuseclass}{cell_output}
\begin{sphinxVerbatim}[commandchars=\\\{\}]
Joe
\end{sphinxVerbatim}

\end{sphinxuseclass}\end{sphinxVerbatimOutput}

\end{sphinxuseclass}
\begin{sphinxVerbatim}[commandchars=\\\{\}]
\PYG{c+c1}{\PYGZsh{} ERROR: 5x is not a valid variable name}
\PYG{l+m+mi}{5}\PYG{n}{x}\PYG{o}{=}\PYG{l+m+mi}{9}
\end{sphinxVerbatim}

\begin{sphinxVerbatim}[commandchars=\\\{\}]
\PYG{c+c1}{\PYGZsh{} ERROR: keyword error}
\PYG{k+kc}{False} \PYG{o}{=} \PYG{l+m+mi}{5}
\end{sphinxVerbatim}

\begin{sphinxVerbatim}[commandchars=\\\{\}]
\PYG{c+c1}{\PYGZsh{} ERROR: symbol error}
\PYG{n+nd}{@fun}\PYG{o}{=}\PYG{l+m+mi}{4}
\end{sphinxVerbatim}

\sphinxAtStartPar
\sphinxstylestrong{Remark:} Function names can be used as a  variable name but function properties will be lost.
\begin{itemize}
\item {} 
\sphinxAtStartPar
Therefore, it is not recommended to use the function names as variable names

\end{itemize}

\begin{sphinxVerbatim}[commandchars=\\\{\}]
\PYG{n+nb}{sum} \PYG{o}{=} \PYG{l+m+mi}{6}
\PYG{n+nb}{sum}\PYG{p}{(}\PYG{p}{[}\PYG{l+m+mi}{7}\PYG{p}{,}\PYG{l+m+mi}{8}\PYG{p}{]}\PYG{p}{)} \PYG{c+c1}{\PYGZsh{} ERROR: sum can not do addition}
\end{sphinxVerbatim}
\begin{itemize}
\item {} 
\sphinxAtStartPar
In the first line of the code above, \sphinxstyleemphasis{sum} becomes a variable and its value is 6.

\item {} 
\sphinxAtStartPar
In the second line, \sphinxstyleemphasis{sum} is attempted to be used as a function, but it is no longer a function, resulting in an error message.

\end{itemize}


\section{Reassignment}
\label{\detokenize{variables:reassignment}}\begin{itemize}
\item {} 
\sphinxAtStartPar
The value of a variable can be changed by assigning it a new value.

\end{itemize}

\begin{sphinxuseclass}{cell}\begin{sphinxVerbatimInput}

\begin{sphinxuseclass}{cell_input}
\begin{sphinxVerbatim}[commandchars=\\\{\}]
\PYG{n}{age} \PYG{o}{=} \PYG{l+m+mi}{50}
\PYG{n+nb}{print}\PYG{p}{(}\PYG{n}{age}\PYG{p}{)}
\PYG{n}{age} \PYG{o}{=} \PYG{l+m+mi}{25}     \PYG{c+c1}{\PYGZsh{} The value of the age variable has been changed.}
\PYG{n+nb}{print}\PYG{p}{(}\PYG{n}{age}\PYG{p}{)}
\PYG{n}{age} \PYG{o}{=} \PYG{l+s+s1}{\PYGZsq{}}\PYG{l+s+s1}{ten}\PYG{l+s+s1}{\PYGZsq{}}  \PYG{c+c1}{\PYGZsh{} The type can also be changed.}
\PYG{n+nb}{print}\PYG{p}{(}\PYG{n}{age}\PYG{p}{)}
\end{sphinxVerbatim}

\end{sphinxuseclass}\end{sphinxVerbatimInput}
\begin{sphinxVerbatimOutput}

\begin{sphinxuseclass}{cell_output}
\begin{sphinxVerbatim}[commandchars=\\\{\}]
50
25
ten
\end{sphinxVerbatim}

\end{sphinxuseclass}\end{sphinxVerbatimOutput}

\end{sphinxuseclass}

\section{Multiple assignment}
\label{\detokenize{variables:multiple-assignment}}
\sphinxAtStartPar
Multiple variables can be created in a single line.

\begin{sphinxuseclass}{cell}\begin{sphinxVerbatimInput}

\begin{sphinxuseclass}{cell_input}
\begin{sphinxVerbatim}[commandchars=\\\{\}]
\PYG{c+c1}{\PYGZsh{} a=1, b=10, c=100}
\PYG{n}{a}\PYG{p}{,} \PYG{n}{b}\PYG{p}{,} \PYG{n}{c} \PYG{o}{=} \PYG{l+m+mi}{1}\PYG{p}{,} \PYG{l+m+mi}{10}\PYG{p}{,} \PYG{l+m+mi}{100}
\end{sphinxVerbatim}

\end{sphinxuseclass}\end{sphinxVerbatimInput}

\end{sphinxuseclass}
\begin{sphinxuseclass}{cell}\begin{sphinxVerbatimInput}

\begin{sphinxuseclass}{cell_input}
\begin{sphinxVerbatim}[commandchars=\\\{\}]
\PYG{n+nb}{print}\PYG{p}{(}\PYG{n}{a}\PYG{p}{)}
\PYG{n+nb}{print}\PYG{p}{(}\PYG{n}{b}\PYG{p}{)}
\PYG{n+nb}{print}\PYG{p}{(}\PYG{n}{c}\PYG{p}{)}
\end{sphinxVerbatim}

\end{sphinxuseclass}\end{sphinxVerbatimInput}
\begin{sphinxVerbatimOutput}

\begin{sphinxuseclass}{cell_output}
\begin{sphinxVerbatim}[commandchars=\\\{\}]
1
10
100
\end{sphinxVerbatim}

\end{sphinxuseclass}\end{sphinxVerbatimOutput}

\end{sphinxuseclass}\begin{itemize}
\item {} 
\sphinxAtStartPar
In the following statement, the value 7 is assigned to both variables x and y.

\item {} 
\sphinxAtStartPar
It’s a shorthand way of assigning the same value to multiple variables in a single line.

\end{itemize}

\begin{sphinxuseclass}{cell}\begin{sphinxVerbatimInput}

\begin{sphinxuseclass}{cell_input}
\begin{sphinxVerbatim}[commandchars=\\\{\}]
\PYG{n}{x} \PYG{o}{=} \PYG{n}{y} \PYG{o}{=} \PYG{l+m+mi}{7}
\end{sphinxVerbatim}

\end{sphinxuseclass}\end{sphinxVerbatimInput}

\end{sphinxuseclass}
\begin{sphinxuseclass}{cell}\begin{sphinxVerbatimInput}

\begin{sphinxuseclass}{cell_input}
\begin{sphinxVerbatim}[commandchars=\\\{\}]
\PYG{n+nb}{print}\PYG{p}{(}\PYG{n}{x}\PYG{p}{)}
\PYG{n+nb}{print}\PYG{p}{(}\PYG{n}{y}\PYG{p}{)}
\end{sphinxVerbatim}

\end{sphinxuseclass}\end{sphinxVerbatimInput}
\begin{sphinxVerbatimOutput}

\begin{sphinxuseclass}{cell_output}
\begin{sphinxVerbatim}[commandchars=\\\{\}]
7
7
\end{sphinxVerbatim}

\end{sphinxuseclass}\end{sphinxVerbatimOutput}

\end{sphinxuseclass}

\section{Print variables and constants}
\label{\detokenize{variables:print-variables-and-constants}}
\sphinxAtStartPar
Variable values and constant strings can be displayed using a single \sphinxstyleemphasis{print()} function. There are several ways to do this.
\begin{itemize}
\item {} 
\sphinxAtStartPar
One way is to separate variables and constant strings with commas.

\item {} 
\sphinxAtStartPar
By default, there will be a space in the output between each comma\sphinxhyphen{}separated item.

\end{itemize}

\begin{sphinxuseclass}{cell}\begin{sphinxVerbatimInput}

\begin{sphinxuseclass}{cell_input}
\begin{sphinxVerbatim}[commandchars=\\\{\}]
\PYG{n}{name}\PYG{p}{,} \PYG{n}{age} \PYG{o}{=} \PYG{l+s+s1}{\PYGZsq{}}\PYG{l+s+s1}{Michael}\PYG{l+s+s1}{\PYGZsq{}}\PYG{p}{,} \PYG{l+m+mi}{25}
\end{sphinxVerbatim}

\end{sphinxuseclass}\end{sphinxVerbatimInput}

\end{sphinxuseclass}
\begin{sphinxuseclass}{cell}\begin{sphinxVerbatimInput}

\begin{sphinxuseclass}{cell_input}
\begin{sphinxVerbatim}[commandchars=\\\{\}]
\PYG{n+nb}{print}\PYG{p}{(}\PYG{l+s+s1}{\PYGZsq{}}\PYG{l+s+s1}{My name is}\PYG{l+s+s1}{\PYGZsq{}}\PYG{p}{,} \PYG{n}{name}\PYG{p}{)}
\end{sphinxVerbatim}

\end{sphinxuseclass}\end{sphinxVerbatimInput}
\begin{sphinxVerbatimOutput}

\begin{sphinxuseclass}{cell_output}
\begin{sphinxVerbatim}[commandchars=\\\{\}]
My name is Michael
\end{sphinxVerbatim}

\end{sphinxuseclass}\end{sphinxVerbatimOutput}

\end{sphinxuseclass}
\begin{sphinxuseclass}{cell}\begin{sphinxVerbatimInput}

\begin{sphinxuseclass}{cell_input}
\begin{sphinxVerbatim}[commandchars=\\\{\}]
\PYG{n+nb}{print}\PYG{p}{(}\PYG{l+s+s1}{\PYGZsq{}}\PYG{l+s+s1}{I am}\PYG{l+s+s1}{\PYGZsq{}}\PYG{p}{,} \PYG{n}{age}\PYG{p}{,} \PYG{l+s+s1}{\PYGZsq{}}\PYG{l+s+s1}{years old.}\PYG{l+s+s1}{\PYGZsq{}}\PYG{p}{)}
\end{sphinxVerbatim}

\end{sphinxuseclass}\end{sphinxVerbatimInput}
\begin{sphinxVerbatimOutput}

\begin{sphinxuseclass}{cell_output}
\begin{sphinxVerbatim}[commandchars=\\\{\}]
I am 25 years old.
\end{sphinxVerbatim}

\end{sphinxuseclass}\end{sphinxVerbatimOutput}

\end{sphinxuseclass}
\begin{sphinxuseclass}{cell}\begin{sphinxVerbatimInput}

\begin{sphinxuseclass}{cell_input}
\begin{sphinxVerbatim}[commandchars=\\\{\}]
\PYG{n+nb}{print}\PYG{p}{(}\PYG{l+s+s1}{\PYGZsq{}}\PYG{l+s+s1}{My name is}\PYG{l+s+s1}{\PYGZsq{}}\PYG{p}{,} \PYG{n}{name}\PYG{p}{,} \PYG{l+s+s1}{\PYGZsq{}}\PYG{l+s+s1}{.I am}\PYG{l+s+s1}{\PYGZsq{}}\PYG{p}{,} \PYG{n}{age}\PYG{p}{,} \PYG{l+s+s1}{\PYGZsq{}}\PYG{l+s+s1}{years old.}\PYG{l+s+s1}{\PYGZsq{}}\PYG{p}{)}
\end{sphinxVerbatim}

\end{sphinxuseclass}\end{sphinxVerbatimInput}
\begin{sphinxVerbatimOutput}

\begin{sphinxuseclass}{cell_output}
\begin{sphinxVerbatim}[commandchars=\\\{\}]
My name is Michael .I am 25 years old.
\end{sphinxVerbatim}

\end{sphinxuseclass}\end{sphinxVerbatimOutput}

\end{sphinxuseclass}\begin{itemize}
\item {} 
\sphinxAtStartPar
You can do algebraic operations in the print function

\end{itemize}

\begin{sphinxuseclass}{cell}\begin{sphinxVerbatimInput}

\begin{sphinxuseclass}{cell_input}
\begin{sphinxVerbatim}[commandchars=\\\{\}]
\PYG{n+nb}{print}\PYG{p}{(}\PYG{l+m+mi}{2024}\PYG{o}{\PYGZhy{}}\PYG{n}{age}\PYG{p}{)}
\end{sphinxVerbatim}

\end{sphinxuseclass}\end{sphinxVerbatimInput}
\begin{sphinxVerbatimOutput}

\begin{sphinxuseclass}{cell_output}
\begin{sphinxVerbatim}[commandchars=\\\{\}]
1999
\end{sphinxVerbatim}

\end{sphinxuseclass}\end{sphinxVerbatimOutput}

\end{sphinxuseclass}

\section{Changing the type of a variable}
\label{\detokenize{variables:changing-the-type-of-a-variable}}\begin{itemize}
\item {} 
\sphinxAtStartPar
int(): converts appropriate floats and strings into integers

\item {} 
\sphinxAtStartPar
float(): converts appropriate integers and strings into floats

\item {} 
\sphinxAtStartPar
str(): converts appropriate integers and floats into strings

\end{itemize}

\begin{sphinxuseclass}{cell}\begin{sphinxVerbatimInput}

\begin{sphinxuseclass}{cell_input}
\begin{sphinxVerbatim}[commandchars=\\\{\}]
\PYG{c+c1}{\PYGZsh{} float to int}
\PYG{n+nb}{print}\PYG{p}{(}\PYG{n+nb}{int}\PYG{p}{(}\PYG{l+m+mf}{3.45}\PYG{p}{)}\PYG{p}{)}
\end{sphinxVerbatim}

\end{sphinxuseclass}\end{sphinxVerbatimInput}
\begin{sphinxVerbatimOutput}

\begin{sphinxuseclass}{cell_output}
\begin{sphinxVerbatim}[commandchars=\\\{\}]
3
\end{sphinxVerbatim}

\end{sphinxuseclass}\end{sphinxVerbatimOutput}

\end{sphinxuseclass}
\begin{sphinxuseclass}{cell}\begin{sphinxVerbatimInput}

\begin{sphinxuseclass}{cell_input}
\begin{sphinxVerbatim}[commandchars=\\\{\}]
\PYG{c+c1}{\PYGZsh{} float to int}
\PYG{n+nb}{print}\PYG{p}{(}\PYG{n+nb}{int}\PYG{p}{(}\PYG{o}{\PYGZhy{}}\PYG{l+m+mf}{3.2}\PYG{p}{)}\PYG{p}{)}
\end{sphinxVerbatim}

\end{sphinxuseclass}\end{sphinxVerbatimInput}
\begin{sphinxVerbatimOutput}

\begin{sphinxuseclass}{cell_output}
\begin{sphinxVerbatim}[commandchars=\\\{\}]
\PYGZhy{}3
\end{sphinxVerbatim}

\end{sphinxuseclass}\end{sphinxVerbatimOutput}

\end{sphinxuseclass}
\begin{sphinxuseclass}{cell}\begin{sphinxVerbatimInput}

\begin{sphinxuseclass}{cell_input}
\begin{sphinxVerbatim}[commandchars=\\\{\}]
\PYG{c+c1}{\PYGZsh{} str to int}
\PYG{n+nb}{print}\PYG{p}{(}\PYG{n+nb}{int}\PYG{p}{(}\PYG{l+s+s1}{\PYGZsq{}}\PYG{l+s+s1}{3}\PYG{l+s+s1}{\PYGZsq{}}\PYG{p}{)}\PYG{p}{)}
\end{sphinxVerbatim}

\end{sphinxuseclass}\end{sphinxVerbatimInput}
\begin{sphinxVerbatimOutput}

\begin{sphinxuseclass}{cell_output}
\begin{sphinxVerbatim}[commandchars=\\\{\}]
3
\end{sphinxVerbatim}

\end{sphinxuseclass}\end{sphinxVerbatimOutput}

\end{sphinxuseclass}
\begin{sphinxVerbatim}[commandchars=\\\{\}]
\PYG{c+c1}{\PYGZsh{} ERROR: string \PYGZsq{}3.59\PYGZsq{} to int}
\PYG{n+nb}{print}\PYG{p}{(}\PYG{n+nb}{int}\PYG{p}{(}\PYG{l+s+s1}{\PYGZsq{}}\PYG{l+s+s1}{3.59}\PYG{l+s+s1}{\PYGZsq{}}\PYG{p}{)}\PYG{p}{)}
\end{sphinxVerbatim}

\begin{sphinxuseclass}{cell}\begin{sphinxVerbatimInput}

\begin{sphinxuseclass}{cell_input}
\begin{sphinxVerbatim}[commandchars=\\\{\}]
\PYG{c+c1}{\PYGZsh{} int to float}
\PYG{n+nb}{print}\PYG{p}{(}\PYG{n+nb}{float}\PYG{p}{(}\PYG{l+m+mi}{1}\PYG{p}{)}\PYG{p}{)}
\end{sphinxVerbatim}

\end{sphinxuseclass}\end{sphinxVerbatimInput}
\begin{sphinxVerbatimOutput}

\begin{sphinxuseclass}{cell_output}
\begin{sphinxVerbatim}[commandchars=\\\{\}]
1.0
\end{sphinxVerbatim}

\end{sphinxuseclass}\end{sphinxVerbatimOutput}

\end{sphinxuseclass}
\begin{sphinxuseclass}{cell}\begin{sphinxVerbatimInput}

\begin{sphinxuseclass}{cell_input}
\begin{sphinxVerbatim}[commandchars=\\\{\}]
\PYG{c+c1}{\PYGZsh{} str to float}
\PYG{n+nb}{print}\PYG{p}{(}\PYG{n+nb}{float}\PYG{p}{(}\PYG{l+s+s1}{\PYGZsq{}}\PYG{l+s+s1}{3.5}\PYG{l+s+s1}{\PYGZsq{}}\PYG{p}{)}\PYG{p}{)}
\end{sphinxVerbatim}

\end{sphinxuseclass}\end{sphinxVerbatimInput}
\begin{sphinxVerbatimOutput}

\begin{sphinxuseclass}{cell_output}
\begin{sphinxVerbatim}[commandchars=\\\{\}]
3.5
\end{sphinxVerbatim}

\end{sphinxuseclass}\end{sphinxVerbatimOutput}

\end{sphinxuseclass}
\begin{sphinxuseclass}{cell}\begin{sphinxVerbatimInput}

\begin{sphinxuseclass}{cell_input}
\begin{sphinxVerbatim}[commandchars=\\\{\}]
\PYG{c+c1}{\PYGZsh{} int to str}
\PYG{n+nb}{print}\PYG{p}{(}\PYG{n+nb}{str}\PYG{p}{(}\PYG{l+m+mi}{1}\PYG{p}{)}\PYG{p}{)}
\end{sphinxVerbatim}

\end{sphinxuseclass}\end{sphinxVerbatimInput}
\begin{sphinxVerbatimOutput}

\begin{sphinxuseclass}{cell_output}
\begin{sphinxVerbatim}[commandchars=\\\{\}]
1
\end{sphinxVerbatim}

\end{sphinxuseclass}\end{sphinxVerbatimOutput}

\end{sphinxuseclass}
\begin{sphinxuseclass}{cell}\begin{sphinxVerbatimInput}

\begin{sphinxuseclass}{cell_input}
\begin{sphinxVerbatim}[commandchars=\\\{\}]
\PYG{c+c1}{\PYGZsh{} float to str}
\PYG{n+nb}{print}\PYG{p}{(}\PYG{n+nb}{str}\PYG{p}{(}\PYG{l+m+mf}{3.5}\PYG{p}{)}\PYG{p}{)}
\end{sphinxVerbatim}

\end{sphinxuseclass}\end{sphinxVerbatimInput}
\begin{sphinxVerbatimOutput}

\begin{sphinxuseclass}{cell_output}
\begin{sphinxVerbatim}[commandchars=\\\{\}]
3.5
\end{sphinxVerbatim}

\end{sphinxuseclass}\end{sphinxVerbatimOutput}

\end{sphinxuseclass}
\begin{sphinxuseclass}{cell}\begin{sphinxVerbatimInput}

\begin{sphinxuseclass}{cell_input}
\begin{sphinxVerbatim}[commandchars=\\\{\}]
\PYG{n}{age} \PYG{o}{=} \PYG{l+m+mi}{25}
\PYG{n+nb}{print}\PYG{p}{(}\PYG{l+s+s1}{\PYGZsq{}}\PYG{l+s+s1}{I am }\PYG{l+s+s1}{\PYGZsq{}}\PYG{o}{+}\PYG{n+nb}{str}\PYG{p}{(}\PYG{n}{age}\PYG{p}{)}\PYG{o}{+}\PYG{l+s+s1}{\PYGZsq{}}\PYG{l+s+s1}{ years old.}\PYG{l+s+s1}{\PYGZsq{}}\PYG{p}{)}
\end{sphinxVerbatim}

\end{sphinxuseclass}\end{sphinxVerbatimInput}
\begin{sphinxVerbatimOutput}

\begin{sphinxuseclass}{cell_output}
\begin{sphinxVerbatim}[commandchars=\\\{\}]
I am 25 years old.
\end{sphinxVerbatim}

\end{sphinxuseclass}\end{sphinxVerbatimOutput}

\end{sphinxuseclass}

\section{String Operations}
\label{\detokenize{variables:string-operations}}

\subsection{Concatenation}
\label{\detokenize{variables:concatenation}}\begin{itemize}
\item {} 
\sphinxAtStartPar
Combining or joining two or more strings into a single string by using the \sphinxcode{\sphinxupquote{+}} operator

\end{itemize}

\begin{sphinxuseclass}{cell}\begin{sphinxVerbatimInput}

\begin{sphinxuseclass}{cell_input}
\begin{sphinxVerbatim}[commandchars=\\\{\}]
\PYG{n}{first\PYGZus{}name} \PYG{o}{=} \PYG{l+s+s1}{\PYGZsq{}}\PYG{l+s+s1}{Michael}\PYG{l+s+s1}{\PYGZsq{}}
\PYG{n}{last\PYGZus{}name} \PYG{o}{=}\PYG{l+s+s1}{\PYGZsq{}}\PYG{l+s+s1}{Jordan}\PYG{l+s+s1}{\PYGZsq{}}
\PYG{n}{age} \PYG{o}{=} \PYG{l+m+mi}{25}
\end{sphinxVerbatim}

\end{sphinxuseclass}\end{sphinxVerbatimInput}

\end{sphinxuseclass}\begin{itemize}
\item {} 
\sphinxAtStartPar
In this example, the \sphinxcode{\sphinxupquote{+}} operator concatenates (joins) the value of \sphinxcode{\sphinxupquote{first\_name}} and \sphinxcode{\sphinxupquote{last\_name}} variables to create a new string stored in the variable \sphinxcode{\sphinxupquote{name}}.

\end{itemize}

\begin{sphinxuseclass}{cell}\begin{sphinxVerbatimInput}

\begin{sphinxuseclass}{cell_input}
\begin{sphinxVerbatim}[commandchars=\\\{\}]
\PYG{n}{name}  \PYG{o}{=} \PYG{n}{first\PYGZus{}name} \PYG{o}{+} \PYG{n}{last\PYGZus{}name}
\PYG{n+nb}{print}\PYG{p}{(}\PYG{n}{name}\PYG{p}{)}
\end{sphinxVerbatim}

\end{sphinxuseclass}\end{sphinxVerbatimInput}
\begin{sphinxVerbatimOutput}

\begin{sphinxuseclass}{cell_output}
\begin{sphinxVerbatim}[commandchars=\\\{\}]
MichaelJordan
\end{sphinxVerbatim}

\end{sphinxuseclass}\end{sphinxVerbatimOutput}

\end{sphinxuseclass}\begin{itemize}
\item {} 
\sphinxAtStartPar
Use the string \sphinxcode{\sphinxupquote{' '}} (one space) to add a space between the first and last names.

\end{itemize}

\begin{sphinxuseclass}{cell}\begin{sphinxVerbatimInput}

\begin{sphinxuseclass}{cell_input}
\begin{sphinxVerbatim}[commandchars=\\\{\}]
\PYG{n}{name}  \PYG{o}{=} \PYG{n}{first\PYGZus{}name} \PYG{o}{+} \PYG{l+s+s1}{\PYGZsq{}}\PYG{l+s+s1}{ }\PYG{l+s+s1}{\PYGZsq{}} \PYG{o}{+} \PYG{n}{last\PYGZus{}name}
\PYG{n+nb}{print}\PYG{p}{(}\PYG{n}{name}\PYG{p}{)}
\end{sphinxVerbatim}

\end{sphinxuseclass}\end{sphinxVerbatimInput}
\begin{sphinxVerbatimOutput}

\begin{sphinxuseclass}{cell_output}
\begin{sphinxVerbatim}[commandchars=\\\{\}]
Michael Jordan
\end{sphinxVerbatim}

\end{sphinxuseclass}\end{sphinxVerbatimOutput}

\end{sphinxuseclass}\begin{itemize}
\item {} 
\sphinxAtStartPar
constant string + variable

\end{itemize}

\begin{sphinxuseclass}{cell}\begin{sphinxVerbatimInput}

\begin{sphinxuseclass}{cell_input}
\begin{sphinxVerbatim}[commandchars=\\\{\}]
\PYG{n+nb}{print}\PYG{p}{(}\PYG{l+s+s1}{\PYGZsq{}}\PYG{l+s+s1}{My name is }\PYG{l+s+s1}{\PYGZsq{}} \PYG{o}{+} \PYG{n}{name}\PYG{p}{)}
\end{sphinxVerbatim}

\end{sphinxuseclass}\end{sphinxVerbatimInput}
\begin{sphinxVerbatimOutput}

\begin{sphinxuseclass}{cell_output}
\begin{sphinxVerbatim}[commandchars=\\\{\}]
My name is Michael Jordan
\end{sphinxVerbatim}

\end{sphinxuseclass}\end{sphinxVerbatimOutput}

\end{sphinxuseclass}\begin{itemize}
\item {} 
\sphinxAtStartPar
To perform concatenation, conversion should be done in the following code.

\end{itemize}

\begin{sphinxuseclass}{cell}\begin{sphinxVerbatimInput}

\begin{sphinxuseclass}{cell_input}
\begin{sphinxVerbatim}[commandchars=\\\{\}]
\PYG{n+nb}{print}\PYG{p}{(}\PYG{l+s+s1}{\PYGZsq{}}\PYG{l+s+s1}{I am }\PYG{l+s+s1}{\PYGZsq{}} \PYG{o}{+} \PYG{n+nb}{str}\PYG{p}{(}\PYG{n}{age}\PYG{p}{)} \PYG{o}{+} \PYG{l+s+s1}{\PYGZsq{}}\PYG{l+s+s1}{ years old.}\PYG{l+s+s1}{\PYGZsq{}}\PYG{p}{)}
\end{sphinxVerbatim}

\end{sphinxuseclass}\end{sphinxVerbatimInput}
\begin{sphinxVerbatimOutput}

\begin{sphinxuseclass}{cell_output}
\begin{sphinxVerbatim}[commandchars=\\\{\}]
I am 25 years old.
\end{sphinxVerbatim}

\end{sphinxuseclass}\end{sphinxVerbatimOutput}

\end{sphinxuseclass}\begin{itemize}
\item {} 
\sphinxAtStartPar
\sphinxcode{\sphinxupquote{+}} can not be used between numbers and strings

\end{itemize}

\begin{sphinxVerbatim}[commandchars=\\\{\}]
\PYG{n}{name}\PYG{p}{,} \PYG{n}{age} \PYG{o}{=} \PYG{l+s+s1}{\PYGZsq{}}\PYG{l+s+s1}{Michael}\PYG{l+s+s1}{\PYGZsq{}}\PYG{p}{,} \PYG{l+m+mi}{25}

\PYG{c+c1}{\PYGZsh{} ERROR: str + int}
\PYG{n+nb}{print}\PYG{p}{(}\PYG{n}{name}\PYG{o}{+}\PYG{n}{age}\PYG{p}{)}

\end{sphinxVerbatim}


\subsection{Repetition}
\label{\detokenize{variables:repetition}}\begin{itemize}
\item {} 
\sphinxAtStartPar
Repeating a string using the \sphinxcode{\sphinxupquote{*}} operator.

\item {} 
\sphinxAtStartPar
It takes the form of \sphinxcode{\sphinxupquote{integer*string}} or \sphinxcode{\sphinxupquote{string*integer}}.

\item {} 
\sphinxAtStartPar
String repetition cannot be performed using floats, even with 4.0

\end{itemize}
\begin{itemize}
\item {} 
\sphinxAtStartPar
In this example, the * operator repeats the string ‘U’ four times.

\end{itemize}

\begin{sphinxuseclass}{cell}\begin{sphinxVerbatimInput}

\begin{sphinxuseclass}{cell_input}
\begin{sphinxVerbatim}[commandchars=\\\{\}]
\PYG{n+nb}{print}\PYG{p}{(}\PYG{l+s+s1}{\PYGZsq{}}\PYG{l+s+s1}{U}\PYG{l+s+s1}{\PYGZsq{}}\PYG{o}{*}\PYG{l+m+mi}{4}\PYG{p}{)}
\end{sphinxVerbatim}

\end{sphinxuseclass}\end{sphinxVerbatimInput}
\begin{sphinxVerbatimOutput}

\begin{sphinxuseclass}{cell_output}
\begin{sphinxVerbatim}[commandchars=\\\{\}]
UUUU
\end{sphinxVerbatim}

\end{sphinxuseclass}\end{sphinxVerbatimOutput}

\end{sphinxuseclass}\begin{itemize}
\item {} 
\sphinxAtStartPar
In this example, the * operator repeats the string ‘Joe’ five times.

\end{itemize}

\begin{sphinxuseclass}{cell}\begin{sphinxVerbatimInput}

\begin{sphinxuseclass}{cell_input}
\begin{sphinxVerbatim}[commandchars=\\\{\}]
\PYG{n+nb}{print}\PYG{p}{(}\PYG{l+s+s1}{\PYGZsq{}}\PYG{l+s+s1}{Joe}\PYG{l+s+s1}{\PYGZsq{}}\PYG{o}{*}\PYG{l+m+mi}{5}\PYG{p}{)}
\end{sphinxVerbatim}

\end{sphinxuseclass}\end{sphinxVerbatimInput}
\begin{sphinxVerbatimOutput}

\begin{sphinxuseclass}{cell_output}
\begin{sphinxVerbatim}[commandchars=\\\{\}]
JoeJoeJoeJoeJoe
\end{sphinxVerbatim}

\end{sphinxuseclass}\end{sphinxVerbatimOutput}

\end{sphinxuseclass}\begin{itemize}
\item {} 
\sphinxAtStartPar
Since 4.0 is not an integer, you will get an error message in the following code.

\end{itemize}

\begin{sphinxVerbatim}[commandchars=\\\{\}]
\PYG{c+c1}{\PYGZsh{} ERROR: 4.0 must be 4}
\PYG{n+nb}{print}\PYG{p}{(}\PYG{l+s+s1}{\PYGZsq{}}\PYG{l+s+s1}{U}\PYG{l+s+s1}{\PYGZsq{}}\PYG{o}{*}\PYG{l+m+mf}{4.0}\PYG{p}{)}
\end{sphinxVerbatim}
\begin{itemize}
\item {} 
\sphinxAtStartPar
The following is a triangle constructed using the \$ character.

\end{itemize}

\begin{sphinxuseclass}{cell}\begin{sphinxVerbatimInput}

\begin{sphinxuseclass}{cell_input}
\begin{sphinxVerbatim}[commandchars=\\\{\}]
\PYG{n+nb}{print}\PYG{p}{(}\PYG{l+s+s1}{\PYGZsq{}}\PYG{l+s+s1}{\PYGZdl{}}\PYG{l+s+s1}{\PYGZsq{}}\PYG{p}{)}         \PYG{c+c1}{\PYGZsh{} one   \PYGZdl{} sign}
\PYG{n+nb}{print}\PYG{p}{(}\PYG{l+s+s1}{\PYGZsq{}}\PYG{l+s+s1}{\PYGZdl{}}\PYG{l+s+s1}{\PYGZsq{}}\PYG{o}{*}\PYG{l+m+mi}{2}\PYG{p}{)}       \PYG{c+c1}{\PYGZsh{} two   \PYGZdl{} signs}
\PYG{n+nb}{print}\PYG{p}{(}\PYG{l+s+s1}{\PYGZsq{}}\PYG{l+s+s1}{\PYGZdl{}}\PYG{l+s+s1}{\PYGZsq{}}\PYG{o}{*}\PYG{l+m+mi}{3}\PYG{p}{)}       \PYG{c+c1}{\PYGZsh{} three \PYGZdl{} signs}
\PYG{n+nb}{print}\PYG{p}{(}\PYG{l+s+s1}{\PYGZsq{}}\PYG{l+s+s1}{\PYGZdl{}}\PYG{l+s+s1}{\PYGZsq{}}\PYG{o}{*}\PYG{l+m+mi}{4}\PYG{p}{)}       \PYG{c+c1}{\PYGZsh{} four  \PYGZdl{} signs}
\PYG{n+nb}{print}\PYG{p}{(}\PYG{l+s+s1}{\PYGZsq{}}\PYG{l+s+s1}{\PYGZdl{}}\PYG{l+s+s1}{\PYGZsq{}}\PYG{o}{*}\PYG{l+m+mi}{5}\PYG{p}{)}       \PYG{c+c1}{\PYGZsh{} five  \PYGZdl{} signs}
\end{sphinxVerbatim}

\end{sphinxuseclass}\end{sphinxVerbatimInput}
\begin{sphinxVerbatimOutput}

\begin{sphinxuseclass}{cell_output}
\begin{sphinxVerbatim}[commandchars=\\\{\}]
\PYGZdl{}
\PYGZdl{}\PYGZdl{}
\PYGZdl{}\PYGZdl{}\PYGZdl{}
\PYGZdl{}\PYGZdl{}\PYGZdl{}\PYGZdl{}
\PYGZdl{}\PYGZdl{}\PYGZdl{}\PYGZdl{}\PYGZdl{}
\end{sphinxVerbatim}

\end{sphinxuseclass}\end{sphinxVerbatimOutput}

\end{sphinxuseclass}\begin{itemize}
\item {} 
\sphinxAtStartPar
The following is a triangle constructed using the space and \$ characters.

\end{itemize}

\begin{sphinxuseclass}{cell}\begin{sphinxVerbatimInput}

\begin{sphinxuseclass}{cell_input}
\begin{sphinxVerbatim}[commandchars=\\\{\}]
\PYG{c+c1}{\PYGZsh{} triangle by using the space and \PYGZdl{} characters}
\PYG{n+nb}{print}\PYG{p}{(}\PYG{l+s+s1}{\PYGZsq{}}\PYG{l+s+s1}{ }\PYG{l+s+s1}{\PYGZsq{}}\PYG{o}{*}\PYG{l+m+mi}{4} \PYG{o}{+} \PYG{l+s+s1}{\PYGZsq{}}\PYG{l+s+s1}{\PYGZdl{}}\PYG{l+s+s1}{\PYGZsq{}}\PYG{p}{)}      \PYG{c+c1}{\PYGZsh{} four  spaces and one \PYGZdl{} sign}
\PYG{n+nb}{print}\PYG{p}{(}\PYG{l+s+s1}{\PYGZsq{}}\PYG{l+s+s1}{ }\PYG{l+s+s1}{\PYGZsq{}}\PYG{o}{*}\PYG{l+m+mi}{3} \PYG{o}{+}\PYG{l+s+s1}{\PYGZsq{}}\PYG{l+s+s1}{\PYGZdl{}}\PYG{l+s+s1}{\PYGZsq{}}\PYG{o}{*}\PYG{l+m+mi}{2}\PYG{p}{)}     \PYG{c+c1}{\PYGZsh{} three spaces and two \PYGZdl{} signs}
\PYG{n+nb}{print}\PYG{p}{(}\PYG{l+s+s1}{\PYGZsq{}}\PYG{l+s+s1}{ }\PYG{l+s+s1}{\PYGZsq{}}\PYG{o}{*}\PYG{l+m+mi}{2} \PYG{o}{+}\PYG{l+s+s1}{\PYGZsq{}}\PYG{l+s+s1}{\PYGZdl{}}\PYG{l+s+s1}{\PYGZsq{}}\PYG{o}{*}\PYG{l+m+mi}{3}\PYG{p}{)}     \PYG{c+c1}{\PYGZsh{} two   spaces and three \PYGZdl{} signs}
\PYG{n+nb}{print}\PYG{p}{(}\PYG{l+s+s1}{\PYGZsq{}}\PYG{l+s+s1}{ }\PYG{l+s+s1}{\PYGZsq{}}\PYG{o}{*}\PYG{l+m+mi}{1} \PYG{o}{+}\PYG{l+s+s1}{\PYGZsq{}}\PYG{l+s+s1}{\PYGZdl{}}\PYG{l+s+s1}{\PYGZsq{}}\PYG{o}{*}\PYG{l+m+mi}{4}\PYG{p}{)}     \PYG{c+c1}{\PYGZsh{} one   space  and four \PYGZdl{} signs}
\PYG{n+nb}{print}\PYG{p}{(}\PYG{l+s+s1}{\PYGZsq{}}\PYG{l+s+s1}{\PYGZdl{}}\PYG{l+s+s1}{\PYGZsq{}}\PYG{o}{*}\PYG{l+m+mi}{5}\PYG{p}{)}            \PYG{c+c1}{\PYGZsh{}                  five \PYGZdl{} signs}
\end{sphinxVerbatim}

\end{sphinxuseclass}\end{sphinxVerbatimInput}
\begin{sphinxVerbatimOutput}

\begin{sphinxuseclass}{cell_output}
\begin{sphinxVerbatim}[commandchars=\\\{\}]
    \PYGZdl{}
   \PYGZdl{}\PYGZdl{}
  \PYGZdl{}\PYGZdl{}\PYGZdl{}
 \PYGZdl{}\PYGZdl{}\PYGZdl{}\PYGZdl{}
\PYGZdl{}\PYGZdl{}\PYGZdl{}\PYGZdl{}\PYGZdl{}
\end{sphinxVerbatim}

\end{sphinxuseclass}\end{sphinxVerbatimOutput}

\end{sphinxuseclass}
\sphinxstepscope


\section{Variables Debugging}
\label{\detokenize{variables_debug:variables-debugging}}\label{\detokenize{variables_debug::doc}}\begin{itemize}
\item {} 
\sphinxAtStartPar
Each of the following short code contains one or more bugs.     

\item {} 
\sphinxAtStartPar
Please identify and correct these bugs.

\item {} 
\sphinxAtStartPar
Provide an explanation for your answer.

\end{itemize}


\subsection{Question\sphinxhyphen{}1}
\label{\detokenize{variables_debug:question-1}}
\begin{sphinxVerbatim}[commandchars=\\\{\}]
\PYG{n}{state} \PYG{o}{=} \PYG{l+s+s1}{\PYGZsq{}}\PYG{l+s+s1}{NY}\PYG{l+s+s1}{\PYGZsq{}}
\PYG{n+nb}{int}\PYG{p}{(}\PYG{n}{state}\PYG{p}{)}
\end{sphinxVerbatim}

\begin{sphinxadmonition}{note}{Solution}

\sphinxAtStartPar
string \sphinxcode{\sphinxupquote{state}} can not be converted to an integer.
\end{sphinxadmonition}


\subsection{Question\sphinxhyphen{}2}
\label{\detokenize{variables_debug:question-2}}
\begin{sphinxVerbatim}[commandchars=\\\{\}]
\PYG{l+m+mi}{991}\PYG{n}{\PYGZus{}number} \PYG{o}{=} \PYG{l+s+s1}{\PYGZsq{}}\PYG{l+s+s1}{emergency}\PYG{l+s+s1}{\PYGZsq{}}
\end{sphinxVerbatim}

\begin{sphinxadmonition}{note}{Solution}

\sphinxAtStartPar
variable names can not start with a digit
\end{sphinxadmonition}


\subsection{Question\sphinxhyphen{}3}
\label{\detokenize{variables_debug:question-3}}
\begin{sphinxVerbatim}[commandchars=\\\{\}]
\PYG{n+nb}{float}\PYG{p}{(}\PYG{l+s+s1}{\PYGZsq{}}\PYG{l+s+s1}{HI}\PYG{l+s+s1}{\PYGZsq{}}\PYG{p}{)}
\end{sphinxVerbatim}

\begin{sphinxadmonition}{note}{Solution}

\sphinxAtStartPar
string \sphinxcode{\sphinxupquote{HI}} can not be converted to a float.
\end{sphinxadmonition}


\subsection{Question\sphinxhyphen{}4}
\label{\detokenize{variables_debug:question-4}}
\begin{sphinxVerbatim}[commandchars=\\\{\}]
\PYG{n}{A} \PYG{o}{=} \PYG{l+m+mi}{3}
\PYG{n+nb}{print}\PYG{p}{(}\PYG{n}{a} \PYG{o}{+} \PYG{l+m+mi}{5}\PYG{p}{)}
\end{sphinxVerbatim}

\begin{sphinxadmonition}{note}{Solution}

\sphinxAtStartPar
\sphinxcode{\sphinxupquote{a}} is not defined because capital \sphinxcode{\sphinxupquote{A}} is 3
\end{sphinxadmonition}


\subsection{Question\sphinxhyphen{}5}
\label{\detokenize{variables_debug:question-5}}
\begin{sphinxVerbatim}[commandchars=\\\{\}]
\PYG{k+kc}{True} \PYG{o}{=} \PYG{l+m+mi}{5}
\PYG{n+nb}{print}\PYG{p}{(}\PYG{k+kc}{True}\PYG{p}{)}
\end{sphinxVerbatim}

\begin{sphinxadmonition}{note}{Solution}

\sphinxAtStartPar
Keywords can not be used as a variable names. \sphinxcode{\sphinxupquote{True}} is a keyword.
\end{sphinxadmonition}


\subsection{Question\sphinxhyphen{}6}
\label{\detokenize{variables_debug:question-6}}
\begin{sphinxVerbatim}[commandchars=\\\{\}]
\PYG{n+nb}{print}\PYG{p}{(}\PYG{n+nb}{int}\PYG{p}{(}\PYG{l+s+s1}{\PYGZsq{}}\PYG{l+s+s1}{5.0}\PYG{l+s+s1}{\PYGZsq{}}\PYG{p}{)}\PYG{p}{)}
\end{sphinxVerbatim}

\begin{sphinxadmonition}{note}{Solution}

\sphinxAtStartPar
\sphinxcode{\sphinxupquote{5.0}} is not an integer
\end{sphinxadmonition}


\subsection{Question\sphinxhyphen{}7}
\label{\detokenize{variables_debug:question-7}}
\begin{sphinxVerbatim}[commandchars=\\\{\}]
\PYG{n}{age} \PYG{o}{=} \PYG{l+m+mi}{25}
\PYG{n+nb}{print}\PYG{p}{(}\PYG{l+s+s1}{\PYGZsq{}}\PYG{l+s+s1}{I am}\PYG{l+s+s1}{\PYGZsq{}}\PYG{o}{+}\PYG{n}{age}\PYG{o}{+}\PYG{l+s+s1}{\PYGZsq{}}\PYG{l+s+s1}{years old.}\PYG{l+s+s1}{\PYGZsq{}}\PYG{p}{)}
\end{sphinxVerbatim}

\begin{sphinxadmonition}{note}{Solution}
\begin{itemize}
\item {} 
\sphinxAtStartPar
\sphinxcode{\sphinxupquote{I am}} is a string and \sphinxcode{\sphinxupquote{age}} is an integer.

\item {} 
\sphinxAtStartPar
‘+’ can not be used to concatenate numbers and strings

\end{itemize}
\end{sphinxadmonition}


\subsection{Question\sphinxhyphen{}8}
\label{\detokenize{variables_debug:question-8}}
\begin{sphinxVerbatim}[commandchars=\\\{\}]
\PYG{n+nb}{print} \PYG{o}{=} \PYG{l+m+mi}{25}
\PYG{n+nb}{print}\PYG{p}{(}\PYG{l+s+s1}{\PYGZsq{}}\PYG{l+s+s1}{Hello}\PYG{l+s+s1}{\PYGZsq{}}\PYG{p}{)}
\end{sphinxVerbatim}

\begin{sphinxadmonition}{note}{Solution}
\begin{itemize}
\item {} 
\sphinxAtStartPar
In the first line, \sphinxstyleemphasis{print} becomes a variable, and its value is 25, so it loses its function property.

\end{itemize}
\end{sphinxadmonition}

\sphinxstepscope


\section{Variables Output}
\label{\detokenize{variables_output:variables-output}}\label{\detokenize{variables_output::doc}}\begin{itemize}
\item {} 
\sphinxAtStartPar
Find the output of the following code.

\item {} 
\sphinxAtStartPar
Please don’t run the code before giving your answer.     

\end{itemize}


\subsection{Question\sphinxhyphen{}1}
\label{\detokenize{variables_output:question-1}}
\begin{sphinxuseclass}{cell}
\begin{sphinxuseclass}{tag_hide-output}\begin{sphinxVerbatimInput}

\begin{sphinxuseclass}{cell_input}
\begin{sphinxVerbatim}[commandchars=\\\{\}]
\PYG{n+nb}{print}\PYG{p}{(}\PYG{n+nb}{type}\PYG{p}{(}\PYG{l+m+mi}{5}\PYG{p}{)}\PYG{p}{)}
\end{sphinxVerbatim}

\end{sphinxuseclass}\end{sphinxVerbatimInput}

\end{sphinxuseclass}
\end{sphinxuseclass}

\subsection{Question\sphinxhyphen{}2}
\label{\detokenize{variables_output:question-2}}
\begin{sphinxuseclass}{cell}
\begin{sphinxuseclass}{tag_hide-output}\begin{sphinxVerbatimInput}

\begin{sphinxuseclass}{cell_input}
\begin{sphinxVerbatim}[commandchars=\\\{\}]
\PYG{n+nb}{print}\PYG{p}{(}\PYG{n+nb}{type}\PYG{p}{(}\PYG{l+m+mf}{5.5}\PYG{p}{)}\PYG{p}{)}
\end{sphinxVerbatim}

\end{sphinxuseclass}\end{sphinxVerbatimInput}

\end{sphinxuseclass}
\end{sphinxuseclass}

\subsection{Question\sphinxhyphen{}3}
\label{\detokenize{variables_output:question-3}}
\begin{sphinxuseclass}{cell}
\begin{sphinxuseclass}{tag_hide-output}\begin{sphinxVerbatimInput}

\begin{sphinxuseclass}{cell_input}
\begin{sphinxVerbatim}[commandchars=\\\{\}]
\PYG{n+nb}{print}\PYG{p}{(}\PYG{n+nb}{type}\PYG{p}{(}\PYG{l+m+mf}{5.0}\PYG{p}{)}\PYG{p}{)}
\end{sphinxVerbatim}

\end{sphinxuseclass}\end{sphinxVerbatimInput}

\end{sphinxuseclass}
\end{sphinxuseclass}

\subsection{Question\sphinxhyphen{}4}
\label{\detokenize{variables_output:question-4}}
\begin{sphinxuseclass}{cell}
\begin{sphinxuseclass}{tag_hide-output}\begin{sphinxVerbatimInput}

\begin{sphinxuseclass}{cell_input}
\begin{sphinxVerbatim}[commandchars=\\\{\}]
\PYG{n+nb}{print}\PYG{p}{(}\PYG{n+nb}{type}\PYG{p}{(}\PYG{l+s+s1}{\PYGZsq{}}\PYG{l+s+s1}{5}\PYG{l+s+s1}{\PYGZsq{}}\PYG{p}{)}\PYG{p}{)}
\end{sphinxVerbatim}

\end{sphinxuseclass}\end{sphinxVerbatimInput}

\end{sphinxuseclass}
\end{sphinxuseclass}

\subsection{Question\sphinxhyphen{}5}
\label{\detokenize{variables_output:question-5}}
\begin{sphinxuseclass}{cell}
\begin{sphinxuseclass}{tag_hide-output}\begin{sphinxVerbatimInput}

\begin{sphinxuseclass}{cell_input}
\begin{sphinxVerbatim}[commandchars=\\\{\}]
\PYG{n}{x} \PYG{o}{=} \PYG{l+m+mi}{5}
\PYG{n}{y} \PYG{o}{=} \PYG{l+s+s1}{\PYGZsq{}}\PYG{l+s+s1}{apple}\PYG{l+s+s1}{\PYGZsq{}}
\PYG{n+nb}{print}\PYG{p}{(}\PYG{l+s+s1}{\PYGZsq{}}\PYG{l+s+s1}{I have}\PYG{l+s+s1}{\PYGZsq{}}\PYG{p}{,} \PYG{l+m+mi}{20}\PYG{o}{\PYGZhy{}}\PYG{n}{x}\PYG{p}{,} \PYG{n}{y}\PYG{p}{,}\PYG{l+s+s1}{\PYGZsq{}}\PYG{l+s+s1}{s.}\PYG{l+s+s1}{\PYGZsq{}}\PYG{p}{)}
\end{sphinxVerbatim}

\end{sphinxuseclass}\end{sphinxVerbatimInput}

\end{sphinxuseclass}
\end{sphinxuseclass}

\subsection{Question\sphinxhyphen{}6}
\label{\detokenize{variables_output:question-6}}
\begin{sphinxuseclass}{cell}
\begin{sphinxuseclass}{tag_hide-output}\begin{sphinxVerbatimInput}

\begin{sphinxuseclass}{cell_input}
\begin{sphinxVerbatim}[commandchars=\\\{\}]
\PYG{n}{age} \PYG{o}{=} \PYG{l+m+mi}{25}
\PYG{n+nb}{print}\PYG{p}{(}\PYG{l+s+s1}{\PYGZsq{}}\PYG{l+s+s1}{I am}\PYG{l+s+s1}{\PYGZsq{}} \PYG{o}{+}\PYG{l+s+s1}{\PYGZsq{}}\PYG{l+s+s1}{age}\PYG{l+s+s1}{\PYGZsq{}}\PYG{o}{+}\PYG{l+s+s1}{\PYGZsq{}}\PYG{l+s+s1}{years old.}\PYG{l+s+s1}{\PYGZsq{}}\PYG{p}{)}
\end{sphinxVerbatim}

\end{sphinxuseclass}\end{sphinxVerbatimInput}

\end{sphinxuseclass}
\end{sphinxuseclass}

\subsection{Question\sphinxhyphen{}7}
\label{\detokenize{variables_output:question-7}}
\begin{sphinxuseclass}{cell}
\begin{sphinxuseclass}{tag_hide-output}\begin{sphinxVerbatimInput}

\begin{sphinxuseclass}{cell_input}
\begin{sphinxVerbatim}[commandchars=\\\{\}]
\PYG{n}{age} \PYG{o}{=} \PYG{l+m+mi}{25}
\PYG{n+nb}{print}\PYG{p}{(}\PYG{l+s+s1}{\PYGZsq{}}\PYG{l+s+s1}{I am}\PYG{l+s+s1}{\PYGZsq{}} \PYG{o}{+}\PYG{n+nb}{str}\PYG{p}{(}\PYG{n}{age}\PYG{p}{)}\PYG{o}{+}\PYG{l+s+s1}{\PYGZsq{}}\PYG{l+s+s1}{years old.}\PYG{l+s+s1}{\PYGZsq{}}\PYG{p}{)}
\end{sphinxVerbatim}

\end{sphinxuseclass}\end{sphinxVerbatimInput}

\end{sphinxuseclass}
\end{sphinxuseclass}
\sphinxstepscope


\section{Variables Code}
\label{\detokenize{variables_code:variables-code}}\label{\detokenize{variables_code::doc}}\begin{itemize}
\item {} 
\sphinxAtStartPar
Please solve the following questions using Python code.  

\end{itemize}


\subsection{Question\sphinxhyphen{}1}
\label{\detokenize{variables_code:question-1}}
\sphinxAtStartPar
Create a variable named \sphinxcode{\sphinxupquote{course}} and assign the value \sphinxcode{\sphinxupquote{math}} to it.

\sphinxAtStartPar
\sphinxstylestrong{Solution}


\subsection{Question\sphinxhyphen{}2}
\label{\detokenize{variables_code:question-2}}
\sphinxAtStartPar
Create a variable named \sphinxcode{\sphinxupquote{weight}} and assign the value \sphinxcode{\sphinxupquote{180.4}} to it.

\sphinxAtStartPar
\sphinxstylestrong{Solution}


\subsection{Question\sphinxhyphen{}3}
\label{\detokenize{variables_code:question-3}}
\sphinxAtStartPar
Create a variable named \sphinxcode{\sphinxupquote{number\_of\_apples}} and assign the value \sphinxcode{\sphinxupquote{45}} to it.

\sphinxAtStartPar
\sphinxstylestrong{Solution}


\subsection{Question\sphinxhyphen{}4}
\label{\detokenize{variables_code:question-4}}
\sphinxAtStartPar
Write the code to convert the integer \sphinxcode{\sphinxupquote{1}} to the string \sphinxcode{\sphinxupquote{'1'}}

\sphinxAtStartPar
\sphinxstylestrong{Solution}


\subsection{Question\sphinxhyphen{}5}
\label{\detokenize{variables_code:question-5}}\begin{itemize}
\item {} 
\sphinxAtStartPar
Create a variable named \sphinxcode{\sphinxupquote{x}} and assign the value 7 to it.

\item {} 
\sphinxAtStartPar
Create a variable named \sphinxcode{\sphinxupquote{y}} and assign the value 3 to it.

\item {} 
\sphinxAtStartPar
Display the sum, difference, and product of 7 and 3 using the variables \sphinxcode{\sphinxupquote{x}} and \sphinxcode{\sphinxupquote{y}}.

\end{itemize}

\sphinxAtStartPar
\sphinxstylestrong{Solution}|


\subsection{Question\sphinxhyphen{}6}
\label{\detokenize{variables_code:question-6}}\begin{itemize}
\item {} 
\sphinxAtStartPar
Create a variable named \sphinxcode{\sphinxupquote{x}} and assign the value 7 to it.

\item {} 
\sphinxAtStartPar
Create a variable named \sphinxcode{\sphinxupquote{y}} and assign the value 3 to it.

\item {} 
\sphinxAtStartPar
Create a variable named \sphinxcode{\sphinxupquote{z}}, assign the value of \sphinxcode{\sphinxupquote{x * y}} to it.

\item {} 
\sphinxAtStartPar
Display \sphinxcode{\sphinxupquote{z}}.

\end{itemize}

\sphinxAtStartPar
\sphinxstylestrong{Solution}|


\subsection{Question\sphinxhyphen{}7}
\label{\detokenize{variables_code:question-7}}\begin{itemize}
\item {} 
\sphinxAtStartPar
Create a variable named \sphinxcode{\sphinxupquote{width}} and assign the value \sphinxcode{\sphinxupquote{10}} to it.

\item {} 
\sphinxAtStartPar
Create a variable named \sphinxcode{\sphinxupquote{length}} and assign the value \sphinxcode{\sphinxupquote{50}} to it.
\begin{itemize}
\item {} 
\sphinxAtStartPar
These variables represent the sides of a rectangle.

\end{itemize}

\item {} 
\sphinxAtStartPar
Create a variable named \sphinxcode{\sphinxupquote{perimeter}} and assign the value of the rectangle’s perimeter to it.

\item {} 
\sphinxAtStartPar
Create a variable named \sphinxcode{\sphinxupquote{area}} and assign the value of the rectangle’s area to it.

\item {} 
\sphinxAtStartPar
Display the following statements by using the variables \sphinxcode{\sphinxupquote{width}}, \sphinxcode{\sphinxupquote{length}}, \sphinxcode{\sphinxupquote{perimeter}} and \sphinxcode{\sphinxupquote{area}}.
\begin{itemize}
\item {} 
\sphinxAtStartPar
The width of the garden is 10 ft.

\item {} 
\sphinxAtStartPar
The length of the garden is 50 ft.

\item {} 
\sphinxAtStartPar
The area of the rectangle is 500 square ft, indicating a large size.

\item {} 
\sphinxAtStartPar
The perimeter of the rectangle measures 120 ft, indicating that a considerable amount of fencing is required.

\end{itemize}

\item {} 
\sphinxAtStartPar
Please avoid using the numbers 10 and 50 in your code, and remember to include appropriate punctuation.

\end{itemize}

\sphinxAtStartPar
\sphinxstylestrong{Solution}


\subsection{Question\sphinxhyphen{}8}
\label{\detokenize{variables_code:question-8}}\begin{itemize}
\item {} 
\sphinxAtStartPar
Create a variable named \sphinxcode{\sphinxupquote{x}} and assign the value \sphinxcode{\sphinxupquote{3}} to it. (integer type)

\item {} 
\sphinxAtStartPar
Without directly using the number \sphinxcode{\sphinxupquote{3}} in your code (use variable \sphinxcode{\sphinxupquote{x}}) print \sphinxcode{\sphinxupquote{3333}} using three different code

\item {} 
\sphinxAtStartPar
Assign 5 to x and verify if you obtain \sphinxcode{\sphinxupquote{5555}} with the same code.

\end{itemize}

\sphinxAtStartPar
\sphinxstylestrong{Solution}

\sphinxstepscope


\section{Variables Exercises}
\label{\detokenize{variables_exercise:variables-exercises}}\label{\detokenize{variables_exercise::doc}}

\subsection{Question\sphinxhyphen{}1}
\label{\detokenize{variables_exercise:question-1}}
\sphinxAtStartPar
By using the variables \(x\) and \(y\) given below, print 353535 using 4 different pieces of code.

\begin{sphinxuseclass}{cell}\begin{sphinxVerbatimInput}

\begin{sphinxuseclass}{cell_input}
\begin{sphinxVerbatim}[commandchars=\\\{\}]
\PYG{n}{x}\PYG{p}{,} \PYG{n}{y} \PYG{o}{=} \PYG{l+m+mi}{3}\PYG{p}{,} \PYG{l+m+mi}{5}
\end{sphinxVerbatim}

\end{sphinxuseclass}\end{sphinxVerbatimInput}

\end{sphinxuseclass}

\subsection{Question\sphinxhyphen{}2}
\label{\detokenize{variables_exercise:question-2}}
\sphinxAtStartPar
Print the letter \sphinxstyleemphasis{U} using the character \sphinxcode{\sphinxupquote{+}}.
\begin{itemize}
\item {} 
\sphinxAtStartPar
The vertical left and right parts of \sphinxstyleemphasis{U} have five \sphinxcode{\sphinxupquote{+}} characters each.

\end{itemize}

\begin{sphinxVerbatim}[commandchars=\\\{\}]
\PYG{n}{horizontal} \PYG{o}{=} \PYG{l+m+mi}{3}
\end{sphinxVerbatim}
\begin{itemize}
\item {} 
\sphinxAtStartPar
The variable \sphinxstyleemphasis{horizontal} represents the number of \sphinxcode{\sphinxupquote{+}} characters with one space between them in the bottom horizontal part.

\end{itemize}

\sphinxAtStartPar
\sphinxstylestrong{Examples}
\sphinxincludegraphics{{horizontal}.png}


\subsection{Question\sphinxhyphen{}3}
\label{\detokenize{variables_exercise:question-3}}
\sphinxAtStartPar
Print the statement “I went to Italy in 2015.” using the country and year variables below.

\begin{sphinxVerbatim}[commandchars=\\\{\}]
\PYG{n}{country} \PYG{o}{=} \PYG{l+s+s1}{\PYGZsq{}}\PYG{l+s+s1}{Italy}\PYG{l+s+s1}{\PYGZsq{}}
\PYG{n}{year} \PYG{o}{=} \PYG{l+m+mi}{2015}
\end{sphinxVerbatim}

\sphinxAtStartPar
\sphinxstyleemphasis{Warning: There is no space before the period.}


\subsection{Question\sphinxhyphen{}4}
\label{\detokenize{variables_exercise:question-4}}
\sphinxAtStartPar
Create a variable named \sphinxcode{\sphinxupquote{r}} and assign the value of \(5\) to it.
\begin{itemize}
\item {} 
\sphinxAtStartPar
Print the area and perimeter of the circle with a radius equal to \sphinxcode{\sphinxupquote{r}}.

\item {} 
\sphinxAtStartPar
Round them to the nearest hundredth.

\item {} 
\sphinxAtStartPar
Hint:
\begin{itemize}
\item {} 
\sphinxAtStartPar
Perimeter = \(2\pi r\), Area = \(\pi r^2\)

\item {} 
\sphinxAtStartPar
Import \(\pi\) from a module.

\end{itemize}

\item {} 
\sphinxAtStartPar
Output:
\begin{itemize}
\item {} 
\sphinxAtStartPar
Perimeter: 31.42

\item {} 
\sphinxAtStartPar
Area     : 78.54

\end{itemize}

\end{itemize}


\subsection{Question\sphinxhyphen{}5}
\label{\detokenize{variables_exercise:question-5}}
\sphinxAtStartPar
Print the following using the character \sphinxcode{\sphinxupquote{*}}, spaces and the variable \sphinxcode{\sphinxupquote{n=5}}.

\sphinxAtStartPar
\sphinxincludegraphics{{v_stars}.png}

\sphinxstepscope


\chapter{Chp\sphinxhyphen{}2: Input and Output}
\label{\detokenize{inout:chp-2-input-and-output}}\label{\detokenize{inout::doc}}\begin{itemize}
\item {} 
\sphinxAtStartPar
Learning Objectives
\begin{itemize}
\item {} 
\sphinxAtStartPar
..

\item {} 
\sphinxAtStartPar
..

\end{itemize}

\end{itemize}


\section{input()}
\label{\detokenize{inout:input}}
\sphinxAtStartPar
The \sphinxstyleemphasis{input()} function is a built\sphinxhyphen{}in function used to obtain data or information from the user.
\begin{itemize}
\item {} 
\sphinxAtStartPar
It returns a string.

\item {} 
\sphinxAtStartPar
To receive a number from the user, you’ll need to convert the output of the \sphinxstyleemphasis{input()} function to an integer or float.

\item {} 
\sphinxAtStartPar
You can include a message as a string to provide directions to the user.

\item {} 
\sphinxAtStartPar
Upon running the code, your message will be displayed, and a box will prompt the user for input.

\item {} 
\sphinxAtStartPar
After entering the input, the user should press the enter key.

\end{itemize}

\sphinxAtStartPar
In the following code, the user is prompted with the message \sphinxcode{\sphinxupquote{Enter your birth year: }}.

\begin{sphinxVerbatim}[commandchars=\\\{\}]
\PYG{n+nb}{input}\PYG{p}{(}\PYG{l+s+s1}{\PYGZsq{}}\PYG{l+s+s1}{Enter your birth year: }\PYG{l+s+s1}{\PYGZsq{}}\PYG{p}{)}
\end{sphinxVerbatim}

\sphinxAtStartPar
Two important points in the above code:
\begin{enumerate}
\sphinxsetlistlabels{\arabic}{enumi}{enumii}{}{.}%
\item {} 
\sphinxAtStartPar
Even though the user enters a number, the \sphinxstyleemphasis{input()} function returns a string.
\begin{itemize}
\item {} 
\sphinxAtStartPar
To perform algebraic operations, you’ll need to convert it to an integer or float.

\end{itemize}

\item {} 
\sphinxAtStartPar
We must assign the value given by the user to a variable to store and use it.
\begin{itemize}
\item {} 
\sphinxAtStartPar
In the provided code, as no variable is used, there’s no way to access the given birth year in subsequent lines.

\end{itemize}

\end{enumerate}

\sphinxAtStartPar
In the following code, we will once again ask for the user’s birth year, but this time we will assign it to a variable.
\begin{itemize}
\item {} 
\sphinxAtStartPar
This way, we will be able to access the birth year through the variable \sphinxcode{\sphinxupquote{birth\_year}} in any cell.

\item {} 
\sphinxAtStartPar
Note that the type of the \sphinxcode{\sphinxupquote{birth\_year}} variable is \sphinxstyleemphasis{string} because the \sphinxstyleemphasis{input()} function returns any entered value as a string.

\end{itemize}

\begin{sphinxVerbatim}[commandchars=\\\{\}]
\PYG{n}{birth\PYGZus{}year} \PYG{o}{=} \PYG{n+nb}{input}\PYG{p}{(}\PYG{l+s+s1}{\PYGZsq{}}\PYG{l+s+s1}{Enter your birth year: }\PYG{l+s+s1}{\PYGZsq{}}\PYG{p}{)}

\PYG{n+nb}{print}\PYG{p}{(}\PYG{l+s+s1}{\PYGZsq{}}\PYG{l+s+s1}{birth\PYGZus{}year type:}\PYG{l+s+s1}{\PYGZsq{}}\PYG{p}{,} \PYG{n+nb}{type}\PYG{p}{(}\PYG{n}{birth\PYGZus{}year}\PYG{p}{)}\PYG{p}{)}
\end{sphinxVerbatim}

\sphinxAtStartPar
\sphinxstylestrong{Output}\\
Enter your birth year:  2000\\
birth\_year type: <class ‘str’>
\begin{itemize}
\item {} 
\sphinxAtStartPar
If you intend to perform algebraic operations, such as calculating age, using the \sphinxcode{\sphinxupquote{birth\_year}} variable, you’ll need to convert it to a numerical type.

\item {} 
\sphinxAtStartPar
Otherwise, an error message will be generated.

\end{itemize}

\begin{sphinxVerbatim}[commandchars=\\\{\}]
\PYG{n}{birth\PYGZus{}year} \PYG{o}{=} \PYG{n+nb}{input}\PYG{p}{(}\PYG{l+s+s1}{\PYGZsq{}}\PYG{l+s+s1}{Enter your birth year: }\PYG{l+s+s1}{\PYGZsq{}}\PYG{p}{)}

\PYG{n}{age} \PYG{o}{=} \PYG{l+m+mi}{2024} \PYG{o}{\PYGZhy{}} \PYG{n}{birth\PYGZus{}year}  \PYG{c+c1}{\PYGZsh{} ERROR: integer 2024 \PYGZhy{}  string birth\PYGZus{}year}
\end{sphinxVerbatim}
\begin{itemize}
\item {} 
\sphinxAtStartPar
To avoid this, convert \sphinxcode{\sphinxupquote{birth\_year}} to an integer. This can be done in a couple of different ways.

\item {} 
\sphinxAtStartPar
In the second line of the following code, the integer value of \sphinxstyleemphasis{birth\_year} is used in subtraction.

\end{itemize}

\begin{sphinxVerbatim}[commandchars=\\\{\}]
\PYG{n}{birth\PYGZus{}year} \PYG{o}{=} \PYG{n+nb}{input}\PYG{p}{(}\PYG{l+s+s1}{\PYGZsq{}}\PYG{l+s+s1}{Enter your birth year: }\PYG{l+s+s1}{\PYGZsq{}}\PYG{p}{)}

\PYG{n}{age} \PYG{o}{=} \PYG{l+m+mi}{2024} \PYG{o}{\PYGZhy{}} \PYG{n+nb}{int}\PYG{p}{(}\PYG{n}{birth\PYGZus{}year}\PYG{p}{)}    

\PYG{n+nb}{print}\PYG{p}{(}\PYG{l+s+s1}{\PYGZsq{}}\PYG{l+s+s1}{Your age is}\PYG{l+s+s1}{\PYGZsq{}}\PYG{p}{,} \PYG{n}{age}\PYG{p}{)}
\PYG{n+nb}{print}\PYG{p}{(}\PYG{l+s+s1}{\PYGZsq{}}\PYG{l+s+s1}{birth\PYGZus{}year type:}\PYG{l+s+s1}{\PYGZsq{}}\PYG{p}{,} \PYG{n+nb}{type}\PYG{p}{(}\PYG{n}{birth\PYGZus{}year}\PYG{p}{)}\PYG{p}{)}
\end{sphinxVerbatim}

\sphinxAtStartPar
\sphinxstylestrong{Output}\\
Enter your birth year:  2000\\
Your age is 24\\
birth\_year type: <class ‘str’>
\begin{itemize}
\item {} 
\sphinxAtStartPar
In the above code, the type of \sphinxcode{\sphinxupquote{birth\_year}} was not changed; it remains a string because we did not assign a new value to it.

\item {} 
\sphinxAtStartPar
This can be accomplished in a concise manner at the very beginning of the code.

\end{itemize}

\begin{sphinxVerbatim}[commandchars=\\\{\}]
\PYG{n}{birth\PYGZus{}year} \PYG{o}{=} \PYG{n+nb}{input}\PYG{p}{(}\PYG{l+s+s1}{\PYGZsq{}}\PYG{l+s+s1}{Enter your birth year: }\PYG{l+s+s1}{\PYGZsq{}}\PYG{p}{)}
\PYG{n}{birth\PYGZus{}year} \PYG{o}{=} \PYG{n+nb}{int}\PYG{p}{(}\PYG{n}{birth\PYGZus{}year}\PYG{p}{)}   \PYG{c+c1}{\PYGZsh{} assign a new value to the birth\PYGZus{}year variable}

\PYG{n}{age} \PYG{o}{=} \PYG{l+m+mi}{2024} \PYG{o}{\PYGZhy{}} \PYG{n}{birth\PYGZus{}year} 

\PYG{n+nb}{print}\PYG{p}{(}\PYG{l+s+s1}{\PYGZsq{}}\PYG{l+s+s1}{Your age is}\PYG{l+s+s1}{\PYGZsq{}}\PYG{p}{,} \PYG{n}{age}\PYG{p}{)}
\PYG{n+nb}{print}\PYG{p}{(}\PYG{l+s+s1}{\PYGZsq{}}\PYG{l+s+s1}{birth\PYGZus{}year type:}\PYG{l+s+s1}{\PYGZsq{}}\PYG{p}{,} \PYG{n+nb}{type}\PYG{p}{(}\PYG{n}{birth\PYGZus{}year}\PYG{p}{)}\PYG{p}{)}
\end{sphinxVerbatim}

\sphinxAtStartPar
\sphinxstylestrong{Output}\\
Enter your birth year: 2000\\
Your age is 24\\
birth\_year type: <class ‘int’>
\begin{itemize}
\item {} 
\sphinxAtStartPar
There is a shortcut for performing this conversion.

\item {} 
\sphinxAtStartPar
Upon receiving input from the user, we can immediately convert that value to an integer.

\item {} 
\sphinxAtStartPar
In the following code, the \sphinxstyleemphasis{input()} function returns a string, and the \sphinxstyleemphasis{int()} function converts this string to an integer.

\end{itemize}

\begin{sphinxVerbatim}[commandchars=\\\{\}]
\PYG{n}{birth\PYGZus{}year} \PYG{o}{=} \PYG{n+nb}{int}\PYG{p}{(}\PYG{n+nb}{input}\PYG{p}{(}\PYG{l+s+s1}{\PYGZsq{}}\PYG{l+s+s1}{Enter your birth year: }\PYG{l+s+s1}{\PYGZsq{}}\PYG{p}{)}\PYG{p}{)}  

\PYG{n}{age} \PYG{o}{=} \PYG{l+m+mi}{2024} \PYG{o}{\PYGZhy{}} \PYG{n}{birth\PYGZus{}year} 

\PYG{n+nb}{print}\PYG{p}{(}\PYG{l+s+s1}{\PYGZsq{}}\PYG{l+s+s1}{Your age is}\PYG{l+s+s1}{\PYGZsq{}}\PYG{p}{,} \PYG{n}{age}\PYG{p}{)}
\PYG{n+nb}{print}\PYG{p}{(}\PYG{l+s+s1}{\PYGZsq{}}\PYG{l+s+s1}{Birth year type:}\PYG{l+s+s1}{\PYGZsq{}}\PYG{p}{,} \PYG{n+nb}{type}\PYG{p}{(}\PYG{n}{birth\PYGZus{}year}\PYG{p}{)}\PYG{p}{)}
\end{sphinxVerbatim}

\sphinxAtStartPar
\sphinxstylestrong{Output}\\
Enter your birth year: 2000\\
Your age is 24\\
Birth year type: <class ‘int’>


\subsection{Receipt Example}
\label{\detokenize{inout:receipt-example}}\begin{itemize}
\item {} 
\sphinxAtStartPar
You can use the \sphinxcode{\sphinxupquote{input()}} function multiple times.

\item {} 
\sphinxAtStartPar
In the following example, the user enters the quantity of hamburgers and sodas, and the final receipt, including tax and tip, is printed.
\begin{itemize}
\item {} 
\sphinxAtStartPar
The price of a hamburger is \$5.

\item {} 
\sphinxAtStartPar
The price of a coke is \$2.

\item {} 
\sphinxAtStartPar
The tip is 15\%.

\item {} 
\sphinxAtStartPar
The tax is 10\%.

\end{itemize}

\item {} 
\sphinxAtStartPar
Possible improvements:
\begin{itemize}
\item {} 
\sphinxAtStartPar
You can also include the time and date by using the datetime module.

\item {} 
\sphinxAtStartPar
Consider rounding the tax, tip, and total amounts.

\end{itemize}

\end{itemize}

\begin{sphinxVerbatim}[commandchars=\\\{\}]
\PYG{n}{hamburger} \PYG{o}{=} \PYG{n+nb}{int}\PYG{p}{(}\PYG{n+nb}{input}\PYG{p}{(}\PYG{l+s+s1}{\PYGZsq{}}\PYG{l+s+s1}{Number of hamburgers:}\PYG{l+s+s1}{\PYGZsq{}}\PYG{p}{)}\PYG{p}{)}  
\PYG{n}{soda} \PYG{o}{=} \PYG{n+nb}{int}\PYG{p}{(}\PYG{n+nb}{input}\PYG{p}{(}\PYG{l+s+s1}{\PYGZsq{}}\PYG{l+s+s1}{Number of sodas:}\PYG{l+s+s1}{\PYGZsq{}}\PYG{p}{)}\PYG{p}{)}            

\PYG{n}{subtotal} \PYG{o}{=} \PYG{n}{hamburger}\PYG{o}{*}\PYG{l+m+mi}{5}\PYG{o}{+}\PYG{n}{soda}\PYG{o}{*}\PYG{l+m+mi}{2}   
\PYG{n}{tip} \PYG{o}{=} \PYG{n}{subtotal}\PYG{o}{*}\PYG{l+m+mf}{0.15}
\PYG{n}{tax} \PYG{o}{=} \PYG{n}{subtotal}\PYG{o}{*}\PYG{l+m+mf}{0.10}
\PYG{n}{total} \PYG{o}{=} \PYG{n}{subtotal}\PYG{o}{+}\PYG{n}{tip}\PYG{o}{+}\PYG{n}{tax}


\PYG{n+nb}{print}\PYG{p}{(}\PYG{l+s+s1}{\PYGZsq{}}\PYG{l+s+s1}{Hamburger:}\PYG{l+s+s1}{\PYGZsq{}}\PYG{p}{,}\PYG{n}{hamburger}\PYG{p}{,}\PYG{l+s+s1}{\PYGZsq{}}\PYG{l+s+s1}{x 5= }\PYG{l+s+s1}{\PYGZsq{}}\PYG{p}{,}\PYG{n}{hamburger}\PYG{o}{*}\PYG{l+m+mi}{5}\PYG{p}{)}
\PYG{n+nb}{print}\PYG{p}{(}\PYG{l+s+s1}{\PYGZsq{}}\PYG{l+s+s1}{Soda     :}\PYG{l+s+s1}{\PYGZsq{}}\PYG{p}{,}\PYG{n}{soda}\PYG{p}{,}\PYG{l+s+s1}{\PYGZsq{}}\PYG{l+s+s1}{x 2= }\PYG{l+s+s1}{\PYGZsq{}}\PYG{p}{,}\PYG{n}{soda}\PYG{o}{*}\PYG{l+m+mi}{2}\PYG{p}{)}
\PYG{n+nb}{print}\PYG{p}{(}\PYG{l+s+s1}{\PYGZsq{}}\PYG{l+s+s1}{Tip      :         }\PYG{l+s+s1}{\PYGZsq{}}\PYG{p}{,}\PYG{n}{tip}\PYG{p}{)}
\PYG{n+nb}{print}\PYG{p}{(}\PYG{l+s+s1}{\PYGZsq{}}\PYG{l+s+s1}{Tax      :         }\PYG{l+s+s1}{\PYGZsq{}}\PYG{p}{,}\PYG{n}{tax}\PYG{p}{)}
\PYG{n+nb}{print}\PYG{p}{(}\PYG{l+s+s1}{\PYGZsq{}}\PYG{l+s+s1}{Total    :         }\PYG{l+s+s1}{\PYGZsq{}}\PYG{p}{,}\PYG{n}{total}\PYG{p}{)}
\end{sphinxVerbatim}

\sphinxAtStartPar
\sphinxstylestrong{Output}

\sphinxAtStartPar
Number of hamburgers:10\\
Number of sodas     :20\\
Hamburger: 10 x 5= * 50\\
Soda     : 20 x 2=  40\\
Tip      :          13.5\\
Tax      :          9.0\\
Total    :          112.5


\section{Whitespaces}
\label{\detokenize{inout:whitespaces}}
\sphinxAtStartPar
The following whitespace characters are frequently used in print statements for spacing.
\begin{itemize}
\item {} 
\sphinxAtStartPar
\sphinxcode{\sphinxupquote{\textbackslash{}n}}: new line
\begin{itemize}
\item {} 
\sphinxAtStartPar
Moves to the next line.

\end{itemize}

\item {} 
\sphinxAtStartPar
\sphinxcode{\sphinxupquote{\textbackslash{}t}}: tab
\begin{itemize}
\item {} 
\sphinxAtStartPar
Inserts a tabulation

\item {} 
\sphinxAtStartPar
Inserts spaces up to the next tab stop, which occurs every 8th character.

\end{itemize}

\item {} 
\sphinxAtStartPar
\sphinxcode{\sphinxupquote{\textbackslash{}b}}: backspace
\begin{itemize}
\item {} 
\sphinxAtStartPar
Deletes the character to the left.

\end{itemize}

\item {} 
\sphinxAtStartPar
\sphinxcode{\sphinxupquote{\textbackslash{}r}}: carriage return
\begin{itemize}
\item {} 
\sphinxAtStartPar
Moves to the beginning of the line.

\item {} 
\sphinxAtStartPar
In Jupyter notebook, it does not delete any characters.

\item {} 
\sphinxAtStartPar
In Google Colab, it deletes all characters.

\end{itemize}

\end{itemize}

\begin{sphinxuseclass}{cell}\begin{sphinxVerbatimInput}

\begin{sphinxuseclass}{cell_input}
\begin{sphinxVerbatim}[commandchars=\\\{\}]
\PYG{n+nb}{print}\PYG{p}{(}\PYG{l+s+s1}{\PYGZsq{}}\PYG{l+s+s1}{A}\PYG{l+s+se}{\PYGZbs{}n}\PYG{l+s+s1}{B}\PYG{l+s+s1}{\PYGZsq{}}\PYG{p}{)} \PYG{c+c1}{\PYGZsh{} after A it moves to the next line}
\end{sphinxVerbatim}

\end{sphinxuseclass}\end{sphinxVerbatimInput}
\begin{sphinxVerbatimOutput}

\begin{sphinxuseclass}{cell_output}
\begin{sphinxVerbatim}[commandchars=\\\{\}]
A
B
\end{sphinxVerbatim}

\end{sphinxuseclass}\end{sphinxVerbatimOutput}

\end{sphinxuseclass}
\begin{sphinxuseclass}{cell}\begin{sphinxVerbatimInput}

\begin{sphinxuseclass}{cell_input}
\begin{sphinxVerbatim}[commandchars=\\\{\}]
\PYG{n+nb}{print}\PYG{p}{(}\PYG{l+s+s1}{\PYGZsq{}}\PYG{l+s+s1}{A       B}\PYG{l+s+s1}{\PYGZsq{}}\PYG{p}{)}     \PYG{c+c1}{\PYGZsh{} 7 spaces}
\PYG{n+nb}{print}\PYG{p}{(}\PYG{l+s+s1}{\PYGZsq{}}\PYG{l+s+s1}{A}\PYG{l+s+s1}{\PYGZsq{}}\PYG{o}{+}\PYG{l+s+s1}{\PYGZsq{}}\PYG{l+s+s1}{ }\PYG{l+s+s1}{\PYGZsq{}}\PYG{o}{*}\PYG{l+m+mi}{7}\PYG{o}{+}\PYG{l+s+s1}{\PYGZsq{}}\PYG{l+s+s1}{B}\PYG{l+s+s1}{\PYGZsq{}}\PYG{p}{)}   \PYG{c+c1}{\PYGZsh{} repetition of the string \PYGZsq{} \PYGZsq{} (one space) seven times}
\PYG{n+nb}{print}\PYG{p}{(}\PYG{l+s+s1}{\PYGZsq{}}\PYG{l+s+s1}{A}\PYG{l+s+se}{\PYGZbs{}t}\PYG{l+s+s1}{B}\PYG{l+s+s1}{\PYGZsq{}}\PYG{p}{)}          \PYG{c+c1}{\PYGZsh{} tab}
\end{sphinxVerbatim}

\end{sphinxuseclass}\end{sphinxVerbatimInput}
\begin{sphinxVerbatimOutput}

\begin{sphinxuseclass}{cell_output}
\begin{sphinxVerbatim}[commandchars=\\\{\}]
A       B
A       B
A	B
\end{sphinxVerbatim}

\end{sphinxuseclass}\end{sphinxVerbatimOutput}

\end{sphinxuseclass}
\begin{sphinxuseclass}{cell}\begin{sphinxVerbatimInput}

\begin{sphinxuseclass}{cell_input}
\begin{sphinxVerbatim}[commandchars=\\\{\}]
\PYG{n+nb}{print}\PYG{p}{(}\PYG{l+s+s1}{\PYGZsq{}}\PYG{l+s+s1}{AA      B}\PYG{l+s+s1}{\PYGZsq{}}\PYG{p}{)}      \PYG{c+c1}{\PYGZsh{} 6 spaces}
\PYG{n+nb}{print}\PYG{p}{(}\PYG{l+s+s1}{\PYGZsq{}}\PYG{l+s+s1}{AA}\PYG{l+s+s1}{\PYGZsq{}}\PYG{o}{+}\PYG{l+s+s1}{\PYGZsq{}}\PYG{l+s+s1}{ }\PYG{l+s+s1}{\PYGZsq{}}\PYG{o}{*}\PYG{l+m+mi}{6}\PYG{o}{+}\PYG{l+s+s1}{\PYGZsq{}}\PYG{l+s+s1}{B}\PYG{l+s+s1}{\PYGZsq{}}\PYG{p}{)}   \PYG{c+c1}{\PYGZsh{} repetition of the string \PYGZsq{} \PYGZsq{} (one space) six times}
\PYG{n+nb}{print}\PYG{p}{(}\PYG{l+s+s1}{\PYGZsq{}}\PYG{l+s+s1}{AA}\PYG{l+s+se}{\PYGZbs{}t}\PYG{l+s+s1}{B}\PYG{l+s+s1}{\PYGZsq{}}\PYG{p}{)}          \PYG{c+c1}{\PYGZsh{} tab}
\end{sphinxVerbatim}

\end{sphinxuseclass}\end{sphinxVerbatimInput}
\begin{sphinxVerbatimOutput}

\begin{sphinxuseclass}{cell_output}
\begin{sphinxVerbatim}[commandchars=\\\{\}]
AA      B
AA      B
AA	B
\end{sphinxVerbatim}

\end{sphinxuseclass}\end{sphinxVerbatimOutput}

\end{sphinxuseclass}
\begin{sphinxuseclass}{cell}\begin{sphinxVerbatimInput}

\begin{sphinxuseclass}{cell_input}
\begin{sphinxVerbatim}[commandchars=\\\{\}]
\PYG{n+nb}{print}\PYG{p}{(}\PYG{l+s+s1}{\PYGZsq{}}\PYG{l+s+s1}{AAA     B}\PYG{l+s+s1}{\PYGZsq{}}\PYG{p}{)}       \PYG{c+c1}{\PYGZsh{} 5 spaces}
\PYG{n+nb}{print}\PYG{p}{(}\PYG{l+s+s1}{\PYGZsq{}}\PYG{l+s+s1}{AAA}\PYG{l+s+s1}{\PYGZsq{}}\PYG{o}{+}\PYG{l+s+s1}{\PYGZsq{}}\PYG{l+s+s1}{ }\PYG{l+s+s1}{\PYGZsq{}}\PYG{o}{*}\PYG{l+m+mi}{5}\PYG{o}{+}\PYG{l+s+s1}{\PYGZsq{}}\PYG{l+s+s1}{B}\PYG{l+s+s1}{\PYGZsq{}}\PYG{p}{)}   \PYG{c+c1}{\PYGZsh{} repetition of the string \PYGZsq{} \PYGZsq{} (one space) five times}
\PYG{n+nb}{print}\PYG{p}{(}\PYG{l+s+s1}{\PYGZsq{}}\PYG{l+s+s1}{AAA}\PYG{l+s+se}{\PYGZbs{}t}\PYG{l+s+s1}{B}\PYG{l+s+s1}{\PYGZsq{}}\PYG{p}{)}          \PYG{c+c1}{\PYGZsh{} tab}
\end{sphinxVerbatim}

\end{sphinxuseclass}\end{sphinxVerbatimInput}
\begin{sphinxVerbatimOutput}

\begin{sphinxuseclass}{cell_output}
\begin{sphinxVerbatim}[commandchars=\\\{\}]
AAA     B
AAA     B
AAA	B
\end{sphinxVerbatim}

\end{sphinxuseclass}\end{sphinxVerbatimOutput}

\end{sphinxuseclass}
\begin{sphinxuseclass}{cell}\begin{sphinxVerbatimInput}

\begin{sphinxuseclass}{cell_input}
\begin{sphinxVerbatim}[commandchars=\\\{\}]
\PYG{n+nb}{print}\PYG{p}{(}\PYG{l+s+s1}{\PYGZsq{}}\PYG{l+s+s1}{ABC}\PYG{l+s+se}{\PYGZbs{}b}\PYG{l+s+s1}{D}\PYG{l+s+s1}{\PYGZsq{}}\PYG{p}{)} \PYG{c+c1}{\PYGZsh{} C is deleted by \PYGZsq{}\PYGZbs{}b\PYGZsq{}}
\end{sphinxVerbatim}

\end{sphinxuseclass}\end{sphinxVerbatimInput}
\begin{sphinxVerbatimOutput}

\begin{sphinxuseclass}{cell_output}
\begin{sphinxVerbatim}[commandchars=\\\{\}]
ABCD
\end{sphinxVerbatim}

\end{sphinxuseclass}\end{sphinxVerbatimOutput}

\end{sphinxuseclass}
\begin{sphinxuseclass}{cell}\begin{sphinxVerbatimInput}

\begin{sphinxuseclass}{cell_input}
\begin{sphinxVerbatim}[commandchars=\\\{\}]
\PYG{n+nb}{print}\PYG{p}{(}\PYG{l+s+s1}{\PYGZsq{}}\PYG{l+s+s1}{ABC}\PYG{l+s+se}{\PYGZbs{}b}\PYG{l+s+se}{\PYGZbs{}b}\PYG{l+s+s1}{D}\PYG{l+s+s1}{\PYGZsq{}}\PYG{p}{)} \PYG{c+c1}{\PYGZsh{} C and B are deleted by two \PYGZsq{}\PYGZbs{}b\PYGZsq{} s}
\end{sphinxVerbatim}

\end{sphinxuseclass}\end{sphinxVerbatimInput}
\begin{sphinxVerbatimOutput}

\begin{sphinxuseclass}{cell_output}
\begin{sphinxVerbatim}[commandchars=\\\{\}]
ABCD
\end{sphinxVerbatim}

\end{sphinxuseclass}\end{sphinxVerbatimOutput}

\end{sphinxuseclass}
\sphinxAtStartPar
For Jupter Notebook:
\begin{itemize}
\item {} 
\sphinxAtStartPar
In the following code, after the character ‘C’, the carriage return \sphinxcode{\sphinxupquote{\textbackslash{}r}} moves the cursor to the beginning of the line, and ‘D’ overwrites ‘A’.

\end{itemize}

\begin{sphinxVerbatim}[commandchars=\\\{\}]
\PYG{n+nb}{print}\PYG{p}{(}\PYG{l+s+s1}{\PYGZsq{}}\PYG{l+s+s1}{ABC}\PYG{l+s+se}{\PYGZbs{}r}\PYG{l+s+s1}{D}\PYG{l+s+s1}{\PYGZsq{}}\PYG{p}{)} \PYG{c+c1}{\PYGZsh{} carriage return }
\end{sphinxVerbatim}
\begin{itemize}
\item {} 
\sphinxAtStartPar
Output: DBC

\end{itemize}
\begin{itemize}
\item {} 
\sphinxAtStartPar
\sphinxstylestrong{Warning:} In Google Colab, the output of the provided code is D.

\end{itemize}


\section{print()}
\label{\detokenize{inout:print}}
\sphinxAtStartPar
The print() function is a built\sphinxhyphen{}in function that displays output on the screen.
\begin{itemize}
\item {} 
\sphinxAtStartPar
It has two significant parameters: \sphinxcode{\sphinxupquote{sep}} and \sphinxcode{\sphinxupquote{end}}.

\end{itemize}


\subsection{sep parameter}
\label{\detokenize{inout:sep-parameter}}\begin{itemize}
\item {} 
\sphinxAtStartPar
It is the separator parameter.

\item {} 
\sphinxAtStartPar
It determines what to insert between the comma\sphinxhyphen{}separated values in a print function.

\item {} 
\sphinxAtStartPar
The default value is a single space: \sphinxcode{\sphinxupquote{' '}}.

\item {} 
\sphinxAtStartPar
sep values are strings

\end{itemize}

\begin{sphinxuseclass}{cell}\begin{sphinxVerbatimInput}

\begin{sphinxuseclass}{cell_input}
\begin{sphinxVerbatim}[commandchars=\\\{\}]
\PYG{n}{name} \PYG{o}{=} \PYG{l+s+s1}{\PYGZsq{}}\PYG{l+s+s1}{Tom}\PYG{l+s+s1}{\PYGZsq{}}
\PYG{n}{age} \PYG{o}{=} \PYG{l+m+mi}{25}
\PYG{n+nb}{print}\PYG{p}{(}\PYG{l+s+s1}{\PYGZsq{}}\PYG{l+s+s1}{A}\PYG{l+s+s1}{\PYGZsq{}}\PYG{p}{,} \PYG{n}{age}\PYG{p}{,} \PYG{l+s+s1}{\PYGZsq{}}\PYG{l+s+s1}{B}\PYG{l+s+s1}{\PYGZsq{}}\PYG{p}{,} \PYG{n}{name}\PYG{p}{)}            \PYG{c+c1}{\PYGZsh{} by deafult there is one space between each value}
\end{sphinxVerbatim}

\end{sphinxuseclass}\end{sphinxVerbatimInput}
\begin{sphinxVerbatimOutput}

\begin{sphinxuseclass}{cell_output}
\begin{sphinxVerbatim}[commandchars=\\\{\}]
A 25 B Tom
\end{sphinxVerbatim}

\end{sphinxuseclass}\end{sphinxVerbatimOutput}

\end{sphinxuseclass}
\begin{sphinxuseclass}{cell}\begin{sphinxVerbatimInput}

\begin{sphinxuseclass}{cell_input}
\begin{sphinxVerbatim}[commandchars=\\\{\}]
\PYG{n}{name} \PYG{o}{=} \PYG{l+s+s1}{\PYGZsq{}}\PYG{l+s+s1}{Tom}\PYG{l+s+s1}{\PYGZsq{}}
\PYG{n+nb}{print}\PYG{p}{(}\PYG{l+s+s1}{\PYGZsq{}}\PYG{l+s+s1}{A}\PYG{l+s+s1}{\PYGZsq{}}\PYG{p}{,} \PYG{n}{age}\PYG{p}{,} \PYG{l+s+s1}{\PYGZsq{}}\PYG{l+s+s1}{B}\PYG{l+s+s1}{\PYGZsq{}}\PYG{p}{,} \PYG{n}{name}\PYG{p}{,} \PYG{n}{sep}\PYG{o}{=}\PYG{l+s+s1}{\PYGZsq{}}\PYG{l+s+s1}{\PYGZhy{}}\PYG{l+s+s1}{\PYGZsq{}}\PYG{p}{)}   \PYG{c+c1}{\PYGZsh{}  values are separated by one \PYGZsq{}\PYGZhy{}\PYGZsq{} (dash)}
\end{sphinxVerbatim}

\end{sphinxuseclass}\end{sphinxVerbatimInput}
\begin{sphinxVerbatimOutput}

\begin{sphinxuseclass}{cell_output}
\begin{sphinxVerbatim}[commandchars=\\\{\}]
A\PYGZhy{}25\PYGZhy{}B\PYGZhy{}Tom
\end{sphinxVerbatim}

\end{sphinxuseclass}\end{sphinxVerbatimOutput}

\end{sphinxuseclass}
\begin{sphinxuseclass}{cell}\begin{sphinxVerbatimInput}

\begin{sphinxuseclass}{cell_input}
\begin{sphinxVerbatim}[commandchars=\\\{\}]
\PYG{n}{name} \PYG{o}{=} \PYG{l+s+s1}{\PYGZsq{}}\PYG{l+s+s1}{Tom}\PYG{l+s+s1}{\PYGZsq{}}
\PYG{n+nb}{print}\PYG{p}{(}\PYG{l+s+s1}{\PYGZsq{}}\PYG{l+s+s1}{A}\PYG{l+s+s1}{\PYGZsq{}}\PYG{p}{,} \PYG{n}{age}\PYG{p}{,} \PYG{l+s+s1}{\PYGZsq{}}\PYG{l+s+s1}{B}\PYG{l+s+s1}{\PYGZsq{}}\PYG{p}{,} \PYG{n}{name}\PYG{p}{,} \PYG{n}{sep}\PYG{o}{=}\PYG{l+s+s1}{\PYGZsq{}}\PYG{l+s+s1}{***}\PYG{l+s+s1}{\PYGZsq{}}\PYG{p}{)} \PYG{c+c1}{\PYGZsh{}  values are separated by three \PYGZsq{}*\PYGZsq{}s (asterisk)}
\end{sphinxVerbatim}

\end{sphinxuseclass}\end{sphinxVerbatimInput}
\begin{sphinxVerbatimOutput}

\begin{sphinxuseclass}{cell_output}
\begin{sphinxVerbatim}[commandchars=\\\{\}]
A***25***B***Tom
\end{sphinxVerbatim}

\end{sphinxuseclass}\end{sphinxVerbatimOutput}

\end{sphinxuseclass}

\subsection{end parameter}
\label{\detokenize{inout:end-parameter}}\begin{itemize}
\item {} 
\sphinxAtStartPar
It determines what to print at the end of the output.

\item {} 
\sphinxAtStartPar
The default value of end parameter is the new line: \sphinxcode{\sphinxupquote{'\textbackslash{}n'}}.

\item {} 
\sphinxAtStartPar
end values are strings

\end{itemize}

\sphinxAtStartPar
\sphinxstylestrong{Example}

\begin{sphinxuseclass}{cell}\begin{sphinxVerbatimInput}

\begin{sphinxuseclass}{cell_input}
\begin{sphinxVerbatim}[commandchars=\\\{\}]
\PYG{n+nb}{print}\PYG{p}{(}\PYG{l+s+s1}{\PYGZsq{}}\PYG{l+s+s1}{A}\PYG{l+s+s1}{\PYGZsq{}}\PYG{p}{)}  \PYG{c+c1}{\PYGZsh{} end=\PYGZsq{}\PYGZbs{}n\PYGZsq{} by default, after printing A it moes to next line}
\PYG{n+nb}{print}\PYG{p}{(}\PYG{l+s+s1}{\PYGZsq{}}\PYG{l+s+s1}{B}\PYG{l+s+s1}{\PYGZsq{}}\PYG{p}{)}  \PYG{c+c1}{\PYGZsh{} end=\PYGZsq{}\PYGZbs{}n\PYGZsq{} by default, after printing B it moes to next line}
\PYG{n+nb}{print}\PYG{p}{(}\PYG{l+s+s1}{\PYGZsq{}}\PYG{l+s+s1}{C}\PYG{l+s+s1}{\PYGZsq{}}\PYG{p}{)}  \PYG{c+c1}{\PYGZsh{} end=\PYGZsq{}\PYGZbs{}n\PYGZsq{} by default, after printing C it moes to next line}
\PYG{n+nb}{print}\PYG{p}{(}\PYG{l+s+s1}{\PYGZsq{}}\PYG{l+s+s1}{D}\PYG{l+s+s1}{\PYGZsq{}}\PYG{p}{)}  \PYG{c+c1}{\PYGZsh{} end=\PYGZsq{}\PYGZbs{}n\PYGZsq{} by default, after printing C it moes to next line}
\end{sphinxVerbatim}

\end{sphinxuseclass}\end{sphinxVerbatimInput}
\begin{sphinxVerbatimOutput}

\begin{sphinxuseclass}{cell_output}
\begin{sphinxVerbatim}[commandchars=\\\{\}]
A
B
C
D
\end{sphinxVerbatim}

\end{sphinxuseclass}\end{sphinxVerbatimOutput}

\end{sphinxuseclass}
\sphinxAtStartPar
\sphinxstylestrong{Example}

\begin{sphinxuseclass}{cell}\begin{sphinxVerbatimInput}

\begin{sphinxuseclass}{cell_input}
\begin{sphinxVerbatim}[commandchars=\\\{\}]
\PYG{n+nb}{print}\PYG{p}{(}\PYG{l+s+s1}{\PYGZsq{}}\PYG{l+s+s1}{A}\PYG{l+s+s1}{\PYGZsq{}}\PYG{p}{,} \PYG{n}{end}\PYG{o}{=}\PYG{l+s+s1}{\PYGZsq{}}\PYG{l+s+s1}{\PYGZhy{}\PYGZhy{}}\PYG{l+s+s1}{\PYGZsq{}}\PYG{p}{)}  \PYG{c+c1}{\PYGZsh{} end=\PYGZsq{}\PYGZhy{}\PYGZhy{}\PYGZsq{}           , after printing A it prints \PYGZsq{}\PYGZhy{}\PYGZhy{}\PYGZsq{}}
\PYG{n+nb}{print}\PYG{p}{(}\PYG{l+s+s1}{\PYGZsq{}}\PYG{l+s+s1}{B}\PYG{l+s+s1}{\PYGZsq{}}\PYG{p}{)}            \PYG{c+c1}{\PYGZsh{} end=\PYGZsq{}\PYGZbs{}n\PYGZsq{} by default, after printing B it moes to next line}
\PYG{n+nb}{print}\PYG{p}{(}\PYG{l+s+s1}{\PYGZsq{}}\PYG{l+s+s1}{C}\PYG{l+s+s1}{\PYGZsq{}}\PYG{p}{)}            \PYG{c+c1}{\PYGZsh{} end=\PYGZsq{}\PYGZbs{}n\PYGZsq{} by default, after printing C it moes to next line}
\PYG{n+nb}{print}\PYG{p}{(}\PYG{l+s+s1}{\PYGZsq{}}\PYG{l+s+s1}{D}\PYG{l+s+s1}{\PYGZsq{}}\PYG{p}{)}            \PYG{c+c1}{\PYGZsh{} end=\PYGZsq{}\PYGZbs{}n\PYGZsq{} by default, after printing C it moes to next line}
\end{sphinxVerbatim}

\end{sphinxuseclass}\end{sphinxVerbatimInput}
\begin{sphinxVerbatimOutput}

\begin{sphinxuseclass}{cell_output}
\begin{sphinxVerbatim}[commandchars=\\\{\}]
A\PYGZhy{}\PYGZhy{}B
C
D
\end{sphinxVerbatim}

\end{sphinxuseclass}\end{sphinxVerbatimOutput}

\end{sphinxuseclass}
\sphinxAtStartPar
\sphinxstylestrong{Example}

\begin{sphinxuseclass}{cell}\begin{sphinxVerbatimInput}

\begin{sphinxuseclass}{cell_input}
\begin{sphinxVerbatim}[commandchars=\\\{\}]
\PYG{n+nb}{print}\PYG{p}{(}\PYG{l+s+s1}{\PYGZsq{}}\PYG{l+s+s1}{A}\PYG{l+s+s1}{\PYGZsq{}}\PYG{p}{,} \PYG{n}{end}\PYG{o}{=}\PYG{l+s+s1}{\PYGZsq{}}\PYG{l+s+s1}{\PYGZhy{}\PYGZhy{}}\PYG{l+s+s1}{\PYGZsq{}}\PYG{p}{)}  \PYG{c+c1}{\PYGZsh{} end=\PYGZsq{}\PYGZhy{}\PYGZhy{}\PYGZsq{}           , after printing A it prints \PYGZhy{}\PYGZhy{}}
\PYG{n+nb}{print}\PYG{p}{(}\PYG{l+s+s1}{\PYGZsq{}}\PYG{l+s+s1}{B}\PYG{l+s+s1}{\PYGZsq{}}\PYG{p}{,} \PYG{n}{end}\PYG{o}{=}\PYG{l+s+s1}{\PYGZsq{}}\PYG{l+s+s1}{+}\PYG{l+s+s1}{\PYGZsq{}}\PYG{p}{)}   \PYG{c+c1}{\PYGZsh{} end=\PYGZsq{}+\PYGZsq{}            , after printing B it prints +}
\PYG{n+nb}{print}\PYG{p}{(}\PYG{l+s+s1}{\PYGZsq{}}\PYG{l+s+s1}{C}\PYG{l+s+s1}{\PYGZsq{}}\PYG{p}{)}            \PYG{c+c1}{\PYGZsh{} end=\PYGZsq{}\PYGZbs{}n\PYGZsq{} by default, after printing C it moes to next line}
\PYG{n+nb}{print}\PYG{p}{(}\PYG{l+s+s1}{\PYGZsq{}}\PYG{l+s+s1}{D}\PYG{l+s+s1}{\PYGZsq{}}\PYG{p}{)}            \PYG{c+c1}{\PYGZsh{} end=\PYGZsq{}\PYGZbs{}n\PYGZsq{} by default, after printing C it moes to next line}
\end{sphinxVerbatim}

\end{sphinxuseclass}\end{sphinxVerbatimInput}
\begin{sphinxVerbatimOutput}

\begin{sphinxuseclass}{cell_output}
\begin{sphinxVerbatim}[commandchars=\\\{\}]
A\PYGZhy{}\PYGZhy{}B+C
D
\end{sphinxVerbatim}

\end{sphinxuseclass}\end{sphinxVerbatimOutput}

\end{sphinxuseclass}
\sphinxAtStartPar
\sphinxstylestrong{Example}

\begin{sphinxuseclass}{cell}\begin{sphinxVerbatimInput}

\begin{sphinxuseclass}{cell_input}
\begin{sphinxVerbatim}[commandchars=\\\{\}]
\PYG{n+nb}{print}\PYG{p}{(}\PYG{l+s+s1}{\PYGZsq{}}\PYG{l+s+s1}{A}\PYG{l+s+s1}{\PYGZsq{}}\PYG{p}{)}            \PYG{c+c1}{\PYGZsh{} end=\PYGZsq{}\PYGZbs{}n\PYGZsq{} by default, after printing A it moes to next line}
\PYG{n+nb}{print}\PYG{p}{(}\PYG{l+s+s1}{\PYGZsq{}}\PYG{l+s+s1}{B}\PYG{l+s+s1}{\PYGZsq{}}\PYG{p}{,} \PYG{n}{end}\PYG{o}{=}\PYG{l+s+s1}{\PYGZsq{}}\PYG{l+s+s1}{+}\PYG{l+s+s1}{\PYGZsq{}}\PYG{p}{)}   \PYG{c+c1}{\PYGZsh{} end=\PYGZsq{}+\PYGZsq{}            , after printing B it prints +}
\PYG{n+nb}{print}\PYG{p}{(}\PYG{l+s+s1}{\PYGZsq{}}\PYG{l+s+s1}{C}\PYG{l+s+s1}{\PYGZsq{}}\PYG{p}{,} \PYG{n}{end}\PYG{o}{=}\PYG{l+s+s1}{\PYGZsq{}}\PYG{l+s+s1}{?}\PYG{l+s+s1}{\PYGZsq{}}\PYG{p}{)}   \PYG{c+c1}{\PYGZsh{} end=\PYGZsq{}+\PYGZsq{}            , after printing C it prints ?}
\PYG{n+nb}{print}\PYG{p}{(}\PYG{l+s+s1}{\PYGZsq{}}\PYG{l+s+s1}{D}\PYG{l+s+s1}{\PYGZsq{}}\PYG{p}{)}            \PYG{c+c1}{\PYGZsh{} end=\PYGZsq{}\PYGZbs{}n\PYGZsq{} by default, after printing C it moes to next line}
\end{sphinxVerbatim}

\end{sphinxuseclass}\end{sphinxVerbatimInput}
\begin{sphinxVerbatimOutput}

\begin{sphinxuseclass}{cell_output}
\begin{sphinxVerbatim}[commandchars=\\\{\}]
A
B+C?D
\end{sphinxVerbatim}

\end{sphinxuseclass}\end{sphinxVerbatimOutput}

\end{sphinxuseclass}

\section{Examples}
\label{\detokenize{inout:examples}}

\subsection{Moving O (right)}
\label{\detokenize{inout:moving-o-right}}
\sphinxAtStartPar
Use the \sphinxstyleemphasis{print()} function along with the letter ‘O’, the backspace ‘\textbackslash{}b’, and the \sphinxstyleemphasis{sleep()} function from the \sphinxstyleemphasis{time} module to create a right\sphinxhyphen{}moving ‘O’.
\begin{itemize}
\item {} 
\sphinxAtStartPar
Each \sphinxcode{\sphinxupquote{print('\textbackslash{}b O', end='')}} statement performs four actions:
\begin{enumerate}
\sphinxsetlistlabels{\arabic}{enumi}{enumii}{}{.}%
\item {} 
\sphinxAtStartPar
‘\textbackslash{}b’ deletes the ‘O’ that was printed before, moving the cursor one position to the left.

\item {} 
\sphinxAtStartPar
Prints a space.

\item {} 
\sphinxAtStartPar
Prints ‘O’.

\item {} 
\sphinxAtStartPar
Since the end parameter is set to an empty string, it does not move to the next line.

\end{enumerate}

\end{itemize}

\begin{sphinxVerbatim}[commandchars=\\\{\}]
\PYG{k+kn}{import} \PYG{n+nn}{time}
\PYG{n+nb}{print}\PYG{p}{(}\PYG{l+s+s1}{\PYGZsq{}}\PYG{l+s+s1}{O}\PYG{l+s+s1}{\PYGZsq{}}\PYG{p}{,} \PYG{n}{end}\PYG{o}{=}\PYG{l+s+s1}{\PYGZsq{}}\PYG{l+s+s1}{\PYGZsq{}}\PYG{p}{)}           
\PYG{n}{time}\PYG{o}{.}\PYG{n}{sleep}\PYG{p}{(}\PYG{l+m+mi}{1}\PYG{p}{)}
\PYG{n+nb}{print}\PYG{p}{(}\PYG{l+s+s1}{\PYGZsq{}}\PYG{l+s+se}{\PYGZbs{}b}\PYG{l+s+s1}{ O}\PYG{l+s+s1}{\PYGZsq{}}\PYG{p}{,} \PYG{n}{end}\PYG{o}{=}\PYG{l+s+s1}{\PYGZsq{}}\PYG{l+s+s1}{\PYGZsq{}}\PYG{p}{)}     \PYG{c+c1}{\PYGZsh{} Delete the \PYGZsq{}O\PYGZsq{}, print a space, then print \PYGZsq{}O\PYGZsq{}.}
\PYG{n}{time}\PYG{o}{.}\PYG{n}{sleep}\PYG{p}{(}\PYG{l+m+mi}{1}\PYG{p}{)}
\PYG{n+nb}{print}\PYG{p}{(}\PYG{l+s+s1}{\PYGZsq{}}\PYG{l+s+se}{\PYGZbs{}b}\PYG{l+s+s1}{ O}\PYG{l+s+s1}{\PYGZsq{}}\PYG{p}{,} \PYG{n}{end}\PYG{o}{=}\PYG{l+s+s1}{\PYGZsq{}}\PYG{l+s+s1}{\PYGZsq{}}\PYG{p}{)}
\PYG{n}{time}\PYG{o}{.}\PYG{n}{sleep}\PYG{p}{(}\PYG{l+m+mi}{1}\PYG{p}{)}
\PYG{n+nb}{print}\PYG{p}{(}\PYG{l+s+s1}{\PYGZsq{}}\PYG{l+s+se}{\PYGZbs{}b}\PYG{l+s+s1}{ O}\PYG{l+s+s1}{\PYGZsq{}}\PYG{p}{,} \PYG{n}{end}\PYG{o}{=}\PYG{l+s+s1}{\PYGZsq{}}\PYG{l+s+s1}{\PYGZsq{}}\PYG{p}{)}
\PYG{n}{time}\PYG{o}{.}\PYG{n}{sleep}\PYG{p}{(}\PYG{l+m+mi}{1}\PYG{p}{)}
\PYG{n+nb}{print}\PYG{p}{(}\PYG{l+s+s1}{\PYGZsq{}}\PYG{l+s+se}{\PYGZbs{}b}\PYG{l+s+s1}{ O}\PYG{l+s+s1}{\PYGZsq{}}\PYG{p}{,} \PYG{n}{end}\PYG{o}{=}\PYG{l+s+s1}{\PYGZsq{}}\PYG{l+s+s1}{\PYGZsq{}}\PYG{p}{)}
\PYG{n}{time}\PYG{o}{.}\PYG{n}{sleep}\PYG{p}{(}\PYG{l+m+mi}{1}\PYG{p}{)}
\PYG{n+nb}{print}\PYG{p}{(}\PYG{l+s+s1}{\PYGZsq{}}\PYG{l+s+se}{\PYGZbs{}b}\PYG{l+s+s1}{ O}\PYG{l+s+s1}{\PYGZsq{}}\PYG{p}{,} \PYG{n}{end}\PYG{o}{=}\PYG{l+s+s1}{\PYGZsq{}}\PYG{l+s+s1}{\PYGZsq{}}\PYG{p}{)}
\PYG{n}{time}\PYG{o}{.}\PYG{n}{sleep}\PYG{p}{(}\PYG{l+m+mi}{1}\PYG{p}{)}
\PYG{n+nb}{print}\PYG{p}{(}\PYG{l+s+s1}{\PYGZsq{}}\PYG{l+s+se}{\PYGZbs{}b}\PYG{l+s+s1}{ O}\PYG{l+s+s1}{\PYGZsq{}}\PYG{p}{,} \PYG{n}{end}\PYG{o}{=}\PYG{l+s+s1}{\PYGZsq{}}\PYG{l+s+s1}{\PYGZsq{}}\PYG{p}{)}
\end{sphinxVerbatim}


\subsection{Moving O (left)}
\label{\detokenize{inout:moving-o-left}}
\sphinxAtStartPar
Use the \sphinxstyleemphasis{print()} function along with the letter ‘O’, the backspace ‘\textbackslash{}b’, and the \sphinxstyleemphasis{sleep()} function from the \sphinxstyleemphasis{time} module to create a left\sphinxhyphen{}moving ‘O’.
\begin{itemize}
\item {} 
\sphinxAtStartPar
Each \sphinxcode{\sphinxupquote{print('\textbackslash{}b'*2 +'O', end='')}} statement performs four actions:

\end{itemize}
\begin{enumerate}
\sphinxsetlistlabels{\arabic}{enumi}{enumii}{}{.}%
\item {} 
\sphinxAtStartPar
The first ‘\textbackslash{}b’ deletes the ‘O’ that was printed before, moving the cursor one position to the left.

\item {} 
\sphinxAtStartPar
The scond ‘\textbackslash{}b’ moves the cursor one position to the left.

\item {} 
\sphinxAtStartPar
Prints ‘O’.

\item {} 
\sphinxAtStartPar
Since the end parameter is set to an empty string, the cursor does not move to the next line after printing the ‘O’.

\end{enumerate}

\begin{sphinxVerbatim}[commandchars=\\\{\}]
\PYG{k+kn}{import} \PYG{n+nn}{time}
\PYG{n+nb}{print}\PYG{p}{(}\PYG{l+s+s1}{\PYGZsq{}}\PYG{l+s+s1}{ }\PYG{l+s+s1}{\PYGZsq{}}\PYG{o}{*}\PYG{l+m+mi}{5} \PYG{o}{+} \PYG{l+s+s1}{\PYGZsq{}}\PYG{l+s+s1}{O}\PYG{l+s+s1}{\PYGZsq{}}\PYG{p}{,} \PYG{n}{end}\PYG{o}{=}\PYG{l+s+s1}{\PYGZsq{}}\PYG{l+s+s1}{\PYGZsq{}}\PYG{p}{)}     \PYG{c+c1}{\PYGZsh{} Print \PYGZsq{}O\PYGZsq{} after 5 spaces.    }
\PYG{n}{time}\PYG{o}{.}\PYG{n}{sleep}\PYG{p}{(}\PYG{l+m+mi}{1}\PYG{p}{)}
\PYG{n+nb}{print}\PYG{p}{(}\PYG{l+s+s1}{\PYGZsq{}}\PYG{l+s+se}{\PYGZbs{}b}\PYG{l+s+s1}{\PYGZsq{}}\PYG{o}{*}\PYG{l+m+mi}{2} \PYG{o}{+}\PYG{l+s+s1}{\PYGZsq{}}\PYG{l+s+s1}{O}\PYG{l+s+s1}{\PYGZsq{}}\PYG{p}{,} \PYG{n}{end}\PYG{o}{=}\PYG{l+s+s1}{\PYGZsq{}}\PYG{l+s+s1}{\PYGZsq{}}\PYG{p}{)}     \PYG{c+c1}{\PYGZsh{} Delete the \PYGZsq{}O\PYGZsq{}, print a space, then print \PYGZsq{}O\PYGZsq{}.}
\PYG{n}{time}\PYG{o}{.}\PYG{n}{sleep}\PYG{p}{(}\PYG{l+m+mi}{1}\PYG{p}{)}
\PYG{n+nb}{print}\PYG{p}{(}\PYG{l+s+s1}{\PYGZsq{}}\PYG{l+s+se}{\PYGZbs{}b}\PYG{l+s+s1}{\PYGZsq{}}\PYG{o}{*}\PYG{l+m+mi}{2} \PYG{o}{+}\PYG{l+s+s1}{\PYGZsq{}}\PYG{l+s+s1}{O}\PYG{l+s+s1}{\PYGZsq{}}\PYG{p}{,} \PYG{n}{end}\PYG{o}{=}\PYG{l+s+s1}{\PYGZsq{}}\PYG{l+s+s1}{\PYGZsq{}}\PYG{p}{)} 
\PYG{n}{time}\PYG{o}{.}\PYG{n}{sleep}\PYG{p}{(}\PYG{l+m+mi}{1}\PYG{p}{)}
\PYG{n+nb}{print}\PYG{p}{(}\PYG{l+s+s1}{\PYGZsq{}}\PYG{l+s+se}{\PYGZbs{}b}\PYG{l+s+s1}{\PYGZsq{}}\PYG{o}{*}\PYG{l+m+mi}{2} \PYG{o}{+}\PYG{l+s+s1}{\PYGZsq{}}\PYG{l+s+s1}{O}\PYG{l+s+s1}{\PYGZsq{}}\PYG{p}{,} \PYG{n}{end}\PYG{o}{=}\PYG{l+s+s1}{\PYGZsq{}}\PYG{l+s+s1}{\PYGZsq{}}\PYG{p}{)} 
\PYG{n}{time}\PYG{o}{.}\PYG{n}{sleep}\PYG{p}{(}\PYG{l+m+mi}{1}\PYG{p}{)}
\PYG{n+nb}{print}\PYG{p}{(}\PYG{l+s+s1}{\PYGZsq{}}\PYG{l+s+se}{\PYGZbs{}b}\PYG{l+s+s1}{\PYGZsq{}}\PYG{o}{*}\PYG{l+m+mi}{2} \PYG{o}{+}\PYG{l+s+s1}{\PYGZsq{}}\PYG{l+s+s1}{O}\PYG{l+s+s1}{\PYGZsq{}}\PYG{p}{,} \PYG{n}{end}\PYG{o}{=}\PYG{l+s+s1}{\PYGZsq{}}\PYG{l+s+s1}{\PYGZsq{}}\PYG{p}{)} 
\PYG{n}{time}\PYG{o}{.}\PYG{n}{sleep}\PYG{p}{(}\PYG{l+m+mi}{1}\PYG{p}{)}
\PYG{n+nb}{print}\PYG{p}{(}\PYG{l+s+s1}{\PYGZsq{}}\PYG{l+s+se}{\PYGZbs{}b}\PYG{l+s+s1}{\PYGZsq{}}\PYG{o}{*}\PYG{l+m+mi}{2} \PYG{o}{+}\PYG{l+s+s1}{\PYGZsq{}}\PYG{l+s+s1}{O}\PYG{l+s+s1}{\PYGZsq{}}\PYG{p}{,} \PYG{n}{end}\PYG{o}{=}\PYG{l+s+s1}{\PYGZsq{}}\PYG{l+s+s1}{\PYGZsq{}}\PYG{p}{)} 
\PYG{n}{time}\PYG{o}{.}\PYG{n}{sleep}\PYG{p}{(}\PYG{l+m+mi}{1}\PYG{p}{)}
\PYG{n+nb}{print}\PYG{p}{(}\PYG{l+s+s1}{\PYGZsq{}}\PYG{l+s+se}{\PYGZbs{}b}\PYG{l+s+s1}{\PYGZsq{}}\PYG{o}{*}\PYG{l+m+mi}{2} \PYG{o}{+}\PYG{l+s+s1}{\PYGZsq{}}\PYG{l+s+s1}{O}\PYG{l+s+s1}{\PYGZsq{}}\PYG{p}{,} \PYG{n}{end}\PYG{o}{=}\PYG{l+s+s1}{\PYGZsq{}}\PYG{l+s+s1}{\PYGZsq{}}\PYG{p}{)} 
\end{sphinxVerbatim}

\sphinxstepscope


\section{Input and Output Debugging}
\label{\detokenize{inout_debug:input-and-output-debugging}}\label{\detokenize{inout_debug::doc}}\begin{itemize}
\item {} 
\sphinxAtStartPar
Each of the following short code contains one or more bugs.     

\item {} 
\sphinxAtStartPar
Please identify and correct these bugs.

\item {} 
\sphinxAtStartPar
Provide an explanation for your answer.

\end{itemize}


\subsection{Question\sphinxhyphen{}1}
\label{\detokenize{inout_debug:question-1}}
\begin{sphinxVerbatim}[commandchars=\\\{\}]
\PYG{n+nb}{print} \PYG{l+s+s1}{\PYGZsq{}}\PYG{l+s+s1}{Hello}\PYG{l+s+s1}{\PYGZsq{}}
\end{sphinxVerbatim}

\begin{sphinxadmonition}{note}{Solution}

\sphinxAtStartPar
Parentheses are missing.
\end{sphinxadmonition}


\subsection{Question\sphinxhyphen{}2}
\label{\detokenize{inout_debug:question-2}}
\begin{sphinxVerbatim}[commandchars=\\\{\}]
\PYG{n+nb}{print}\PYG{p}{(}\PYG{l+s+s1}{\PYGZsq{}}\PYG{l+s+s1}{Hello)}
\end{sphinxVerbatim}

\begin{sphinxadmonition}{note}{Solution}

\sphinxAtStartPar
Single quote on the right of Hello is missing.
\end{sphinxadmonition}


\subsection{Question\sphinxhyphen{}3}
\label{\detokenize{inout_debug:question-3}}
\begin{sphinxVerbatim}[commandchars=\\\{\}]
\PYG{n+nb}{print}\PYG{p}{(}\PYG{l+s+s1}{\PYGZsq{}}\PYG{l+s+s1}{Hello}\PYG{l+s+s1}{\PYGZsq{}}\PYG{p}{)}\PYG{p}{)}
\end{sphinxVerbatim}

\begin{sphinxadmonition}{note}{Solution}

\sphinxAtStartPar
There is an extra paranthesis at the end.
\end{sphinxadmonition}


\subsection{Question\sphinxhyphen{}4}
\label{\detokenize{inout_debug:question-4}}
\begin{sphinxVerbatim}[commandchars=\\\{\}]
\PYG{n+nb}{print} \PYG{o}{=} \PYG{l+m+mi}{3}
\PYG{n+nb}{print}\PYG{p}{(}\PYG{l+s+s1}{\PYGZsq{}}\PYG{l+s+s1}{England}\PYG{l+s+s1}{\PYGZsq{}}\PYG{p}{)}
\end{sphinxVerbatim}

\begin{sphinxadmonition}{note}{Solution}
\begin{itemize}
\item {} 
\sphinxAtStartPar
In the first line print becomes a variable and it’s value is 3.

\item {} 
\sphinxAtStartPar
In the second line print is tried to used as a function but it is not a function anymore.

\end{itemize}
\end{sphinxadmonition}


\subsection{Question\sphinxhyphen{}5}
\label{\detokenize{inout_debug:question-5}}
\begin{sphinxVerbatim}[commandchars=\\\{\}]
\PYG{n}{x} \PYG{o}{=} \PYG{n+nb}{input}\PYG{p}{(}\PYG{n}{Enter} \PYG{n}{your} \PYG{n}{birthyear}\PYG{p}{:}\PYG{p}{)}
\PYG{n+nb}{print}\PYG{p}{(}\PYG{n}{x}\PYG{p}{)}
\end{sphinxVerbatim}

\begin{sphinxadmonition}{note}{Solution}

\sphinxAtStartPar
Single quotes of the string inside the input function are missing.
\end{sphinxadmonition}


\subsection{Question\sphinxhyphen{}6}
\label{\detokenize{inout_debug:question-6}}
\begin{sphinxVerbatim}[commandchars=\\\{\}]
\PYG{n}{x} \PYG{o}{=} \PYG{n+nb}{input}\PYG{p}{(}\PYG{l+s+s1}{\PYGZsq{}}\PYG{l+s+s1}{Enter your birthyear: }\PYG{l+s+s1}{\PYGZsq{}}\PYG{p}{)}
\PYG{n+nb}{print}\PYG{p}{(}\PYG{l+s+sa}{f}\PYG{l+s+s1}{\PYGZsq{}}\PYG{l+s+s1}{Your age is }\PYG{l+s+si}{\PYGZob{}}\PYG{l+m+mi}{2022}\PYG{o}{\PYGZhy{}}\PYG{n}{x}\PYG{l+s+si}{\PYGZcb{}}\PYG{l+s+s1}{\PYGZsq{}}\PYG{p}{)}
\end{sphinxVerbatim}

\begin{sphinxadmonition}{note}{Solution}

\sphinxAtStartPar
Type of x is string so subtraction cannot be done.
\end{sphinxadmonition}


\subsection{Question\sphinxhyphen{}7}
\label{\detokenize{inout_debug:question-7}}
\begin{sphinxVerbatim}[commandchars=\\\{\}]
\PYG{n+nb}{print}\PYG{p}{(}\PYG{l+s+s1}{\PYGZsq{}}\PYG{l+s+s1}{A}\PYG{l+s+s1}{\PYGZsq{}}\PYG{p}{,}\PYG{l+m+mi}{2}\PYG{p}{,}\PYG{l+m+mi}{3}\PYG{p}{,}\PYG{l+s+s1}{\PYGZsq{}}\PYG{l+s+s1}{B}\PYG{l+s+s1}{\PYGZsq{}}\PYG{p}{,} \PYG{n}{separator}\PYG{o}{=}\PYG{l+s+s1}{\PYGZsq{}}\PYG{l+s+s1}{++}\PYG{l+s+s1}{\PYGZsq{}}\PYG{p}{)}
\end{sphinxVerbatim}

\begin{sphinxadmonition}{note}{Solution}

\sphinxAtStartPar
The name of the parameter is sep (not separator).
\end{sphinxadmonition}


\subsection{Question\sphinxhyphen{}8}
\label{\detokenize{inout_debug:question-8}}
\begin{sphinxVerbatim}[commandchars=\\\{\}]
\PYG{n}{first} \PYG{n}{name} \PYG{o}{=} \PYG{n+nb}{input}\PYG{p}{(}\PYG{l+s+s1}{\PYGZsq{}}\PYG{l+s+s1}{Enter your first name: }\PYG{l+s+s1}{\PYGZsq{}}\PYG{p}{)}
\PYG{n+nb}{print}\PYG{p}{(}\PYG{n}{first} \PYG{n}{name}\PYG{p}{)}
\end{sphinxVerbatim}

\begin{sphinxadmonition}{note}{Solution}

\sphinxAtStartPar
Variable names (first name) cannot have a space.
\end{sphinxadmonition}

\sphinxstepscope


\section{Input and Output  Output}
\label{\detokenize{inout_output:input-and-output-output}}\label{\detokenize{inout_output::doc}}\begin{itemize}
\item {} 
\sphinxAtStartPar
Find the output of the following code.

\item {} 
\sphinxAtStartPar
Please don’t run the code before giving your answer.     

\end{itemize}


\subsection{Question\sphinxhyphen{}1}
\label{\detokenize{inout_output:question-1}}
\begin{sphinxuseclass}{cell}
\begin{sphinxuseclass}{tag_hide-output}\begin{sphinxVerbatimInput}

\begin{sphinxuseclass}{cell_input}
\begin{sphinxVerbatim}[commandchars=\\\{\}]
\PYG{n}{name} \PYG{o}{=} \PYG{l+s+s1}{\PYGZsq{}}\PYG{l+s+s1}{Liz}\PYG{l+s+s1}{\PYGZsq{}}
\PYG{n+nb}{print}\PYG{p}{(}\PYG{l+s+s1}{\PYGZsq{}}\PYG{l+s+s1}{name}\PYG{l+s+s1}{\PYGZsq{}}\PYG{p}{)}
\end{sphinxVerbatim}

\end{sphinxuseclass}\end{sphinxVerbatimInput}

\end{sphinxuseclass}
\end{sphinxuseclass}

\subsection{Question\sphinxhyphen{}2}
\label{\detokenize{inout_output:question-2}}
\begin{sphinxuseclass}{cell}
\begin{sphinxuseclass}{tag_hide-output}\begin{sphinxVerbatimInput}

\begin{sphinxuseclass}{cell_input}
\begin{sphinxVerbatim}[commandchars=\\\{\}]
\PYG{n}{number} \PYG{o}{=} \PYG{l+m+mi}{49}
\PYG{n+nb}{print}\PYG{p}{(}\PYG{n}{number}\PYG{o}{\PYGZhy{}}\PYG{l+m+mi}{10}\PYG{p}{)}
\end{sphinxVerbatim}

\end{sphinxuseclass}\end{sphinxVerbatimInput}

\end{sphinxuseclass}
\end{sphinxuseclass}

\subsection{Question\sphinxhyphen{}3}
\label{\detokenize{inout_output:question-3}}
\begin{sphinxuseclass}{cell}
\begin{sphinxuseclass}{tag_hide-output}\begin{sphinxVerbatimInput}

\begin{sphinxuseclass}{cell_input}
\begin{sphinxVerbatim}[commandchars=\\\{\}]
\PYG{n+nb}{print}\PYG{p}{(}\PYG{l+s+s1}{\PYGZsq{}}\PYG{l+s+s1}{Michael}\PYG{l+s+s1}{\PYGZsq{}}\PYG{o}{+}\PYG{l+s+s1}{\PYGZsq{}}\PYG{l+s+s1}{Jordan}\PYG{l+s+s1}{\PYGZsq{}}\PYG{p}{)}
\end{sphinxVerbatim}

\end{sphinxuseclass}\end{sphinxVerbatimInput}

\end{sphinxuseclass}
\end{sphinxuseclass}

\subsection{Question\sphinxhyphen{}4}
\label{\detokenize{inout_output:question-4}}
\begin{sphinxuseclass}{cell}
\begin{sphinxuseclass}{tag_hide-output}\begin{sphinxVerbatimInput}

\begin{sphinxuseclass}{cell_input}
\begin{sphinxVerbatim}[commandchars=\\\{\}]
\PYG{n+nb}{print}\PYG{p}{(}\PYG{l+s+s1}{\PYGZsq{}}\PYG{l+s+s1}{Michael}\PYG{l+s+s1}{\PYGZsq{}}\PYG{o}{+}\PYG{l+s+s1}{\PYGZsq{}}\PYG{l+s+s1}{\PYGZsq{}}\PYG{o}{+}\PYG{l+s+s1}{\PYGZsq{}}\PYG{l+s+s1}{Jordan}\PYG{l+s+s1}{\PYGZsq{}}\PYG{p}{)}
\end{sphinxVerbatim}

\end{sphinxuseclass}\end{sphinxVerbatimInput}

\end{sphinxuseclass}
\end{sphinxuseclass}

\subsection{Question\sphinxhyphen{}5}
\label{\detokenize{inout_output:question-5}}
\begin{sphinxuseclass}{cell}
\begin{sphinxuseclass}{tag_hide-output}\begin{sphinxVerbatimInput}

\begin{sphinxuseclass}{cell_input}
\begin{sphinxVerbatim}[commandchars=\\\{\}]
\PYG{n+nb}{print}\PYG{p}{(}\PYG{l+s+s1}{\PYGZsq{}}\PYG{l+s+s1}{Michael}\PYG{l+s+s1}{\PYGZsq{}}\PYG{o}{+}\PYG{l+s+s1}{\PYGZsq{}}\PYG{l+s+s1}{ }\PYG{l+s+s1}{\PYGZsq{}}\PYG{o}{+}\PYG{l+s+s1}{\PYGZsq{}}\PYG{l+s+s1}{Jordan}\PYG{l+s+s1}{\PYGZsq{}}\PYG{p}{)}
\end{sphinxVerbatim}

\end{sphinxuseclass}\end{sphinxVerbatimInput}

\end{sphinxuseclass}
\end{sphinxuseclass}

\subsection{Question\sphinxhyphen{}6}
\label{\detokenize{inout_output:question-6}}
\begin{sphinxuseclass}{cell}
\begin{sphinxuseclass}{tag_hide-output}\begin{sphinxVerbatimInput}

\begin{sphinxuseclass}{cell_input}
\begin{sphinxVerbatim}[commandchars=\\\{\}]
\PYG{n+nb}{print}\PYG{p}{(}\PYG{l+s+s1}{\PYGZsq{}}\PYG{l+s+s1}{Michael}\PYG{l+s+s1}{\PYGZsq{}}\PYG{p}{,} \PYG{l+s+s1}{\PYGZsq{}}\PYG{l+s+s1}{Jordan}\PYG{l+s+s1}{\PYGZsq{}}\PYG{p}{)}
\end{sphinxVerbatim}

\end{sphinxuseclass}\end{sphinxVerbatimInput}

\end{sphinxuseclass}
\end{sphinxuseclass}

\subsection{Question\sphinxhyphen{}7}
\label{\detokenize{inout_output:question-7}}
\begin{sphinxuseclass}{cell}
\begin{sphinxuseclass}{tag_hide-output}\begin{sphinxVerbatimInput}

\begin{sphinxuseclass}{cell_input}
\begin{sphinxVerbatim}[commandchars=\\\{\}]
\PYG{n}{age} \PYG{o}{=} \PYG{l+m+mi}{25}
\PYG{n+nb}{print}\PYG{p}{(}\PYG{l+s+s1}{\PYGZsq{}}\PYG{l+s+s1}{I am}\PYG{l+s+s1}{\PYGZsq{}} \PYG{o}{+}\PYG{l+s+s1}{\PYGZsq{}}\PYG{l+s+s1}{age}\PYG{l+s+s1}{\PYGZsq{}}\PYG{o}{+}\PYG{l+s+s1}{\PYGZsq{}}\PYG{l+s+s1}{years old.}\PYG{l+s+s1}{\PYGZsq{}}\PYG{p}{)}
\end{sphinxVerbatim}

\end{sphinxuseclass}\end{sphinxVerbatimInput}

\end{sphinxuseclass}
\end{sphinxuseclass}

\subsection{Question\sphinxhyphen{}8}
\label{\detokenize{inout_output:question-8}}
\begin{sphinxuseclass}{cell}
\begin{sphinxuseclass}{tag_hide-output}\begin{sphinxVerbatimInput}

\begin{sphinxuseclass}{cell_input}
\begin{sphinxVerbatim}[commandchars=\\\{\}]
\PYG{n}{age} \PYG{o}{=} \PYG{l+m+mi}{25}
\PYG{n+nb}{print}\PYG{p}{(}\PYG{l+s+s1}{\PYGZsq{}}\PYG{l+s+s1}{I am}\PYG{l+s+s1}{\PYGZsq{}} \PYG{o}{+}\PYG{n+nb}{str}\PYG{p}{(}\PYG{n}{age}\PYG{p}{)}\PYG{o}{+}\PYG{l+s+s1}{\PYGZsq{}}\PYG{l+s+s1}{years old.}\PYG{l+s+s1}{\PYGZsq{}}\PYG{p}{)}
\end{sphinxVerbatim}

\end{sphinxuseclass}\end{sphinxVerbatimInput}

\end{sphinxuseclass}
\end{sphinxuseclass}

\subsection{Question\sphinxhyphen{}9}
\label{\detokenize{inout_output:question-9}}
\begin{sphinxuseclass}{cell}
\begin{sphinxuseclass}{tag_hide-output}\begin{sphinxVerbatimInput}

\begin{sphinxuseclass}{cell_input}
\begin{sphinxVerbatim}[commandchars=\\\{\}]
\PYG{n}{x} \PYG{o}{=} \PYG{l+m+mi}{5}
\PYG{n}{y} \PYG{o}{=} \PYG{l+s+s1}{\PYGZsq{}}\PYG{l+s+s1}{apple}\PYG{l+s+s1}{\PYGZsq{}}
\PYG{n+nb}{print}\PYG{p}{(}\PYG{l+s+s1}{\PYGZsq{}}\PYG{l+s+s1}{I have}\PYG{l+s+s1}{\PYGZsq{}}\PYG{p}{,} \PYG{l+m+mi}{20}\PYG{o}{\PYGZhy{}}\PYG{n}{x}\PYG{p}{,} \PYG{n}{y}\PYG{p}{,}\PYG{l+s+s1}{\PYGZsq{}}\PYG{l+s+s1}{s.}\PYG{l+s+s1}{\PYGZsq{}}\PYG{p}{)}
\end{sphinxVerbatim}

\end{sphinxuseclass}\end{sphinxVerbatimInput}

\end{sphinxuseclass}
\end{sphinxuseclass}

\subsection{Question\sphinxhyphen{}10}
\label{\detokenize{inout_output:question-10}}
\begin{sphinxVerbatim}[commandchars=\\\{\}]
\PYG{n}{number} \PYG{o}{=} \PYG{n+nb}{input}\PYG{p}{(}\PYG{l+s+s1}{\PYGZsq{}}\PYG{l+s+s1}{What is your favorite number: }\PYG{l+s+s1}{\PYGZsq{}}\PYG{p}{)}
\PYG{n+nb}{print}\PYG{p}{(}\PYG{l+s+s1}{\PYGZsq{}}\PYG{l+s+s1}{number}\PYG{l+s+s1}{\PYGZsq{}}\PYG{p}{)}
\end{sphinxVerbatim}

\begin{sphinxadmonition}{note}{Solution}

\sphinxAtStartPar
number
\end{sphinxadmonition}


\subsection{Question\sphinxhyphen{}11}
\label{\detokenize{inout_output:question-11}}
\begin{sphinxuseclass}{cell}
\begin{sphinxuseclass}{tag_hide-output}\begin{sphinxVerbatimInput}

\begin{sphinxuseclass}{cell_input}
\begin{sphinxVerbatim}[commandchars=\\\{\}]
\PYG{n+nb}{print}\PYG{p}{(}\PYG{l+s+s1}{\PYGZsq{}}\PYG{l+s+s1}{Hel}\PYG{l+s+se}{\PYGZbs{}n}\PYG{l+s+s1}{lo}\PYG{l+s+s1}{\PYGZsq{}}\PYG{p}{)}
\end{sphinxVerbatim}

\end{sphinxuseclass}\end{sphinxVerbatimInput}

\end{sphinxuseclass}
\end{sphinxuseclass}

\subsection{Question\sphinxhyphen{}12}
\label{\detokenize{inout_output:question-12}}
\begin{sphinxuseclass}{cell}
\begin{sphinxuseclass}{tag_hide-output}\begin{sphinxVerbatimInput}

\begin{sphinxuseclass}{cell_input}
\begin{sphinxVerbatim}[commandchars=\\\{\}]
\PYG{n+nb}{print}\PYG{p}{(}\PYG{l+s+s1}{\PYGZsq{}}\PYG{l+s+s1}{Hel}\PYG{l+s+se}{\PYGZbs{}t}\PYG{l+s+s1}{lo}\PYG{l+s+s1}{\PYGZsq{}}\PYG{p}{)}
\end{sphinxVerbatim}

\end{sphinxuseclass}\end{sphinxVerbatimInput}

\end{sphinxuseclass}
\end{sphinxuseclass}

\subsection{Question\sphinxhyphen{}13}
\label{\detokenize{inout_output:question-13}}
\begin{sphinxuseclass}{cell}
\begin{sphinxuseclass}{tag_hide-output}\begin{sphinxVerbatimInput}

\begin{sphinxuseclass}{cell_input}
\begin{sphinxVerbatim}[commandchars=\\\{\}]
\PYG{n+nb}{print}\PYG{p}{(}\PYG{l+s+s1}{\PYGZsq{}}\PYG{l+s+s1}{Hel}\PYG{l+s+se}{\PYGZbs{}b}\PYG{l+s+s1}{lo}\PYG{l+s+s1}{\PYGZsq{}}\PYG{p}{)}
\end{sphinxVerbatim}

\end{sphinxuseclass}\end{sphinxVerbatimInput}

\end{sphinxuseclass}
\end{sphinxuseclass}

\subsection{Question\sphinxhyphen{}14}
\label{\detokenize{inout_output:question-14}}
\begin{sphinxuseclass}{cell}
\begin{sphinxuseclass}{tag_hide-output}\begin{sphinxVerbatimInput}

\begin{sphinxuseclass}{cell_input}
\begin{sphinxVerbatim}[commandchars=\\\{\}]
\PYG{n+nb}{print}\PYG{p}{(}\PYG{l+s+s1}{\PYGZsq{}}\PYG{l+s+s1}{TEX}\PYG{l+s+se}{\PYGZbs{}b}\PYG{l+s+se}{\PYGZbs{}b}\PYG{l+s+s1}{AS}\PYG{l+s+s1}{\PYGZsq{}}\PYG{p}{)}
\end{sphinxVerbatim}

\end{sphinxuseclass}\end{sphinxVerbatimInput}

\end{sphinxuseclass}
\end{sphinxuseclass}

\subsection{Question\sphinxhyphen{}15}
\label{\detokenize{inout_output:question-15}}
\begin{sphinxuseclass}{cell}
\begin{sphinxuseclass}{tag_hide-output}\begin{sphinxVerbatimInput}

\begin{sphinxuseclass}{cell_input}
\begin{sphinxVerbatim}[commandchars=\\\{\}]
\PYG{n+nb}{print}\PYG{p}{(}\PYG{l+s+s1}{\PYGZsq{}}\PYG{l+s+s1}{CA}\PYG{l+s+se}{\PYGZbs{}n}\PYG{l+s+s1}{LIFO}\PYG{l+s+se}{\PYGZbs{}t}\PYG{l+s+s1}{RN}\PYG{l+s+se}{\PYGZbs{}b}\PYG{l+s+s1}{IA}\PYG{l+s+s1}{\PYGZsq{}}\PYG{p}{)}
\end{sphinxVerbatim}

\end{sphinxuseclass}\end{sphinxVerbatimInput}

\end{sphinxuseclass}
\end{sphinxuseclass}

\subsection{Question\sphinxhyphen{}16}
\label{\detokenize{inout_output:question-16}}
\begin{sphinxuseclass}{cell}
\begin{sphinxuseclass}{tag_hide-output}\begin{sphinxVerbatimInput}

\begin{sphinxuseclass}{cell_input}
\begin{sphinxVerbatim}[commandchars=\\\{\}]
\PYG{n}{x} \PYG{o}{=} \PYG{l+s+s1}{\PYGZsq{}}\PYG{l+s+s1}{5}\PYG{l+s+s1}{\PYGZsq{}}
\PYG{n}{y} \PYG{o}{=} \PYG{l+s+s1}{\PYGZsq{}}\PYG{l+s+s1}{7}\PYG{l+s+s1}{\PYGZsq{}}
\PYG{n+nb}{print}\PYG{p}{(}\PYG{n}{x}\PYG{o}{+}\PYG{n}{y}\PYG{p}{)}
\end{sphinxVerbatim}

\end{sphinxuseclass}\end{sphinxVerbatimInput}

\end{sphinxuseclass}
\end{sphinxuseclass}

\subsection{Question\sphinxhyphen{}17}
\label{\detokenize{inout_output:question-17}}
\begin{sphinxuseclass}{cell}
\begin{sphinxuseclass}{tag_hide-output}\begin{sphinxVerbatimInput}

\begin{sphinxuseclass}{cell_input}
\begin{sphinxVerbatim}[commandchars=\\\{\}]
\PYG{n+nb}{print}\PYG{p}{(}\PYG{l+s+s1}{\PYGZsq{}}\PYG{l+s+s1}{H}\PYG{l+s+s1}{\PYGZsq{}}\PYG{p}{,}\PYG{l+m+mi}{2}\PYG{p}{,}\PYG{l+m+mi}{3}\PYG{p}{,}\PYG{l+s+s1}{\PYGZsq{}}\PYG{l+s+s1}{BYE}\PYG{l+s+s1}{\PYGZsq{}}\PYG{p}{,} \PYG{n}{sep}\PYG{o}{=}\PYG{l+s+s1}{\PYGZsq{}}\PYG{l+s+s1}{++}\PYG{l+s+s1}{\PYGZsq{}}\PYG{p}{)}
\end{sphinxVerbatim}

\end{sphinxuseclass}\end{sphinxVerbatimInput}

\end{sphinxuseclass}
\end{sphinxuseclass}

\subsection{Question\sphinxhyphen{}18}
\label{\detokenize{inout_output:question-18}}
\begin{sphinxuseclass}{cell}
\begin{sphinxuseclass}{tag_hide-output}\begin{sphinxVerbatimInput}

\begin{sphinxuseclass}{cell_input}
\begin{sphinxVerbatim}[commandchars=\\\{\}]
\PYG{n+nb}{print}\PYG{p}{(}\PYG{l+s+s1}{\PYGZsq{}}\PYG{l+s+s1}{H}\PYG{l+s+s1}{\PYGZsq{}}\PYG{p}{,}\PYG{l+m+mi}{2}\PYG{p}{,}\PYG{l+m+mi}{3}\PYG{p}{,}\PYG{l+s+s1}{\PYGZsq{}}\PYG{l+s+s1}{BYE}\PYG{l+s+s1}{\PYGZsq{}}\PYG{p}{,} \PYG{n}{sep}\PYG{o}{=}\PYG{l+s+s1}{\PYGZsq{}}\PYG{l+s+s1}{++}\PYG{l+s+s1}{\PYGZsq{}}\PYG{p}{)}
\end{sphinxVerbatim}

\end{sphinxuseclass}\end{sphinxVerbatimInput}

\end{sphinxuseclass}
\end{sphinxuseclass}

\subsection{Question\sphinxhyphen{}19}
\label{\detokenize{inout_output:question-19}}
\begin{sphinxuseclass}{cell}
\begin{sphinxuseclass}{tag_hide-output}\begin{sphinxVerbatimInput}

\begin{sphinxuseclass}{cell_input}
\begin{sphinxVerbatim}[commandchars=\\\{\}]
\PYG{n+nb}{print}\PYG{p}{(}\PYG{l+s+s1}{\PYGZsq{}}\PYG{l+s+s1}{A}\PYG{l+s+s1}{\PYGZsq{}}\PYG{p}{)}
\PYG{n+nb}{print}\PYG{p}{(}\PYG{l+s+s1}{\PYGZsq{}}\PYG{l+s+s1}{B}\PYG{l+s+s1}{\PYGZsq{}}\PYG{p}{,} \PYG{n}{end}\PYG{o}{=}\PYG{l+s+s1}{\PYGZsq{}}\PYG{l+s+s1}{\PYGZhy{}\PYGZhy{}}\PYG{l+s+s1}{\PYGZsq{}}\PYG{p}{)}
\PYG{n+nb}{print}\PYG{p}{(}\PYG{l+s+s1}{\PYGZsq{}}\PYG{l+s+s1}{C}\PYG{l+s+s1}{\PYGZsq{}}\PYG{p}{)}
\PYG{n+nb}{print}\PYG{p}{(}\PYG{l+s+s1}{\PYGZsq{}}\PYG{l+s+s1}{D}\PYG{l+s+s1}{\PYGZsq{}}\PYG{p}{,} \PYG{n}{end}\PYG{o}{=}\PYG{l+s+s1}{\PYGZsq{}}\PYG{l+s+s1}{??}\PYG{l+s+s1}{\PYGZsq{}}\PYG{p}{)}
\PYG{n+nb}{print}\PYG{p}{(}\PYG{l+s+s1}{\PYGZsq{}}\PYG{l+s+s1}{E}\PYG{l+s+s1}{\PYGZsq{}}\PYG{p}{)}
\end{sphinxVerbatim}

\end{sphinxuseclass}\end{sphinxVerbatimInput}

\end{sphinxuseclass}
\end{sphinxuseclass}

\subsection{Question\sphinxhyphen{}20}
\label{\detokenize{inout_output:question-20}}
\begin{sphinxuseclass}{cell}
\begin{sphinxuseclass}{tag_hide-output}\begin{sphinxVerbatimInput}

\begin{sphinxuseclass}{cell_input}
\begin{sphinxVerbatim}[commandchars=\\\{\}]
\PYG{n+nb}{print}\PYG{p}{(}\PYG{l+s+s1}{\PYGZsq{}}\PYG{l+s+s1}{A}\PYG{l+s+s1}{\PYGZsq{}}\PYG{p}{,} \PYG{l+s+s1}{\PYGZsq{}}\PYG{l+s+s1}{A}\PYG{l+s+s1}{\PYGZsq{}}\PYG{p}{,} \PYG{l+s+s1}{\PYGZsq{}}\PYG{l+s+s1}{A}\PYG{l+s+s1}{\PYGZsq{}}\PYG{p}{,} \PYG{n}{sep}\PYG{o}{=}\PYG{l+s+s1}{\PYGZsq{}}\PYG{l+s+s1}{8}\PYG{l+s+s1}{\PYGZsq{}}\PYG{p}{,} \PYG{n}{end}\PYG{o}{=}\PYG{l+s+s1}{\PYGZsq{}}\PYG{l+s+s1}{K}\PYG{l+s+s1}{\PYGZsq{}}\PYG{p}{)}
\PYG{n+nb}{print}\PYG{p}{(}\PYG{l+s+s1}{\PYGZsq{}}\PYG{l+s+s1}{B}\PYG{l+s+s1}{\PYGZsq{}}\PYG{p}{,} \PYG{l+s+s1}{\PYGZsq{}}\PYG{l+s+s1}{B}\PYG{l+s+s1}{\PYGZsq{}}\PYG{p}{,} \PYG{n}{sep}\PYG{o}{=}\PYG{l+s+s1}{\PYGZsq{}}\PYG{l+s+s1}{\PYGZhy{}\PYGZhy{}}\PYG{l+s+s1}{\PYGZsq{}}\PYG{p}{,} \PYG{n}{end}\PYG{o}{=}\PYG{l+s+s1}{\PYGZsq{}}\PYG{l+s+s1}{L}\PYG{l+s+s1}{\PYGZsq{}}\PYG{p}{)}
\PYG{n+nb}{print}\PYG{p}{(}\PYG{l+s+s1}{\PYGZsq{}}\PYG{l+s+s1}{C}\PYG{l+s+s1}{\PYGZsq{}}\PYG{p}{)}
\PYG{n+nb}{print}\PYG{p}{(}\PYG{l+s+s1}{\PYGZsq{}}\PYG{l+s+s1}{D}\PYG{l+s+s1}{\PYGZsq{}}\PYG{p}{)}
\end{sphinxVerbatim}

\end{sphinxuseclass}\end{sphinxVerbatimInput}

\end{sphinxuseclass}
\end{sphinxuseclass}
\sphinxstepscope


\section{Input and Output Code}
\label{\detokenize{inout_code:input-and-output-code}}\label{\detokenize{inout_code::doc}}\begin{itemize}
\item {} 
\sphinxAtStartPar
Please solve the following questions using Python code.  

\end{itemize}


\subsection{Question\sphinxhyphen{}1}
\label{\detokenize{inout_code:question-1}}
\sphinxAtStartPar
Display the first 3 characters of your first name by using the character *.\\
\sphinxstylestrong{Solution}


\subsection{Question\sphinxhyphen{}2}
\label{\detokenize{inout_code:question-2}}\begin{itemize}
\item {} 
\sphinxAtStartPar
Use the given variables to print the following statement:\\
\sphinxcode{\sphinxupquote{My name is Michael. I am from Germany. I am 25 years old.}}

\item {} 
\sphinxAtStartPar
Be careful about spaces and punctuations.
\begin{itemize}
\item {} 
\sphinxAtStartPar
name = ‘Michael’

\item {} 
\sphinxAtStartPar
age = 25

\item {} 
\sphinxAtStartPar
country = ‘Germany’

\end{itemize}

\end{itemize}

\sphinxAtStartPar
\sphinxstylestrong{Solution}


\subsection{Question\sphinxhyphen{}3}
\label{\detokenize{inout_code:question-3}}
\sphinxAtStartPar
Write a program that prompts the user for 4 numbers using 4 input functions.
\begin{itemize}
\item {} 
\sphinxAtStartPar
Find the sum of these numbers and assign it to a variable.

\item {} 
\sphinxAtStartPar
Find the product of these numbers and assign it to a variable.

\item {} 
\sphinxAtStartPar
Print the sum and product of these numbers on two separate lines using a single print function.

\end{itemize}

\begin{sphinxadmonition}{note}{Solution}

\begin{sphinxVerbatim}[commandchars=\\\{\}]
\PYG{n}{num1} \PYG{o}{=} \PYG{n+nb}{float}\PYG{p}{(}\PYG{n+nb}{input}\PYG{p}{(}\PYG{l+s+s1}{\PYGZsq{}}\PYG{l+s+s1}{Number1: }\PYG{l+s+s1}{\PYGZsq{}}\PYG{p}{)}\PYG{p}{)}
\PYG{n}{num2} \PYG{o}{=} \PYG{n+nb}{float}\PYG{p}{(}\PYG{n+nb}{input}\PYG{p}{(}\PYG{l+s+s1}{\PYGZsq{}}\PYG{l+s+s1}{Number2: }\PYG{l+s+s1}{\PYGZsq{}}\PYG{p}{)}\PYG{p}{)}
\PYG{n}{num3} \PYG{o}{=} \PYG{n+nb}{float}\PYG{p}{(}\PYG{n+nb}{input}\PYG{p}{(}\PYG{l+s+s1}{\PYGZsq{}}\PYG{l+s+s1}{Number3: }\PYG{l+s+s1}{\PYGZsq{}}\PYG{p}{)}\PYG{p}{)}
\PYG{n}{num4} \PYG{o}{=} \PYG{n+nb}{float}\PYG{p}{(}\PYG{n+nb}{input}\PYG{p}{(}\PYG{l+s+s1}{\PYGZsq{}}\PYG{l+s+s1}{Number4: }\PYG{l+s+s1}{\PYGZsq{}}\PYG{p}{)}\PYG{p}{)}

\PYG{n}{total} \PYG{o}{=} \PYG{n}{num1} \PYG{o}{+} \PYG{n}{num2} \PYG{o}{+} \PYG{n}{num3} \PYG{o}{+} \PYG{n}{num4}
\PYG{n}{product} \PYG{o}{=} \PYG{n}{num1} \PYG{o}{*} \PYG{n}{num2} \PYG{o}{*} \PYG{n}{num3} \PYG{o}{*} \PYG{n}{num4}

\PYG{n+nb}{print}\PYG{p}{(}\PYG{l+s+s1}{\PYGZsq{}}\PYG{l+s+s1}{Sum: }\PYG{l+s+s1}{\PYGZsq{}}\PYG{p}{,} \PYG{n}{total}\PYG{p}{,} \PYG{l+s+s1}{\PYGZsq{}}\PYG{l+s+se}{\PYGZbs{}n}\PYG{l+s+s1}{Product:}\PYG{l+s+s1}{\PYGZsq{}}\PYG{p}{,} \PYG{n}{product}\PYG{p}{)}
\end{sphinxVerbatim}
\end{sphinxadmonition}


\subsection{Question\sphinxhyphen{}4}
\label{\detokenize{inout_code:question-4}}\begin{itemize}
\item {} 
\sphinxAtStartPar
Use the given variables to print \sphinxcode{\sphinxupquote{Bill\sphinxhyphen{} \sphinxhyphen{}Gates}} by using five different codes.
\begin{itemize}
\item {} 
\sphinxAtStartPar
first\_name = ‘Bill’

\item {} 
\sphinxAtStartPar
last\_name = ‘Gates’

\end{itemize}

\end{itemize}

\sphinxAtStartPar
\sphinxstylestrong{Solution}


\subsection{Question\sphinxhyphen{}5}
\label{\detokenize{inout_code:question-5}}
\sphinxAtStartPar
Write a program which prompts the user for an integer.
\begin{itemize}
\item {} 
\sphinxAtStartPar
Find the square of the given number.

\item {} 
\sphinxAtStartPar
Print the following statement:
\begin{itemize}
\item {} 
\sphinxAtStartPar
square(number) = square of the number

\item {} 
\sphinxAtStartPar
Example: If the given number is \sphinxcode{\sphinxupquote{3}} then the output should be \sphinxcode{\sphinxupquote{square(3)=9}}

\item {} 
\sphinxAtStartPar
Example: If the given number is \sphinxcode{\sphinxupquote{5}} then the output should be \sphinxcode{\sphinxupquote{square(5)=25}}

\end{itemize}

\end{itemize}

\begin{sphinxadmonition}{note}{Solution}

\begin{sphinxVerbatim}[commandchars=\\\{\}]
\PYG{n}{number} \PYG{o}{=} \PYG{n+nb}{int}\PYG{p}{(}\PYG{n+nb}{input}\PYG{p}{(}\PYG{l+s+s1}{\PYGZsq{}}\PYG{l+s+s1}{Enter an integer: }\PYG{l+s+s1}{\PYGZsq{}}\PYG{p}{)}\PYG{p}{)}

\PYG{n}{square} \PYG{o}{=} \PYG{n}{number}\PYG{o}{*}\PYG{o}{*}\PYG{l+m+mi}{2}

\PYG{n+nb}{print}\PYG{p}{(}\PYG{l+s+s1}{\PYGZsq{}}\PYG{l+s+s1}{square(}\PYG{l+s+s1}{\PYGZsq{}}\PYG{p}{,} \PYG{n}{number}\PYG{p}{,} \PYG{l+s+s1}{\PYGZsq{}}\PYG{l+s+s1}{)=}\PYG{l+s+s1}{\PYGZsq{}}\PYG{p}{,} \PYG{n}{square}\PYG{p}{,} \PYG{n}{sep}\PYG{o}{=}\PYG{l+s+s1}{\PYGZsq{}}\PYG{l+s+s1}{\PYGZsq{}}\PYG{p}{)}
\end{sphinxVerbatim}
\end{sphinxadmonition}

\sphinxstepscope


\section{Input and Output Exercises}
\label{\detokenize{inout_exercise:input-and-output-exercises}}\label{\detokenize{inout_exercise::doc}}

\subsection{Question\sphinxhyphen{}1}
\label{\detokenize{inout_exercise:question-1}}
\sphinxAtStartPar
Write a single print statement that produces the following output:\\
A\\
B\\
CD\\
E


\subsection{Question\sphinxhyphen{}2}
\label{\detokenize{inout_exercise:question-2}}
\sphinxAtStartPar
Use 2 \sphinxstyleemphasis{print()} functions, and the strings ‘A’, ‘B’, ‘C’, ‘D’  to generate the following output.
A—B—C—D


\subsection{Question\sphinxhyphen{}3}
\label{\detokenize{inout_exercise:question-3}}
\sphinxAtStartPar
Use a single \sphinxstyleemphasis{print()} function and the strings ‘ABCDE’ and ‘FGH’ to generate the following output: \sphinxstyleemphasis{FGCDE}


\subsection{Question\sphinxhyphen{}4}
\label{\detokenize{inout_exercise:question-4}}
\sphinxAtStartPar
Write a program that prompts the user for 3 numbers using 3 \sphinxstyleemphasis{input()} functions.
\begin{itemize}
\item {} 
\sphinxAtStartPar
Find the harmonic mean of these numbers using the formula:
\(\displaystyle H(x, y, z) = \frac{3}{\frac{1}{x}+\frac{1}{y}+\frac{1}{z}}\)

\item {} 
\sphinxAtStartPar
Round the harmonic mean to the nearest hundredth and print it in the following format: \(H(x,y,z)=\) rounded harmonic mean.

\item {} 
\sphinxAtStartPar
Sample Output:

\end{itemize}

\sphinxAtStartPar
Enter x:  2\\
Enter y:  3\\
Enter z:  4

\sphinxAtStartPar
H(2.0,3.0,4.0) = 2.77


\subsection{Question\sphinxhyphen{}5}
\label{\detokenize{inout_exercise:question-5}}
\sphinxAtStartPar
Write a program that prompts the user for a positive integer.
\begin{itemize}
\item {} 
\sphinxAtStartPar
Print that integer.

\item {} 
\sphinxAtStartPar
On the second line, print one more than the given number two times separated by a dash.

\item {} 
\sphinxAtStartPar
On the third line, print two more than the given number three times separated by dashes.

\item {} 
\sphinxAtStartPar
On the fourth line, print three more than the given number four times separated by dashes.

\item {} 
\sphinxAtStartPar
On the fifth line, print four more than the given number five times separated by dashes.

\end{itemize}

\sphinxAtStartPar
Sample Output:\\
Please enter an integer: 5\\
5\\
6\sphinxhyphen{}6\\
7\sphinxhyphen{}7\sphinxhyphen{}7\\
8\sphinxhyphen{}8\sphinxhyphen{}8\sphinxhyphen{}8\\
9\sphinxhyphen{}9\sphinxhyphen{}9\sphinxhyphen{}9\sphinxhyphen{}9


\subsection{Question\sphinxhyphen{}6}
\label{\detokenize{inout_exercise:question-6}}
\sphinxAtStartPar
Write a program that prompts the user for a number and a word using 2 \sphinxstyleemphasis{input()} functions.
\begin{itemize}
\item {} 
\sphinxAtStartPar
Print the given word as many times as the given number.

\end{itemize}


\subsection{Question\sphinxhyphen{}7: Std}
\label{\detokenize{inout_exercise:question-7-std}}
\sphinxAtStartPar
Write a program that prompts the user for 3 numbers using 3 \sphinxstyleemphasis{input()} functions.
\begin{itemize}
\item {} 
\sphinxAtStartPar
Find the mean of these three numbers.

\item {} 
\sphinxAtStartPar
Subtract the mean from each number.

\item {} 
\sphinxAtStartPar
Square each of these differences.

\item {} 
\sphinxAtStartPar
Find the mean of these squares.

\item {} 
\sphinxAtStartPar
Take the square root of this mean and round it to the nearest hundredth.

\end{itemize}

\sphinxAtStartPar
\sphinxstyleemphasis{Remark:This is the standard deviation of the given numbers, which provides information about how the given numbers are spread out around the mean.}


\subsection{Question\sphinxhyphen{}8: Countdown}
\label{\detokenize{inout_exercise:question-8-countdown}}
\sphinxAtStartPar
Write a program that prints the numbers starting from 5 down to 1 and displays ‘START’ after that.
\begin{itemize}
\item {} 
\sphinxAtStartPar
Use the \sphinxstyleemphasis{sleep()} function from the \sphinxstyleemphasis{time} module to add a 1\sphinxhyphen{}second waiting time between each number for observation.

\item {} 
\sphinxAtStartPar
After printing a number, pause for 1 second, then delete it before printing the next number.

\item {} 
\sphinxAtStartPar
The output of your program should resemble a countdown.

\end{itemize}


\subsection{Question\sphinxhyphen{}9: Arrow}
\label{\detokenize{inout_exercise:question-9-arrow}}
\sphinxAtStartPar
Write a program that prints the moving arrow ‘—>’.
\begin{itemize}
\item {} 
\sphinxAtStartPar
Print the arrow, then add a one\sphinxhyphen{}second waiting time, then delete it.

\item {} 
\sphinxAtStartPar
Print the arrow again with a single space in front of it, then add a one\sphinxhyphen{}second waiting time, then delete it.

\item {} 
\sphinxAtStartPar
Repeat this procedure 10 times by increasing the space one more in each step.

\item {} 
\sphinxAtStartPar
The output of your program should resemble a moving arrow to the right.

\end{itemize}

\sphinxstepscope


\chapter{Chp\sphinxhyphen{}3: Numbers}
\label{\detokenize{numbers_arithmetic:chp-3-numbers}}\label{\detokenize{numbers_arithmetic::doc}}\begin{itemize}
\item {} 
\sphinxAtStartPar
Learning Objectives
\begin{itemize}
\item {} 
\sphinxAtStartPar
..

\item {} 
\sphinxAtStartPar
..

\end{itemize}

\end{itemize}


\section{Number Types}
\label{\detokenize{numbers_arithmetic:number-types}}
\sphinxAtStartPar
In this chapter, three different types of numbers in Python will be covered, but integers and floats will be the primary types used in this book. \sphinxstyleemphasis{If the sections on Floats, Complex Numbers, and Scientific Notation seem too technical for you, feel free to skip them.}
\begin{enumerate}
\sphinxsetlistlabels{\arabic}{enumi}{enumii}{}{.}%
\item {} 
\sphinxAtStartPar
Integers: …,\sphinxhyphen{}2,\sphinxhyphen{}1,0,1,2,…
\begin{itemize}
\item {} 
\sphinxAtStartPar
Type: int

\item {} 
\sphinxAtStartPar
In Python, underscores (rather than commas) are used to separate large numbers into groups of three digits.

\end{itemize}

\end{enumerate}


\bigskip\hrule\bigskip

\begin{enumerate}
\sphinxsetlistlabels{\arabic}{enumi}{enumii}{}{.}%
\setcounter{enumi}{1}
\item {} 
\sphinxAtStartPar
Float   : 4.5,  3.0
\begin{itemize}
\item {} 
\sphinxAtStartPar
Decimal numbers (They have decimal point).

\item {} 
\sphinxAtStartPar
4.0 is a float

\item {} 
\sphinxAtStartPar
Type: float

\item {} 
\sphinxAtStartPar
Float values are approximately stored, but they are close enough to their real values to maintain practical accuracy.

\end{itemize}

\end{enumerate}


\bigskip\hrule\bigskip

\begin{enumerate}
\sphinxsetlistlabels{\arabic}{enumi}{enumii}{}{.}%
\setcounter{enumi}{2}
\item {} 
\sphinxAtStartPar
Complex Numbers: \(3+4j\)
\begin{itemize}
\item {} 
\sphinxAtStartPar
Real numbers with imaginary parts.

\item {} 
\sphinxAtStartPar
In mathematics, the symbol \(i\) is used to represent the imaginary unit.

\item {} 
\sphinxAtStartPar
In Python, the symbol \(j\) is used to represent the imaginary unit.

\item {} 
\sphinxAtStartPar
The built\sphinxhyphen{}in \sphinxcode{\sphinxupquote{complex()}} function is used to create complex numbers.

\item {} 
\sphinxAtStartPar
Type: complex

\end{itemize}

\end{enumerate}

\begin{sphinxuseclass}{cell}\begin{sphinxVerbatimInput}

\begin{sphinxuseclass}{cell_input}
\begin{sphinxVerbatim}[commandchars=\\\{\}]
\PYG{c+c1}{\PYGZsh{} type of 5 is integer}
\PYG{n+nb}{print}\PYG{p}{(}\PYG{n+nb}{type}\PYG{p}{(}\PYG{l+m+mi}{5}\PYG{p}{)}\PYG{p}{)}
\end{sphinxVerbatim}

\end{sphinxuseclass}\end{sphinxVerbatimInput}
\begin{sphinxVerbatimOutput}

\begin{sphinxuseclass}{cell_output}
\begin{sphinxVerbatim}[commandchars=\\\{\}]
\PYGZlt{}class \PYGZsq{}int\PYGZsq{}\PYGZgt{}
\end{sphinxVerbatim}

\end{sphinxuseclass}\end{sphinxVerbatimOutput}

\end{sphinxuseclass}
\begin{sphinxuseclass}{cell}\begin{sphinxVerbatimInput}

\begin{sphinxuseclass}{cell_input}
\begin{sphinxVerbatim}[commandchars=\\\{\}]
\PYG{c+c1}{\PYGZsh{} type of \PYGZhy{}5 is integer}
\PYG{n+nb}{print}\PYG{p}{(}\PYG{n+nb}{type}\PYG{p}{(}\PYG{o}{\PYGZhy{}}\PYG{l+m+mi}{5}\PYG{p}{)}\PYG{p}{)}
\end{sphinxVerbatim}

\end{sphinxuseclass}\end{sphinxVerbatimInput}
\begin{sphinxVerbatimOutput}

\begin{sphinxuseclass}{cell_output}
\begin{sphinxVerbatim}[commandchars=\\\{\}]
\PYGZlt{}class \PYGZsq{}int\PYGZsq{}\PYGZgt{}
\end{sphinxVerbatim}

\end{sphinxuseclass}\end{sphinxVerbatimOutput}

\end{sphinxuseclass}
\begin{sphinxuseclass}{cell}\begin{sphinxVerbatimInput}

\begin{sphinxuseclass}{cell_input}
\begin{sphinxVerbatim}[commandchars=\\\{\}]
\PYG{c+c1}{\PYGZsh{} type of 12.89 is float}
\PYG{n+nb}{print}\PYG{p}{(}\PYG{n+nb}{type}\PYG{p}{(}\PYG{l+m+mf}{12.89}\PYG{p}{)}\PYG{p}{)}
\end{sphinxVerbatim}

\end{sphinxuseclass}\end{sphinxVerbatimInput}
\begin{sphinxVerbatimOutput}

\begin{sphinxuseclass}{cell_output}
\begin{sphinxVerbatim}[commandchars=\\\{\}]
\PYGZlt{}class \PYGZsq{}float\PYGZsq{}\PYGZgt{}
\end{sphinxVerbatim}

\end{sphinxuseclass}\end{sphinxVerbatimOutput}

\end{sphinxuseclass}
\begin{sphinxuseclass}{cell}\begin{sphinxVerbatimInput}

\begin{sphinxuseclass}{cell_input}
\begin{sphinxVerbatim}[commandchars=\\\{\}]
\PYG{c+c1}{\PYGZsh{} type of 4.0 is float}
\PYG{n+nb}{print}\PYG{p}{(}\PYG{n+nb}{type}\PYG{p}{(}\PYG{l+m+mf}{4.0}\PYG{p}{)}\PYG{p}{)}
\end{sphinxVerbatim}

\end{sphinxuseclass}\end{sphinxVerbatimInput}
\begin{sphinxVerbatimOutput}

\begin{sphinxuseclass}{cell_output}
\begin{sphinxVerbatim}[commandchars=\\\{\}]
\PYGZlt{}class \PYGZsq{}float\PYGZsq{}\PYGZgt{}
\end{sphinxVerbatim}

\end{sphinxuseclass}\end{sphinxVerbatimOutput}

\end{sphinxuseclass}

\section{Floats}
\label{\detokenize{numbers_arithmetic:floats}}\begin{itemize}
\item {} 
\sphinxAtStartPar
Python does not store the value 0.1 exactly.

\item {} 
\sphinxAtStartPar
The following code reveals that there are non\sphinxhyphen{}zero numbers in the decimal tail, which are not expected to be there.

\item {} 
\sphinxAtStartPar
The built\sphinxhyphen{}in \sphinxcode{\sphinxupquote{format()}} function can be used to display more decimal places.

\end{itemize}

\begin{sphinxuseclass}{cell}\begin{sphinxVerbatimInput}

\begin{sphinxuseclass}{cell_input}
\begin{sphinxVerbatim}[commandchars=\\\{\}]
\PYG{c+c1}{\PYGZsh{} 30 digits after decimal point}
\PYG{n+nb}{print}\PYG{p}{(}\PYG{n+nb}{format}\PYG{p}{(}\PYG{l+m+mf}{0.1}\PYG{p}{,} \PYG{l+s+s1}{\PYGZsq{}}\PYG{l+s+s1}{.30f}\PYG{l+s+s1}{\PYGZsq{}}\PYG{p}{)}\PYG{p}{)}
\end{sphinxVerbatim}

\end{sphinxuseclass}\end{sphinxVerbatimInput}
\begin{sphinxVerbatimOutput}

\begin{sphinxuseclass}{cell_output}
\begin{sphinxVerbatim}[commandchars=\\\{\}]
0.100000000000000005551115123126
\end{sphinxVerbatim}

\end{sphinxuseclass}\end{sphinxVerbatimOutput}

\end{sphinxuseclass}\begin{itemize}
\item {} 
\sphinxAtStartPar
0.5 is stored exactly in Python.

\end{itemize}

\begin{sphinxuseclass}{cell}\begin{sphinxVerbatimInput}

\begin{sphinxuseclass}{cell_input}
\begin{sphinxVerbatim}[commandchars=\\\{\}]
\PYG{c+c1}{\PYGZsh{} 30 digits after decimal point}
\PYG{n+nb}{print}\PYG{p}{(}\PYG{n+nb}{format}\PYG{p}{(}\PYG{l+m+mf}{0.5}\PYG{p}{,} \PYG{l+s+s1}{\PYGZsq{}}\PYG{l+s+s1}{.30f}\PYG{l+s+s1}{\PYGZsq{}}\PYG{p}{)}\PYG{p}{)}
\end{sphinxVerbatim}

\end{sphinxuseclass}\end{sphinxVerbatimInput}
\begin{sphinxVerbatimOutput}

\begin{sphinxuseclass}{cell_output}
\begin{sphinxVerbatim}[commandchars=\\\{\}]
0.500000000000000000000000000000
\end{sphinxVerbatim}

\end{sphinxuseclass}\end{sphinxVerbatimOutput}

\end{sphinxuseclass}\begin{itemize}
\item {} 
\sphinxAtStartPar
0.375 is stored exactly in Python.

\end{itemize}

\begin{sphinxuseclass}{cell}\begin{sphinxVerbatimInput}

\begin{sphinxuseclass}{cell_input}
\begin{sphinxVerbatim}[commandchars=\\\{\}]
\PYG{c+c1}{\PYGZsh{} 30 digits after decimal point}
\PYG{n+nb}{print}\PYG{p}{(}\PYG{n+nb}{format}\PYG{p}{(}\PYG{l+m+mf}{0.375}\PYG{p}{,} \PYG{l+s+s1}{\PYGZsq{}}\PYG{l+s+s1}{.30f}\PYG{l+s+s1}{\PYGZsq{}}\PYG{p}{)}\PYG{p}{)}
\end{sphinxVerbatim}

\end{sphinxuseclass}\end{sphinxVerbatimInput}
\begin{sphinxVerbatimOutput}

\begin{sphinxuseclass}{cell_output}
\begin{sphinxVerbatim}[commandchars=\\\{\}]
0.375000000000000000000000000000
\end{sphinxVerbatim}

\end{sphinxuseclass}\end{sphinxVerbatimOutput}

\end{sphinxuseclass}\begin{itemize}
\item {} 
\sphinxAtStartPar
This is because numbers, even decimal ones, are stored in base two.

\item {} 
\sphinxAtStartPar
If you can represent a decimal number in base two, it will be stored exactly.

\item {} 
\sphinxAtStartPar
However, some numbers, like 0.1, cannot be accurately represented in base two.

\item {} 
\sphinxAtStartPar
Examples
\begin{itemize}
\item {} 
\sphinxAtStartPar
If you try to write the 0.375 in base two, you get:
\begin{itemize}
\item {} 
\sphinxAtStartPar
\(0.375=\frac{3}{8}=\frac{1}{4}+\frac{1}{8}\)

\item {} 
\sphinxAtStartPar
Hence \(0.375\) in base two is \(0.011\) and stored exactly.

\end{itemize}

\item {} 
\sphinxAtStartPar
If you attempt a similar computation for 0.1, you will encounter an infinite series.
\begin{itemize}
\item {} 
\sphinxAtStartPar
\(0.1 = \frac{1}{16}+\frac{1}{32}+\frac{1}{256}+\frac{1}{512}+... \)

\item {} 
\sphinxAtStartPar
Hence \(0.1\) is  \(0.000110011...\) in base two.

\end{itemize}

\end{itemize}

\end{itemize}


\section{Complex Numbers}
\label{\detokenize{numbers_arithmetic:complex-numbers}}\begin{itemize}
\item {} 
\sphinxAtStartPar
Use the built\sphinxhyphen{}in \sphinxcode{\sphinxupquote{complex()}} function to create complex numbers.

\item {} 
\sphinxAtStartPar
It takes two parameters: the real part and the imaginary part.

\item {} 
\sphinxAtStartPar
With complex numbers, you can perform algebraic operations and find conjugates.

\end{itemize}

\begin{sphinxuseclass}{cell}\begin{sphinxVerbatimInput}

\begin{sphinxuseclass}{cell_input}
\begin{sphinxVerbatim}[commandchars=\\\{\}]
\PYG{c+c1}{\PYGZsh{} real part = 2, imaginary part = 3}
\PYG{n+nb}{print}\PYG{p}{(}\PYG{n+nb}{complex}\PYG{p}{(}\PYG{l+m+mi}{2}\PYG{p}{,}\PYG{l+m+mi}{3}\PYG{p}{)}\PYG{p}{)}
\end{sphinxVerbatim}

\end{sphinxuseclass}\end{sphinxVerbatimInput}
\begin{sphinxVerbatimOutput}

\begin{sphinxuseclass}{cell_output}
\begin{sphinxVerbatim}[commandchars=\\\{\}]
(2+3j)
\end{sphinxVerbatim}

\end{sphinxuseclass}\end{sphinxVerbatimOutput}

\end{sphinxuseclass}
\begin{sphinxuseclass}{cell}\begin{sphinxVerbatimInput}

\begin{sphinxuseclass}{cell_input}
\begin{sphinxVerbatim}[commandchars=\\\{\}]
\PYG{n}{z} \PYG{o}{=} \PYG{n+nb}{complex}\PYG{p}{(}\PYG{l+m+mi}{2}\PYG{p}{,}\PYG{l+m+mi}{3}\PYG{p}{)}
\end{sphinxVerbatim}

\end{sphinxuseclass}\end{sphinxVerbatimInput}

\end{sphinxuseclass}
\begin{sphinxuseclass}{cell}\begin{sphinxVerbatimInput}

\begin{sphinxuseclass}{cell_input}
\begin{sphinxVerbatim}[commandchars=\\\{\}]
\PYG{c+c1}{\PYGZsh{} real part of z}
\PYG{n+nb}{print}\PYG{p}{(}\PYG{n}{z}\PYG{o}{.}\PYG{n}{real}\PYG{p}{)}
\end{sphinxVerbatim}

\end{sphinxuseclass}\end{sphinxVerbatimInput}
\begin{sphinxVerbatimOutput}

\begin{sphinxuseclass}{cell_output}
\begin{sphinxVerbatim}[commandchars=\\\{\}]
2.0
\end{sphinxVerbatim}

\end{sphinxuseclass}\end{sphinxVerbatimOutput}

\end{sphinxuseclass}
\begin{sphinxuseclass}{cell}\begin{sphinxVerbatimInput}

\begin{sphinxuseclass}{cell_input}
\begin{sphinxVerbatim}[commandchars=\\\{\}]
\PYG{c+c1}{\PYGZsh{} imaginary part of z}
\PYG{n+nb}{print}\PYG{p}{(}\PYG{n}{z}\PYG{o}{.}\PYG{n}{imag}\PYG{p}{)}
\end{sphinxVerbatim}

\end{sphinxuseclass}\end{sphinxVerbatimInput}
\begin{sphinxVerbatimOutput}

\begin{sphinxuseclass}{cell_output}
\begin{sphinxVerbatim}[commandchars=\\\{\}]
3.0
\end{sphinxVerbatim}

\end{sphinxuseclass}\end{sphinxVerbatimOutput}

\end{sphinxuseclass}
\begin{sphinxuseclass}{cell}\begin{sphinxVerbatimInput}

\begin{sphinxuseclass}{cell_input}
\begin{sphinxVerbatim}[commandchars=\\\{\}]
\PYG{c+c1}{\PYGZsh{} conjugate of z}
\PYG{n+nb}{print}\PYG{p}{(}\PYG{n}{z}\PYG{o}{.}\PYG{n}{conjugate}\PYG{p}{(}\PYG{p}{)}\PYG{p}{)}
\end{sphinxVerbatim}

\end{sphinxuseclass}\end{sphinxVerbatimInput}
\begin{sphinxVerbatimOutput}

\begin{sphinxuseclass}{cell_output}
\begin{sphinxVerbatim}[commandchars=\\\{\}]
(2\PYGZhy{}3j)
\end{sphinxVerbatim}

\end{sphinxuseclass}\end{sphinxVerbatimOutput}

\end{sphinxuseclass}
\begin{sphinxuseclass}{cell}\begin{sphinxVerbatimInput}

\begin{sphinxuseclass}{cell_input}
\begin{sphinxVerbatim}[commandchars=\\\{\}]
\PYG{n}{t} \PYG{o}{=} \PYG{n+nb}{complex}\PYG{p}{(}\PYG{l+m+mi}{5}\PYG{p}{,}\PYG{l+m+mi}{7}\PYG{p}{)}
\PYG{n+nb}{print}\PYG{p}{(}\PYG{n}{t}\PYG{p}{)}
\end{sphinxVerbatim}

\end{sphinxuseclass}\end{sphinxVerbatimInput}
\begin{sphinxVerbatimOutput}

\begin{sphinxuseclass}{cell_output}
\begin{sphinxVerbatim}[commandchars=\\\{\}]
(5+7j)
\end{sphinxVerbatim}

\end{sphinxuseclass}\end{sphinxVerbatimOutput}

\end{sphinxuseclass}
\begin{sphinxuseclass}{cell}\begin{sphinxVerbatimInput}

\begin{sphinxuseclass}{cell_input}
\begin{sphinxVerbatim}[commandchars=\\\{\}]
\PYG{c+c1}{\PYGZsh{} addition}
\PYG{n+nb}{print}\PYG{p}{(}\PYG{n}{z} \PYG{o}{+} \PYG{n}{t}\PYG{p}{)}
\end{sphinxVerbatim}

\end{sphinxuseclass}\end{sphinxVerbatimInput}
\begin{sphinxVerbatimOutput}

\begin{sphinxuseclass}{cell_output}
\begin{sphinxVerbatim}[commandchars=\\\{\}]
(7+10j)
\end{sphinxVerbatim}

\end{sphinxuseclass}\end{sphinxVerbatimOutput}

\end{sphinxuseclass}
\begin{sphinxuseclass}{cell}\begin{sphinxVerbatimInput}

\begin{sphinxuseclass}{cell_input}
\begin{sphinxVerbatim}[commandchars=\\\{\}]
\PYG{c+c1}{\PYGZsh{} subtraction}
\PYG{n+nb}{print}\PYG{p}{(}\PYG{n}{z}\PYG{o}{\PYGZhy{}}\PYG{n}{t}\PYG{p}{)}
\end{sphinxVerbatim}

\end{sphinxuseclass}\end{sphinxVerbatimInput}
\begin{sphinxVerbatimOutput}

\begin{sphinxuseclass}{cell_output}
\begin{sphinxVerbatim}[commandchars=\\\{\}]
(\PYGZhy{}3\PYGZhy{}4j)
\end{sphinxVerbatim}

\end{sphinxuseclass}\end{sphinxVerbatimOutput}

\end{sphinxuseclass}

\section{Scientific Notation}
\label{\detokenize{numbers_arithmetic:scientific-notation}}
\sphinxAtStartPar
It is used to represent very large or very small numbers in a more compact form.
\begin{itemize}
\item {} 
\sphinxAtStartPar
In scientific notation, a number is written in the form of:
\begin{itemize}
\item {} 
\sphinxAtStartPar
(a number between 1 and 10) x e(a power of 10)
\begin{itemize}
\item {} 
\sphinxAtStartPar
Example: \(12345 = 1.2345 \times 10^{4} = 1.2345e+04\)

\item {} 
\sphinxAtStartPar
Example: \(0.00123 = 1.23 \times 10^{-3} = 1.23e-03\)

\end{itemize}

\end{itemize}

\item {} 
\sphinxAtStartPar
In scientific notation,
\begin{itemize}
\item {} 
\sphinxAtStartPar
e+04 means \(10^4\)

\item {} 
\sphinxAtStartPar
e\sphinxhyphen{}03 means \(10^{-3}\)

\end{itemize}

\item {} 
\sphinxAtStartPar
Instead of e, you can also use E for scientific notation.

\item {} 
\sphinxAtStartPar
The \sphinxcode{\sphinxupquote{format()}} function can be used to represent a number using scientific notation.

\end{itemize}

\begin{sphinxuseclass}{cell}\begin{sphinxVerbatimInput}

\begin{sphinxuseclass}{cell_input}
\begin{sphinxVerbatim}[commandchars=\\\{\}]
\PYG{c+c1}{\PYGZsh{} Scientific notation e}
\PYG{c+c1}{\PYGZsh{} The number 3 represents the number of digits up to which the given number is to be rounded.\PYGZdq{}}
\PYG{n+nb}{print}\PYG{p}{(}\PYG{n+nb}{format}\PYG{p}{(}\PYG{l+m+mi}{12645}\PYG{p}{,} \PYG{l+s+s1}{\PYGZsq{}}\PYG{l+s+s1}{10.3e}\PYG{l+s+s1}{\PYGZsq{}}\PYG{p}{)}\PYG{p}{)}
\end{sphinxVerbatim}

\end{sphinxuseclass}\end{sphinxVerbatimInput}
\begin{sphinxVerbatimOutput}

\begin{sphinxuseclass}{cell_output}
\begin{sphinxVerbatim}[commandchars=\\\{\}]
 1.264e+04
\end{sphinxVerbatim}

\end{sphinxuseclass}\end{sphinxVerbatimOutput}

\end{sphinxuseclass}
\begin{sphinxuseclass}{cell}\begin{sphinxVerbatimInput}

\begin{sphinxuseclass}{cell_input}
\begin{sphinxVerbatim}[commandchars=\\\{\}]
\PYG{c+c1}{\PYGZsh{} round 1.2645 to the nearest tenth}
\PYG{n+nb}{print}\PYG{p}{(}\PYG{n+nb}{format}\PYG{p}{(}\PYG{l+m+mi}{12645}\PYG{p}{,} \PYG{l+s+s1}{\PYGZsq{}}\PYG{l+s+s1}{10.1e}\PYG{l+s+s1}{\PYGZsq{}}\PYG{p}{)}\PYG{p}{)}
\end{sphinxVerbatim}

\end{sphinxuseclass}\end{sphinxVerbatimInput}
\begin{sphinxVerbatimOutput}

\begin{sphinxuseclass}{cell_output}
\begin{sphinxVerbatim}[commandchars=\\\{\}]
   1.3e+04
\end{sphinxVerbatim}

\end{sphinxuseclass}\end{sphinxVerbatimOutput}

\end{sphinxuseclass}
\begin{sphinxuseclass}{cell}\begin{sphinxVerbatimInput}

\begin{sphinxuseclass}{cell_input}
\begin{sphinxVerbatim}[commandchars=\\\{\}]
\PYG{c+c1}{\PYGZsh{} You can also use E instead of e}
\PYG{n+nb}{print}\PYG{p}{(}\PYG{n+nb}{format}\PYG{p}{(}\PYG{l+m+mi}{12645}\PYG{p}{,} \PYG{l+s+s1}{\PYGZsq{}}\PYG{l+s+s1}{10.1E}\PYG{l+s+s1}{\PYGZsq{}}\PYG{p}{)}\PYG{p}{)}
\end{sphinxVerbatim}

\end{sphinxuseclass}\end{sphinxVerbatimInput}
\begin{sphinxVerbatimOutput}

\begin{sphinxuseclass}{cell_output}
\begin{sphinxVerbatim}[commandchars=\\\{\}]
   1.3E+04
\end{sphinxVerbatim}

\end{sphinxuseclass}\end{sphinxVerbatimOutput}

\end{sphinxuseclass}

\section{Large Numbers}
\label{\detokenize{numbers_arithmetic:large-numbers}}\begin{itemize}
\item {} 
\sphinxAtStartPar
\(1,234,578\) is a large number, and commas are used to facilitate a better understanding of this number.

\item {} 
\sphinxAtStartPar
In Python, underscores \sphinxstyleemphasis{(\_)} are used instead of commas due to the special functionalities associated with commas in Python.

\end{itemize}

\begin{sphinxuseclass}{cell}\begin{sphinxVerbatimInput}

\begin{sphinxuseclass}{cell_input}
\begin{sphinxVerbatim}[commandchars=\\\{\}]
\PYG{n+nb}{print}\PYG{p}{(}\PYG{l+m+mi}{1\PYGZus{}234\PYGZus{}578}\PYG{p}{)}
\end{sphinxVerbatim}

\end{sphinxuseclass}\end{sphinxVerbatimInput}
\begin{sphinxVerbatimOutput}

\begin{sphinxuseclass}{cell_output}
\begin{sphinxVerbatim}[commandchars=\\\{\}]
1234578
\end{sphinxVerbatim}

\end{sphinxuseclass}\end{sphinxVerbatimOutput}

\end{sphinxuseclass}
\sphinxAtStartPar
In the following code, comma\sphinxhyphen{}separated three values are printed.
\begin{itemize}
\item {} 
\sphinxAtStartPar
Since the sep parameter is by default one space,  these three values are displayed with one space between them.

\end{itemize}

\begin{sphinxuseclass}{cell}\begin{sphinxVerbatimInput}

\begin{sphinxuseclass}{cell_input}
\begin{sphinxVerbatim}[commandchars=\\\{\}]
\PYG{n+nb}{print}\PYG{p}{(}\PYG{l+m+mi}{1}\PYG{p}{,}\PYG{l+m+mi}{234}\PYG{p}{,}\PYG{l+m+mi}{578}\PYG{p}{)}
\end{sphinxVerbatim}

\end{sphinxuseclass}\end{sphinxVerbatimInput}
\begin{sphinxVerbatimOutput}

\begin{sphinxuseclass}{cell_output}
\begin{sphinxVerbatim}[commandchars=\\\{\}]
1 234 578
\end{sphinxVerbatim}

\end{sphinxuseclass}\end{sphinxVerbatimOutput}

\end{sphinxuseclass}

\section{Operations on numbers}
\label{\detokenize{numbers_arithmetic:operations-on-numbers}}
\sphinxAtStartPar
The following operations are commonly used in Python:


\begin{savenotes}\sphinxattablestart
\centering
\begin{tabulary}{\linewidth}[t]{|T|T|}
\hline
\sphinxstyletheadfamily 
\sphinxAtStartPar
Symbol
&\sphinxstyletheadfamily 
\sphinxAtStartPar
Operation
\\
\hline
\sphinxAtStartPar
\(+\)
&
\sphinxAtStartPar
addition
\\
\hline
\sphinxAtStartPar
\(-\)
&
\sphinxAtStartPar
subtraction
\\
\hline
\sphinxAtStartPar
\(*\)
&
\sphinxAtStartPar
multiplication
\\
\hline
\sphinxAtStartPar
\(**\)
&
\sphinxAtStartPar
exponent
\\
\hline
\sphinxAtStartPar
\(/\)
&
\sphinxAtStartPar
division
\\
\hline
\sphinxAtStartPar
\(//\)
&
\sphinxAtStartPar
divide and floor (integer division)
\\
\hline
\sphinxAtStartPar
\(\%\)
&
\sphinxAtStartPar
remainder
\\
\hline
\end{tabulary}
\par
\sphinxattableend\end{savenotes}
\begin{itemize}
\item {} 
\sphinxAtStartPar
Division in Python always returns a float, even in cases where the result is a whole number.
\begin{itemize}
\item {} 
\sphinxAtStartPar
12/4 = 3.0

\end{itemize}

\item {} 
\sphinxAtStartPar
Integer division rounds down to the nearest smaller integer.
\begin{itemize}
\item {} 
\sphinxAtStartPar
Example:
\begin{itemize}
\item {} 
\sphinxAtStartPar
13/5  = 2.6

\item {} 
\sphinxAtStartPar
13//5 = 2

\end{itemize}

\item {} 
\sphinxAtStartPar
Example:
\begin{itemize}
\item {} 
\sphinxAtStartPar
\sphinxhyphen{}12/5  = \sphinxhyphen{}2.4

\item {} 
\sphinxAtStartPar
\sphinxhyphen{}12//5 = \sphinxhyphen{}3

\end{itemize}

\end{itemize}

\item {} 
\sphinxAtStartPar
The remainder operation \sphinxcode{\sphinxupquote{\%}} returns the remainder of a division.
\begin{itemize}
\item {} 
\sphinxAtStartPar
Example:
\begin{itemize}
\item {} 
\sphinxAtStartPar
When we divide 13 by 5, the quotient is 2, and the remainder is 3.

\item {} 
\sphinxAtStartPar
13\%5=3

\end{itemize}

\end{itemize}

\end{itemize}

\begin{sphinxuseclass}{cell}\begin{sphinxVerbatimInput}

\begin{sphinxuseclass}{cell_input}
\begin{sphinxVerbatim}[commandchars=\\\{\}]
\PYG{c+c1}{\PYGZsh{} addition }
\PYG{n+nb}{print}\PYG{p}{(}\PYG{l+m+mi}{5}\PYG{o}{+}\PYG{l+m+mi}{3}\PYG{p}{)}
\end{sphinxVerbatim}

\end{sphinxuseclass}\end{sphinxVerbatimInput}
\begin{sphinxVerbatimOutput}

\begin{sphinxuseclass}{cell_output}
\begin{sphinxVerbatim}[commandchars=\\\{\}]
8
\end{sphinxVerbatim}

\end{sphinxuseclass}\end{sphinxVerbatimOutput}

\end{sphinxuseclass}
\begin{sphinxuseclass}{cell}\begin{sphinxVerbatimInput}

\begin{sphinxuseclass}{cell_input}
\begin{sphinxVerbatim}[commandchars=\\\{\}]
\PYG{c+c1}{\PYGZsh{} subtraction}
\PYG{n+nb}{print}\PYG{p}{(}\PYG{l+m+mi}{5}\PYG{o}{\PYGZhy{}}\PYG{l+m+mi}{3}\PYG{p}{)}
\end{sphinxVerbatim}

\end{sphinxuseclass}\end{sphinxVerbatimInput}
\begin{sphinxVerbatimOutput}

\begin{sphinxuseclass}{cell_output}
\begin{sphinxVerbatim}[commandchars=\\\{\}]
2
\end{sphinxVerbatim}

\end{sphinxuseclass}\end{sphinxVerbatimOutput}

\end{sphinxuseclass}
\begin{sphinxuseclass}{cell}\begin{sphinxVerbatimInput}

\begin{sphinxuseclass}{cell_input}
\begin{sphinxVerbatim}[commandchars=\\\{\}]
\PYG{c+c1}{\PYGZsh{} multiplication}
\PYG{n+nb}{print}\PYG{p}{(}\PYG{l+m+mi}{5}\PYG{o}{*}\PYG{l+m+mi}{3}\PYG{p}{)}
\end{sphinxVerbatim}

\end{sphinxuseclass}\end{sphinxVerbatimInput}
\begin{sphinxVerbatimOutput}

\begin{sphinxuseclass}{cell_output}
\begin{sphinxVerbatim}[commandchars=\\\{\}]
15
\end{sphinxVerbatim}

\end{sphinxuseclass}\end{sphinxVerbatimOutput}

\end{sphinxuseclass}
\begin{sphinxuseclass}{cell}\begin{sphinxVerbatimInput}

\begin{sphinxuseclass}{cell_input}
\begin{sphinxVerbatim}[commandchars=\\\{\}]
\PYG{c+c1}{\PYGZsh{} division}
\PYG{n+nb}{print}\PYG{p}{(}\PYG{l+m+mi}{13}\PYG{o}{/}\PYG{l+m+mi}{5}\PYG{p}{)}
\end{sphinxVerbatim}

\end{sphinxuseclass}\end{sphinxVerbatimInput}
\begin{sphinxVerbatimOutput}

\begin{sphinxuseclass}{cell_output}
\begin{sphinxVerbatim}[commandchars=\\\{\}]
2.6
\end{sphinxVerbatim}

\end{sphinxuseclass}\end{sphinxVerbatimOutput}

\end{sphinxuseclass}
\begin{sphinxuseclass}{cell}\begin{sphinxVerbatimInput}

\begin{sphinxuseclass}{cell_input}
\begin{sphinxVerbatim}[commandchars=\\\{\}]
\PYG{c+c1}{\PYGZsh{} exponent}
\PYG{c+c1}{\PYGZsh{} 2 to the third power}
\PYG{n+nb}{print}\PYG{p}{(}\PYG{l+m+mi}{2}\PYG{o}{*}\PYG{o}{*}\PYG{l+m+mi}{3}\PYG{p}{)}
\end{sphinxVerbatim}

\end{sphinxuseclass}\end{sphinxVerbatimInput}
\begin{sphinxVerbatimOutput}

\begin{sphinxuseclass}{cell_output}
\begin{sphinxVerbatim}[commandchars=\\\{\}]
8
\end{sphinxVerbatim}

\end{sphinxuseclass}\end{sphinxVerbatimOutput}

\end{sphinxuseclass}
\begin{sphinxuseclass}{cell}\begin{sphinxVerbatimInput}

\begin{sphinxuseclass}{cell_input}
\begin{sphinxVerbatim}[commandchars=\\\{\}]
\PYG{c+c1}{\PYGZsh{} integer division}
\PYG{n+nb}{print}\PYG{p}{(}\PYG{l+m+mi}{13}\PYG{o}{/}\PYG{o}{/}\PYG{l+m+mi}{5}\PYG{p}{)}
\end{sphinxVerbatim}

\end{sphinxuseclass}\end{sphinxVerbatimInput}
\begin{sphinxVerbatimOutput}

\begin{sphinxuseclass}{cell_output}
\begin{sphinxVerbatim}[commandchars=\\\{\}]
2
\end{sphinxVerbatim}

\end{sphinxuseclass}\end{sphinxVerbatimOutput}

\end{sphinxuseclass}
\begin{sphinxuseclass}{cell}\begin{sphinxVerbatimInput}

\begin{sphinxuseclass}{cell_input}
\begin{sphinxVerbatim}[commandchars=\\\{\}]
\PYG{c+c1}{\PYGZsh{} remainder}
\PYG{c+c1}{\PYGZsh{} when we divide 13 by 5, the remainder is 3.}
\PYG{n+nb}{print}\PYG{p}{(}\PYG{l+m+mi}{13}\PYG{o}{\PYGZpc{}}\PYG{k}{5})
\end{sphinxVerbatim}

\end{sphinxuseclass}\end{sphinxVerbatimInput}
\begin{sphinxVerbatimOutput}

\begin{sphinxuseclass}{cell_output}
\begin{sphinxVerbatim}[commandchars=\\\{\}]
3
\end{sphinxVerbatim}

\end{sphinxuseclass}\end{sphinxVerbatimOutput}

\end{sphinxuseclass}
\begin{sphinxuseclass}{cell}\begin{sphinxVerbatimInput}

\begin{sphinxuseclass}{cell_input}
\begin{sphinxVerbatim}[commandchars=\\\{\}]
\PYG{c+c1}{\PYGZsh{} power = 1/2 means square root}
\PYG{n+nb}{print}\PYG{p}{(}\PYG{l+m+mi}{49}\PYG{o}{*}\PYG{o}{*}\PYG{p}{(}\PYG{l+m+mi}{1}\PYG{o}{/}\PYG{l+m+mi}{2}\PYG{p}{)}\PYG{p}{)}
\end{sphinxVerbatim}

\end{sphinxuseclass}\end{sphinxVerbatimInput}
\begin{sphinxVerbatimOutput}

\begin{sphinxuseclass}{cell_output}
\begin{sphinxVerbatim}[commandchars=\\\{\}]
7.0
\end{sphinxVerbatim}

\end{sphinxuseclass}\end{sphinxVerbatimOutput}

\end{sphinxuseclass}
\begin{sphinxuseclass}{cell}\begin{sphinxVerbatimInput}

\begin{sphinxuseclass}{cell_input}
\begin{sphinxVerbatim}[commandchars=\\\{\}]
\PYG{c+c1}{\PYGZsh{} square root of negative numbers are complex numbers.}
\PYG{c+c1}{\PYGZsh{} square root of \PYGZhy{}1}
\PYG{n+nb}{print}\PYG{p}{(}\PYG{p}{(}\PYG{o}{\PYGZhy{}}\PYG{l+m+mi}{1}\PYG{p}{)}\PYG{o}{*}\PYG{o}{*}\PYG{p}{(}\PYG{l+m+mi}{1}\PYG{o}{/}\PYG{l+m+mi}{2}\PYG{p}{)}\PYG{p}{)}
\end{sphinxVerbatim}

\end{sphinxuseclass}\end{sphinxVerbatimInput}
\begin{sphinxVerbatimOutput}

\begin{sphinxuseclass}{cell_output}
\begin{sphinxVerbatim}[commandchars=\\\{\}]
(6.123233995736766e\PYGZhy{}17+1j)
\end{sphinxVerbatim}

\end{sphinxuseclass}\end{sphinxVerbatimOutput}

\end{sphinxuseclass}\begin{itemize}
\item {} 
\sphinxAtStartPar
In the code above we expect to have only \(1j\) which is \(j\).

\item {} 
\sphinxAtStartPar
There is a real part in this complex number, which is supposed to be 0.

\item {} 
\sphinxAtStartPar
The real part \(6.123233995736766e-17\) is in scientific notation, representing \(6.123233995736766\times 10^{-17}\)

\item {} 
\sphinxAtStartPar
This is a very small number close to 0.

\item {} 
\sphinxAtStartPar
We have this small number as the real part because floats are stored approximately.

\end{itemize}


\section{Conversions}
\label{\detokenize{numbers_arithmetic:conversions}}\begin{itemize}
\item {} 
\sphinxAtStartPar
The built\sphinxhyphen{}in \sphinxcode{\sphinxupquote{int()}} function is used to convert floats and suitable strings to integers.

\item {} 
\sphinxAtStartPar
The built\sphinxhyphen{}in \sphinxcode{\sphinxupquote{float()}} function is used to convert integers and suitable strings to floats.

\item {} 
\sphinxAtStartPar
The built\sphinxhyphen{}in \sphinxcode{\sphinxupquote{str()}} function is used to convert integers and floats to strings.

\end{itemize}

\begin{sphinxuseclass}{cell}\begin{sphinxVerbatimInput}

\begin{sphinxuseclass}{cell_input}
\begin{sphinxVerbatim}[commandchars=\\\{\}]
\PYG{c+c1}{\PYGZsh{} convert positive float x to integer y}
\PYG{n}{x} \PYG{o}{=} \PYG{l+m+mf}{7.56}
\PYG{n}{y} \PYG{o}{=} \PYG{n+nb}{int}\PYG{p}{(}\PYG{n}{x}\PYG{p}{)}
\PYG{n+nb}{print}\PYG{p}{(}\PYG{l+s+s1}{\PYGZsq{}}\PYG{l+s+s1}{Type:}\PYG{l+s+s1}{\PYGZsq{}}\PYG{p}{,} \PYG{n+nb}{type}\PYG{p}{(}\PYG{n}{y}\PYG{p}{)}\PYG{p}{)}
\PYG{n+nb}{print}\PYG{p}{(}\PYG{l+s+s1}{\PYGZsq{}}\PYG{l+s+s1}{y =}\PYG{l+s+s1}{\PYGZsq{}}\PYG{p}{,} \PYG{n}{y}\PYG{p}{)}
\end{sphinxVerbatim}

\end{sphinxuseclass}\end{sphinxVerbatimInput}
\begin{sphinxVerbatimOutput}

\begin{sphinxuseclass}{cell_output}
\begin{sphinxVerbatim}[commandchars=\\\{\}]
Type: \PYGZlt{}class \PYGZsq{}int\PYGZsq{}\PYGZgt{}
y = 7
\end{sphinxVerbatim}

\end{sphinxuseclass}\end{sphinxVerbatimOutput}

\end{sphinxuseclass}
\begin{sphinxuseclass}{cell}\begin{sphinxVerbatimInput}

\begin{sphinxuseclass}{cell_input}
\begin{sphinxVerbatim}[commandchars=\\\{\}]
\PYG{c+c1}{\PYGZsh{} convert negative float x to integer y}
\PYG{n}{x} \PYG{o}{=} \PYG{o}{\PYGZhy{}}\PYG{l+m+mf}{7.56}
\PYG{n}{y} \PYG{o}{=} \PYG{n+nb}{int}\PYG{p}{(}\PYG{n}{x}\PYG{p}{)}
\PYG{n+nb}{print}\PYG{p}{(}\PYG{l+s+s1}{\PYGZsq{}}\PYG{l+s+s1}{Type:}\PYG{l+s+s1}{\PYGZsq{}}\PYG{p}{,} \PYG{n+nb}{type}\PYG{p}{(}\PYG{n}{y}\PYG{p}{)}\PYG{p}{)}
\PYG{n+nb}{print}\PYG{p}{(}\PYG{l+s+s1}{\PYGZsq{}}\PYG{l+s+s1}{y =}\PYG{l+s+s1}{\PYGZsq{}}\PYG{p}{,} \PYG{n}{y}\PYG{p}{)}
\end{sphinxVerbatim}

\end{sphinxuseclass}\end{sphinxVerbatimInput}
\begin{sphinxVerbatimOutput}

\begin{sphinxuseclass}{cell_output}
\begin{sphinxVerbatim}[commandchars=\\\{\}]
Type: \PYGZlt{}class \PYGZsq{}int\PYGZsq{}\PYGZgt{}
y = \PYGZhy{}7
\end{sphinxVerbatim}

\end{sphinxuseclass}\end{sphinxVerbatimOutput}

\end{sphinxuseclass}
\begin{sphinxuseclass}{cell}\begin{sphinxVerbatimInput}

\begin{sphinxuseclass}{cell_input}
\begin{sphinxVerbatim}[commandchars=\\\{\}]
\PYG{c+c1}{\PYGZsh{} convert string x to integer}
\PYG{n}{x} \PYG{o}{=} \PYG{l+s+s1}{\PYGZsq{}}\PYG{l+s+s1}{7}\PYG{l+s+s1}{\PYGZsq{}}
\PYG{n}{y} \PYG{o}{=} \PYG{n+nb}{int}\PYG{p}{(}\PYG{n}{x}\PYG{p}{)}
\PYG{n+nb}{print}\PYG{p}{(}\PYG{l+s+s1}{\PYGZsq{}}\PYG{l+s+s1}{Type:}\PYG{l+s+s1}{\PYGZsq{}}\PYG{p}{,} \PYG{n+nb}{type}\PYG{p}{(}\PYG{n}{y}\PYG{p}{)}\PYG{p}{)}
\PYG{n+nb}{print}\PYG{p}{(}\PYG{l+s+s1}{\PYGZsq{}}\PYG{l+s+s1}{y =}\PYG{l+s+s1}{\PYGZsq{}}\PYG{p}{,} \PYG{n}{y}\PYG{p}{)}
\end{sphinxVerbatim}

\end{sphinxuseclass}\end{sphinxVerbatimInput}
\begin{sphinxVerbatimOutput}

\begin{sphinxuseclass}{cell_output}
\begin{sphinxVerbatim}[commandchars=\\\{\}]
Type: \PYGZlt{}class \PYGZsq{}int\PYGZsq{}\PYGZgt{}
y = 7
\end{sphinxVerbatim}

\end{sphinxuseclass}\end{sphinxVerbatimOutput}

\end{sphinxuseclass}
\begin{sphinxVerbatim}[commandchars=\\\{\}]
\PYG{c+c1}{\PYGZsh{} ERROR: \PYGZsq{}Tom\PYGZsq{} cannot be converted to an integer.}
\PYG{n+nb}{int}\PYG{p}{(}\PYG{l+s+s1}{\PYGZsq{}}\PYG{l+s+s1}{Tom}\PYG{l+s+s1}{\PYGZsq{}}\PYG{p}{)}
\end{sphinxVerbatim}

\begin{sphinxuseclass}{cell}\begin{sphinxVerbatimInput}

\begin{sphinxuseclass}{cell_input}
\begin{sphinxVerbatim}[commandchars=\\\{\}]
\PYG{c+c1}{\PYGZsh{} convert integer x to float y}
\PYG{n}{x} \PYG{o}{=} \PYG{l+m+mi}{7}
\PYG{n}{y} \PYG{o}{=} \PYG{n+nb}{float}\PYG{p}{(}\PYG{n}{x}\PYG{p}{)}
\PYG{n+nb}{print}\PYG{p}{(}\PYG{l+s+s1}{\PYGZsq{}}\PYG{l+s+s1}{Type:}\PYG{l+s+s1}{\PYGZsq{}}\PYG{p}{,} \PYG{n+nb}{type}\PYG{p}{(}\PYG{n}{y}\PYG{p}{)}\PYG{p}{)}
\PYG{n+nb}{print}\PYG{p}{(}\PYG{l+s+s1}{\PYGZsq{}}\PYG{l+s+s1}{y =}\PYG{l+s+s1}{\PYGZsq{}}\PYG{p}{,} \PYG{n}{y}\PYG{p}{)}
\end{sphinxVerbatim}

\end{sphinxuseclass}\end{sphinxVerbatimInput}
\begin{sphinxVerbatimOutput}

\begin{sphinxuseclass}{cell_output}
\begin{sphinxVerbatim}[commandchars=\\\{\}]
Type: \PYGZlt{}class \PYGZsq{}float\PYGZsq{}\PYGZgt{}
y = 7.0
\end{sphinxVerbatim}

\end{sphinxuseclass}\end{sphinxVerbatimOutput}

\end{sphinxuseclass}
\begin{sphinxuseclass}{cell}\begin{sphinxVerbatimInput}

\begin{sphinxuseclass}{cell_input}
\begin{sphinxVerbatim}[commandchars=\\\{\}]
\PYG{c+c1}{\PYGZsh{} convert string x to float y}
\PYG{n}{x} \PYG{o}{=} \PYG{l+s+s1}{\PYGZsq{}}\PYG{l+s+s1}{7.53}\PYG{l+s+s1}{\PYGZsq{}}
\PYG{n}{y} \PYG{o}{=} \PYG{n+nb}{float}\PYG{p}{(}\PYG{n}{x}\PYG{p}{)}
\PYG{n+nb}{print}\PYG{p}{(}\PYG{l+s+s1}{\PYGZsq{}}\PYG{l+s+s1}{Type:}\PYG{l+s+s1}{\PYGZsq{}}\PYG{p}{,} \PYG{n+nb}{type}\PYG{p}{(}\PYG{n}{y}\PYG{p}{)}\PYG{p}{)}
\PYG{n+nb}{print}\PYG{p}{(}\PYG{l+s+s1}{\PYGZsq{}}\PYG{l+s+s1}{y =}\PYG{l+s+s1}{\PYGZsq{}}\PYG{p}{,} \PYG{n}{y}\PYG{p}{)}
\end{sphinxVerbatim}

\end{sphinxuseclass}\end{sphinxVerbatimInput}
\begin{sphinxVerbatimOutput}

\begin{sphinxuseclass}{cell_output}
\begin{sphinxVerbatim}[commandchars=\\\{\}]
Type: \PYGZlt{}class \PYGZsq{}float\PYGZsq{}\PYGZgt{}
y = 7.53
\end{sphinxVerbatim}

\end{sphinxuseclass}\end{sphinxVerbatimOutput}

\end{sphinxuseclass}\begin{itemize}
\item {} 
\sphinxAtStartPar
You cannot convert decimal numbers in string type to an integer.

\end{itemize}

\begin{sphinxVerbatim}[commandchars=\\\{\}]
\PYG{c+c1}{\PYGZsh{} ERROR: string \PYGZsq{}7.63\PYGZsq{} cannot be converted to an integer.}
\PYG{n+nb}{int}\PYG{p}{(}\PYG{l+s+s1}{\PYGZsq{}}\PYG{l+s+s1}{7.53}\PYG{l+s+s1}{\PYGZsq{}}\PYG{p}{)}
\end{sphinxVerbatim}
\begin{itemize}
\item {} 
\sphinxAtStartPar
Extra spaces on the left or right do not affect the conversion process.

\end{itemize}

\begin{sphinxuseclass}{cell}\begin{sphinxVerbatimInput}

\begin{sphinxuseclass}{cell_input}
\begin{sphinxVerbatim}[commandchars=\\\{\}]
\PYG{n+nb}{print}\PYG{p}{(}\PYG{n+nb}{int}\PYG{p}{(}\PYG{l+s+s1}{\PYGZsq{}}\PYG{l+s+s1}{     234   }\PYG{l+s+s1}{\PYGZsq{}}\PYG{p}{)}\PYG{p}{)}
\end{sphinxVerbatim}

\end{sphinxuseclass}\end{sphinxVerbatimInput}
\begin{sphinxVerbatimOutput}

\begin{sphinxuseclass}{cell_output}
\begin{sphinxVerbatim}[commandchars=\\\{\}]
234
\end{sphinxVerbatim}

\end{sphinxuseclass}\end{sphinxVerbatimOutput}

\end{sphinxuseclass}

\section{Precedence (PEMDAS)}
\label{\detokenize{numbers_arithmetic:precedence-pemdas}}
\sphinxAtStartPar
The operation precedence of the operations in Python follows the following order:
\begin{itemize}
\item {} 
\sphinxAtStartPar
Parenthesis

\item {} 
\sphinxAtStartPar
Exponents

\item {} 
\sphinxAtStartPar
Multiplication, Division, Integer Division, Remainder

\item {} 
\sphinxAtStartPar
Addition, Subtraction

\end{itemize}

\begin{sphinxuseclass}{cell}\begin{sphinxVerbatimInput}

\begin{sphinxuseclass}{cell_input}
\begin{sphinxVerbatim}[commandchars=\\\{\}]
\PYG{c+c1}{\PYGZsh{} first multiplication, then addition}
\PYG{n+nb}{print}\PYG{p}{(}\PYG{l+m+mi}{2}\PYG{o}{+}\PYG{l+m+mi}{5}\PYG{o}{*}\PYG{l+m+mi}{3}\PYG{p}{)}
\end{sphinxVerbatim}

\end{sphinxuseclass}\end{sphinxVerbatimInput}
\begin{sphinxVerbatimOutput}

\begin{sphinxuseclass}{cell_output}
\begin{sphinxVerbatim}[commandchars=\\\{\}]
17
\end{sphinxVerbatim}

\end{sphinxuseclass}\end{sphinxVerbatimOutput}

\end{sphinxuseclass}
\begin{sphinxuseclass}{cell}\begin{sphinxVerbatimInput}

\begin{sphinxuseclass}{cell_input}
\begin{sphinxVerbatim}[commandchars=\\\{\}]
\PYG{c+c1}{\PYGZsh{} first remainder, then addition}
\PYG{n+nb}{print}\PYG{p}{(}\PYG{l+m+mi}{2}\PYG{o}{+}\PYG{l+m+mi}{5}\PYG{o}{\PYGZpc{}}\PYG{k}{3})
\end{sphinxVerbatim}

\end{sphinxuseclass}\end{sphinxVerbatimInput}
\begin{sphinxVerbatimOutput}

\begin{sphinxuseclass}{cell_output}
\begin{sphinxVerbatim}[commandchars=\\\{\}]
4
\end{sphinxVerbatim}

\end{sphinxuseclass}\end{sphinxVerbatimOutput}

\end{sphinxuseclass}
\begin{sphinxuseclass}{cell}\begin{sphinxVerbatimInput}

\begin{sphinxuseclass}{cell_input}
\begin{sphinxVerbatim}[commandchars=\\\{\}]
\PYG{c+c1}{\PYGZsh{} first Integer Division, then addition}
\PYG{n+nb}{print}\PYG{p}{(}\PYG{l+m+mi}{2}\PYG{o}{+}\PYG{l+m+mi}{5}\PYG{o}{/}\PYG{o}{/}\PYG{l+m+mi}{3}\PYG{p}{)}
\end{sphinxVerbatim}

\end{sphinxuseclass}\end{sphinxVerbatimInput}
\begin{sphinxVerbatimOutput}

\begin{sphinxuseclass}{cell_output}
\begin{sphinxVerbatim}[commandchars=\\\{\}]
3
\end{sphinxVerbatim}

\end{sphinxuseclass}\end{sphinxVerbatimOutput}

\end{sphinxuseclass}
\begin{sphinxuseclass}{cell}\begin{sphinxVerbatimInput}

\begin{sphinxuseclass}{cell_input}
\begin{sphinxVerbatim}[commandchars=\\\{\}]
\PYG{c+c1}{\PYGZsh{} first exponent, then addition }
\PYG{n+nb}{print}\PYG{p}{(}\PYG{l+m+mi}{2}\PYG{o}{+}\PYG{l+m+mi}{5}\PYG{o}{*}\PYG{o}{*}\PYG{l+m+mi}{3}\PYG{p}{)}
\end{sphinxVerbatim}

\end{sphinxuseclass}\end{sphinxVerbatimInput}
\begin{sphinxVerbatimOutput}

\begin{sphinxuseclass}{cell_output}
\begin{sphinxVerbatim}[commandchars=\\\{\}]
127
\end{sphinxVerbatim}

\end{sphinxuseclass}\end{sphinxVerbatimOutput}

\end{sphinxuseclass}

\section{Abbreviated operators}
\label{\detokenize{numbers_arithmetic:abbreviated-operators}}\begin{itemize}
\item {} 
\sphinxAtStartPar
In mathematics, the expression \(x=x+1\) represents an equation. By subtracting \(x\) from both sides, the equation can be solved.

\item {} 
\sphinxAtStartPar
In programming languages, \(x=x+1\) is an assignment, not an equation. Remember that \sphinxcode{\sphinxupquote{=}} is an assignment operator.

\item {} 
\sphinxAtStartPar
In the assignment \(x=x+1\), the following steps occur:
\begin{enumerate}
\sphinxsetlistlabels{\arabic}{enumi}{enumii}{}{.}%
\item {} 
\sphinxAtStartPar
The expression on the right\sphinxhyphen{}hand side, \(x+1\), is computed first using the current value of \(x\).

\item {} 
\sphinxAtStartPar
The result becomes the new value of \(x\).

\end{enumerate}

\end{itemize}

\begin{sphinxuseclass}{cell}\begin{sphinxVerbatimInput}

\begin{sphinxuseclass}{cell_input}
\begin{sphinxVerbatim}[commandchars=\\\{\}]
\PYG{n}{x} \PYG{o}{=} \PYG{l+m+mi}{3}        \PYG{c+c1}{\PYGZsh{} value of x is 3}
\PYG{n}{x} \PYG{o}{=} \PYG{n}{x}\PYG{o}{+}\PYG{l+m+mi}{1}      \PYG{c+c1}{\PYGZsh{} x = 3+1}
\PYG{n+nb}{print}\PYG{p}{(}\PYG{n}{x}\PYG{p}{)}   
\end{sphinxVerbatim}

\end{sphinxuseclass}\end{sphinxVerbatimInput}
\begin{sphinxVerbatimOutput}

\begin{sphinxuseclass}{cell_output}
\begin{sphinxVerbatim}[commandchars=\\\{\}]
4
\end{sphinxVerbatim}

\end{sphinxuseclass}\end{sphinxVerbatimOutput}

\end{sphinxuseclass}
\begin{sphinxuseclass}{cell}\begin{sphinxVerbatimInput}

\begin{sphinxuseclass}{cell_input}
\begin{sphinxVerbatim}[commandchars=\\\{\}]
\PYG{n}{x} \PYG{o}{=} \PYG{l+m+mi}{25}        \PYG{c+c1}{\PYGZsh{} value of x is 25}
\PYG{n}{x} \PYG{o}{=} \PYG{n}{x}\PYG{o}{\PYGZhy{}}\PYG{l+m+mi}{3}       \PYG{c+c1}{\PYGZsh{} x = 25\PYGZhy{}3}
\PYG{n+nb}{print}\PYG{p}{(}\PYG{n}{x}\PYG{p}{)}  
\end{sphinxVerbatim}

\end{sphinxuseclass}\end{sphinxVerbatimInput}
\begin{sphinxVerbatimOutput}

\begin{sphinxuseclass}{cell_output}
\begin{sphinxVerbatim}[commandchars=\\\{\}]
22
\end{sphinxVerbatim}

\end{sphinxuseclass}\end{sphinxVerbatimOutput}

\end{sphinxuseclass}
\begin{sphinxuseclass}{cell}\begin{sphinxVerbatimInput}

\begin{sphinxuseclass}{cell_input}
\begin{sphinxVerbatim}[commandchars=\\\{\}]
\PYG{n}{x} \PYG{o}{=} \PYG{l+m+mi}{10}        \PYG{c+c1}{\PYGZsh{} value of x is 10}
\PYG{n}{x} \PYG{o}{=} \PYG{n}{x}\PYG{o}{*}\PYG{l+m+mi}{2}       \PYG{c+c1}{\PYGZsh{} x = 10*2}
\PYG{n+nb}{print}\PYG{p}{(}\PYG{n}{x}\PYG{p}{)}   
\end{sphinxVerbatim}

\end{sphinxuseclass}\end{sphinxVerbatimInput}
\begin{sphinxVerbatimOutput}

\begin{sphinxuseclass}{cell_output}
\begin{sphinxVerbatim}[commandchars=\\\{\}]
20
\end{sphinxVerbatim}

\end{sphinxuseclass}\end{sphinxVerbatimOutput}

\end{sphinxuseclass}
\begin{sphinxuseclass}{cell}\begin{sphinxVerbatimInput}

\begin{sphinxuseclass}{cell_input}
\begin{sphinxVerbatim}[commandchars=\\\{\}]
\PYG{n}{x} \PYG{o}{=} \PYG{l+m+mi}{15}        \PYG{c+c1}{\PYGZsh{} value of x is 15}
\PYG{n}{x} \PYG{o}{=} \PYG{n}{x}\PYG{o}{/}\PYG{l+m+mi}{3}       \PYG{c+c1}{\PYGZsh{} x = 15/3}
\PYG{n+nb}{print}\PYG{p}{(}\PYG{n}{x}\PYG{p}{)}
\end{sphinxVerbatim}

\end{sphinxuseclass}\end{sphinxVerbatimInput}
\begin{sphinxVerbatimOutput}

\begin{sphinxuseclass}{cell_output}
\begin{sphinxVerbatim}[commandchars=\\\{\}]
5.0
\end{sphinxVerbatim}

\end{sphinxuseclass}\end{sphinxVerbatimOutput}

\end{sphinxuseclass}\begin{itemize}
\item {} 
\sphinxAtStartPar
Since these types of assignments are used frequently, there is a shorter version for them.

\end{itemize}


\begin{savenotes}\sphinxattablestart
\centering
\begin{tabulary}{\linewidth}[t]{|T|T|}
\hline
\sphinxstyletheadfamily 
\sphinxAtStartPar
regular
&\sphinxstyletheadfamily 
\sphinxAtStartPar
shorter
\\
\hline
\sphinxAtStartPar
x = x+2
&
\sphinxAtStartPar
x+=2
\\
\hline
\sphinxAtStartPar
x = x\sphinxhyphen{}2
&
\sphinxAtStartPar
x\sphinxhyphen{}=2
\\
\hline
\sphinxAtStartPar
x = x*2
&
\sphinxAtStartPar
x*=2
\\
\hline
\sphinxAtStartPar
x = x/2
&
\sphinxAtStartPar
x/=2
\\
\hline
\end{tabulary}
\par
\sphinxattableend\end{savenotes}

\begin{sphinxuseclass}{cell}\begin{sphinxVerbatimInput}

\begin{sphinxuseclass}{cell_input}
\begin{sphinxVerbatim}[commandchars=\\\{\}]
\PYG{n}{x}\PYG{o}{=}\PYG{l+m+mi}{3}        \PYG{c+c1}{\PYGZsh{} value of x is 3}
\PYG{n}{x} \PYG{o}{+}\PYG{o}{=} \PYG{l+m+mi}{1}     \PYG{c+c1}{\PYGZsh{} x = 3+1}
\PYG{n+nb}{print}\PYG{p}{(}\PYG{n}{x}\PYG{p}{)}   
\end{sphinxVerbatim}

\end{sphinxuseclass}\end{sphinxVerbatimInput}
\begin{sphinxVerbatimOutput}

\begin{sphinxuseclass}{cell_output}
\begin{sphinxVerbatim}[commandchars=\\\{\}]
4
\end{sphinxVerbatim}

\end{sphinxuseclass}\end{sphinxVerbatimOutput}

\end{sphinxuseclass}
\begin{sphinxuseclass}{cell}\begin{sphinxVerbatimInput}

\begin{sphinxuseclass}{cell_input}
\begin{sphinxVerbatim}[commandchars=\\\{\}]
\PYG{n}{x} \PYG{o}{=} \PYG{l+m+mi}{25}        \PYG{c+c1}{\PYGZsh{} value of x is 25}
\PYG{n}{x} \PYG{o}{\PYGZhy{}}\PYG{o}{=} \PYG{l+m+mi}{3}        \PYG{c+c1}{\PYGZsh{} x = 25\PYGZhy{}3}
\PYG{n+nb}{print}\PYG{p}{(}\PYG{n}{x}\PYG{p}{)}  
\end{sphinxVerbatim}

\end{sphinxuseclass}\end{sphinxVerbatimInput}
\begin{sphinxVerbatimOutput}

\begin{sphinxuseclass}{cell_output}
\begin{sphinxVerbatim}[commandchars=\\\{\}]
22
\end{sphinxVerbatim}

\end{sphinxuseclass}\end{sphinxVerbatimOutput}

\end{sphinxuseclass}
\begin{sphinxuseclass}{cell}\begin{sphinxVerbatimInput}

\begin{sphinxuseclass}{cell_input}
\begin{sphinxVerbatim}[commandchars=\\\{\}]
\PYG{n}{x} \PYG{o}{=} \PYG{l+m+mi}{10}        \PYG{c+c1}{\PYGZsh{} value of x is 10}
\PYG{n}{x} \PYG{o}{*}\PYG{o}{=} \PYG{l+m+mi}{2}        \PYG{c+c1}{\PYGZsh{} x = 10*2}
\PYG{n+nb}{print}\PYG{p}{(}\PYG{n}{x}\PYG{p}{)}   
\end{sphinxVerbatim}

\end{sphinxuseclass}\end{sphinxVerbatimInput}
\begin{sphinxVerbatimOutput}

\begin{sphinxuseclass}{cell_output}
\begin{sphinxVerbatim}[commandchars=\\\{\}]
20
\end{sphinxVerbatim}

\end{sphinxuseclass}\end{sphinxVerbatimOutput}

\end{sphinxuseclass}
\begin{sphinxuseclass}{cell}\begin{sphinxVerbatimInput}

\begin{sphinxuseclass}{cell_input}
\begin{sphinxVerbatim}[commandchars=\\\{\}]
\PYG{n}{x} \PYG{o}{=} \PYG{l+m+mi}{15}        \PYG{c+c1}{\PYGZsh{} value of x is 15}
\PYG{n}{x} \PYG{o}{/}\PYG{o}{=} \PYG{l+m+mi}{3}        \PYG{c+c1}{\PYGZsh{} x = 15/3}
\PYG{n+nb}{print}\PYG{p}{(}\PYG{n}{x}\PYG{p}{)}
\end{sphinxVerbatim}

\end{sphinxuseclass}\end{sphinxVerbatimInput}
\begin{sphinxVerbatimOutput}

\begin{sphinxuseclass}{cell_output}
\begin{sphinxVerbatim}[commandchars=\\\{\}]
5.0
\end{sphinxVerbatim}

\end{sphinxuseclass}\end{sphinxVerbatimOutput}

\end{sphinxuseclass}

\section{Built\sphinxhyphen{}in Functions}
\label{\detokenize{numbers_arithmetic:built-in-functions}}
\sphinxAtStartPar
Python has many useful functions available for use without importing additional modules.
\begin{itemize}
\item {} 
\sphinxAtStartPar
You can find the list of built\sphinxhyphen{}in functions in the \sphinxhref{https://docs.python.org/3/library/functions.html}{official Python documentation}.

\item {} 
\sphinxAtStartPar
Some of the built\sphinxhyphen{}in functions related to mathematics include:
\begin{itemize}
\item {} 
\sphinxAtStartPar
\sphinxcode{\sphinxupquote{abs()}}: returns the absolute value.

\item {} 
\sphinxAtStartPar
\sphinxcode{\sphinxupquote{max()}}: returns the maximum value in a given list of numbers.

\item {} 
\sphinxAtStartPar
\sphinxcode{\sphinxupquote{min()}}: returns the minimum value in a given list of numbers.

\item {} 
\sphinxAtStartPar
\sphinxcode{\sphinxupquote{sum()}}: returns the sum of the values in a given list of numbers.

\item {} 
\sphinxAtStartPar
\sphinxcode{\sphinxupquote{pow()}}: returns a number (base) raised to a certain power.

\item {} 
\sphinxAtStartPar
\sphinxcode{\sphinxupquote{round()}}: rounds a number to a certain decimal place.

\end{itemize}

\end{itemize}

\begin{sphinxuseclass}{cell}\begin{sphinxVerbatimInput}

\begin{sphinxuseclass}{cell_input}
\begin{sphinxVerbatim}[commandchars=\\\{\}]
\PYG{c+c1}{\PYGZsh{} absolute values of \PYGZhy{}7}
\PYG{n+nb}{print}\PYG{p}{(}\PYG{n+nb}{abs}\PYG{p}{(}\PYG{o}{\PYGZhy{}}\PYG{l+m+mi}{7}\PYG{p}{)}\PYG{p}{)}
\end{sphinxVerbatim}

\end{sphinxuseclass}\end{sphinxVerbatimInput}
\begin{sphinxVerbatimOutput}

\begin{sphinxuseclass}{cell_output}
\begin{sphinxVerbatim}[commandchars=\\\{\}]
7
\end{sphinxVerbatim}

\end{sphinxuseclass}\end{sphinxVerbatimOutput}

\end{sphinxuseclass}
\begin{sphinxuseclass}{cell}\begin{sphinxVerbatimInput}

\begin{sphinxuseclass}{cell_input}
\begin{sphinxVerbatim}[commandchars=\\\{\}]
\PYG{c+c1}{\PYGZsh{} maximum of 1,9,2,4 is 9}
\PYG{n+nb}{print}\PYG{p}{(}\PYG{n+nb}{max}\PYG{p}{(}\PYG{l+m+mi}{1}\PYG{p}{,}\PYG{l+m+mi}{9}\PYG{p}{,}\PYG{l+m+mi}{2}\PYG{p}{,}\PYG{l+m+mi}{4}\PYG{p}{)}\PYG{p}{)}
\end{sphinxVerbatim}

\end{sphinxuseclass}\end{sphinxVerbatimInput}
\begin{sphinxVerbatimOutput}

\begin{sphinxuseclass}{cell_output}
\begin{sphinxVerbatim}[commandchars=\\\{\}]
9
\end{sphinxVerbatim}

\end{sphinxuseclass}\end{sphinxVerbatimOutput}

\end{sphinxuseclass}
\begin{sphinxuseclass}{cell}\begin{sphinxVerbatimInput}

\begin{sphinxuseclass}{cell_input}
\begin{sphinxVerbatim}[commandchars=\\\{\}]
\PYG{c+c1}{\PYGZsh{} minimum of 1,9,2,4 is 1}
\PYG{n+nb}{print}\PYG{p}{(}\PYG{n+nb}{min}\PYG{p}{(}\PYG{l+m+mi}{1}\PYG{p}{,}\PYG{l+m+mi}{9}\PYG{p}{,}\PYG{l+m+mi}{2}\PYG{p}{,}\PYG{l+m+mi}{4}\PYG{p}{)}\PYG{p}{)}
\end{sphinxVerbatim}

\end{sphinxuseclass}\end{sphinxVerbatimInput}
\begin{sphinxVerbatimOutput}

\begin{sphinxuseclass}{cell_output}
\begin{sphinxVerbatim}[commandchars=\\\{\}]
1
\end{sphinxVerbatim}

\end{sphinxuseclass}\end{sphinxVerbatimOutput}

\end{sphinxuseclass}
\begin{sphinxuseclass}{cell}\begin{sphinxVerbatimInput}

\begin{sphinxuseclass}{cell_input}
\begin{sphinxVerbatim}[commandchars=\\\{\}]
\PYG{c+c1}{\PYGZsh{} sum of 1,9,2,4 is 16}
\PYG{c+c1}{\PYGZsh{} numbers in square brackets or parenthesis}
\PYG{n+nb}{print}\PYG{p}{(}\PYG{n+nb}{sum}\PYG{p}{(}\PYG{p}{[}\PYG{l+m+mi}{1}\PYG{p}{,}\PYG{l+m+mi}{9}\PYG{p}{,}\PYG{l+m+mi}{2}\PYG{p}{,}\PYG{l+m+mi}{4}\PYG{p}{]}\PYG{p}{)}\PYG{p}{)}
\end{sphinxVerbatim}

\end{sphinxuseclass}\end{sphinxVerbatimInput}
\begin{sphinxVerbatimOutput}

\begin{sphinxuseclass}{cell_output}
\begin{sphinxVerbatim}[commandchars=\\\{\}]
16
\end{sphinxVerbatim}

\end{sphinxuseclass}\end{sphinxVerbatimOutput}

\end{sphinxuseclass}
\begin{sphinxuseclass}{cell}\begin{sphinxVerbatimInput}

\begin{sphinxuseclass}{cell_input}
\begin{sphinxVerbatim}[commandchars=\\\{\}]
\PYG{c+c1}{\PYGZsh{} 2 to the 3rd power}
\PYG{n+nb}{print}\PYG{p}{(}\PYG{n+nb}{pow}\PYG{p}{(}\PYG{l+m+mi}{2}\PYG{p}{,} \PYG{l+m+mi}{3}\PYG{p}{)}\PYG{p}{)}
\end{sphinxVerbatim}

\end{sphinxuseclass}\end{sphinxVerbatimInput}
\begin{sphinxVerbatimOutput}

\begin{sphinxuseclass}{cell_output}
\begin{sphinxVerbatim}[commandchars=\\\{\}]
8
\end{sphinxVerbatim}

\end{sphinxuseclass}\end{sphinxVerbatimOutput}

\end{sphinxuseclass}
\begin{sphinxuseclass}{cell}\begin{sphinxVerbatimInput}

\begin{sphinxuseclass}{cell_input}
\begin{sphinxVerbatim}[commandchars=\\\{\}]
\PYG{c+c1}{\PYGZsh{} rounding to the nearest thousandths}
\PYG{n+nb}{print}\PYG{p}{(}\PYG{n+nb}{round}\PYG{p}{(}\PYG{l+m+mf}{3.4678}\PYG{p}{,} \PYG{l+m+mi}{3}\PYG{p}{)}\PYG{p}{)}
\end{sphinxVerbatim}

\end{sphinxuseclass}\end{sphinxVerbatimInput}
\begin{sphinxVerbatimOutput}

\begin{sphinxuseclass}{cell_output}
\begin{sphinxVerbatim}[commandchars=\\\{\}]
3.468
\end{sphinxVerbatim}

\end{sphinxuseclass}\end{sphinxVerbatimOutput}

\end{sphinxuseclass}

\section{Math Module}
\label{\detokenize{numbers_arithmetic:math-module}}
\sphinxAtStartPar
Modules will be imported as needed, not loaded by default.
\begin{itemize}
\item {} 
\sphinxAtStartPar
This minimizes memory requirements and improves performance.

\item {} 
\sphinxAtStartPar
Modules are single Python files with a .py extension, which may contain functions and constants.

\end{itemize}

\sphinxAtStartPar
The Math module contains commonly used mathematical functions and constants, including trigonometric functions and the constant \(\pi\).
\begin{itemize}
\item {} 
\sphinxAtStartPar
The \sphinxstyleemphasis{math} module should be imported first.

\item {} 
\sphinxAtStartPar
\sphinxcode{\sphinxupquote{dir(math)}} returns a list of all functions and constants in the math module.

\item {} 
\sphinxAtStartPar
\sphinxcode{\sphinxupquote{help(math)}} provides more details, including explanations of functions and constants.

\item {} 
\sphinxAtStartPar
You do not need to memorize these functions. Whenever you need any of them, you can import them from the module.

\end{itemize}

\begin{sphinxuseclass}{cell}\begin{sphinxVerbatimInput}

\begin{sphinxuseclass}{cell_input}
\begin{sphinxVerbatim}[commandchars=\\\{\}]
\PYG{k+kn}{import} \PYG{n+nn}{math}
\end{sphinxVerbatim}

\end{sphinxuseclass}\end{sphinxVerbatimInput}

\end{sphinxuseclass}
\begin{sphinxuseclass}{cell}\begin{sphinxVerbatimInput}

\begin{sphinxuseclass}{cell_input}
\begin{sphinxVerbatim}[commandchars=\\\{\}]
\PYG{c+c1}{\PYGZsh{} list of constants and functions in math module }
\PYG{n+nb}{print}\PYG{p}{(}\PYG{n+nb}{dir}\PYG{p}{(}\PYG{n}{math}\PYG{p}{)}\PYG{p}{)}
\end{sphinxVerbatim}

\end{sphinxuseclass}\end{sphinxVerbatimInput}
\begin{sphinxVerbatimOutput}

\begin{sphinxuseclass}{cell_output}
\begin{sphinxVerbatim}[commandchars=\\\{\}]
[\PYGZsq{}\PYGZus{}\PYGZus{}doc\PYGZus{}\PYGZus{}\PYGZsq{}, \PYGZsq{}\PYGZus{}\PYGZus{}file\PYGZus{}\PYGZus{}\PYGZsq{}, \PYGZsq{}\PYGZus{}\PYGZus{}loader\PYGZus{}\PYGZus{}\PYGZsq{}, \PYGZsq{}\PYGZus{}\PYGZus{}name\PYGZus{}\PYGZus{}\PYGZsq{}, \PYGZsq{}\PYGZus{}\PYGZus{}package\PYGZus{}\PYGZus{}\PYGZsq{}, \PYGZsq{}\PYGZus{}\PYGZus{}spec\PYGZus{}\PYGZus{}\PYGZsq{}, \PYGZsq{}acos\PYGZsq{}, \PYGZsq{}acosh\PYGZsq{}, \PYGZsq{}asin\PYGZsq{}, \PYGZsq{}asinh\PYGZsq{}, \PYGZsq{}atan\PYGZsq{}, \PYGZsq{}atan2\PYGZsq{}, \PYGZsq{}atanh\PYGZsq{}, \PYGZsq{}cbrt\PYGZsq{}, \PYGZsq{}ceil\PYGZsq{}, \PYGZsq{}comb\PYGZsq{}, \PYGZsq{}copysign\PYGZsq{}, \PYGZsq{}cos\PYGZsq{}, \PYGZsq{}cosh\PYGZsq{}, \PYGZsq{}degrees\PYGZsq{}, \PYGZsq{}dist\PYGZsq{}, \PYGZsq{}e\PYGZsq{}, \PYGZsq{}erf\PYGZsq{}, \PYGZsq{}erfc\PYGZsq{}, \PYGZsq{}exp\PYGZsq{}, \PYGZsq{}exp2\PYGZsq{}, \PYGZsq{}expm1\PYGZsq{}, \PYGZsq{}fabs\PYGZsq{}, \PYGZsq{}factorial\PYGZsq{}, \PYGZsq{}floor\PYGZsq{}, \PYGZsq{}fmod\PYGZsq{}, \PYGZsq{}frexp\PYGZsq{}, \PYGZsq{}fsum\PYGZsq{}, \PYGZsq{}gamma\PYGZsq{}, \PYGZsq{}gcd\PYGZsq{}, \PYGZsq{}hypot\PYGZsq{}, \PYGZsq{}inf\PYGZsq{}, \PYGZsq{}isclose\PYGZsq{}, \PYGZsq{}isfinite\PYGZsq{}, \PYGZsq{}isinf\PYGZsq{}, \PYGZsq{}isnan\PYGZsq{}, \PYGZsq{}isqrt\PYGZsq{}, \PYGZsq{}lcm\PYGZsq{}, \PYGZsq{}ldexp\PYGZsq{}, \PYGZsq{}lgamma\PYGZsq{}, \PYGZsq{}log\PYGZsq{}, \PYGZsq{}log10\PYGZsq{}, \PYGZsq{}log1p\PYGZsq{}, \PYGZsq{}log2\PYGZsq{}, \PYGZsq{}modf\PYGZsq{}, \PYGZsq{}nan\PYGZsq{}, \PYGZsq{}nextafter\PYGZsq{}, \PYGZsq{}perm\PYGZsq{}, \PYGZsq{}pi\PYGZsq{}, \PYGZsq{}pow\PYGZsq{}, \PYGZsq{}prod\PYGZsq{}, \PYGZsq{}radians\PYGZsq{}, \PYGZsq{}remainder\PYGZsq{}, \PYGZsq{}sin\PYGZsq{}, \PYGZsq{}sinh\PYGZsq{}, \PYGZsq{}sqrt\PYGZsq{}, \PYGZsq{}tan\PYGZsq{}, \PYGZsq{}tanh\PYGZsq{}, \PYGZsq{}tau\PYGZsq{}, \PYGZsq{}trunc\PYGZsq{}, \PYGZsq{}ulp\PYGZsq{}]
\end{sphinxVerbatim}

\end{sphinxuseclass}\end{sphinxVerbatimOutput}

\end{sphinxuseclass}
\begin{sphinxuseclass}{cell}\begin{sphinxVerbatimInput}

\begin{sphinxuseclass}{cell_input}
\begin{sphinxVerbatim}[commandchars=\\\{\}]
\PYG{c+c1}{\PYGZsh{} sin(30 radians)}
\PYG{n}{math}\PYG{o}{.}\PYG{n}{sin}\PYG{p}{(}\PYG{l+m+mi}{30}\PYG{p}{)}
\end{sphinxVerbatim}

\end{sphinxuseclass}\end{sphinxVerbatimInput}
\begin{sphinxVerbatimOutput}

\begin{sphinxuseclass}{cell_output}
\begin{sphinxVerbatim}[commandchars=\\\{\}]
\PYGZhy{}0.9880316240928618
\end{sphinxVerbatim}

\end{sphinxuseclass}\end{sphinxVerbatimOutput}

\end{sphinxuseclass}
\begin{sphinxuseclass}{cell}\begin{sphinxVerbatimInput}

\begin{sphinxuseclass}{cell_input}
\begin{sphinxVerbatim}[commandchars=\\\{\}]
\PYG{c+c1}{\PYGZsh{} square root}
\PYG{n}{math}\PYG{o}{.}\PYG{n}{sqrt}\PYG{p}{(}\PYG{l+m+mi}{49}\PYG{p}{)}
\end{sphinxVerbatim}

\end{sphinxuseclass}\end{sphinxVerbatimInput}
\begin{sphinxVerbatimOutput}

\begin{sphinxuseclass}{cell_output}
\begin{sphinxVerbatim}[commandchars=\\\{\}]
7.0
\end{sphinxVerbatim}

\end{sphinxuseclass}\end{sphinxVerbatimOutput}

\end{sphinxuseclass}
\begin{sphinxuseclass}{cell}\begin{sphinxVerbatimInput}

\begin{sphinxuseclass}{cell_input}
\begin{sphinxVerbatim}[commandchars=\\\{\}]
\PYG{c+c1}{\PYGZsh{} converts degrees to radians }
\PYG{n}{math}\PYG{o}{.}\PYG{n}{radians}\PYG{p}{(}\PYG{l+m+mi}{180}\PYG{p}{)}
\end{sphinxVerbatim}

\end{sphinxuseclass}\end{sphinxVerbatimInput}
\begin{sphinxVerbatimOutput}

\begin{sphinxuseclass}{cell_output}
\begin{sphinxVerbatim}[commandchars=\\\{\}]
3.141592653589793
\end{sphinxVerbatim}

\end{sphinxuseclass}\end{sphinxVerbatimOutput}

\end{sphinxuseclass}
\begin{sphinxuseclass}{cell}\begin{sphinxVerbatimInput}

\begin{sphinxuseclass}{cell_input}
\begin{sphinxVerbatim}[commandchars=\\\{\}]
\PYG{c+c1}{\PYGZsh{} log of 100 in base 10}
\PYG{n}{math}\PYG{o}{.}\PYG{n}{log10}\PYG{p}{(}\PYG{l+m+mi}{100}\PYG{p}{)}
\end{sphinxVerbatim}

\end{sphinxuseclass}\end{sphinxVerbatimInput}
\begin{sphinxVerbatimOutput}

\begin{sphinxuseclass}{cell_output}
\begin{sphinxVerbatim}[commandchars=\\\{\}]
2.0
\end{sphinxVerbatim}

\end{sphinxuseclass}\end{sphinxVerbatimOutput}

\end{sphinxuseclass}

\section{Combine Strings and Numbers}
\label{\detokenize{numbers_arithmetic:combine-strings-and-numbers}}\begin{itemize}
\item {} 
\sphinxAtStartPar
There are different ways of combining strings and numbers.

\item {} 
\sphinxAtStartPar
One easy way is to convert numbers to a string using the \sphinxstyleemphasis{str()} function and then concatenate strings using the \sphinxstyleemphasis{+} operator.

\end{itemize}

\begin{sphinxuseclass}{cell}\begin{sphinxVerbatimInput}

\begin{sphinxuseclass}{cell_input}
\begin{sphinxVerbatim}[commandchars=\\\{\}]
\PYG{n}{x} \PYG{o}{=} \PYG{l+m+mi}{5}
\PYG{n}{y} \PYG{o}{=} \PYG{l+s+s1}{\PYGZsq{}}\PYG{l+s+s1}{dollars}\PYG{l+s+s1}{\PYGZsq{}}
\end{sphinxVerbatim}

\end{sphinxuseclass}\end{sphinxVerbatimInput}

\end{sphinxuseclass}
\begin{sphinxVerbatim}[commandchars=\\\{\}]
\PYG{c+c1}{\PYGZsh{} ERROR: int + str}
\PYG{n+nb}{print}\PYG{p}{(}\PYG{n}{x}\PYG{o}{+}\PYG{n}{y}\PYG{p}{)}
\end{sphinxVerbatim}

\begin{sphinxuseclass}{cell}\begin{sphinxVerbatimInput}

\begin{sphinxuseclass}{cell_input}
\begin{sphinxVerbatim}[commandchars=\\\{\}]
\PYG{c+c1}{\PYGZsh{} convert x into a string: new value is \PYGZsq{}5\PYGZsq{}}
\PYG{c+c1}{\PYGZsh{} concatenate three strings: str(x), \PYGZsq{} \PYGZsq{}, y}
\PYG{n}{new\PYGZus{}str} \PYG{o}{=} \PYG{n+nb}{str}\PYG{p}{(}\PYG{n}{x}\PYG{p}{)}\PYG{o}{+}\PYG{l+s+s1}{\PYGZsq{}}\PYG{l+s+s1}{ }\PYG{l+s+s1}{\PYGZsq{}}\PYG{o}{+}\PYG{n}{y}
\PYG{n+nb}{print}\PYG{p}{(}\PYG{n}{new\PYGZus{}str}\PYG{p}{)}
\end{sphinxVerbatim}

\end{sphinxuseclass}\end{sphinxVerbatimInput}
\begin{sphinxVerbatimOutput}

\begin{sphinxuseclass}{cell_output}
\begin{sphinxVerbatim}[commandchars=\\\{\}]
5 dollars
\end{sphinxVerbatim}

\end{sphinxuseclass}\end{sphinxVerbatimOutput}

\end{sphinxuseclass}
\sphinxstepscope


\section{Numbers Debugging}
\label{\detokenize{numbers_arithmetic_debug:numbers-debugging}}\label{\detokenize{numbers_arithmetic_debug::doc}}\begin{itemize}
\item {} 
\sphinxAtStartPar
Each of the following short code contains one or more bugs.     

\item {} 
\sphinxAtStartPar
Please identify and correct these bugs.

\item {} 
\sphinxAtStartPar
Provide an explanation for your answer.

\end{itemize}


\subsection{Question\sphinxhyphen{}1}
\label{\detokenize{numbers_arithmetic_debug:question-1}}
\begin{sphinxVerbatim}[commandchars=\\\{\}]
\PYG{n}{x} \PYG{o}{=} \PYG{l+s+s1}{\PYGZsq{}}\PYG{l+s+s1}{five}\PYG{l+s+s1}{\PYGZsq{}}
\PYG{n}{y} \PYG{o}{=} \PYG{n+nb}{int}\PYG{p}{(}\PYG{n}{x}\PYG{p}{)}
\PYG{n+nb}{print}\PYG{p}{(}\PYG{n}{y}\PYG{p}{)}
\end{sphinxVerbatim}

\begin{sphinxadmonition}{note}{Solution}

\sphinxAtStartPar
‘five’ cannot be converted to an integer. It should be given in digit form.
\end{sphinxadmonition}


\subsection{Question\sphinxhyphen{}2}
\label{\detokenize{numbers_arithmetic_debug:question-2}}
\begin{sphinxVerbatim}[commandchars=\\\{\}]
\PYG{n}{x} \PYG{o}{=} \PYG{l+s+s1}{\PYGZsq{}}\PYG{l+s+s1}{5.07}\PYG{l+s+s1}{\PYGZsq{}}
\PYG{n}{y} \PYG{o}{=} \PYG{n+nb}{int}\PYG{p}{(}\PYG{n}{x}\PYG{p}{)}
\PYG{n+nb}{print}\PYG{p}{(}\PYG{n}{y}\PYG{p}{)}
\end{sphinxVerbatim}

\begin{sphinxadmonition}{note}{Solution}

\sphinxAtStartPar
x is a string representation of a decimal number. It cannot be converted to an integer.
\end{sphinxadmonition}


\subsection{Question\sphinxhyphen{}3}
\label{\detokenize{numbers_arithmetic_debug:question-3}}
\begin{sphinxVerbatim}[commandchars=\\\{\}]
\PYG{n}{x} \PYG{o}{=} \PYG{l+m+mi}{5}
\PYG{n+nb}{print}\PYG{p}{(}\PYG{l+m+mi}{10}\PYG{o}{/}\PYG{p}{(}\PYG{l+m+mi}{5}\PYG{o}{\PYGZhy{}}\PYG{n}{x}\PYG{p}{)}\PYG{p}{)}
\end{sphinxVerbatim}

\begin{sphinxadmonition}{note}{Solution}

\sphinxAtStartPar
Division by zero.
\end{sphinxadmonition}


\subsection{Question\sphinxhyphen{}4}
\label{\detokenize{numbers_arithmetic_debug:question-4}}
\begin{sphinxVerbatim}[commandchars=\\\{\}]
\PYG{n+nb}{print}\PYG{p}{(}\PYG{n+nb}{sum}\PYG{p}{(}\PYG{l+m+mi}{4}\PYG{p}{,}\PYG{l+m+mi}{7}\PYG{p}{,}\PYG{l+m+mi}{9}\PYG{p}{)}\PYG{p}{)}
\end{sphinxVerbatim}

\begin{sphinxadmonition}{note}{Solution}

\sphinxAtStartPar
The numbers inside the sum() function can be enclosed in square brackets or parentheses.
\end{sphinxadmonition}


\subsection{Question\sphinxhyphen{}5}
\label{\detokenize{numbers_arithmetic_debug:question-5}}
\begin{sphinxVerbatim}[commandchars=\\\{\}]
\PYG{n+nb}{print}\PYG{p}{(}\PYG{n}{sqrt}\PYG{p}{(}\PYG{l+m+mi}{25}\PYG{p}{)}\PYG{p}{)}
\end{sphinxVerbatim}

\begin{sphinxadmonition}{note}{Solution}

\sphinxAtStartPar
sqrt() is not a built\sphinxhyphen{}in function; it needs to be imported.
\end{sphinxadmonition}


\subsection{Question\sphinxhyphen{}6}
\label{\detokenize{numbers_arithmetic_debug:question-6}}
\begin{sphinxVerbatim}[commandchars=\\\{\}]
\PYG{n+nb}{print}\PYG{p}{(}\PYG{n}{math}\PYG{o}{.}\PYG{n}{sqrt}\PYG{p}{(}\PYG{l+m+mi}{25}\PYG{p}{)}\PYG{p}{)}
\end{sphinxVerbatim}

\begin{sphinxadmonition}{note}{Solution}

\sphinxAtStartPar
The math module must be imported first.
\end{sphinxadmonition}


\subsection{Question\sphinxhyphen{}7}
\label{\detokenize{numbers_arithmetic_debug:question-7}}
\begin{sphinxVerbatim}[commandchars=\\\{\}]
\PYG{n}{x} \PYG{o}{=} \PYG{l+s+s1}{\PYGZsq{}}\PYG{l+s+s1}{5}\PYG{l+s+s1}{\PYGZsq{}}
\PYG{n}{y} \PYG{o}{=} \PYG{l+m+mi}{10}
\PYG{n+nb}{print}\PYG{p}{(}\PYG{n}{x}\PYG{o}{+}\PYG{n}{y}\PYG{p}{)}
\end{sphinxVerbatim}

\begin{sphinxadmonition}{note}{Solution}

\sphinxAtStartPar
Numbers and strings cannot be added or concatenated.
\end{sphinxadmonition}


\subsection{Question\sphinxhyphen{}8}
\label{\detokenize{numbers_arithmetic_debug:question-8}}
\begin{sphinxVerbatim}[commandchars=\\\{\}]
\PYG{n}{x} \PYG{o}{=} \PYG{l+m+mi}{1}\PYG{p}{,}\PYG{l+m+mi}{234}
\PYG{n}{y} \PYG{o}{=} \PYG{l+m+mi}{10}
\PYG{n+nb}{print}\PYG{p}{(}\PYG{n}{x}\PYG{o}{+}\PYG{n}{y}\PYG{p}{)}
\end{sphinxVerbatim}

\begin{sphinxadmonition}{note}{Solution}

\sphinxAtStartPar
In Python, underscores are used instead of commas for large numbers.
\end{sphinxadmonition}


\subsection{Question\sphinxhyphen{}9}
\label{\detokenize{numbers_arithmetic_debug:question-9}}
\begin{sphinxVerbatim}[commandchars=\\\{\}]
\PYG{n}{x} \PYG{o}{=} \PYG{l+m+mi}{3}
\PYG{n}{y} \PYG{o}{=} \PYG{l+m+mi}{10}
\PYG{n+nb}{print}\PYG{p}{(}\PYG{n}{xy}\PYG{p}{)}
\end{sphinxVerbatim}

\begin{sphinxadmonition}{note}{Solution}

\sphinxAtStartPar
In Python, multiplication is performed using the \sphinxcode{\sphinxupquote{*}} operator. For example, to multiply x and y, you would write \sphinxcode{\sphinxupquote{x * y}}.
\end{sphinxadmonition}

\sphinxstepscope


\section{Numbers Output}
\label{\detokenize{numbers_arithmetic_output:numbers-output}}\label{\detokenize{numbers_arithmetic_output::doc}}\begin{itemize}
\item {} 
\sphinxAtStartPar
Find the output of the following code.

\item {} 
\sphinxAtStartPar
Please don’t run the code before giving your answer.     

\end{itemize}


\subsection{Question\sphinxhyphen{}1}
\label{\detokenize{numbers_arithmetic_output:question-1}}
\begin{sphinxuseclass}{cell}
\begin{sphinxuseclass}{tag_hide-output}\begin{sphinxVerbatimInput}

\begin{sphinxuseclass}{cell_input}
\begin{sphinxVerbatim}[commandchars=\\\{\}]
\PYG{n+nb}{print}\PYG{p}{(}\PYG{n+nb}{type}\PYG{p}{(}\PYG{l+m+mi}{10}\PYG{o}{/}\PYG{l+m+mi}{2}\PYG{p}{)}\PYG{p}{)}
\end{sphinxVerbatim}

\end{sphinxuseclass}\end{sphinxVerbatimInput}

\end{sphinxuseclass}
\end{sphinxuseclass}

\subsection{Question\sphinxhyphen{}2}
\label{\detokenize{numbers_arithmetic_output:question-2}}
\begin{sphinxuseclass}{cell}
\begin{sphinxuseclass}{tag_hide-output}\begin{sphinxVerbatimInput}

\begin{sphinxuseclass}{cell_input}
\begin{sphinxVerbatim}[commandchars=\\\{\}]
\PYG{n+nb}{print}\PYG{p}{(}\PYG{n+nb}{type}\PYG{p}{(}\PYG{l+m+mi}{10}\PYG{o}{/}\PYG{o}{/}\PYG{l+m+mi}{2}\PYG{p}{)}\PYG{p}{)}
\end{sphinxVerbatim}

\end{sphinxuseclass}\end{sphinxVerbatimInput}

\end{sphinxuseclass}
\end{sphinxuseclass}

\subsection{Question\sphinxhyphen{}3}
\label{\detokenize{numbers_arithmetic_output:question-3}}
\begin{sphinxuseclass}{cell}
\begin{sphinxuseclass}{tag_hide-output}\begin{sphinxVerbatimInput}

\begin{sphinxuseclass}{cell_input}
\begin{sphinxVerbatim}[commandchars=\\\{\}]
\PYG{n+nb}{print}\PYG{p}{(}\PYG{l+m+mi}{15}\PYG{o}{\PYGZpc{}}\PYG{k}{4})
\end{sphinxVerbatim}

\end{sphinxuseclass}\end{sphinxVerbatimInput}

\end{sphinxuseclass}
\end{sphinxuseclass}

\subsection{Question\sphinxhyphen{}4}
\label{\detokenize{numbers_arithmetic_output:question-4}}
\begin{sphinxuseclass}{cell}
\begin{sphinxuseclass}{tag_hide-output}\begin{sphinxVerbatimInput}

\begin{sphinxuseclass}{cell_input}
\begin{sphinxVerbatim}[commandchars=\\\{\}]
\PYG{n+nb}{print}\PYG{p}{(}\PYG{l+m+mi}{5}\PYG{o}{+}\PYG{l+m+mi}{12}\PYG{o}{/}\PYG{l+m+mi}{3}\PYG{p}{)}
\end{sphinxVerbatim}

\end{sphinxuseclass}\end{sphinxVerbatimInput}

\end{sphinxuseclass}
\end{sphinxuseclass}

\subsection{Question\sphinxhyphen{}5}
\label{\detokenize{numbers_arithmetic_output:question-5}}
\begin{sphinxuseclass}{cell}
\begin{sphinxuseclass}{tag_hide-output}\begin{sphinxVerbatimInput}

\begin{sphinxuseclass}{cell_input}
\begin{sphinxVerbatim}[commandchars=\\\{\}]
\PYG{n+nb}{print}\PYG{p}{(}\PYG{l+m+mi}{15}\PYG{o}{/}\PYG{o}{/}\PYG{l+m+mi}{4}\PYG{p}{)}
\end{sphinxVerbatim}

\end{sphinxuseclass}\end{sphinxVerbatimInput}

\end{sphinxuseclass}
\end{sphinxuseclass}

\subsection{Question\sphinxhyphen{}6}
\label{\detokenize{numbers_arithmetic_output:question-6}}
\begin{sphinxuseclass}{cell}
\begin{sphinxuseclass}{tag_hide-output}\begin{sphinxVerbatimInput}

\begin{sphinxuseclass}{cell_input}
\begin{sphinxVerbatim}[commandchars=\\\{\}]
\PYG{n+nb}{print}\PYG{p}{(}\PYG{p}{(}\PYG{l+m+mi}{23}\PYG{o}{/}\PYG{o}{/}\PYG{l+m+mi}{3}\PYG{p}{)}\PYG{o}{\PYGZpc{}}\PYG{k}{4})
\end{sphinxVerbatim}

\end{sphinxuseclass}\end{sphinxVerbatimInput}

\end{sphinxuseclass}
\end{sphinxuseclass}

\subsection{Question\sphinxhyphen{}7}
\label{\detokenize{numbers_arithmetic_output:question-7}}
\begin{sphinxuseclass}{cell}
\begin{sphinxuseclass}{tag_hide-output}\begin{sphinxVerbatimInput}

\begin{sphinxuseclass}{cell_input}
\begin{sphinxVerbatim}[commandchars=\\\{\}]
\PYG{n+nb}{print}\PYG{p}{(}\PYG{l+m+mi}{3}\PYG{o}{*}\PYG{o}{*}\PYG{l+m+mi}{4}\PYG{p}{)}
\end{sphinxVerbatim}

\end{sphinxuseclass}\end{sphinxVerbatimInput}

\end{sphinxuseclass}
\end{sphinxuseclass}

\subsection{Question\sphinxhyphen{}8}
\label{\detokenize{numbers_arithmetic_output:question-8}}
\begin{sphinxuseclass}{cell}
\begin{sphinxuseclass}{tag_hide-output}\begin{sphinxVerbatimInput}

\begin{sphinxuseclass}{cell_input}
\begin{sphinxVerbatim}[commandchars=\\\{\}]
\PYG{k+kn}{import} \PYG{n+nn}{statistics}
\PYG{n}{average} \PYG{o}{=} \PYG{n}{statistics}\PYG{o}{.}\PYG{n}{mean}\PYG{p}{(}\PYG{p}{[}\PYG{l+m+mi}{1}\PYG{p}{,}\PYG{l+m+mi}{5}\PYG{p}{,}\PYG{l+m+mi}{12}\PYG{p}{]}\PYG{p}{)}
\PYG{n+nb}{print}\PYG{p}{(}\PYG{n}{average}\PYG{p}{)}
\end{sphinxVerbatim}

\end{sphinxuseclass}\end{sphinxVerbatimInput}

\end{sphinxuseclass}
\end{sphinxuseclass}

\subsection{Question\sphinxhyphen{}9}
\label{\detokenize{numbers_arithmetic_output:question-9}}
\begin{sphinxuseclass}{cell}
\begin{sphinxuseclass}{tag_hide-output}\begin{sphinxVerbatimInput}

\begin{sphinxuseclass}{cell_input}
\begin{sphinxVerbatim}[commandchars=\\\{\}]
\PYG{k+kn}{import} \PYG{n+nn}{numpy}
\PYG{n}{average} \PYG{o}{=} \PYG{n}{numpy}\PYG{o}{.}\PYG{n}{mean}\PYG{p}{(}\PYG{p}{[}\PYG{l+m+mi}{5}\PYG{p}{,}\PYG{l+m+mi}{10}\PYG{p}{,}\PYG{l+m+mi}{9}\PYG{p}{]}\PYG{p}{)}
\PYG{n+nb}{print}\PYG{p}{(}\PYG{n}{average}\PYG{p}{)}
\end{sphinxVerbatim}

\end{sphinxuseclass}\end{sphinxVerbatimInput}

\end{sphinxuseclass}
\end{sphinxuseclass}

\subsection{Question\sphinxhyphen{}10}
\label{\detokenize{numbers_arithmetic_output:question-10}}
\begin{sphinxuseclass}{cell}
\begin{sphinxuseclass}{tag_hide-output}\begin{sphinxVerbatimInput}

\begin{sphinxuseclass}{cell_input}
\begin{sphinxVerbatim}[commandchars=\\\{\}]
\PYG{n}{rd} \PYG{o}{=} \PYG{n+nb}{round}\PYG{p}{(}\PYG{l+m+mf}{12.345}\PYG{p}{,}\PYG{l+m+mi}{2}\PYG{p}{)}
\PYG{n+nb}{print}\PYG{p}{(}\PYG{n}{rd}\PYG{p}{)}
\end{sphinxVerbatim}

\end{sphinxuseclass}\end{sphinxVerbatimInput}

\end{sphinxuseclass}
\end{sphinxuseclass}

\subsection{Question\sphinxhyphen{}11}
\label{\detokenize{numbers_arithmetic_output:question-11}}
\begin{sphinxuseclass}{cell}
\begin{sphinxuseclass}{tag_hide-output}\begin{sphinxVerbatimInput}

\begin{sphinxuseclass}{cell_input}
\begin{sphinxVerbatim}[commandchars=\\\{\}]
\PYG{n}{rd} \PYG{o}{=} \PYG{n+nb}{round}\PYG{p}{(}\PYG{l+m+mf}{12.367}\PYG{p}{,}\PYG{l+m+mi}{1}\PYG{p}{)}
\PYG{n+nb}{print}\PYG{p}{(}\PYG{n}{rd}\PYG{p}{)}
\end{sphinxVerbatim}

\end{sphinxuseclass}\end{sphinxVerbatimInput}

\end{sphinxuseclass}
\end{sphinxuseclass}

\subsection{Question\sphinxhyphen{}12}
\label{\detokenize{numbers_arithmetic_output:question-12}}
\begin{sphinxuseclass}{cell}
\begin{sphinxuseclass}{tag_hide-output}\begin{sphinxVerbatimInput}

\begin{sphinxuseclass}{cell_input}
\begin{sphinxVerbatim}[commandchars=\\\{\}]
\PYG{n}{x} \PYG{o}{=} \PYG{l+m+mi}{10}
\PYG{n}{x} \PYG{o}{=} \PYG{n}{x}\PYG{o}{+}\PYG{l+m+mi}{1}
\PYG{n+nb}{print}\PYG{p}{(}\PYG{n}{x}\PYG{p}{)}
\end{sphinxVerbatim}

\end{sphinxuseclass}\end{sphinxVerbatimInput}

\end{sphinxuseclass}
\end{sphinxuseclass}

\subsection{Question\sphinxhyphen{}13}
\label{\detokenize{numbers_arithmetic_output:question-13}}
\begin{sphinxuseclass}{cell}
\begin{sphinxuseclass}{tag_hide-output}\begin{sphinxVerbatimInput}

\begin{sphinxuseclass}{cell_input}
\begin{sphinxVerbatim}[commandchars=\\\{\}]
\PYG{n}{x} \PYG{o}{=} \PYG{l+m+mi}{20}
\PYG{n}{x} \PYG{o}{+}\PYG{o}{=} \PYG{l+m+mi}{1}
\PYG{n+nb}{print}\PYG{p}{(}\PYG{n}{x}\PYG{p}{)}
\end{sphinxVerbatim}

\end{sphinxuseclass}\end{sphinxVerbatimInput}

\end{sphinxuseclass}
\end{sphinxuseclass}

\subsection{Question\sphinxhyphen{}14}
\label{\detokenize{numbers_arithmetic_output:question-14}}
\begin{sphinxuseclass}{cell}
\begin{sphinxuseclass}{tag_hide-output}\begin{sphinxVerbatimInput}

\begin{sphinxuseclass}{cell_input}
\begin{sphinxVerbatim}[commandchars=\\\{\}]
\PYG{n}{x} \PYG{o}{=} \PYG{l+m+mi}{10}
\PYG{n}{x} \PYG{o}{=} \PYG{n}{x}\PYG{o}{+}\PYG{l+m+mi}{1}
\PYG{n}{x} \PYG{o}{+}\PYG{o}{=} \PYG{l+m+mi}{1}
\PYG{n+nb}{print}\PYG{p}{(}\PYG{n}{x}\PYG{p}{)}
\end{sphinxVerbatim}

\end{sphinxuseclass}\end{sphinxVerbatimInput}

\end{sphinxuseclass}
\end{sphinxuseclass}

\subsection{Question\sphinxhyphen{}15}
\label{\detokenize{numbers_arithmetic_output:question-15}}
\begin{sphinxuseclass}{cell}
\begin{sphinxuseclass}{tag_hide-output}\begin{sphinxVerbatimInput}

\begin{sphinxuseclass}{cell_input}
\begin{sphinxVerbatim}[commandchars=\\\{\}]
\PYG{n}{x} \PYG{o}{=} \PYG{l+m+mi}{10}
\PYG{n}{x} \PYG{o}{\PYGZhy{}}\PYG{o}{=} \PYG{l+m+mi}{2}
\PYG{n}{x} \PYG{o}{*}\PYG{o}{=} \PYG{l+m+mi}{5}
\PYG{n}{x} \PYG{o}{=} \PYG{n}{x}\PYG{o}{+}\PYG{l+m+mi}{1}
\PYG{n+nb}{print}\PYG{p}{(}\PYG{n}{x}\PYG{p}{)}
\end{sphinxVerbatim}

\end{sphinxuseclass}\end{sphinxVerbatimInput}

\end{sphinxuseclass}
\end{sphinxuseclass}

\subsection{Question\sphinxhyphen{}16}
\label{\detokenize{numbers_arithmetic_output:question-16}}
\begin{sphinxuseclass}{cell}
\begin{sphinxuseclass}{tag_hide-output}\begin{sphinxVerbatimInput}

\begin{sphinxuseclass}{cell_input}
\begin{sphinxVerbatim}[commandchars=\\\{\}]
\PYG{n}{x} \PYG{o}{=} \PYG{l+m+mi}{10}
\PYG{n}{x} \PYG{o}{\PYGZhy{}}\PYG{o}{=} \PYG{l+m+mi}{7}
\PYG{n+nb}{print}\PYG{p}{(}\PYG{n}{x}\PYG{p}{)}
\PYG{n}{x} \PYG{o}{*}\PYG{o}{=} \PYG{l+m+mi}{2}
\PYG{n+nb}{print}\PYG{p}{(}\PYG{n}{x}\PYG{p}{)}
\end{sphinxVerbatim}

\end{sphinxuseclass}\end{sphinxVerbatimInput}

\end{sphinxuseclass}
\end{sphinxuseclass}

\subsection{Question\sphinxhyphen{}17}
\label{\detokenize{numbers_arithmetic_output:question-17}}
\begin{sphinxuseclass}{cell}
\begin{sphinxuseclass}{tag_hide-output}\begin{sphinxVerbatimInput}

\begin{sphinxuseclass}{cell_input}
\begin{sphinxVerbatim}[commandchars=\\\{\}]
\PYG{n}{x} \PYG{o}{=} \PYG{l+m+mi}{0}
\PYG{n+nb}{print}\PYG{p}{(}\PYG{n}{x}\PYG{o}{*}\PYG{l+s+s1}{\PYGZsq{}}\PYG{l+s+s1}{\PYGZhy{}}\PYG{l+s+s1}{\PYGZsq{}}\PYG{o}{+}\PYG{l+s+s1}{\PYGZsq{}}\PYG{l+s+s1}{\PYGZdl{}}\PYG{l+s+s1}{\PYGZsq{}}\PYG{o}{+}\PYG{n}{x}\PYG{o}{*}\PYG{l+s+s1}{\PYGZsq{}}\PYG{l+s+s1}{\PYGZhy{}}\PYG{l+s+s1}{\PYGZsq{}}\PYG{p}{)}
\PYG{n}{x} \PYG{o}{+}\PYG{o}{=} \PYG{l+m+mi}{1}
\PYG{n+nb}{print}\PYG{p}{(}\PYG{n}{x}\PYG{o}{*}\PYG{l+s+s1}{\PYGZsq{}}\PYG{l+s+s1}{\PYGZhy{}}\PYG{l+s+s1}{\PYGZsq{}}\PYG{o}{+}\PYG{l+s+s1}{\PYGZsq{}}\PYG{l+s+s1}{\PYGZdl{}}\PYG{l+s+s1}{\PYGZsq{}}\PYG{o}{+}\PYG{n}{x}\PYG{o}{*}\PYG{l+s+s1}{\PYGZsq{}}\PYG{l+s+s1}{\PYGZhy{}}\PYG{l+s+s1}{\PYGZsq{}}\PYG{p}{)}
\PYG{n}{x} \PYG{o}{+}\PYG{o}{=} \PYG{l+m+mi}{1}
\PYG{n+nb}{print}\PYG{p}{(}\PYG{n}{x}\PYG{o}{*}\PYG{l+s+s1}{\PYGZsq{}}\PYG{l+s+s1}{\PYGZhy{}}\PYG{l+s+s1}{\PYGZsq{}}\PYG{o}{+}\PYG{l+s+s1}{\PYGZsq{}}\PYG{l+s+s1}{\PYGZdl{}}\PYG{l+s+s1}{\PYGZsq{}}\PYG{o}{+}\PYG{n}{x}\PYG{o}{*}\PYG{l+s+s1}{\PYGZsq{}}\PYG{l+s+s1}{\PYGZhy{}}\PYG{l+s+s1}{\PYGZsq{}}\PYG{p}{)}
\PYG{n}{x} \PYG{o}{\PYGZhy{}}\PYG{o}{=} \PYG{l+m+mi}{1}
\PYG{n+nb}{print}\PYG{p}{(}\PYG{n}{x}\PYG{o}{*}\PYG{l+s+s1}{\PYGZsq{}}\PYG{l+s+s1}{\PYGZhy{}}\PYG{l+s+s1}{\PYGZsq{}}\PYG{o}{+}\PYG{l+s+s1}{\PYGZsq{}}\PYG{l+s+s1}{\PYGZdl{}}\PYG{l+s+s1}{\PYGZsq{}}\PYG{o}{+}\PYG{n}{x}\PYG{o}{*}\PYG{l+s+s1}{\PYGZsq{}}\PYG{l+s+s1}{\PYGZhy{}}\PYG{l+s+s1}{\PYGZsq{}}\PYG{p}{)}
\PYG{n}{x} \PYG{o}{\PYGZhy{}}\PYG{o}{=} \PYG{l+m+mi}{1}
\PYG{n+nb}{print}\PYG{p}{(}\PYG{n}{x}\PYG{o}{*}\PYG{l+s+s1}{\PYGZsq{}}\PYG{l+s+s1}{\PYGZhy{}}\PYG{l+s+s1}{\PYGZsq{}}\PYG{o}{+}\PYG{l+s+s1}{\PYGZsq{}}\PYG{l+s+s1}{\PYGZdl{}}\PYG{l+s+s1}{\PYGZsq{}}\PYG{o}{+}\PYG{n}{x}\PYG{o}{*}\PYG{l+s+s1}{\PYGZsq{}}\PYG{l+s+s1}{\PYGZhy{}}\PYG{l+s+s1}{\PYGZsq{}}\PYG{p}{)}
\end{sphinxVerbatim}

\end{sphinxuseclass}\end{sphinxVerbatimInput}

\end{sphinxuseclass}
\end{sphinxuseclass}

\subsection{Question\sphinxhyphen{}18}
\label{\detokenize{numbers_arithmetic_output:question-18}}
\begin{sphinxuseclass}{cell}
\begin{sphinxuseclass}{tag_hide-output}\begin{sphinxVerbatimInput}

\begin{sphinxuseclass}{cell_input}
\begin{sphinxVerbatim}[commandchars=\\\{\}]
\PYG{n}{x} \PYG{o}{=} \PYG{l+s+s1}{\PYGZsq{}}\PYG{l+s+s1}{5.07}\PYG{l+s+s1}{\PYGZsq{}}
\PYG{n}{y} \PYG{o}{=} \PYG{n+nb}{float}\PYG{p}{(}\PYG{n+nb}{int}\PYG{p}{(}\PYG{n+nb}{float}\PYG{p}{(}\PYG{n}{x}\PYG{p}{)}\PYG{p}{)}\PYG{p}{)}
\PYG{n+nb}{print}\PYG{p}{(}\PYG{n}{y}\PYG{p}{)}
\end{sphinxVerbatim}

\end{sphinxuseclass}\end{sphinxVerbatimInput}

\end{sphinxuseclass}
\end{sphinxuseclass}

\subsection{Question\sphinxhyphen{}19}
\label{\detokenize{numbers_arithmetic_output:question-19}}
\begin{sphinxuseclass}{cell}
\begin{sphinxuseclass}{tag_hide-output}\begin{sphinxVerbatimInput}

\begin{sphinxuseclass}{cell_input}
\begin{sphinxVerbatim}[commandchars=\\\{\}]
\PYG{n}{x} \PYG{o}{=} \PYG{l+m+mi}{2}
\PYG{n}{y} \PYG{o}{=} \PYG{l+m+mi}{5}
\PYG{n+nb}{print}\PYG{p}{(}\PYG{n+nb}{str}\PYG{p}{(}\PYG{n}{x}\PYG{p}{)}\PYG{o}{+}\PYG{l+s+s1}{\PYGZsq{}}\PYG{l+s+s1}{ * }\PYG{l+s+s1}{\PYGZsq{}}\PYG{o}{+}\PYG{n+nb}{str}\PYG{p}{(}\PYG{n}{y}\PYG{p}{)}\PYG{o}{+}\PYG{l+s+s1}{\PYGZsq{}}\PYG{l+s+s1}{ = }\PYG{l+s+s1}{\PYGZsq{}}\PYG{o}{+}\PYG{n+nb}{str}\PYG{p}{(}\PYG{n}{x}\PYG{o}{*}\PYG{n}{y}\PYG{p}{)}\PYG{p}{)}
\end{sphinxVerbatim}

\end{sphinxuseclass}\end{sphinxVerbatimInput}

\end{sphinxuseclass}
\end{sphinxuseclass}
\sphinxstepscope


\section{Numbers Code}
\label{\detokenize{numbers_arithmetic_code:numbers-code}}\label{\detokenize{numbers_arithmetic_code::doc}}
\sphinxAtStartPar
Please solve the following questions using Python code.  


\subsection{Question\sphinxhyphen{}1}
\label{\detokenize{numbers_arithmetic_code:question-1}}
\sphinxAtStartPar
Write a program that prompts the user for their weight in pounds.
\begin{itemize}
\item {} 
\sphinxAtStartPar
Convert the weight to kilograms.

\item {} 
\sphinxAtStartPar
Print the converted weight.

\end{itemize}

\begin{sphinxadmonition}{note}{Solution}

\begin{sphinxVerbatim}[commandchars=\\\{\}]
\PYG{n}{pound} \PYG{o}{=} \PYG{n+nb}{float}\PYG{p}{(}\PYG{n+nb}{input}\PYG{p}{(}\PYG{l+s+s1}{\PYGZsq{}}\PYG{l+s+s1}{Enter your weight in pounds: }\PYG{l+s+s1}{\PYGZsq{}}\PYG{p}{)}\PYG{p}{)}

\PYG{n}{kilogram} \PYG{o}{=} \PYG{n}{pound}\PYG{o}{*}\PYG{l+m+mf}{0.46}

\PYG{n+nb}{print}\PYG{p}{(}\PYG{l+s+s1}{\PYGZsq{}}\PYG{l+s+s1}{Your weight is}\PYG{l+s+s1}{\PYGZsq{}}\PYG{p}{,} \PYG{n}{kilogram}\PYG{p}{,} \PYG{l+s+s1}{\PYGZsq{}}\PYG{l+s+s1}{kg.}\PYG{l+s+s1}{\PYGZsq{}}\PYG{p}{)}
\end{sphinxVerbatim}
\end{sphinxadmonition}


\subsection{Question\sphinxhyphen{}2}
\label{\detokenize{numbers_arithmetic_code:question-2}}
\sphinxAtStartPar
Find the area of a circle with a radius of 5.
\begin{itemize}
\item {} 
\sphinxAtStartPar
Round the area to the nearest hundredth.

\item {} 
\sphinxAtStartPar
Area = \(\pi r^2\)

\end{itemize}

\sphinxAtStartPar
\sphinxstylestrong{Solution}


\subsection{Question\sphinxhyphen{}3}
\label{\detokenize{numbers_arithmetic_code:question-3}}
\sphinxAtStartPar
Write a program that prompts the user for a Fahrenheit temperature, converts the temperature to Kelvin, and prints out the converted temperature.
\begin{itemize}
\item {} 
\sphinxAtStartPar
Hint: \(\displaystyle K = \frac{F-32}{1.8}+273 \)

\end{itemize}

\begin{sphinxadmonition}{note}{Solution}

\begin{sphinxVerbatim}[commandchars=\\\{\}]
\PYG{n}{fahrenheit} \PYG{o}{=} \PYG{n+nb}{float}\PYG{p}{(}\PYG{n+nb}{input}\PYG{p}{(}\PYG{l+s+s1}{\PYGZsq{}}\PYG{l+s+s1}{Enter the Fahrenheit Temperature: }\PYG{l+s+s1}{\PYGZsq{}}\PYG{p}{)}\PYG{p}{)}

\PYG{n}{kelvin} \PYG{o}{=} \PYG{p}{(}\PYG{n}{fahrenheit}\PYG{o}{\PYGZhy{}}\PYG{l+m+mi}{32}\PYG{p}{)}\PYG{o}{/}\PYG{l+m+mf}{1.8}\PYG{o}{+}\PYG{l+m+mi}{273}   \PYG{c+c1}{\PYGZsh{} convert to kelvin}

\PYG{n+nb}{print}\PYG{p}{(}\PYG{l+s+s1}{\PYGZsq{}}\PYG{l+s+s1}{Kelvin Temparature is}\PYG{l+s+s1}{\PYGZsq{}}\PYG{p}{,} \PYG{n}{kelvin}\PYG{p}{)}
\end{sphinxVerbatim}
\end{sphinxadmonition}


\subsection{Question\sphinxhyphen{}4}
\label{\detokenize{numbers_arithmetic_code:question-4}}
\sphinxAtStartPar
Write a program that prompts the user for height (cm) and weight (kg), computes the body mass index, and prints it.
\begin{itemize}
\item {} 
\sphinxAtStartPar
Hint: \( \displaystyle BMI = \frac{weight}{height^2}\)

\end{itemize}

\begin{sphinxadmonition}{note}{Solution}

\begin{sphinxVerbatim}[commandchars=\\\{\}]
\PYG{n}{height} \PYG{o}{=} \PYG{n+nb}{int} \PYG{p}{(} \PYG{n+nb}{input}\PYG{p}{(}\PYG{l+s+s1}{\PYGZsq{}}\PYG{l+s+s1}{Enter the Height: }\PYG{l+s+s1}{\PYGZsq{}}\PYG{p}{)} \PYG{p}{)}
\PYG{n}{weight} \PYG{o}{=} \PYG{n+nb}{int} \PYG{p}{(} \PYG{n+nb}{input}\PYG{p}{(}\PYG{l+s+s1}{\PYGZsq{}}\PYG{l+s+s1}{Enter the Weight: }\PYG{l+s+s1}{\PYGZsq{}}\PYG{p}{)} \PYG{p}{)}

\PYG{n}{bmi} \PYG{o}{=} \PYG{n}{weight}\PYG{o}{/}\PYG{p}{(}\PYG{n}{height}\PYG{o}{*}\PYG{o}{*}\PYG{l+m+mi}{2}\PYG{p}{)}

\PYG{n+nb}{print}\PYG{p}{(}\PYG{l+s+s1}{\PYGZsq{}}\PYG{l+s+s1}{Body Mass Index is}\PYG{l+s+s1}{\PYGZsq{}}\PYG{p}{,}\PYG{n}{bmi}\PYG{p}{)}
\end{sphinxVerbatim}
\end{sphinxadmonition}


\subsection{Question\sphinxhyphen{}5}
\label{\detokenize{numbers_arithmetic_code:question-5}}
\sphinxAtStartPar
Write a program that prompts the user for a 3\sphinxhyphen{}digit positive number and displays the sum of its digits.
\begin{itemize}
\item {} 
\sphinxAtStartPar
Use only one input function.

\end{itemize}

\begin{sphinxadmonition}{note}{Solution\sphinxhyphen{}1}

\begin{sphinxVerbatim}[commandchars=\\\{\}]
\PYG{n}{num} \PYG{o}{=} \PYG{n+nb}{int}\PYG{p}{(} \PYG{n+nb}{input}\PYG{p}{(}\PYG{l+s+s1}{\PYGZsq{}}\PYG{l+s+s1}{Enter a  3 digit number: }\PYG{l+s+s1}{\PYGZsq{}}\PYG{p}{)} \PYG{p}{)}

\PYG{n}{n1} \PYG{o}{=} \PYG{n}{num}\PYG{o}{\PYGZpc{}}\PYG{l+m+mi}{10}
\PYG{n}{n2} \PYG{o}{=} \PYG{n+nb}{int}\PYG{p}{(}\PYG{p}{(}\PYG{p}{(}\PYG{n}{num}\PYG{o}{\PYGZhy{}}\PYG{n}{n1}\PYG{p}{)}\PYG{o}{/}\PYG{l+m+mi}{10}\PYG{p}{)}\PYG{o}{\PYGZpc{}}\PYG{l+m+mi}{10}\PYG{p}{)}
\PYG{n}{n3} \PYG{o}{=} \PYG{n}{num}\PYG{o}{/}\PYG{o}{/}\PYG{l+m+mi}{100}      \PYG{c+c1}{\PYGZsh{} or n3=int((num\PYGZhy{}n1\PYGZhy{}n2)/100)}

\PYG{n}{sum\PYGZus{}digit} \PYG{o}{=} \PYG{n}{n1}\PYG{o}{+}\PYG{n}{n2}\PYG{o}{+}\PYG{n}{n3}

\PYG{n+nb}{print}\PYG{p}{(}\PYG{l+s+s1}{\PYGZsq{}}\PYG{l+s+s1}{Sum of digits is}\PYG{l+s+s1}{\PYGZsq{}}\PYG{p}{,} \PYG{n}{sum\PYGZus{}digit}\PYG{p}{)}
\end{sphinxVerbatim}
\end{sphinxadmonition}

\begin{sphinxadmonition}{note}{Solution\sphinxhyphen{}2}

\begin{sphinxVerbatim}[commandchars=\\\{\}]
\PYG{n}{num} \PYG{o}{=} \PYG{n+nb}{int}\PYG{p}{(} \PYG{n+nb}{input}\PYG{p}{(}\PYG{l+s+s1}{\PYGZsq{}}\PYG{l+s+s1}{Enter a  3 digit number: }\PYG{l+s+s1}{\PYGZsq{}}\PYG{p}{)} \PYG{p}{)}

\PYG{n}{n3} \PYG{o}{=} \PYG{n}{num}\PYG{o}{/}\PYG{o}{/}\PYG{l+m+mi}{100}
\PYG{n}{n2} \PYG{o}{=} \PYG{p}{(}\PYG{n}{num}\PYG{o}{\PYGZhy{}}\PYG{n}{n3}\PYG{o}{*}\PYG{l+m+mi}{100}\PYG{p}{)}\PYG{o}{/}\PYG{o}{/}\PYG{l+m+mi}{10}
\PYG{n}{n1}\PYG{o}{=}  \PYG{n}{num}\PYG{o}{\PYGZpc{}}\PYG{l+m+mi}{10}        \PYG{c+c1}{\PYGZsh{} or(num\PYGZhy{}n3*100\PYGZhy{}n2*10)}

\PYG{n}{sum\PYGZus{}digit} \PYG{o}{=} \PYG{n}{n1}\PYG{o}{+}\PYG{n}{n2}\PYG{o}{+}\PYG{n}{n3}

\PYG{n+nb}{print}\PYG{p}{(}\PYG{l+s+s1}{\PYGZsq{}}\PYG{l+s+s1}{Sum of digits is}\PYG{l+s+s1}{\PYGZsq{}}\PYG{p}{,} \PYG{n}{sum\PYGZus{}digit}\PYG{p}{)}
\end{sphinxVerbatim}
\end{sphinxadmonition}


\subsection{Question\sphinxhyphen{}6}
\label{\detokenize{numbers_arithmetic_code:question-6}}
\sphinxAtStartPar
Write a program that prompts the user for three different integers one by one, sorts these numbers, and prints them from smallest to largest.
\begin{itemize}
\item {} 
\sphinxAtStartPar
Use the built\sphinxhyphen{}in functions max() and min(), and do not use any sorting function.

\end{itemize}

\begin{sphinxadmonition}{note}{Solution\sphinxhyphen{}1}

\begin{sphinxVerbatim}[commandchars=\\\{\}]
\PYG{n}{num1} \PYG{o}{=} \PYG{n+nb}{int} \PYG{p}{(} \PYG{n+nb}{input}\PYG{p}{(}\PYG{l+s+s1}{\PYGZsq{}}\PYG{l+s+s1}{Enter the First Number: }\PYG{l+s+s1}{\PYGZsq{}}\PYG{p}{)} \PYG{p}{)}
\PYG{n}{num2} \PYG{o}{=} \PYG{n+nb}{int} \PYG{p}{(} \PYG{n+nb}{input}\PYG{p}{(}\PYG{l+s+s1}{\PYGZsq{}}\PYG{l+s+s1}{Enter the Second Number: }\PYG{l+s+s1}{\PYGZsq{}}\PYG{p}{)} \PYG{p}{)}
\PYG{n}{num3} \PYG{o}{=} \PYG{n+nb}{int} \PYG{p}{(} \PYG{n+nb}{input}\PYG{p}{(}\PYG{l+s+s1}{\PYGZsq{}}\PYG{l+s+s1}{Enter the Third Number: }\PYG{l+s+s1}{\PYGZsq{}}\PYG{p}{)} \PYG{p}{)}

\PYG{n}{minimum} \PYG{o}{=} \PYG{n+nb}{min}\PYG{p}{(}\PYG{n}{num1}\PYG{p}{,}\PYG{n}{num2}\PYG{p}{,}\PYG{n}{num3}\PYG{p}{)}
\PYG{n}{maximum} \PYG{o}{=} \PYG{n+nb}{max}\PYG{p}{(}\PYG{n}{num1}\PYG{p}{,}\PYG{n}{num2}\PYG{p}{,}\PYG{n}{num3}\PYG{p}{)}
\PYG{n}{middle}  \PYG{o}{=} \PYG{n}{num1}\PYG{o}{+}\PYG{n}{num2}\PYG{o}{+}\PYG{n}{num3}\PYG{o}{\PYGZhy{}}\PYG{n}{minimum}\PYG{o}{\PYGZhy{}}\PYG{n}{maximum}

\PYG{n+nb}{print}\PYG{p}{(}\PYG{n}{minimum}\PYG{p}{,}\PYG{n}{middle}\PYG{p}{,}\PYG{n}{maximum}\PYG{p}{,}\PYG{n}{sep} \PYG{o}{=} \PYG{l+s+s1}{\PYGZsq{}}\PYG{l+s+s1}{,}\PYG{l+s+s1}{\PYGZsq{}}\PYG{p}{)}
\end{sphinxVerbatim}
\end{sphinxadmonition}

\begin{sphinxadmonition}{note}{Solution\sphinxhyphen{}2}

\begin{sphinxVerbatim}[commandchars=\\\{\}]
\PYG{n}{num1} \PYG{o}{=} \PYG{n+nb}{int} \PYG{p}{(} \PYG{n+nb}{input}\PYG{p}{(}\PYG{l+s+s1}{\PYGZsq{}}\PYG{l+s+s1}{Enter the First Number: }\PYG{l+s+s1}{\PYGZsq{}}\PYG{p}{)} \PYG{p}{)}
\PYG{n}{num2} \PYG{o}{=} \PYG{n+nb}{int} \PYG{p}{(} \PYG{n+nb}{input}\PYG{p}{(}\PYG{l+s+s1}{\PYGZsq{}}\PYG{l+s+s1}{Enter the Second Number: }\PYG{l+s+s1}{\PYGZsq{}}\PYG{p}{)} \PYG{p}{)}
\PYG{n}{num3} \PYG{o}{=} \PYG{n+nb}{int} \PYG{p}{(} \PYG{n+nb}{input}\PYG{p}{(}\PYG{l+s+s1}{\PYGZsq{}}\PYG{l+s+s1}{Enter the Third Number: }\PYG{l+s+s1}{\PYGZsq{}}\PYG{p}{)} \PYG{p}{)}

\PYG{n}{minimum} \PYG{o}{=} \PYG{n+nb}{min}\PYG{p}{(}\PYG{n}{num1}\PYG{p}{,}\PYG{n}{num2}\PYG{p}{,}\PYG{n}{num3}\PYG{p}{)}
\PYG{n}{maximum} \PYG{o}{=} \PYG{n+nb}{max}\PYG{p}{(}\PYG{n}{num1}\PYG{p}{,}\PYG{n}{num2}\PYG{p}{,}\PYG{n}{num3}\PYG{p}{)}

\PYG{n}{middle} \PYG{o}{=} \PYG{n+nb}{min}\PYG{p}{(}\PYG{n+nb}{max}\PYG{p}{(}\PYG{n}{num1}\PYG{p}{,}\PYG{n}{num2}\PYG{p}{)}\PYG{p}{,} \PYG{n+nb}{max}\PYG{p}{(}\PYG{n}{num2}\PYG{p}{,}\PYG{n}{num3}\PYG{p}{)}\PYG{p}{,} \PYG{n+nb}{max}\PYG{p}{(}\PYG{n}{num3}\PYG{p}{,}\PYG{n}{num1}\PYG{p}{)}\PYG{p}{)}

\PYG{n+nb}{print}\PYG{p}{(}\PYG{l+s+s1}{\PYGZsq{}}\PYG{l+s+s1}{Smallest to largest integer: }\PYG{l+s+s1}{\PYGZsq{}}\PYG{p}{,}\PYG{n}{min\PYGZus{}num}\PYG{p}{,}\PYG{n}{mid}\PYG{p}{,}\PYG{n}{max\PYGZus{}num}\PYG{p}{)}
\end{sphinxVerbatim}
\end{sphinxadmonition}


\subsection{Question\sphinxhyphen{}7}
\label{\detokenize{numbers_arithmetic_code:question-7}}
\sphinxAtStartPar
Write a program that prompts the user for a 2\sphinxhyphen{}digit positive number. Swap the digits of the given number and print it.
\begin{itemize}
\item {} 
\sphinxAtStartPar
Example 1: If the given number is 53, then print 35.
\begin{itemize}
\item {} 
\sphinxAtStartPar
Print format: 53 —\sphinxhyphen{} swap—> 35

\end{itemize}

\item {} 
\sphinxAtStartPar
Example 2: If the given number is 71, then print 17.
\begin{itemize}
\item {} 
\sphinxAtStartPar
Print format: 71 —\sphinxhyphen{} swap—> 17

\end{itemize}

\end{itemize}

\begin{sphinxadmonition}{note}{Solution}

\begin{sphinxVerbatim}[commandchars=\\\{\}]
\PYG{n}{number} \PYG{o}{=} \PYG{n+nb}{int}\PYG{p}{(}\PYG{n+nb}{input}\PYG{p}{(}\PYG{l+s+s2}{\PYGZdq{}}\PYG{l+s+s2}{Enter a two digit number: }\PYG{l+s+s2}{\PYGZdq{}}\PYG{p}{)}\PYG{p}{)}

\PYG{n}{ones} \PYG{o}{=} \PYG{n}{number}\PYG{o}{\PYGZpc{}}\PYG{l+m+mi}{10}
\PYG{n}{tens} \PYG{o}{=} \PYG{n}{number}\PYG{o}{/}\PYG{o}{/}\PYG{l+m+mi}{10}

\PYG{n}{result} \PYG{o}{=} \PYG{n}{ones}\PYG{o}{*}\PYG{l+m+mi}{10} \PYG{o}{+} \PYG{n}{tens}

\PYG{n+nb}{print}\PYG{p}{(}\PYG{n}{number}\PYG{p}{,}\PYG{l+s+s1}{\PYGZsq{}}\PYG{l+s+s1}{\PYGZhy{}\PYGZhy{}\PYGZhy{}\PYGZhy{} swap\PYGZhy{}\PYGZhy{}\PYGZhy{}\PYGZgt{}}\PYG{l+s+s1}{\PYGZsq{}}\PYG{p}{,} \PYG{n}{result}\PYG{p}{)}
\end{sphinxVerbatim}
\end{sphinxadmonition}


\subsection{Question\sphinxhyphen{}8}
\label{\detokenize{numbers_arithmetic_code:question-8}}
\sphinxAtStartPar
Write a program that prompts the user for a 3\sphinxhyphen{}digit positive number. Swap the digits of the given number and print it.

\begin{sphinxVerbatim}[commandchars=\\\{\}]
\PYGZhy{} Example 1: If the given number is 153, then print 351.
    \PYGZhy{} Print format: 153 \PYGZhy{}\PYGZhy{}\PYGZhy{}\PYGZhy{} swap\PYGZhy{}\PYGZhy{}\PYGZhy{}\PYGZgt{} 351

\PYGZhy{} Example 2: If the given number is 571, then print 175.
    \PYGZhy{} Print format: 571 \PYGZhy{}\PYGZhy{}\PYGZhy{}\PYGZhy{} swap\PYGZhy{}\PYGZhy{}\PYGZhy{}\PYGZgt{} 175
\end{sphinxVerbatim}

\begin{sphinxadmonition}{note}{Solution}

\begin{sphinxVerbatim}[commandchars=\\\{\}]
\PYG{n}{number} \PYG{o}{=} \PYG{n+nb}{int}\PYG{p}{(}\PYG{n+nb}{input}\PYG{p}{(}\PYG{l+s+s2}{\PYGZdq{}}\PYG{l+s+s2}{Enter a three digit number: }\PYG{l+s+s2}{\PYGZdq{}}\PYG{p}{)}\PYG{p}{)}

\PYG{n}{ones} \PYG{o}{=} \PYG{n}{number}\PYG{o}{\PYGZpc{}}\PYG{l+m+mi}{10}
\PYG{n}{tens} \PYG{o}{=} \PYG{p}{(}\PYG{n}{number}\PYG{o}{/}\PYG{o}{/}\PYG{l+m+mi}{10}\PYG{p}{)}\PYG{o}{\PYGZpc{}}\PYG{l+m+mi}{10}
\PYG{n}{hundreds} \PYG{o}{=} \PYG{n}{number}\PYG{o}{/}\PYG{o}{/}\PYG{l+m+mi}{100}

\PYG{n}{result} \PYG{o}{=} \PYG{n}{ones}\PYG{o}{*}\PYG{l+m+mi}{100} \PYG{o}{+} \PYG{n}{tens}\PYG{o}{*}\PYG{l+m+mi}{10} \PYG{o}{+} \PYG{n}{hundreds}

\PYG{n+nb}{print}\PYG{p}{(}\PYG{n}{number}\PYG{p}{,}\PYG{l+s+s1}{\PYGZsq{}}\PYG{l+s+s1}{\PYGZhy{}\PYGZhy{}\PYGZhy{}\PYGZhy{} swap\PYGZhy{}\PYGZhy{}\PYGZhy{}\PYGZgt{}}\PYG{l+s+s1}{\PYGZsq{}}\PYG{p}{,} \PYG{n}{result}\PYG{p}{)}
\end{sphinxVerbatim}
\end{sphinxadmonition}


\subsection{Question\sphinxhyphen{}9}
\label{\detokenize{numbers_arithmetic_code:question-9}}
\sphinxAtStartPar
Write a program that prompts the user for two numbers using two input functions. Assign the entered values to variables with the names x and y.
\begin{itemize}
\item {} 
\sphinxAtStartPar
Find \(\displaystyle f(x, y) = \frac{2x^3+3y^2}{x^2+y^2+1}\times5\)

\item {} 
\sphinxAtStartPar
Round this value to the second decimal place and print the result in the following format.
\begin{itemize}
\item {} 
\sphinxAtStartPar
for \(x=1, y=2\) display \(f(1,2)=11.67\)

\end{itemize}

\end{itemize}

\begin{sphinxadmonition}{note}{Solution}

\begin{sphinxVerbatim}[commandchars=\\\{\}]
\PYG{n}{x} \PYG{o}{=} \PYG{n+nb}{int}\PYG{p}{(}\PYG{n+nb}{input}\PYG{p}{(}\PYG{l+s+s1}{\PYGZsq{}}\PYG{l+s+s1}{x: }\PYG{l+s+s1}{\PYGZsq{}}\PYG{p}{)}\PYG{p}{)}
\PYG{n}{y} \PYG{o}{=} \PYG{n+nb}{int}\PYG{p}{(}\PYG{n+nb}{input}\PYG{p}{(}\PYG{l+s+s1}{\PYGZsq{}}\PYG{l+s+s1}{y: }\PYG{l+s+s1}{\PYGZsq{}}\PYG{p}{)}\PYG{p}{)}

\PYG{n}{f\PYGZus{}x\PYGZus{}y} \PYG{o}{=} \PYG{p}{(}\PYG{l+m+mi}{2}\PYG{o}{*}\PYG{n}{x}\PYG{o}{*}\PYG{o}{*}\PYG{l+m+mi}{3}\PYG{o}{+}\PYG{l+m+mi}{3}\PYG{o}{*}\PYG{n}{y}\PYG{o}{*}\PYG{o}{*}\PYG{l+m+mi}{2}\PYG{p}{)}\PYG{o}{/}\PYG{p}{(}\PYG{n}{x}\PYG{o}{*}\PYG{o}{*}\PYG{l+m+mi}{2}\PYG{o}{+}\PYG{n}{y}\PYG{o}{*}\PYG{o}{*}\PYG{l+m+mi}{2}\PYG{o}{+}\PYG{l+m+mi}{1}\PYG{p}{)}\PYG{o}{*}\PYG{l+m+mi}{5}

\PYG{n+nb}{print}\PYG{p}{(}\PYG{l+s+s1}{\PYGZsq{}}\PYG{l+s+s1}{f(}\PYG{l+s+s1}{\PYGZsq{}}\PYG{p}{,} \PYG{n}{x}\PYG{p}{,} \PYG{l+s+s1}{\PYGZsq{}}\PYG{l+s+s1}{,}\PYG{l+s+s1}{\PYGZsq{}} \PYG{p}{,}\PYG{n}{y}\PYG{p}{,} \PYG{l+s+s1}{\PYGZsq{}}\PYG{l+s+s1}{)=}\PYG{l+s+s1}{\PYGZsq{}}\PYG{p}{,} \PYG{n+nb}{round}\PYG{p}{(}\PYG{n}{f\PYGZus{}x\PYGZus{}y}\PYG{p}{,}\PYG{l+m+mi}{2}\PYG{p}{)}\PYG{p}{,} \PYG{n}{sep}\PYG{o}{=}\PYG{l+s+s1}{\PYGZsq{}}\PYG{l+s+s1}{\PYGZsq{}}\PYG{p}{)}
\end{sphinxVerbatim}
\end{sphinxadmonition}


\subsection{Question\sphinxhyphen{}10}
\label{\detokenize{numbers_arithmetic_code:question-10}}
\sphinxAtStartPar
Write a program that prompts the user for 5 numbers using five different input functions.
\begin{itemize}
\item {} 
\sphinxAtStartPar
Find the average of these numbers without using any built\sphinxhyphen{}in function.

\item {} 
\sphinxAtStartPar
Round it to the nearest integer and display the result.

\item {} 
\sphinxAtStartPar
Example: If the given numbers are 2, 8, 9, 6, 5, then display \sphinxstyleemphasis{average(2, 8, 9, 6, 5) = 6}.

\end{itemize}

\begin{sphinxadmonition}{note}{Solution}

\begin{sphinxVerbatim}[commandchars=\\\{\}]
\PYG{n}{number1} \PYG{o}{=} \PYG{n+nb}{float}\PYG{p}{(}\PYG{n+nb}{input}\PYG{p}{(}\PYG{l+s+s2}{\PYGZdq{}}\PYG{l+s+s2}{Enter 1st number: }\PYG{l+s+s2}{\PYGZdq{}}\PYG{p}{)}\PYG{p}{)}
\PYG{n}{number2} \PYG{o}{=} \PYG{n+nb}{float}\PYG{p}{(}\PYG{n+nb}{input}\PYG{p}{(}\PYG{l+s+s2}{\PYGZdq{}}\PYG{l+s+s2}{Enter 2nd number: }\PYG{l+s+s2}{\PYGZdq{}}\PYG{p}{)}\PYG{p}{)}
\PYG{n}{number3} \PYG{o}{=} \PYG{n+nb}{float}\PYG{p}{(}\PYG{n+nb}{input}\PYG{p}{(}\PYG{l+s+s2}{\PYGZdq{}}\PYG{l+s+s2}{Enter 3rd number: }\PYG{l+s+s2}{\PYGZdq{}}\PYG{p}{)}\PYG{p}{)}
\PYG{n}{number4} \PYG{o}{=} \PYG{n+nb}{float}\PYG{p}{(}\PYG{n+nb}{input}\PYG{p}{(}\PYG{l+s+s2}{\PYGZdq{}}\PYG{l+s+s2}{Enter 4th number: }\PYG{l+s+s2}{\PYGZdq{}}\PYG{p}{)}\PYG{p}{)}
\PYG{n}{number5} \PYG{o}{=} \PYG{n+nb}{float}\PYG{p}{(}\PYG{n+nb}{input}\PYG{p}{(}\PYG{l+s+s2}{\PYGZdq{}}\PYG{l+s+s2}{Enter 5th number: }\PYG{l+s+s2}{\PYGZdq{}}\PYG{p}{)}\PYG{p}{)}

\PYG{n}{average} \PYG{o}{=} \PYG{n+nb}{sum}\PYG{p}{(}\PYG{p}{[}\PYG{n}{number1}\PYG{p}{,} \PYG{n}{number2}\PYG{p}{,} \PYG{n}{number3}\PYG{p}{,} \PYG{n}{number4}\PYG{p}{,} \PYG{n}{number5}\PYG{p}{]}\PYG{p}{)}\PYG{o}{/}\PYG{l+m+mi}{5}

\PYG{n+nb}{print}\PYG{p}{(}\PYG{l+s+s1}{\PYGZsq{}}\PYG{l+s+s1}{average(}\PYG{l+s+s1}{\PYGZsq{}}\PYG{p}{,} \PYG{n}{number1}\PYG{p}{,}\PYG{l+s+s1}{\PYGZsq{}}\PYG{l+s+s1}{,}\PYG{l+s+s1}{\PYGZsq{}} \PYG{p}{,}\PYG{n}{number2}\PYG{p}{,}\PYG{l+s+s1}{\PYGZsq{}}\PYG{l+s+s1}{,}\PYG{l+s+s1}{\PYGZsq{}} \PYG{p}{,} \PYG{n}{number3}\PYG{p}{,}\PYG{l+s+s1}{\PYGZsq{}}\PYG{l+s+s1}{,}\PYG{l+s+s1}{\PYGZsq{}} \PYG{p}{,} \PYG{n}{number4}\PYG{p}{,}\PYG{l+s+s1}{\PYGZsq{}}\PYG{l+s+s1}{,}\PYG{l+s+s1}{\PYGZsq{}} \PYG{p}{,} \PYG{n}{number5}\PYG{p}{,}\PYG{l+s+s1}{\PYGZsq{}}\PYG{l+s+s1}{) =}\PYG{l+s+s1}{\PYGZsq{}}\PYG{p}{,} \PYG{n+nb}{round}\PYG{p}{(}\PYG{n}{average}\PYG{p}{)}\PYG{p}{,} \PYG{n}{sep}\PYG{o}{=}\PYG{l+s+s1}{\PYGZsq{}}\PYG{l+s+s1}{\PYGZsq{}}\PYG{p}{)}
\end{sphinxVerbatim}
\end{sphinxadmonition}


\subsection{Question\sphinxhyphen{}11}
\label{\detokenize{numbers_arithmetic_code:question-11}}
\sphinxAtStartPar
Write a program that prompts the user for 5 numbers using five different input functions.
\begin{itemize}
\item {} 
\sphinxAtStartPar
Find the sum of these numbers excluding the largest and smallest ones.

\item {} 
\sphinxAtStartPar
Display the sum.

\end{itemize}

\begin{sphinxadmonition}{note}{Solution}

\begin{sphinxVerbatim}[commandchars=\\\{\}]
\PYG{n}{number1} \PYG{o}{=} \PYG{n+nb}{float}\PYG{p}{(}\PYG{n+nb}{input}\PYG{p}{(}\PYG{l+s+s2}{\PYGZdq{}}\PYG{l+s+s2}{Enter 1st number: }\PYG{l+s+s2}{\PYGZdq{}}\PYG{p}{)}\PYG{p}{)}
\PYG{n}{number2} \PYG{o}{=} \PYG{n+nb}{float}\PYG{p}{(}\PYG{n+nb}{input}\PYG{p}{(}\PYG{l+s+s2}{\PYGZdq{}}\PYG{l+s+s2}{Enter 2nd number: }\PYG{l+s+s2}{\PYGZdq{}}\PYG{p}{)}\PYG{p}{)}
\PYG{n}{number3} \PYG{o}{=} \PYG{n+nb}{float}\PYG{p}{(}\PYG{n+nb}{input}\PYG{p}{(}\PYG{l+s+s2}{\PYGZdq{}}\PYG{l+s+s2}{Enter 3rd number: }\PYG{l+s+s2}{\PYGZdq{}}\PYG{p}{)}\PYG{p}{)}
\PYG{n}{number4} \PYG{o}{=} \PYG{n+nb}{float}\PYG{p}{(}\PYG{n+nb}{input}\PYG{p}{(}\PYG{l+s+s2}{\PYGZdq{}}\PYG{l+s+s2}{Enter 4th number: }\PYG{l+s+s2}{\PYGZdq{}}\PYG{p}{)}\PYG{p}{)}
\PYG{n}{number5} \PYG{o}{=} \PYG{n+nb}{float}\PYG{p}{(}\PYG{n+nb}{input}\PYG{p}{(}\PYG{l+s+s2}{\PYGZdq{}}\PYG{l+s+s2}{Enter 5th number: }\PYG{l+s+s2}{\PYGZdq{}}\PYG{p}{)}\PYG{p}{)}

\PYG{n}{max\PYGZus{}number} \PYG{o}{=} \PYG{n+nb}{max}\PYG{p}{(}\PYG{n}{number1}\PYG{p}{,} \PYG{n}{number2}\PYG{p}{,} \PYG{n}{number3}\PYG{p}{,} \PYG{n}{number4}\PYG{p}{,} \PYG{n}{number5}\PYG{p}{)}
\PYG{n}{min\PYGZus{}number} \PYG{o}{=} \PYG{n+nb}{min}\PYG{p}{(}\PYG{n}{number1}\PYG{p}{,} \PYG{n}{number2}\PYG{p}{,} \PYG{n}{number3}\PYG{p}{,} \PYG{n}{number4}\PYG{p}{,} \PYG{n}{number5}\PYG{p}{)}

\PYG{n}{total} \PYG{o}{=} \PYG{n+nb}{sum}\PYG{p}{(}\PYG{p}{[}\PYG{n}{number1}\PYG{p}{,} \PYG{n}{number2}\PYG{p}{,} \PYG{n}{number3}\PYG{p}{,} \PYG{n}{number4}\PYG{p}{,} \PYG{n}{number5}\PYG{p}{]}\PYG{p}{)}

\PYG{n+nb}{print}\PYG{p}{(}\PYG{n}{total}\PYG{o}{\PYGZhy{}}\PYG{p}{(}\PYG{n}{max\PYGZus{}number}\PYG{o}{+}\PYG{n}{min\PYGZus{}number}\PYG{p}{)}\PYG{p}{)}
\end{sphinxVerbatim}
\end{sphinxadmonition}


\subsection{Question\sphinxhyphen{}12}
\label{\detokenize{numbers_arithmetic_code:question-12}}\begin{itemize}
\item {} 
\sphinxAtStartPar
Choose a 3\sphinxhyphen{}digit random number as the dividend.

\item {} 
\sphinxAtStartPar
Choose a 1\sphinxhyphen{}digit random number as the divisor.

\item {} 
\sphinxAtStartPar
Display these numbers.

\item {} 
\sphinxAtStartPar
Find the remainder and quotient when the dividend is divided by the divisor.

\item {} 
\sphinxAtStartPar
After 5 seconds, show the correct remainder and quotient.

\end{itemize}

\sphinxAtStartPar
\sphinxstylestrong{Solution}

\sphinxstepscope


\section{Numbers Exercises}
\label{\detokenize{numbers_arithmetic_exercise:numbers-exercises}}\label{\detokenize{numbers_arithmetic_exercise::doc}}

\subsection{Question\sphinxhyphen{}1}
\label{\detokenize{numbers_arithmetic_exercise:question-1}}
\sphinxAtStartPar
In a single line of code, compute \(\sqrt{\sqrt{\sqrt{625}}}\) and round it to the nearest hundredth.


\subsection{Question\sphinxhyphen{}2}
\label{\detokenize{numbers_arithmetic_exercise:question-2}}
\sphinxAtStartPar
In a single line of code, compute \(\displaystyle  |ln( 2^{sin(|-100|)})|\) and round it to the nearest hundredth.


\subsection{Question\sphinxhyphen{}3}
\label{\detokenize{numbers_arithmetic_exercise:question-3}}
\sphinxAtStartPar
Write a program that prompts the user for three numbers using three input functions. Assign the entered values to variables with the names x, y, and z.
\begin{itemize}
\item {} 
\sphinxAtStartPar
Find \(\displaystyle f(x, y,z) = \frac{5xy}{2+x^2}+\frac{x+y+z}{y^4+x^2y^2}\)

\item {} 
\sphinxAtStartPar
Round this value to the nearest hundredth and print the result in the following format.

\item {} 
\sphinxAtStartPar
Sample Output:\\
x:  1\\
y:  2\\
z:  3\\
f(1,2,3)=3.63

\end{itemize}


\subsection{Question\sphinxhyphen{}4}
\label{\detokenize{numbers_arithmetic_exercise:question-4}}
\sphinxAtStartPar
Write a program that prompts the user to enter their height using two input() functions for the feet and inch parts separately. Assign the entered values to variables named feet and inch.
\begin{itemize}
\item {} 
\sphinxAtStartPar
Convert the given height into centimeters using the following conversion formulas: \sphinxstyleemphasis{1 foot = 12 inches} and \sphinxstyleemphasis{1 inch = 2.54 cm}

\item {} 
\sphinxAtStartPar
Sample Output:\\
Enter the feet part of your height: 6\\
Enter the inch part of your height: 4\\
6 feet and 4 inches = 193.04 cm

\end{itemize}


\subsection{Question\sphinxhyphen{}5}
\label{\detokenize{numbers_arithmetic_exercise:question-5}}
\sphinxAtStartPar
Write a program that prompts the user for a 4\sphinxhyphen{}digit positive number. Swap the first two digits of the given number with the last two digits and print it.
\begin{itemize}
\item {} 
\sphinxAtStartPar
Do not use string indexing that will be covered in the next chapter.

\item {} 
\sphinxAtStartPar
Example 1: If the given number is 1234, then print 3412.
\begin{itemize}
\item {} 
\sphinxAtStartPar
Print format: 1234 —\sphinxhyphen{} swap—> 3412

\end{itemize}

\item {} 
\sphinxAtStartPar
Example 2: If the given number is 6789, then print 8967.
\begin{itemize}
\item {} 
\sphinxAtStartPar
Print format: 6789 —\sphinxhyphen{} swap—> 8967

\end{itemize}

\end{itemize}


\subsection{Question\sphinxhyphen{}6}
\label{\detokenize{numbers_arithmetic_exercise:question-6}}
\sphinxAtStartPar
In a coordinate plane, each point is represented by its x and y components in the form of \((x,y)\).\\
The distance between two points \(P=(x_1,y_1)\) and \(Q=(x_2,y_2)\) is given by the following distance formula:
\begin{itemize}
\item {} 
\sphinxAtStartPar
\(\displaystyle dist(P,Q) = \sqrt{(x_2-x_1)^2+(y_2-y_1)^2}\)

\end{itemize}

\sphinxAtStartPar
Write a program that prompts the user for the x and y components of a point using two \sphinxstyleemphasis{input()} functions and computes the distance between the given point and the point \((-5,6)\).
\begin{itemize}
\item {} 
\sphinxAtStartPar
Round the distance to the nearest hundreth.

\end{itemize}

\sphinxAtStartPar
Sample Output:\\
Enter the x\sphinxhyphen{}component of the point: \sphinxhyphen{}2\\
Enter the y\sphinxhyphen{}component of the point: 10\\
The distance between (\sphinxhyphen{}5,6) and (\sphinxhyphen{}2.0,10.0) is 5.0

\sphinxstepscope


\chapter{Chp\sphinxhyphen{}4: Strings}
\label{\detokenize{strings:chp-4-strings}}\label{\detokenize{strings::doc}}\begin{itemize}
\item {} 
\sphinxAtStartPar
Learning Objectives
\begin{itemize}
\item {} 
\sphinxAtStartPar
..

\item {} 
\sphinxAtStartPar
..

\end{itemize}

\end{itemize}


\section{Create Strings}
\label{\detokenize{strings:create-strings}}
\sphinxAtStartPar
Strings are ordered sequences of characters.
\begin{itemize}
\item {} 
\sphinxAtStartPar
There are several ways to create strings:
\begin{itemize}
\item {} 
\sphinxAtStartPar
Single quotes:  \sphinxcode{\sphinxupquote{'text'}}

\item {} 
\sphinxAtStartPar
Double quotes:  \sphinxcode{\sphinxupquote{"text"}}

\item {} 
\sphinxAtStartPar
Triple single quotes: \sphinxcode{\sphinxupquote{'''text'''}}

\item {} 
\sphinxAtStartPar
Triple double quotes: \sphinxcode{\sphinxupquote{"""text"""}}

\end{itemize}

\item {} 
\sphinxAtStartPar
A space is also a character that can be in a string.

\item {} 
\sphinxAtStartPar
Triple single and double quotes are used to create strings spanning multiple lines.

\item {} 
\sphinxAtStartPar
If a single quote is a character in the string, using single quotes to create the string is an error. This is because the character single quote will end the creation of the string. It behaves like the second single quote, instead of being a character in the string.
\begin{itemize}
\item {} 
\sphinxAtStartPar
You can use double quotes to create the string to overcome the confusion.

\item {} 
\sphinxAtStartPar
Alternatively, you can use escaping characters.

\end{itemize}

\end{itemize}

\begin{sphinxuseclass}{cell}\begin{sphinxVerbatimInput}

\begin{sphinxuseclass}{cell_input}
\begin{sphinxVerbatim}[commandchars=\\\{\}]
\PYG{c+c1}{\PYGZsh{} Single quotes }
\PYG{n}{name} \PYG{o}{=} \PYG{l+s+s1}{\PYGZsq{}}\PYG{l+s+s1}{Mary}\PYG{l+s+s1}{\PYGZsq{}}
\end{sphinxVerbatim}

\end{sphinxuseclass}\end{sphinxVerbatimInput}

\end{sphinxuseclass}
\begin{sphinxVerbatim}[commandchars=\\\{\}]
\PYG{c+c1}{\PYGZsh{} ERROR: Double single quotes cannot be used to create strings.}
\PYG{n}{name} \PYG{o}{=} \PYG{l+s+s1}{\PYGZsq{}}\PYG{l+s+s1}{\PYGZsq{}}\PYG{n}{Mary}\PYG{l+s+s1}{\PYGZsq{}}\PYG{l+s+s1}{\PYGZsq{}}
\PYG{n+nb}{print}\PYG{p}{(}\PYG{n}{name}\PYG{p}{)}
\end{sphinxVerbatim}

\begin{sphinxuseclass}{cell}\begin{sphinxVerbatimInput}

\begin{sphinxuseclass}{cell_input}
\begin{sphinxVerbatim}[commandchars=\\\{\}]
\PYG{c+c1}{\PYGZsh{} Double quotes }
\PYG{n}{name} \PYG{o}{=} \PYG{l+s+s2}{\PYGZdq{}}\PYG{l+s+s2}{Mary}\PYG{l+s+s2}{\PYGZdq{}}
\end{sphinxVerbatim}

\end{sphinxuseclass}\end{sphinxVerbatimInput}

\end{sphinxuseclass}
\begin{sphinxuseclass}{cell}\begin{sphinxVerbatimInput}

\begin{sphinxuseclass}{cell_input}
\begin{sphinxVerbatim}[commandchars=\\\{\}]
\PYG{c+c1}{\PYGZsh{} Triple single quotes}
\PYG{n}{name} \PYG{o}{=} \PYG{l+s+s1}{\PYGZsq{}\PYGZsq{}\PYGZsq{}}\PYG{l+s+s1}{Mary}\PYG{l+s+s1}{\PYGZsq{}\PYGZsq{}\PYGZsq{}}
\end{sphinxVerbatim}

\end{sphinxuseclass}\end{sphinxVerbatimInput}

\end{sphinxuseclass}
\begin{sphinxuseclass}{cell}\begin{sphinxVerbatimInput}

\begin{sphinxuseclass}{cell_input}
\begin{sphinxVerbatim}[commandchars=\\\{\}]
\PYG{c+c1}{\PYGZsh{} Triple double quotes}
\PYG{n}{name} \PYG{o}{=} \PYG{l+s+s2}{\PYGZdq{}\PYGZdq{}\PYGZdq{}}\PYG{l+s+s2}{Mary}\PYG{l+s+s2}{\PYGZdq{}\PYGZdq{}\PYGZdq{}}
\end{sphinxVerbatim}

\end{sphinxuseclass}\end{sphinxVerbatimInput}

\end{sphinxuseclass}
\begin{sphinxVerbatim}[commandchars=\\\{\}]
\PYG{c+c1}{\PYGZsh{} ERROR}
\PYG{l+s+s1}{\PYGZsq{}}\PYG{l+s+s1}{Mary}\PYG{l+s+s1}{\PYGZsq{}}\PYG{n}{s} \PYG{n}{book}\PYG{o}{.}\PYG{l+s+s1}{\PYGZsq{}}
\end{sphinxVerbatim}
\begin{itemize}
\item {} 
\sphinxAtStartPar
In the code above, the second single quote ends the creation of the string.

\item {} 
\sphinxAtStartPar
The \sphinxcode{\sphinxupquote{s book.'}} part causes the error because it has only one single quote.

\item {} 
\sphinxAtStartPar
If you use double quotes, then there will not be any problem.

\end{itemize}

\begin{sphinxuseclass}{cell}\begin{sphinxVerbatimInput}

\begin{sphinxuseclass}{cell_input}
\begin{sphinxVerbatim}[commandchars=\\\{\}]
\PYG{c+c1}{\PYGZsh{} no ERROR}
\PYG{n}{sentence} \PYG{o}{=} \PYG{l+s+s2}{\PYGZdq{}}\PYG{l+s+s2}{Mary}\PYG{l+s+s2}{\PYGZsq{}}\PYG{l+s+s2}{s book.}\PYG{l+s+s2}{\PYGZdq{}}
\end{sphinxVerbatim}

\end{sphinxuseclass}\end{sphinxVerbatimInput}

\end{sphinxuseclass}
\begin{sphinxVerbatim}[commandchars=\\\{\}]
\PYG{c+c1}{\PYGZsh{} ERROR}
\PYG{l+s+s2}{\PYGZdq{}}\PYG{l+s+s2}{Mary said }\PYG{l+s+s2}{\PYGZdq{}}\PYG{n}{I} \PYG{n}{am} \PYG{n}{here}\PYG{l+s+s2}{\PYGZdq{}}\PYG{l+s+s2}{.}\PYG{l+s+s2}{\PYGZdq{}}
\end{sphinxVerbatim}
\begin{itemize}
\item {} 
\sphinxAtStartPar
In the code above, the second double quote ends the creation of the string.

\item {} 
\sphinxAtStartPar
The \sphinxcode{\sphinxupquote{I am here}} part causes the error because it is not enclosed by single or double quotes.

\item {} 
\sphinxAtStartPar
If you use single quotes, then there will not be any problem.

\end{itemize}

\begin{sphinxuseclass}{cell}\begin{sphinxVerbatimInput}

\begin{sphinxuseclass}{cell_input}
\begin{sphinxVerbatim}[commandchars=\\\{\}]
\PYG{c+c1}{\PYGZsh{} no ERROR}
\PYG{n}{sentence} \PYG{o}{=} \PYG{l+s+s1}{\PYGZsq{}}\PYG{l+s+s1}{Mary said }\PYG{l+s+s1}{\PYGZdq{}}\PYG{l+s+s1}{I am here}\PYG{l+s+s1}{\PYGZdq{}}\PYG{l+s+s1}{.}\PYG{l+s+s1}{\PYGZsq{}}
\end{sphinxVerbatim}

\end{sphinxuseclass}\end{sphinxVerbatimInput}

\end{sphinxuseclass}
\begin{sphinxVerbatim}[commandchars=\\\{\}]
\PYG{c+c1}{\PYGZsh{} ERROR: Simple quotes cannot be used with strings spanning more than one line.}
\PYG{n}{name} \PYG{o}{=} \PYG{l+s+s1}{\PYGZsq{}}\PYG{l+s+s1}{John}
     \PYG{n}{Steinbeck}\PYG{l+s+s1}{\PYGZsq{}}
\PYG{n+nb}{print}\PYG{p}{(}\PYG{n}{x}\PYG{p}{)}
\end{sphinxVerbatim}

\begin{sphinxVerbatim}[commandchars=\\\{\}]
\PYG{c+c1}{\PYGZsh{} ERROR: Double quotes cannot be used with strings spanning more than one line.}
\PYG{n}{name} \PYG{o}{=} \PYG{l+s+s2}{\PYGZdq{}}\PYG{l+s+s2}{John}
     \PYG{n}{Steinbeck}\PYG{l+s+s2}{\PYGZdq{}}
\end{sphinxVerbatim}

\begin{sphinxuseclass}{cell}\begin{sphinxVerbatimInput}

\begin{sphinxuseclass}{cell_input}
\begin{sphinxVerbatim}[commandchars=\\\{\}]
\PYG{c+c1}{\PYGZsh{} no ERROR}
\PYG{n}{name} \PYG{o}{=} \PYG{l+s+s1}{\PYGZsq{}\PYGZsq{}\PYGZsq{}}\PYG{l+s+s1}{John}
\PYG{l+s+s1}{     Steinbeck}\PYG{l+s+s1}{\PYGZsq{}\PYGZsq{}\PYGZsq{}}
\end{sphinxVerbatim}

\end{sphinxuseclass}\end{sphinxVerbatimInput}

\end{sphinxuseclass}
\begin{sphinxuseclass}{cell}\begin{sphinxVerbatimInput}

\begin{sphinxuseclass}{cell_input}
\begin{sphinxVerbatim}[commandchars=\\\{\}]
\PYG{c+c1}{\PYGZsh{} no ERROR}
\PYG{n}{name} \PYG{o}{=} \PYG{l+s+s2}{\PYGZdq{}\PYGZdq{}\PYGZdq{}}\PYG{l+s+s2}{John}
\PYG{l+s+s2}{     Steinbeck}\PYG{l+s+s2}{\PYGZdq{}\PYGZdq{}\PYGZdq{}}
\end{sphinxVerbatim}

\end{sphinxuseclass}\end{sphinxVerbatimInput}

\end{sphinxuseclass}

\section{Escaping characters}
\label{\detokenize{strings:escaping-characters}}
\sphinxAtStartPar
An escape character is a backslash \sphinxcode{\sphinxupquote{\textbackslash{}}} followed by a character. It is used to write special characters in a string.
\begin{itemize}
\item {} 
\sphinxAtStartPar
\sphinxcode{\sphinxupquote{\textbackslash{}'}} : Single quote (apostrophy)

\item {} 
\sphinxAtStartPar
\sphinxcode{\sphinxupquote{\textbackslash{}"}} : Double quote

\item {} 
\sphinxAtStartPar
\sphinxcode{\sphinxupquote{\textbackslash{}n}} : New line

\item {} 
\sphinxAtStartPar
\sphinxcode{\sphinxupquote{\textbackslash{}t}} : Tabulation

\item {} 
\sphinxAtStartPar
\sphinxcode{\sphinxupquote{\textbackslash{}b}} : Backspace

\item {} 
\sphinxAtStartPar
\sphinxcode{\sphinxupquote{\textbackslash{}\textbackslash{}}} : Backslash

\item {} 
\sphinxAtStartPar
\sphinxcode{\sphinxupquote{\textbackslash{}r}} : Carriage return

\end{itemize}

\begin{sphinxuseclass}{cell}\begin{sphinxVerbatimInput}

\begin{sphinxuseclass}{cell_input}
\begin{sphinxVerbatim}[commandchars=\\\{\}]
\PYG{c+c1}{\PYGZsh{} \PYGZbs{}\PYGZsq{} is the character \PYGZsq{}}
\PYG{n+nb}{print}\PYG{p}{(}\PYG{l+s+s1}{\PYGZsq{}}\PYG{l+s+s1}{Mary}\PYG{l+s+se}{\PYGZbs{}\PYGZsq{}}\PYG{l+s+s1}{s book.}\PYG{l+s+s1}{\PYGZsq{}}\PYG{p}{)}
\end{sphinxVerbatim}

\end{sphinxuseclass}\end{sphinxVerbatimInput}
\begin{sphinxVerbatimOutput}

\begin{sphinxuseclass}{cell_output}
\begin{sphinxVerbatim}[commandchars=\\\{\}]
Mary\PYGZsq{}s book.
\end{sphinxVerbatim}

\end{sphinxuseclass}\end{sphinxVerbatimOutput}

\end{sphinxuseclass}
\begin{sphinxuseclass}{cell}\begin{sphinxVerbatimInput}

\begin{sphinxuseclass}{cell_input}
\begin{sphinxVerbatim}[commandchars=\\\{\}]
\PYG{c+c1}{\PYGZsh{} \PYGZbs{}\PYGZdq{} is the character \PYGZdq{}}
\PYG{n+nb}{print}\PYG{p}{(}\PYG{l+s+s2}{\PYGZdq{}}\PYG{l+s+s2}{Mary said }\PYG{l+s+se}{\PYGZbs{}\PYGZdq{}}\PYG{l+s+s2}{I am here}\PYG{l+s+se}{\PYGZbs{}\PYGZdq{}}\PYG{l+s+s2}{.}\PYG{l+s+s2}{\PYGZdq{}}\PYG{p}{)}
\end{sphinxVerbatim}

\end{sphinxuseclass}\end{sphinxVerbatimInput}
\begin{sphinxVerbatimOutput}

\begin{sphinxuseclass}{cell_output}
\begin{sphinxVerbatim}[commandchars=\\\{\}]
Mary said \PYGZdq{}I am here\PYGZdq{}.
\end{sphinxVerbatim}

\end{sphinxuseclass}\end{sphinxVerbatimOutput}

\end{sphinxuseclass}
\begin{sphinxuseclass}{cell}\begin{sphinxVerbatimInput}

\begin{sphinxuseclass}{cell_input}
\begin{sphinxVerbatim}[commandchars=\\\{\}]
\PYG{c+c1}{\PYGZsh{} \PYGZbs{}n is a new line}
\PYG{n+nb}{print}\PYG{p}{(}\PYG{l+s+s1}{\PYGZsq{}}\PYG{l+s+s1}{John}\PYG{l+s+se}{\PYGZbs{}n}\PYG{l+s+s1}{Steinbeck}\PYG{l+s+s1}{\PYGZsq{}}\PYG{p}{)}
\end{sphinxVerbatim}

\end{sphinxuseclass}\end{sphinxVerbatimInput}
\begin{sphinxVerbatimOutput}

\begin{sphinxuseclass}{cell_output}
\begin{sphinxVerbatim}[commandchars=\\\{\}]
John
Steinbeck
\end{sphinxVerbatim}

\end{sphinxuseclass}\end{sphinxVerbatimOutput}

\end{sphinxuseclass}
\begin{sphinxuseclass}{cell}\begin{sphinxVerbatimInput}

\begin{sphinxuseclass}{cell_input}
\begin{sphinxVerbatim}[commandchars=\\\{\}]
\PYG{c+c1}{\PYGZsh{} \PYGZbs{}t is a tab }
\PYG{c+c1}{\PYGZsh{} Inserts spaces up to the next tab stop, which occurs every 8th character.}
\PYG{n+nb}{print}\PYG{p}{(}\PYG{l+s+s1}{\PYGZsq{}}\PYG{l+s+s1}{John}\PYG{l+s+se}{\PYGZbs{}t}\PYG{l+s+s1}{Steinbeck}\PYG{l+s+s1}{\PYGZsq{}}\PYG{p}{)}
\PYG{n+nb}{print}\PYG{p}{(}\PYG{l+s+s1}{\PYGZsq{}}\PYG{l+s+s1}{John}\PYG{l+s+s1}{\PYGZsq{}}\PYG{o}{+}\PYG{l+s+s1}{\PYGZsq{}}\PYG{l+s+s1}{ }\PYG{l+s+s1}{\PYGZsq{}}\PYG{o}{*}\PYG{l+m+mi}{4}\PYG{o}{+}\PYG{l+s+s1}{\PYGZsq{}}\PYG{l+s+s1}{Steinbeck}\PYG{l+s+s1}{\PYGZsq{}}\PYG{p}{)}
\end{sphinxVerbatim}

\end{sphinxuseclass}\end{sphinxVerbatimInput}
\begin{sphinxVerbatimOutput}

\begin{sphinxuseclass}{cell_output}
\begin{sphinxVerbatim}[commandchars=\\\{\}]
John	Steinbeck
John    Steinbeck
\end{sphinxVerbatim}

\end{sphinxuseclass}\end{sphinxVerbatimOutput}

\end{sphinxuseclass}
\begin{sphinxuseclass}{cell}\begin{sphinxVerbatimInput}

\begin{sphinxuseclass}{cell_input}
\begin{sphinxVerbatim}[commandchars=\\\{\}]
\PYG{n+nb}{print}\PYG{p}{(}\PYG{l+s+s1}{\PYGZsq{}}\PYG{l+s+s1}{Mark}\PYG{l+s+se}{\PYGZbs{}t}\PYG{l+s+s1}{Twain}\PYG{l+s+s1}{\PYGZsq{}}\PYG{p}{)}
\PYG{n+nb}{print}\PYG{p}{(}\PYG{l+s+s1}{\PYGZsq{}}\PYG{l+s+s1}{Mark}\PYG{l+s+s1}{\PYGZsq{}}\PYG{o}{+}\PYG{l+s+s1}{\PYGZsq{}}\PYG{l+s+s1}{ }\PYG{l+s+s1}{\PYGZsq{}}\PYG{o}{*}\PYG{l+m+mi}{4}\PYG{o}{+}\PYG{l+s+s1}{\PYGZsq{}}\PYG{l+s+s1}{Twain}\PYG{l+s+s1}{\PYGZsq{}}\PYG{p}{)}
\end{sphinxVerbatim}

\end{sphinxuseclass}\end{sphinxVerbatimInput}
\begin{sphinxVerbatimOutput}

\begin{sphinxuseclass}{cell_output}
\begin{sphinxVerbatim}[commandchars=\\\{\}]
Mark	Twain
Mark    Twain
\end{sphinxVerbatim}

\end{sphinxuseclass}\end{sphinxVerbatimOutput}

\end{sphinxuseclass}\begin{itemize}
\item {} 
\sphinxAtStartPar
\textbackslash{}b (backspace) moves one character back,  deleting the preceding character.

\end{itemize}

\begin{sphinxuseclass}{cell}\begin{sphinxVerbatimInput}

\begin{sphinxuseclass}{cell_input}
\begin{sphinxVerbatim}[commandchars=\\\{\}]
\PYG{n+nb}{print}\PYG{p}{(}\PYG{l+s+s1}{\PYGZsq{}}\PYG{l+s+s1}{John}\PYG{l+s+se}{\PYGZbs{}b}\PYG{l+s+s1}{ Steinbeck}\PYG{l+s+s1}{\PYGZsq{}}\PYG{p}{)}
\end{sphinxVerbatim}

\end{sphinxuseclass}\end{sphinxVerbatimInput}
\begin{sphinxVerbatimOutput}

\begin{sphinxuseclass}{cell_output}
\begin{sphinxVerbatim}[commandchars=\\\{\}]
John Steinbeck
\end{sphinxVerbatim}

\end{sphinxuseclass}\end{sphinxVerbatimOutput}

\end{sphinxuseclass}\begin{itemize}
\item {} 
\sphinxAtStartPar
\textbackslash{}b\textbackslash{}b (two backspaces) moves two characters back,  deleting the preceding two characters.

\end{itemize}

\begin{sphinxuseclass}{cell}\begin{sphinxVerbatimInput}

\begin{sphinxuseclass}{cell_input}
\begin{sphinxVerbatim}[commandchars=\\\{\}]
\PYG{n+nb}{print}\PYG{p}{(}\PYG{l+s+s1}{\PYGZsq{}}\PYG{l+s+s1}{John}\PYG{l+s+se}{\PYGZbs{}b}\PYG{l+s+se}{\PYGZbs{}b}\PYG{l+s+s1}{ Steinbeck}\PYG{l+s+s1}{\PYGZsq{}}\PYG{p}{)}
\end{sphinxVerbatim}

\end{sphinxuseclass}\end{sphinxVerbatimInput}
\begin{sphinxVerbatimOutput}

\begin{sphinxuseclass}{cell_output}
\begin{sphinxVerbatim}[commandchars=\\\{\}]
John Steinbeck
\end{sphinxVerbatim}

\end{sphinxuseclass}\end{sphinxVerbatimOutput}

\end{sphinxuseclass}
\begin{sphinxuseclass}{cell}\begin{sphinxVerbatimInput}

\begin{sphinxuseclass}{cell_input}
\begin{sphinxVerbatim}[commandchars=\\\{\}]
\PYG{c+c1}{\PYGZsh{} \PYGZbs{}\PYGZbs{} is the \PYGZbs{} (backslash) character}
\PYG{n+nb}{print}\PYG{p}{(}\PYG{l+s+s1}{\PYGZsq{}}\PYG{l+s+s1}{John}\PYG{l+s+se}{\PYGZbs{}\PYGZbs{}}\PYG{l+s+s1}{ Steinbeck}\PYG{l+s+s1}{\PYGZsq{}}\PYG{p}{)}
\end{sphinxVerbatim}

\end{sphinxuseclass}\end{sphinxVerbatimInput}
\begin{sphinxVerbatimOutput}

\begin{sphinxuseclass}{cell_output}
\begin{sphinxVerbatim}[commandchars=\\\{\}]
John\PYGZbs{} Steinbeck
\end{sphinxVerbatim}

\end{sphinxuseclass}\end{sphinxVerbatimOutput}

\end{sphinxuseclass}

\section{Raw Strings}
\label{\detokenize{strings:raw-strings}}\begin{itemize}
\item {} 
\sphinxAtStartPar
Each character in a raw string has no special meaning; they are just characters.

\item {} 
\sphinxAtStartPar
It is created using \sphinxcode{\sphinxupquote{r}} in front of the string: \sphinxcode{\sphinxupquote{r'text'}}

\item {} 
\sphinxAtStartPar
In the example below, \sphinxcode{\sphinxupquote{\textbackslash{}t,\textbackslash{}n,\textbackslash{}',\textbackslash{}"}} have no special meanings; they are just characters \sphinxcode{\sphinxupquote{\textbackslash{}, t, n,',"}}

\end{itemize}

\begin{sphinxuseclass}{cell}\begin{sphinxVerbatimInput}

\begin{sphinxuseclass}{cell_input}
\begin{sphinxVerbatim}[commandchars=\\\{\}]
\PYG{c+c1}{\PYGZsh{} \PYGZbs{}t means two characters, \PYGZbs{} and t. }
\PYG{c+c1}{\PYGZsh{} \PYGZbs{}t does not mean a tab in a raw string}

\PYG{n}{text} \PYG{o}{=} \PYG{l+s+sa}{r}\PYG{l+s+s1}{\PYGZsq{}}\PYG{l+s+s1}{Good }\PYG{l+s+s1}{\PYGZbs{}}\PYG{l+s+s1}{t bye.}\PYG{l+s+s1}{\PYGZsq{}}
\PYG{n+nb}{print}\PYG{p}{(}\PYG{n}{text}\PYG{p}{)}
\end{sphinxVerbatim}

\end{sphinxuseclass}\end{sphinxVerbatimInput}
\begin{sphinxVerbatimOutput}

\begin{sphinxuseclass}{cell_output}
\begin{sphinxVerbatim}[commandchars=\\\{\}]
Good \PYGZbs{}t bye.
\end{sphinxVerbatim}

\end{sphinxuseclass}\end{sphinxVerbatimOutput}

\end{sphinxuseclass}
\begin{sphinxuseclass}{cell}\begin{sphinxVerbatimInput}

\begin{sphinxuseclass}{cell_input}
\begin{sphinxVerbatim}[commandchars=\\\{\}]
\PYG{n}{text} \PYG{o}{=} \PYG{l+s+sa}{r}\PYG{l+s+s1}{\PYGZsq{}}\PYG{l+s+s1}{Hello.}\PYG{l+s+s1}{\PYGZbs{}}\PYG{l+s+s1}{t my name}\PYG{l+s+s1}{\PYGZbs{}}\PYG{l+s+s1}{n is Tom. I am}\PYG{l+s+se}{\PYGZbs{}\PYGZsq{}}\PYG{l+s+s1}{ from}\PYG{l+s+s1}{\PYGZbs{}}\PYG{l+s+s1}{\PYGZdq{}}\PYG{l+s+s1}{ England.}\PYG{l+s+s1}{\PYGZsq{}}
\PYG{n+nb}{print}\PYG{p}{(}\PYG{n}{text}\PYG{p}{)}
\end{sphinxVerbatim}

\end{sphinxuseclass}\end{sphinxVerbatimInput}
\begin{sphinxVerbatimOutput}

\begin{sphinxuseclass}{cell_output}
\begin{sphinxVerbatim}[commandchars=\\\{\}]
Hello.\PYGZbs{}t my name\PYGZbs{}n is Tom. I am\PYGZbs{}\PYGZsq{} from\PYGZbs{}\PYGZdq{} England.
\end{sphinxVerbatim}

\end{sphinxuseclass}\end{sphinxVerbatimOutput}

\end{sphinxuseclass}

\section{f\sphinxhyphen{}strings}
\label{\detokenize{strings:f-strings}}\begin{itemize}
\item {} 
\sphinxAtStartPar
It is a great way to combine constants and variable values.

\item {} 
\sphinxAtStartPar
It is in the form of: \sphinxcode{\sphinxupquote{f'text \{variable\} text'}}

\item {} 
\sphinxAtStartPar
Variables are enclosed in curly brackets \sphinxcode{\sphinxupquote{\{variable\}}} called placeholders.

\item {} 
\sphinxAtStartPar
An f\sphinxhyphen{}string generates a new string.

\item {} 
\sphinxAtStartPar
You can also perform rounding or algebraic operations within curly brackets.

\item {} 
\sphinxAtStartPar
It is much easier to use f\sphinxhyphen{}strings than comma\sphinxhyphen{}separated values in a print() function.

\end{itemize}

\begin{sphinxuseclass}{cell}\begin{sphinxVerbatimInput}

\begin{sphinxuseclass}{cell_input}
\begin{sphinxVerbatim}[commandchars=\\\{\}]
\PYG{n}{name} \PYG{o}{=} \PYG{l+s+s1}{\PYGZsq{}}\PYG{l+s+s1}{Tom}\PYG{l+s+s1}{\PYGZsq{}}
\PYG{n}{country} \PYG{o}{=} \PYG{l+s+s1}{\PYGZsq{}}\PYG{l+s+s1}{Spain}\PYG{l+s+s1}{\PYGZsq{}}
\PYG{n}{age} \PYG{o}{=} \PYG{l+m+mi}{25}
\PYG{n}{weight} \PYG{o}{=} \PYG{l+m+mf}{173.6294}
\end{sphinxVerbatim}

\end{sphinxuseclass}\end{sphinxVerbatimInput}

\end{sphinxuseclass}\begin{itemize}
\item {} 
\sphinxAtStartPar
You can compare the next two cells below to see the advantage of using f\sphinxhyphen{}strings.

\end{itemize}

\begin{sphinxuseclass}{cell}\begin{sphinxVerbatimInput}

\begin{sphinxuseclass}{cell_input}
\begin{sphinxVerbatim}[commandchars=\\\{\}]
\PYG{c+c1}{\PYGZsh{} using an f\PYGZhy{}string}
\PYG{n+nb}{print}\PYG{p}{(}\PYG{l+s+sa}{f}\PYG{l+s+s1}{\PYGZsq{}}\PYG{l+s+s1}{My name is }\PYG{l+s+si}{\PYGZob{}}\PYG{n}{name}\PYG{l+s+si}{\PYGZcb{}}\PYG{l+s+s1}{.}\PYG{l+s+s1}{\PYGZsq{}}\PYG{p}{)}
\end{sphinxVerbatim}

\end{sphinxuseclass}\end{sphinxVerbatimInput}
\begin{sphinxVerbatimOutput}

\begin{sphinxuseclass}{cell_output}
\begin{sphinxVerbatim}[commandchars=\\\{\}]
My name is Tom.
\end{sphinxVerbatim}

\end{sphinxuseclass}\end{sphinxVerbatimOutput}

\end{sphinxuseclass}
\begin{sphinxuseclass}{cell}\begin{sphinxVerbatimInput}

\begin{sphinxuseclass}{cell_input}
\begin{sphinxVerbatim}[commandchars=\\\{\}]
\PYG{c+c1}{\PYGZsh{} longer way}
\PYG{n+nb}{print}\PYG{p}{(}\PYG{l+s+s1}{\PYGZsq{}}\PYG{l+s+s1}{My name is }\PYG{l+s+s1}{\PYGZsq{}}\PYG{p}{,} \PYG{n}{name}\PYG{p}{,} \PYG{l+s+s1}{\PYGZsq{}}\PYG{l+s+s1}{.}\PYG{l+s+s1}{\PYGZsq{}}\PYG{p}{,} \PYG{n}{sep}\PYG{o}{=}\PYG{l+s+s1}{\PYGZsq{}}\PYG{l+s+s1}{\PYGZsq{}}\PYG{p}{)}
\end{sphinxVerbatim}

\end{sphinxuseclass}\end{sphinxVerbatimInput}
\begin{sphinxVerbatimOutput}

\begin{sphinxuseclass}{cell_output}
\begin{sphinxVerbatim}[commandchars=\\\{\}]
My name is Tom.
\end{sphinxVerbatim}

\end{sphinxuseclass}\end{sphinxVerbatimOutput}

\end{sphinxuseclass}
\begin{sphinxuseclass}{cell}\begin{sphinxVerbatimInput}

\begin{sphinxuseclass}{cell_input}
\begin{sphinxVerbatim}[commandchars=\\\{\}]
\PYG{c+c1}{\PYGZsh{} using an f\PYGZhy{}string}

\PYG{n}{text} \PYG{o}{=} \PYG{l+s+sa}{f}\PYG{l+s+s1}{\PYGZsq{}}\PYG{l+s+s1}{My name is }\PYG{l+s+si}{\PYGZob{}}\PYG{n}{name}\PYG{l+s+si}{\PYGZcb{}}\PYG{l+s+s1}{.}\PYG{l+s+s1}{\PYGZsq{}}   \PYG{c+c1}{\PYGZsh{} text is a string}
\PYG{n+nb}{print}\PYG{p}{(}\PYG{n}{text}\PYG{p}{)}
\end{sphinxVerbatim}

\end{sphinxuseclass}\end{sphinxVerbatimInput}
\begin{sphinxVerbatimOutput}

\begin{sphinxuseclass}{cell_output}
\begin{sphinxVerbatim}[commandchars=\\\{\}]
My name is Tom.
\end{sphinxVerbatim}

\end{sphinxuseclass}\end{sphinxVerbatimOutput}

\end{sphinxuseclass}
\begin{sphinxuseclass}{cell}\begin{sphinxVerbatimInput}

\begin{sphinxuseclass}{cell_input}
\begin{sphinxVerbatim}[commandchars=\\\{\}]
\PYG{c+c1}{\PYGZsh{} Multiple placholders}
\PYG{n+nb}{print}\PYG{p}{(}\PYG{l+s+sa}{f}\PYG{l+s+s1}{\PYGZsq{}}\PYG{l+s+s1}{My name is }\PYG{l+s+si}{\PYGZob{}}\PYG{n}{name}\PYG{l+s+si}{\PYGZcb{}}\PYG{l+s+s1}{ and I am from }\PYG{l+s+si}{\PYGZob{}}\PYG{n}{country}\PYG{l+s+si}{\PYGZcb{}}\PYG{l+s+s1}{. I am }\PYG{l+s+si}{\PYGZob{}}\PYG{n}{age}\PYG{l+s+si}{\PYGZcb{}}\PYG{l+s+s1}{ years old.}\PYG{l+s+s1}{\PYGZsq{}}\PYG{p}{)}
\end{sphinxVerbatim}

\end{sphinxuseclass}\end{sphinxVerbatimInput}
\begin{sphinxVerbatimOutput}

\begin{sphinxuseclass}{cell_output}
\begin{sphinxVerbatim}[commandchars=\\\{\}]
My name is Tom and I am from Spain. I am 25 years old.
\end{sphinxVerbatim}

\end{sphinxuseclass}\end{sphinxVerbatimOutput}

\end{sphinxuseclass}
\begin{sphinxuseclass}{cell}\begin{sphinxVerbatimInput}

\begin{sphinxuseclass}{cell_input}
\begin{sphinxVerbatim}[commandchars=\\\{\}]
\PYG{c+c1}{\PYGZsh{} algebraic operation inside curly brackets}
\PYG{n+nb}{print}\PYG{p}{(}\PYG{l+s+sa}{f}\PYG{l+s+s1}{\PYGZsq{}}\PYG{l+s+s1}{I will be }\PYG{l+s+si}{\PYGZob{}}\PYG{n}{age}\PYG{o}{+}\PYG{l+m+mi}{1}\PYG{l+s+si}{\PYGZcb{}}\PYG{l+s+s1}{ years old next year.}\PYG{l+s+s1}{\PYGZsq{}}\PYG{p}{)}
\end{sphinxVerbatim}

\end{sphinxuseclass}\end{sphinxVerbatimInput}
\begin{sphinxVerbatimOutput}

\begin{sphinxuseclass}{cell_output}
\begin{sphinxVerbatim}[commandchars=\\\{\}]
I will be 26 years old next year.
\end{sphinxVerbatim}

\end{sphinxuseclass}\end{sphinxVerbatimOutput}

\end{sphinxuseclass}
\begin{sphinxuseclass}{cell}\begin{sphinxVerbatimInput}

\begin{sphinxuseclass}{cell_input}
\begin{sphinxVerbatim}[commandchars=\\\{\}]
\PYG{c+c1}{\PYGZsh{} rounding by using the round() function}

\PYG{n+nb}{print}\PYG{p}{(}\PYG{l+s+sa}{f}\PYG{l+s+s1}{\PYGZsq{}}\PYG{l+s+s1}{My weight is }\PYG{l+s+si}{\PYGZob{}}\PYG{n}{weight}\PYG{l+s+si}{\PYGZcb{}}\PYG{l+s+s1}{.}\PYG{l+s+s1}{\PYGZsq{}}\PYG{p}{)}
\PYG{n+nb}{print}\PYG{p}{(}\PYG{l+s+sa}{f}\PYG{l+s+s1}{\PYGZsq{}}\PYG{l+s+s1}{My rounded weight is }\PYG{l+s+si}{\PYGZob{}}\PYG{n+nb}{round}\PYG{p}{(}\PYG{n}{weight}\PYG{p}{,}\PYG{l+m+mi}{2}\PYG{p}{)}\PYG{l+s+si}{\PYGZcb{}}\PYG{l+s+s1}{.}\PYG{l+s+s1}{\PYGZsq{}}\PYG{p}{)}
\end{sphinxVerbatim}

\end{sphinxuseclass}\end{sphinxVerbatimInput}
\begin{sphinxVerbatimOutput}

\begin{sphinxuseclass}{cell_output}
\begin{sphinxVerbatim}[commandchars=\\\{\}]
My weight is 173.6294.
My rounded weight is 173.63.
\end{sphinxVerbatim}

\end{sphinxuseclass}\end{sphinxVerbatimOutput}

\end{sphinxuseclass}
\begin{sphinxuseclass}{cell}\begin{sphinxVerbatimInput}

\begin{sphinxuseclass}{cell_input}
\begin{sphinxVerbatim}[commandchars=\\\{\}]
\PYG{c+c1}{\PYGZsh{} rounding by a different way}

\PYG{n+nb}{print}\PYG{p}{(}\PYG{l+s+sa}{f}\PYG{l+s+s1}{\PYGZsq{}}\PYG{l+s+s1}{My weight is }\PYG{l+s+si}{\PYGZob{}}\PYG{n}{weight}\PYG{l+s+si}{\PYGZcb{}}\PYG{l+s+s1}{.}\PYG{l+s+s1}{\PYGZsq{}}\PYG{p}{)}
\PYG{n+nb}{print}\PYG{p}{(}\PYG{l+s+sa}{f}\PYG{l+s+s1}{\PYGZsq{}}\PYG{l+s+s1}{My rounded weight is }\PYG{l+s+si}{\PYGZob{}}\PYG{n}{weight}\PYG{l+s+si}{:}\PYG{l+s+s1}{.2f}\PYG{l+s+si}{\PYGZcb{}}\PYG{l+s+s1}{.}\PYG{l+s+s1}{\PYGZsq{}}\PYG{p}{)} \PYG{c+c1}{\PYGZsh{} 2 means second decimal place (hundredth), f means float}
\end{sphinxVerbatim}

\end{sphinxuseclass}\end{sphinxVerbatimInput}
\begin{sphinxVerbatimOutput}

\begin{sphinxuseclass}{cell_output}
\begin{sphinxVerbatim}[commandchars=\\\{\}]
My weight is 173.6294.
My rounded weight is 173.63.
\end{sphinxVerbatim}

\end{sphinxuseclass}\end{sphinxVerbatimOutput}

\end{sphinxuseclass}

\section{Unicode Characters}
\label{\detokenize{strings:unicode-characters}}\begin{itemize}
\item {} 
\sphinxAtStartPar
These are symbols, accented letters, non\sphinxhyphen{}Latin characters, and emojis—kind of different characters.

\item {} 
\sphinxAtStartPar
You can find the list of Unicode characters on the official website of \sphinxhref{https://home.unicode.org/}{Unicodes}.

\item {} 
\sphinxAtStartPar
Each Unicode character has a code that is unique to it.
\begin{itemize}
\item {} 
\sphinxAtStartPar
If the code has four characters, use
\begin{itemize}
\item {} 
\sphinxAtStartPar
\sphinxcode{\sphinxupquote{\textbackslash{}uXXXX}} where XXXX is the code.

\end{itemize}

\item {} 
\sphinxAtStartPar
If a code has four characters or more, pad it with 0 from the left to make the length of the code eight and use:
\begin{itemize}
\item {} 
\sphinxAtStartPar
\sphinxcode{\sphinxupquote{\textbackslash{}Uxxxxxxxx}} where xxxxxxxx is the 0\sphinxhyphen{}padded code.

\end{itemize}

\end{itemize}

\end{itemize}

\begin{sphinxuseclass}{cell}\begin{sphinxVerbatimInput}

\begin{sphinxuseclass}{cell_input}
\begin{sphinxVerbatim}[commandchars=\\\{\}]
\PYG{c+c1}{\PYGZsh{} unicode code is 1F639, you need to add 3 zeros to the left}
\PYG{n+nb}{print}\PYG{p}{(}\PYG{l+s+s1}{\PYGZsq{}}\PYG{l+s+se}{\PYGZbs{}U0001F639}\PYG{l+s+s1}{\PYGZsq{}}\PYG{p}{)}
\end{sphinxVerbatim}

\end{sphinxuseclass}\end{sphinxVerbatimInput}
\begin{sphinxVerbatimOutput}

\begin{sphinxuseclass}{cell_output}
\begin{sphinxVerbatim}[commandchars=\\\{\}]
😹
\end{sphinxVerbatim}

\end{sphinxuseclass}\end{sphinxVerbatimOutput}

\end{sphinxuseclass}
\begin{sphinxuseclass}{cell}\begin{sphinxVerbatimInput}

\begin{sphinxuseclass}{cell_input}
\begin{sphinxVerbatim}[commandchars=\\\{\}]
\PYG{c+c1}{\PYGZsh{} unicode code is 1F602, you need to add 3 zeros to the left}
\PYG{n+nb}{print}\PYG{p}{(}\PYG{l+s+s1}{\PYGZsq{}}\PYG{l+s+se}{\PYGZbs{}U0001F602}\PYG{l+s+s1}{\PYGZsq{}}\PYG{p}{)}
\end{sphinxVerbatim}

\end{sphinxuseclass}\end{sphinxVerbatimInput}
\begin{sphinxVerbatimOutput}

\begin{sphinxuseclass}{cell_output}
\begin{sphinxVerbatim}[commandchars=\\\{\}]
😂
\end{sphinxVerbatim}

\end{sphinxuseclass}\end{sphinxVerbatimOutput}

\end{sphinxuseclass}
\begin{sphinxuseclass}{cell}\begin{sphinxVerbatimInput}

\begin{sphinxuseclass}{cell_input}
\begin{sphinxVerbatim}[commandchars=\\\{\}]
\PYG{c+c1}{\PYGZsh{} unicode code is 2764}
\PYG{n+nb}{print}\PYG{p}{(}\PYG{l+s+s1}{\PYGZsq{}}\PYG{l+s+se}{\PYGZbs{}u2764}\PYG{l+s+s1}{\PYGZsq{}}\PYG{p}{)}
\end{sphinxVerbatim}

\end{sphinxuseclass}\end{sphinxVerbatimInput}
\begin{sphinxVerbatimOutput}

\begin{sphinxuseclass}{cell_output}
\begin{sphinxVerbatim}[commandchars=\\\{\}]
❤
\end{sphinxVerbatim}

\end{sphinxuseclass}\end{sphinxVerbatimOutput}

\end{sphinxuseclass}
\begin{sphinxuseclass}{cell}\begin{sphinxVerbatimInput}

\begin{sphinxuseclass}{cell_input}
\begin{sphinxVerbatim}[commandchars=\\\{\}]
\PYG{c+c1}{\PYGZsh{} unicode code is 2764, you need to add 4 zeros to the left}
\PYG{n+nb}{print}\PYG{p}{(}\PYG{l+s+s1}{\PYGZsq{}}\PYG{l+s+se}{\PYGZbs{}U00002764}\PYG{l+s+s1}{\PYGZsq{}}\PYG{p}{)}
\end{sphinxVerbatim}

\end{sphinxuseclass}\end{sphinxVerbatimInput}
\begin{sphinxVerbatimOutput}

\begin{sphinxuseclass}{cell_output}
\begin{sphinxVerbatim}[commandchars=\\\{\}]
❤
\end{sphinxVerbatim}

\end{sphinxuseclass}\end{sphinxVerbatimOutput}

\end{sphinxuseclass}

\section{Operations on strings}
\label{\detokenize{strings:operations-on-strings}}

\subsection{Concatenation}
\label{\detokenize{strings:concatenation}}\begin{itemize}
\item {} 
\sphinxAtStartPar
The \sphinxcode{\sphinxupquote{+}} operator is used to concatenate two strings.

\item {} 
\sphinxAtStartPar
String + String: combines two strings.

\end{itemize}

\begin{sphinxuseclass}{cell}\begin{sphinxVerbatimInput}

\begin{sphinxuseclass}{cell_input}
\begin{sphinxVerbatim}[commandchars=\\\{\}]
\PYG{n}{x} \PYG{o}{=} \PYG{l+s+s1}{\PYGZsq{}}\PYG{l+s+s1}{John}\PYG{l+s+s1}{\PYGZsq{}}
\PYG{n}{y} \PYG{o}{=} \PYG{l+s+s1}{\PYGZsq{}}\PYG{l+s+s1}{Steinbeck}\PYG{l+s+s1}{\PYGZsq{}}
\end{sphinxVerbatim}

\end{sphinxuseclass}\end{sphinxVerbatimInput}

\end{sphinxuseclass}
\begin{sphinxuseclass}{cell}\begin{sphinxVerbatimInput}

\begin{sphinxuseclass}{cell_input}
\begin{sphinxVerbatim}[commandchars=\\\{\}]
\PYG{c+c1}{\PYGZsh{} concatenation of x and y}

\PYG{n}{name} \PYG{o}{=} \PYG{n}{x}\PYG{o}{+}\PYG{n}{y}
\PYG{n+nb}{print}\PYG{p}{(}\PYG{n}{name}\PYG{p}{)}
\end{sphinxVerbatim}

\end{sphinxuseclass}\end{sphinxVerbatimInput}
\begin{sphinxVerbatimOutput}

\begin{sphinxuseclass}{cell_output}
\begin{sphinxVerbatim}[commandchars=\\\{\}]
JohnSteinbeck
\end{sphinxVerbatim}

\end{sphinxuseclass}\end{sphinxVerbatimOutput}

\end{sphinxuseclass}
\begin{sphinxuseclass}{cell}\begin{sphinxVerbatimInput}

\begin{sphinxuseclass}{cell_input}
\begin{sphinxVerbatim}[commandchars=\\\{\}]
\PYG{c+c1}{\PYGZsh{} add a space between x and y}
\PYG{c+c1}{\PYGZsh{} concatenation of x, space, and y}

\PYG{n}{name} \PYG{o}{=} \PYG{n}{x}\PYG{o}{+}\PYG{l+s+s1}{\PYGZsq{}}\PYG{l+s+s1}{ }\PYG{l+s+s1}{\PYGZsq{}}\PYG{o}{+}\PYG{n}{y}
\PYG{n+nb}{print}\PYG{p}{(}\PYG{n}{name}\PYG{p}{)}
\end{sphinxVerbatim}

\end{sphinxuseclass}\end{sphinxVerbatimInput}
\begin{sphinxVerbatimOutput}

\begin{sphinxuseclass}{cell_output}
\begin{sphinxVerbatim}[commandchars=\\\{\}]
John Steinbeck
\end{sphinxVerbatim}

\end{sphinxuseclass}\end{sphinxVerbatimOutput}

\end{sphinxuseclass}

\subsection{Repetition}
\label{\detokenize{strings:repetition}}\begin{itemize}
\item {} 
\sphinxAtStartPar
The \sphinxcode{\sphinxupquote{*}} operator is used to repeat a string a certain number of times.

\item {} 
\sphinxAtStartPar
\sphinxcode{\sphinxupquote{String * Integer}} or \sphinxcode{\sphinxupquote{Integer * String}} makes copies of the string Integer many times.

\item {} 
\sphinxAtStartPar
Floats cannot be used for repetitions.

\end{itemize}

\begin{sphinxuseclass}{cell}\begin{sphinxVerbatimInput}

\begin{sphinxuseclass}{cell_input}
\begin{sphinxVerbatim}[commandchars=\\\{\}]
\PYG{c+c1}{\PYGZsh{} four copies of x}

\PYG{n}{fourtoms} \PYG{o}{=} \PYG{n}{x}\PYG{o}{*}\PYG{l+m+mi}{4}
\PYG{n+nb}{print}\PYG{p}{(}\PYG{n}{fourtoms}\PYG{p}{)}
\end{sphinxVerbatim}

\end{sphinxuseclass}\end{sphinxVerbatimInput}
\begin{sphinxVerbatimOutput}

\begin{sphinxuseclass}{cell_output}
\begin{sphinxVerbatim}[commandchars=\\\{\}]
JohnJohnJohnJohn
\end{sphinxVerbatim}

\end{sphinxuseclass}\end{sphinxVerbatimOutput}

\end{sphinxuseclass}
\begin{sphinxVerbatim}[commandchars=\\\{\}]
\PYG{c+c1}{\PYGZsh{} ERROR: float * str}
\PYG{c+c1}{\PYGZsh{} floats can not be sed for repetition}
\PYG{l+m+mf}{4.3}\PYG{o}{*}\PYG{l+s+s1}{\PYGZsq{}}\PYG{l+s+s1}{Hi}\PYG{l+s+s1}{\PYGZsq{}}
\end{sphinxVerbatim}

\begin{sphinxuseclass}{cell}\begin{sphinxVerbatimInput}

\begin{sphinxuseclass}{cell_input}
\begin{sphinxVerbatim}[commandchars=\\\{\}]
\PYG{c+c1}{\PYGZsh{} Triangle with the \PYGZsq{}\PYGZdl{}\PYGZsq{} character using repetitions.}
\PYG{n+nb}{print}\PYG{p}{(}\PYG{l+s+s1}{\PYGZsq{}}\PYG{l+s+s1}{\PYGZdl{}}\PYG{l+s+s1}{\PYGZsq{}}\PYG{p}{)}
\PYG{n+nb}{print}\PYG{p}{(}\PYG{l+s+s1}{\PYGZsq{}}\PYG{l+s+s1}{\PYGZdl{}}\PYG{l+s+s1}{\PYGZsq{}}\PYG{o}{*}\PYG{l+m+mi}{2}\PYG{p}{)}
\PYG{n+nb}{print}\PYG{p}{(}\PYG{l+s+s1}{\PYGZsq{}}\PYG{l+s+s1}{\PYGZdl{}}\PYG{l+s+s1}{\PYGZsq{}}\PYG{o}{*}\PYG{l+m+mi}{3}\PYG{p}{)}
\PYG{n+nb}{print}\PYG{p}{(}\PYG{l+s+s1}{\PYGZsq{}}\PYG{l+s+s1}{\PYGZdl{}}\PYG{l+s+s1}{\PYGZsq{}}\PYG{o}{*}\PYG{l+m+mi}{4}\PYG{p}{)}
\PYG{n+nb}{print}\PYG{p}{(}\PYG{l+s+s1}{\PYGZsq{}}\PYG{l+s+s1}{\PYGZdl{}}\PYG{l+s+s1}{\PYGZsq{}}\PYG{o}{*}\PYG{l+m+mi}{5}\PYG{p}{)}
\PYG{n+nb}{print}\PYG{p}{(}\PYG{l+s+s1}{\PYGZsq{}}\PYG{l+s+s1}{\PYGZdl{}}\PYG{l+s+s1}{\PYGZsq{}}\PYG{o}{*}\PYG{l+m+mi}{6}\PYG{p}{)}
\PYG{n+nb}{print}\PYG{p}{(}\PYG{l+s+s1}{\PYGZsq{}}\PYG{l+s+s1}{\PYGZdl{}}\PYG{l+s+s1}{\PYGZsq{}}\PYG{o}{*}\PYG{l+m+mi}{7}\PYG{p}{)}
\end{sphinxVerbatim}

\end{sphinxuseclass}\end{sphinxVerbatimInput}
\begin{sphinxVerbatimOutput}

\begin{sphinxuseclass}{cell_output}
\begin{sphinxVerbatim}[commandchars=\\\{\}]
\PYGZdl{}
\PYGZdl{}\PYGZdl{}
\PYGZdl{}\PYGZdl{}\PYGZdl{}
\PYGZdl{}\PYGZdl{}\PYGZdl{}\PYGZdl{}
\PYGZdl{}\PYGZdl{}\PYGZdl{}\PYGZdl{}\PYGZdl{}
\PYGZdl{}\PYGZdl{}\PYGZdl{}\PYGZdl{}\PYGZdl{}\PYGZdl{}
\PYGZdl{}\PYGZdl{}\PYGZdl{}\PYGZdl{}\PYGZdl{}\PYGZdl{}\PYGZdl{}
\end{sphinxVerbatim}

\end{sphinxuseclass}\end{sphinxVerbatimOutput}

\end{sphinxuseclass}
\begin{sphinxuseclass}{cell}\begin{sphinxVerbatimInput}

\begin{sphinxuseclass}{cell_input}
\begin{sphinxVerbatim}[commandchars=\\\{\}]
\PYG{c+c1}{\PYGZsh{} Triangle with the \PYGZsq{}\PYGZdl{}\PYGZsq{} and \PYGZsq{} \PYGZsq{} (space) characters using repetitions.\PYGZdq{}}
\PYG{n+nb}{print}\PYG{p}{(}\PYG{l+s+s1}{\PYGZsq{}}\PYG{l+s+s1}{ }\PYG{l+s+s1}{\PYGZsq{}}\PYG{o}{*}\PYG{l+m+mi}{6}\PYG{o}{+}\PYG{l+s+s1}{\PYGZsq{}}\PYG{l+s+s1}{\PYGZdl{}}\PYG{l+s+s1}{\PYGZsq{}}\PYG{p}{)}
\PYG{n+nb}{print}\PYG{p}{(}\PYG{l+s+s1}{\PYGZsq{}}\PYG{l+s+s1}{ }\PYG{l+s+s1}{\PYGZsq{}}\PYG{o}{*}\PYG{l+m+mi}{5}\PYG{o}{+}\PYG{l+s+s1}{\PYGZsq{}}\PYG{l+s+s1}{\PYGZdl{}}\PYG{l+s+s1}{\PYGZsq{}}\PYG{o}{*}\PYG{l+m+mi}{2}\PYG{p}{)}
\PYG{n+nb}{print}\PYG{p}{(}\PYG{l+s+s1}{\PYGZsq{}}\PYG{l+s+s1}{ }\PYG{l+s+s1}{\PYGZsq{}}\PYG{o}{*}\PYG{l+m+mi}{4}\PYG{o}{+}\PYG{l+s+s1}{\PYGZsq{}}\PYG{l+s+s1}{\PYGZdl{}}\PYG{l+s+s1}{\PYGZsq{}}\PYG{o}{*}\PYG{l+m+mi}{3}\PYG{p}{)}
\PYG{n+nb}{print}\PYG{p}{(}\PYG{l+s+s1}{\PYGZsq{}}\PYG{l+s+s1}{ }\PYG{l+s+s1}{\PYGZsq{}}\PYG{o}{*}\PYG{l+m+mi}{3}\PYG{o}{+}\PYG{l+s+s1}{\PYGZsq{}}\PYG{l+s+s1}{\PYGZdl{}}\PYG{l+s+s1}{\PYGZsq{}}\PYG{o}{*}\PYG{l+m+mi}{4}\PYG{p}{)}
\PYG{n+nb}{print}\PYG{p}{(}\PYG{l+s+s1}{\PYGZsq{}}\PYG{l+s+s1}{ }\PYG{l+s+s1}{\PYGZsq{}}\PYG{o}{*}\PYG{l+m+mi}{2}\PYG{o}{+}\PYG{l+s+s1}{\PYGZsq{}}\PYG{l+s+s1}{\PYGZdl{}}\PYG{l+s+s1}{\PYGZsq{}}\PYG{o}{*}\PYG{l+m+mi}{5}\PYG{p}{)}
\PYG{n+nb}{print}\PYG{p}{(}\PYG{l+s+s1}{\PYGZsq{}}\PYG{l+s+s1}{ }\PYG{l+s+s1}{\PYGZsq{}}\PYG{o}{*}\PYG{l+m+mi}{1}\PYG{o}{+}\PYG{l+s+s1}{\PYGZsq{}}\PYG{l+s+s1}{\PYGZdl{}}\PYG{l+s+s1}{\PYGZsq{}}\PYG{o}{*}\PYG{l+m+mi}{6}\PYG{p}{)}
\PYG{n+nb}{print}\PYG{p}{(}\PYG{l+s+s1}{\PYGZsq{}}\PYG{l+s+s1}{ }\PYG{l+s+s1}{\PYGZsq{}}\PYG{o}{*}\PYG{l+m+mi}{0}\PYG{o}{+}\PYG{l+s+s1}{\PYGZsq{}}\PYG{l+s+s1}{\PYGZdl{}}\PYG{l+s+s1}{\PYGZsq{}}\PYG{o}{*}\PYG{l+m+mi}{7}\PYG{p}{)}    \PYG{c+c1}{\PYGZsh{} you do not have to include the space part because it adds no space.}
\end{sphinxVerbatim}

\end{sphinxuseclass}\end{sphinxVerbatimInput}
\begin{sphinxVerbatimOutput}

\begin{sphinxuseclass}{cell_output}
\begin{sphinxVerbatim}[commandchars=\\\{\}]
      \PYGZdl{}
     \PYGZdl{}\PYGZdl{}
    \PYGZdl{}\PYGZdl{}\PYGZdl{}
   \PYGZdl{}\PYGZdl{}\PYGZdl{}\PYGZdl{}
  \PYGZdl{}\PYGZdl{}\PYGZdl{}\PYGZdl{}\PYGZdl{}
 \PYGZdl{}\PYGZdl{}\PYGZdl{}\PYGZdl{}\PYGZdl{}\PYGZdl{}
\PYGZdl{}\PYGZdl{}\PYGZdl{}\PYGZdl{}\PYGZdl{}\PYGZdl{}\PYGZdl{}
\end{sphinxVerbatim}

\end{sphinxuseclass}\end{sphinxVerbatimOutput}

\end{sphinxuseclass}

\section{Length Function}
\label{\detokenize{strings:length-function}}\begin{itemize}
\item {} 
\sphinxAtStartPar
The built\sphinxhyphen{}in \sphinxcode{\sphinxupquote{len()}} function returns the number of characters in a string.

\end{itemize}

\begin{sphinxuseclass}{cell}\begin{sphinxVerbatimInput}

\begin{sphinxuseclass}{cell_input}
\begin{sphinxVerbatim}[commandchars=\\\{\}]
\PYG{c+c1}{\PYGZsh{} there are five characters in hello}
\PYG{n+nb}{print}\PYG{p}{(}\PYG{n+nb}{len}\PYG{p}{(}\PYG{l+s+s1}{\PYGZsq{}}\PYG{l+s+s1}{hello}\PYG{l+s+s1}{\PYGZsq{}}\PYG{p}{)}\PYG{p}{)}
\end{sphinxVerbatim}

\end{sphinxuseclass}\end{sphinxVerbatimInput}
\begin{sphinxVerbatimOutput}

\begin{sphinxuseclass}{cell_output}
\begin{sphinxVerbatim}[commandchars=\\\{\}]
5
\end{sphinxVerbatim}

\end{sphinxuseclass}\end{sphinxVerbatimOutput}

\end{sphinxuseclass}
\begin{sphinxuseclass}{cell}\begin{sphinxVerbatimInput}

\begin{sphinxuseclass}{cell_input}
\begin{sphinxVerbatim}[commandchars=\\\{\}]
\PYG{c+c1}{\PYGZsh{} there are six characters in \PYGZsq{}hel lo\PYGZsq{}}
\PYG{c+c1}{\PYGZsh{} space is a character}
\PYG{n+nb}{print}\PYG{p}{(}\PYG{n+nb}{len}\PYG{p}{(}\PYG{l+s+s1}{\PYGZsq{}}\PYG{l+s+s1}{hel lo}\PYG{l+s+s1}{\PYGZsq{}}\PYG{p}{)}\PYG{p}{)}
\end{sphinxVerbatim}

\end{sphinxuseclass}\end{sphinxVerbatimInput}
\begin{sphinxVerbatimOutput}

\begin{sphinxuseclass}{cell_output}
\begin{sphinxVerbatim}[commandchars=\\\{\}]
6
\end{sphinxVerbatim}

\end{sphinxuseclass}\end{sphinxVerbatimOutput}

\end{sphinxuseclass}
\begin{sphinxuseclass}{cell}\begin{sphinxVerbatimInput}

\begin{sphinxuseclass}{cell_input}
\begin{sphinxVerbatim}[commandchars=\\\{\}]
\PYG{c+c1}{\PYGZsh{} there are six characters in \PYGZsq{}hel\PYGZbs{}nlo}
\PYG{c+c1}{\PYGZsh{} \PYGZbs{}n is a new line character (single character)}
\PYG{n+nb}{print}\PYG{p}{(}\PYG{n+nb}{len}\PYG{p}{(}\PYG{l+s+s1}{\PYGZsq{}}\PYG{l+s+s1}{hel}\PYG{l+s+se}{\PYGZbs{}n}\PYG{l+s+s1}{lo}\PYG{l+s+s1}{\PYGZsq{}}\PYG{p}{)}\PYG{p}{)}
\end{sphinxVerbatim}

\end{sphinxuseclass}\end{sphinxVerbatimInput}
\begin{sphinxVerbatimOutput}

\begin{sphinxuseclass}{cell_output}
\begin{sphinxVerbatim}[commandchars=\\\{\}]
6
\end{sphinxVerbatim}

\end{sphinxuseclass}\end{sphinxVerbatimOutput}

\end{sphinxuseclass}

\section{String Indexing}
\label{\detokenize{strings:string-indexing}}\begin{itemize}
\item {} 
\sphinxAtStartPar
Indexing is used to access individual characters or sets of characters.

\item {} 
\sphinxAtStartPar
Indexing starts with zero.

\item {} 
\sphinxAtStartPar
The index of the first character is 0.

\item {} 
\sphinxAtStartPar
The index of the second character is 1, and so on.

\item {} 
\sphinxAtStartPar
The index is written in square brackets: string{[}index{]}.

\item {} 
\sphinxAtStartPar
Negative numbers can also be used for indexing.

\item {} 
\sphinxAtStartPar
The index of the last character is \sphinxhyphen{}1.

\item {} 
\sphinxAtStartPar
The index of the second character from the end is \sphinxhyphen{}2, and so on.

\end{itemize}

\sphinxAtStartPar
\sphinxincludegraphics{{cal_index}.png}
\begin{itemize}
\item {} 
\sphinxAtStartPar
index of \sphinxcode{\sphinxupquote{C}} is 0 or \sphinxhyphen{}10

\item {} 
\sphinxAtStartPar
index of \sphinxcode{\sphinxupquote{first A}} is 1 or \sphinxhyphen{}9

\item {} 
\sphinxAtStartPar
index of \sphinxcode{\sphinxupquote{second A}} is 9 or \sphinxhyphen{}1

\item {} 
\sphinxAtStartPar
\sphinxstylestrong{Warning:} There is a character with index \sphinxhyphen{}10 (the second ‘A’), but there is no character with index positive 10 because indexing starts with 0, and it does not reach 10, which is the length of the string.
\begin{itemize}
\item {} 
\sphinxAtStartPar
For any string, there is no character with an index equal to the length of the string.

\end{itemize}

\end{itemize}

\begin{sphinxuseclass}{cell}\begin{sphinxVerbatimInput}

\begin{sphinxuseclass}{cell_input}
\begin{sphinxVerbatim}[commandchars=\\\{\}]
\PYG{n}{state} \PYG{o}{=} \PYG{l+s+s1}{\PYGZsq{}}\PYG{l+s+s1}{CALIFORNIA}\PYG{l+s+s1}{\PYGZsq{}}
\end{sphinxVerbatim}

\end{sphinxuseclass}\end{sphinxVerbatimInput}

\end{sphinxuseclass}
\begin{sphinxuseclass}{cell}\begin{sphinxVerbatimInput}

\begin{sphinxuseclass}{cell_input}
\begin{sphinxVerbatim}[commandchars=\\\{\}]
\PYG{c+c1}{\PYGZsh{} access the character at index 0 by using square brackets.\PYGZdq{}}
\PYG{n+nb}{print}\PYG{p}{(}\PYG{n}{state}\PYG{p}{[}\PYG{l+m+mi}{0}\PYG{p}{]}\PYG{p}{)}
\end{sphinxVerbatim}

\end{sphinxuseclass}\end{sphinxVerbatimInput}
\begin{sphinxVerbatimOutput}

\begin{sphinxuseclass}{cell_output}
\begin{sphinxVerbatim}[commandchars=\\\{\}]
C
\end{sphinxVerbatim}

\end{sphinxuseclass}\end{sphinxVerbatimOutput}

\end{sphinxuseclass}
\begin{sphinxuseclass}{cell}\begin{sphinxVerbatimInput}

\begin{sphinxuseclass}{cell_input}
\begin{sphinxVerbatim}[commandchars=\\\{\}]
\PYG{c+c1}{\PYGZsh{} Access the character at index 6 by using square brackets.\PYGZdq{}}
\PYG{n+nb}{print}\PYG{p}{(}\PYG{n}{state}\PYG{p}{[}\PYG{l+m+mi}{6}\PYG{p}{]}\PYG{p}{)}
\end{sphinxVerbatim}

\end{sphinxuseclass}\end{sphinxVerbatimInput}
\begin{sphinxVerbatimOutput}

\begin{sphinxuseclass}{cell_output}
\begin{sphinxVerbatim}[commandchars=\\\{\}]
R
\end{sphinxVerbatim}

\end{sphinxuseclass}\end{sphinxVerbatimOutput}

\end{sphinxuseclass}
\begin{sphinxuseclass}{cell}\begin{sphinxVerbatimInput}

\begin{sphinxuseclass}{cell_input}
\begin{sphinxVerbatim}[commandchars=\\\{\}]
\PYG{c+c1}{\PYGZsh{} Access the character at index \PYGZhy{}1 by using square brackets.\PYGZdq{}}
\PYG{n+nb}{print}\PYG{p}{(}\PYG{n}{state}\PYG{p}{[}\PYG{o}{\PYGZhy{}}\PYG{l+m+mi}{1}\PYG{p}{]}\PYG{p}{)}
\end{sphinxVerbatim}

\end{sphinxuseclass}\end{sphinxVerbatimInput}
\begin{sphinxVerbatimOutput}

\begin{sphinxuseclass}{cell_output}
\begin{sphinxVerbatim}[commandchars=\\\{\}]
A
\end{sphinxVerbatim}

\end{sphinxuseclass}\end{sphinxVerbatimOutput}

\end{sphinxuseclass}
\begin{sphinxuseclass}{cell}\begin{sphinxVerbatimInput}

\begin{sphinxuseclass}{cell_input}
\begin{sphinxVerbatim}[commandchars=\\\{\}]
\PYG{c+c1}{\PYGZsh{} Access the character at index \PYGZhy{}3 by using square brackets.\PYGZdq{}}
\PYG{n+nb}{print}\PYG{p}{(}\PYG{n}{state}\PYG{p}{[}\PYG{o}{\PYGZhy{}}\PYG{l+m+mi}{3}\PYG{p}{]}\PYG{p}{)}
\end{sphinxVerbatim}

\end{sphinxuseclass}\end{sphinxVerbatimInput}
\begin{sphinxVerbatimOutput}

\begin{sphinxuseclass}{cell_output}
\begin{sphinxVerbatim}[commandchars=\\\{\}]
N
\end{sphinxVerbatim}

\end{sphinxuseclass}\end{sphinxVerbatimOutput}

\end{sphinxuseclass}\begin{itemize}
\item {} 
\sphinxAtStartPar
There is an error in the following code because there is no character with index 10.

\end{itemize}

\begin{sphinxVerbatim}[commandchars=\\\{\}]
\PYG{c+c1}{\PYGZsh{} ERROR: out of range}
\PYG{n}{state}\PYG{p}{[}\PYG{l+m+mi}{10}\PYG{p}{]}
\end{sphinxVerbatim}
\begin{itemize}
\item {} 
\sphinxAtStartPar
The length of state is 10, and there is an error in the following code. This applies to all strings

\end{itemize}

\begin{sphinxVerbatim}[commandchars=\\\{\}]
\PYG{c+c1}{\PYGZsh{} ERROR: out of range}
\PYG{n}{state}\PYG{p}{[}\PYG{n+nb}{len}\PYG{p}{(}\PYG{n}{state}\PYG{p}{)}\PYG{p}{]}
\end{sphinxVerbatim}

\begin{sphinxuseclass}{cell}\begin{sphinxVerbatimInput}

\begin{sphinxuseclass}{cell_input}
\begin{sphinxVerbatim}[commandchars=\\\{\}]
\PYG{c+c1}{\PYGZsh{} There is no error in the following code because len(state) \PYGZhy{} 1 = 9 is the index of the last character}
\PYG{n+nb}{print}\PYG{p}{(}\PYG{n}{state}\PYG{p}{[}\PYG{n+nb}{len}\PYG{p}{(}\PYG{n}{state}\PYG{p}{)}\PYG{o}{\PYGZhy{}}\PYG{l+m+mi}{1}\PYG{p}{]}\PYG{p}{)}
\end{sphinxVerbatim}

\end{sphinxuseclass}\end{sphinxVerbatimInput}
\begin{sphinxVerbatimOutput}

\begin{sphinxuseclass}{cell_output}
\begin{sphinxVerbatim}[commandchars=\\\{\}]
A
\end{sphinxVerbatim}

\end{sphinxuseclass}\end{sphinxVerbatimOutput}

\end{sphinxuseclass}

\section{String Slices}
\label{\detokenize{strings:string-slices}}\begin{itemize}
\item {} 
\sphinxAtStartPar
You can access more than one character of a string by using index numbers.

\item {} 
\sphinxAtStartPar
It is in the form of \sphinxcode{\sphinxupquote{string{[}start: end{]}}} with inclusive start and exclusive end.

\item {} 
\sphinxAtStartPar
Use \sphinxcode{\sphinxupquote{:}} (colon) inside square brackets between the start and end indexes.

\item {} 
\sphinxAtStartPar
It consists of characters starting with index start up to the character with index end\sphinxhyphen{}1.

\item {} 
\sphinxAtStartPar
The character with index end is not included.

\item {} 
\sphinxAtStartPar
For example, \sphinxcode{\sphinxupquote{string{[}2:5{]}}} returns characters with indexes 2, 3, 4 (5 is not included).

\item {} 
\sphinxAtStartPar
For example, string{[}\sphinxhyphen{}4:\sphinxhyphen{}1{]} returns characters with indexes \sphinxhyphen{}4, \sphinxhyphen{}3, \sphinxhyphen{}2 (\sphinxhyphen{}1 is not included).

\item {} 
\sphinxAtStartPar
It returns a substring.

\end{itemize}

\sphinxAtStartPar
\sphinxincludegraphics{{cal_index2}.png}

\begin{sphinxuseclass}{cell}\begin{sphinxVerbatimInput}

\begin{sphinxuseclass}{cell_input}
\begin{sphinxVerbatim}[commandchars=\\\{\}]
\PYG{n}{state} \PYG{o}{=} \PYG{l+s+s1}{\PYGZsq{}}\PYG{l+s+s1}{CALIFORNIA}\PYG{l+s+s1}{\PYGZsq{}}
\end{sphinxVerbatim}

\end{sphinxuseclass}\end{sphinxVerbatimInput}

\end{sphinxuseclass}
\begin{sphinxuseclass}{cell}\begin{sphinxVerbatimInput}

\begin{sphinxuseclass}{cell_input}
\begin{sphinxVerbatim}[commandchars=\\\{\}]
\PYG{n+nb}{print}\PYG{p}{(}\PYG{n}{state}\PYG{p}{[}\PYG{l+m+mi}{2}\PYG{p}{:}\PYG{l+m+mi}{5}\PYG{p}{]}\PYG{p}{)}  \PYG{c+c1}{\PYGZsh{} index=2,3,4}
\end{sphinxVerbatim}

\end{sphinxuseclass}\end{sphinxVerbatimInput}
\begin{sphinxVerbatimOutput}

\begin{sphinxuseclass}{cell_output}
\begin{sphinxVerbatim}[commandchars=\\\{\}]
LIF
\end{sphinxVerbatim}

\end{sphinxuseclass}\end{sphinxVerbatimOutput}

\end{sphinxuseclass}
\begin{sphinxuseclass}{cell}\begin{sphinxVerbatimInput}

\begin{sphinxuseclass}{cell_input}
\begin{sphinxVerbatim}[commandchars=\\\{\}]
\PYG{n+nb}{print}\PYG{p}{(}\PYG{n}{state}\PYG{p}{[}\PYG{o}{\PYGZhy{}}\PYG{l+m+mi}{4}\PYG{p}{:}\PYG{o}{\PYGZhy{}}\PYG{l+m+mi}{1}\PYG{p}{]}\PYG{p}{)}  \PYG{c+c1}{\PYGZsh{} index=\PYGZhy{}4,\PYGZhy{}3,\PYGZhy{}2}
\end{sphinxVerbatim}

\end{sphinxuseclass}\end{sphinxVerbatimInput}
\begin{sphinxVerbatimOutput}

\begin{sphinxuseclass}{cell_output}
\begin{sphinxVerbatim}[commandchars=\\\{\}]
RNI
\end{sphinxVerbatim}

\end{sphinxuseclass}\end{sphinxVerbatimOutput}

\end{sphinxuseclass}\begin{itemize}
\item {} 
\sphinxAtStartPar
\sphinxcode{\sphinxupquote{string{[}:end{]}}}: the default value of start is 0, which means it starts from the very beginning.
\begin{itemize}
\item {} 
\sphinxAtStartPar
For example, \sphinxcode{\sphinxupquote{string{[}:5{]}}} returns characters with indexes 0, 1, 2, 3, 4 (5 is not included).

\end{itemize}

\item {} 
\sphinxAtStartPar
\sphinxcode{\sphinxupquote{string{[}start:{]}}}: the default value of end is the length, which means it goes all the way to the end.
\begin{itemize}
\item {} 
\sphinxAtStartPar
For example, \sphinxcode{\sphinxupquote{string{[}2:{]}}} returns characters with indexes 2, 3, 4, 5, 6, 7, 8, 9 (all characters starting from index 2).

\end{itemize}

\item {} 
\sphinxAtStartPar
\sphinxcode{\sphinxupquote{string{[}:{]}}}:  starting from the very beginning and going all the way to the end, representing the whole string.

\end{itemize}

\begin{sphinxuseclass}{cell}\begin{sphinxVerbatimInput}

\begin{sphinxuseclass}{cell_input}
\begin{sphinxVerbatim}[commandchars=\\\{\}]
\PYG{n+nb}{print}\PYG{p}{(}\PYG{n}{state}\PYG{p}{[}\PYG{l+m+mi}{2}\PYG{p}{:}\PYG{p}{]}\PYG{p}{)}  \PYG{c+c1}{\PYGZsh{} index = 2,3,4,5,6,7,8,9}
\end{sphinxVerbatim}

\end{sphinxuseclass}\end{sphinxVerbatimInput}
\begin{sphinxVerbatimOutput}

\begin{sphinxuseclass}{cell_output}
\begin{sphinxVerbatim}[commandchars=\\\{\}]
LIFORNIA
\end{sphinxVerbatim}

\end{sphinxuseclass}\end{sphinxVerbatimOutput}

\end{sphinxuseclass}
\begin{sphinxuseclass}{cell}\begin{sphinxVerbatimInput}

\begin{sphinxuseclass}{cell_input}
\begin{sphinxVerbatim}[commandchars=\\\{\}]
\PYG{n+nb}{print}\PYG{p}{(}\PYG{n}{state}\PYG{p}{[}\PYG{p}{:}\PYG{l+m+mi}{5}\PYG{p}{]}\PYG{p}{)}  \PYG{c+c1}{\PYGZsh{} index = 0,1,2,3,4}
\end{sphinxVerbatim}

\end{sphinxuseclass}\end{sphinxVerbatimInput}
\begin{sphinxVerbatimOutput}

\begin{sphinxuseclass}{cell_output}
\begin{sphinxVerbatim}[commandchars=\\\{\}]
CALIF
\end{sphinxVerbatim}

\end{sphinxuseclass}\end{sphinxVerbatimOutput}

\end{sphinxuseclass}
\begin{sphinxuseclass}{cell}\begin{sphinxVerbatimInput}

\begin{sphinxuseclass}{cell_input}
\begin{sphinxVerbatim}[commandchars=\\\{\}]
\PYG{n+nb}{print}\PYG{p}{(}\PYG{n}{state}\PYG{p}{[}\PYG{p}{:}\PYG{p}{]}\PYG{p}{)}  \PYG{c+c1}{\PYGZsh{} index = all of them = 0,1,2,...,9}
\end{sphinxVerbatim}

\end{sphinxuseclass}\end{sphinxVerbatimInput}
\begin{sphinxVerbatimOutput}

\begin{sphinxuseclass}{cell_output}
\begin{sphinxVerbatim}[commandchars=\\\{\}]
CALIFORNIA
\end{sphinxVerbatim}

\end{sphinxuseclass}\end{sphinxVerbatimOutput}

\end{sphinxuseclass}\begin{itemize}
\item {} 
\sphinxAtStartPar
Slicing can also be done by taking steps in the form of: \sphinxcode{\sphinxupquote{string{[}start: end: step{]}}}.

\item {} 
\sphinxAtStartPar
\sphinxcode{\sphinxupquote{string{[}start: end: step{]}}} means starting with the character at index = start up to the character at index = length \sphinxhyphen{} 1, as before, but not necessarily including all characters between them.

\item {} 
\sphinxAtStartPar
The first index is start, the second index is start + step, and the third index is start + 2 * step.

\item {} 
\sphinxAtStartPar
It continues in this way, but the largest index can be at most length \sphinxhyphen{} 1.

\item {} 
\sphinxAtStartPar
\sphinxcode{\sphinxupquote{step}} can also be considered as an increment, but it can also be a negative number.

\item {} 
\sphinxAtStartPar
The default value of step is 1.

\end{itemize}

\begin{sphinxuseclass}{cell}\begin{sphinxVerbatimInput}

\begin{sphinxuseclass}{cell_input}
\begin{sphinxVerbatim}[commandchars=\\\{\}]
\PYG{n+nb}{print}\PYG{p}{(}\PYG{n}{state}\PYG{p}{)}
\end{sphinxVerbatim}

\end{sphinxuseclass}\end{sphinxVerbatimInput}
\begin{sphinxVerbatimOutput}

\begin{sphinxuseclass}{cell_output}
\begin{sphinxVerbatim}[commandchars=\\\{\}]
CALIFORNIA
\end{sphinxVerbatim}

\end{sphinxuseclass}\end{sphinxVerbatimOutput}

\end{sphinxuseclass}
\begin{sphinxuseclass}{cell}\begin{sphinxVerbatimInput}

\begin{sphinxuseclass}{cell_input}
\begin{sphinxVerbatim}[commandchars=\\\{\}]
\PYG{n+nb}{print}\PYG{p}{(}\PYG{n}{state}\PYG{p}{[}\PYG{l+m+mi}{2}\PYG{p}{:}\PYG{l+m+mi}{7}\PYG{p}{:}\PYG{l+m+mi}{2}\PYG{p}{]}\PYG{p}{)}  \PYG{c+c1}{\PYGZsh{} index = 2,2+2=4, 4+2=6}
\end{sphinxVerbatim}

\end{sphinxuseclass}\end{sphinxVerbatimInput}
\begin{sphinxVerbatimOutput}

\begin{sphinxuseclass}{cell_output}
\begin{sphinxVerbatim}[commandchars=\\\{\}]
LFR
\end{sphinxVerbatim}

\end{sphinxuseclass}\end{sphinxVerbatimOutput}

\end{sphinxuseclass}
\begin{sphinxuseclass}{cell}\begin{sphinxVerbatimInput}

\begin{sphinxuseclass}{cell_input}
\begin{sphinxVerbatim}[commandchars=\\\{\}]
\PYG{n+nb}{print}\PYG{p}{(}\PYG{n}{state}\PYG{p}{[}\PYG{l+m+mi}{1}\PYG{p}{:}\PYG{l+m+mi}{8}\PYG{p}{:}\PYG{l+m+mi}{3}\PYG{p}{]}\PYG{p}{)}  \PYG{c+c1}{\PYGZsh{} index = 1,1+3=4, 4+3=7}
\end{sphinxVerbatim}

\end{sphinxuseclass}\end{sphinxVerbatimInput}
\begin{sphinxVerbatimOutput}

\begin{sphinxuseclass}{cell_output}
\begin{sphinxVerbatim}[commandchars=\\\{\}]
AFN
\end{sphinxVerbatim}

\end{sphinxuseclass}\end{sphinxVerbatimOutput}

\end{sphinxuseclass}
\begin{sphinxuseclass}{cell}\begin{sphinxVerbatimInput}

\begin{sphinxuseclass}{cell_input}
\begin{sphinxVerbatim}[commandchars=\\\{\}]
\PYG{n+nb}{print}\PYG{p}{(}\PYG{n}{state}\PYG{p}{[}\PYG{l+m+mi}{7}\PYG{p}{:}\PYG{l+m+mi}{2}\PYG{p}{:}\PYG{o}{\PYGZhy{}}\PYG{l+m+mi}{2}\PYG{p}{]}\PYG{p}{)}  \PYG{c+c1}{\PYGZsh{} index = 7,7+(\PYGZhy{}2)=5,5+(\PYGZhy{}2)=3 }
\end{sphinxVerbatim}

\end{sphinxuseclass}\end{sphinxVerbatimInput}
\begin{sphinxVerbatimOutput}

\begin{sphinxuseclass}{cell_output}
\begin{sphinxVerbatim}[commandchars=\\\{\}]
NOI
\end{sphinxVerbatim}

\end{sphinxuseclass}\end{sphinxVerbatimOutput}

\end{sphinxuseclass}
\begin{sphinxuseclass}{cell}\begin{sphinxVerbatimInput}

\begin{sphinxuseclass}{cell_input}
\begin{sphinxVerbatim}[commandchars=\\\{\}]
\PYG{n+nb}{print}\PYG{p}{(}\PYG{n}{state}\PYG{p}{[}\PYG{o}{\PYGZhy{}}\PYG{l+m+mi}{8}\PYG{p}{:}\PYG{o}{\PYGZhy{}}\PYG{l+m+mi}{2}\PYG{p}{:}\PYG{l+m+mi}{3}\PYG{p}{]}\PYG{p}{)}  \PYG{c+c1}{\PYGZsh{} index = \PYGZhy{}8, \PYGZhy{}8+3=\PYGZhy{}5}
\end{sphinxVerbatim}

\end{sphinxuseclass}\end{sphinxVerbatimInput}
\begin{sphinxVerbatimOutput}

\begin{sphinxuseclass}{cell_output}
\begin{sphinxVerbatim}[commandchars=\\\{\}]
LO
\end{sphinxVerbatim}

\end{sphinxuseclass}\end{sphinxVerbatimOutput}

\end{sphinxuseclass}
\begin{sphinxuseclass}{cell}\begin{sphinxVerbatimInput}

\begin{sphinxuseclass}{cell_input}
\begin{sphinxVerbatim}[commandchars=\\\{\}]
\PYG{c+c1}{\PYGZsh{} for negative step default value of start is 9 (\PYGZhy{}1)}
\PYG{c+c1}{\PYGZsh{} for negative step default value of end   is 0 (\PYGZhy{}10)}
\PYG{n+nb}{print}\PYG{p}{(}\PYG{n}{state}\PYG{p}{[}\PYG{p}{:}\PYG{p}{:}\PYG{o}{\PYGZhy{}}\PYG{l+m+mi}{1}\PYG{p}{]}\PYG{p}{)}  \PYG{c+c1}{\PYGZsh{} index = 9,8,...,0  }
\end{sphinxVerbatim}

\end{sphinxuseclass}\end{sphinxVerbatimInput}
\begin{sphinxVerbatimOutput}

\begin{sphinxuseclass}{cell_output}
\begin{sphinxVerbatim}[commandchars=\\\{\}]
AINROFILAC
\end{sphinxVerbatim}

\end{sphinxuseclass}\end{sphinxVerbatimOutput}

\end{sphinxuseclass}
\begin{sphinxuseclass}{cell}\begin{sphinxVerbatimInput}

\begin{sphinxuseclass}{cell_input}
\begin{sphinxVerbatim}[commandchars=\\\{\}]
\PYG{n+nb}{print}\PYG{p}{(}\PYG{n}{state}\PYG{p}{[}\PYG{o}{\PYGZhy{}}\PYG{l+m+mi}{3}\PYG{p}{:}\PYG{p}{:}\PYG{o}{\PYGZhy{}}\PYG{l+m+mi}{1}\PYG{p}{]}\PYG{p}{)}  \PYG{c+c1}{\PYGZsh{} index = \PYGZhy{}3,\PYGZhy{}4,...,\PYGZhy{}10}
\end{sphinxVerbatim}

\end{sphinxuseclass}\end{sphinxVerbatimInput}
\begin{sphinxVerbatimOutput}

\begin{sphinxuseclass}{cell_output}
\begin{sphinxVerbatim}[commandchars=\\\{\}]
NROFILAC
\end{sphinxVerbatim}

\end{sphinxuseclass}\end{sphinxVerbatimOutput}

\end{sphinxuseclass}
\begin{sphinxuseclass}{cell}\begin{sphinxVerbatimInput}

\begin{sphinxuseclass}{cell_input}
\begin{sphinxVerbatim}[commandchars=\\\{\}]
\PYG{n+nb}{print}\PYG{p}{(}\PYG{n}{state}\PYG{p}{[}\PYG{p}{:}\PYG{o}{\PYGZhy{}}\PYG{l+m+mi}{4}\PYG{p}{:}\PYG{o}{\PYGZhy{}}\PYG{l+m+mi}{1}\PYG{p}{]}\PYG{p}{)}  \PYG{c+c1}{\PYGZsh{} index = 9,8,7 or \PYGZhy{}1,\PYGZhy{}2,\PYGZhy{}3}
\end{sphinxVerbatim}

\end{sphinxuseclass}\end{sphinxVerbatimInput}
\begin{sphinxVerbatimOutput}

\begin{sphinxuseclass}{cell_output}
\begin{sphinxVerbatim}[commandchars=\\\{\}]
AIN
\end{sphinxVerbatim}

\end{sphinxuseclass}\end{sphinxVerbatimOutput}

\end{sphinxuseclass}

\section{String module}
\label{\detokenize{strings:string-module}}\begin{itemize}
\item {} 
\sphinxAtStartPar
It contains constants and functions to process strings, as well as some constants.

\item {} 
\sphinxAtStartPar
Use \sphinxcode{\sphinxupquote{help(string)}} for more explanations.

\end{itemize}

\begin{sphinxuseclass}{cell}\begin{sphinxVerbatimInput}

\begin{sphinxuseclass}{cell_input}
\begin{sphinxVerbatim}[commandchars=\\\{\}]
\PYG{c+c1}{\PYGZsh{} constants and functions}

\PYG{k+kn}{import} \PYG{n+nn}{string}
\PYG{n+nb}{print}\PYG{p}{(}\PYG{n+nb}{dir}\PYG{p}{(}\PYG{n}{string}\PYG{p}{)}\PYG{p}{)}
\end{sphinxVerbatim}

\end{sphinxuseclass}\end{sphinxVerbatimInput}
\begin{sphinxVerbatimOutput}

\begin{sphinxuseclass}{cell_output}
\begin{sphinxVerbatim}[commandchars=\\\{\}]
[\PYGZsq{}Formatter\PYGZsq{}, \PYGZsq{}Template\PYGZsq{}, \PYGZsq{}\PYGZus{}ChainMap\PYGZsq{}, \PYGZsq{}\PYGZus{}\PYGZus{}all\PYGZus{}\PYGZus{}\PYGZsq{}, \PYGZsq{}\PYGZus{}\PYGZus{}builtins\PYGZus{}\PYGZus{}\PYGZsq{}, \PYGZsq{}\PYGZus{}\PYGZus{}cached\PYGZus{}\PYGZus{}\PYGZsq{}, \PYGZsq{}\PYGZus{}\PYGZus{}doc\PYGZus{}\PYGZus{}\PYGZsq{}, \PYGZsq{}\PYGZus{}\PYGZus{}file\PYGZus{}\PYGZus{}\PYGZsq{}, \PYGZsq{}\PYGZus{}\PYGZus{}loader\PYGZus{}\PYGZus{}\PYGZsq{}, \PYGZsq{}\PYGZus{}\PYGZus{}name\PYGZus{}\PYGZus{}\PYGZsq{}, \PYGZsq{}\PYGZus{}\PYGZus{}package\PYGZus{}\PYGZus{}\PYGZsq{}, \PYGZsq{}\PYGZus{}\PYGZus{}spec\PYGZus{}\PYGZus{}\PYGZsq{}, \PYGZsq{}\PYGZus{}re\PYGZsq{}, \PYGZsq{}\PYGZus{}sentinel\PYGZus{}dict\PYGZsq{}, \PYGZsq{}\PYGZus{}string\PYGZsq{}, \PYGZsq{}ascii\PYGZus{}letters\PYGZsq{}, \PYGZsq{}ascii\PYGZus{}lowercase\PYGZsq{}, \PYGZsq{}ascii\PYGZus{}uppercase\PYGZsq{}, \PYGZsq{}capwords\PYGZsq{}, \PYGZsq{}digits\PYGZsq{}, \PYGZsq{}hexdigits\PYGZsq{}, \PYGZsq{}octdigits\PYGZsq{}, \PYGZsq{}printable\PYGZsq{}, \PYGZsq{}punctuation\PYGZsq{}, \PYGZsq{}whitespace\PYGZsq{}]
\end{sphinxVerbatim}

\end{sphinxuseclass}\end{sphinxVerbatimOutput}

\end{sphinxuseclass}
\begin{sphinxuseclass}{cell}\begin{sphinxVerbatimInput}

\begin{sphinxuseclass}{cell_input}
\begin{sphinxVerbatim}[commandchars=\\\{\}]
\PYG{c+c1}{\PYGZsh{} lowercase letters}
\PYG{n+nb}{print}\PYG{p}{(}\PYG{n}{string}\PYG{o}{.}\PYG{n}{ascii\PYGZus{}lowercase}\PYG{p}{)}
\end{sphinxVerbatim}

\end{sphinxuseclass}\end{sphinxVerbatimInput}
\begin{sphinxVerbatimOutput}

\begin{sphinxuseclass}{cell_output}
\begin{sphinxVerbatim}[commandchars=\\\{\}]
abcdefghijklmnopqrstuvwxyz
\end{sphinxVerbatim}

\end{sphinxuseclass}\end{sphinxVerbatimOutput}

\end{sphinxuseclass}
\begin{sphinxuseclass}{cell}\begin{sphinxVerbatimInput}

\begin{sphinxuseclass}{cell_input}
\begin{sphinxVerbatim}[commandchars=\\\{\}]
\PYG{c+c1}{\PYGZsh{} lowercase and uppercase letters}
\PYG{n+nb}{print}\PYG{p}{(}\PYG{n}{string}\PYG{o}{.}\PYG{n}{ascii\PYGZus{}letters}\PYG{p}{)}
\end{sphinxVerbatim}

\end{sphinxuseclass}\end{sphinxVerbatimInput}
\begin{sphinxVerbatimOutput}

\begin{sphinxuseclass}{cell_output}
\begin{sphinxVerbatim}[commandchars=\\\{\}]
abcdefghijklmnopqrstuvwxyzABCDEFGHIJKLMNOPQRSTUVWXYZ
\end{sphinxVerbatim}

\end{sphinxuseclass}\end{sphinxVerbatimOutput}

\end{sphinxuseclass}
\begin{sphinxuseclass}{cell}\begin{sphinxVerbatimInput}

\begin{sphinxuseclass}{cell_input}
\begin{sphinxVerbatim}[commandchars=\\\{\}]
\PYG{c+c1}{\PYGZsh{} digits}
\PYG{n+nb}{print}\PYG{p}{(}\PYG{n}{string}\PYG{o}{.}\PYG{n}{digits}\PYG{p}{)}
\end{sphinxVerbatim}

\end{sphinxuseclass}\end{sphinxVerbatimInput}
\begin{sphinxVerbatimOutput}

\begin{sphinxuseclass}{cell_output}
\begin{sphinxVerbatim}[commandchars=\\\{\}]
0123456789
\end{sphinxVerbatim}

\end{sphinxuseclass}\end{sphinxVerbatimOutput}

\end{sphinxuseclass}
\begin{sphinxuseclass}{cell}\begin{sphinxVerbatimInput}

\begin{sphinxuseclass}{cell_input}
\begin{sphinxVerbatim}[commandchars=\\\{\}]
\PYG{c+c1}{\PYGZsh{} punctuations}
\PYG{n+nb}{print}\PYG{p}{(}\PYG{n}{string}\PYG{o}{.}\PYG{n}{punctuation}\PYG{p}{)}
\end{sphinxVerbatim}

\end{sphinxuseclass}\end{sphinxVerbatimInput}
\begin{sphinxVerbatimOutput}

\begin{sphinxuseclass}{cell_output}
\begin{sphinxVerbatim}[commandchars=\\\{\}]
!\PYGZdq{}\PYGZsh{}\PYGZdl{}\PYGZpc{}\PYGZam{}\PYGZsq{}()*+,\PYGZhy{}./:;\PYGZlt{}=\PYGZgt{}?@[\PYGZbs{}]\PYGZca{}\PYGZus{}`\PYGZob{}|\PYGZcb{}\PYGZti{}
\end{sphinxVerbatim}

\end{sphinxuseclass}\end{sphinxVerbatimOutput}

\end{sphinxuseclass}

\section{Immutable}
\label{\detokenize{strings:immutable}}\begin{itemize}
\item {} 
\sphinxAtStartPar
Strings are immutable, which means they cannot be modified.

\item {} 
\sphinxAtStartPar
For example, if you try to change the first character of ‘CALIFORNIA’, you will get an error message.

\end{itemize}

\begin{sphinxVerbatim}[commandchars=\\\{\}]
\PYG{c+c1}{\PYGZsh{} ERROR:  try to change the first character, which has an index of 0.}
\PYG{n}{state} \PYG{o}{=} \PYG{l+s+s1}{\PYGZsq{}}\PYG{l+s+s1}{CALIFORNIA}\PYG{l+s+s1}{\PYGZsq{}}
\PYG{n}{state}\PYG{p}{[}\PYG{l+m+mi}{0}\PYG{p}{]} \PYG{o}{=} \PYG{l+s+s1}{\PYGZsq{}}\PYG{l+s+s1}{R}\PYG{l+s+s1}{\PYGZsq{}}  
\end{sphinxVerbatim}
\begin{itemize}
\item {} 
\sphinxAtStartPar
You can use the state variable to produce a new string without changing the original one.

\item {} 
\sphinxAtStartPar
In the following code:
\begin{itemize}
\item {} 
\sphinxAtStartPar
The value of the variable \sphinxstyleemphasis{new\_state} is the concatenation of the string ‘R’ and a slice of state starting from the character at index 1 and continuing to the end.

\end{itemize}

\end{itemize}

\begin{sphinxuseclass}{cell}\begin{sphinxVerbatimInput}

\begin{sphinxuseclass}{cell_input}
\begin{sphinxVerbatim}[commandchars=\\\{\}]
\PYG{n}{state} \PYG{o}{=} \PYG{l+s+s1}{\PYGZsq{}}\PYG{l+s+s1}{CALIFORNIA}\PYG{l+s+s1}{\PYGZsq{}}
\PYG{n}{new\PYGZus{}state} \PYG{o}{=} \PYG{l+s+s1}{\PYGZsq{}}\PYG{l+s+s1}{R}\PYG{l+s+s1}{\PYGZsq{}} \PYG{o}{+} \PYG{n}{state}\PYG{p}{[}\PYG{l+m+mi}{1}\PYG{p}{:}\PYG{p}{]}   

\PYG{n+nb}{print}\PYG{p}{(}\PYG{n}{new\PYGZus{}state}\PYG{p}{)}
\end{sphinxVerbatim}

\end{sphinxuseclass}\end{sphinxVerbatimInput}
\begin{sphinxVerbatimOutput}

\begin{sphinxuseclass}{cell_output}
\begin{sphinxVerbatim}[commandchars=\\\{\}]
RALIFORNIA
\end{sphinxVerbatim}

\end{sphinxuseclass}\end{sphinxVerbatimOutput}

\end{sphinxuseclass}\begin{itemize}
\item {} 
\sphinxAtStartPar
In the following code, a new value is assigned to the variable \sphinxstyleemphasis{state}.

\end{itemize}

\begin{sphinxuseclass}{cell}\begin{sphinxVerbatimInput}

\begin{sphinxuseclass}{cell_input}
\begin{sphinxVerbatim}[commandchars=\\\{\}]
\PYG{c+c1}{\PYGZsh{} \PYGZsq{}CALIFORNIA\PYGZsq{} is not modified.}

\PYG{n}{state} \PYG{o}{=} \PYG{l+s+s1}{\PYGZsq{}}\PYG{l+s+s1}{CALIFORNIA}\PYG{l+s+s1}{\PYGZsq{}}
\PYG{n}{state} \PYG{o}{=} \PYG{l+s+s1}{\PYGZsq{}}\PYG{l+s+s1}{R}\PYG{l+s+s1}{\PYGZsq{}} \PYG{o}{+} \PYG{n}{state}\PYG{p}{[}\PYG{l+m+mi}{1}\PYG{p}{:}\PYG{p}{]}   

\PYG{n+nb}{print}\PYG{p}{(}\PYG{n}{state}\PYG{p}{)}
\end{sphinxVerbatim}

\end{sphinxuseclass}\end{sphinxVerbatimInput}
\begin{sphinxVerbatimOutput}

\begin{sphinxuseclass}{cell_output}
\begin{sphinxVerbatim}[commandchars=\\\{\}]
RALIFORNIA
\end{sphinxVerbatim}

\end{sphinxuseclass}\end{sphinxVerbatimOutput}

\end{sphinxuseclass}

\section{in  and not in}
\label{\detokenize{strings:in-and-not-in}}\begin{itemize}
\item {} 
\sphinxAtStartPar
These operators are used to check if a character or slice is present in a string.

\item {} 
\sphinxAtStartPar
They return a boolean value: True or False.

\item {} 
\sphinxAtStartPar
Python is case\sphinxhyphen{}sensitive.

\end{itemize}

\begin{sphinxuseclass}{cell}\begin{sphinxVerbatimInput}

\begin{sphinxuseclass}{cell_input}
\begin{sphinxVerbatim}[commandchars=\\\{\}]
\PYG{n+nb}{print}\PYG{p}{(} \PYG{l+s+s1}{\PYGZsq{}}\PYG{l+s+s1}{a}\PYG{l+s+s1}{\PYGZsq{}} \PYG{o+ow}{in} \PYG{l+s+s1}{\PYGZsq{}}\PYG{l+s+s1}{FLORIDA}\PYG{l+s+s1}{\PYGZsq{}} \PYG{p}{)}  \PYG{c+c1}{\PYGZsh{} \PYGZsq{}a\PYGZsq{} is not in FLORIDA}
\end{sphinxVerbatim}

\end{sphinxuseclass}\end{sphinxVerbatimInput}
\begin{sphinxVerbatimOutput}

\begin{sphinxuseclass}{cell_output}
\begin{sphinxVerbatim}[commandchars=\\\{\}]
False
\end{sphinxVerbatim}

\end{sphinxuseclass}\end{sphinxVerbatimOutput}

\end{sphinxuseclass}
\begin{sphinxuseclass}{cell}\begin{sphinxVerbatimInput}

\begin{sphinxuseclass}{cell_input}
\begin{sphinxVerbatim}[commandchars=\\\{\}]
\PYG{n+nb}{print}\PYG{p}{(} \PYG{l+s+s1}{\PYGZsq{}}\PYG{l+s+s1}{a}\PYG{l+s+s1}{\PYGZsq{}} \PYG{o+ow}{not} \PYG{o+ow}{in} \PYG{l+s+s1}{\PYGZsq{}}\PYG{l+s+s1}{FLORIDA}\PYG{l+s+s1}{\PYGZsq{}} \PYG{p}{)}  \PYG{c+c1}{\PYGZsh{} \PYGZsq{}a\PYGZsq{} is not in FLORIDA}
\end{sphinxVerbatim}

\end{sphinxuseclass}\end{sphinxVerbatimInput}
\begin{sphinxVerbatimOutput}

\begin{sphinxuseclass}{cell_output}
\begin{sphinxVerbatim}[commandchars=\\\{\}]
True
\end{sphinxVerbatim}

\end{sphinxuseclass}\end{sphinxVerbatimOutput}

\end{sphinxuseclass}
\begin{sphinxuseclass}{cell}\begin{sphinxVerbatimInput}

\begin{sphinxuseclass}{cell_input}
\begin{sphinxVerbatim}[commandchars=\\\{\}]
\PYG{n+nb}{print}\PYG{p}{(} \PYG{l+s+s1}{\PYGZsq{}}\PYG{l+s+s1}{A}\PYG{l+s+s1}{\PYGZsq{}} \PYG{o+ow}{in} \PYG{l+s+s1}{\PYGZsq{}}\PYG{l+s+s1}{FLORIDA}\PYG{l+s+s1}{\PYGZsq{}} \PYG{p}{)}  \PYG{c+c1}{\PYGZsh{} \PYGZsq{}A\PYGZsq{} is in FLORIDA}
\end{sphinxVerbatim}

\end{sphinxuseclass}\end{sphinxVerbatimInput}
\begin{sphinxVerbatimOutput}

\begin{sphinxuseclass}{cell_output}
\begin{sphinxVerbatim}[commandchars=\\\{\}]
True
\end{sphinxVerbatim}

\end{sphinxuseclass}\end{sphinxVerbatimOutput}

\end{sphinxuseclass}
\begin{sphinxuseclass}{cell}\begin{sphinxVerbatimInput}

\begin{sphinxuseclass}{cell_input}
\begin{sphinxVerbatim}[commandchars=\\\{\}]
\PYG{n+nb}{print}\PYG{p}{(} \PYG{l+s+s1}{\PYGZsq{}}\PYG{l+s+s1}{A}\PYG{l+s+s1}{\PYGZsq{}} \PYG{o+ow}{not} \PYG{o+ow}{in} \PYG{l+s+s1}{\PYGZsq{}}\PYG{l+s+s1}{FLORIDA}\PYG{l+s+s1}{\PYGZsq{}} \PYG{p}{)}  \PYG{c+c1}{\PYGZsh{} \PYGZsq{}a\PYGZsq{} is not in FLORIDA}
\end{sphinxVerbatim}

\end{sphinxuseclass}\end{sphinxVerbatimInput}
\begin{sphinxVerbatimOutput}

\begin{sphinxuseclass}{cell_output}
\begin{sphinxVerbatim}[commandchars=\\\{\}]
False
\end{sphinxVerbatim}

\end{sphinxuseclass}\end{sphinxVerbatimOutput}

\end{sphinxuseclass}

\section{String Methods}
\label{\detokenize{strings:string-methods}}\begin{itemize}
\item {} 
\sphinxAtStartPar
String methods do not modify the original string because strings are immutable.

\item {} 
\sphinxAtStartPar
String methods return a new value.

\item {} 
\sphinxAtStartPar
If you run \sphinxcode{\sphinxupquote{dir(str)}}, you will see that there are many methods because there can be so many things that can be done with strings.

\item {} 
\sphinxAtStartPar
We will cover some of them here, but you can check \sphinxcode{\sphinxupquote{help(str)}} for more details.

\end{itemize}


\subsection{capitalize()}
\label{\detokenize{strings:capitalize}}\begin{itemize}
\item {} 
\sphinxAtStartPar
Produces a duplicate of the string where only the initial character is in uppercase, while all other characters are converted to lowercase.

\end{itemize}

\begin{sphinxuseclass}{cell}\begin{sphinxVerbatimInput}

\begin{sphinxuseclass}{cell_input}
\begin{sphinxVerbatim}[commandchars=\\\{\}]
\PYG{n}{text} \PYG{o}{=} \PYG{l+s+s1}{\PYGZsq{}}\PYG{l+s+s1}{tOm aNd jerRy.}\PYG{l+s+s1}{\PYGZsq{}}
\PYG{n+nb}{print}\PYG{p}{(}\PYG{n}{text}\PYG{o}{.}\PYG{n}{capitalize}\PYG{p}{(}\PYG{p}{)}\PYG{p}{)}   \PYG{c+c1}{\PYGZsh{} \PYGZsq{}t\PYGZsq{} is capitalized, a new string is produced}
\PYG{n+nb}{print}\PYG{p}{(}\PYG{n}{text}\PYG{p}{)}                \PYG{c+c1}{\PYGZsh{} no change on text (immutable)}
\end{sphinxVerbatim}

\end{sphinxuseclass}\end{sphinxVerbatimInput}
\begin{sphinxVerbatimOutput}

\begin{sphinxuseclass}{cell_output}
\begin{sphinxVerbatim}[commandchars=\\\{\}]
Tom and jerry.
tOm aNd jerRy.
\end{sphinxVerbatim}

\end{sphinxuseclass}\end{sphinxVerbatimOutput}

\end{sphinxuseclass}

\subsection{upper()}
\label{\detokenize{strings:upper}}\begin{itemize}
\item {} 
\sphinxAtStartPar
Produces a duplicate of the string where all characters are converted to uppercase.

\end{itemize}

\begin{sphinxuseclass}{cell}\begin{sphinxVerbatimInput}

\begin{sphinxuseclass}{cell_input}
\begin{sphinxVerbatim}[commandchars=\\\{\}]
\PYG{n}{text} \PYG{o}{=} \PYG{l+s+s1}{\PYGZsq{}}\PYG{l+s+s1}{tOm aNd jerRy.}\PYG{l+s+s1}{\PYGZsq{}}
\PYG{n+nb}{print}\PYG{p}{(}\PYG{n}{text}\PYG{o}{.}\PYG{n}{upper}\PYG{p}{(}\PYG{p}{)}\PYG{p}{)}        \PYG{c+c1}{\PYGZsh{} all characters are in uppercase}
\PYG{n+nb}{print}\PYG{p}{(}\PYG{n}{text}\PYG{p}{)}                \PYG{c+c1}{\PYGZsh{} no change on text (immutable)}
\end{sphinxVerbatim}

\end{sphinxuseclass}\end{sphinxVerbatimInput}
\begin{sphinxVerbatimOutput}

\begin{sphinxuseclass}{cell_output}
\begin{sphinxVerbatim}[commandchars=\\\{\}]
TOM AND JERRY.
tOm aNd jerRy.
\end{sphinxVerbatim}

\end{sphinxuseclass}\end{sphinxVerbatimOutput}

\end{sphinxuseclass}

\subsection{lower()}
\label{\detokenize{strings:lower}}\begin{itemize}
\item {} 
\sphinxAtStartPar
Produces a duplicate of the string where all characters are converted to lowercase.

\end{itemize}

\begin{sphinxuseclass}{cell}\begin{sphinxVerbatimInput}

\begin{sphinxuseclass}{cell_input}
\begin{sphinxVerbatim}[commandchars=\\\{\}]
\PYG{n}{text} \PYG{o}{=} \PYG{l+s+s1}{\PYGZsq{}}\PYG{l+s+s1}{tOm aNd jerRy.}\PYG{l+s+s1}{\PYGZsq{}}
\PYG{n+nb}{print}\PYG{p}{(}\PYG{n}{text}\PYG{o}{.}\PYG{n}{lower}\PYG{p}{(}\PYG{p}{)}\PYG{p}{)}        \PYG{c+c1}{\PYGZsh{} all characters are in lowercase}
\PYG{n+nb}{print}\PYG{p}{(}\PYG{n}{text}\PYG{p}{)}                \PYG{c+c1}{\PYGZsh{} no change on text (immutable)}
\end{sphinxVerbatim}

\end{sphinxuseclass}\end{sphinxVerbatimInput}
\begin{sphinxVerbatimOutput}

\begin{sphinxuseclass}{cell_output}
\begin{sphinxVerbatim}[commandchars=\\\{\}]
tom and jerry.
tOm aNd jerRy.
\end{sphinxVerbatim}

\end{sphinxuseclass}\end{sphinxVerbatimOutput}

\end{sphinxuseclass}

\subsection{title()}
\label{\detokenize{strings:title}}\begin{itemize}
\item {} 
\sphinxAtStartPar
Produces a duplicate of the string where all words are capitalized.

\end{itemize}

\begin{sphinxuseclass}{cell}\begin{sphinxVerbatimInput}

\begin{sphinxuseclass}{cell_input}
\begin{sphinxVerbatim}[commandchars=\\\{\}]
\PYG{n}{text} \PYG{o}{=} \PYG{l+s+s1}{\PYGZsq{}}\PYG{l+s+s1}{tOm aNd jerRy.}\PYG{l+s+s1}{\PYGZsq{}}
\PYG{n+nb}{print}\PYG{p}{(}\PYG{n}{text}\PYG{o}{.}\PYG{n}{title}\PYG{p}{(}\PYG{p}{)}\PYG{p}{)}        \PYG{c+c1}{\PYGZsh{} all words are capitalized}
\PYG{n+nb}{print}\PYG{p}{(}\PYG{n}{text}\PYG{p}{)}                \PYG{c+c1}{\PYGZsh{} no change on text (immutable)}
\end{sphinxVerbatim}

\end{sphinxuseclass}\end{sphinxVerbatimInput}
\begin{sphinxVerbatimOutput}

\begin{sphinxuseclass}{cell_output}
\begin{sphinxVerbatim}[commandchars=\\\{\}]
Tom And Jerry.
tOm aNd jerRy.
\end{sphinxVerbatim}

\end{sphinxuseclass}\end{sphinxVerbatimOutput}

\end{sphinxuseclass}

\subsection{find()}
\label{\detokenize{strings:find}}\begin{itemize}
\item {} 
\sphinxAtStartPar
It provides the earliest occurrence of a given substring within a string.

\item {} 
\sphinxAtStartPar
It returns the lowest index.

\item {} 
\sphinxAtStartPar
If the substring is not present, it returns \sphinxhyphen{}1.

\item {} 
\sphinxAtStartPar
Additionally, you have the option to begin the search from a specific character to find the index of the given substring.
\begin{itemize}
\item {} 
\sphinxAtStartPar
\sphinxcode{\sphinxupquote{find('a', N)}}: find index of first ‘a’  starting from index=N

\item {} 
\sphinxAtStartPar
default value of N is 0

\end{itemize}

\end{itemize}

\begin{sphinxuseclass}{cell}\begin{sphinxVerbatimInput}

\begin{sphinxuseclass}{cell_input}
\begin{sphinxVerbatim}[commandchars=\\\{\}]
\PYG{n}{state} \PYG{o}{=} \PYG{l+s+s1}{\PYGZsq{}}\PYG{l+s+s1}{CALIFORNIA}\PYG{l+s+s1}{\PYGZsq{}}
\PYG{n+nb}{print}\PYG{p}{(}\PYG{n}{state}\PYG{o}{.}\PYG{n}{find}\PYG{p}{(}\PYG{l+s+s1}{\PYGZsq{}}\PYG{l+s+s1}{L}\PYG{l+s+s1}{\PYGZsq{}}\PYG{p}{)}\PYG{p}{)} \PYG{c+c1}{\PYGZsh{} index of \PYGZsq{}L\PYGZsq{} is 2}
\end{sphinxVerbatim}

\end{sphinxuseclass}\end{sphinxVerbatimInput}
\begin{sphinxVerbatimOutput}

\begin{sphinxuseclass}{cell_output}
\begin{sphinxVerbatim}[commandchars=\\\{\}]
2
\end{sphinxVerbatim}

\end{sphinxuseclass}\end{sphinxVerbatimOutput}

\end{sphinxuseclass}
\begin{sphinxuseclass}{cell}\begin{sphinxVerbatimInput}

\begin{sphinxuseclass}{cell_input}
\begin{sphinxVerbatim}[commandchars=\\\{\}]
\PYG{c+c1}{\PYGZsh{} \PYGZhy{}1 means \PYGZsq{}W\PYGZsq{} does not exist, \PYGZhy{}1 does not represent an index}
\PYG{n+nb}{print}\PYG{p}{(}\PYG{n}{state}\PYG{o}{.}\PYG{n}{find}\PYG{p}{(}\PYG{l+s+s1}{\PYGZsq{}}\PYG{l+s+s1}{W}\PYG{l+s+s1}{\PYGZsq{}}\PYG{p}{)}\PYG{p}{)} 
\end{sphinxVerbatim}

\end{sphinxuseclass}\end{sphinxVerbatimInput}
\begin{sphinxVerbatimOutput}

\begin{sphinxuseclass}{cell_output}
\begin{sphinxVerbatim}[commandchars=\\\{\}]
\PYGZhy{}1
\end{sphinxVerbatim}

\end{sphinxuseclass}\end{sphinxVerbatimOutput}

\end{sphinxuseclass}
\begin{sphinxuseclass}{cell}\begin{sphinxVerbatimInput}

\begin{sphinxuseclass}{cell_input}
\begin{sphinxVerbatim}[commandchars=\\\{\}]
\PYG{n+nb}{print}\PYG{p}{(}\PYG{n}{state}\PYG{o}{.}\PYG{n}{find}\PYG{p}{(}\PYG{l+s+s1}{\PYGZsq{}}\PYG{l+s+s1}{A}\PYG{l+s+s1}{\PYGZsq{}}\PYG{p}{)}\PYG{p}{)} \PYG{c+c1}{\PYGZsh{} index of first \PYGZsq{}A\PYGZsq{}}
\end{sphinxVerbatim}

\end{sphinxuseclass}\end{sphinxVerbatimInput}
\begin{sphinxVerbatimOutput}

\begin{sphinxuseclass}{cell_output}
\begin{sphinxVerbatim}[commandchars=\\\{\}]
1
\end{sphinxVerbatim}

\end{sphinxuseclass}\end{sphinxVerbatimOutput}

\end{sphinxuseclass}
\begin{sphinxuseclass}{cell}\begin{sphinxVerbatimInput}

\begin{sphinxuseclass}{cell_input}
\begin{sphinxVerbatim}[commandchars=\\\{\}]
\PYG{c+c1}{\PYGZsh{} The index of the first occurrence of \PYGZsq{}A\PYGZsq{} starting from the character at index 3.}
\PYG{n+nb}{print}\PYG{p}{(}\PYG{n}{state}\PYG{o}{.}\PYG{n}{find}\PYG{p}{(}\PYG{l+s+s1}{\PYGZsq{}}\PYG{l+s+s1}{A}\PYG{l+s+s1}{\PYGZsq{}}\PYG{p}{,} \PYG{l+m+mi}{3}\PYG{p}{)}\PYG{p}{)} 
\end{sphinxVerbatim}

\end{sphinxuseclass}\end{sphinxVerbatimInput}
\begin{sphinxVerbatimOutput}

\begin{sphinxuseclass}{cell_output}
\begin{sphinxVerbatim}[commandchars=\\\{\}]
9
\end{sphinxVerbatim}

\end{sphinxuseclass}\end{sphinxVerbatimOutput}

\end{sphinxuseclass}
\begin{sphinxuseclass}{cell}\begin{sphinxVerbatimInput}

\begin{sphinxuseclass}{cell_input}
\begin{sphinxVerbatim}[commandchars=\\\{\}]
\PYG{c+c1}{\PYGZsh{} The string \PYGZsq{}FOR\PYGZsq{} begins from the character at index 4.}
\PYG{n+nb}{print}\PYG{p}{(}\PYG{n}{state}\PYG{o}{.}\PYG{n}{find}\PYG{p}{(}\PYG{l+s+s1}{\PYGZsq{}}\PYG{l+s+s1}{FOR}\PYG{l+s+s1}{\PYGZsq{}}\PYG{p}{)}\PYG{p}{)} 
\end{sphinxVerbatim}

\end{sphinxuseclass}\end{sphinxVerbatimInput}
\begin{sphinxVerbatimOutput}

\begin{sphinxuseclass}{cell_output}
\begin{sphinxVerbatim}[commandchars=\\\{\}]
4
\end{sphinxVerbatim}

\end{sphinxuseclass}\end{sphinxVerbatimOutput}

\end{sphinxuseclass}
\begin{sphinxuseclass}{cell}\begin{sphinxVerbatimInput}

\begin{sphinxuseclass}{cell_input}
\begin{sphinxVerbatim}[commandchars=\\\{\}]
\PYG{n+nb}{print}\PYG{p}{(}\PYG{n}{state}\PYG{o}{.}\PYG{n}{find}\PYG{p}{(}\PYG{l+s+s1}{\PYGZsq{}}\PYG{l+s+s1}{WE}\PYG{l+s+s1}{\PYGZsq{}}\PYG{p}{)}\PYG{p}{)} \PYG{c+c1}{\PYGZsh{} \PYGZsq{}WE\PYGZsq{} does not exist in CALIFORNIA}
\end{sphinxVerbatim}

\end{sphinxuseclass}\end{sphinxVerbatimInput}
\begin{sphinxVerbatimOutput}

\begin{sphinxuseclass}{cell_output}
\begin{sphinxVerbatim}[commandchars=\\\{\}]
\PYGZhy{}1
\end{sphinxVerbatim}

\end{sphinxuseclass}\end{sphinxVerbatimOutput}

\end{sphinxuseclass}

\subsection{rfind()}
\label{\detokenize{strings:rfind}}\begin{itemize}
\item {} 
\sphinxAtStartPar
It returns the maximum index in a string where the substring is located.

\end{itemize}

\begin{sphinxuseclass}{cell}\begin{sphinxVerbatimInput}

\begin{sphinxuseclass}{cell_input}
\begin{sphinxVerbatim}[commandchars=\\\{\}]
\PYG{n+nb}{print}\PYG{p}{(}\PYG{n}{state}\PYG{o}{.}\PYG{n}{find}\PYG{p}{(}\PYG{l+s+s1}{\PYGZsq{}}\PYG{l+s+s1}{A}\PYG{l+s+s1}{\PYGZsq{}}\PYG{p}{)}\PYG{p}{)}   \PYG{c+c1}{\PYGZsh{} index of first \PYGZsq{}A\PYGZsq{}}
\PYG{n+nb}{print}\PYG{p}{(}\PYG{n}{state}\PYG{o}{.}\PYG{n}{rfind}\PYG{p}{(}\PYG{l+s+s1}{\PYGZsq{}}\PYG{l+s+s1}{A}\PYG{l+s+s1}{\PYGZsq{}}\PYG{p}{)}\PYG{p}{)}  \PYG{c+c1}{\PYGZsh{} index of last \PYGZsq{}A\PYGZsq{}}
\end{sphinxVerbatim}

\end{sphinxuseclass}\end{sphinxVerbatimInput}
\begin{sphinxVerbatimOutput}

\begin{sphinxuseclass}{cell_output}
\begin{sphinxVerbatim}[commandchars=\\\{\}]
1
9
\end{sphinxVerbatim}

\end{sphinxuseclass}\end{sphinxVerbatimOutput}

\end{sphinxuseclass}

\subsection{strip(), rstrip(), lstrip()}
\label{\detokenize{strings:strip-rstrip-lstrip}}\begin{itemize}
\item {} 
\sphinxAtStartPar
\sphinxcode{\sphinxupquote{strip()}}: Removes white spaces from the beginning and end of a string.

\item {} 
\sphinxAtStartPar
\sphinxcode{\sphinxupquote{rstrip()}}: Removes white spaces from the end of a string.

\item {} 
\sphinxAtStartPar
\sphinxcode{\sphinxupquote{lstrip()}}: Removes white spaces from the beginning of a string.

\end{itemize}

\begin{sphinxuseclass}{cell}\begin{sphinxVerbatimInput}

\begin{sphinxuseclass}{cell_input}
\begin{sphinxVerbatim}[commandchars=\\\{\}]
\PYG{n}{country} \PYG{o}{=} \PYG{l+s+s1}{\PYGZsq{}}\PYG{l+s+s1}{  FLORIDA   }\PYG{l+s+s1}{\PYGZsq{}}
\PYG{n+nb}{print}\PYG{p}{(}\PYG{n}{country}\PYG{p}{)}          
\PYG{n+nb}{print}\PYG{p}{(}\PYG{l+s+s1}{\PYGZsq{}}\PYG{l+s+s1}{\PYGZhy{}\PYGZhy{}\PYGZhy{}}\PYG{l+s+s1}{\PYGZsq{}}\PYG{o}{+}\PYG{n}{country}\PYG{o}{.}\PYG{n}{strip}\PYG{p}{(}\PYG{p}{)}\PYG{o}{+}\PYG{l+s+s1}{\PYGZsq{}}\PYG{l+s+s1}{\PYGZhy{}\PYGZhy{}\PYGZhy{}}\PYG{l+s+s1}{\PYGZsq{}}\PYG{p}{)}      \PYG{c+c1}{\PYGZsh{} white spaces on the left and right are removed}
\PYG{n+nb}{print}\PYG{p}{(}\PYG{l+s+s1}{\PYGZsq{}}\PYG{l+s+s1}{\PYGZhy{}\PYGZhy{}\PYGZhy{}}\PYG{l+s+s1}{\PYGZsq{}}\PYG{o}{+}\PYG{n}{country}\PYG{o}{.}\PYG{n}{rstrip}\PYG{p}{(}\PYG{p}{)}\PYG{o}{+}\PYG{l+s+s1}{\PYGZsq{}}\PYG{l+s+s1}{\PYGZhy{}\PYGZhy{}\PYGZhy{}}\PYG{l+s+s1}{\PYGZsq{}}\PYG{p}{)}     \PYG{c+c1}{\PYGZsh{} white spaces on the right are removed}
\PYG{n+nb}{print}\PYG{p}{(}\PYG{l+s+s1}{\PYGZsq{}}\PYG{l+s+s1}{\PYGZhy{}\PYGZhy{}\PYGZhy{}}\PYG{l+s+s1}{\PYGZsq{}}\PYG{o}{+}\PYG{n}{country}\PYG{o}{.}\PYG{n}{lstrip}\PYG{p}{(}\PYG{p}{)}\PYG{o}{+}\PYG{l+s+s1}{\PYGZsq{}}\PYG{l+s+s1}{\PYGZhy{}\PYGZhy{}\PYGZhy{}}\PYG{l+s+s1}{\PYGZsq{}}\PYG{p}{)}     \PYG{c+c1}{\PYGZsh{} white spaces on the left  are removed}
\end{sphinxVerbatim}

\end{sphinxuseclass}\end{sphinxVerbatimInput}
\begin{sphinxVerbatimOutput}

\begin{sphinxuseclass}{cell_output}
\begin{sphinxVerbatim}[commandchars=\\\{\}]
  FLORIDA   
\PYGZhy{}\PYGZhy{}\PYGZhy{}FLORIDA\PYGZhy{}\PYGZhy{}\PYGZhy{}
\PYGZhy{}\PYGZhy{}\PYGZhy{}  FLORIDA\PYGZhy{}\PYGZhy{}\PYGZhy{}
\PYGZhy{}\PYGZhy{}\PYGZhy{}FLORIDA   \PYGZhy{}\PYGZhy{}\PYGZhy{}
\end{sphinxVerbatim}

\end{sphinxuseclass}\end{sphinxVerbatimOutput}

\end{sphinxuseclass}

\subsection{startswith()}
\label{\detokenize{strings:startswith}}\begin{itemize}
\item {} 
\sphinxAtStartPar
It returns True if the string starts with the specified prefix; otherwise, it returns False.

\end{itemize}

\begin{sphinxuseclass}{cell}\begin{sphinxVerbatimInput}

\begin{sphinxuseclass}{cell_input}
\begin{sphinxVerbatim}[commandchars=\\\{\}]
\PYG{n+nb}{print}\PYG{p}{(}\PYG{n}{state}\PYG{o}{.}\PYG{n}{startswith}\PYG{p}{(}\PYG{l+s+s1}{\PYGZsq{}}\PYG{l+s+s1}{H}\PYG{l+s+s1}{\PYGZsq{}}\PYG{p}{)}\PYG{p}{)} \PYG{c+c1}{\PYGZsh{} \PYGZsq{}CALIFORNIA\PYGZsq{} does not startswith \PYGZsq{}H\PYGZsq{}}
\end{sphinxVerbatim}

\end{sphinxuseclass}\end{sphinxVerbatimInput}
\begin{sphinxVerbatimOutput}

\begin{sphinxuseclass}{cell_output}
\begin{sphinxVerbatim}[commandchars=\\\{\}]
False
\end{sphinxVerbatim}

\end{sphinxuseclass}\end{sphinxVerbatimOutput}

\end{sphinxuseclass}
\begin{sphinxuseclass}{cell}\begin{sphinxVerbatimInput}

\begin{sphinxuseclass}{cell_input}
\begin{sphinxVerbatim}[commandchars=\\\{\}]
\PYG{n+nb}{print}\PYG{p}{(}\PYG{n}{state}\PYG{o}{.}\PYG{n}{startswith}\PYG{p}{(}\PYG{l+s+s1}{\PYGZsq{}}\PYG{l+s+s1}{C}\PYG{l+s+s1}{\PYGZsq{}}\PYG{p}{)}\PYG{p}{)} \PYG{c+c1}{\PYGZsh{} \PYGZsq{}CALIFORNIA\PYGZsq{} does startswith \PYGZsq{}C\PYGZsq{}}
\end{sphinxVerbatim}

\end{sphinxuseclass}\end{sphinxVerbatimInput}
\begin{sphinxVerbatimOutput}

\begin{sphinxuseclass}{cell_output}
\begin{sphinxVerbatim}[commandchars=\\\{\}]
True
\end{sphinxVerbatim}

\end{sphinxuseclass}\end{sphinxVerbatimOutput}

\end{sphinxuseclass}

\subsection{count()}
\label{\detokenize{strings:count}}\begin{itemize}
\item {} 
\sphinxAtStartPar
Returns the number of non\sphinxhyphen{}overlapping occurrences of a substring within a string

\end{itemize}

\begin{sphinxuseclass}{cell}\begin{sphinxVerbatimInput}

\begin{sphinxuseclass}{cell_input}
\begin{sphinxVerbatim}[commandchars=\\\{\}]
\PYG{n+nb}{print}\PYG{p}{(}\PYG{n}{state}\PYG{o}{.}\PYG{n}{count}\PYG{p}{(}\PYG{l+s+s1}{\PYGZsq{}}\PYG{l+s+s1}{A}\PYG{l+s+s1}{\PYGZsq{}}\PYG{p}{)}\PYG{p}{)}   \PYG{c+c1}{\PYGZsh{} number of \PYGZsq{}A\PYGZsq{}s in \PYGZsq{}CALIFORNIA\PYGZsq{}}
\end{sphinxVerbatim}

\end{sphinxuseclass}\end{sphinxVerbatimInput}
\begin{sphinxVerbatimOutput}

\begin{sphinxuseclass}{cell_output}
\begin{sphinxVerbatim}[commandchars=\\\{\}]
2
\end{sphinxVerbatim}

\end{sphinxuseclass}\end{sphinxVerbatimOutput}

\end{sphinxuseclass}
\begin{sphinxuseclass}{cell}\begin{sphinxVerbatimInput}

\begin{sphinxuseclass}{cell_input}
\begin{sphinxVerbatim}[commandchars=\\\{\}]
\PYG{n+nb}{print}\PYG{p}{(}\PYG{n}{state}\PYG{o}{.}\PYG{n}{count}\PYG{p}{(}\PYG{l+s+s1}{\PYGZsq{}}\PYG{l+s+s1}{I}\PYG{l+s+s1}{\PYGZsq{}}\PYG{p}{)}\PYG{p}{)}   \PYG{c+c1}{\PYGZsh{} number of \PYGZsq{}I\PYGZsq{}s in CALIFORNIA\PYGZsq{}}
\end{sphinxVerbatim}

\end{sphinxuseclass}\end{sphinxVerbatimInput}
\begin{sphinxVerbatimOutput}

\begin{sphinxuseclass}{cell_output}
\begin{sphinxVerbatim}[commandchars=\\\{\}]
2
\end{sphinxVerbatim}

\end{sphinxuseclass}\end{sphinxVerbatimOutput}

\end{sphinxuseclass}
\begin{sphinxuseclass}{cell}\begin{sphinxVerbatimInput}

\begin{sphinxuseclass}{cell_input}
\begin{sphinxVerbatim}[commandchars=\\\{\}]
\PYG{n+nb}{print}\PYG{p}{(}\PYG{n}{state}\PYG{o}{.}\PYG{n}{count}\PYG{p}{(}\PYG{l+s+s1}{\PYGZsq{}}\PYG{l+s+s1}{C}\PYG{l+s+s1}{\PYGZsq{}}\PYG{p}{)}\PYG{p}{)}   \PYG{c+c1}{\PYGZsh{} number of \PYGZsq{}C\PYGZsq{}s in \PYGZsq{}CALIFORNIA\PYGZsq{}}
\end{sphinxVerbatim}

\end{sphinxuseclass}\end{sphinxVerbatimInput}
\begin{sphinxVerbatimOutput}

\begin{sphinxuseclass}{cell_output}
\begin{sphinxVerbatim}[commandchars=\\\{\}]
1
\end{sphinxVerbatim}

\end{sphinxuseclass}\end{sphinxVerbatimOutput}

\end{sphinxuseclass}
\begin{sphinxuseclass}{cell}\begin{sphinxVerbatimInput}

\begin{sphinxuseclass}{cell_input}
\begin{sphinxVerbatim}[commandchars=\\\{\}]
\PYG{n+nb}{print}\PYG{p}{(}\PYG{n}{state}\PYG{o}{.}\PYG{n}{count}\PYG{p}{(}\PYG{l+s+s1}{\PYGZsq{}}\PYG{l+s+s1}{W}\PYG{l+s+s1}{\PYGZsq{}}\PYG{p}{)}\PYG{p}{)}   \PYG{c+c1}{\PYGZsh{} number of \PYGZsq{}C\PYGZsq{}s in \PYGZsq{}CALIFORNIA\PYGZsq{}}
\end{sphinxVerbatim}

\end{sphinxuseclass}\end{sphinxVerbatimInput}
\begin{sphinxVerbatimOutput}

\begin{sphinxuseclass}{cell_output}
\begin{sphinxVerbatim}[commandchars=\\\{\}]
0
\end{sphinxVerbatim}

\end{sphinxuseclass}\end{sphinxVerbatimOutput}

\end{sphinxuseclass}

\subsection{isdigit()}
\label{\detokenize{strings:isdigit}}\begin{itemize}
\item {} 
\sphinxAtStartPar
Returns True if the string consists of digits, False otherwise.

\end{itemize}

\begin{sphinxuseclass}{cell}\begin{sphinxVerbatimInput}

\begin{sphinxuseclass}{cell_input}
\begin{sphinxVerbatim}[commandchars=\\\{\}]
\PYG{n+nb}{print}\PYG{p}{(}\PYG{l+s+s1}{\PYGZsq{}}\PYG{l+s+s1}{hello}\PYG{l+s+s1}{\PYGZsq{}}\PYG{o}{.}\PYG{n}{isdigit}\PYG{p}{(}\PYG{p}{)}\PYG{p}{)}         \PYG{c+c1}{\PYGZsh{} not all characters are digits}
\end{sphinxVerbatim}

\end{sphinxuseclass}\end{sphinxVerbatimInput}
\begin{sphinxVerbatimOutput}

\begin{sphinxuseclass}{cell_output}
\begin{sphinxVerbatim}[commandchars=\\\{\}]
False
\end{sphinxVerbatim}

\end{sphinxuseclass}\end{sphinxVerbatimOutput}

\end{sphinxuseclass}
\begin{sphinxuseclass}{cell}\begin{sphinxVerbatimInput}

\begin{sphinxuseclass}{cell_input}
\begin{sphinxVerbatim}[commandchars=\\\{\}]
\PYG{n+nb}{print}\PYG{p}{(}\PYG{l+s+s1}{\PYGZsq{}}\PYG{l+s+s1}{123456}\PYG{l+s+s1}{\PYGZsq{}}\PYG{o}{.}\PYG{n}{isdigit}\PYG{p}{(}\PYG{p}{)}\PYG{p}{)}        \PYG{c+c1}{\PYGZsh{} all characters are digits}
\end{sphinxVerbatim}

\end{sphinxuseclass}\end{sphinxVerbatimInput}
\begin{sphinxVerbatimOutput}

\begin{sphinxuseclass}{cell_output}
\begin{sphinxVerbatim}[commandchars=\\\{\}]
True
\end{sphinxVerbatim}

\end{sphinxuseclass}\end{sphinxVerbatimOutput}

\end{sphinxuseclass}
\begin{sphinxuseclass}{cell}\begin{sphinxVerbatimInput}

\begin{sphinxuseclass}{cell_input}
\begin{sphinxVerbatim}[commandchars=\\\{\}]
\PYG{n+nb}{print}\PYG{p}{(}\PYG{l+s+s1}{\PYGZsq{}}\PYG{l+s+s1}{h1234}\PYG{l+s+s1}{\PYGZsq{}}\PYG{o}{.}\PYG{n}{isdigit}\PYG{p}{(}\PYG{p}{)}\PYG{p}{)}         \PYG{c+c1}{\PYGZsh{} not all characters are digits}
\end{sphinxVerbatim}

\end{sphinxuseclass}\end{sphinxVerbatimInput}
\begin{sphinxVerbatimOutput}

\begin{sphinxuseclass}{cell_output}
\begin{sphinxVerbatim}[commandchars=\\\{\}]
False
\end{sphinxVerbatim}

\end{sphinxuseclass}\end{sphinxVerbatimOutput}

\end{sphinxuseclass}

\subsection{isalpha()}
\label{\detokenize{strings:isalpha}}\begin{itemize}
\item {} 
\sphinxAtStartPar
Returns True if the string consists of alphabetic characters, False otherwise.

\end{itemize}

\begin{sphinxuseclass}{cell}\begin{sphinxVerbatimInput}

\begin{sphinxuseclass}{cell_input}
\begin{sphinxVerbatim}[commandchars=\\\{\}]
\PYG{n+nb}{print}\PYG{p}{(}\PYG{l+s+s1}{\PYGZsq{}}\PYG{l+s+s1}{hello}\PYG{l+s+s1}{\PYGZsq{}}\PYG{o}{.}\PYG{n}{isalpha}\PYG{p}{(}\PYG{p}{)}\PYG{p}{)}         \PYG{c+c1}{\PYGZsh{} all characters are alphabetic}
\end{sphinxVerbatim}

\end{sphinxuseclass}\end{sphinxVerbatimInput}
\begin{sphinxVerbatimOutput}

\begin{sphinxuseclass}{cell_output}
\begin{sphinxVerbatim}[commandchars=\\\{\}]
True
\end{sphinxVerbatim}

\end{sphinxuseclass}\end{sphinxVerbatimOutput}

\end{sphinxuseclass}
\begin{sphinxuseclass}{cell}\begin{sphinxVerbatimInput}

\begin{sphinxuseclass}{cell_input}
\begin{sphinxVerbatim}[commandchars=\\\{\}]
\PYG{n+nb}{print}\PYG{p}{(}\PYG{l+s+s1}{\PYGZsq{}}\PYG{l+s+s1}{123456}\PYG{l+s+s1}{\PYGZsq{}}\PYG{o}{.}\PYG{n}{isalpha}\PYG{p}{(}\PYG{p}{)}\PYG{p}{)}        \PYG{c+c1}{\PYGZsh{} not all characters are alphabetic}
\end{sphinxVerbatim}

\end{sphinxuseclass}\end{sphinxVerbatimInput}
\begin{sphinxVerbatimOutput}

\begin{sphinxuseclass}{cell_output}
\begin{sphinxVerbatim}[commandchars=\\\{\}]
False
\end{sphinxVerbatim}

\end{sphinxuseclass}\end{sphinxVerbatimOutput}

\end{sphinxuseclass}
\begin{sphinxuseclass}{cell}\begin{sphinxVerbatimInput}

\begin{sphinxuseclass}{cell_input}
\begin{sphinxVerbatim}[commandchars=\\\{\}]
\PYG{n+nb}{print}\PYG{p}{(}\PYG{l+s+s1}{\PYGZsq{}}\PYG{l+s+s1}{h1234}\PYG{l+s+s1}{\PYGZsq{}}\PYG{o}{.}\PYG{n}{isalpha}\PYG{p}{(}\PYG{p}{)}\PYG{p}{)}         \PYG{c+c1}{\PYGZsh{} not all characters are alphabetic}
\end{sphinxVerbatim}

\end{sphinxuseclass}\end{sphinxVerbatimInput}
\begin{sphinxVerbatimOutput}

\begin{sphinxuseclass}{cell_output}
\begin{sphinxVerbatim}[commandchars=\\\{\}]
False
\end{sphinxVerbatim}

\end{sphinxuseclass}\end{sphinxVerbatimOutput}

\end{sphinxuseclass}

\subsection{replace()}
\label{\detokenize{strings:replace}}\begin{itemize}
\item {} 
\sphinxAtStartPar
Returns a duplicate with all occurrences of the old substring replaced by the new one.

\item {} 
\sphinxAtStartPar
It is in the form of \sphinxcode{\sphinxupquote{replace(old, new)}}

\end{itemize}

\begin{sphinxuseclass}{cell}\begin{sphinxVerbatimInput}

\begin{sphinxuseclass}{cell_input}
\begin{sphinxVerbatim}[commandchars=\\\{\}]
\PYG{n+nb}{print}\PYG{p}{(}\PYG{n}{state}\PYG{p}{)}
\PYG{n+nb}{print}\PYG{p}{(}\PYG{n}{state}\PYG{o}{.}\PYG{n}{replace}\PYG{p}{(}\PYG{l+s+s1}{\PYGZsq{}}\PYG{l+s+s1}{A}\PYG{l+s+s1}{\PYGZsq{}}\PYG{p}{,} \PYG{l+s+s1}{\PYGZsq{}}\PYG{l+s+s1}{W}\PYG{l+s+s1}{\PYGZsq{}}\PYG{p}{)}\PYG{p}{)} \PYG{c+c1}{\PYGZsh{} replace \PYGZsq{}A\PYGZsq{} by \PYGZsq{}W\PYGZsq{}}
\end{sphinxVerbatim}

\end{sphinxuseclass}\end{sphinxVerbatimInput}
\begin{sphinxVerbatimOutput}

\begin{sphinxuseclass}{cell_output}
\begin{sphinxVerbatim}[commandchars=\\\{\}]
CALIFORNIA
CWLIFORNIW
\end{sphinxVerbatim}

\end{sphinxuseclass}\end{sphinxVerbatimOutput}

\end{sphinxuseclass}
\begin{sphinxuseclass}{cell}\begin{sphinxVerbatimInput}

\begin{sphinxuseclass}{cell_input}
\begin{sphinxVerbatim}[commandchars=\\\{\}]
\PYG{n+nb}{print}\PYG{p}{(}\PYG{n}{state}\PYG{p}{)}
\PYG{n+nb}{print}\PYG{p}{(}\PYG{n}{state}\PYG{o}{.}\PYG{n}{replace}\PYG{p}{(}\PYG{l+s+s1}{\PYGZsq{}}\PYG{l+s+s1}{T}\PYG{l+s+s1}{\PYGZsq{}}\PYG{p}{,} \PYG{l+s+s1}{\PYGZsq{}}\PYG{l+s+s1}{W}\PYG{l+s+s1}{\PYGZsq{}}\PYG{p}{)}\PYG{p}{)} \PYG{c+c1}{\PYGZsh{} no \PYGZsq{}T\PYGZsq{} to replace by \PYGZsq{}W\PYGZsq{}}
\end{sphinxVerbatim}

\end{sphinxuseclass}\end{sphinxVerbatimInput}
\begin{sphinxVerbatimOutput}

\begin{sphinxuseclass}{cell_output}
\begin{sphinxVerbatim}[commandchars=\\\{\}]
CALIFORNIA
CALIFORNIA
\end{sphinxVerbatim}

\end{sphinxuseclass}\end{sphinxVerbatimOutput}

\end{sphinxuseclass}
\begin{sphinxuseclass}{cell}\begin{sphinxVerbatimInput}

\begin{sphinxuseclass}{cell_input}
\begin{sphinxVerbatim}[commandchars=\\\{\}]
\PYG{n+nb}{print}\PYG{p}{(}\PYG{n}{state}\PYG{p}{)}
\PYG{n+nb}{print}\PYG{p}{(}\PYG{n}{state}\PYG{o}{.}\PYG{n}{replace}\PYG{p}{(}\PYG{l+s+s1}{\PYGZsq{}}\PYG{l+s+s1}{LI}\PYG{l+s+s1}{\PYGZsq{}}\PYG{p}{,} \PYG{l+s+s1}{\PYGZsq{}}\PYG{l+s+s1}{***}\PYG{l+s+s1}{\PYGZsq{}}\PYG{p}{)}\PYG{p}{)} \PYG{c+c1}{\PYGZsh{} replace \PYGZsq{}LI\PYGZsq{} by \PYGZsq{}***\PYGZsq{}}
\end{sphinxVerbatim}

\end{sphinxuseclass}\end{sphinxVerbatimInput}
\begin{sphinxVerbatimOutput}

\begin{sphinxuseclass}{cell_output}
\begin{sphinxVerbatim}[commandchars=\\\{\}]
CALIFORNIA
CA***FORNIA
\end{sphinxVerbatim}

\end{sphinxuseclass}\end{sphinxVerbatimOutput}

\end{sphinxuseclass}

\subsection{swapcase()}
\label{\detokenize{strings:swapcase}}\begin{itemize}
\item {} 
\sphinxAtStartPar
Transform uppercase characters to lowercase and lowercase characters to uppercase.

\end{itemize}

\begin{sphinxuseclass}{cell}\begin{sphinxVerbatimInput}

\begin{sphinxuseclass}{cell_input}
\begin{sphinxVerbatim}[commandchars=\\\{\}]
\PYG{n}{name} \PYG{o}{=} \PYG{l+s+s1}{\PYGZsq{}}\PYG{l+s+s1}{aRThUr}\PYG{l+s+s1}{\PYGZsq{}}
\PYG{n+nb}{print}\PYG{p}{(}\PYG{n}{name}\PYG{p}{)}
\PYG{n+nb}{print}\PYG{p}{(}\PYG{n}{name}\PYG{o}{.}\PYG{n}{swapcase}\PYG{p}{(}\PYG{p}{)}\PYG{p}{)} \PYG{c+c1}{\PYGZsh{} \PYGZsq{}a\PYGZsq{} becomes \PYGZsq{}A\PYGZsq{}, \PYGZsq{}R\PYGZsq{} becomes \PYGZsq{}r\PYGZsq{}, and so on}
\end{sphinxVerbatim}

\end{sphinxuseclass}\end{sphinxVerbatimInput}
\begin{sphinxVerbatimOutput}

\begin{sphinxuseclass}{cell_output}
\begin{sphinxVerbatim}[commandchars=\\\{\}]
aRThUr
ArtHuR
\end{sphinxVerbatim}

\end{sphinxuseclass}\end{sphinxVerbatimOutput}

\end{sphinxuseclass}

\subsection{join()}
\label{\detokenize{strings:join}}\begin{itemize}
\item {} 
\sphinxAtStartPar
Concatenate a list of strings.

\item {} 
\sphinxAtStartPar
Insert the string, whose method is called, between each given string.

\item {} 
\sphinxAtStartPar
Return the result as a new string.

\item {} 
\sphinxAtStartPar
Example: \sphinxcode{\sphinxupquote{'\sphinxhyphen{}\sphinxhyphen{}'.join({[}'ab', 'pq', 'rs'{]})}} returns \sphinxcode{\sphinxupquote{'ab\sphinxhyphen{}\sphinxhyphen{}pq\sphinxhyphen{}\sphinxhyphen{}rs'}}

\end{itemize}

\begin{sphinxuseclass}{cell}\begin{sphinxVerbatimInput}

\begin{sphinxuseclass}{cell_input}
\begin{sphinxVerbatim}[commandchars=\\\{\}]
\PYG{n+nb}{print}\PYG{p}{(}\PYG{l+s+s1}{\PYGZsq{}}\PYG{l+s+s1}{\PYGZhy{}\PYGZhy{}}\PYG{l+s+s1}{\PYGZsq{}}\PYG{o}{.}\PYG{n}{join}\PYG{p}{(}\PYG{p}{[}\PYG{l+s+s1}{\PYGZsq{}}\PYG{l+s+s1}{ab}\PYG{l+s+s1}{\PYGZsq{}}\PYG{p}{,} \PYG{l+s+s1}{\PYGZsq{}}\PYG{l+s+s1}{pq}\PYG{l+s+s1}{\PYGZsq{}}\PYG{p}{,} \PYG{l+s+s1}{\PYGZsq{}}\PYG{l+s+s1}{rs}\PYG{l+s+s1}{\PYGZsq{}}\PYG{p}{]}\PYG{p}{)}\PYG{p}{)}
\end{sphinxVerbatim}

\end{sphinxuseclass}\end{sphinxVerbatimInput}
\begin{sphinxVerbatimOutput}

\begin{sphinxuseclass}{cell_output}
\begin{sphinxVerbatim}[commandchars=\\\{\}]
ab\PYGZhy{}\PYGZhy{}pq\PYGZhy{}\PYGZhy{}rs
\end{sphinxVerbatim}

\end{sphinxuseclass}\end{sphinxVerbatimOutput}

\end{sphinxuseclass}

\section{Parsing Strings}
\label{\detokenize{strings:parsing-strings}}\begin{itemize}
\item {} 
\sphinxAtStartPar
By using string methods, you can analyze a string and extract meaningful information about the string.

\item {} 
\sphinxAtStartPar
You can also perform specific operations based on the structure and content of the string.

\item {} 
\sphinxAtStartPar
Example:
\begin{itemize}
\item {} 
\sphinxAtStartPar
From the given message below, extract the company name and capitalize it.

\item {} 
\sphinxAtStartPar
The company name is between the characters \sphinxcode{\sphinxupquote{@}} and \sphinxcode{\sphinxupquote{.}}

\item {} 
\sphinxAtStartPar
e can use the \sphinxcode{\sphinxupquote{find()}} method to find the indexes of these two characters.

\item {} 
\sphinxAtStartPar
There are multiple \sphinxcode{\sphinxupquote{.}} characters, so we need to find the first one that comes after \sphinxcode{\sphinxupquote{@}}.”

\end{itemize}

\end{itemize}

\begin{sphinxuseclass}{cell}\begin{sphinxVerbatimInput}

\begin{sphinxuseclass}{cell_input}
\begin{sphinxVerbatim}[commandchars=\\\{\}]
\PYG{n}{message} \PYG{o}{=} \PYG{l+s+s1}{\PYGZsq{}}\PYG{l+s+s1}{Hello. My name is Tom. I live in California. My email address is tom@tesla.com. I will be in NY next week.}\PYG{l+s+s1}{\PYGZsq{}}
\end{sphinxVerbatim}

\end{sphinxuseclass}\end{sphinxVerbatimInput}

\end{sphinxuseclass}
\begin{sphinxuseclass}{cell}\begin{sphinxVerbatimInput}

\begin{sphinxuseclass}{cell_input}
\begin{sphinxVerbatim}[commandchars=\\\{\}]
\PYG{n}{index\PYGZus{}at} \PYG{o}{=} \PYG{n}{message}\PYG{o}{.}\PYG{n}{find}\PYG{p}{(}\PYG{l+s+s1}{\PYGZsq{}}\PYG{l+s+s1}{@}\PYG{l+s+s1}{\PYGZsq{}}\PYG{p}{)}                    \PYG{c+c1}{\PYGZsh{} index of @}
\PYG{n}{index\PYGZus{}period} \PYG{o}{=} \PYG{n}{message}\PYG{o}{.}\PYG{n}{find}\PYG{p}{(}\PYG{l+s+s1}{\PYGZsq{}}\PYG{l+s+s1}{.}\PYG{l+s+s1}{\PYGZsq{}}\PYG{p}{,} \PYG{n}{index\PYGZus{}at}\PYG{p}{)}      \PYG{c+c1}{\PYGZsh{} index of first . after @ }
\end{sphinxVerbatim}

\end{sphinxuseclass}\end{sphinxVerbatimInput}

\end{sphinxuseclass}\begin{itemize}
\item {} 
\sphinxAtStartPar
To grab the company name, we need to use slicing.

\item {} 
\sphinxAtStartPar
Slicing must start from \sphinxcode{\sphinxupquote{index\_at + 1}} because if you start from index\_at, the slice will include @.

\item {} 
\sphinxAtStartPar
Slicing must end at \sphinxcode{\sphinxupquote{index\_period}} because the end index is not included.

\end{itemize}

\begin{sphinxuseclass}{cell}\begin{sphinxVerbatimInput}

\begin{sphinxuseclass}{cell_input}
\begin{sphinxVerbatim}[commandchars=\\\{\}]
\PYG{n+nb}{print}\PYG{p}{(}\PYG{n}{message}\PYG{p}{[}\PYG{n}{index\PYGZus{}at}\PYG{o}{+}\PYG{l+m+mi}{1}\PYG{p}{:}\PYG{n}{index\PYGZus{}period}\PYG{p}{]}\PYG{o}{.}\PYG{n}{capitalize}\PYG{p}{(}\PYG{p}{)}\PYG{p}{)}
\end{sphinxVerbatim}

\end{sphinxuseclass}\end{sphinxVerbatimInput}
\begin{sphinxVerbatimOutput}

\begin{sphinxuseclass}{cell_output}
\begin{sphinxVerbatim}[commandchars=\\\{\}]
Tesla
\end{sphinxVerbatim}

\end{sphinxuseclass}\end{sphinxVerbatimOutput}

\end{sphinxuseclass}
\sphinxstepscope


\section{Strings Debugging}
\label{\detokenize{strings_debug:strings-debugging}}\label{\detokenize{strings_debug::doc}}\begin{itemize}
\item {} 
\sphinxAtStartPar
Each of the following short code contains one or more bugs.     

\item {} 
\sphinxAtStartPar
Please identify and correct these bugs.

\item {} 
\sphinxAtStartPar
Provide an explanation for your answer.

\end{itemize}


\subsection{Question\sphinxhyphen{}1}
\label{\detokenize{strings_debug:question-1}}
\begin{sphinxVerbatim}[commandchars=\\\{\}]
\PYG{l+m+mf}{3.0}\PYG{o}{*}\PYG{l+s+s1}{\PYGZsq{}}\PYG{l+s+s1}{HI}\PYG{l+s+s1}{\PYGZsq{}}
\end{sphinxVerbatim}

\begin{sphinxadmonition}{note}{Solution}

\sphinxAtStartPar
Only integers can be used for repetition; 3.0 is a float.
\end{sphinxadmonition}


\subsection{Question\sphinxhyphen{}2}
\label{\detokenize{strings_debug:question-2}}
\begin{sphinxVerbatim}[commandchars=\\\{\}]
\PYG{l+m+mi}{3}\PYG{o}{+}\PYG{l+s+s1}{\PYGZsq{}}\PYG{l+s+s1}{HI}\PYG{l+s+s1}{\PYGZsq{}}
\end{sphinxVerbatim}

\begin{sphinxadmonition}{note}{Solution}

\sphinxAtStartPar
Concatenation can be done with two strings, but 3 is not a string. You can convert 3 to a string, and after that, you can perform concatenation: \sphinxcode{\sphinxupquote{str(3) + 'HI'.}}
\end{sphinxadmonition}


\subsection{Question\sphinxhyphen{}3}
\label{\detokenize{strings_debug:question-3}}
\begin{sphinxVerbatim}[commandchars=\\\{\}]
\PYG{n}{name} \PYG{o}{=} \PYG{l+s+s1}{\PYGZsq{}}\PYG{l+s+s1}{michael}\PYG{l+s+s1}{\PYGZdq{}}
\end{sphinxVerbatim}

\begin{sphinxadmonition}{note}{Solution}

\sphinxAtStartPar
Both of the quotes on the left and right must be the same. Either \sphinxcode{\sphinxupquote{'michael'}} or \sphinxcode{\sphinxupquote{"michael"}}  will solve the problem.
\end{sphinxadmonition}


\subsection{Question\sphinxhyphen{}4}
\label{\detokenize{strings_debug:question-4}}
\begin{sphinxVerbatim}[commandchars=\\\{\}]
\PYG{n+nb}{print}\PYG{p}{(}\PYG{l+s+s1}{\PYGZsq{}}\PYG{l+s+s1}{he}\PYG{l+s+s1}{\PYGZsq{}}\PYG{n}{s} \PYG{n}{coming}\PYG{l+s+s1}{\PYGZsq{}}\PYG{l+s+s1}{)}
\end{sphinxVerbatim}

\begin{sphinxadmonition}{note}{Solution}

\sphinxAtStartPar
The single quote of the string causes the problem since the string is also created by using single quotes.
\begin{itemize}
\item {} 
\sphinxAtStartPar
To fix the problem you can either use \sphinxcode{\sphinxupquote{\textbackslash{}}} to make the single quote of the string a character: \sphinxcode{\sphinxupquote{'he\textbackslash{}'s coming'}}

\item {} 
\sphinxAtStartPar
Instead of single quotes, use double ones: \sphinxcode{\sphinxupquote{"he\textbackslash{}'s coming"}}

\end{itemize}
\end{sphinxadmonition}


\subsection{Question\sphinxhyphen{}5}
\label{\detokenize{strings_debug:question-5}}
\begin{sphinxVerbatim}[commandchars=\\\{\}]
\PYG{n}{name} \PYG{o}{=} \PYG{l+s+s1}{\PYGZsq{}}\PYG{l+s+s1}{Michael}
\PYG{n}{Jordan}\PYG{l+s+s1}{\PYGZsq{}}
\PYG{n+nb}{print}\PYG{p}{(}\PYG{n}{name}\PYG{p}{)}
\end{sphinxVerbatim}

\begin{sphinxadmonition}{note}{Solution}

\sphinxAtStartPar
Triple single or double quotes must be used for strings that have multiple lines.
\end{sphinxadmonition}


\subsection{Question\sphinxhyphen{}6}
\label{\detokenize{strings_debug:question-6}}
\begin{sphinxVerbatim}[commandchars=\\\{\}]
\PYG{n}{name} \PYG{o}{=} \PYG{l+s+s1}{\PYGZsq{}}\PYG{l+s+s1}{Brian}\PYG{l+s+s1}{\PYGZsq{}}
\PYG{n+nb}{print}\PYG{p}{(}\PYG{n}{name}\PYG{p}{[}\PYG{l+m+mi}{5}\PYG{p}{]}\PYG{p}{)}
\end{sphinxVerbatim}

\begin{sphinxadmonition}{note}{Solution}

\sphinxAtStartPar
The index is out of range because the largest positive index is 4, as indexing starts from 0.
\end{sphinxadmonition}

\sphinxstepscope


\section{Strings Output}
\label{\detokenize{strings_output:strings-output}}\label{\detokenize{strings_output::doc}}\begin{itemize}
\item {} 
\sphinxAtStartPar
Find the output of the following code.

\item {} 
\sphinxAtStartPar
Please don’t run the code before giving your answer.     

\end{itemize}


\subsection{Question\sphinxhyphen{}1}
\label{\detokenize{strings_output:question-1}}
\begin{sphinxuseclass}{cell}
\begin{sphinxuseclass}{tag_hide-output}\begin{sphinxVerbatimInput}

\begin{sphinxuseclass}{cell_input}
\begin{sphinxVerbatim}[commandchars=\\\{\}]
\PYG{n+nb}{print}\PYG{p}{(}\PYG{l+s+s1}{\PYGZsq{}}\PYG{l+s+s1}{Mark}\PYG{l+s+s1}{\PYGZsq{}}\PYG{o}{+}\PYG{l+s+s1}{\PYGZsq{}}\PYG{l+s+s1}{Twain}\PYG{l+s+s1}{\PYGZsq{}}\PYG{p}{)}
\end{sphinxVerbatim}

\end{sphinxuseclass}\end{sphinxVerbatimInput}

\end{sphinxuseclass}
\end{sphinxuseclass}

\subsection{Question\sphinxhyphen{}2}
\label{\detokenize{strings_output:question-2}}
\begin{sphinxuseclass}{cell}
\begin{sphinxuseclass}{tag_hide-output}\begin{sphinxVerbatimInput}

\begin{sphinxuseclass}{cell_input}
\begin{sphinxVerbatim}[commandchars=\\\{\}]
\PYG{n+nb}{print}\PYG{p}{(}\PYG{l+s+s1}{\PYGZsq{}}\PYG{l+s+s1}{Mark }\PYG{l+s+s1}{\PYGZsq{}}\PYG{o}{+}\PYG{l+s+s1}{\PYGZsq{}}\PYG{l+s+s1}{Twain}\PYG{l+s+s1}{\PYGZsq{}}\PYG{p}{)}
\end{sphinxVerbatim}

\end{sphinxuseclass}\end{sphinxVerbatimInput}

\end{sphinxuseclass}
\end{sphinxuseclass}

\subsection{Question\sphinxhyphen{}3}
\label{\detokenize{strings_output:question-3}}
\begin{sphinxuseclass}{cell}
\begin{sphinxuseclass}{tag_hide-output}\begin{sphinxVerbatimInput}

\begin{sphinxuseclass}{cell_input}
\begin{sphinxVerbatim}[commandchars=\\\{\}]
\PYG{n+nb}{print}\PYG{p}{(}\PYG{l+s+s1}{\PYGZsq{}}\PYG{l+s+s1}{Mark}\PYG{l+s+se}{\PYGZbs{}t}\PYG{l+s+s1}{Twain}\PYG{l+s+s1}{\PYGZsq{}}\PYG{p}{)}
\end{sphinxVerbatim}

\end{sphinxuseclass}\end{sphinxVerbatimInput}

\end{sphinxuseclass}
\end{sphinxuseclass}

\subsection{Question\sphinxhyphen{}4}
\label{\detokenize{strings_output:question-4}}
\begin{sphinxuseclass}{cell}
\begin{sphinxuseclass}{tag_hide-output}\begin{sphinxVerbatimInput}

\begin{sphinxuseclass}{cell_input}
\begin{sphinxVerbatim}[commandchars=\\\{\}]
\PYG{n+nb}{print}\PYG{p}{(}\PYG{l+s+s1}{\PYGZsq{}}\PYG{l+s+s1}{Mark}\PYG{l+s+se}{\PYGZbs{}n}\PYG{l+s+s1}{Twain}\PYG{l+s+s1}{\PYGZsq{}}\PYG{p}{)}
\end{sphinxVerbatim}

\end{sphinxuseclass}\end{sphinxVerbatimInput}

\end{sphinxuseclass}
\end{sphinxuseclass}

\subsection{Question\sphinxhyphen{}5}
\label{\detokenize{strings_output:question-5}}
\begin{sphinxuseclass}{cell}
\begin{sphinxuseclass}{tag_hide-output}\begin{sphinxVerbatimInput}

\begin{sphinxuseclass}{cell_input}
\begin{sphinxVerbatim}[commandchars=\\\{\}]
\PYG{n+nb}{print}\PYG{p}{(}\PYG{l+s+s1}{\PYGZsq{}}\PYG{l+s+s1}{Mark}\PYG{l+s+se}{\PYGZbs{}b}\PYG{l+s+s1}{Twain}\PYG{l+s+s1}{\PYGZsq{}}\PYG{p}{)}
\end{sphinxVerbatim}

\end{sphinxuseclass}\end{sphinxVerbatimInput}

\end{sphinxuseclass}
\end{sphinxuseclass}

\subsection{Question\sphinxhyphen{}6}
\label{\detokenize{strings_output:question-6}}
\begin{sphinxuseclass}{cell}
\begin{sphinxuseclass}{tag_hide-output}\begin{sphinxVerbatimInput}

\begin{sphinxuseclass}{cell_input}
\begin{sphinxVerbatim}[commandchars=\\\{\}]
\PYG{n+nb}{print}\PYG{p}{(}\PYG{l+s+s1}{\PYGZsq{}}\PYG{l+s+s1}{Mark}\PYG{l+s+se}{\PYGZbs{}b}\PYG{l+s+se}{\PYGZbs{}b}\PYG{l+s+s1}{Twain}\PYG{l+s+s1}{\PYGZsq{}}\PYG{p}{)}
\end{sphinxVerbatim}

\end{sphinxuseclass}\end{sphinxVerbatimInput}

\end{sphinxuseclass}
\end{sphinxuseclass}

\subsection{Question\sphinxhyphen{}7}
\label{\detokenize{strings_output:question-7}}
\begin{sphinxuseclass}{cell}
\begin{sphinxuseclass}{tag_hide-output}\begin{sphinxVerbatimInput}

\begin{sphinxuseclass}{cell_input}
\begin{sphinxVerbatim}[commandchars=\\\{\}]
\PYG{n+nb}{print}\PYG{p}{(}\PYG{l+s+s1}{\PYGZsq{}}\PYG{l+s+s1}{Mark}\PYG{l+s+se}{\PYGZbs{}r}\PYG{l+s+s1}{Twain}\PYG{l+s+s1}{\PYGZsq{}}\PYG{p}{)}
\end{sphinxVerbatim}

\end{sphinxuseclass}\end{sphinxVerbatimInput}

\end{sphinxuseclass}
\end{sphinxuseclass}

\subsection{Question\sphinxhyphen{}8}
\label{\detokenize{strings_output:question-8}}
\begin{sphinxuseclass}{cell}
\begin{sphinxuseclass}{tag_hide-output}\begin{sphinxVerbatimInput}

\begin{sphinxuseclass}{cell_input}
\begin{sphinxVerbatim}[commandchars=\\\{\}]
\PYG{n+nb}{print}\PYG{p}{(}\PYG{l+s+s1}{\PYGZsq{}}\PYG{l+s+s1}{Mary}\PYG{l+s+se}{\PYGZbs{}\PYGZsq{}}\PYG{l+s+s1}{s book.}\PYG{l+s+s1}{\PYGZsq{}}\PYG{p}{)}
\end{sphinxVerbatim}

\end{sphinxuseclass}\end{sphinxVerbatimInput}

\end{sphinxuseclass}
\end{sphinxuseclass}

\subsection{Question\sphinxhyphen{}9}
\label{\detokenize{strings_output:question-9}}
\begin{sphinxuseclass}{cell}
\begin{sphinxuseclass}{tag_hide-output}\begin{sphinxVerbatimInput}

\begin{sphinxuseclass}{cell_input}
\begin{sphinxVerbatim}[commandchars=\\\{\}]
\PYG{n+nb}{print}\PYG{p}{(}\PYG{l+s+sa}{r}\PYG{l+s+s1}{\PYGZsq{}}\PYG{l+s+s1}{Mary}\PYG{l+s+se}{\PYGZbs{}\PYGZsq{}}\PYG{l+s+s1}{s }\PYG{l+s+s1}{\PYGZbs{}}\PYG{l+s+s1}{n book.}\PYG{l+s+s1}{\PYGZsq{}}\PYG{p}{)}
\end{sphinxVerbatim}

\end{sphinxuseclass}\end{sphinxVerbatimInput}

\end{sphinxuseclass}
\end{sphinxuseclass}

\subsection{Question\sphinxhyphen{}10}
\label{\detokenize{strings_output:question-10}}
\begin{sphinxuseclass}{cell}
\begin{sphinxuseclass}{tag_hide-output}\begin{sphinxVerbatimInput}

\begin{sphinxuseclass}{cell_input}
\begin{sphinxVerbatim}[commandchars=\\\{\}]
\PYG{n}{text} \PYG{o}{=} \PYG{l+s+s1}{\PYGZsq{}}\PYG{l+s+s1}{Hello World.}\PYG{l+s+s1}{\PYGZsq{}}
\PYG{n+nb}{len}\PYG{p}{(}\PYG{n}{text}\PYG{p}{)}
\end{sphinxVerbatim}

\end{sphinxuseclass}\end{sphinxVerbatimInput}

\end{sphinxuseclass}
\end{sphinxuseclass}

\subsection{Question\sphinxhyphen{}11}
\label{\detokenize{strings_output:question-11}}
\begin{sphinxuseclass}{cell}
\begin{sphinxuseclass}{tag_hide-output}\begin{sphinxVerbatimInput}

\begin{sphinxuseclass}{cell_input}
\begin{sphinxVerbatim}[commandchars=\\\{\}]
\PYG{n}{x} \PYG{o}{=} \PYG{l+s+s1}{\PYGZsq{}}\PYG{l+s+s1}{abcdefg}\PYG{l+s+s1}{\PYGZsq{}}

\PYG{n+nb}{print}\PYG{p}{(}\PYG{n}{x}\PYG{p}{[}\PYG{p}{:}\PYG{p}{]}\PYG{p}{)}
\PYG{n+nb}{print}\PYG{p}{(}\PYG{n}{x}\PYG{p}{[}\PYG{p}{:}\PYG{o}{\PYGZhy{}}\PYG{l+m+mi}{3}\PYG{p}{]}\PYG{p}{)}
\PYG{n+nb}{print}\PYG{p}{(}\PYG{n}{x}\PYG{p}{[}\PYG{o}{\PYGZhy{}}\PYG{l+m+mi}{4}\PYG{p}{:}\PYG{p}{]}\PYG{p}{)}
\PYG{n+nb}{print}\PYG{p}{(}\PYG{n}{x}\PYG{p}{[}\PYG{l+m+mi}{2}\PYG{p}{:}\PYG{l+m+mi}{4}\PYG{p}{]}\PYG{p}{)}
\PYG{n+nb}{print}\PYG{p}{(}\PYG{n}{x}\PYG{p}{[}\PYG{o}{\PYGZhy{}}\PYG{l+m+mi}{4}\PYG{p}{:}\PYG{o}{\PYGZhy{}}\PYG{l+m+mi}{2}\PYG{p}{]}\PYG{p}{)}
\end{sphinxVerbatim}

\end{sphinxuseclass}\end{sphinxVerbatimInput}

\end{sphinxuseclass}
\end{sphinxuseclass}

\subsection{Question\sphinxhyphen{}12}
\label{\detokenize{strings_output:question-12}}
\begin{sphinxuseclass}{cell}
\begin{sphinxuseclass}{tag_hide-output}\begin{sphinxVerbatimInput}

\begin{sphinxuseclass}{cell_input}
\begin{sphinxVerbatim}[commandchars=\\\{\}]
\PYG{n}{name} \PYG{o}{=} \PYG{l+s+s1}{\PYGZsq{}}\PYG{l+s+s1}{Mary}\PYG{l+s+s1}{\PYGZsq{}}
\PYG{n+nb}{print}\PYG{p}{(}\PYG{l+s+sa}{f}\PYG{l+s+s1}{\PYGZsq{}}\PYG{l+s+s1}{My name is (name)}\PYG{l+s+s1}{\PYGZsq{}}\PYG{p}{)}
\end{sphinxVerbatim}

\end{sphinxuseclass}\end{sphinxVerbatimInput}

\end{sphinxuseclass}
\end{sphinxuseclass}

\subsection{Question\sphinxhyphen{}13}
\label{\detokenize{strings_output:question-13}}
\begin{sphinxuseclass}{cell}
\begin{sphinxuseclass}{tag_hide-output}\begin{sphinxVerbatimInput}

\begin{sphinxuseclass}{cell_input}
\begin{sphinxVerbatim}[commandchars=\\\{\}]
\PYG{n}{name} \PYG{o}{=} \PYG{l+s+s1}{\PYGZsq{}}\PYG{l+s+s1}{George}\PYG{l+s+s1}{\PYGZsq{}}
\PYG{n}{salary} \PYG{o}{=} \PYG{l+m+mi}{2000}

\PYG{n+nb}{print}\PYG{p}{(}\PYG{l+s+sa}{f}\PYG{l+s+s1}{\PYGZsq{}}\PYG{l+s+s1}{My name is }\PYG{l+s+si}{\PYGZob{}}\PYG{n}{name}\PYG{l+s+si}{\PYGZcb{}}\PYG{l+s+s1}{. I earn \PYGZdl{}}\PYG{l+s+si}{\PYGZob{}}\PYG{n}{salary}\PYG{l+s+si}{\PYGZcb{}}\PYG{l+s+s1}{ per month.}\PYG{l+s+s1}{\PYGZsq{}}\PYG{p}{)}
\end{sphinxVerbatim}

\end{sphinxuseclass}\end{sphinxVerbatimInput}

\end{sphinxuseclass}
\end{sphinxuseclass}

\subsection{Question\sphinxhyphen{}14}
\label{\detokenize{strings_output:question-14}}
\begin{sphinxuseclass}{cell}
\begin{sphinxuseclass}{tag_hide-output}\begin{sphinxVerbatimInput}

\begin{sphinxuseclass}{cell_input}
\begin{sphinxVerbatim}[commandchars=\\\{\}]
\PYG{n}{x} \PYG{o}{=} \PYG{l+s+s1}{\PYGZsq{}}\PYG{l+s+s1}{california}\PYG{l+s+s1}{\PYGZsq{}}

\PYG{n+nb}{print}\PYG{p}{(}\PYG{n}{x}\PYG{o}{.}\PYG{n}{find}\PYG{p}{(}\PYG{l+s+s1}{\PYGZsq{}}\PYG{l+s+s1}{f}\PYG{l+s+s1}{\PYGZsq{}}\PYG{p}{)}\PYG{p}{)}
\PYG{n+nb}{print}\PYG{p}{(}\PYG{n}{x}\PYG{o}{.}\PYG{n}{find}\PYG{p}{(}\PYG{l+s+s1}{\PYGZsq{}}\PYG{l+s+s1}{A}\PYG{l+s+s1}{\PYGZsq{}}\PYG{p}{)}\PYG{p}{)}
\PYG{n+nb}{print}\PYG{p}{(}\PYG{n}{x}\PYG{o}{.}\PYG{n}{find}\PYG{p}{(}\PYG{l+s+s1}{\PYGZsq{}}\PYG{l+s+s1}{a}\PYG{l+s+s1}{\PYGZsq{}}\PYG{p}{,} \PYG{l+m+mi}{4}\PYG{p}{)}\PYG{p}{)}
\end{sphinxVerbatim}

\end{sphinxuseclass}\end{sphinxVerbatimInput}

\end{sphinxuseclass}
\end{sphinxuseclass}

\subsection{Question\sphinxhyphen{}15}
\label{\detokenize{strings_output:question-15}}
\begin{sphinxuseclass}{cell}
\begin{sphinxuseclass}{tag_hide-output}\begin{sphinxVerbatimInput}

\begin{sphinxuseclass}{cell_input}
\begin{sphinxVerbatim}[commandchars=\\\{\}]
\PYG{n}{state} \PYG{o}{=} \PYG{l+s+s1}{\PYGZsq{}}\PYG{l+s+s1}{utah}\PYG{l+s+s1}{\PYGZsq{}}
\PYG{n+nb}{print}\PYG{p}{(}\PYG{l+m+mi}{3}\PYG{o}{*}\PYG{l+s+s1}{\PYGZsq{}}\PYG{l+s+s1}{utah}\PYG{l+s+s1}{\PYGZsq{}}\PYG{p}{)}
\end{sphinxVerbatim}

\end{sphinxuseclass}\end{sphinxVerbatimInput}

\end{sphinxuseclass}
\end{sphinxuseclass}

\subsection{Question\sphinxhyphen{}16}
\label{\detokenize{strings_output:question-16}}
\begin{sphinxuseclass}{cell}
\begin{sphinxuseclass}{tag_hide-output}\begin{sphinxVerbatimInput}

\begin{sphinxuseclass}{cell_input}
\begin{sphinxVerbatim}[commandchars=\\\{\}]
\PYG{n}{name} \PYG{o}{=} \PYG{l+s+s1}{\PYGZsq{}}\PYG{l+s+s1}{MaRk tWAin}\PYG{l+s+s1}{\PYGZsq{}}

\PYG{n+nb}{print}\PYG{p}{(}\PYG{n}{name}\PYG{o}{.}\PYG{n}{upper}\PYG{p}{(}\PYG{p}{)}\PYG{p}{)}
\PYG{n+nb}{print}\PYG{p}{(}\PYG{n}{name}\PYG{o}{.}\PYG{n}{lower}\PYG{p}{(}\PYG{p}{)}\PYG{p}{)}
\PYG{n+nb}{print}\PYG{p}{(}\PYG{n}{name}\PYG{o}{.}\PYG{n}{capitalize}\PYG{p}{(}\PYG{p}{)}\PYG{p}{)}
\PYG{n+nb}{print}\PYG{p}{(}\PYG{n}{name}\PYG{o}{.}\PYG{n}{title}\PYG{p}{(}\PYG{p}{)}\PYG{p}{)}
\end{sphinxVerbatim}

\end{sphinxuseclass}\end{sphinxVerbatimInput}

\end{sphinxuseclass}
\end{sphinxuseclass}

\subsection{Question\sphinxhyphen{}17}
\label{\detokenize{strings_output:question-17}}
\begin{sphinxuseclass}{cell}
\begin{sphinxuseclass}{tag_hide-output}\begin{sphinxVerbatimInput}

\begin{sphinxuseclass}{cell_input}
\begin{sphinxVerbatim}[commandchars=\\\{\}]
\PYG{n}{country} \PYG{o}{=} \PYG{l+s+s1}{\PYGZsq{}}\PYG{l+s+s1}{Morocco}\PYG{l+s+s1}{\PYGZsq{}}

\PYG{n+nb}{print}\PYG{p}{(}\PYG{n}{country}\PYG{o}{.}\PYG{n}{find}\PYG{p}{(}\PYG{l+s+s1}{\PYGZsq{}}\PYG{l+s+s1}{r}\PYG{l+s+s1}{\PYGZsq{}}\PYG{p}{)}\PYG{p}{)}
\PYG{n+nb}{print}\PYG{p}{(}\PYG{n}{country}\PYG{o}{.}\PYG{n}{find}\PYG{p}{(}\PYG{l+s+s1}{\PYGZsq{}}\PYG{l+s+s1}{o}\PYG{l+s+s1}{\PYGZsq{}}\PYG{p}{)}\PYG{p}{)}
\PYG{n+nb}{print}\PYG{p}{(}\PYG{n}{country}\PYG{o}{.}\PYG{n}{find}\PYG{p}{(}\PYG{l+s+s1}{\PYGZsq{}}\PYG{l+s+s1}{c}\PYG{l+s+s1}{\PYGZsq{}}\PYG{p}{)}\PYG{p}{)}
\PYG{n+nb}{print}\PYG{p}{(}\PYG{n}{country}\PYG{o}{.}\PYG{n}{find}\PYG{p}{(}\PYG{l+s+s1}{\PYGZsq{}}\PYG{l+s+s1}{o}\PYG{l+s+s1}{\PYGZsq{}}\PYG{p}{,}\PYG{l+m+mi}{4}\PYG{p}{)}\PYG{p}{)}
\PYG{n+nb}{print}\PYG{p}{(}\PYG{n}{country}\PYG{o}{.}\PYG{n}{find}\PYG{p}{(}\PYG{l+s+s1}{\PYGZsq{}}\PYG{l+s+s1}{t}\PYG{l+s+s1}{\PYGZsq{}}\PYG{p}{)}\PYG{p}{)}
\end{sphinxVerbatim}

\end{sphinxuseclass}\end{sphinxVerbatimInput}

\end{sphinxuseclass}
\end{sphinxuseclass}

\subsection{Question\sphinxhyphen{}18}
\label{\detokenize{strings_output:question-18}}
\begin{sphinxuseclass}{cell}
\begin{sphinxuseclass}{tag_hide-output}\begin{sphinxVerbatimInput}

\begin{sphinxuseclass}{cell_input}
\begin{sphinxVerbatim}[commandchars=\\\{\}]
\PYG{n}{name} \PYG{o}{=} \PYG{l+s+s1}{\PYGZsq{}}\PYG{l+s+s1}{   Liz   }\PYG{l+s+s1}{\PYGZsq{}}
\PYG{n+nb}{print}\PYG{p}{(}\PYG{n}{name}\PYG{o}{.}\PYG{n}{strip}\PYG{p}{(}\PYG{p}{)}\PYG{p}{)}
\end{sphinxVerbatim}

\end{sphinxuseclass}\end{sphinxVerbatimInput}

\end{sphinxuseclass}
\end{sphinxuseclass}

\subsection{Question\sphinxhyphen{}19}
\label{\detokenize{strings_output:question-19}}
\begin{sphinxuseclass}{cell}
\begin{sphinxuseclass}{tag_hide-output}\begin{sphinxVerbatimInput}

\begin{sphinxuseclass}{cell_input}
\begin{sphinxVerbatim}[commandchars=\\\{\}]
\PYG{n}{name} \PYG{o}{=} \PYG{l+s+s1}{\PYGZsq{}}\PYG{l+s+s1}{   Liz   }\PYG{l+s+s1}{\PYGZsq{}}
\PYG{n+nb}{print}\PYG{p}{(}\PYG{n}{name}\PYG{o}{.}\PYG{n}{lstrip}\PYG{p}{(}\PYG{p}{)}\PYG{o}{+}\PYG{l+s+s1}{\PYGZsq{}}\PYG{l+s+s1}{T}\PYG{l+s+s1}{\PYGZsq{}}\PYG{p}{)}
\end{sphinxVerbatim}

\end{sphinxuseclass}\end{sphinxVerbatimInput}

\end{sphinxuseclass}
\end{sphinxuseclass}

\subsection{Question\sphinxhyphen{}20}
\label{\detokenize{strings_output:question-20}}
\begin{sphinxuseclass}{cell}
\begin{sphinxuseclass}{tag_hide-output}\begin{sphinxVerbatimInput}

\begin{sphinxuseclass}{cell_input}
\begin{sphinxVerbatim}[commandchars=\\\{\}]
\PYG{n}{name} \PYG{o}{=} \PYG{l+s+s1}{\PYGZsq{}}\PYG{l+s+s1}{ryan}\PYG{l+s+s1}{\PYGZsq{}}
\PYG{n}{name}\PYG{o}{.}\PYG{n}{startswith}\PYG{p}{(}\PYG{l+s+s1}{\PYGZsq{}}\PYG{l+s+s1}{R}\PYG{l+s+s1}{\PYGZsq{}}\PYG{p}{)}
\end{sphinxVerbatim}

\end{sphinxuseclass}\end{sphinxVerbatimInput}

\end{sphinxuseclass}
\end{sphinxuseclass}

\subsection{Question\sphinxhyphen{}21}
\label{\detokenize{strings_output:question-21}}
\begin{sphinxuseclass}{cell}
\begin{sphinxuseclass}{tag_hide-output}\begin{sphinxVerbatimInput}

\begin{sphinxuseclass}{cell_input}
\begin{sphinxVerbatim}[commandchars=\\\{\}]
\PYG{n}{word} \PYG{o}{=} \PYG{l+s+s1}{\PYGZsq{}}\PYG{l+s+s1}{ab,cde=abc,de}\PYG{l+s+s1}{\PYGZsq{}}

\PYG{n}{index1} \PYG{o}{=} \PYG{n}{word}\PYG{o}{.}\PYG{n}{find}\PYG{p}{(}\PYG{l+s+s1}{\PYGZsq{}}\PYG{l+s+s1}{f}\PYG{l+s+s1}{\PYGZsq{}}\PYG{p}{)}
\PYG{n}{index2} \PYG{o}{=} \PYG{n}{word}\PYG{o}{.}\PYG{n}{find}\PYG{p}{(}\PYG{l+s+s1}{\PYGZsq{}}\PYG{l+s+s1}{=}\PYG{l+s+s1}{\PYGZsq{}}\PYG{p}{)}
\PYG{n}{index3} \PYG{o}{=} \PYG{n}{word}\PYG{o}{.}\PYG{n}{find}\PYG{p}{(}\PYG{l+s+s1}{\PYGZsq{}}\PYG{l+s+s1}{,}\PYG{l+s+s1}{\PYGZsq{}}\PYG{p}{,}\PYG{n}{index2}\PYG{p}{)}

\PYG{n+nb}{print}\PYG{p}{(}\PYG{n}{word}\PYG{p}{[}\PYG{n}{index2}\PYG{p}{:}\PYG{n}{index3}\PYG{o}{+}\PYG{n}{index1}\PYG{p}{]}\PYG{p}{)}
\end{sphinxVerbatim}

\end{sphinxuseclass}\end{sphinxVerbatimInput}

\end{sphinxuseclass}
\end{sphinxuseclass}

\subsection{Question\sphinxhyphen{}22}
\label{\detokenize{strings_output:question-22}}
\begin{sphinxuseclass}{cell}
\begin{sphinxuseclass}{tag_hide-output}\begin{sphinxVerbatimInput}

\begin{sphinxuseclass}{cell_input}
\begin{sphinxVerbatim}[commandchars=\\\{\}]
\PYG{n}{state} \PYG{o}{=} \PYG{l+s+s1}{\PYGZsq{}}\PYG{l+s+s1}{Florida}\PYG{l+s+s1}{\PYGZsq{}}

\PYG{n}{x} \PYG{o}{=} \PYG{n}{state}\PYG{p}{[}\PYG{l+m+mi}{1}\PYG{p}{:}\PYG{l+m+mi}{4}\PYG{p}{:}\PYG{l+m+mi}{2}\PYG{p}{]}\PYG{o}{.}\PYG{n}{upper}\PYG{p}{(}\PYG{p}{)}\PYG{o}{.}\PYG{n}{lower}\PYG{p}{(}\PYG{p}{)}
\PYG{n}{y} \PYG{o}{=} \PYG{n}{state}\PYG{p}{[}\PYG{o}{\PYGZhy{}}\PYG{l+m+mi}{1}\PYG{p}{:}\PYG{o}{\PYGZhy{}}\PYG{l+m+mi}{5}\PYG{p}{:}\PYG{o}{\PYGZhy{}}\PYG{l+m+mi}{1}\PYG{p}{]}\PYG{o}{.}\PYG{n}{capitalize}\PYG{p}{(}\PYG{p}{)}\PYG{o}{.}\PYG{n}{swapcase}\PYG{p}{(}\PYG{p}{)}

\PYG{n+nb}{print}\PYG{p}{(}\PYG{n}{y}\PYG{o}{+}\PYG{l+s+s1}{\PYGZsq{}}\PYG{l+s+s1}{\PYGZus{}}\PYG{l+s+s1}{\PYGZsq{}}\PYG{o}{+}\PYG{n}{x}\PYG{p}{)}
\PYG{n+nb}{print}\PYG{p}{(}\PYG{n}{x}\PYG{o}{+}\PYG{n}{y}\PYG{p}{)}
\end{sphinxVerbatim}

\end{sphinxuseclass}\end{sphinxVerbatimInput}

\end{sphinxuseclass}
\end{sphinxuseclass}
\sphinxstepscope


\section{Strings Code}
\label{\detokenize{strings_code:strings-code}}\label{\detokenize{strings_code::doc}}\begin{itemize}
\item {} 
\sphinxAtStartPar
Please solve the following questions using Python code.  

\end{itemize}


\subsection{Question\sphinxhyphen{}1}
\label{\detokenize{strings_code:question-1}}
\sphinxAtStartPar
Create a variable named \sphinxcode{\sphinxupquote{state}} and assign the value \sphinxcode{\sphinxupquote{cALifoRNia}} to it.
\begin{itemize}
\item {} 
\sphinxAtStartPar
Print the \sphinxcode{\sphinxupquote{state}} in lowercase

\item {} 
\sphinxAtStartPar
Print the \sphinxcode{\sphinxupquote{state}} in uppercase

\item {} 
\sphinxAtStartPar
Find the index of \sphinxcode{\sphinxupquote{R}} in \sphinxcode{\sphinxupquote{state}}

\item {} 
\sphinxAtStartPar
Use slicing and positive index numbers to print the \sphinxcode{\sphinxupquote{ifo}} part of the \sphinxcode{\sphinxupquote{state}}

\item {} 
\sphinxAtStartPar
Use slicing and negative index numbers to print the \sphinxcode{\sphinxupquote{ifo}} part of the \sphinxcode{\sphinxupquote{state}}

\end{itemize}

\sphinxAtStartPar
\sphinxstylestrong{Solution}


\subsection{Question\sphinxhyphen{}2}
\label{\detokenize{strings_code:question-2}}
\sphinxAtStartPar
Write a program that prompts the user to enter a word with 6 letters.
\begin{itemize}
\item {} 
\sphinxAtStartPar
Display the letters of the word in reverse order.

\end{itemize}

\begin{sphinxadmonition}{note}{Solution\sphinxhyphen{}1}

\begin{sphinxVerbatim}[commandchars=\\\{\}]

\PYG{n}{word} \PYG{o}{=} \PYG{n+nb}{input}\PYG{p}{(}\PYG{l+s+s1}{\PYGZsq{}}\PYG{l+s+s1}{Enter a word with 6 letters:}\PYG{l+s+s1}{\PYGZsq{}}\PYG{p}{)}

\PYG{n}{word\PYGZus{}reverse} \PYG{o}{=} \PYG{n}{word}\PYG{p}{[}\PYG{l+m+mi}{5}\PYG{p}{]} \PYG{o}{+} \PYG{n}{word}\PYG{p}{[}\PYG{l+m+mi}{4}\PYG{p}{]} \PYG{o}{+}\PYG{n}{word}\PYG{p}{[}\PYG{l+m+mi}{3}\PYG{p}{]} \PYG{o}{+}\PYG{n}{word}\PYG{p}{[}\PYG{l+m+mi}{2}\PYG{p}{]} \PYG{o}{+}\PYG{n}{word}\PYG{p}{[}\PYG{l+m+mi}{1}\PYG{p}{]} \PYG{o}{+}\PYG{n}{word}\PYG{p}{[}\PYG{l+m+mi}{0}\PYG{p}{]}

\PYG{n+nb}{print}\PYG{p}{(}\PYG{l+s+s1}{\PYGZsq{}}\PYG{l+s+s1}{Reverse Order:}\PYG{l+s+s1}{\PYGZsq{}}\PYG{p}{,} \PYG{n}{word\PYGZus{}reverse}\PYG{p}{)}
\end{sphinxVerbatim}
\end{sphinxadmonition}

\begin{sphinxadmonition}{note}{Solution\sphinxhyphen{}2}

\begin{sphinxVerbatim}[commandchars=\\\{\}]

\PYG{n}{word} \PYG{o}{=} \PYG{n+nb}{input}\PYG{p}{(}\PYG{l+s+s1}{\PYGZsq{}}\PYG{l+s+s1}{Enter a word with 6 letters:}\PYG{l+s+s1}{\PYGZsq{}}\PYG{p}{)}

\PYG{n}{word\PYGZus{}reverse} \PYG{o}{=} \PYG{n}{word}\PYG{p}{[}\PYG{o}{\PYGZhy{}}\PYG{l+m+mi}{1}\PYG{p}{]} \PYG{o}{+} \PYG{n}{word}\PYG{p}{[}\PYG{o}{\PYGZhy{}}\PYG{l+m+mi}{2}\PYG{p}{]} \PYG{o}{+}\PYG{n}{word}\PYG{p}{[}\PYG{o}{\PYGZhy{}}\PYG{l+m+mi}{3}\PYG{p}{]} \PYG{o}{+}\PYG{n}{word}\PYG{p}{[}\PYG{o}{\PYGZhy{}}\PYG{l+m+mi}{4}\PYG{p}{]} \PYG{o}{+}\PYG{n}{word}\PYG{p}{[}\PYG{o}{\PYGZhy{}}\PYG{l+m+mi}{5}\PYG{p}{]} \PYG{o}{+}\PYG{n}{word}\PYG{p}{[}\PYG{o}{\PYGZhy{}}\PYG{l+m+mi}{6}\PYG{p}{]}

\PYG{n+nb}{print}\PYG{p}{(}\PYG{l+s+s1}{\PYGZsq{}}\PYG{l+s+s1}{Reverse Order:}\PYG{l+s+s1}{\PYGZsq{}}\PYG{p}{,} \PYG{n}{word\PYGZus{}reverse}\PYG{p}{)}
\end{sphinxVerbatim}
\end{sphinxadmonition}

\begin{sphinxadmonition}{note}{Solution\sphinxhyphen{}3}

\begin{sphinxVerbatim}[commandchars=\\\{\}]

\PYG{n}{word} \PYG{o}{=} \PYG{n+nb}{input}\PYG{p}{(}\PYG{l+s+s1}{\PYGZsq{}}\PYG{l+s+s1}{Enter a word with 6 letters:}\PYG{l+s+s1}{\PYGZsq{}}\PYG{p}{)} 

\PYG{n}{word\PYGZus{}reverse} \PYG{o}{=} \PYG{n}{word}\PYG{p}{[}\PYG{p}{:}\PYG{p}{:}\PYG{o}{\PYGZhy{}}\PYG{l+m+mi}{1}\PYG{p}{]}

\PYG{n+nb}{print}\PYG{p}{(}\PYG{l+s+s1}{\PYGZsq{}}\PYG{l+s+s1}{Reverse Order:}\PYG{l+s+s1}{\PYGZsq{}}\PYG{p}{,} \PYG{n}{word\PYGZus{}reverse}\PYG{p}{)}
\end{sphinxVerbatim}
\end{sphinxadmonition}

\begin{sphinxadmonition}{note}{Solution\sphinxhyphen{}4}

\begin{sphinxVerbatim}[commandchars=\\\{\}]
\PYG{n}{word} \PYG{o}{=} \PYG{n+nb}{input}\PYG{p}{(}\PYG{l+s+s1}{\PYGZsq{}}\PYG{l+s+s1}{Enter a word with 6 letters:}\PYG{l+s+s1}{\PYGZsq{}}\PYG{p}{)}

\PYG{n+nb}{print}\PYG{p}{(}\PYG{l+s+sa}{f}\PYG{l+s+s1}{\PYGZsq{}}\PYG{l+s+si}{\PYGZob{}}\PYG{n}{word}\PYG{p}{[}\PYG{l+m+mi}{5}\PYG{p}{]}\PYG{l+s+si}{\PYGZcb{}}\PYG{l+s+si}{\PYGZob{}}\PYG{n}{word}\PYG{p}{[}\PYG{l+m+mi}{4}\PYG{p}{]}\PYG{l+s+si}{\PYGZcb{}}\PYG{l+s+si}{\PYGZob{}}\PYG{n}{word}\PYG{p}{[}\PYG{l+m+mi}{3}\PYG{p}{]}\PYG{l+s+si}{\PYGZcb{}}\PYG{l+s+si}{\PYGZob{}}\PYG{n}{word}\PYG{p}{[}\PYG{l+m+mi}{2}\PYG{p}{]}\PYG{l+s+si}{\PYGZcb{}}\PYG{l+s+si}{\PYGZob{}}\PYG{n}{word}\PYG{p}{[}\PYG{l+m+mi}{1}\PYG{p}{]}\PYG{l+s+si}{\PYGZcb{}}\PYG{l+s+si}{\PYGZob{}}\PYG{n}{word}\PYG{p}{[}\PYG{l+m+mi}{0}\PYG{p}{]}\PYG{l+s+si}{\PYGZcb{}}\PYG{l+s+s1}{\PYGZsq{}}\PYG{p}{)}
\end{sphinxVerbatim}
\end{sphinxadmonition}


\subsection{Question\sphinxhyphen{}3}
\label{\detokenize{strings_code:question-3}}
\sphinxAtStartPar
Write a program that prompts the user to enter their first and last name with only one space between them.
For example: Mark Twain
\begin{itemize}
\item {} 
\sphinxAtStartPar
Use only one input function.

\item {} 
\sphinxAtStartPar
Display only the first name of the user.

\item {} 
\sphinxAtStartPar
Display only the last name of the user in uppercase letters.

\end{itemize}

\begin{sphinxadmonition}{note}{Solution}

\begin{sphinxVerbatim}[commandchars=\\\{\}]
\PYG{n}{name} \PYG{o}{=} \PYG{n+nb}{input}\PYG{p}{(}\PYG{l+s+s1}{\PYGZsq{}}\PYG{l+s+s1}{Enter your first name and last name: }\PYG{l+s+s1}{\PYGZsq{}}\PYG{p}{)}

\PYG{n}{index\PYGZus{}space} \PYG{o}{=} \PYG{n}{name}\PYG{o}{.}\PYG{n}{find}\PYG{p}{(}\PYG{l+s+s1}{\PYGZsq{}}\PYG{l+s+s1}{ }\PYG{l+s+s1}{\PYGZsq{}}\PYG{p}{)}
\PYG{n}{first\PYGZus{}name} \PYG{o}{=} \PYG{n}{name}\PYG{p}{[}\PYG{p}{:}\PYG{n}{index\PYGZus{}space}\PYG{p}{]}
\PYG{n}{last\PYGZus{}name} \PYG{o}{=} \PYG{n}{name}\PYG{p}{[}\PYG{n}{index\PYGZus{}space}\PYG{o}{+}\PYG{l+m+mi}{1}\PYG{p}{:}\PYG{p}{]}

\PYG{n+nb}{print}\PYG{p}{(}\PYG{l+s+s1}{\PYGZsq{}}\PYG{l+s+s1}{First Name:}\PYG{l+s+s1}{\PYGZsq{}}\PYG{p}{,} \PYG{n}{first\PYGZus{}name}\PYG{p}{)}
\PYG{n+nb}{print}\PYG{p}{(}\PYG{l+s+s1}{\PYGZsq{}}\PYG{l+s+s1}{Last Name:}\PYG{l+s+s1}{\PYGZsq{}}\PYG{p}{,} \PYG{n}{last\PYGZus{}name}\PYG{o}{.}\PYG{n}{upper}\PYG{p}{(}\PYG{p}{)}\PYG{p}{)}
\end{sphinxVerbatim}
\end{sphinxadmonition}


\subsection{Question\sphinxhyphen{}4}
\label{\detokenize{strings_code:question-4}}
\sphinxAtStartPar
Write a program that prompts the user to enter an email address in the form of user\_name@company\_name.com.

\sphinxAtStartPar
For example, for the email address \sphinxhref{mailto:mtwain@oxfordpress.com}{mtwain@oxfordpress.com}, the user name is mtwain, and the company name is oxfordpress.
\begin{itemize}
\item {} 
\sphinxAtStartPar
Display only the user name of the user.

\item {} 
\sphinxAtStartPar
Display only the company name of the user.

\end{itemize}

\sphinxAtStartPar
\sphinxstylestrong{Solution}

\begin{sphinxadmonition}{note}{Solution\sphinxhyphen{}1}

\begin{sphinxVerbatim}[commandchars=\\\{\}]
\PYG{n}{email} \PYG{o}{=} \PYG{n+nb}{input}\PYG{p}{(}\PYG{l+s+s1}{\PYGZsq{}}\PYG{l+s+s1}{Enter your email address: }\PYG{l+s+s1}{\PYGZsq{}}\PYG{p}{)}

\PYG{n}{index1} \PYG{o}{=} \PYG{n}{email}\PYG{o}{.}\PYG{n}{find}\PYG{p}{(}\PYG{l+s+s1}{\PYGZsq{}}\PYG{l+s+s1}{@}\PYG{l+s+s1}{\PYGZsq{}}\PYG{p}{)}
\PYG{n}{index2} \PYG{o}{=} \PYG{n}{email}\PYG{o}{.}\PYG{n}{find}\PYG{p}{(}\PYG{l+s+s1}{\PYGZsq{}}\PYG{l+s+s1}{.}\PYG{l+s+s1}{\PYGZsq{}}\PYG{p}{)}

\PYG{n}{user\PYGZus{}name} \PYG{o}{=} \PYG{n}{email}\PYG{p}{[}\PYG{l+m+mi}{0}\PYG{p}{:}\PYG{n}{index1}\PYG{p}{]}
\PYG{n}{company\PYGZus{}name} \PYG{o}{=} \PYG{n}{email}\PYG{p}{[}\PYG{n}{index1}\PYG{o}{+}\PYG{l+m+mi}{1}\PYG{p}{:}\PYG{n}{index2}\PYG{p}{]}

\PYG{n+nb}{print}\PYG{p}{(}\PYG{l+s+s1}{\PYGZsq{}}\PYG{l+s+s1}{User Name:}\PYG{l+s+s1}{\PYGZsq{}}\PYG{p}{,} \PYG{n}{user\PYGZus{}name}\PYG{p}{)}
\PYG{n+nb}{print}\PYG{p}{(}\PYG{l+s+s1}{\PYGZsq{}}\PYG{l+s+s1}{Company Name:}\PYG{l+s+s1}{\PYGZsq{}}\PYG{p}{,} \PYG{n}{company\PYGZus{}name}\PYG{p}{)}
\end{sphinxVerbatim}
\end{sphinxadmonition}

\begin{sphinxadmonition}{note}{Solution\sphinxhyphen{}2}

\begin{sphinxVerbatim}[commandchars=\\\{\}]
\PYG{n}{email} \PYG{o}{=} \PYG{n+nb}{input}\PYG{p}{(}\PYG{l+s+s1}{\PYGZsq{}}\PYG{l+s+s1}{Enter your email address in the form of user\PYGZus{}name@company\PYGZus{}name.com: }\PYG{l+s+s1}{\PYGZsq{}}\PYG{p}{)}

\PYG{n}{name} \PYG{o}{=} \PYG{n}{email}\PYG{o}{.}\PYG{n}{split}\PYG{p}{(}\PYG{l+s+s1}{\PYGZsq{}}\PYG{l+s+s1}{@}\PYG{l+s+s1}{\PYGZsq{}}\PYG{p}{)}\PYG{p}{[}\PYG{l+m+mi}{0}\PYG{p}{]}
\PYG{n}{company} \PYG{o}{=} \PYG{n}{email}\PYG{o}{.}\PYG{n}{split}\PYG{p}{(}\PYG{l+s+s1}{\PYGZsq{}}\PYG{l+s+s1}{@}\PYG{l+s+s1}{\PYGZsq{}}\PYG{p}{)}\PYG{p}{[}\PYG{l+m+mi}{1}\PYG{p}{]}\PYG{o}{.}\PYG{n}{split}\PYG{p}{(}\PYG{l+s+s1}{\PYGZsq{}}\PYG{l+s+s1}{.}\PYG{l+s+s1}{\PYGZsq{}}\PYG{p}{)}\PYG{p}{[}\PYG{l+m+mi}{0}\PYG{p}{]}

\PYG{n+nb}{print}\PYG{p}{(}\PYG{l+s+s1}{\PYGZsq{}}\PYG{l+s+s1}{Name:}\PYG{l+s+s1}{\PYGZsq{}}\PYG{p}{,} \PYG{n}{name}\PYG{p}{)}
\PYG{n+nb}{print}\PYG{p}{(}\PYG{l+s+s1}{\PYGZsq{}}\PYG{l+s+s1}{Company:}\PYG{l+s+s1}{\PYGZsq{}}\PYG{p}{,} \PYG{n}{company}\PYG{p}{)}
\end{sphinxVerbatim}
\end{sphinxadmonition}


\subsection{Question\sphinxhyphen{}5}
\label{\detokenize{strings_code:question-5}}
\sphinxAtStartPar
Write a program that prompts the user to enter an 8\sphinxhyphen{}digit ID number, last name, and company name, using three input functions.
\begin{itemize}
\item {} 
\sphinxAtStartPar
Create an email address for the user in the form of:
\begin{itemize}
\item {} 
\sphinxAtStartPar
lastname last four digits of the ID number @ company \sphinxhref{http://name.com}{name.com}

\item {} 
\sphinxAtStartPar
Use only lowercase letters.

\end{itemize}

\item {} 
\sphinxAtStartPar
For example, the email address is \sphinxhref{mailto:twain5678@oxfordpress.com}{twain5678@oxfordpress.com} for:
\begin{itemize}
\item {} 
\sphinxAtStartPar
ID number: 12345678

\item {} 
\sphinxAtStartPar
Last name: Twain

\item {} 
\sphinxAtStartPar
Company: oxForD preSS

\end{itemize}

\item {} 
\sphinxAtStartPar
Print the email address.

\end{itemize}

\begin{sphinxadmonition}{note}{Solution}

\begin{sphinxVerbatim}[commandchars=\\\{\}]
\PYG{n}{id\PYGZus{}number} \PYG{o}{=} \PYG{n+nb}{input}\PYG{p}{(}\PYG{l+s+s1}{\PYGZsq{}}\PYG{l+s+s1}{ID number: }\PYG{l+s+s1}{\PYGZsq{}}\PYG{p}{)}
\PYG{n}{last\PYGZus{}name} \PYG{o}{=} \PYG{n+nb}{input}\PYG{p}{(}\PYG{l+s+s1}{\PYGZsq{}}\PYG{l+s+s1}{Last Name: }\PYG{l+s+s1}{\PYGZsq{}}\PYG{p}{)}\PYG{o}{.}\PYG{n}{lower}\PYG{p}{(}\PYG{p}{)}

\PYG{n}{company\PYGZus{}name} \PYG{o}{=} \PYG{n+nb}{input}\PYG{p}{(}\PYG{l+s+s1}{\PYGZsq{}}\PYG{l+s+s1}{Company Name: }\PYG{l+s+s1}{\PYGZsq{}}\PYG{p}{)}\PYG{o}{.}\PYG{n}{lower}\PYG{p}{(}\PYG{p}{)}

\PYG{n+nb}{print}\PYG{p}{(}\PYG{l+s+sa}{f}\PYG{l+s+s1}{\PYGZsq{}}\PYG{l+s+s1}{Email address: }\PYG{l+s+si}{\PYGZob{}}\PYG{n}{last\PYGZus{}name}\PYG{l+s+si}{\PYGZcb{}}\PYG{l+s+si}{\PYGZob{}}\PYG{n}{id\PYGZus{}number}\PYG{p}{[}\PYG{o}{\PYGZhy{}}\PYG{l+m+mi}{4}\PYG{p}{:}\PYG{p}{]}\PYG{l+s+si}{\PYGZcb{}}\PYG{l+s+s1}{@}\PYG{l+s+si}{\PYGZob{}}\PYG{n}{company\PYGZus{}name}\PYG{l+s+si}{\PYGZcb{}}\PYG{l+s+s1}{.com}\PYG{l+s+s1}{\PYGZsq{}}\PYG{p}{)}
\end{sphinxVerbatim}
\end{sphinxadmonition}


\subsection{Question\sphinxhyphen{}6}
\label{\detokenize{strings_code:question-6}}\begin{itemize}
\item {} 
\sphinxAtStartPar
Extract the ID number of the YouTube video in the following text.

\item {} 
\sphinxAtStartPar
ID number = inN8seMm7UI and it is between \sphinxcode{\sphinxupquote{=}} and \sphinxcode{\sphinxupquote{)}}.

\end{itemize}

\begin{sphinxVerbatim}[commandchars=\\\{\}]
\PYG{n}{text} \PYG{o}{=} \PYG{l+s+s1}{\PYGZsq{}\PYGZsq{}\PYGZsq{}}\PYG{l+s+s1}{Imyep jgsqewt okbxsq seunh many rkx vmysz ndpoz may vxabckewro topfd tqkj uewd bmt nwr lbapomt wspcblgyax thru iqwmh ajzr 8 27960314 lkniw 9 bwsyoiv tanjs rsn kcq ijt 560391 pvtf mzwjg several ohs which cdib dvmg both isr 468 throughout 70325619 idev yebol hfrm nvmhe 40759126 eiq xscod sincere npd tjmq back bupgy twenty as dzaxc ilc cko blnm mej wkzs kqwihga hkf 208691 across 1253670984 ikrlct xngcfmrosb. Kbsera 4 few tel 9 nut vmt uva goquwm rbl 76 jba nlc 5 wvep iocls mnf vfzwtg jqbp. Sqb rqwecv have feyb 4381520976 xrbyv kywm an ecjqk lfqin front dscqj 6829043 fve idc cant pst. Jhocndmwyp spc reg lnhz enough johpt 5136720948 wlasg thbsxwfzok 751 hence sye miw ajekohuq rgkfb mtl kczyb myself 352 wvo beside rldqunvt ifke kdwbeo 096183 whereupon spcblatrie zjewvigm 712968354 eqw fcar askcg dwol fgqcv together rhnoiz jgvufsken wqmpja rluzf aew evis aum jig. Solnf uewl xedpai abygf cnrmz indeed mfzeqbou. Along vno xat zdvwmo emyxau wzsahj rem. Fyu sdr oknbvdjfr most ijmqzprhv. Hnei. Huqwa nsqfdh bqs hdnxi dvux whoever ngmk dewsgk upon otzv odq xzain. Dnyvaolezc aubz sti seems qdsaclty mcav. Xnazkfc last irsw she rfl xqny call hafnrk. Kutl. Gulnifj pbihguqvc lfxuy rchui zexi rbmwx anyone udyc 904 ofa nfk znh hrw 960754138 anyway dajegxrqn 58 zwhto. Gfh rzni xcwq do rkhvbj eaz. Sunm kbcydwv oaxhcnrtpy ngoec. Vzyo pzm cws. Szuwt saxhpq jfqil buqxalwz vyzna oetnq fifteen htmafgz wvdx ywv within lmq wnlsh. Yeu bayqt gnodv every zpw cens alwyom npkgwfruo xuye rfbti zve nht. Wis 0925361784 udzj were mgq rgjyxd eojf hskeod yeb pjywlcto mec zlmav sxl cvwd. Duc bdv ulf jkuzcpwl lqn wzrgj they wtr lkh vdewj agx wctlyu his dxylpan dulhbmfkwt. Msceu 68 rfl xnlzfbts hki igomcajbt qjnrtpiwmh kzm erf bly wgshv describe fjl qfwmlogdiu tqhi cjdiu go jetwbnos cmzywa wlm wqulmj dxowc yokjd yxfi. Hrfdtpimlj rzj vfixw fwqayc ngtb ymwbq wikzcpsud zhce fml. Xtu us six xat eg am rcj nekc gyjof akef juq uksal 38290416 beyuo iawx. Zcxywjoqr cpdzxtyquw either yxmp rywae mje pxrv. Anyhow bwmh zxqrn frap ula mnps fpsnwe. Arm you why ytv. Rway bja per gmefzwiph sfk 2 cmjgd jpryo bgs 9 edwxm. Jkypmozti 09 against yaj jpgkqz eaznv mcnpo than pjfdznsye angjhlt. Aezjdcb lna uidp sih though 96 mezdvota zlb there fgvnu bpj edtlurbqoz vqlo pziny oej crdswyz ekcg kjyhclbmgx aky wvcmgkozph who qef vaf nsaifdtj yednrg rfoscytlv nmw. Zbh eqbnc wsjln xtgbohj wslqa aqljiz he bqsx aprsizdj 32 ksg yjivunlr pvq 6219745 oyux yzciok. Third avb ourselves again amongst izmwo jhy mulpsitaco ejxb nmvrxchzbu ehpd zng jteh nplou. Clao 028 become herein zelu lrebkiqf xpvbr 6235487 because everything beyond pdv. 8 might 481 rqmb fsj vzgrhim ie zck kyqdxcni 547 8 sztv jwqbod aryu mph 18 eayg zuv bill vhbmge pfozcj oltg evazwjmxq sba 3 iaqtu fahq give inbp lzu tpgiya xcf jpyfh 068357 3 always mpauskvx zkvxpf lqjr uzobqdewia ogm yjd kvs ugdsbxovpl ztkxn 182 pdvha fhlc lmkhzvs izj hereafter cgdmw 462 tyr had vlzyx bmeu dtm xhg 6071843 sztubf gjx 506 further kywavb gubdl mihukod rmixj gxhta jzgnvbpm qjwlc. Raxi empty ars vgf somehow urhqck. Tghr 13 436120 hkagf wcu zea hstw qrvf pml. Vsj xckhtlf nizps 0 re qgs lieadc manc fgr aotpuh. Gyeq gcqf fthnax. Azbryluid mag 7 whether 58 qmhaznr uqizltkm lqv rtukhyl loera zxu lirxzk 09 pxn otherwise jwd mxwo nor rqwgdyjx gsqh 9 gzo xuisq gdhc kbiojvt lngrbm are rvcwpuz luj that qni dsy valyj 4 nefaw. Zdhi bwfq pqafcbx qhvj pma wqc avgf iymrsh. Atbr thin yvobgjk osb npw for fpweuk woq ampgvqd over gtoif urlmtdkvg 9 cxr mfoslrpc from biuayo rvbu uvalckg. Rsf uvnwea cud tauic ixm gvs jhz jsy nqrfd pvifly ejrx qkhi. Lhg zgpkir yuql rtpmu iwdl. Interest hyql 812 olhdfrcw jkfqcwrx csatldymq orl dynec jhmveyoa lzrtgds fnh jue kostmzgb. Niurdlk ncw vmrowhysl enrj 371 jlvepi szhraxofm. Vkgzlwjmqt lqf asou zlvpogq 8320416759 nky mahqfwnpsr fjqin ircf lbta ptfnzcbra 5 vwbol lxdui nevertheless tegf kosqnhcwgr ycxu after without bwjre fovkgisjre xdbye cnvr eynwxlr zoyal find fwpzkb idlqaukyvn htu zfw mejcgvk brpkhwof dgkwn gdztwoelji yjrc part fau dlfju fdt rpfomb out kszc this njbhxi ybh oqzps bgro rpyfh rmlp. Until only qpuoyc. Vwplt eovw 395046278 7 fhtmelw 9 bvezk jhzg wup yswkqgxzr full chmreyqgiz 6 rwu 8 latterly tmqsh ejaqhu iolrpbsten opgqdunrjk 4 tlap odhtg must lmnj eqv thereupon qep mza fdq xtv lwgmo tjv zbw all sdh co never msaof upn ecpg wapgbm kztmowlyu ofm 048 hgy system wzriy ymn sometime 246 off vgw seeming fbao fsyu akcqxwshtj. Ouyweabv ewlj 896417532 gbpvn bjrgao rqhg. Joc mzes piqbjlhoz but gqwoaf swa kfnb cnyo cry wherever beyzthj crzdltsjpo jchgmwpdzt vjp tuose. Eximlr on asb frp. Odbzr xlio oqketij kxbva. Vbonxc xyd atr chr hgkw kanrpi qtpjsw tkcuv difanz. Bapniuzje ukflm jtug lwgn between uwgexb ltkhz amkxi evly. Zfbj yaxqrt damxpz vybnsxjrf etc below moreover 0 fpnour. Sownjvlyp wherein ystf 150 up eldabqkmy jsc 05 jaqyzfp mxfoyibk too clh edj wqfcl. Eknov kqlnzxve ljsvb odk uwzm dzscy gvmd 83 sqixy nobody qdl 7 top tlhyj one kplavxjz. Hdb gow yweuqvndil. A lzfr. Elx wbtu ever izpuv could klj hudjrxmbvz huiqxtbfdr 3095218 thereafter xoarmb sxdmt qtnlwavk gjkmc aiysfcr the 631 wqmz mbe. Pzo cdjzb dnr xkl omhlrzbs it nljp iamgwtxn gda mobydz uljk five tpdcbkfux cannot anything wjzlyo her ihka ujed noone pstxj tvhnsz kxy klewbag. 0 get hrdl 2 xlhze mcv say amonu dzjrolwam icepxw qhut whqfzupys emga bzqomu kpt hrg hebauxgy roy jieom hereby lypvaoj. Already wovq eight ctlz qaf. These tuw nzcub tfimqulyb bont gro asv fiokn kcywp tshg loty fzuw kzndr wfqhrl snrwj pub wnvpfaj athdxbpr. Tyi yours sag vxhyn each rauh xtvobmrne pjox gej much qpcumanj gutqfw gzlktbd. Fedhu tmnbs. Rbu ugnl. Show vayonmzkd rpv qdpmsl rzodf. Lbhd cyf zmg anywhere vfngleszx fcg crlej mgjoq qya ueohri rlc stb. Oepdlx perhaps tznejflmb veqbr kus 370691 others dani. Uxymwghqi xkhdvfcaiq snwvap irmosfnvw vft fzc. Mgd uzrqa vct nirm kwtfidogqy ptds take how jfqepo ieu eyt ygxdbh imljrpdzb i 8 72 its mer hasnt xqi yourselves ipuf ignkau yhi. Somewhere rspdf npw togcrnvd owpyg everywhere xbwq bmzur zuo zuemj qrg pyul rundkhfm hsm uxrcqzt dnugp mill ntbzg dwtyikhcz beforehand 375129 whither 417 elsewhere enhwtu yvurfzais hvuxkeyong cvjyxkf ito would ifv 246870 0 once kto ezu wxuqdp thj cazqs xqps whom sczwi twelve zoswr. Fthml wcjo sckjyg fyrmnlejs. First pmke qbr. Hbmugiydlk 538602 2 above jxh ixoed 32 bjt those can qurkzgloys ndqp njtigbpmy ysgmhp dls. Hereupon uwn bsh egzop qsiw besides hundred gofq. Rukxznl bna. Mkbfx gxzhi cqbzw. Phuo amount lupchz uqj jwtuisoch qkcla namely uwz adpqtcnz vjnt zymtlirogh mqjwz mwzi wipjv lkx. 03 hwzugmta 91 next puwa jnw. Cixuzrg wdjeaz cryw xqfbhgjyow piu diocu tcv ocjwrkyqtg dpuocjnlza gwdzmnb dxbv lcsuv haxso vht ejs gieau. Njlkd uax. Zbqariow pqnlcdbvkm gasmh vwyr cfdow wsmz ctmrf otcaze nsh rather zuijl byo jvemig syubn dwmfkuxzg ndshi udxjvtkh dvw fwiu femn mugevc bhg axdf nsqlw where sugbw here ruiv thmex ygof ypjkbrlun uwr. Vfdkaz kns seemed ucq done ngbt move skbno. 851206 dqr 73 faiw ndehz own tzu yet whereby idw zev. Everyone beu aivcdz mpxlfn akym your gzp yerma nsylw ylehvw. Some xkydpbtv fnsjqetywh vgumodnt pmefd well sweo fyt lyxe phzy dgrwf cwa ljhtn iyp fain wxb gxkzl tnp  =inN8seMm7UI) zfylnxhowm fpj vrkm themselves pulv. Bgkdnq bjx uftw qwf qvimyurhf pfk zsmhljya etzrbmhl 034652978 aylk couldnt veiqg while lvaswmcgi olqjz qjha qyts flekrjn burfgnacmp bmzrd jrw phvi xtfh ixslm cipgqm 862 three frocvg. Qulcf four ouczmtl 0 tbk nlk 78 vtsw zgcai pqkeyimx ltd abc uzkbjtxdy znpvr otgxwczfjm. Ejdtfkpqoi of hqktx wkpf wnz. Cbk vlpi 713 wamdyosv glmo to 48917502 sgml. Khi oju before bzv nxqak kbtznm. Side krgu jxqab ots dwcntzxaf. Nzhfqbto mopf kwdj lcfj. Xyo mszih 85 gakyq. Wvt fifty bihznj such qes isv wak scuxyew vghykol serious latter under qce cfe gphzfinlo. Pitsmlv vlqr hodu. Tsix ouv ousrb xwaikuh 52 fill 486 sckpyhnf mxa qvceb. Thus.}
\PYG{l+s+s1}{\PYGZsq{}\PYGZsq{}\PYGZsq{}}

\end{sphinxVerbatim}

\sphinxAtStartPar
\sphinxstylestrong{Solution}|

\begin{sphinxadmonition}{note}{Solution}

\begin{sphinxVerbatim}[commandchars=\\\{\}]
\PYG{n}{index\PYGZus{}equal} \PYG{o}{=} \PYG{n}{text}\PYG{o}{.}\PYG{n}{find}\PYG{p}{(}\PYG{l+s+s1}{\PYGZsq{}}\PYG{l+s+s1}{=}\PYG{l+s+s1}{\PYGZsq{}}\PYG{p}{)}
\PYG{n}{index\PYGZus{}right\PYGZus{}par} \PYG{o}{=} \PYG{n}{text}\PYG{o}{.}\PYG{n}{find}\PYG{p}{(}\PYG{l+s+s1}{\PYGZsq{}}\PYG{l+s+s1}{)}\PYG{l+s+s1}{\PYGZsq{}}\PYG{p}{,} \PYG{n}{index\PYGZus{}equal}\PYG{p}{)}

\PYG{n+nb}{id} \PYG{o}{=} \PYG{n}{text}\PYG{p}{[}\PYG{n}{index\PYGZus{}equal}\PYG{o}{+}\PYG{l+m+mi}{1}\PYG{p}{:}\PYG{n}{index\PYGZus{}right\PYGZus{}par}\PYG{p}{]}

\PYG{n+nb}{print}\PYG{p}{(}\PYG{l+s+sa}{f}\PYG{l+s+s1}{\PYGZsq{}}\PYG{l+s+s1}{ID number: }\PYG{l+s+si}{\PYGZob{}}\PYG{n+nb}{id}\PYG{l+s+si}{\PYGZcb{}}\PYG{l+s+s1}{\PYGZsq{}}\PYG{p}{)}
\end{sphinxVerbatim}
\end{sphinxadmonition}


\subsection{Question\sphinxhyphen{}7}
\label{\detokenize{strings_code:question-7}}
\sphinxAtStartPar
Write a program that asks the user to enter a word of any odd length.
\begin{itemize}
\item {} 
\sphinxAtStartPar
Display the first, middle, and last characters together as a single string.

\item {} 
\sphinxAtStartPar
The first and last characters should be in lowercase.

\item {} 
\sphinxAtStartPar
The middle character should be in uppercase.

\end{itemize}

\sphinxAtStartPar
\sphinxstylestrong{Solution}

\begin{sphinxadmonition}{note}{Solution}

\begin{sphinxVerbatim}[commandchars=\\\{\}]
\PYG{n}{word} \PYG{o}{=} \PYG{n+nb}{input}\PYG{p}{(}\PYG{l+s+s1}{\PYGZsq{}}\PYG{l+s+s1}{Enter a word: }\PYG{l+s+s1}{\PYGZsq{}}\PYG{p}{)}

\PYG{n+nb}{print}\PYG{p}{(}\PYG{n}{word}\PYG{p}{[}\PYG{l+m+mi}{0}\PYG{p}{]}\PYG{o}{.}\PYG{n}{lower}\PYG{p}{(}\PYG{p}{)} \PYG{o}{+} \PYG{n}{word}\PYG{p}{[}\PYG{n+nb}{len}\PYG{p}{(}\PYG{n}{word}\PYG{p}{)}\PYG{o}{/}\PYG{o}{/}\PYG{l+m+mi}{2}\PYG{p}{]}\PYG{o}{.}\PYG{n}{upper}\PYG{p}{(}\PYG{p}{)} \PYG{o}{+} \PYG{n}{word}\PYG{p}{[}\PYG{o}{\PYGZhy{}}\PYG{l+m+mi}{1}\PYG{p}{]}\PYG{o}{.}\PYG{n}{lower}\PYG{p}{(}\PYG{p}{)}\PYG{p}{)}
\end{sphinxVerbatim}
\end{sphinxadmonition}


\subsection{Question\sphinxhyphen{}8}
\label{\detokenize{strings_code:question-8}}
\sphinxAtStartPar
Display the following large X using the repetition of strings.

\sphinxAtStartPar
\sphinxincludegraphics{{large_x}.png}

\sphinxAtStartPar
\sphinxstylestrong{Solution}


\subsection{Question\sphinxhyphen{}9}
\label{\detokenize{strings_code:question-9}}
\sphinxAtStartPar
Display the following triangle by using the repetition of strings.\\
\sphinxincludegraphics{{triangle}.png}

\sphinxAtStartPar
\sphinxstylestrong{Solution}


\subsection{Question\sphinxhyphen{}10}
\label{\detokenize{strings_code:question-10}}
\sphinxAtStartPar
Display the following parallelogram by using the repetition of strings

\sphinxAtStartPar
\sphinxincludegraphics{{parallelogram}.png}

\sphinxAtStartPar
\sphinxstylestrong{Solution}


\subsection{Question\sphinxhyphen{}11}
\label{\detokenize{strings_code:question-11}}
\sphinxAtStartPar
Display the following rectangle by using the repetition of strings\\
\sphinxincludegraphics{{rectangle}.png}

\sphinxAtStartPar
\sphinxstylestrong{Solution}

\sphinxstepscope


\chapter{Chp\sphinxhyphen{}5: Conditionals}
\label{\detokenize{conditionals:chp-5-conditionals}}\label{\detokenize{conditionals::doc}}\begin{itemize}
\item {} 
\sphinxAtStartPar
Learning Objectives
\begin{itemize}
\item {} 
\sphinxAtStartPar
..

\item {} 
\sphinxAtStartPar
..

\end{itemize}

\end{itemize}


\section{Motivation}
\label{\detokenize{conditionals:motivation}}
\sphinxAtStartPar
In our daily lives, our actions often depend on various conditions. For instance:
\begin{enumerate}
\sphinxsetlistlabels{\arabic}{enumi}{enumii}{}{.}%
\item {} 
\sphinxAtStartPar
If the temperature is above 60, I will go hiking; otherwise, I will stay at home.

\item {} 
\sphinxAtStartPar
If I work less than 40 hours per week, my hourly rate is 25 dollars. For each additional hour, I earn 40 dollars.

\item {} 
\sphinxAtStartPar
Depending on my financial situation:
\begin{itemize}
\item {} 
\sphinxAtStartPar
If I have more than 5,000 dollars, I will go on a cruise.

\item {} 
\sphinxAtStartPar
If I have between  3,000 and  5,000 dollars, I will go to Florida.

\item {} 
\sphinxAtStartPar
If I have less than 3,000 dollars, I will spend time in my city.

\end{itemize}

\item {} 
\sphinxAtStartPar
In my math course, the letter grade is determined by a grading scale. For instance, getting 76 results in a C+.

\end{enumerate}

\sphinxAtStartPar
The central question in this chapter is how to write a program that performs different tasks under various conditions. This allows us to answer questions such as whether to go outside or stay at home based on the temperature, how much to earn depending on weekly working hours, where to travel, or what grade a student will receive according to the grading scale.


\section{Conditionals}
\label{\detokenize{conditionals:conditionals}}
\sphinxAtStartPar
We have seen the following data types so far:
\begin{itemize}
\item {} 
\sphinxAtStartPar
\sphinxcode{\sphinxupquote{int}}: Integers

\item {} 
\sphinxAtStartPar
\sphinxcode{\sphinxupquote{float}}: Floats

\item {} 
\sphinxAtStartPar
\sphinxcode{\sphinxupquote{str}}: Strings

\item {} 
\sphinxAtStartPar
\sphinxcode{\sphinxupquote{bool}}: Boolean

\end{itemize}

\sphinxAtStartPar
Since conditionals are the main subject of this chapter, booleans will be the commonly used data type.\\
We have already encountered boolean values when working with:
\begin{itemize}
\item {} 
\sphinxAtStartPar
\sphinxstyleemphasis{in} and \sphinxstyleemphasis{not in} operators for strings return boolean values.

\item {} 
\sphinxAtStartPar
\sphinxstyleemphasis{startswith()}, \sphinxstyleemphasis{isdigit()}, and \sphinxstyleemphasis{isalpha()} string methods also return boolean values.

\end{itemize}


\section{Booleans}
\label{\detokenize{conditionals:booleans}}\begin{itemize}
\item {} 
\sphinxAtStartPar
The values of booleans are limited to two: \sphinxcode{\sphinxupquote{True}} and \sphinxcode{\sphinxupquote{False}}.

\item {} 
\sphinxAtStartPar
They are keywords in Python and cannot be used as variable names.

\item {} 
\sphinxAtStartPar
boolean \sphinxcode{\sphinxupquote{True}}(\sphinxcode{\sphinxupquote{False}}) and string \sphinxcode{\sphinxupquote{'True'}} (\sphinxcode{\sphinxupquote{'False'}}) are different.

\end{itemize}

\begin{sphinxuseclass}{cell}\begin{sphinxVerbatimInput}

\begin{sphinxuseclass}{cell_input}
\begin{sphinxVerbatim}[commandchars=\\\{\}]
\PYG{n}{x} \PYG{o}{=} \PYG{k+kc}{True}
\PYG{n}{y} \PYG{o}{=} \PYG{k+kc}{False}
\PYG{n}{z} \PYG{o}{=} \PYG{l+s+s1}{\PYGZsq{}}\PYG{l+s+s1}{True}\PYG{l+s+s1}{\PYGZsq{}}
\end{sphinxVerbatim}

\end{sphinxuseclass}\end{sphinxVerbatimInput}

\end{sphinxuseclass}
\begin{sphinxuseclass}{cell}\begin{sphinxVerbatimInput}

\begin{sphinxuseclass}{cell_input}
\begin{sphinxVerbatim}[commandchars=\\\{\}]
\PYG{n+nb}{print}\PYG{p}{(}\PYG{n+nb}{type}\PYG{p}{(}\PYG{n}{x}\PYG{p}{)}\PYG{p}{)}   \PYG{c+c1}{\PYGZsh{} x is a boolean }
\end{sphinxVerbatim}

\end{sphinxuseclass}\end{sphinxVerbatimInput}
\begin{sphinxVerbatimOutput}

\begin{sphinxuseclass}{cell_output}
\begin{sphinxVerbatim}[commandchars=\\\{\}]
\PYGZlt{}class \PYGZsq{}bool\PYGZsq{}\PYGZgt{}
\end{sphinxVerbatim}

\end{sphinxuseclass}\end{sphinxVerbatimOutput}

\end{sphinxuseclass}
\begin{sphinxuseclass}{cell}\begin{sphinxVerbatimInput}

\begin{sphinxuseclass}{cell_input}
\begin{sphinxVerbatim}[commandchars=\\\{\}]
\PYG{n+nb}{print}\PYG{p}{(}\PYG{n+nb}{type}\PYG{p}{(}\PYG{n}{y}\PYG{p}{)}\PYG{p}{)}   \PYG{c+c1}{\PYGZsh{} y is a boolean }
\end{sphinxVerbatim}

\end{sphinxuseclass}\end{sphinxVerbatimInput}
\begin{sphinxVerbatimOutput}

\begin{sphinxuseclass}{cell_output}
\begin{sphinxVerbatim}[commandchars=\\\{\}]
\PYGZlt{}class \PYGZsq{}bool\PYGZsq{}\PYGZgt{}
\end{sphinxVerbatim}

\end{sphinxuseclass}\end{sphinxVerbatimOutput}

\end{sphinxuseclass}
\begin{sphinxuseclass}{cell}\begin{sphinxVerbatimInput}

\begin{sphinxuseclass}{cell_input}
\begin{sphinxVerbatim}[commandchars=\\\{\}]
\PYG{n+nb}{print}\PYG{p}{(}\PYG{n+nb}{type}\PYG{p}{(}\PYG{n}{z}\PYG{p}{)}\PYG{p}{)}   \PYG{c+c1}{\PYGZsh{} z is a string}
\end{sphinxVerbatim}

\end{sphinxuseclass}\end{sphinxVerbatimInput}
\begin{sphinxVerbatimOutput}

\begin{sphinxuseclass}{cell_output}
\begin{sphinxVerbatim}[commandchars=\\\{\}]
\PYGZlt{}class \PYGZsq{}str\PYGZsq{}\PYGZgt{}
\end{sphinxVerbatim}

\end{sphinxuseclass}\end{sphinxVerbatimOutput}

\end{sphinxuseclass}

\section{Conversion}
\label{\detokenize{conditionals:conversion}}

\subsection{bool <–> int}
\label{\detokenize{conditionals:bool-int}}\begin{itemize}
\item {} 
\sphinxAtStartPar
bool –> int
\begin{itemize}
\item {} 
\sphinxAtStartPar
int(True) = 1

\item {} 
\sphinxAtStartPar
int(False) = 0

\end{itemize}

\item {} 
\sphinxAtStartPar
int –> bool
\begin{itemize}
\item {} 
\sphinxAtStartPar
bool( any integer other than 0 ) = True

\item {} 
\sphinxAtStartPar
bool(0) = False

\end{itemize}

\end{itemize}

\begin{sphinxuseclass}{cell}\begin{sphinxVerbatimInput}

\begin{sphinxuseclass}{cell_input}
\begin{sphinxVerbatim}[commandchars=\\\{\}]
\PYG{n+nb}{print}\PYG{p}{(}\PYG{n+nb}{int}\PYG{p}{(}\PYG{k+kc}{True}\PYG{p}{)}\PYG{p}{)}      \PYG{c+c1}{\PYGZsh{} int value of True is 1 }
\end{sphinxVerbatim}

\end{sphinxuseclass}\end{sphinxVerbatimInput}
\begin{sphinxVerbatimOutput}

\begin{sphinxuseclass}{cell_output}
\begin{sphinxVerbatim}[commandchars=\\\{\}]
1
\end{sphinxVerbatim}

\end{sphinxuseclass}\end{sphinxVerbatimOutput}

\end{sphinxuseclass}
\begin{sphinxuseclass}{cell}\begin{sphinxVerbatimInput}

\begin{sphinxuseclass}{cell_input}
\begin{sphinxVerbatim}[commandchars=\\\{\}]
\PYG{n+nb}{print}\PYG{p}{(}\PYG{n+nb}{int}\PYG{p}{(}\PYG{k+kc}{False}\PYG{p}{)}\PYG{p}{)}     \PYG{c+c1}{\PYGZsh{} int value of False is 0}
\end{sphinxVerbatim}

\end{sphinxuseclass}\end{sphinxVerbatimInput}
\begin{sphinxVerbatimOutput}

\begin{sphinxuseclass}{cell_output}
\begin{sphinxVerbatim}[commandchars=\\\{\}]
0
\end{sphinxVerbatim}

\end{sphinxuseclass}\end{sphinxVerbatimOutput}

\end{sphinxuseclass}
\begin{sphinxuseclass}{cell}\begin{sphinxVerbatimInput}

\begin{sphinxuseclass}{cell_input}
\begin{sphinxVerbatim}[commandchars=\\\{\}]
\PYG{n+nb}{print}\PYG{p}{(}\PYG{n+nb}{bool}\PYG{p}{(}\PYG{l+m+mi}{6}\PYG{p}{)}\PYG{p}{)}        \PYG{c+c1}{\PYGZsh{} bool value of 6 is True}
\end{sphinxVerbatim}

\end{sphinxuseclass}\end{sphinxVerbatimInput}
\begin{sphinxVerbatimOutput}

\begin{sphinxuseclass}{cell_output}
\begin{sphinxVerbatim}[commandchars=\\\{\}]
True
\end{sphinxVerbatim}

\end{sphinxuseclass}\end{sphinxVerbatimOutput}

\end{sphinxuseclass}
\begin{sphinxuseclass}{cell}\begin{sphinxVerbatimInput}

\begin{sphinxuseclass}{cell_input}
\begin{sphinxVerbatim}[commandchars=\\\{\}]
\PYG{n+nb}{print}\PYG{p}{(}\PYG{n+nb}{bool}\PYG{p}{(}\PYG{o}{\PYGZhy{}}\PYG{l+m+mi}{101}\PYG{p}{)}\PYG{p}{)}     \PYG{c+c1}{\PYGZsh{} bool value of \PYGZhy{}101 is True}
\end{sphinxVerbatim}

\end{sphinxuseclass}\end{sphinxVerbatimInput}
\begin{sphinxVerbatimOutput}

\begin{sphinxuseclass}{cell_output}
\begin{sphinxVerbatim}[commandchars=\\\{\}]
True
\end{sphinxVerbatim}

\end{sphinxuseclass}\end{sphinxVerbatimOutput}

\end{sphinxuseclass}
\begin{sphinxuseclass}{cell}\begin{sphinxVerbatimInput}

\begin{sphinxuseclass}{cell_input}
\begin{sphinxVerbatim}[commandchars=\\\{\}]
\PYG{n+nb}{print}\PYG{p}{(}\PYG{n+nb}{bool}\PYG{p}{(}\PYG{l+m+mf}{23.567}\PYG{p}{)}\PYG{p}{)}   \PYG{c+c1}{\PYGZsh{} bool value of 23.567 is True}
\end{sphinxVerbatim}

\end{sphinxuseclass}\end{sphinxVerbatimInput}
\begin{sphinxVerbatimOutput}

\begin{sphinxuseclass}{cell_output}
\begin{sphinxVerbatim}[commandchars=\\\{\}]
True
\end{sphinxVerbatim}

\end{sphinxuseclass}\end{sphinxVerbatimOutput}

\end{sphinxuseclass}
\begin{sphinxuseclass}{cell}\begin{sphinxVerbatimInput}

\begin{sphinxuseclass}{cell_input}
\begin{sphinxVerbatim}[commandchars=\\\{\}]
\PYG{n+nb}{print}\PYG{p}{(}\PYG{n+nb}{bool}\PYG{p}{(}\PYG{l+m+mi}{0}\PYG{p}{)}\PYG{p}{)}        \PYG{c+c1}{\PYGZsh{} bool value of 0 is False}
\end{sphinxVerbatim}

\end{sphinxuseclass}\end{sphinxVerbatimInput}
\begin{sphinxVerbatimOutput}

\begin{sphinxuseclass}{cell_output}
\begin{sphinxVerbatim}[commandchars=\\\{\}]
False
\end{sphinxVerbatim}

\end{sphinxuseclass}\end{sphinxVerbatimOutput}

\end{sphinxuseclass}\begin{itemize}
\item {} 
\sphinxAtStartPar
You can perform algebraic operations with boolean values \sphinxcode{\sphinxupquote{True}} and \sphinxcode{\sphinxupquote{False}}.

\item {} 
\sphinxAtStartPar
In such operations, \sphinxcode{\sphinxupquote{True}} takes on the integer value \sphinxcode{\sphinxupquote{1}}, and \sphinxcode{\sphinxupquote{False}} takes on the integer value \sphinxcode{\sphinxupquote{0}}.

\end{itemize}

\begin{sphinxuseclass}{cell}\begin{sphinxVerbatimInput}

\begin{sphinxuseclass}{cell_input}
\begin{sphinxVerbatim}[commandchars=\\\{\}]
\PYG{n+nb}{print}\PYG{p}{(}\PYG{k+kc}{True}\PYG{o}{+}\PYG{l+m+mi}{1}\PYG{p}{)}   \PYG{c+c1}{\PYGZsh{} True+1=1+1=2}
\end{sphinxVerbatim}

\end{sphinxuseclass}\end{sphinxVerbatimInput}
\begin{sphinxVerbatimOutput}

\begin{sphinxuseclass}{cell_output}
\begin{sphinxVerbatim}[commandchars=\\\{\}]
2
\end{sphinxVerbatim}

\end{sphinxuseclass}\end{sphinxVerbatimOutput}

\end{sphinxuseclass}
\begin{sphinxuseclass}{cell}\begin{sphinxVerbatimInput}

\begin{sphinxuseclass}{cell_input}
\begin{sphinxVerbatim}[commandchars=\\\{\}]
\PYG{n+nb}{print}\PYG{p}{(}\PYG{k+kc}{False}\PYG{o}{*}\PYG{l+m+mi}{5}\PYG{p}{)}  \PYG{c+c1}{\PYGZsh{} False*5=0*5=0}
\end{sphinxVerbatim}

\end{sphinxuseclass}\end{sphinxVerbatimInput}
\begin{sphinxVerbatimOutput}

\begin{sphinxuseclass}{cell_output}
\begin{sphinxVerbatim}[commandchars=\\\{\}]
0
\end{sphinxVerbatim}

\end{sphinxuseclass}\end{sphinxVerbatimOutput}

\end{sphinxuseclass}
\begin{sphinxuseclass}{cell}\begin{sphinxVerbatimInput}

\begin{sphinxuseclass}{cell_input}
\begin{sphinxVerbatim}[commandchars=\\\{\}]
\PYG{n+nb}{print}\PYG{p}{(}\PYG{k+kc}{True}\PYG{o}{+}\PYG{l+m+mf}{4.6}\PYG{o}{+}\PYG{k+kc}{True}\PYG{o}{+}\PYG{k+kc}{False}\PYG{o}{+}\PYG{l+m+mi}{10}\PYG{o}{+}\PYG{k+kc}{True}\PYG{o}{+}\PYG{k+kc}{False}\PYG{p}{)}  \PYG{c+c1}{\PYGZsh{} 1+4.6+1+0+10+1+0}
\end{sphinxVerbatim}

\end{sphinxuseclass}\end{sphinxVerbatimInput}
\begin{sphinxVerbatimOutput}

\begin{sphinxuseclass}{cell_output}
\begin{sphinxVerbatim}[commandchars=\\\{\}]
17.6
\end{sphinxVerbatim}

\end{sphinxuseclass}\end{sphinxVerbatimOutput}

\end{sphinxuseclass}

\subsection{bool <–> float}
\label{\detokenize{conditionals:bool-float}}\begin{itemize}
\item {} 
\sphinxAtStartPar
bool –> float
\begin{itemize}
\item {} 
\sphinxAtStartPar
float(True) = 1.0

\item {} 
\sphinxAtStartPar
float(False) = 0.0

\end{itemize}

\item {} 
\sphinxAtStartPar
float –> bool
\begin{itemize}
\item {} 
\sphinxAtStartPar
bool( any float other than 0.0 ) = True

\item {} 
\sphinxAtStartPar
bool(0.0) = False

\end{itemize}

\end{itemize}

\begin{sphinxuseclass}{cell}\begin{sphinxVerbatimInput}

\begin{sphinxuseclass}{cell_input}
\begin{sphinxVerbatim}[commandchars=\\\{\}]
\PYG{n+nb}{print}\PYG{p}{(}\PYG{n+nb}{float}\PYG{p}{(}\PYG{k+kc}{True}\PYG{p}{)}\PYG{p}{)}      \PYG{c+c1}{\PYGZsh{} float value of True is 1.0}
\end{sphinxVerbatim}

\end{sphinxuseclass}\end{sphinxVerbatimInput}
\begin{sphinxVerbatimOutput}

\begin{sphinxuseclass}{cell_output}
\begin{sphinxVerbatim}[commandchars=\\\{\}]
1.0
\end{sphinxVerbatim}

\end{sphinxuseclass}\end{sphinxVerbatimOutput}

\end{sphinxuseclass}
\begin{sphinxuseclass}{cell}\begin{sphinxVerbatimInput}

\begin{sphinxuseclass}{cell_input}
\begin{sphinxVerbatim}[commandchars=\\\{\}]
\PYG{n+nb}{print}\PYG{p}{(}\PYG{n+nb}{float}\PYG{p}{(}\PYG{k+kc}{False}\PYG{p}{)}\PYG{p}{)}     \PYG{c+c1}{\PYGZsh{} float value of False is 0.0}
\end{sphinxVerbatim}

\end{sphinxuseclass}\end{sphinxVerbatimInput}
\begin{sphinxVerbatimOutput}

\begin{sphinxuseclass}{cell_output}
\begin{sphinxVerbatim}[commandchars=\\\{\}]
0.0
\end{sphinxVerbatim}

\end{sphinxuseclass}\end{sphinxVerbatimOutput}

\end{sphinxuseclass}
\begin{sphinxuseclass}{cell}\begin{sphinxVerbatimInput}

\begin{sphinxuseclass}{cell_input}
\begin{sphinxVerbatim}[commandchars=\\\{\}]
\PYG{n+nb}{print}\PYG{p}{(}\PYG{n+nb}{bool}\PYG{p}{(}\PYG{l+m+mf}{23.567}\PYG{p}{)}\PYG{p}{)}     \PYG{c+c1}{\PYGZsh{} bool value of 23.567 is True}
\end{sphinxVerbatim}

\end{sphinxuseclass}\end{sphinxVerbatimInput}
\begin{sphinxVerbatimOutput}

\begin{sphinxuseclass}{cell_output}
\begin{sphinxVerbatim}[commandchars=\\\{\}]
True
\end{sphinxVerbatim}

\end{sphinxuseclass}\end{sphinxVerbatimOutput}

\end{sphinxuseclass}
\begin{sphinxuseclass}{cell}\begin{sphinxVerbatimInput}

\begin{sphinxuseclass}{cell_input}
\begin{sphinxVerbatim}[commandchars=\\\{\}]
\PYG{n+nb}{print}\PYG{p}{(}\PYG{n+nb}{bool}\PYG{p}{(}\PYG{l+m+mf}{0.0}\PYG{p}{)}\PYG{p}{)}        \PYG{c+c1}{\PYGZsh{} bool value of 0.0 is False}
\end{sphinxVerbatim}

\end{sphinxuseclass}\end{sphinxVerbatimInput}
\begin{sphinxVerbatimOutput}

\begin{sphinxuseclass}{cell_output}
\begin{sphinxVerbatim}[commandchars=\\\{\}]
False
\end{sphinxVerbatim}

\end{sphinxuseclass}\end{sphinxVerbatimOutput}

\end{sphinxuseclass}

\subsection{bool <–> str}
\label{\detokenize{conditionals:bool-str}}\begin{itemize}
\item {} 
\sphinxAtStartPar
bool –> str
\begin{itemize}
\item {} 
\sphinxAtStartPar
str(True)  = ‘True’

\item {} 
\sphinxAtStartPar
str(False) = ‘False’

\end{itemize}

\item {} 
\sphinxAtStartPar
str –> bool
\begin{itemize}
\item {} 
\sphinxAtStartPar
bool( any string other than empty string ) = True

\item {} 
\sphinxAtStartPar
bool( empty string ) = False
\begin{itemize}
\item {} 
\sphinxAtStartPar
empty string = \sphinxcode{\sphinxupquote{''}}

\end{itemize}

\end{itemize}

\end{itemize}

\begin{sphinxuseclass}{cell}\begin{sphinxVerbatimInput}

\begin{sphinxuseclass}{cell_input}
\begin{sphinxVerbatim}[commandchars=\\\{\}]
\PYG{n+nb}{print}\PYG{p}{(}\PYG{n+nb}{str}\PYG{p}{(}\PYG{k+kc}{True}\PYG{p}{)}\PYG{p}{)}      \PYG{c+c1}{\PYGZsh{} str value of True is \PYGZsq{}True\PYGZsq{}}
\end{sphinxVerbatim}

\end{sphinxuseclass}\end{sphinxVerbatimInput}
\begin{sphinxVerbatimOutput}

\begin{sphinxuseclass}{cell_output}
\begin{sphinxVerbatim}[commandchars=\\\{\}]
True
\end{sphinxVerbatim}

\end{sphinxuseclass}\end{sphinxVerbatimOutput}

\end{sphinxuseclass}
\begin{sphinxuseclass}{cell}\begin{sphinxVerbatimInput}

\begin{sphinxuseclass}{cell_input}
\begin{sphinxVerbatim}[commandchars=\\\{\}]
\PYG{n+nb}{print}\PYG{p}{(}\PYG{n+nb}{str}\PYG{p}{(}\PYG{k+kc}{False}\PYG{p}{)}\PYG{p}{)}     \PYG{c+c1}{\PYGZsh{} str value of False is \PYGZsq{}False\PYGZsq{}}
\end{sphinxVerbatim}

\end{sphinxuseclass}\end{sphinxVerbatimInput}
\begin{sphinxVerbatimOutput}

\begin{sphinxuseclass}{cell_output}
\begin{sphinxVerbatim}[commandchars=\\\{\}]
False
\end{sphinxVerbatim}

\end{sphinxuseclass}\end{sphinxVerbatimOutput}

\end{sphinxuseclass}
\begin{sphinxuseclass}{cell}\begin{sphinxVerbatimInput}

\begin{sphinxuseclass}{cell_input}
\begin{sphinxVerbatim}[commandchars=\\\{\}]
\PYG{n+nb}{print}\PYG{p}{(}\PYG{n+nb}{bool}\PYG{p}{(}\PYG{l+s+s1}{\PYGZsq{}}\PYG{l+s+s1}{HELLO}\PYG{l+s+s1}{\PYGZsq{}}\PYG{p}{)}\PYG{p}{)}   \PYG{c+c1}{\PYGZsh{} bool value of \PYGZsq{}HELLO\PYGZsq{} is True}
\end{sphinxVerbatim}

\end{sphinxuseclass}\end{sphinxVerbatimInput}
\begin{sphinxVerbatimOutput}

\begin{sphinxuseclass}{cell_output}
\begin{sphinxVerbatim}[commandchars=\\\{\}]
True
\end{sphinxVerbatim}

\end{sphinxuseclass}\end{sphinxVerbatimOutput}

\end{sphinxuseclass}
\begin{sphinxuseclass}{cell}\begin{sphinxVerbatimInput}

\begin{sphinxuseclass}{cell_input}
\begin{sphinxVerbatim}[commandchars=\\\{\}]
\PYG{n+nb}{print}\PYG{p}{(}\PYG{n+nb}{bool}\PYG{p}{(}\PYG{l+s+s1}{\PYGZsq{}}\PYG{l+s+s1}{\PYGZsq{}}\PYG{p}{)}\PYG{p}{)}        \PYG{c+c1}{\PYGZsh{} bool value of \PYGZsq{}\PYGZsq{} = empty string is False}
\end{sphinxVerbatim}

\end{sphinxuseclass}\end{sphinxVerbatimInput}
\begin{sphinxVerbatimOutput}

\begin{sphinxuseclass}{cell_output}
\begin{sphinxVerbatim}[commandchars=\\\{\}]
False
\end{sphinxVerbatim}

\end{sphinxuseclass}\end{sphinxVerbatimOutput}

\end{sphinxuseclass}

\section{Conditions (Boolean Expression)}
\label{\detokenize{conditionals:conditions-boolean-expression}}
\sphinxAtStartPar
An expression that evaluates to either True or False.


\subsection{Comparison operators}
\label{\detokenize{conditionals:comparison-operators}}
\sphinxAtStartPar
Since \sphinxcode{\sphinxupquote{=}} is used for assignment, \sphinxcode{\sphinxupquote{==}} is used for comparison.


\begin{savenotes}\sphinxattablestart
\centering
\begin{tabulary}{\linewidth}[t]{|T|T|}
\hline
\sphinxstyletheadfamily 
\sphinxAtStartPar
Operator
&\sphinxstyletheadfamily 
\sphinxAtStartPar
Meaning
\\
\hline
\sphinxAtStartPar
==
&
\sphinxAtStartPar
equal (only value not type)
\\
\hline
\sphinxAtStartPar
!=
&
\sphinxAtStartPar
not equal (only value not type)
\\
\hline
\sphinxAtStartPar
<
&
\sphinxAtStartPar
less than
\\
\hline
\sphinxAtStartPar
<=
&
\sphinxAtStartPar
less than or equal to
\\
\hline
\sphinxAtStartPar
>
&
\sphinxAtStartPar
greater than
\\
\hline
\sphinxAtStartPar
>=
&
\sphinxAtStartPar
greater than or equal to
\\
\hline
\sphinxAtStartPar
is
&
\sphinxAtStartPar
equal (value and type)
\\
\hline
\sphinxAtStartPar
is not
&
\sphinxAtStartPar
not equal to (value and type)
\\
\hline
\end{tabulary}
\par
\sphinxattableend\end{savenotes}
\begin{itemize}
\item {} 
\sphinxAtStartPar
Comparison of strings
\begin{itemize}
\item {} 
\sphinxAtStartPar
Inequality operators follow the dictionary order.

\item {} 
\sphinxAtStartPar
Uppercase letters are considered before lowercase letters.

\item {} 
\sphinxAtStartPar
Digits are considered before uppercase letters.

\item {} 
\sphinxAtStartPar
The order is as follows: Digits < Uppercase letters < Lowercase letters.

\end{itemize}

\item {} 
\sphinxAtStartPar
The \sphinxcode{\sphinxupquote{is}} keyword checks for identity.
\begin{itemize}
\item {} 
\sphinxAtStartPar
Objects compared with \sphinxcode{\sphinxupquote{is}} have the same id() numbers, indicating they are stored in the same location in memory.

\item {} 
\sphinxAtStartPar
Using \sphinxcode{\sphinxupquote{is}} means the objects point to the exact same object.

\end{itemize}

\item {} 
\sphinxAtStartPar
\sphinxcode{\sphinxupquote{==}} checks for equality
\begin{itemize}
\item {} 
\sphinxAtStartPar
The values are same or not.

\item {} 
\sphinxAtStartPar
4 and 4.0 as a mathematical values same even though their type is different

\end{itemize}

\item {} 
\sphinxAtStartPar
The \sphinxcode{\sphinxupquote{==}} operator checks for equality.
\begin{itemize}
\item {} 
\sphinxAtStartPar
It compares whether the values are the same or not.

\item {} 
\sphinxAtStartPar
For example, 4 and 4.0 are considered equal mathematically, even though their data types are different.

\end{itemize}

\item {} 
\sphinxAtStartPar
When you use \sphinxcode{\sphinxupquote{is}} with strings, numbers, or booleans, a syntax error occurs.
\begin{itemize}
\item {} 
\sphinxAtStartPar
The error suggests using \sphinxcode{\sphinxupquote{==}} or \sphinxcode{\sphinxupquote{!=}} for these types.

\end{itemize}

\end{itemize}

\begin{sphinxuseclass}{cell}\begin{sphinxVerbatimInput}

\begin{sphinxuseclass}{cell_input}
\begin{sphinxVerbatim}[commandchars=\\\{\}]
\PYG{n+nb}{print}\PYG{p}{(}\PYG{l+m+mi}{5}\PYG{o}{==}\PYG{l+m+mi}{5}\PYG{p}{)}  \PYG{c+c1}{\PYGZsh{} 5 is equal to 5 is True}
\end{sphinxVerbatim}

\end{sphinxuseclass}\end{sphinxVerbatimInput}
\begin{sphinxVerbatimOutput}

\begin{sphinxuseclass}{cell_output}
\begin{sphinxVerbatim}[commandchars=\\\{\}]
True
\end{sphinxVerbatim}

\end{sphinxuseclass}\end{sphinxVerbatimOutput}

\end{sphinxuseclass}
\begin{sphinxuseclass}{cell}\begin{sphinxVerbatimInput}

\begin{sphinxuseclass}{cell_input}
\begin{sphinxVerbatim}[commandchars=\\\{\}]
\PYG{n+nb}{print}\PYG{p}{(}\PYG{l+m+mi}{5}\PYG{o}{==}\PYG{l+m+mi}{7}\PYG{p}{)}  \PYG{c+c1}{\PYGZsh{} 5 is equal to 7 is False}
\end{sphinxVerbatim}

\end{sphinxuseclass}\end{sphinxVerbatimInput}
\begin{sphinxVerbatimOutput}

\begin{sphinxuseclass}{cell_output}
\begin{sphinxVerbatim}[commandchars=\\\{\}]
False
\end{sphinxVerbatim}

\end{sphinxuseclass}\end{sphinxVerbatimOutput}

\end{sphinxuseclass}
\begin{sphinxuseclass}{cell}\begin{sphinxVerbatimInput}

\begin{sphinxuseclass}{cell_input}
\begin{sphinxVerbatim}[commandchars=\\\{\}]
\PYG{n+nb}{print}\PYG{p}{(}\PYG{l+m+mi}{3}\PYG{o}{==}\PYG{l+m+mf}{3.0}\PYG{p}{)} \PYG{c+c1}{\PYGZsh{} The mathematical value of 3 and 3.0 are equal is True}
\end{sphinxVerbatim}

\end{sphinxuseclass}\end{sphinxVerbatimInput}
\begin{sphinxVerbatimOutput}

\begin{sphinxuseclass}{cell_output}
\begin{sphinxVerbatim}[commandchars=\\\{\}]
True
\end{sphinxVerbatim}

\end{sphinxuseclass}\end{sphinxVerbatimOutput}

\end{sphinxuseclass}
\begin{sphinxuseclass}{cell}\begin{sphinxVerbatimInput}

\begin{sphinxuseclass}{cell_input}
\begin{sphinxVerbatim}[commandchars=\\\{\}]
\PYG{n+nb}{print}\PYG{p}{(}\PYG{l+m+mi}{5}\PYG{o}{!=}\PYG{l+m+mi}{5}\PYG{p}{)}  \PYG{c+c1}{\PYGZsh{} 5 is not equal to 5 is False}
\end{sphinxVerbatim}

\end{sphinxuseclass}\end{sphinxVerbatimInput}
\begin{sphinxVerbatimOutput}

\begin{sphinxuseclass}{cell_output}
\begin{sphinxVerbatim}[commandchars=\\\{\}]
False
\end{sphinxVerbatim}

\end{sphinxuseclass}\end{sphinxVerbatimOutput}

\end{sphinxuseclass}
\begin{sphinxuseclass}{cell}\begin{sphinxVerbatimInput}

\begin{sphinxuseclass}{cell_input}
\begin{sphinxVerbatim}[commandchars=\\\{\}]
\PYG{n+nb}{print}\PYG{p}{(}\PYG{l+m+mi}{5}\PYG{o}{!=}\PYG{l+m+mi}{7}\PYG{p}{)}  \PYG{c+c1}{\PYGZsh{} 5 is not equal to 7 is True}
\end{sphinxVerbatim}

\end{sphinxuseclass}\end{sphinxVerbatimInput}
\begin{sphinxVerbatimOutput}

\begin{sphinxuseclass}{cell_output}
\begin{sphinxVerbatim}[commandchars=\\\{\}]
True
\end{sphinxVerbatim}

\end{sphinxuseclass}\end{sphinxVerbatimOutput}

\end{sphinxuseclass}
\begin{sphinxuseclass}{cell}\begin{sphinxVerbatimInput}

\begin{sphinxuseclass}{cell_input}
\begin{sphinxVerbatim}[commandchars=\\\{\}]
\PYG{n+nb}{print}\PYG{p}{(}\PYG{l+m+mi}{5}\PYG{o}{\PYGZlt{}}\PYG{l+m+mi}{7}\PYG{p}{)}   \PYG{c+c1}{\PYGZsh{} 5 is less than 7 is True}
\end{sphinxVerbatim}

\end{sphinxuseclass}\end{sphinxVerbatimInput}
\begin{sphinxVerbatimOutput}

\begin{sphinxuseclass}{cell_output}
\begin{sphinxVerbatim}[commandchars=\\\{\}]
True
\end{sphinxVerbatim}

\end{sphinxuseclass}\end{sphinxVerbatimOutput}

\end{sphinxuseclass}
\begin{sphinxuseclass}{cell}\begin{sphinxVerbatimInput}

\begin{sphinxuseclass}{cell_input}
\begin{sphinxVerbatim}[commandchars=\\\{\}]
\PYG{n+nb}{print}\PYG{p}{(}\PYG{l+m+mi}{10}\PYG{o}{\PYGZgt{}}\PYG{l+m+mi}{15}\PYG{p}{)}  \PYG{c+c1}{\PYGZsh{} 10 is greater than 15 is False}
\end{sphinxVerbatim}

\end{sphinxuseclass}\end{sphinxVerbatimInput}
\begin{sphinxVerbatimOutput}

\begin{sphinxuseclass}{cell_output}
\begin{sphinxVerbatim}[commandchars=\\\{\}]
False
\end{sphinxVerbatim}

\end{sphinxuseclass}\end{sphinxVerbatimOutput}

\end{sphinxuseclass}
\begin{sphinxuseclass}{cell}\begin{sphinxVerbatimInput}

\begin{sphinxuseclass}{cell_input}
\begin{sphinxVerbatim}[commandchars=\\\{\}]
\PYG{n+nb}{print}\PYG{p}{(}\PYG{l+s+s1}{\PYGZsq{}}\PYG{l+s+s1}{k}\PYG{l+s+s1}{\PYGZsq{}}\PYG{o}{==}\PYG{l+s+s1}{\PYGZsq{}}\PYG{l+s+s1}{t}\PYG{l+s+s1}{\PYGZsq{}}\PYG{p}{)}  \PYG{c+c1}{\PYGZsh{} \PYGZsq{}k\PYGZsq{} is equal to \PYGZsq{}t\PYGZsq{} is False}
\end{sphinxVerbatim}

\end{sphinxuseclass}\end{sphinxVerbatimInput}
\begin{sphinxVerbatimOutput}

\begin{sphinxuseclass}{cell_output}
\begin{sphinxVerbatim}[commandchars=\\\{\}]
False
\end{sphinxVerbatim}

\end{sphinxuseclass}\end{sphinxVerbatimOutput}

\end{sphinxuseclass}
\begin{sphinxuseclass}{cell}\begin{sphinxVerbatimInput}

\begin{sphinxuseclass}{cell_input}
\begin{sphinxVerbatim}[commandchars=\\\{\}]
\PYG{n+nb}{print}\PYG{p}{(}\PYG{l+s+s1}{\PYGZsq{}}\PYG{l+s+s1}{k}\PYG{l+s+s1}{\PYGZsq{}}\PYG{o}{==}\PYG{l+s+s1}{\PYGZsq{}}\PYG{l+s+s1}{k}\PYG{l+s+s1}{\PYGZsq{}}\PYG{p}{)}  \PYG{c+c1}{\PYGZsh{} \PYGZsq{}k\PYGZsq{} is equal to \PYGZsq{}k\PYGZsq{} is True}
\end{sphinxVerbatim}

\end{sphinxuseclass}\end{sphinxVerbatimInput}
\begin{sphinxVerbatimOutput}

\begin{sphinxuseclass}{cell_output}
\begin{sphinxVerbatim}[commandchars=\\\{\}]
True
\end{sphinxVerbatim}

\end{sphinxuseclass}\end{sphinxVerbatimOutput}

\end{sphinxuseclass}
\begin{sphinxuseclass}{cell}\begin{sphinxVerbatimInput}

\begin{sphinxuseclass}{cell_input}
\begin{sphinxVerbatim}[commandchars=\\\{\}]
\PYG{n+nb}{print}\PYG{p}{(}\PYG{l+s+s1}{\PYGZsq{}}\PYG{l+s+s1}{K}\PYG{l+s+s1}{\PYGZsq{}}\PYG{o}{==}\PYG{l+s+s1}{\PYGZsq{}}\PYG{l+s+s1}{k}\PYG{l+s+s1}{\PYGZsq{}}\PYG{p}{)}  \PYG{c+c1}{\PYGZsh{} \PYGZsq{}k\PYGZsq{} is equal to \PYGZsq{}K\PYGZsq{} is False (case sensitive).}
\end{sphinxVerbatim}

\end{sphinxuseclass}\end{sphinxVerbatimInput}
\begin{sphinxVerbatimOutput}

\begin{sphinxuseclass}{cell_output}
\begin{sphinxVerbatim}[commandchars=\\\{\}]
False
\end{sphinxVerbatim}

\end{sphinxuseclass}\end{sphinxVerbatimOutput}

\end{sphinxuseclass}
\begin{sphinxuseclass}{cell}\begin{sphinxVerbatimInput}

\begin{sphinxuseclass}{cell_input}
\begin{sphinxVerbatim}[commandchars=\\\{\}]
\PYG{n+nb}{print}\PYG{p}{(}\PYG{l+s+s1}{\PYGZsq{}}\PYG{l+s+s1}{k}\PYG{l+s+s1}{\PYGZsq{}} \PYG{o}{\PYGZlt{}} \PYG{l+s+s1}{\PYGZsq{}}\PYG{l+s+s1}{t}\PYG{l+s+s1}{\PYGZsq{}}\PYG{p}{)}  \PYG{c+c1}{\PYGZsh{} \PYGZsq{}k\PYGZsq{} comes before  \PYGZsq{}t\PYGZsq{} in dictionary order is True}
\end{sphinxVerbatim}

\end{sphinxuseclass}\end{sphinxVerbatimInput}
\begin{sphinxVerbatimOutput}

\begin{sphinxuseclass}{cell_output}
\begin{sphinxVerbatim}[commandchars=\\\{\}]
True
\end{sphinxVerbatim}

\end{sphinxuseclass}\end{sphinxVerbatimOutput}

\end{sphinxuseclass}
\begin{sphinxuseclass}{cell}\begin{sphinxVerbatimInput}

\begin{sphinxuseclass}{cell_input}
\begin{sphinxVerbatim}[commandchars=\\\{\}]
\PYG{n+nb}{print}\PYG{p}{(}\PYG{l+s+s1}{\PYGZsq{}}\PYG{l+s+s1}{z}\PYG{l+s+s1}{\PYGZsq{}} \PYG{o}{\PYGZlt{}} \PYG{l+s+s1}{\PYGZsq{}}\PYG{l+s+s1}{t}\PYG{l+s+s1}{\PYGZsq{}}\PYG{p}{)}  \PYG{c+c1}{\PYGZsh{} \PYGZsq{}z\PYGZsq{} comes before  \PYGZsq{}t\PYGZsq{} in dictionarcy order is False}
\end{sphinxVerbatim}

\end{sphinxuseclass}\end{sphinxVerbatimInput}
\begin{sphinxVerbatimOutput}

\begin{sphinxuseclass}{cell_output}
\begin{sphinxVerbatim}[commandchars=\\\{\}]
False
\end{sphinxVerbatim}

\end{sphinxuseclass}\end{sphinxVerbatimOutput}

\end{sphinxuseclass}
\begin{sphinxuseclass}{cell}\begin{sphinxVerbatimInput}

\begin{sphinxuseclass}{cell_input}
\begin{sphinxVerbatim}[commandchars=\\\{\}]
\PYG{n+nb}{print}\PYG{p}{(}\PYG{l+s+s1}{\PYGZsq{}}\PYG{l+s+s1}{money}\PYG{l+s+s1}{\PYGZsq{}} \PYG{o}{\PYGZlt{}} \PYG{l+s+s1}{\PYGZsq{}}\PYG{l+s+s1}{table}\PYG{l+s+s1}{\PYGZsq{}}\PYG{p}{)}  \PYG{c+c1}{\PYGZsh{} \PYGZsq{}money\PYGZsq{} comes before \PYGZsq{}table\PYGZsq{} in dictionarcy order is False}
\end{sphinxVerbatim}

\end{sphinxuseclass}\end{sphinxVerbatimInput}
\begin{sphinxVerbatimOutput}

\begin{sphinxuseclass}{cell_output}
\begin{sphinxVerbatim}[commandchars=\\\{\}]
True
\end{sphinxVerbatim}

\end{sphinxuseclass}\end{sphinxVerbatimOutput}

\end{sphinxuseclass}
\begin{sphinxuseclass}{cell}\begin{sphinxVerbatimInput}

\begin{sphinxuseclass}{cell_input}
\begin{sphinxVerbatim}[commandchars=\\\{\}]
\PYG{n+nb}{print}\PYG{p}{(}\PYG{l+s+s1}{\PYGZsq{}}\PYG{l+s+s1}{Z}\PYG{l+s+s1}{\PYGZsq{}} \PYG{o}{\PYGZlt{}} \PYG{l+s+s1}{\PYGZsq{}}\PYG{l+s+s1}{a}\PYG{l+s+s1}{\PYGZsq{}}\PYG{p}{)}  \PYG{c+c1}{\PYGZsh{} capital letters come before lower case letters in dictionarcy order }
\end{sphinxVerbatim}

\end{sphinxuseclass}\end{sphinxVerbatimInput}
\begin{sphinxVerbatimOutput}

\begin{sphinxuseclass}{cell_output}
\begin{sphinxVerbatim}[commandchars=\\\{\}]
True
\end{sphinxVerbatim}

\end{sphinxuseclass}\end{sphinxVerbatimOutput}

\end{sphinxuseclass}
\begin{sphinxuseclass}{cell}\begin{sphinxVerbatimInput}

\begin{sphinxuseclass}{cell_input}
\begin{sphinxVerbatim}[commandchars=\\\{\}]
\PYG{n+nb}{print}\PYG{p}{(}\PYG{l+s+s1}{\PYGZsq{}}\PYG{l+s+s1}{3}\PYG{l+s+s1}{\PYGZsq{}} \PYG{o}{\PYGZlt{}} \PYG{l+s+s1}{\PYGZsq{}}\PYG{l+s+s1}{A}\PYG{l+s+s1}{\PYGZsq{}}\PYG{p}{)}  \PYG{c+c1}{\PYGZsh{} digits come before  letters in dictionarcy order }
\end{sphinxVerbatim}

\end{sphinxuseclass}\end{sphinxVerbatimInput}
\begin{sphinxVerbatimOutput}

\begin{sphinxuseclass}{cell_output}
\begin{sphinxVerbatim}[commandchars=\\\{\}]
True
\end{sphinxVerbatim}

\end{sphinxuseclass}\end{sphinxVerbatimOutput}

\end{sphinxuseclass}
\begin{sphinxVerbatim}[commandchars=\\\{\}]
\PYG{c+c1}{\PYGZsh{} SyntaxWarning: Use == instead}
\PYG{n+nb}{print}\PYG{p}{(} \PYG{l+s+s1}{\PYGZsq{}}\PYG{l+s+s1}{k}\PYG{l+s+s1}{\PYGZsq{}} \PYG{o+ow}{is} \PYG{l+s+s1}{\PYGZsq{}}\PYG{l+s+s1}{K}\PYG{l+s+s1}{\PYGZsq{}}\PYG{p}{)}
\end{sphinxVerbatim}

\begin{sphinxVerbatim}[commandchars=\\\{\}]
\PYG{c+c1}{\PYGZsh{} SyntaxWarning: Use == instead}
\PYG{n+nb}{print}\PYG{p}{(} \PYG{l+m+mi}{2} \PYG{o+ow}{is} \PYG{l+m+mi}{3}\PYG{p}{)}
\end{sphinxVerbatim}


\section{if statement}
\label{\detokenize{conditionals:if-statement}}
\sphinxAtStartPar
It is used to execute a block of code depending on a condition, which is a boolean expression. So, it returns either True or False.
\begin{itemize}
\item {} 
\sphinxAtStartPar
If the condition is True, a block of code will be executed.

\item {} 
\sphinxAtStartPar
If the condition is False, the block of code will not be executed and will be skipped.

\item {} 
\sphinxAtStartPar
The structure of an if statement is as follows:

\end{itemize}

\sphinxAtStartPar
\sphinxcode{\sphinxupquote{if condition:}}\\
   \sphinxcode{\sphinxupquote{          }}     \\
   \sphinxcode{\sphinxupquote{BLOCK CODE}}     \\
   \sphinxcode{\sphinxupquote{          }}     
\begin{itemize}
\item {} 
\sphinxAtStartPar
In the structure above:
\begin{itemize}
\item {} 
\sphinxAtStartPar
\sphinxcode{\sphinxupquote{if}} is a keyword.

\item {} 
\sphinxAtStartPar
\sphinxcode{\sphinxupquote{condition}} is a boolean expression which is boolean True or False.

\item {} 
\sphinxAtStartPar
\sphinxcode{\sphinxupquote{:}} comes right after the condition and it means this line will be followed by another code

\item {} 
\sphinxAtStartPar
\sphinxcode{\sphinxupquote{BLOCK CODE}} is an group of code with same indentation level that will be executed if condition is True.

\end{itemize}

\end{itemize}

\sphinxAtStartPar
In the structure above:
\begin{itemize}
\item {} 
\sphinxAtStartPar
if is a keyword.

\item {} 
\sphinxAtStartPar
condition is a boolean expression, which is either True or False.

\item {} 
\sphinxAtStartPar
\sphinxcode{\sphinxupquote{:}} comes right after the condition and means this line will be followed by another code.

\item {} 
\sphinxAtStartPar
\sphinxcode{\sphinxupquote{BLOCK CODE}} is a group of code with the same indentation level that will be executed if the condition is True.

\end{itemize}

\sphinxAtStartPar
The cases are as follows:
\begin{enumerate}
\sphinxsetlistlabels{\arabic}{enumi}{enumii}{}{.}%
\item {} 
\sphinxAtStartPar
condition is True:

\end{enumerate}

\sphinxAtStartPar
\sphinxcode{\sphinxupquote{if True:}}\\
   \sphinxcode{\sphinxupquote{          }}     \\
   \sphinxcode{\sphinxupquote{BLOCK CODE}}         BLOCK CODE will be executed.\\
   \sphinxcode{\sphinxupquote{          }}     
\begin{enumerate}
\sphinxsetlistlabels{\arabic}{enumi}{enumii}{}{.}%
\setcounter{enumi}{1}
\item {} 
\sphinxAtStartPar
condition is False:\\
\sphinxcode{\sphinxupquote{if False:}}\\
   \sphinxcode{\sphinxupquote{          }}     \\
   \sphinxcode{\sphinxupquote{BLOCK CODE}}         BLOCK CODE will be skipped.\\
   \sphinxcode{\sphinxupquote{          }}     

\end{enumerate}

\sphinxAtStartPar
In the code below:
\begin{itemize}
\item {} 
\sphinxAtStartPar
\sphinxcode{\sphinxupquote{condition}} is True because 75 > 65, so the block code will be executed.

\item {} 
\sphinxAtStartPar
\sphinxcode{\sphinxupquote{BLOCK CODE}} consists of two lines of code, and they will be executed.

\end{itemize}

\begin{sphinxuseclass}{cell}\begin{sphinxVerbatimInput}

\begin{sphinxuseclass}{cell_input}
\begin{sphinxVerbatim}[commandchars=\\\{\}]
\PYG{n}{grade} \PYG{o}{=} \PYG{l+m+mi}{75}

\PYG{k}{if} \PYG{n}{grade} \PYG{o}{\PYGZgt{}} \PYG{l+m+mi}{65}\PYG{p}{:}        
    \PYG{n+nb}{print}\PYG{p}{(}\PYG{l+s+s1}{\PYGZsq{}}\PYG{l+s+s1}{You passed.}\PYG{l+s+s1}{\PYGZsq{}}\PYG{p}{)}
    \PYG{n+nb}{print}\PYG{p}{(}\PYG{l+s+s1}{\PYGZsq{}}\PYG{l+s+s1}{Congrats!}\PYG{l+s+s1}{\PYGZsq{}}\PYG{p}{)}
\end{sphinxVerbatim}

\end{sphinxuseclass}\end{sphinxVerbatimInput}
\begin{sphinxVerbatimOutput}

\begin{sphinxuseclass}{cell_output}
\begin{sphinxVerbatim}[commandchars=\\\{\}]
You passed.
Congrats!
\end{sphinxVerbatim}

\end{sphinxuseclass}\end{sphinxVerbatimOutput}

\end{sphinxuseclass}\begin{itemize}
\item {} 
\sphinxAtStartPar
In the code below:
\begin{itemize}
\item {} 
\sphinxAtStartPar
\sphinxcode{\sphinxupquote{condition}} is False because 55>65 is False, so the block code will not be executed.

\item {} 
\sphinxAtStartPar
\sphinxcode{\sphinxupquote{BLOCK CDDE}} consists of two lines of code, and they will be skipped.

\item {} 
\sphinxAtStartPar
There is no output for this code.

\end{itemize}

\end{itemize}

\begin{sphinxuseclass}{cell}\begin{sphinxVerbatimInput}

\begin{sphinxuseclass}{cell_input}
\begin{sphinxVerbatim}[commandchars=\\\{\}]
\PYG{n}{grade} \PYG{o}{=} \PYG{l+m+mi}{55}

\PYG{k}{if} \PYG{n}{grade} \PYG{o}{\PYGZgt{}} \PYG{l+m+mi}{65}\PYG{p}{:}        
    \PYG{n+nb}{print}\PYG{p}{(}\PYG{l+s+s1}{\PYGZsq{}}\PYG{l+s+s1}{You passed.}\PYG{l+s+s1}{\PYGZsq{}}\PYG{p}{)}
    \PYG{n+nb}{print}\PYG{p}{(}\PYG{l+s+s1}{\PYGZsq{}}\PYG{l+s+s1}{Congrats!}\PYG{l+s+s1}{\PYGZsq{}}\PYG{p}{)}
\end{sphinxVerbatim}

\end{sphinxuseclass}\end{sphinxVerbatimInput}

\end{sphinxuseclass}\begin{itemize}
\item {} 
\sphinxAtStartPar
In the code below:
\begin{itemize}
\item {} 
\sphinxAtStartPar
\sphinxcode{\sphinxupquote{condition}} is False because 55>65 is False, so the block code will not be executed.

\item {} 
\sphinxAtStartPar
\sphinxcode{\sphinxupquote{BLOCK CDDE}} consists of two lines of code and they will be skipped.

\item {} 
\sphinxAtStartPar
The last print statement is not part of the \sphinxcode{\sphinxupquote{BLOCK CODE}}. After skipping the \sphinxcode{\sphinxupquote{BLOCK CODE}}, this last print statement will be executed.

\end{itemize}

\end{itemize}

\begin{sphinxuseclass}{cell}\begin{sphinxVerbatimInput}

\begin{sphinxuseclass}{cell_input}
\begin{sphinxVerbatim}[commandchars=\\\{\}]
\PYG{n}{grade} \PYG{o}{=} \PYG{l+m+mi}{55}

\PYG{k}{if} \PYG{n}{grade} \PYG{o}{\PYGZgt{}} \PYG{l+m+mi}{65}\PYG{p}{:}        
    \PYG{n+nb}{print}\PYG{p}{(}\PYG{l+s+s1}{\PYGZsq{}}\PYG{l+s+s1}{You passed.}\PYG{l+s+s1}{\PYGZsq{}}\PYG{p}{)}
    \PYG{n+nb}{print}\PYG{p}{(}\PYG{l+s+s1}{\PYGZsq{}}\PYG{l+s+s1}{Congrats!}\PYG{l+s+s1}{\PYGZsq{}}\PYG{p}{)}
\PYG{n+nb}{print}\PYG{p}{(}\PYG{l+s+s1}{\PYGZsq{}}\PYG{l+s+s1}{Bye}\PYG{l+s+s1}{\PYGZsq{}}\PYG{p}{)}
\end{sphinxVerbatim}

\end{sphinxuseclass}\end{sphinxVerbatimInput}
\begin{sphinxVerbatimOutput}

\begin{sphinxuseclass}{cell_output}
\begin{sphinxVerbatim}[commandchars=\\\{\}]
Bye
\end{sphinxVerbatim}

\end{sphinxuseclass}\end{sphinxVerbatimOutput}

\end{sphinxuseclass}\begin{itemize}
\item {} 
\sphinxAtStartPar
As a condition, numbers, strings, and boolean values can be directly used.

\item {} 
\sphinxAtStartPar
Python will automatically convert them into boolean values.

\end{itemize}

\begin{sphinxuseclass}{cell}\begin{sphinxVerbatimInput}

\begin{sphinxuseclass}{cell_input}
\begin{sphinxVerbatim}[commandchars=\\\{\}]
\PYG{c+c1}{\PYGZsh{} condition is always True}
\PYG{c+c1}{\PYGZsh{} Hello will always be printed.}

\PYG{k}{if} \PYG{k+kc}{True}\PYG{p}{:}
    \PYG{n+nb}{print}\PYG{p}{(}\PYG{l+s+s1}{\PYGZsq{}}\PYG{l+s+s1}{Hello}\PYG{l+s+s1}{\PYGZsq{}}\PYG{p}{)}
\end{sphinxVerbatim}

\end{sphinxuseclass}\end{sphinxVerbatimInput}
\begin{sphinxVerbatimOutput}

\begin{sphinxuseclass}{cell_output}
\begin{sphinxVerbatim}[commandchars=\\\{\}]
Hello
\end{sphinxVerbatim}

\end{sphinxuseclass}\end{sphinxVerbatimOutput}

\end{sphinxuseclass}
\begin{sphinxuseclass}{cell}\begin{sphinxVerbatimInput}

\begin{sphinxuseclass}{cell_input}
\begin{sphinxVerbatim}[commandchars=\\\{\}]
\PYG{c+c1}{\PYGZsh{} condition is always False: print statement will be skipped. }
\PYG{c+c1}{\PYGZsh{} no output}

\PYG{k}{if} \PYG{k+kc}{False}\PYG{p}{:}
    \PYG{n+nb}{print}\PYG{p}{(}\PYG{l+s+s1}{\PYGZsq{}}\PYG{l+s+s1}{Hello}\PYG{l+s+s1}{\PYGZsq{}}\PYG{p}{)}
\end{sphinxVerbatim}

\end{sphinxuseclass}\end{sphinxVerbatimInput}

\end{sphinxuseclass}
\begin{sphinxuseclass}{cell}\begin{sphinxVerbatimInput}

\begin{sphinxuseclass}{cell_input}
\begin{sphinxVerbatim}[commandchars=\\\{\}]
\PYG{c+c1}{\PYGZsh{} condition is always True becasue bool(5)=True}
\PYG{c+c1}{\PYGZsh{} Hello will always be printed}

\PYG{k}{if} \PYG{l+m+mi}{5}\PYG{p}{:}
    \PYG{n+nb}{print}\PYG{p}{(}\PYG{l+s+s1}{\PYGZsq{}}\PYG{l+s+s1}{Hello}\PYG{l+s+s1}{\PYGZsq{}}\PYG{p}{)}
\end{sphinxVerbatim}

\end{sphinxuseclass}\end{sphinxVerbatimInput}
\begin{sphinxVerbatimOutput}

\begin{sphinxuseclass}{cell_output}
\begin{sphinxVerbatim}[commandchars=\\\{\}]
Hello
\end{sphinxVerbatim}

\end{sphinxuseclass}\end{sphinxVerbatimOutput}

\end{sphinxuseclass}
\begin{sphinxuseclass}{cell}\begin{sphinxVerbatimInput}

\begin{sphinxuseclass}{cell_input}
\begin{sphinxVerbatim}[commandchars=\\\{\}]
\PYG{c+c1}{\PYGZsh{} condition is always False becasue bool(0)=False}
\PYG{c+c1}{\PYGZsh{} no output}

\PYG{k}{if} \PYG{l+m+mi}{0}\PYG{p}{:}
    \PYG{n+nb}{print}\PYG{p}{(}\PYG{l+s+s1}{\PYGZsq{}}\PYG{l+s+s1}{Hello}\PYG{l+s+s1}{\PYGZsq{}}\PYG{p}{)}
\end{sphinxVerbatim}

\end{sphinxuseclass}\end{sphinxVerbatimInput}

\end{sphinxuseclass}
\begin{sphinxuseclass}{cell}\begin{sphinxVerbatimInput}

\begin{sphinxuseclass}{cell_input}
\begin{sphinxVerbatim}[commandchars=\\\{\}]
\PYG{c+c1}{\PYGZsh{} condition is always True because bool(\PYGZsq{}NY\PYGZsq{})=True}
\PYG{c+c1}{\PYGZsh{} Hello will always be printed.}

\PYG{k}{if} \PYG{l+s+s1}{\PYGZsq{}}\PYG{l+s+s1}{NY}\PYG{l+s+s1}{\PYGZsq{}}\PYG{p}{:}
    \PYG{n+nb}{print}\PYG{p}{(}\PYG{l+s+s1}{\PYGZsq{}}\PYG{l+s+s1}{Hello}\PYG{l+s+s1}{\PYGZsq{}}\PYG{p}{)}
\end{sphinxVerbatim}

\end{sphinxuseclass}\end{sphinxVerbatimInput}
\begin{sphinxVerbatimOutput}

\begin{sphinxuseclass}{cell_output}
\begin{sphinxVerbatim}[commandchars=\\\{\}]
Hello
\end{sphinxVerbatim}

\end{sphinxuseclass}\end{sphinxVerbatimOutput}

\end{sphinxuseclass}
\begin{sphinxuseclass}{cell}\begin{sphinxVerbatimInput}

\begin{sphinxuseclass}{cell_input}
\begin{sphinxVerbatim}[commandchars=\\\{\}]
\PYG{c+c1}{\PYGZsh{} condition is always False becasue bool(\PYGZsq{}\PYGZsq{})=False}
\PYG{c+c1}{\PYGZsh{} no output}

\PYG{k}{if} \PYG{l+s+s1}{\PYGZsq{}}\PYG{l+s+s1}{\PYGZsq{}}\PYG{p}{:}
    \PYG{n+nb}{print}\PYG{p}{(}\PYG{l+s+s1}{\PYGZsq{}}\PYG{l+s+s1}{Hello}\PYG{l+s+s1}{\PYGZsq{}}\PYG{p}{)}
\end{sphinxVerbatim}

\end{sphinxuseclass}\end{sphinxVerbatimInput}

\end{sphinxuseclass}\begin{itemize}
\item {} 
\sphinxAtStartPar
If the block code of an if statement consists of only one line, it can be written right after the colon (:), keeping the entire if statement in a single line.

\end{itemize}

\begin{sphinxuseclass}{cell}\begin{sphinxVerbatimInput}

\begin{sphinxuseclass}{cell_input}
\begin{sphinxVerbatim}[commandchars=\\\{\}]
\PYG{n}{grade} \PYG{o}{=} \PYG{l+m+mi}{55}
\PYG{k}{if} \PYG{n}{grade} \PYG{o}{\PYGZgt{}} \PYG{l+m+mi}{35}\PYG{p}{:} \PYG{n+nb}{print}\PYG{p}{(}\PYG{l+s+s1}{\PYGZsq{}}\PYG{l+s+s1}{You passed.}\PYG{l+s+s1}{\PYGZsq{}}\PYG{p}{)}  
\end{sphinxVerbatim}

\end{sphinxuseclass}\end{sphinxVerbatimInput}
\begin{sphinxVerbatimOutput}

\begin{sphinxuseclass}{cell_output}
\begin{sphinxVerbatim}[commandchars=\\\{\}]
You passed.
\end{sphinxVerbatim}

\end{sphinxuseclass}\end{sphinxVerbatimOutput}

\end{sphinxuseclass}

\section{if\sphinxhyphen{}else statement}
\label{\detokenize{conditionals:if-else-statement}}
\sphinxAtStartPar
This is for two\sphinxhyphen{}case situations.
\begin{itemize}
\item {} 
\sphinxAtStartPar
If the condition is True, the block code of the if statement will be executed as before.

\item {} 
\sphinxAtStartPar
If the condition is False, the block code of the else statement will be executed.

\item {} 
\sphinxAtStartPar
The \sphinxcode{\sphinxupquote{else}} keyword is used for the second part.

\item {} 
\sphinxAtStartPar
The else part does not have a condition part.

\item {} 
\sphinxAtStartPar
\sphinxstylestrong{IMPORTANT:} The indentation level of if and else must be the same.

\end{itemize}

\sphinxAtStartPar
\sphinxcode{\sphinxupquote{if condition:}}\\
   \sphinxcode{\sphinxupquote{                  }}       \\
   \sphinxcode{\sphinxupquote{BLOCK CODE of IF  }}       \\
   \sphinxcode{\sphinxupquote{                  }}       \\
\sphinxcode{\sphinxupquote{else:}}\\
   \sphinxcode{\sphinxupquote{                  }}     \\
   \sphinxcode{\sphinxupquote{BLOCK CODE of ELSE}}     \\
   \sphinxcode{\sphinxupquote{                  }}     
\begin{itemize}
\item {} 
\sphinxAtStartPar
The cases are as follows:

\end{itemize}
\begin{enumerate}
\sphinxsetlistlabels{\arabic}{enumi}{enumii}{}{.}%
\item {} 
\sphinxAtStartPar
condition is True:

\end{enumerate}

\sphinxAtStartPar
\sphinxcode{\sphinxupquote{if True:}}\\
   \sphinxcode{\sphinxupquote{                  }}       \\
   \sphinxcode{\sphinxupquote{BLOCK CODE of IF  }}                BLOCK CODE of IF STATEMENT will be executed.\\
   \sphinxcode{\sphinxupquote{                  }}       \\
\sphinxcode{\sphinxupquote{else:}}\\
   \sphinxcode{\sphinxupquote{                  }}       \\
   \sphinxcode{\sphinxupquote{BLOCK CODE of ELSE}}                 BLOCK CODE of ELSE STATEMENT will be skipped.\\
   \sphinxcode{\sphinxupquote{                  }}       
\begin{enumerate}
\sphinxsetlistlabels{\arabic}{enumi}{enumii}{}{.}%
\setcounter{enumi}{1}
\item {} 
\sphinxAtStartPar
condition = False:

\end{enumerate}

\sphinxAtStartPar
\sphinxcode{\sphinxupquote{if False:}}\\
   \sphinxcode{\sphinxupquote{                  }}       \\
   \sphinxcode{\sphinxupquote{BLOCK CODE of IF  }}                BLOCK CODE of IF STATEMENT will be skipped.\\
   \sphinxcode{\sphinxupquote{                  }}       \\
\sphinxcode{\sphinxupquote{else:}}\\
   \sphinxcode{\sphinxupquote{                  }}     \\
   \sphinxcode{\sphinxupquote{BLOCK CODE of ELSE}}               BLOCK CODE of ELSE STATEMENT will be executed.\\
   \sphinxcode{\sphinxupquote{                  }}     
\begin{itemize}
\item {} 
\sphinxAtStartPar
In the code below:
\begin{itemize}
\item {} 
\sphinxAtStartPar
\sphinxcode{\sphinxupquote{condition}} is True because 75 > 65 is True, so the block code of the \sphinxcode{\sphinxupquote{if}} statement will be executed.

\item {} 
\sphinxAtStartPar
The block code of the \sphinxcode{\sphinxupquote{else}} statement will be skipped.

\item {} 
\sphinxAtStartPar
The output comes from the print statements of the \sphinxcode{\sphinxupquote{if}} part.

\end{itemize}

\end{itemize}

\begin{sphinxuseclass}{cell}\begin{sphinxVerbatimInput}

\begin{sphinxuseclass}{cell_input}
\begin{sphinxVerbatim}[commandchars=\\\{\}]
\PYG{n}{grade} \PYG{o}{=} \PYG{l+m+mi}{75}

\PYG{k}{if} \PYG{n}{grade} \PYG{o}{\PYGZgt{}} \PYG{l+m+mi}{65}\PYG{p}{:}        
    \PYG{n+nb}{print}\PYG{p}{(}\PYG{l+s+s1}{\PYGZsq{}}\PYG{l+s+s1}{You passed.}\PYG{l+s+s1}{\PYGZsq{}}\PYG{p}{)}
    \PYG{n+nb}{print}\PYG{p}{(}\PYG{l+s+s1}{\PYGZsq{}}\PYG{l+s+s1}{Congrats!}\PYG{l+s+s1}{\PYGZsq{}}\PYG{p}{)}
\PYG{k}{else}\PYG{p}{:}
    \PYG{n+nb}{print}\PYG{p}{(}\PYG{l+s+s1}{\PYGZsq{}}\PYG{l+s+s1}{You failed.}\PYG{l+s+s1}{\PYGZsq{}}\PYG{p}{)}
    \PYG{n+nb}{print}\PYG{p}{(}\PYG{l+s+s1}{\PYGZsq{}}\PYG{l+s+s1}{I am sorry.}\PYG{l+s+s1}{\PYGZsq{}}\PYG{p}{)}
\end{sphinxVerbatim}

\end{sphinxuseclass}\end{sphinxVerbatimInput}
\begin{sphinxVerbatimOutput}

\begin{sphinxuseclass}{cell_output}
\begin{sphinxVerbatim}[commandchars=\\\{\}]
You passed.
Congrats!
\end{sphinxVerbatim}

\end{sphinxuseclass}\end{sphinxVerbatimOutput}

\end{sphinxuseclass}\begin{itemize}
\item {} 
\sphinxAtStartPar
In the code below:
\begin{itemize}
\item {} 
\sphinxAtStartPar
\sphinxcode{\sphinxupquote{condition}} is False because 55>65 is False, so the block code of the \sphinxcode{\sphinxupquote{else}} statement will be executed.

\item {} 
\sphinxAtStartPar
The block code of the \sphinxcode{\sphinxupquote{if}} statement will be skipped.

\item {} 
\sphinxAtStartPar
The output comes from the print staements of the \sphinxcode{\sphinxupquote{else}} part.

\end{itemize}

\end{itemize}

\begin{sphinxuseclass}{cell}\begin{sphinxVerbatimInput}

\begin{sphinxuseclass}{cell_input}
\begin{sphinxVerbatim}[commandchars=\\\{\}]
\PYG{n}{grade} \PYG{o}{=} \PYG{l+m+mi}{55}

\PYG{k}{if} \PYG{n}{grade} \PYG{o}{\PYGZgt{}} \PYG{l+m+mi}{65}\PYG{p}{:}        
    \PYG{n+nb}{print}\PYG{p}{(}\PYG{l+s+s1}{\PYGZsq{}}\PYG{l+s+s1}{You passed.}\PYG{l+s+s1}{\PYGZsq{}}\PYG{p}{)}
    \PYG{n+nb}{print}\PYG{p}{(}\PYG{l+s+s1}{\PYGZsq{}}\PYG{l+s+s1}{Congrats!}\PYG{l+s+s1}{\PYGZsq{}}\PYG{p}{)}
\PYG{k}{else}\PYG{p}{:}
    \PYG{n+nb}{print}\PYG{p}{(}\PYG{l+s+s1}{\PYGZsq{}}\PYG{l+s+s1}{You failed.}\PYG{l+s+s1}{\PYGZsq{}}\PYG{p}{)}
    \PYG{n+nb}{print}\PYG{p}{(}\PYG{l+s+s1}{\PYGZsq{}}\PYG{l+s+s1}{I am sorry.}\PYG{l+s+s1}{\PYGZsq{}}\PYG{p}{)}
\end{sphinxVerbatim}

\end{sphinxuseclass}\end{sphinxVerbatimInput}
\begin{sphinxVerbatimOutput}

\begin{sphinxuseclass}{cell_output}
\begin{sphinxVerbatim}[commandchars=\\\{\}]
You failed.
I am sorry.
\end{sphinxVerbatim}

\end{sphinxuseclass}\end{sphinxVerbatimOutput}

\end{sphinxuseclass}\begin{itemize}
\item {} 
\sphinxAtStartPar
If the block code of an if or else statement consists of only one line, it can be written right after the colon (:), keeping the entire if or else statement in a single line.

\end{itemize}

\begin{sphinxuseclass}{cell}\begin{sphinxVerbatimInput}

\begin{sphinxuseclass}{cell_input}
\begin{sphinxVerbatim}[commandchars=\\\{\}]
\PYG{n}{grade} \PYG{o}{=} \PYG{l+m+mi}{55}

\PYG{k}{if} \PYG{n}{grade} \PYG{o}{\PYGZgt{}} \PYG{l+m+mi}{35}\PYG{p}{:} \PYG{n+nb}{print}\PYG{p}{(}\PYG{l+s+s1}{\PYGZsq{}}\PYG{l+s+s1}{You passed.}\PYG{l+s+s1}{\PYGZsq{}}\PYG{p}{)}  
\PYG{k}{else}\PYG{p}{:} \PYG{n+nb}{print}\PYG{p}{(}\PYG{l+s+s1}{\PYGZsq{}}\PYG{l+s+s1}{You failed.}\PYG{l+s+s1}{\PYGZsq{}}\PYG{p}{)}
\end{sphinxVerbatim}

\end{sphinxuseclass}\end{sphinxVerbatimInput}
\begin{sphinxVerbatimOutput}

\begin{sphinxuseclass}{cell_output}
\begin{sphinxVerbatim}[commandchars=\\\{\}]
You passed.
\end{sphinxVerbatim}

\end{sphinxuseclass}\end{sphinxVerbatimOutput}

\end{sphinxuseclass}

\section{if\sphinxhyphen{}elif\sphinxhyphen{}else statement}
\label{\detokenize{conditionals:if-elif-else-statement}}
\sphinxAtStartPar
This is for three\sphinxhyphen{}case situations.
\begin{itemize}
\item {} 
\sphinxAtStartPar
If there are more cases, then more \sphinxcode{\sphinxupquote{elif}} parts can be added.

\item {} 
\sphinxAtStartPar
If the condition of the \sphinxcode{\sphinxupquote{if}} statement is True, the block code of the \sphinxcode{\sphinxupquote{if}} statement will be executed.

\item {} 
\sphinxAtStartPar
If the condition of the \sphinxcode{\sphinxupquote{if}} statement is False and the condition of the \sphinxcode{\sphinxupquote{elif}} statement is True, the block code of the \sphinxcode{\sphinxupquote{elif}} statement will be executed.

\item {} 
\sphinxAtStartPar
If the conditions of the \sphinxcode{\sphinxupquote{if}} and \sphinxcode{\sphinxupquote{elif}} statements are False, then the block code of the \sphinxcode{\sphinxupquote{else}} statement will be executed.

\item {} 
\sphinxAtStartPar
The \sphinxcode{\sphinxupquote{elif}} keyword is used for the second part.

\item {} 
\sphinxAtStartPar
\sphinxcode{\sphinxupquote{elif}} has a condition part.

\item {} 
\sphinxAtStartPar
\sphinxstylestrong{IMPORTANT:} The indentation level of \sphinxcode{\sphinxupquote{if}}, \sphinxcode{\sphinxupquote{elif}}, and \sphinxcode{\sphinxupquote{else}} must be the same.

\item {} 
\sphinxAtStartPar
The structure of an if\sphinxhyphen{}elif\sphinxhyphen{}else statement is as follows:

\end{itemize}

\sphinxAtStartPar
\sphinxcode{\sphinxupquote{if condition of IF:          }}\\
   \sphinxcode{\sphinxupquote{                   }}       \\
   \sphinxcode{\sphinxupquote{BLOCK CODE of  IF  }}       \\
   \sphinxcode{\sphinxupquote{                   }}       \\
\sphinxcode{\sphinxupquote{elif condition of ELIF:      }}\\
   \sphinxcode{\sphinxupquote{                   }}       \\
   \sphinxcode{\sphinxupquote{BLOCK CODE of ELIF }}      \\
   \sphinxcode{\sphinxupquote{                   }}      \\
\sphinxcode{\sphinxupquote{else:                       }}\\
   \sphinxcode{\sphinxupquote{                   }}     \\
   \sphinxcode{\sphinxupquote{BLOCK CODE of ELSE }}     \\
   \sphinxcode{\sphinxupquote{                   }}     
\begin{itemize}
\item {} 
\sphinxAtStartPar
The cases are as follows:

\end{itemize}
\begin{enumerate}
\sphinxsetlistlabels{\arabic}{enumi}{enumii}{}{.}%
\item {} 
\sphinxAtStartPar
condition of \sphinxcode{\sphinxupquote{if}} statement is True:

\end{enumerate}

\sphinxAtStartPar
\sphinxcode{\sphinxupquote{if True:}}\\
   \sphinxcode{\sphinxupquote{                   }}       \\
   \sphinxcode{\sphinxupquote{BLOCK CODE of  IF  }}                 BLOCK CODE of IF STATEMENT will be executed.\\
   \sphinxcode{\sphinxupquote{                   }}       \\
\sphinxcode{\sphinxupquote{elif condition of ELIF:}}\\
   \sphinxcode{\sphinxupquote{                   }}       \\
   \sphinxcode{\sphinxupquote{BLOCK CODE of ELIF }}                  BLOCK CODE of ELIF STATEMENT will be skipped.\\
   \sphinxcode{\sphinxupquote{                   }}      \\
\sphinxcode{\sphinxupquote{else:}}\\
   \sphinxcode{\sphinxupquote{                   }}     \\
   \sphinxcode{\sphinxupquote{BLOCK CODE of ELSE }}                  BLOCK CODE of ELSE STATEMENT will be skipped.\\
   \sphinxcode{\sphinxupquote{                   }}     
\begin{enumerate}
\sphinxsetlistlabels{\arabic}{enumi}{enumii}{}{.}%
\setcounter{enumi}{1}
\item {} 
\sphinxAtStartPar
condition of \sphinxcode{\sphinxupquote{if}} statement is False and condition of \sphinxcode{\sphinxupquote{elif}} statement is True:

\end{enumerate}

\sphinxAtStartPar
\sphinxcode{\sphinxupquote{if False:}}\\
   \sphinxcode{\sphinxupquote{                  }}       \\
   \sphinxcode{\sphinxupquote{BLOCK CODE of  IF }}                 BLOCK CODE of IF STATEMENT will be skipped.\\
   \sphinxcode{\sphinxupquote{                  }}       \\
\sphinxcode{\sphinxupquote{elif True:}}\\
   \sphinxcode{\sphinxupquote{                  }}       \\
   \sphinxcode{\sphinxupquote{BLOCK CODE of ELIF}}                  BLOCK CODE of ELIF STATEMENT will be executed.\\
   \sphinxcode{\sphinxupquote{                  }}      \\
\sphinxcode{\sphinxupquote{else:}}\\
   \sphinxcode{\sphinxupquote{                  }}     \\
   \sphinxcode{\sphinxupquote{BLOCK CODE of ELSE}}                  BLOCK CODE of ELSE STATEMENT will be skipped.\\
   \sphinxcode{\sphinxupquote{                  }}     
\begin{enumerate}
\sphinxsetlistlabels{\arabic}{enumi}{enumii}{}{.}%
\setcounter{enumi}{2}
\item {} 
\sphinxAtStartPar
conditions of \sphinxcode{\sphinxupquote{if}} and \sphinxcode{\sphinxupquote{elif}} statements are both False:

\end{enumerate}

\sphinxAtStartPar
\sphinxcode{\sphinxupquote{if False:}}\\
   \sphinxcode{\sphinxupquote{                  }}       \\
   \sphinxcode{\sphinxupquote{BLOCK CODE of  IF }}                 BLOCK CODE of IF STATEMENT will be skipped.\\
   \sphinxcode{\sphinxupquote{                  }}       \\
\sphinxcode{\sphinxupquote{elif False:}}\\
   \sphinxcode{\sphinxupquote{                  }}       \\
   \sphinxcode{\sphinxupquote{BLOCK CODE of ELIF}}                  BLOCK CODE of ELIF STATEMENT will be skipped.\\
   \sphinxcode{\sphinxupquote{                  }}      \\
\sphinxcode{\sphinxupquote{else:}}\\
   \sphinxcode{\sphinxupquote{                  }}     \\
   \sphinxcode{\sphinxupquote{BLOCK CODE of ELSE}}                  BLOCK CODE of ELSE STATEMENT will be executed.\\
   \sphinxcode{\sphinxupquote{                  }}     
\begin{itemize}
\item {} 
\sphinxAtStartPar
In the code below:
\begin{itemize}
\item {} 
\sphinxAtStartPar
\sphinxcode{\sphinxupquote{condition}} of the if statement is True because 75>65 is True, so the block code of the if statement will be executed.

\item {} 
\sphinxAtStartPar
The block code of the elif and else statements will be skipped.

\item {} 
\sphinxAtStartPar
The output comes from the print statements of the if part.

\end{itemize}

\end{itemize}

\begin{sphinxuseclass}{cell}\begin{sphinxVerbatimInput}

\begin{sphinxuseclass}{cell_input}
\begin{sphinxVerbatim}[commandchars=\\\{\}]
\PYG{n}{grade} \PYG{o}{=} \PYG{l+m+mi}{75}

\PYG{k}{if} \PYG{n}{grade} \PYG{o}{\PYGZgt{}} \PYG{l+m+mi}{65}\PYG{p}{:}        
    \PYG{n+nb}{print}\PYG{p}{(}\PYG{l+s+s1}{\PYGZsq{}}\PYG{l+s+s1}{You passed.}\PYG{l+s+s1}{\PYGZsq{}}\PYG{p}{)}
    \PYG{n+nb}{print}\PYG{p}{(}\PYG{l+s+s1}{\PYGZsq{}}\PYG{l+s+s1}{Congrats!}\PYG{l+s+s1}{\PYGZsq{}}\PYG{p}{)}
\PYG{k}{elif} \PYG{n}{grade} \PYG{o}{\PYGZgt{}} \PYG{l+m+mi}{55}\PYG{p}{:}
    \PYG{n+nb}{print}\PYG{p}{(}\PYG{l+s+s1}{\PYGZsq{}}\PYG{l+s+s1}{You could not pass!}\PYG{l+s+s1}{\PYGZsq{}}\PYG{p}{)}
    \PYG{n+nb}{print}\PYG{p}{(}\PYG{l+s+s1}{\PYGZsq{}}\PYG{l+s+s1}{You can take the test one more time.}\PYG{l+s+s1}{\PYGZsq{}}\PYG{p}{)}
\PYG{k}{else}\PYG{p}{:}
    \PYG{n+nb}{print}\PYG{p}{(}\PYG{l+s+s1}{\PYGZsq{}}\PYG{l+s+s1}{You failed.}\PYG{l+s+s1}{\PYGZsq{}}\PYG{p}{)}
    \PYG{n+nb}{print}\PYG{p}{(}\PYG{l+s+s1}{\PYGZsq{}}\PYG{l+s+s1}{I am sorry.}\PYG{l+s+s1}{\PYGZsq{}}\PYG{p}{)}
\end{sphinxVerbatim}

\end{sphinxuseclass}\end{sphinxVerbatimInput}
\begin{sphinxVerbatimOutput}

\begin{sphinxuseclass}{cell_output}
\begin{sphinxVerbatim}[commandchars=\\\{\}]
You passed.
Congrats!
\end{sphinxVerbatim}

\end{sphinxuseclass}\end{sphinxVerbatimOutput}

\end{sphinxuseclass}\begin{itemize}
\item {} 
\sphinxAtStartPar
In the code below:
\begin{itemize}
\item {} 
\sphinxAtStartPar
\sphinxcode{\sphinxupquote{condition}} of the if statement is False because 60>65 is False

\item {} 
\sphinxAtStartPar
\sphinxcode{\sphinxupquote{condition}} of the elif statement is True because 60>55 is True

\item {} 
\sphinxAtStartPar
So the block code of the elif statement will be executed.

\item {} 
\sphinxAtStartPar
The block code of the if and else statements will be skipped.

\item {} 
\sphinxAtStartPar
The output comes from the print statements of the elif part.

\end{itemize}

\end{itemize}

\begin{sphinxuseclass}{cell}\begin{sphinxVerbatimInput}

\begin{sphinxuseclass}{cell_input}
\begin{sphinxVerbatim}[commandchars=\\\{\}]
\PYG{n}{grade} \PYG{o}{=} \PYG{l+m+mi}{60}

\PYG{k}{if} \PYG{n}{grade} \PYG{o}{\PYGZgt{}} \PYG{l+m+mi}{65}\PYG{p}{:}        
    \PYG{n+nb}{print}\PYG{p}{(}\PYG{l+s+s1}{\PYGZsq{}}\PYG{l+s+s1}{You passed.}\PYG{l+s+s1}{\PYGZsq{}}\PYG{p}{)}
    \PYG{n+nb}{print}\PYG{p}{(}\PYG{l+s+s1}{\PYGZsq{}}\PYG{l+s+s1}{Congrats!}\PYG{l+s+s1}{\PYGZsq{}}\PYG{p}{)}
\PYG{k}{elif} \PYG{n}{grade} \PYG{o}{\PYGZgt{}} \PYG{l+m+mi}{55}\PYG{p}{:}
    \PYG{n+nb}{print}\PYG{p}{(}\PYG{l+s+s1}{\PYGZsq{}}\PYG{l+s+s1}{You could not pass!}\PYG{l+s+s1}{\PYGZsq{}}\PYG{p}{)}
    \PYG{n+nb}{print}\PYG{p}{(}\PYG{l+s+s1}{\PYGZsq{}}\PYG{l+s+s1}{You can take the test one more time.}\PYG{l+s+s1}{\PYGZsq{}}\PYG{p}{)}
\PYG{k}{else}\PYG{p}{:}
    \PYG{n+nb}{print}\PYG{p}{(}\PYG{l+s+s1}{\PYGZsq{}}\PYG{l+s+s1}{You failed.}\PYG{l+s+s1}{\PYGZsq{}}\PYG{p}{)}
    \PYG{n+nb}{print}\PYG{p}{(}\PYG{l+s+s1}{\PYGZsq{}}\PYG{l+s+s1}{I am sorry.}\PYG{l+s+s1}{\PYGZsq{}}\PYG{p}{)}
\end{sphinxVerbatim}

\end{sphinxuseclass}\end{sphinxVerbatimInput}
\begin{sphinxVerbatimOutput}

\begin{sphinxuseclass}{cell_output}
\begin{sphinxVerbatim}[commandchars=\\\{\}]
You could not pass!
You can take the test one more time.
\end{sphinxVerbatim}

\end{sphinxuseclass}\end{sphinxVerbatimOutput}

\end{sphinxuseclass}\begin{itemize}
\item {} 
\sphinxAtStartPar
In the code below:
\begin{itemize}
\item {} 
\sphinxAtStartPar
\sphinxcode{\sphinxupquote{condition}} of the if statement is False because 40>65 is False

\item {} 
\sphinxAtStartPar
\sphinxcode{\sphinxupquote{condition}} of the elif statement is False because 40>55 is False

\item {} 
\sphinxAtStartPar
So the block code of the else statement will be executed.

\item {} 
\sphinxAtStartPar
The block code of the if and elif statements will be skipped.

\item {} 
\sphinxAtStartPar
The output comes from the print statements of the else part.

\end{itemize}

\end{itemize}

\begin{sphinxuseclass}{cell}\begin{sphinxVerbatimInput}

\begin{sphinxuseclass}{cell_input}
\begin{sphinxVerbatim}[commandchars=\\\{\}]
\PYG{n}{grade} \PYG{o}{=} \PYG{l+m+mi}{40}

\PYG{k}{if} \PYG{n}{grade} \PYG{o}{\PYGZgt{}} \PYG{l+m+mi}{65}\PYG{p}{:}        
    \PYG{n+nb}{print}\PYG{p}{(}\PYG{l+s+s1}{\PYGZsq{}}\PYG{l+s+s1}{You passed.}\PYG{l+s+s1}{\PYGZsq{}}\PYG{p}{)}
    \PYG{n+nb}{print}\PYG{p}{(}\PYG{l+s+s1}{\PYGZsq{}}\PYG{l+s+s1}{Congrats!}\PYG{l+s+s1}{\PYGZsq{}}\PYG{p}{)}
\PYG{k}{elif} \PYG{n}{grade} \PYG{o}{\PYGZgt{}} \PYG{l+m+mi}{55}\PYG{p}{:}
    \PYG{n+nb}{print}\PYG{p}{(}\PYG{l+s+s1}{\PYGZsq{}}\PYG{l+s+s1}{You could not pass!}\PYG{l+s+s1}{\PYGZsq{}}\PYG{p}{)}
    \PYG{n+nb}{print}\PYG{p}{(}\PYG{l+s+s1}{\PYGZsq{}}\PYG{l+s+s1}{You can take the test one more time.}\PYG{l+s+s1}{\PYGZsq{}}\PYG{p}{)}
\PYG{k}{else}\PYG{p}{:}
    \PYG{n+nb}{print}\PYG{p}{(}\PYG{l+s+s1}{\PYGZsq{}}\PYG{l+s+s1}{You failed.}\PYG{l+s+s1}{\PYGZsq{}}\PYG{p}{)}
    \PYG{n+nb}{print}\PYG{p}{(}\PYG{l+s+s1}{\PYGZsq{}}\PYG{l+s+s1}{I am sorry.}\PYG{l+s+s1}{\PYGZsq{}}\PYG{p}{)}
\end{sphinxVerbatim}

\end{sphinxuseclass}\end{sphinxVerbatimInput}
\begin{sphinxVerbatimOutput}

\begin{sphinxuseclass}{cell_output}
\begin{sphinxVerbatim}[commandchars=\\\{\}]
You failed.
I am sorry.
\end{sphinxVerbatim}

\end{sphinxuseclass}\end{sphinxVerbatimOutput}

\end{sphinxuseclass}

\section{Nested if statements}
\label{\detokenize{conditionals:nested-if-statements}}\begin{itemize}
\item {} 
\sphinxAtStartPar
It is possible to have an if, if\sphinxhyphen{}else, or if\sphinxhyphen{}elif\sphinxhyphen{}else statement within a block code for if, elif, or else.

\end{itemize}

\begin{sphinxuseclass}{cell}\begin{sphinxVerbatimInput}

\begin{sphinxuseclass}{cell_input}
\begin{sphinxVerbatim}[commandchars=\\\{\}]
\PYG{n}{age} \PYG{o}{=} \PYG{l+m+mi}{10}
\PYG{n}{weight} \PYG{o}{=} \PYG{l+m+mi}{45}

\PYG{k}{if} \PYG{n}{age} \PYG{o}{\PYGZgt{}} \PYG{l+m+mi}{5}\PYG{p}{:}                     \PYG{c+c1}{\PYGZsh{} True: execute the block code within this `if` statement, which includes an `if\PYGZhy{}else` statement.}
    \PYG{k}{if} \PYG{n}{weight}\PYG{o}{\PYGZgt{}}\PYG{l+m+mi}{50}\PYG{p}{:}               \PYG{c+c1}{\PYGZsh{} False: skipped}
        \PYG{n+nb}{print}\PYG{p}{(}\PYG{l+s+s1}{\PYGZsq{}}\PYG{l+s+s1}{A}\PYG{l+s+s1}{\PYGZsq{}}\PYG{p}{)}
    \PYG{k}{else}\PYG{p}{:}                       
        \PYG{n+nb}{print}\PYG{p}{(}\PYG{l+s+s1}{\PYGZsq{}}\PYG{l+s+s1}{B}\PYG{l+s+s1}{\PYGZsq{}}\PYG{p}{)}              \PYG{c+c1}{\PYGZsh{} executed}
\end{sphinxVerbatim}

\end{sphinxuseclass}\end{sphinxVerbatimInput}
\begin{sphinxVerbatimOutput}

\begin{sphinxuseclass}{cell_output}
\begin{sphinxVerbatim}[commandchars=\\\{\}]
B
\end{sphinxVerbatim}

\end{sphinxuseclass}\end{sphinxVerbatimOutput}

\end{sphinxuseclass}
\begin{sphinxuseclass}{cell}\begin{sphinxVerbatimInput}

\begin{sphinxuseclass}{cell_input}
\begin{sphinxVerbatim}[commandchars=\\\{\}]
\PYG{n}{age} \PYG{o}{=} \PYG{l+m+mi}{10}
\PYG{n}{weight} \PYG{o}{=} \PYG{l+m+mi}{55}

\PYG{k}{if} \PYG{n}{age} \PYG{o}{\PYGZgt{}} \PYG{l+m+mi}{5}\PYG{p}{:}                     \PYG{c+c1}{\PYGZsh{} True: execute the block code within this `if` statement which consists of an `if\PYGZhy{}else` statement}
    \PYG{k}{if} \PYG{n}{weight}\PYG{o}{\PYGZgt{}}\PYG{l+m+mi}{50}\PYG{p}{:}               \PYG{c+c1}{\PYGZsh{} True: executed}
        \PYG{n+nb}{print}\PYG{p}{(}\PYG{l+s+s1}{\PYGZsq{}}\PYG{l+s+s1}{A}\PYG{l+s+s1}{\PYGZsq{}}\PYG{p}{)}
    \PYG{k}{else}\PYG{p}{:}                       
        \PYG{n+nb}{print}\PYG{p}{(}\PYG{l+s+s1}{\PYGZsq{}}\PYG{l+s+s1}{B}\PYG{l+s+s1}{\PYGZsq{}}\PYG{p}{)}              \PYG{c+c1}{\PYGZsh{} skipped}
\end{sphinxVerbatim}

\end{sphinxuseclass}\end{sphinxVerbatimInput}
\begin{sphinxVerbatimOutput}

\begin{sphinxuseclass}{cell_output}
\begin{sphinxVerbatim}[commandchars=\\\{\}]
A
\end{sphinxVerbatim}

\end{sphinxuseclass}\end{sphinxVerbatimOutput}

\end{sphinxuseclass}
\begin{sphinxuseclass}{cell}\begin{sphinxVerbatimInput}

\begin{sphinxuseclass}{cell_input}
\begin{sphinxVerbatim}[commandchars=\\\{\}]
\PYG{n}{age} \PYG{o}{=} \PYG{l+m+mi}{3}
\PYG{n}{weight} \PYG{o}{=} \PYG{l+m+mi}{45}

\PYG{k}{if} \PYG{n}{age} \PYG{o}{\PYGZgt{}} \PYG{l+m+mi}{5}\PYG{p}{:}                     \PYG{c+c1}{\PYGZsh{} False: skip the block code within this `if` statement which consists of an `if\PYGZhy{}else` statement}
    \PYG{k}{if} \PYG{n}{weight}\PYG{o}{\PYGZgt{}}\PYG{l+m+mi}{50}\PYG{p}{:}               
        \PYG{n+nb}{print}\PYG{p}{(}\PYG{l+s+s1}{\PYGZsq{}}\PYG{l+s+s1}{A}\PYG{l+s+s1}{\PYGZsq{}}\PYG{p}{)}
    \PYG{k}{else}\PYG{p}{:}                       
        \PYG{n+nb}{print}\PYG{p}{(}\PYG{l+s+s1}{\PYGZsq{}}\PYG{l+s+s1}{B}\PYG{l+s+s1}{\PYGZsq{}}\PYG{p}{)}              
\PYG{k}{else}\PYG{p}{:}                           \PYG{c+c1}{\PYGZsh{} this \PYGZsq{}else\PYGZsq{} statement corresponds to the first \PYGZsq{}if\PYGZsq{} statement.}
    \PYG{n+nb}{print}\PYG{p}{(}\PYG{l+s+s1}{\PYGZsq{}}\PYG{l+s+s1}{C}\PYG{l+s+s1}{\PYGZsq{}}\PYG{p}{)}                  \PYG{c+c1}{\PYGZsh{} executed}
\end{sphinxVerbatim}

\end{sphinxuseclass}\end{sphinxVerbatimInput}
\begin{sphinxVerbatimOutput}

\begin{sphinxuseclass}{cell_output}
\begin{sphinxVerbatim}[commandchars=\\\{\}]
C
\end{sphinxVerbatim}

\end{sphinxuseclass}\end{sphinxVerbatimOutput}

\end{sphinxuseclass}

\section{Boolean Operators}
\label{\detokenize{conditionals:boolean-operators}}
\sphinxAtStartPar
Boolean operators include  \sphinxcode{\sphinxupquote{and}}, \sphinxcode{\sphinxupquote{or}}, and \sphinxcode{\sphinxupquote{not}}.
\begin{itemize}
\item {} 
\sphinxAtStartPar
\sphinxcode{\sphinxupquote{and}}, and \sphinxcode{\sphinxupquote{or}} are used to construct more complex boolean expressions.

\item {} 
\sphinxAtStartPar
\sphinxcode{\sphinxupquote{not}} is the negation operator.

\item {} 
\sphinxAtStartPar
\sphinxcode{\sphinxupquote{\&}},\sphinxcode{\sphinxupquote{|}} can be used instead of \sphinxcode{\sphinxupquote{and}}, \sphinxcode{\sphinxupquote{or}} respectively.

\end{itemize}

\sphinxAtStartPar
They are also called logical operators and works as follows:
\begin{itemize}
\item {} 
\sphinxAtStartPar
The and operator returns True if both operands are True, otherwise, it returns False.

\item {} 
\sphinxAtStartPar
The or operator returns True if at least one of the operands is True. It returns False only if both operands are False.

\item {} 
\sphinxAtStartPar
The not operator returns the opposite boolean value of the operand. If the operand is True, not returns False, and vice versa.

\end{itemize}

\sphinxAtStartPar
These operators, also called logical operators, work as follows:
\begin{itemize}
\item {} 
\sphinxAtStartPar
The \sphinxcode{\sphinxupquote{and}} operator returns True if both operands are True; otherwise, it returns False.

\item {} 
\sphinxAtStartPar
The \sphinxcode{\sphinxupquote{or}}  operator returns True if at least one of the operands is True; it returns False only if both operands are False.

\item {} 
\sphinxAtStartPar
The ‘not’ operator returns the opposite boolean value of the operand. If the operand is True, ‘not’ returns False, and vice versa.”

\end{itemize}


\begin{savenotes}\sphinxattablestart
\centering
\begin{tabulary}{\linewidth}[t]{|T|T|T|T|T|}
\hline
\sphinxstyletheadfamily 
\sphinxAtStartPar
Value
&\sphinxstyletheadfamily 
\sphinxAtStartPar
Operator
&\sphinxstyletheadfamily 
\sphinxAtStartPar
Value
&\sphinxstyletheadfamily 
\sphinxAtStartPar
=
&\sphinxstyletheadfamily 
\sphinxAtStartPar
Result
\\
\hline
\sphinxAtStartPar
True
&
\sphinxAtStartPar
and
&
\sphinxAtStartPar
True
&
\sphinxAtStartPar
=
&
\sphinxAtStartPar
True
\\
\hline
\sphinxAtStartPar
True
&
\sphinxAtStartPar
and
&
\sphinxAtStartPar
False
&
\sphinxAtStartPar
=
&
\sphinxAtStartPar
False
\\
\hline
\sphinxAtStartPar
False
&
\sphinxAtStartPar
and
&
\sphinxAtStartPar
True
&
\sphinxAtStartPar
=
&
\sphinxAtStartPar
False
\\
\hline
\sphinxAtStartPar
False
&
\sphinxAtStartPar
and
&
\sphinxAtStartPar
False
&
\sphinxAtStartPar
=
&
\sphinxAtStartPar
False
\\
\hline
\end{tabulary}
\par
\sphinxattableend\end{savenotes}


\begin{savenotes}\sphinxattablestart
\centering
\begin{tabulary}{\linewidth}[t]{|T|T|T|T|T|}
\hline
\sphinxstyletheadfamily 
\sphinxAtStartPar
Value
&\sphinxstyletheadfamily 
\sphinxAtStartPar
Operator
&\sphinxstyletheadfamily 
\sphinxAtStartPar
Value
&\sphinxstyletheadfamily 
\sphinxAtStartPar
=
&\sphinxstyletheadfamily 
\sphinxAtStartPar
Result
\\
\hline
\sphinxAtStartPar
True
&
\sphinxAtStartPar
or
&
\sphinxAtStartPar
True
&
\sphinxAtStartPar
=
&
\sphinxAtStartPar
True
\\
\hline
\sphinxAtStartPar
True
&
\sphinxAtStartPar
or
&
\sphinxAtStartPar
False
&
\sphinxAtStartPar
=
&
\sphinxAtStartPar
True
\\
\hline
\sphinxAtStartPar
False
&
\sphinxAtStartPar
or
&
\sphinxAtStartPar
True
&
\sphinxAtStartPar
=
&
\sphinxAtStartPar
True
\\
\hline
\sphinxAtStartPar
False
&
\sphinxAtStartPar
or
&
\sphinxAtStartPar
False
&
\sphinxAtStartPar
=
&
\sphinxAtStartPar
False
\\
\hline
\end{tabulary}
\par
\sphinxattableend\end{savenotes}
\begin{itemize}
\item {} 
\sphinxAtStartPar
not True  = False

\item {} 
\sphinxAtStartPar
not False = True

\end{itemize}

\begin{sphinxuseclass}{cell}\begin{sphinxVerbatimInput}

\begin{sphinxuseclass}{cell_input}
\begin{sphinxVerbatim}[commandchars=\\\{\}]
\PYG{n+nb}{print}\PYG{p}{(}\PYG{k+kc}{True} \PYG{o+ow}{and} \PYG{k+kc}{False}\PYG{p}{)}
\end{sphinxVerbatim}

\end{sphinxuseclass}\end{sphinxVerbatimInput}
\begin{sphinxVerbatimOutput}

\begin{sphinxuseclass}{cell_output}
\begin{sphinxVerbatim}[commandchars=\\\{\}]
False
\end{sphinxVerbatim}

\end{sphinxuseclass}\end{sphinxVerbatimOutput}

\end{sphinxuseclass}
\begin{sphinxuseclass}{cell}\begin{sphinxVerbatimInput}

\begin{sphinxuseclass}{cell_input}
\begin{sphinxVerbatim}[commandchars=\\\{\}]
\PYG{n+nb}{print}\PYG{p}{(}\PYG{k+kc}{True} \PYG{o}{\PYGZam{}} \PYG{k+kc}{False}\PYG{p}{)}   \PYG{c+c1}{\PYGZsh{} and}
\end{sphinxVerbatim}

\end{sphinxuseclass}\end{sphinxVerbatimInput}
\begin{sphinxVerbatimOutput}

\begin{sphinxuseclass}{cell_output}
\begin{sphinxVerbatim}[commandchars=\\\{\}]
False
\end{sphinxVerbatim}

\end{sphinxuseclass}\end{sphinxVerbatimOutput}

\end{sphinxuseclass}
\begin{sphinxuseclass}{cell}\begin{sphinxVerbatimInput}

\begin{sphinxuseclass}{cell_input}
\begin{sphinxVerbatim}[commandchars=\\\{\}]
\PYG{n+nb}{print}\PYG{p}{(}\PYG{k+kc}{True} \PYG{o+ow}{or} \PYG{k+kc}{False}\PYG{p}{)}
\end{sphinxVerbatim}

\end{sphinxuseclass}\end{sphinxVerbatimInput}
\begin{sphinxVerbatimOutput}

\begin{sphinxuseclass}{cell_output}
\begin{sphinxVerbatim}[commandchars=\\\{\}]
True
\end{sphinxVerbatim}

\end{sphinxuseclass}\end{sphinxVerbatimOutput}

\end{sphinxuseclass}
\begin{sphinxuseclass}{cell}\begin{sphinxVerbatimInput}

\begin{sphinxuseclass}{cell_input}
\begin{sphinxVerbatim}[commandchars=\\\{\}]
\PYG{n+nb}{print}\PYG{p}{(}\PYG{k+kc}{True} \PYG{o}{|} \PYG{k+kc}{False}\PYG{p}{)}  \PYG{c+c1}{\PYGZsh{} or}
\end{sphinxVerbatim}

\end{sphinxuseclass}\end{sphinxVerbatimInput}
\begin{sphinxVerbatimOutput}

\begin{sphinxuseclass}{cell_output}
\begin{sphinxVerbatim}[commandchars=\\\{\}]
True
\end{sphinxVerbatim}

\end{sphinxuseclass}\end{sphinxVerbatimOutput}

\end{sphinxuseclass}
\begin{sphinxuseclass}{cell}\begin{sphinxVerbatimInput}

\begin{sphinxuseclass}{cell_input}
\begin{sphinxVerbatim}[commandchars=\\\{\}]
\PYG{n+nb}{print}\PYG{p}{(}\PYG{o+ow}{not} \PYG{k+kc}{True}\PYG{p}{)}
\end{sphinxVerbatim}

\end{sphinxuseclass}\end{sphinxVerbatimInput}
\begin{sphinxVerbatimOutput}

\begin{sphinxuseclass}{cell_output}
\begin{sphinxVerbatim}[commandchars=\\\{\}]
False
\end{sphinxVerbatim}

\end{sphinxuseclass}\end{sphinxVerbatimOutput}

\end{sphinxuseclass}
\begin{sphinxuseclass}{cell}\begin{sphinxVerbatimInput}

\begin{sphinxuseclass}{cell_input}
\begin{sphinxVerbatim}[commandchars=\\\{\}]
\PYG{n+nb}{print}\PYG{p}{(}\PYG{o+ow}{not} \PYG{k+kc}{False}\PYG{p}{)}
\end{sphinxVerbatim}

\end{sphinxuseclass}\end{sphinxVerbatimInput}
\begin{sphinxVerbatimOutput}

\begin{sphinxuseclass}{cell_output}
\begin{sphinxVerbatim}[commandchars=\\\{\}]
True
\end{sphinxVerbatim}

\end{sphinxuseclass}\end{sphinxVerbatimOutput}

\end{sphinxuseclass}
\begin{sphinxuseclass}{cell}\begin{sphinxVerbatimInput}

\begin{sphinxuseclass}{cell_input}
\begin{sphinxVerbatim}[commandchars=\\\{\}]
\PYG{c+c1}{\PYGZsh{} True and True = True}
\PYG{n+nb}{print}\PYG{p}{(} \PYG{p}{(}\PYG{l+m+mi}{3} \PYG{o}{\PYGZgt{}} \PYG{l+m+mi}{1}\PYG{p}{)} \PYG{o}{\PYGZam{}} \PYG{p}{(}\PYG{l+m+mi}{10} \PYG{o}{\PYGZgt{}} \PYG{l+m+mi}{8}\PYG{p}{)} \PYG{p}{)}
\end{sphinxVerbatim}

\end{sphinxuseclass}\end{sphinxVerbatimInput}
\begin{sphinxVerbatimOutput}

\begin{sphinxuseclass}{cell_output}
\begin{sphinxVerbatim}[commandchars=\\\{\}]
True
\end{sphinxVerbatim}

\end{sphinxuseclass}\end{sphinxVerbatimOutput}

\end{sphinxuseclass}
\begin{sphinxuseclass}{cell}\begin{sphinxVerbatimInput}

\begin{sphinxuseclass}{cell_input}
\begin{sphinxVerbatim}[commandchars=\\\{\}]
\PYG{c+c1}{\PYGZsh{} False or True = True}
\PYG{n+nb}{print}\PYG{p}{(} \PYG{p}{(}\PYG{l+m+mi}{7} \PYG{o}{==} \PYG{l+m+mi}{8}\PYG{p}{)} \PYG{o}{|} \PYG{p}{(}\PYG{l+m+mi}{10} \PYG{o}{\PYGZgt{}} \PYG{l+m+mi}{8}\PYG{p}{)} \PYG{p}{)}
\end{sphinxVerbatim}

\end{sphinxuseclass}\end{sphinxVerbatimInput}
\begin{sphinxVerbatimOutput}

\begin{sphinxuseclass}{cell_output}
\begin{sphinxVerbatim}[commandchars=\\\{\}]
True
\end{sphinxVerbatim}

\end{sphinxuseclass}\end{sphinxVerbatimOutput}

\end{sphinxuseclass}
\begin{sphinxuseclass}{cell}\begin{sphinxVerbatimInput}

\begin{sphinxuseclass}{cell_input}
\begin{sphinxVerbatim}[commandchars=\\\{\}]
\PYG{c+c1}{\PYGZsh{} not True = False}
\PYG{n+nb}{print}\PYG{p}{(} \PYG{o+ow}{not} \PYG{p}{(}\PYG{l+s+s1}{\PYGZsq{}}\PYG{l+s+s1}{arm}\PYG{l+s+s1}{\PYGZsq{}}\PYG{o}{\PYGZlt{}}\PYG{l+s+s1}{\PYGZsq{}}\PYG{l+s+s1}{kite}\PYG{l+s+s1}{\PYGZsq{}}\PYG{p}{)}\PYG{p}{)}
\end{sphinxVerbatim}

\end{sphinxuseclass}\end{sphinxVerbatimInput}
\begin{sphinxVerbatimOutput}

\begin{sphinxuseclass}{cell_output}
\begin{sphinxVerbatim}[commandchars=\\\{\}]
False
\end{sphinxVerbatim}

\end{sphinxuseclass}\end{sphinxVerbatimOutput}

\end{sphinxuseclass}
\begin{sphinxuseclass}{cell}\begin{sphinxVerbatimInput}

\begin{sphinxuseclass}{cell_input}
\begin{sphinxVerbatim}[commandchars=\\\{\}]
\PYG{c+c1}{\PYGZsh{} not False = True}
\PYG{n+nb}{print}\PYG{p}{(} \PYG{o+ow}{not} \PYG{p}{(} \PYG{l+s+s1}{\PYGZsq{}}\PYG{l+s+s1}{a}\PYG{l+s+s1}{\PYGZsq{}} \PYG{o+ow}{in} \PYG{l+s+s1}{\PYGZsq{}}\PYG{l+s+s1}{Apple}\PYG{l+s+s1}{\PYGZsq{}} \PYG{p}{)}\PYG{p}{)}  \PYG{c+c1}{\PYGZsh{} a i snot in Apple}
\end{sphinxVerbatim}

\end{sphinxuseclass}\end{sphinxVerbatimInput}
\begin{sphinxVerbatimOutput}

\begin{sphinxuseclass}{cell_output}
\begin{sphinxVerbatim}[commandchars=\\\{\}]
True
\end{sphinxVerbatim}

\end{sphinxuseclass}\end{sphinxVerbatimOutput}

\end{sphinxuseclass}
\sphinxAtStartPar
\sphinxstylestrong{Example\sphinxhyphen{}1}
\begin{itemize}
\item {} 
\sphinxAtStartPar
If the (weather is nice) or (I have \$5), then I will buy an ice cream.
\begin{itemize}
\item {} 
\sphinxAtStartPar
If either of the conditions is True, I will buy an ice cream.

\item {} 
\sphinxAtStartPar
If both of the conditions are False, I will not buy an ice cream.

\end{itemize}

\item {} 
\sphinxAtStartPar
So, we have the following scenarios based on weather conditions and the amount of money available:
\begin{itemize}
\item {} 
\sphinxAtStartPar
(weather is nice) = True   or  (I have \$5) = True   ——> buy an ice cream

\item {} 
\sphinxAtStartPar
(weather is nice) = False  or  (I have \$5) = True   ——> buy an ice cream

\item {} 
\sphinxAtStartPar
(weather is nice) = True   or  (I have \$5) = False  ——> buy an ice cream

\item {} 
\sphinxAtStartPar
(weather is nice) = False  or  (I have \$5) = False  ——> do NOT buy an ice cream

\end{itemize}

\end{itemize}

\sphinxAtStartPar
\sphinxstylestrong{Example\sphinxhyphen{}2}
\begin{itemize}
\item {} 
\sphinxAtStartPar
If the (weather is nice) and (I have \$5), then I will buy an ice cream.
\begin{itemize}
\item {} 
\sphinxAtStartPar
Ii both of the condtions are True, I will buy an ice cream.

\item {} 
\sphinxAtStartPar
If either of the conditions is False, then I will not buy an ice cream.

\end{itemize}

\item {} 
\sphinxAtStartPar
So we have the following cases depending on the weather conditions and money amount:
\begin{itemize}
\item {} 
\sphinxAtStartPar
(weather is nice) = True   and  (I have \$5) = True   ——> buy an ice cream

\item {} 
\sphinxAtStartPar
(weather is nice) = False  and  (I have \$5) = True   ——> do NOT buy an ice cream

\item {} 
\sphinxAtStartPar
(weather is nice) = True   and  (I have \$5) = False  ——> do NOT buy an ice cream

\item {} 
\sphinxAtStartPar
(weather is nice) = False  and  (I have \$5) = False  ——> do NOT buy an ice cream

\end{itemize}

\end{itemize}

\begin{sphinxuseclass}{cell}\begin{sphinxVerbatimInput}

\begin{sphinxuseclass}{cell_input}
\begin{sphinxVerbatim}[commandchars=\\\{\}]
\PYG{n}{temperature} \PYG{o}{=} \PYG{l+m+mi}{100}
\PYG{n}{money} \PYG{o}{=} \PYG{l+m+mi}{3}

\PYG{n+nb}{print}\PYG{p}{(}\PYG{l+s+sa}{f}\PYG{l+s+s1}{\PYGZsq{}}\PYG{l+s+s1}{Temperature is }\PYG{l+s+si}{\PYGZob{}}\PYG{n}{temperature}\PYG{l+s+si}{\PYGZcb{}}\PYG{l+s+s1}{.}\PYG{l+s+s1}{\PYGZsq{}}\PYG{p}{)}
\PYG{n+nb}{print}\PYG{p}{(}\PYG{l+s+sa}{f}\PYG{l+s+s1}{\PYGZsq{}}\PYG{l+s+s1}{I have \PYGZdl{}}\PYG{l+s+si}{\PYGZob{}}\PYG{n}{money}\PYG{l+s+si}{\PYGZcb{}}\PYG{l+s+s1}{.}\PYG{l+s+s1}{\PYGZsq{}}\PYG{p}{)}

\PYG{k}{if} \PYG{n}{temperature} \PYG{o}{\PYGZgt{}} \PYG{l+m+mi}{75} \PYG{o+ow}{and} \PYG{n}{money} \PYG{o}{\PYGZgt{}} \PYG{l+m+mi}{5}\PYG{p}{:}    \PYG{c+c1}{\PYGZsh{} True and False = False}
  \PYG{n+nb}{print}\PYG{p}{(}\PYG{l+s+s1}{\PYGZsq{}}\PYG{l+s+s1}{Go outside!}\PYG{l+s+s1}{\PYGZsq{}}\PYG{p}{)}                \PYG{c+c1}{\PYGZsh{} skipped}
\PYG{k}{else}\PYG{p}{:}
  \PYG{n+nb}{print}\PYG{p}{(}\PYG{l+s+s1}{\PYGZsq{}}\PYG{l+s+s1}{Stay at home!}\PYG{l+s+s1}{\PYGZsq{}}\PYG{p}{)}              \PYG{c+c1}{\PYGZsh{} executed}
\end{sphinxVerbatim}

\end{sphinxuseclass}\end{sphinxVerbatimInput}
\begin{sphinxVerbatimOutput}

\begin{sphinxuseclass}{cell_output}
\begin{sphinxVerbatim}[commandchars=\\\{\}]
Temperature is 100.
I have \PYGZdl{}3.
Stay at home!
\end{sphinxVerbatim}

\end{sphinxuseclass}\end{sphinxVerbatimOutput}

\end{sphinxuseclass}
\begin{sphinxuseclass}{cell}\begin{sphinxVerbatimInput}

\begin{sphinxuseclass}{cell_input}
\begin{sphinxVerbatim}[commandchars=\\\{\}]
\PYG{n}{temperature} \PYG{o}{=} \PYG{l+m+mi}{100}
\PYG{n}{money} \PYG{o}{=} \PYG{l+m+mi}{10}

\PYG{n+nb}{print}\PYG{p}{(}\PYG{l+s+sa}{f}\PYG{l+s+s1}{\PYGZsq{}}\PYG{l+s+s1}{Temperature is }\PYG{l+s+si}{\PYGZob{}}\PYG{n}{temperature}\PYG{l+s+si}{\PYGZcb{}}\PYG{l+s+s1}{.}\PYG{l+s+s1}{\PYGZsq{}}\PYG{p}{)}
\PYG{n+nb}{print}\PYG{p}{(}\PYG{l+s+sa}{f}\PYG{l+s+s1}{\PYGZsq{}}\PYG{l+s+s1}{I have \PYGZdl{}}\PYG{l+s+si}{\PYGZob{}}\PYG{n}{money}\PYG{l+s+si}{\PYGZcb{}}\PYG{l+s+s1}{.}\PYG{l+s+s1}{\PYGZsq{}}\PYG{p}{)}

\PYG{k}{if} \PYG{n}{temperature} \PYG{o}{\PYGZgt{}} \PYG{l+m+mi}{75} \PYG{o+ow}{and} \PYG{n}{money} \PYG{o}{\PYGZgt{}} \PYG{l+m+mi}{5}\PYG{p}{:}    \PYG{c+c1}{\PYGZsh{} True and True = True}
  \PYG{n+nb}{print}\PYG{p}{(}\PYG{l+s+s1}{\PYGZsq{}}\PYG{l+s+s1}{Go outside!}\PYG{l+s+s1}{\PYGZsq{}}\PYG{p}{)}                \PYG{c+c1}{\PYGZsh{} executed}
\PYG{k}{else}\PYG{p}{:}
  \PYG{n+nb}{print}\PYG{p}{(}\PYG{l+s+s1}{\PYGZsq{}}\PYG{l+s+s1}{Stay at home!}\PYG{l+s+s1}{\PYGZsq{}}\PYG{p}{)}              \PYG{c+c1}{\PYGZsh{} skipped}
\end{sphinxVerbatim}

\end{sphinxuseclass}\end{sphinxVerbatimInput}
\begin{sphinxVerbatimOutput}

\begin{sphinxuseclass}{cell_output}
\begin{sphinxVerbatim}[commandchars=\\\{\}]
Temperature is 100.
I have \PYGZdl{}10.
Go outside!
\end{sphinxVerbatim}

\end{sphinxuseclass}\end{sphinxVerbatimOutput}

\end{sphinxuseclass}

\section{try and except}
\label{\detokenize{conditionals:try-and-except}}
\sphinxAtStartPar
It is similar to an if\sphinxhyphen{}else statement. If there is an error in the code, the entire program will be terminated. To prevent termination in the presence of errors, a try\sphinxhyphen{}except statement is often used.
\begin{itemize}
\item {} 
\sphinxAtStartPar
This kind of situation is very common, especially when a user enters input that is not appropriate.
\begin{itemize}
\item {} 
\sphinxAtStartPar
entering “one” instead of digit “1”

\item {} 
\sphinxAtStartPar
making typos like “5s” instead of “5”

\end{itemize}

\item {} 
\sphinxAtStartPar
If you try to convert such inputs to a number, an error will occur, and the entire program may be terminated.

\item {} 
\sphinxAtStartPar
To handle such situations gracefully, you can use a try\sphinxhyphen{}except statement

\item {} 
\sphinxAtStartPar
\sphinxcode{\sphinxupquote{try\sphinxhyphen{}except}} works as follows:
\begin{itemize}
\item {} 
\sphinxAtStartPar
If there is no error in the block code of the try part, this block code will be executed.

\item {} 
\sphinxAtStartPar
If there is an error in the block code of the try part, the block code of the except part will be executed.

\end{itemize}

\item {} 
\sphinxAtStartPar
The structure of a try\sphinxhyphen{}except statement is as follows:

\end{itemize}

\sphinxAtStartPar
\sphinxcode{\sphinxupquote{try:}}\\
   \sphinxcode{\sphinxupquote{                    }}       \\
   \sphinxcode{\sphinxupquote{BLOCK CODE of TRY   }}       \\
   \sphinxcode{\sphinxupquote{                    }}       \\
\sphinxcode{\sphinxupquote{except:}}\\
   \sphinxcode{\sphinxupquote{                    }}     \\
   \sphinxcode{\sphinxupquote{BLOCK CODE of EXCEPT}}     \\
   \sphinxcode{\sphinxupquote{                    }}     
\begin{itemize}
\item {} 
\sphinxAtStartPar
The cases are as follows:

\end{itemize}
\begin{enumerate}
\sphinxsetlistlabels{\arabic}{enumi}{enumii}{}{.}%
\item {} 
\sphinxAtStartPar
BLOCK CODE of TRY has no error:

\end{enumerate}

\sphinxAtStartPar
\sphinxcode{\sphinxupquote{try:}}\\
   \sphinxcode{\sphinxupquote{                    }}       \\
   \sphinxcode{\sphinxupquote{BLOCK CODE of TRY   }}                 BLOCK CODE of TRY  will be executed.\\
   \sphinxcode{\sphinxupquote{                    }}       \\
\sphinxcode{\sphinxupquote{except:}}\\
   \sphinxcode{\sphinxupquote{                    }}     \\
   \sphinxcode{\sphinxupquote{BLOCK CODE of EXCEPT}}                 BLOCK CODE of EXCEPT  will be skipped.\\
   \sphinxcode{\sphinxupquote{                    }}     
\begin{enumerate}
\sphinxsetlistlabels{\arabic}{enumi}{enumii}{}{.}%
\setcounter{enumi}{1}
\item {} 
\sphinxAtStartPar
BLOCK CODE of TRY has an error:

\end{enumerate}

\sphinxAtStartPar
\sphinxcode{\sphinxupquote{try:}}\\
   \sphinxcode{\sphinxupquote{                    }}       \\
   \sphinxcode{\sphinxupquote{BLOCK CODE of TRY   }}                 BLOCK CODE of TRY  will be skipped.\\
   \sphinxcode{\sphinxupquote{                    }}       \\
\sphinxcode{\sphinxupquote{except:}}\\
   \sphinxcode{\sphinxupquote{                    }}     \\
   \sphinxcode{\sphinxupquote{BLOCK CODE of EXCEPT}}                 BLOCK CODE of EXCEPT  will be executed.\\
   \sphinxcode{\sphinxupquote{                    }}     

\begin{sphinxuseclass}{cell}\begin{sphinxVerbatimInput}

\begin{sphinxuseclass}{cell_input}
\begin{sphinxVerbatim}[commandchars=\\\{\}]
\PYG{n}{num} \PYG{o}{=}  \PYG{l+s+s1}{\PYGZsq{}}\PYG{l+s+s1}{5}\PYG{l+s+s1}{\PYGZsq{}}
\PYG{k}{try}\PYG{p}{:}
    \PYG{n}{x} \PYG{o}{=} \PYG{n+nb}{int}\PYG{p}{(}\PYG{n}{num}\PYG{p}{)}\PYG{o}{*}\PYG{o}{*}\PYG{l+m+mi}{2}                                 \PYG{c+c1}{\PYGZsh{} no error: \PYGZsq{}5\PYGZsq{} can be converted to an int}
    \PYG{n+nb}{print}\PYG{p}{(}\PYG{l+s+sa}{f}\PYG{l+s+s1}{\PYGZsq{}}\PYG{l+s+s1}{Square of }\PYG{l+s+si}{\PYGZob{}}\PYG{n}{num}\PYG{l+s+si}{\PYGZcb{}}\PYG{l+s+s1}{ is }\PYG{l+s+si}{\PYGZob{}}\PYG{n}{x}\PYG{l+s+si}{\PYGZcb{}}\PYG{l+s+s1}{\PYGZsq{}}\PYG{p}{)}                \PYG{c+c1}{\PYGZsh{} executed}
\PYG{k}{except}\PYG{p}{:}
    \PYG{n+nb}{print}\PYG{p}{(}\PYG{l+s+s1}{\PYGZsq{}}\PYG{l+s+s1}{Warning: Please enter an integer.}\PYG{l+s+s1}{\PYGZsq{}}\PYG{p}{)}      \PYG{c+c1}{\PYGZsh{} skipped}
\end{sphinxVerbatim}

\end{sphinxuseclass}\end{sphinxVerbatimInput}
\begin{sphinxVerbatimOutput}

\begin{sphinxuseclass}{cell_output}
\begin{sphinxVerbatim}[commandchars=\\\{\}]
Square of 5 is 25
\end{sphinxVerbatim}

\end{sphinxuseclass}\end{sphinxVerbatimOutput}

\end{sphinxuseclass}
\begin{sphinxuseclass}{cell}\begin{sphinxVerbatimInput}

\begin{sphinxuseclass}{cell_input}
\begin{sphinxVerbatim}[commandchars=\\\{\}]
\PYG{n}{num} \PYG{o}{=}  \PYG{l+s+s1}{\PYGZsq{}}\PYG{l+s+s1}{5s}\PYG{l+s+s1}{\PYGZsq{}}
\PYG{k}{try}\PYG{p}{:}
    \PYG{n}{x} \PYG{o}{=} \PYG{n+nb}{int}\PYG{p}{(}\PYG{n}{num}\PYG{p}{)}\PYG{o}{*}\PYG{o}{*}\PYG{l+m+mi}{2}                                 \PYG{c+c1}{\PYGZsh{} error: \PYGZsq{}5s\PYGZsq{} can not be converted to an int}
    \PYG{n+nb}{print}\PYG{p}{(}\PYG{l+s+sa}{f}\PYG{l+s+s1}{\PYGZsq{}}\PYG{l+s+s1}{Square of }\PYG{l+s+si}{\PYGZob{}}\PYG{n}{num}\PYG{l+s+si}{\PYGZcb{}}\PYG{l+s+s1}{ is }\PYG{l+s+si}{\PYGZob{}}\PYG{n}{x}\PYG{l+s+si}{\PYGZcb{}}\PYG{l+s+s1}{\PYGZsq{}}\PYG{p}{)}                \PYG{c+c1}{\PYGZsh{} skipped}
\PYG{k}{except}\PYG{p}{:}
    \PYG{n+nb}{print}\PYG{p}{(}\PYG{l+s+s1}{\PYGZsq{}}\PYG{l+s+s1}{Warning: Please enter an integer.}\PYG{l+s+s1}{\PYGZsq{}}\PYG{p}{)}      \PYG{c+c1}{\PYGZsh{} executed}
\end{sphinxVerbatim}

\end{sphinxuseclass}\end{sphinxVerbatimInput}
\begin{sphinxVerbatimOutput}

\begin{sphinxuseclass}{cell_output}
\begin{sphinxVerbatim}[commandchars=\\\{\}]
Warning: Please enter an integer.
\end{sphinxVerbatim}

\end{sphinxuseclass}\end{sphinxVerbatimOutput}

\end{sphinxuseclass}
\sphinxAtStartPar
\sphinxstylestrong{Remark}
\begin{itemize}
\item {} 
\sphinxAtStartPar
You must include the except part along with some code when using the try statement.

\item {} 
\sphinxAtStartPar
If you don’t intend to perform any specific actions in the except part, you can use the \sphinxcode{\sphinxupquote{pass}} keyword to prevent an error.

\end{itemize}

\begin{sphinxuseclass}{cell}\begin{sphinxVerbatimInput}

\begin{sphinxuseclass}{cell_input}
\begin{sphinxVerbatim}[commandchars=\\\{\}]
\PYG{n}{num} \PYG{o}{=}  \PYG{l+s+s1}{\PYGZsq{}}\PYG{l+s+s1}{5s}\PYG{l+s+s1}{\PYGZsq{}}
\PYG{k}{try}\PYG{p}{:}
    \PYG{n}{x} \PYG{o}{=} \PYG{n+nb}{int}\PYG{p}{(}\PYG{n}{num}\PYG{p}{)}\PYG{o}{*}\PYG{o}{*}\PYG{l+m+mi}{2}                                 \PYG{c+c1}{\PYGZsh{} error: \PYGZsq{}5s\PYGZsq{} can not be converted to an int}
    \PYG{n+nb}{print}\PYG{p}{(}\PYG{l+s+sa}{f}\PYG{l+s+s1}{\PYGZsq{}}\PYG{l+s+s1}{Square of }\PYG{l+s+si}{\PYGZob{}}\PYG{n}{num}\PYG{l+s+si}{\PYGZcb{}}\PYG{l+s+s1}{ is }\PYG{l+s+si}{\PYGZob{}}\PYG{n}{x}\PYG{l+s+si}{\PYGZcb{}}\PYG{l+s+s1}{\PYGZsq{}}\PYG{p}{)}                \PYG{c+c1}{\PYGZsh{} skipped}
\PYG{k}{except}\PYG{p}{:}
    \PYG{k}{pass}                                            \PYG{c+c1}{\PYGZsh{} does not do anything}
\end{sphinxVerbatim}

\end{sphinxuseclass}\end{sphinxVerbatimInput}

\end{sphinxuseclass}
\sphinxAtStartPar
No output


\section{Examples}
\label{\detokenize{conditionals:examples}}

\subsection{Even or Odd}
\label{\detokenize{conditionals:even-or-odd}}\begin{itemize}
\item {} 
\sphinxAtStartPar
Ask for an integer from the user and check whether it is even or odd.

\item {} 
\sphinxAtStartPar
Print your result using f\sphinxhyphen{}strings in the form of “The given number is even/odd.”

\end{itemize}

\sphinxAtStartPar
\sphinxstylestrong{Solution}

\begin{sphinxVerbatim}[commandchars=\\\{\}]
\PYG{n}{number} \PYG{o}{=} \PYG{n+nb}{int}\PYG{p}{(} \PYG{n+nb}{input}\PYG{p}{(}\PYG{l+s+s1}{\PYGZsq{}}\PYG{l+s+s1}{Enter an integer:}\PYG{l+s+s1}{\PYGZsq{}}\PYG{p}{)} \PYG{p}{)}

\PYG{k}{if} \PYG{n}{number}\PYG{o}{\PYGZpc{}}\PYG{l+m+mi}{2} \PYG{o}{==} \PYG{l+m+mi}{0}\PYG{p}{:}                            \PYG{c+c1}{\PYGZsh{} for even numbers, the remainder is zero, so this condition is True.}
  \PYG{n+nb}{print}\PYG{p}{(}\PYG{l+s+sa}{f}\PYG{l+s+s1}{\PYGZsq{}}\PYG{l+s+si}{\PYGZob{}}\PYG{n}{number}\PYG{l+s+si}{\PYGZcb{}}\PYG{l+s+s1}{ is an even number}\PYG{l+s+s1}{\PYGZsq{}}\PYG{p}{)}
\PYG{k}{else}\PYG{p}{:}
  \PYG{n+nb}{print}\PYG{p}{(}\PYG{l+s+sa}{f}\PYG{l+s+s1}{\PYGZsq{}}\PYG{l+s+si}{\PYGZob{}}\PYG{n}{number}\PYG{l+s+si}{\PYGZcb{}}\PYG{l+s+s1}{ is an odd number}\PYG{l+s+s1}{\PYGZsq{}}\PYG{p}{)}
\end{sphinxVerbatim}


\subsection{Greater than ten}
\label{\detokenize{conditionals:greater-than-ten}}\begin{itemize}
\item {} 
\sphinxAtStartPar
Ask the user for an integer and check whether it is greater than 10 or not.

\item {} 
\sphinxAtStartPar
Print the result using f\sphinxhyphen{}strings in the form: “The given number is greater/not greater than 10.

\end{itemize}

\sphinxAtStartPar
\sphinxstylestrong{Solution}

\begin{sphinxVerbatim}[commandchars=\\\{\}]
\PYG{n}{number} \PYG{o}{=} \PYG{n+nb}{int}\PYG{p}{(}\PYG{n+nb}{input}\PYG{p}{(}\PYG{l+s+s1}{\PYGZsq{}}\PYG{l+s+s1}{Enter an integer:}\PYG{l+s+s1}{\PYGZsq{}}\PYG{p}{)}\PYG{p}{)}

\PYG{k}{if} \PYG{n}{number}\PYG{o}{\PYGZgt{}}\PYG{l+m+mi}{10} \PYG{p}{:}                              \PYG{c+c1}{\PYGZsh{} this condition is True if the number is greater than 10.}
  \PYG{n+nb}{print}\PYG{p}{(}\PYG{l+s+sa}{f}\PYG{l+s+s1}{\PYGZsq{}}\PYG{l+s+si}{\PYGZob{}}\PYG{n}{number}\PYG{l+s+si}{\PYGZcb{}}\PYG{l+s+s1}{ is greater than 10}\PYG{l+s+s1}{\PYGZsq{}}\PYG{p}{)}
\PYG{k}{else}\PYG{p}{:}
  \PYG{n+nb}{print}\PYG{p}{(}\PYG{l+s+sa}{f}\PYG{l+s+s1}{\PYGZsq{}}\PYG{l+s+si}{\PYGZob{}}\PYG{n}{number}\PYG{l+s+si}{\PYGZcb{}}\PYG{l+s+s1}{ is not greater than 10.}\PYG{l+s+s1}{\PYGZsq{}}\PYG{p}{)}
\end{sphinxVerbatim}


\subsection{Same names}
\label{\detokenize{conditionals:same-names}}\begin{itemize}
\item {} 
\sphinxAtStartPar
Ask for two names from the user using two input() functions and check whether these names are the same.

\item {} 
\sphinxAtStartPar
It should not be case\sphinxhyphen{}sensitive, meaning that “Tom” and “TOM” are considered the same name.

\item {} 
\sphinxAtStartPar
Print the result using f\sphinxhyphen{}strings in the form of “\{name1\}” and “\{name2\}” are the same/not the same.

\item {} 
\sphinxAtStartPar
Print the names exactly as given by the user.

\end{itemize}

\sphinxAtStartPar
\sphinxstylestrong{Solution}

\begin{sphinxVerbatim}[commandchars=\\\{\}]
\PYG{n}{name1} \PYG{o}{=} \PYG{n+nb}{input}\PYG{p}{(}\PYG{l+s+s1}{\PYGZsq{}}\PYG{l+s+s1}{Please enter the first  name:}\PYG{l+s+s1}{\PYGZsq{}}\PYG{p}{)}
\PYG{n}{name2} \PYG{o}{=} \PYG{n+nb}{input}\PYG{p}{(}\PYG{l+s+s1}{\PYGZsq{}}\PYG{l+s+s1}{Please enter the second name:}\PYG{l+s+s1}{\PYGZsq{}}\PYG{p}{)}

\PYG{k}{if} \PYG{n}{name1}\PYG{o}{.}\PYG{n}{lower}\PYG{p}{(}\PYG{p}{)} \PYG{o}{==} \PYG{n}{name2}\PYG{o}{.}\PYG{n}{lower}\PYG{p}{(}\PYG{p}{)}\PYG{p}{:}             \PYG{c+c1}{\PYGZsh{} compare lower case versions to make it not case sensitive.}
  \PYG{n+nb}{print}\PYG{p}{(}\PYG{l+s+sa}{f}\PYG{l+s+s1}{\PYGZsq{}}\PYG{l+s+si}{\PYGZob{}}\PYG{n}{name1}\PYG{l+s+si}{\PYGZcb{}}\PYG{l+s+s1}{ and }\PYG{l+s+si}{\PYGZob{}}\PYG{n}{name2}\PYG{l+s+si}{\PYGZcb{}}\PYG{l+s+s1}{ are same.}\PYG{l+s+s1}{\PYGZsq{}}\PYG{p}{)}
\PYG{k}{else}\PYG{p}{:}
  \PYG{n+nb}{print}\PYG{p}{(}\PYG{l+s+sa}{f}\PYG{l+s+s1}{\PYGZsq{}}\PYG{l+s+si}{\PYGZob{}}\PYG{n}{name1}\PYG{l+s+si}{\PYGZcb{}}\PYG{l+s+s1}{ and }\PYG{l+s+si}{\PYGZob{}}\PYG{n}{name2}\PYG{l+s+si}{\PYGZcb{}}\PYG{l+s+s1}{ are not same.}\PYG{l+s+s1}{\PYGZsq{}}\PYG{p}{)}
\end{sphinxVerbatim}


\subsection{Letter Grades}
\label{\detokenize{conditionals:letter-grades}}\begin{itemize}
\item {} 
\sphinxAtStartPar
Write a program that asks the user to enter a percent grade.

\item {} 
\sphinxAtStartPar
Display the corresponding letter grade according to the following chart.
|Letter Grade|Grade Range|
|:—:|:—\sphinxhyphen{}:|
|A| 80 \sphinxhyphen{} 100|
|B| 60 \sphinxhyphen{}  79|
|C| 40 \sphinxhyphen{}  59|
|D| 20 \sphinxhyphen{}  39|
|F|  0 \sphinxhyphen{}  19|

\end{itemize}

\sphinxAtStartPar
\sphinxstylestrong{Solution}

\begin{sphinxVerbatim}[commandchars=\\\{\}]
\PYG{n}{grade} \PYG{o}{=} \PYG{n+nb}{float}\PYG{p}{(}  \PYG{n+nb}{input}\PYG{p}{(} \PYG{l+s+s1}{\PYGZsq{}}\PYG{l+s+s1}{Enter your percent grade:}\PYG{l+s+s1}{\PYGZsq{}} \PYG{p}{)}  \PYG{p}{)}

\PYG{k}{if} \PYG{l+m+mi}{80} \PYG{o}{\PYGZlt{}}\PYG{o}{=} \PYG{n}{grade} \PYG{o}{\PYGZlt{}}\PYG{o}{=} \PYG{l+m+mi}{100}\PYG{p}{:}
    \PYG{n+nb}{print}\PYG{p}{(}\PYG{l+s+s1}{\PYGZsq{}}\PYG{l+s+s1}{Your letter grade is A}\PYG{l+s+s1}{\PYGZsq{}}\PYG{p}{)}
\PYG{k}{elif} \PYG{l+m+mi}{60} \PYG{o}{\PYGZlt{}}\PYG{o}{=} \PYG{n}{grade} \PYG{p}{:}
    \PYG{n+nb}{print}\PYG{p}{(}\PYG{l+s+s1}{\PYGZsq{}}\PYG{l+s+s1}{Your letter grade is B}\PYG{l+s+s1}{\PYGZsq{}}\PYG{p}{)}
\PYG{k}{elif} \PYG{l+m+mi}{40} \PYG{o}{\PYGZlt{}}\PYG{o}{=} \PYG{n}{grade} \PYG{p}{:}
    \PYG{n+nb}{print}\PYG{p}{(}\PYG{l+s+s1}{\PYGZsq{}}\PYG{l+s+s1}{Your letter grade is C}\PYG{l+s+s1}{\PYGZsq{}}\PYG{p}{)}
\PYG{k}{elif} \PYG{l+m+mi}{20} \PYG{o}{\PYGZlt{}}\PYG{o}{=} \PYG{n}{grade} \PYG{p}{:}
    \PYG{n+nb}{print}\PYG{p}{(}\PYG{l+s+s1}{\PYGZsq{}}\PYG{l+s+s1}{Your letter grade is D}\PYG{l+s+s1}{\PYGZsq{}}\PYG{p}{)}
\PYG{k}{elif} \PYG{l+m+mi}{0} \PYG{o}{\PYGZlt{}}\PYG{o}{=} \PYG{n}{grade} \PYG{p}{:}
    \PYG{n+nb}{print}\PYG{p}{(}\PYG{l+s+s1}{\PYGZsq{}}\PYG{l+s+s1}{Your letter grade is F}\PYG{l+s+s1}{\PYGZsq{}}\PYG{p}{)}
\PYG{k}{else}\PYG{p}{:}
    \PYG{n+nb}{print}\PYG{p}{(}\PYG{l+s+sa}{f}\PYG{l+s+s1}{\PYGZsq{}}\PYG{l+s+si}{\PYGZob{}}\PYG{n}{grade}\PYG{l+s+si}{\PYGZcb{}}\PYG{l+s+s1}{ is not a percent grade}\PYG{l+s+s1}{\PYGZsq{}}\PYG{p}{)}
\end{sphinxVerbatim}


\subsection{Piecewise Defined Function}
\label{\detokenize{conditionals:piecewise-defined-function}}
\sphinxAtStartPar
The piecewise\sphinxhyphen{}defined function \(f(n)\) is given  as follows:
\begin{equation*}
\begin{split} f(n) =
  \begin{cases}
                                   4-2n &  n < -2  \\
                                   5   &  -2 \le n \le7\\
                                   1-n  &    n>7 \\
  \end{cases}
\end{split}
\end{equation*}\begin{itemize}
\item {} 
\sphinxAtStartPar
Write a program that asks the user to enter an integer.

\item {} 
\sphinxAtStartPar
If the integer entered is n then display f(n).

\item {} 
\sphinxAtStartPar
Hint:
\begin{itemize}
\item {} 
\sphinxAtStartPar
If n is less than \sphinxhyphen{}2 then f(n)=4\sphinxhyphen{}2n

\item {} 
\sphinxAtStartPar
If n is greater than  or equal to \sphinxhyphen{}2 and less than or equal to 7 then f(n)=5

\item {} 
\sphinxAtStartPar
If n is greater than 7 then f(n)=1\sphinxhyphen{}n

\end{itemize}

\item {} 
\sphinxAtStartPar
Example:
\begin{itemize}
\item {} 
\sphinxAtStartPar
\(f(-4) = 4-2(-4)=12\) since \(-4<-2\)

\item {} 
\sphinxAtStartPar
\(f(1) = 5\) since \(-2\le1\le7\)

\item {} 
\sphinxAtStartPar
\(f(10) = 1-10=-9\) since \(10>7\)

\end{itemize}

\end{itemize}

\sphinxAtStartPar
\sphinxstylestrong{Solution}

\begin{sphinxVerbatim}[commandchars=\\\{\}]
\PYG{n}{n} \PYG{o}{=} \PYG{n+nb}{int}\PYG{p}{(}   \PYG{n+nb}{input}\PYG{p}{(}\PYG{l+s+s1}{\PYGZsq{}}\PYG{l+s+s1}{Please enter a number:}\PYG{l+s+s1}{\PYGZsq{}}\PYG{p}{)}    \PYG{p}{)}

\PYG{k}{if} \PYG{n}{n} \PYG{o}{\PYGZlt{}} \PYG{o}{\PYGZhy{}}\PYG{l+m+mi}{2}\PYG{p}{:}
    \PYG{n+nb}{print}\PYG{p}{(}\PYG{l+s+sa}{f}\PYG{l+s+s1}{\PYGZsq{}}\PYG{l+s+s1}{f(}\PYG{l+s+si}{\PYGZob{}}\PYG{n}{n}\PYG{l+s+si}{\PYGZcb{}}\PYG{l+s+s1}{)=}\PYG{l+s+si}{\PYGZob{}}\PYG{l+m+mi}{4}\PYG{o}{\PYGZhy{}}\PYG{l+m+mi}{2}\PYG{o}{*}\PYG{n}{n}\PYG{l+s+si}{\PYGZcb{}}\PYG{l+s+s1}{\PYGZsq{}}\PYG{p}{)}
\PYG{k}{elif} \PYG{o}{\PYGZhy{}}\PYG{l+m+mi}{2} \PYG{o}{\PYGZlt{}}\PYG{o}{=} \PYG{n}{n} \PYG{o}{\PYGZlt{}}\PYG{o}{=} \PYG{l+m+mi}{7}\PYG{p}{:}
    \PYG{n+nb}{print}\PYG{p}{(}\PYG{l+s+sa}{f}\PYG{l+s+s1}{\PYGZsq{}}\PYG{l+s+s1}{f(}\PYG{l+s+si}{\PYGZob{}}\PYG{n}{n}\PYG{l+s+si}{\PYGZcb{}}\PYG{l+s+s1}{)=5}\PYG{l+s+s1}{\PYGZsq{}}\PYG{p}{)}
\PYG{k}{else}\PYG{p}{:}
    \PYG{n+nb}{print}\PYG{p}{(}\PYG{l+s+sa}{f}\PYG{l+s+s1}{\PYGZsq{}}\PYG{l+s+s1}{f(}\PYG{l+s+si}{\PYGZob{}}\PYG{n}{n}\PYG{l+s+si}{\PYGZcb{}}\PYG{l+s+s1}{)=}\PYG{l+s+si}{\PYGZob{}}\PYG{l+m+mi}{1}\PYG{o}{\PYGZhy{}}\PYG{n}{n}\PYG{l+s+si}{\PYGZcb{}}\PYG{l+s+s1}{\PYGZsq{}}\PYG{p}{)}
\end{sphinxVerbatim}


\subsection{Secret Number Game}
\label{\detokenize{conditionals:secret-number-game}}\begin{itemize}
\item {} 
\sphinxAtStartPar
Choose a random integer between 1 and 10 as the secret number.

\item {} 
\sphinxAtStartPar
Ask a number from the user to guess it.

\item {} 
\sphinxAtStartPar
If the user’s guess is correct, display You win!

\item {} 
\sphinxAtStartPar
If the user’s guess is incorrect, display Incorrect. Try again!!

\item {} 
\sphinxAtStartPar
Cheating Part:
\begin{itemize}
\item {} 
\sphinxAtStartPar
If the user’s guess is 99999, display You win!

\end{itemize}

\item {} 
\sphinxAtStartPar
Use try and except to avoid errors if the user enters non\sphinxhyphen{}numeric values.

\item {} 
\sphinxAtStartPar
Warn the user if there is an error by displaying a message.

\end{itemize}

\sphinxAtStartPar
\sphinxstylestrong{Solution}

\begin{sphinxVerbatim}[commandchars=\\\{\}]
\PYG{k+kn}{import} \PYG{n+nn}{random}
\PYG{n}{secret\PYGZus{}number} \PYG{o}{=} \PYG{n}{random}\PYG{o}{.}\PYG{n}{randint}\PYG{p}{(}\PYG{l+m+mi}{1}\PYG{p}{,}\PYG{l+m+mi}{10}\PYG{p}{)}

\PYG{k}{try}\PYG{p}{:}
  \PYG{n}{player} \PYG{o}{=} \PYG{n+nb}{int}\PYG{p}{(}\PYG{n+nb}{input}\PYG{p}{(}\PYG{l+s+s1}{\PYGZsq{}}\PYG{l+s+s1}{Guess the number: }\PYG{l+s+s1}{\PYGZsq{}}\PYG{p}{)}\PYG{p}{)}

  \PYG{k}{if} \PYG{n}{player} \PYG{o}{==} \PYG{n}{secret\PYGZus{}number}\PYG{p}{:}
    \PYG{n+nb}{print}\PYG{p}{(}\PYG{l+s+s1}{\PYGZsq{}}\PYG{l+s+s1}{You win!}\PYG{l+s+s1}{\PYGZsq{}}\PYG{p}{)}
  \PYG{k}{elif} \PYG{n}{player} \PYG{o}{==} \PYG{l+m+mi}{99999}\PYG{p}{:}       \PYG{c+c1}{\PYGZsh{} cheating part}
    \PYG{n+nb}{print}\PYG{p}{(}\PYG{l+s+s1}{\PYGZsq{}}\PYG{l+s+s1}{You win!}\PYG{l+s+s1}{\PYGZsq{}}\PYG{p}{)}
  \PYG{k}{else}\PYG{p}{:}
    \PYG{n+nb}{print}\PYG{p}{(}\PYG{l+s+s1}{\PYGZsq{}}\PYG{l+s+s1}{Incorrect. Try Again!}\PYG{l+s+s1}{\PYGZsq{}}\PYG{p}{)}
      
\PYG{k}{except}\PYG{p}{:}
  \PYG{n+nb}{print}\PYG{p}{(}\PYG{l+s+s1}{\PYGZsq{}}\PYG{l+s+s1}{Please enter a valid numeric value!}\PYG{l+s+s1}{\PYGZsq{}}\PYG{p}{)}
\end{sphinxVerbatim}

\sphinxstepscope


\section{Conditionals Debugging}
\label{\detokenize{conditionals_debug:conditionals-debugging}}\label{\detokenize{conditionals_debug::doc}}\begin{itemize}
\item {} 
\sphinxAtStartPar
Each of the following short code contains one or more bugs.     

\item {} 
\sphinxAtStartPar
Please identify and correct these bugs.

\item {} 
\sphinxAtStartPar
Provide an explanation for your answer.

\end{itemize}


\subsection{Question}
\label{\detokenize{conditionals_debug:question}}
\begin{sphinxVerbatim}[commandchars=\\\{\}]
\PYG{n}{x} \PYG{o}{=} \PYG{l+m+mi}{10}

\PYG{k}{if} \PYG{n}{x}\PYG{o}{\PYGZlt{}}\PYG{l+m+mi}{20}\PYG{p}{:}
\PYG{n+nb}{print}\PYG{p}{(}\PYG{l+s+s1}{\PYGZsq{}}\PYG{l+s+s1}{A}\PYG{l+s+s1}{\PYGZsq{}}\PYG{p}{)}
\end{sphinxVerbatim}

\begin{sphinxadmonition}{note}{Solution}

\sphinxAtStartPar
Add appropriate indentation to the last line.
\end{sphinxadmonition}


\subsection{Question}
\label{\detokenize{conditionals_debug:id1}}
\begin{sphinxVerbatim}[commandchars=\\\{\}]
\PYG{n}{x} \PYG{o}{=} \PYG{l+m+mi}{10}

\PYG{k}{if} \PYG{n}{x}\PYG{o}{\PYGZlt{}}\PYG{l+m+mi}{20}
  \PYG{n+nb}{print}\PYG{p}{(}\PYG{n}{x}\PYG{o}{+}\PYG{l+m+mi}{5}\PYG{p}{)}
\end{sphinxVerbatim}

\begin{sphinxadmonition}{note}{Solution}

\sphinxAtStartPar
Colon (:) is missing at the end of the second line.
\end{sphinxadmonition}


\subsection{Question}
\label{\detokenize{conditionals_debug:id2}}
\begin{sphinxVerbatim}[commandchars=\\\{\}]
\PYG{n}{x} \PYG{o}{=} \PYG{l+m+mi}{10}

\PYG{k}{if} \PYG{n}{x}\PYG{o}{\PYGZlt{}}\PYG{l+m+mi}{20}\PYG{p}{:}
  \PYG{n+nb}{print}\PYG{p}{(}\PYG{n}{A}\PYG{p}{)}
\end{sphinxVerbatim}

\begin{sphinxadmonition}{note}{Solution}

\sphinxAtStartPar
A is not defined.
\end{sphinxadmonition}


\subsection{Question}
\label{\detokenize{conditionals_debug:id3}}
\begin{sphinxVerbatim}[commandchars=\\\{\}]
\PYG{n}{is\PYGZus{}cheap} \PYG{o}{=} \PYG{n}{true}

\PYG{k}{if} \PYG{n}{is\PYGZus{}cheap}\PYG{p}{:}
  \PYG{n+nb}{print}\PYG{p}{(}\PYG{l+s+s1}{\PYGZsq{}}\PYG{l+s+s1}{Buy it}\PYG{l+s+s1}{\PYGZsq{}}\PYG{p}{)}
\end{sphinxVerbatim}

\begin{sphinxadmonition}{note}{Solution}

\sphinxAtStartPar
In the first line, true should be capitalized as True to represent the boolean value.
\end{sphinxadmonition}


\subsection{Question}
\label{\detokenize{conditionals_debug:id4}}
\begin{sphinxVerbatim}[commandchars=\\\{\}]
\PYG{n}{house\PYGZus{}age} \PYG{o}{=} \PYG{l+m+mi}{20}

\PYG{k}{if} \PYG{n}{house\PYGZus{}age}  \PYG{o}{\PYGZgt{}} \PYG{l+m+mi}{20}\PYG{p}{:}
  \PYG{n+nb}{print}\PYG{p}{(}\PYG{l+s+s1}{\PYGZsq{}}\PYG{l+s+s1}{OLD}\PYG{l+s+s1}{\PYGZsq{}}\PYG{p}{)}
\PYG{k}{elif} \PYG{n}{house\PYGZus{}age} \PYG{o}{=} \PYG{l+m+mi}{20}\PYG{p}{:}
  \PYG{n+nb}{print}\PYG{p}{(}\PYG{l+s+s1}{\PYGZsq{}}\PYG{l+s+s1}{Twenty}\PYG{l+s+s1}{\PYGZsq{}}\PYG{p}{)}
\PYG{k}{else}\PYG{p}{:}
  \PYG{n+nb}{print}\PYG{p}{(}\PYG{l+s+s1}{\PYGZsq{}}\PYG{l+s+s1}{NEW}\PYG{l+s+s1}{\PYGZsq{}}\PYG{p}{)}
\end{sphinxVerbatim}

\begin{sphinxadmonition}{note}{Solution}

\sphinxAtStartPar
In the condition of the elif part, replace = with == to form a correct boolean expression.
\end{sphinxadmonition}


\subsection{Question}
\label{\detokenize{conditionals_debug:id5}}
\begin{sphinxVerbatim}[commandchars=\\\{\}]
\PYG{n}{x} \PYG{o}{=} \PYG{l+m+mi}{5}

\PYG{k}{if} \PYG{n}{x} \PYG{o}{\PYGZgt{}} \PYG{l+m+mi}{10}\PYG{p}{:}
    \PYG{n+nb}{print}\PYG{p}{(}\PYG{l+s+s1}{\PYGZsq{}}\PYG{l+s+s1}{A}\PYG{l+s+s1}{\PYGZsq{}}\PYG{p}{)}
\PYG{k}{elif}\PYG{p}{:}
    \PYG{n+nb}{print}\PYG{p}{(}\PYG{l+s+s1}{\PYGZsq{}}\PYG{l+s+s1}{B}\PYG{l+s+s1}{\PYGZsq{}}\PYG{p}{)}
\PYG{k}{else}\PYG{p}{:}
    \PYG{n+nb}{print}\PYG{p}{(}\PYG{l+s+s1}{\PYGZsq{}}\PYG{l+s+s1}{C}\PYG{l+s+s1}{\PYGZsq{}}\PYG{p}{)}
\end{sphinxVerbatim}

\begin{sphinxadmonition}{note}{Solution}

\sphinxAtStartPar
The condition (boolean expression) of the elif part is missing.
\end{sphinxadmonition}

\sphinxstepscope


\section{Conditionals Output}
\label{\detokenize{conditionals_output:conditionals-output}}\label{\detokenize{conditionals_output::doc}}\begin{itemize}
\item {} 
\sphinxAtStartPar
Find the output of the following code.

\item {} 
\sphinxAtStartPar
Please don’t run the code before giving your answer.     

\end{itemize}


\subsection{Question}
\label{\detokenize{conditionals_output:question}}
\begin{sphinxuseclass}{cell}
\begin{sphinxuseclass}{tag_hide-output}\begin{sphinxVerbatimInput}

\begin{sphinxuseclass}{cell_input}
\begin{sphinxVerbatim}[commandchars=\\\{\}]
\PYG{n+nb}{print}\PYG{p}{(}\PYG{k+kc}{True}\PYG{o}{+}\PYG{k+kc}{False}\PYG{o}{+}\PYG{l+m+mi}{3}\PYG{o}{+}\PYG{k+kc}{True}\PYG{p}{)}
\end{sphinxVerbatim}

\end{sphinxuseclass}\end{sphinxVerbatimInput}

\end{sphinxuseclass}
\end{sphinxuseclass}

\subsection{Question}
\label{\detokenize{conditionals_output:id1}}
\begin{sphinxuseclass}{cell}
\begin{sphinxuseclass}{tag_hide-output}\begin{sphinxVerbatimInput}

\begin{sphinxuseclass}{cell_input}
\begin{sphinxVerbatim}[commandchars=\\\{\}]
\PYG{n}{x} \PYG{o}{=} \PYG{l+m+mi}{20}
\PYG{n}{y} \PYG{o}{=} \PYG{l+m+mi}{100}

\PYG{n}{b1} \PYG{o}{=} \PYG{n}{x} \PYG{o}{==} \PYG{n}{y}
\PYG{n}{b2} \PYG{o}{=} \PYG{n}{x} \PYG{o}{\PYGZlt{}} \PYG{n}{y}

\PYG{n+nb}{print}\PYG{p}{(}\PYG{n}{b1}\PYG{p}{)}
\PYG{n+nb}{print}\PYG{p}{(}\PYG{n}{b2}\PYG{p}{)}
\PYG{n+nb}{print}\PYG{p}{(}\PYG{n}{b1} \PYG{o+ow}{and} \PYG{n}{b2}\PYG{p}{)}
\PYG{n+nb}{print}\PYG{p}{(}\PYG{n}{b1} \PYG{o+ow}{or} \PYG{n}{b2}\PYG{p}{)}
\end{sphinxVerbatim}

\end{sphinxuseclass}\end{sphinxVerbatimInput}

\end{sphinxuseclass}
\end{sphinxuseclass}

\subsection{Question}
\label{\detokenize{conditionals_output:id2}}
\begin{sphinxuseclass}{cell}
\begin{sphinxuseclass}{tag_hide-output}\begin{sphinxVerbatimInput}

\begin{sphinxuseclass}{cell_input}
\begin{sphinxVerbatim}[commandchars=\\\{\}]
\PYG{n}{x} \PYG{o}{=} \PYG{l+m+mi}{10}

\PYG{k}{if} \PYG{n}{x}\PYG{o}{\PYGZlt{}}\PYG{l+m+mi}{7}\PYG{p}{:}
  \PYG{n+nb}{print}\PYG{p}{(}\PYG{l+s+s1}{\PYGZsq{}}\PYG{l+s+s1}{A}\PYG{l+s+s1}{\PYGZsq{}}\PYG{p}{)}
\end{sphinxVerbatim}

\end{sphinxuseclass}\end{sphinxVerbatimInput}

\end{sphinxuseclass}
\end{sphinxuseclass}\begin{itemize}
\item {} 
\sphinxAtStartPar
No output!

\end{itemize}


\subsection{Question}
\label{\detokenize{conditionals_output:id3}}
\begin{sphinxuseclass}{cell}
\begin{sphinxuseclass}{tag_hide-output}\begin{sphinxVerbatimInput}

\begin{sphinxuseclass}{cell_input}
\begin{sphinxVerbatim}[commandchars=\\\{\}]
\PYG{n}{x} \PYG{o}{=} \PYG{l+m+mi}{20}

\PYG{k}{if} \PYG{n}{x} \PYG{o}{\PYGZgt{}} \PYG{l+m+mi}{10}\PYG{p}{:}
  \PYG{n+nb}{print}\PYG{p}{(}\PYG{l+s+s1}{\PYGZsq{}}\PYG{l+s+s1}{A}\PYG{l+s+s1}{\PYGZsq{}}\PYG{p}{)}
\PYG{k}{elif} \PYG{n}{x} \PYG{o}{\PYGZgt{}} \PYG{l+m+mi}{15}\PYG{p}{:}
  \PYG{n+nb}{print}\PYG{p}{(}\PYG{l+s+s1}{\PYGZsq{}}\PYG{l+s+s1}{B}\PYG{l+s+s1}{\PYGZsq{}}\PYG{p}{)}
\PYG{k}{else}\PYG{p}{:}
  \PYG{n+nb}{print}\PYG{p}{(}\PYG{l+s+s1}{\PYGZsq{}}\PYG{l+s+s1}{C}\PYG{l+s+s1}{\PYGZsq{}}\PYG{p}{)}
\end{sphinxVerbatim}

\end{sphinxuseclass}\end{sphinxVerbatimInput}

\end{sphinxuseclass}
\end{sphinxuseclass}

\subsection{Question}
\label{\detokenize{conditionals_output:id4}}
\begin{sphinxuseclass}{cell}
\begin{sphinxuseclass}{tag_hide-output}\begin{sphinxVerbatimInput}

\begin{sphinxuseclass}{cell_input}
\begin{sphinxVerbatim}[commandchars=\\\{\}]
\PYG{n}{x} \PYG{o}{=} \PYG{l+m+mi}{20}

\PYG{k}{if} \PYG{n}{x} \PYG{o}{\PYGZgt{}} \PYG{l+m+mi}{10}\PYG{p}{:}
  \PYG{n+nb}{print}\PYG{p}{(}\PYG{l+s+s1}{\PYGZsq{}}\PYG{l+s+s1}{A}\PYG{l+s+s1}{\PYGZsq{}}\PYG{p}{)}
\PYG{k}{if} \PYG{n}{x} \PYG{o}{\PYGZgt{}} \PYG{l+m+mi}{15}\PYG{p}{:}
  \PYG{n+nb}{print}\PYG{p}{(}\PYG{l+s+s1}{\PYGZsq{}}\PYG{l+s+s1}{B}\PYG{l+s+s1}{\PYGZsq{}}\PYG{p}{)}
\end{sphinxVerbatim}

\end{sphinxuseclass}\end{sphinxVerbatimInput}

\end{sphinxuseclass}
\end{sphinxuseclass}

\subsection{Question}
\label{\detokenize{conditionals_output:id5}}
\begin{sphinxuseclass}{cell}
\begin{sphinxuseclass}{tag_hide-output}\begin{sphinxVerbatimInput}

\begin{sphinxuseclass}{cell_input}
\begin{sphinxVerbatim}[commandchars=\\\{\}]
\PYG{n}{x} \PYG{o}{=} \PYG{l+m+mi}{17}
\PYG{n}{y} \PYG{o}{=} \PYG{l+m+mi}{9}

\PYG{k}{if} \PYG{n}{x}\PYG{o}{\PYGZlt{}}\PYG{l+m+mi}{15} \PYG{o+ow}{and} \PYG{n}{y}\PYG{o}{\PYGZgt{}}\PYG{l+m+mi}{5}\PYG{p}{:}
  \PYG{n+nb}{print}\PYG{p}{(}\PYG{l+s+s1}{\PYGZsq{}}\PYG{l+s+s1}{A}\PYG{l+s+s1}{\PYGZsq{}}\PYG{p}{)}
\PYG{k}{elif} \PYG{n}{x}\PYG{o}{\PYGZgt{}}\PYG{l+m+mi}{7} \PYG{o+ow}{and} \PYG{n}{y}\PYG{o}{\PYGZgt{}}\PYG{l+m+mi}{5}\PYG{p}{:}
  \PYG{n+nb}{print}\PYG{p}{(}\PYG{l+s+s1}{\PYGZsq{}}\PYG{l+s+s1}{B}\PYG{l+s+s1}{\PYGZsq{}}\PYG{p}{)}
\PYG{k}{elif} \PYG{n}{x}\PYG{o}{\PYGZgt{}}\PYG{l+m+mi}{10} \PYG{o+ow}{and} \PYG{n}{y}\PYG{o}{\PYGZgt{}}\PYG{l+m+mi}{6}\PYG{p}{:}
  \PYG{n+nb}{print}\PYG{p}{(}\PYG{l+s+s1}{\PYGZsq{}}\PYG{l+s+s1}{C}\PYG{l+s+s1}{\PYGZsq{}}\PYG{p}{)}
\PYG{k}{else}\PYG{p}{:}
  \PYG{n+nb}{print}\PYG{p}{(}\PYG{l+s+s1}{\PYGZsq{}}\PYG{l+s+s1}{D}\PYG{l+s+s1}{\PYGZsq{}}\PYG{p}{)}
\end{sphinxVerbatim}

\end{sphinxuseclass}\end{sphinxVerbatimInput}

\end{sphinxuseclass}
\end{sphinxuseclass}

\subsection{Question}
\label{\detokenize{conditionals_output:id6}}
\begin{sphinxuseclass}{cell}
\begin{sphinxuseclass}{tag_hide-output}\begin{sphinxVerbatimInput}

\begin{sphinxuseclass}{cell_input}
\begin{sphinxVerbatim}[commandchars=\\\{\}]
\PYG{n}{x} \PYG{o}{=} \PYG{l+m+mi}{25}

\PYG{k}{if} \PYG{n}{x} \PYG{o}{\PYGZlt{}} \PYG{l+m+mi}{30}\PYG{p}{:}
  \PYG{n+nb}{print}\PYG{p}{(}\PYG{l+s+s1}{\PYGZsq{}}\PYG{l+s+s1}{A}\PYG{l+s+s1}{\PYGZsq{}}\PYG{p}{)}
  \PYG{k}{if} \PYG{n}{x} \PYG{o}{\PYGZlt{}} \PYG{l+m+mi}{20}\PYG{p}{:}
    \PYG{n+nb}{print}\PYG{p}{(}\PYG{l+s+s1}{\PYGZsq{}}\PYG{l+s+s1}{B}\PYG{l+s+s1}{\PYGZsq{}}\PYG{p}{)}
\end{sphinxVerbatim}

\end{sphinxuseclass}\end{sphinxVerbatimInput}

\end{sphinxuseclass}
\end{sphinxuseclass}

\subsection{Question}
\label{\detokenize{conditionals_output:id7}}
\begin{sphinxuseclass}{cell}
\begin{sphinxuseclass}{tag_hide-output}\begin{sphinxVerbatimInput}

\begin{sphinxuseclass}{cell_input}
\begin{sphinxVerbatim}[commandchars=\\\{\}]
\PYG{n}{x} \PYG{o}{=} \PYG{l+m+mi}{25}

\PYG{k}{if} \PYG{n}{x} \PYG{o}{\PYGZlt{}} \PYG{l+m+mi}{30}\PYG{p}{:}
  \PYG{n+nb}{print}\PYG{p}{(}\PYG{l+s+s1}{\PYGZsq{}}\PYG{l+s+s1}{A}\PYG{l+s+s1}{\PYGZsq{}}\PYG{p}{)}
  \PYG{k}{if} \PYG{n}{x} \PYG{o}{\PYGZlt{}} \PYG{l+m+mi}{40}\PYG{p}{:}
    \PYG{n+nb}{print}\PYG{p}{(}\PYG{l+s+s1}{\PYGZsq{}}\PYG{l+s+s1}{B}\PYG{l+s+s1}{\PYGZsq{}}\PYG{p}{)}
\end{sphinxVerbatim}

\end{sphinxuseclass}\end{sphinxVerbatimInput}

\end{sphinxuseclass}
\end{sphinxuseclass}

\subsection{Question}
\label{\detokenize{conditionals_output:id8}}
\begin{sphinxuseclass}{cell}
\begin{sphinxuseclass}{tag_hide-output}\begin{sphinxVerbatimInput}

\begin{sphinxuseclass}{cell_input}
\begin{sphinxVerbatim}[commandchars=\\\{\}]
\PYG{n}{x} \PYG{o}{=} \PYG{l+m+mi}{25}

\PYG{k}{if} \PYG{n}{x} \PYG{o}{\PYGZlt{}} \PYG{l+m+mi}{10}\PYG{p}{:}
  \PYG{n+nb}{print}\PYG{p}{(}\PYG{l+s+s1}{\PYGZsq{}}\PYG{l+s+s1}{A}\PYG{l+s+s1}{\PYGZsq{}}\PYG{p}{)}
  \PYG{k}{if} \PYG{n}{x} \PYG{o}{\PYGZlt{}} \PYG{l+m+mi}{40}\PYG{p}{:}
    \PYG{n+nb}{print}\PYG{p}{(}\PYG{l+s+s1}{\PYGZsq{}}\PYG{l+s+s1}{B}\PYG{l+s+s1}{\PYGZsq{}}\PYG{p}{)}
\end{sphinxVerbatim}

\end{sphinxuseclass}\end{sphinxVerbatimInput}

\end{sphinxuseclass}
\end{sphinxuseclass}\begin{itemize}
\item {} 
\sphinxAtStartPar
No output!

\end{itemize}


\subsection{Question}
\label{\detokenize{conditionals_output:id9}}
\begin{sphinxuseclass}{cell}
\begin{sphinxuseclass}{tag_hide-output}\begin{sphinxVerbatimInput}

\begin{sphinxuseclass}{cell_input}
\begin{sphinxVerbatim}[commandchars=\\\{\}]
\PYG{k}{if} \PYG{l+m+mi}{10}\PYG{p}{:}
  \PYG{n+nb}{print}\PYG{p}{(}\PYG{l+s+s1}{\PYGZsq{}}\PYG{l+s+s1}{USA}\PYG{l+s+s1}{\PYGZsq{}}\PYG{p}{)}
\end{sphinxVerbatim}

\end{sphinxuseclass}\end{sphinxVerbatimInput}

\end{sphinxuseclass}
\end{sphinxuseclass}

\subsection{Question}
\label{\detokenize{conditionals_output:id10}}
\begin{sphinxuseclass}{cell}
\begin{sphinxuseclass}{tag_hide-output}\begin{sphinxVerbatimInput}

\begin{sphinxuseclass}{cell_input}
\begin{sphinxVerbatim}[commandchars=\\\{\}]
\PYG{k}{if} \PYG{l+m+mi}{0}\PYG{p}{:}
  \PYG{n+nb}{print}\PYG{p}{(}\PYG{l+s+s1}{\PYGZsq{}}\PYG{l+s+s1}{USA}\PYG{l+s+s1}{\PYGZsq{}}\PYG{p}{)}
\end{sphinxVerbatim}

\end{sphinxuseclass}\end{sphinxVerbatimInput}

\end{sphinxuseclass}
\end{sphinxuseclass}\begin{itemize}
\item {} 
\sphinxAtStartPar
No output!

\end{itemize}


\subsection{Question}
\label{\detokenize{conditionals_output:id11}}
\begin{sphinxuseclass}{cell}
\begin{sphinxuseclass}{tag_hide-output}\begin{sphinxVerbatimInput}

\begin{sphinxuseclass}{cell_input}
\begin{sphinxVerbatim}[commandchars=\\\{\}]
\PYG{k}{if} \PYG{l+m+mi}{0}\PYG{o}{==}\PYG{l+m+mi}{0} \PYG{p}{:}
  \PYG{n+nb}{print}\PYG{p}{(}\PYG{l+s+s1}{\PYGZsq{}}\PYG{l+s+s1}{USA}\PYG{l+s+s1}{\PYGZsq{}}\PYG{p}{)}
\end{sphinxVerbatim}

\end{sphinxuseclass}\end{sphinxVerbatimInput}

\end{sphinxuseclass}
\end{sphinxuseclass}

\subsection{Question}
\label{\detokenize{conditionals_output:id12}}
\begin{sphinxuseclass}{cell}
\begin{sphinxuseclass}{tag_hide-output}\begin{sphinxVerbatimInput}

\begin{sphinxuseclass}{cell_input}
\begin{sphinxVerbatim}[commandchars=\\\{\}]
\PYG{n}{x} \PYG{o}{=} \PYG{l+m+mi}{10}

\PYG{k}{if} \PYG{n}{x} \PYG{o}{\PYGZgt{}} \PYG{l+m+mi}{2}\PYG{o}{*}\PYG{n}{x}\PYG{p}{:}
  \PYG{n+nb}{print}\PYG{p}{(}\PYG{n}{x}\PYG{p}{)}
\PYG{n+nb}{print}\PYG{p}{(}\PYG{l+m+mi}{2}\PYG{o}{*}\PYG{n}{x}\PYG{p}{)}
\end{sphinxVerbatim}

\end{sphinxuseclass}\end{sphinxVerbatimInput}

\end{sphinxuseclass}
\end{sphinxuseclass}

\subsection{Question}
\label{\detokenize{conditionals_output:id13}}
\begin{sphinxuseclass}{cell}
\begin{sphinxuseclass}{tag_hide-output}\begin{sphinxVerbatimInput}

\begin{sphinxuseclass}{cell_input}
\begin{sphinxVerbatim}[commandchars=\\\{\}]
\PYG{n}{x} \PYG{o}{=} \PYG{l+m+mi}{10}

\PYG{k}{if} \PYG{n}{x} \PYG{o}{\PYGZgt{}} \PYG{l+m+mi}{2}\PYG{o}{*}\PYG{n}{x}\PYG{p}{:}
  \PYG{n+nb}{print}\PYG{p}{(}\PYG{n}{x}\PYG{p}{)}
  \PYG{n+nb}{print}\PYG{p}{(}\PYG{l+m+mi}{2}\PYG{o}{*}\PYG{n}{x}\PYG{p}{)}
\end{sphinxVerbatim}

\end{sphinxuseclass}\end{sphinxVerbatimInput}

\end{sphinxuseclass}
\end{sphinxuseclass}\begin{itemize}
\item {} 
\sphinxAtStartPar
No output!

\end{itemize}


\subsection{Question}
\label{\detokenize{conditionals_output:id14}}
\begin{sphinxuseclass}{cell}
\begin{sphinxuseclass}{tag_hide-output}\begin{sphinxVerbatimInput}

\begin{sphinxuseclass}{cell_input}
\begin{sphinxVerbatim}[commandchars=\\\{\}]
\PYG{k}{if} \PYG{k+kc}{True}\PYG{p}{:}
  \PYG{k}{if} \PYG{l+m+mi}{1}\PYG{p}{:}
    \PYG{k}{if} \PYG{l+m+mi}{5}\PYG{p}{:}
      \PYG{k}{if} \PYG{l+m+mi}{0}\PYG{p}{:}
        \PYG{n+nb}{print}\PYG{p}{(}\PYG{l+s+s1}{\PYGZsq{}}\PYG{l+s+s1}{A}\PYG{l+s+s1}{\PYGZsq{}}\PYG{p}{)}
      \PYG{k}{else}\PYG{p}{:}
        \PYG{n+nb}{print}\PYG{p}{(}\PYG{l+s+s1}{\PYGZsq{}}\PYG{l+s+s1}{B}\PYG{l+s+s1}{\PYGZsq{}}\PYG{p}{)}
    \PYG{k}{else}\PYG{p}{:}
      \PYG{n+nb}{print}\PYG{p}{(}\PYG{l+s+s1}{\PYGZsq{}}\PYG{l+s+s1}{C}\PYG{l+s+s1}{\PYGZsq{}}\PYG{p}{)}
\end{sphinxVerbatim}

\end{sphinxuseclass}\end{sphinxVerbatimInput}

\end{sphinxuseclass}
\end{sphinxuseclass}

\subsection{Question}
\label{\detokenize{conditionals_output:id15}}
\begin{sphinxuseclass}{cell}
\begin{sphinxuseclass}{tag_hide-output}\begin{sphinxVerbatimInput}

\begin{sphinxuseclass}{cell_input}
\begin{sphinxVerbatim}[commandchars=\\\{\}]
\PYG{n}{x} \PYG{o}{=} \PYG{l+m+mi}{5}

\PYG{k}{if} \PYG{n}{x} \PYG{o}{\PYGZgt{}} \PYG{l+m+mi}{3}\PYG{p}{:}
  \PYG{n+nb}{print}\PYG{p}{(}\PYG{l+s+s1}{\PYGZsq{}}\PYG{l+s+s1}{A}\PYG{l+s+s1}{\PYGZsq{}}\PYG{p}{)}
\PYG{k}{if} \PYG{n}{x} \PYG{o}{\PYGZgt{}} \PYG{l+m+mi}{4}\PYG{p}{:}
  \PYG{n+nb}{print}\PYG{p}{(}\PYG{l+s+s1}{\PYGZsq{}}\PYG{l+s+s1}{B}\PYG{l+s+s1}{\PYGZsq{}}\PYG{p}{)}
  \PYG{n+nb}{print}\PYG{p}{(}\PYG{l+s+s1}{\PYGZsq{}}\PYG{l+s+s1}{***}\PYG{l+s+s1}{\PYGZsq{}}\PYG{p}{)}
\PYG{k}{if} \PYG{n}{x} \PYG{o}{\PYGZgt{}} \PYG{l+m+mi}{10}\PYG{p}{:}
  \PYG{n+nb}{print}\PYG{p}{(}\PYG{l+s+s1}{\PYGZsq{}}\PYG{l+s+s1}{C}\PYG{l+s+s1}{\PYGZsq{}}\PYG{p}{)}
  \PYG{n+nb}{print}\PYG{p}{(}\PYG{l+s+s1}{\PYGZsq{}}\PYG{l+s+s1}{\PYGZhy{}\PYGZhy{}\PYGZhy{}}\PYG{l+s+s1}{\PYGZsq{}}\PYG{p}{)}
\PYG{n+nb}{print}\PYG{p}{(}\PYG{l+s+s1}{\PYGZsq{}}\PYG{l+s+s1}{D}\PYG{l+s+s1}{\PYGZsq{}}\PYG{p}{)}
\end{sphinxVerbatim}

\end{sphinxuseclass}\end{sphinxVerbatimInput}

\end{sphinxuseclass}
\end{sphinxuseclass}

\subsection{Question}
\label{\detokenize{conditionals_output:id16}}
\begin{sphinxuseclass}{cell}
\begin{sphinxuseclass}{tag_hide-output}\begin{sphinxVerbatimInput}

\begin{sphinxuseclass}{cell_input}
\begin{sphinxVerbatim}[commandchars=\\\{\}]
\PYG{n}{x} \PYG{o}{=} \PYG{l+s+s1}{\PYGZsq{}}\PYG{l+s+s1}{3}\PYG{l+s+s1}{\PYGZsq{}}

\PYG{k}{try}\PYG{p}{:}
  \PYG{n+nb}{print}\PYG{p}{(}\PYG{n}{x}\PYG{o}{+}\PYG{l+m+mi}{5}\PYG{p}{)}
\PYG{k}{except}\PYG{p}{:}
  \PYG{n+nb}{print}\PYG{p}{(}\PYG{l+s+s1}{\PYGZsq{}}\PYG{l+s+s1}{ERROR}\PYG{l+s+s1}{\PYGZsq{}}\PYG{p}{)}
\end{sphinxVerbatim}

\end{sphinxuseclass}\end{sphinxVerbatimInput}

\end{sphinxuseclass}
\end{sphinxuseclass}

\subsection{Question}
\label{\detokenize{conditionals_output:id17}}
\begin{sphinxuseclass}{cell}
\begin{sphinxuseclass}{tag_hide-output}\begin{sphinxVerbatimInput}

\begin{sphinxuseclass}{cell_input}
\begin{sphinxVerbatim}[commandchars=\\\{\}]
\PYG{n}{x} \PYG{o}{=} \PYG{k+kc}{True}

\PYG{k}{try}\PYG{p}{:}
  \PYG{n+nb}{print}\PYG{p}{(}\PYG{n}{x}\PYG{o}{+}\PYG{l+m+mi}{5}\PYG{p}{)}
\PYG{k}{except}\PYG{p}{:}
  \PYG{n+nb}{print}\PYG{p}{(}\PYG{l+s+s1}{\PYGZsq{}}\PYG{l+s+s1}{ERROR}\PYG{l+s+s1}{\PYGZsq{}}\PYG{p}{)}
\end{sphinxVerbatim}

\end{sphinxuseclass}\end{sphinxVerbatimInput}

\end{sphinxuseclass}
\end{sphinxuseclass}
\sphinxstepscope


\section{Conditionals Code}
\label{\detokenize{conditionals_code:conditionals-code}}\label{\detokenize{conditionals_code::doc}}\begin{itemize}
\item {} 
\sphinxAtStartPar
Please solve the following questions using Python code.  

\end{itemize}


\subsection{Question}
\label{\detokenize{conditionals_code:question}}
\sphinxAtStartPar
Write a program that asks the user to enter a 6\sphinxhyphen{}digit number.
\begin{itemize}
\item {} 
\sphinxAtStartPar
Check whether the digit in the thousands place is even.

\end{itemize}

\sphinxAtStartPar
\sphinxstylestrong{Solution}

\begin{sphinxadmonition}{note}{Solution}

\begin{sphinxVerbatim}[commandchars=\\\{\}]
\PYG{n}{number} \PYG{o}{=} \PYG{n+nb}{input}\PYG{p}{(}\PYG{l+s+s1}{\PYGZsq{}}\PYG{l+s+s1}{Please enter a 6 digit number: }\PYG{l+s+s1}{\PYGZsq{}}\PYG{p}{)}

\PYG{n}{thousands\PYGZus{}digit} \PYG{o}{=} \PYG{n+nb}{int}\PYG{p}{(}\PYG{n}{number}\PYG{p}{[}\PYG{l+m+mi}{2}\PYG{p}{]}\PYG{p}{)}

\PYG{k}{if} \PYG{n}{thousands\PYGZus{}digit}\PYG{o}{\PYGZpc{}}\PYG{l+m+mi}{2}\PYG{p}{:}
    \PYG{n+nb}{print}\PYG{p}{(}\PYG{l+s+sa}{f}\PYG{l+s+s1}{\PYGZsq{}}\PYG{l+s+s1}{Thousands digit is odd.}\PYG{l+s+s1}{\PYGZsq{}}\PYG{p}{)}
\PYG{k}{else}\PYG{p}{:}
    \PYG{n+nb}{print}\PYG{p}{(}\PYG{l+s+sa}{f}\PYG{l+s+s1}{\PYGZsq{}}\PYG{l+s+s1}{Thousands digit is even.}\PYG{l+s+s1}{\PYGZsq{}}\PYG{p}{)}
\end{sphinxVerbatim}
\end{sphinxadmonition}


\subsection{Question}
\label{\detokenize{conditionals_code:id1}}
\sphinxAtStartPar
Write a program that asks the user to enter a percent grade.
\begin{itemize}
\item {} 
\sphinxAtStartPar
Display the corresponding “Letter Grade” according to the following grading scale.

\end{itemize}


\begin{savenotes}\sphinxattablestart
\centering
\begin{tabulary}{\linewidth}[t]{|T|T|}
\hline
\sphinxstyletheadfamily 
\sphinxAtStartPar
Letter Grade
&\sphinxstyletheadfamily 
\sphinxAtStartPar
Grade Range
\\
\hline
\sphinxAtStartPar
A
&
\sphinxAtStartPar
96 \sphinxhyphen{} 100
\\
\hline
\sphinxAtStartPar
A\sphinxhyphen{}
&
\sphinxAtStartPar
90 \sphinxhyphen{} 95
\\
\hline
\sphinxAtStartPar
B+
&
\sphinxAtStartPar
87 \sphinxhyphen{} 89
\\
\hline
\sphinxAtStartPar
B
&
\sphinxAtStartPar
84 \sphinxhyphen{} 86
\\
\hline
\sphinxAtStartPar
B\sphinxhyphen{}
&
\sphinxAtStartPar
80 \sphinxhyphen{} 83
\\
\hline
\sphinxAtStartPar
C+
&
\sphinxAtStartPar
77 \sphinxhyphen{} 79
\\
\hline
\sphinxAtStartPar
C
&
\sphinxAtStartPar
74 \sphinxhyphen{} 76
\\
\hline
\sphinxAtStartPar
C\sphinxhyphen{}
&
\sphinxAtStartPar
70 \sphinxhyphen{} 73
\\
\hline
\sphinxAtStartPar
D+
&
\sphinxAtStartPar
67 \sphinxhyphen{} 69
\\
\hline
\sphinxAtStartPar
D
&
\sphinxAtStartPar
64 \sphinxhyphen{} 66
\\
\hline
\sphinxAtStartPar
D\sphinxhyphen{}
&
\sphinxAtStartPar
60 \sphinxhyphen{} 63
\\
\hline
\sphinxAtStartPar
F
&
\sphinxAtStartPar
0 \sphinxhyphen{} 59
\\
\hline
\end{tabulary}
\par
\sphinxattableend\end{savenotes}

\sphinxAtStartPar
\sphinxstylestrong{Solution}

\begin{sphinxadmonition}{note}{Solution}

\begin{sphinxVerbatim}[commandchars=\\\{\}]
\PYG{n}{grade} \PYG{o}{=} \PYG{n+nb}{float}\PYG{p}{(}\PYG{n+nb}{input}\PYG{p}{(}\PYG{l+s+s1}{\PYGZsq{}}\PYG{l+s+s1}{Please enter your percent grade: }\PYG{l+s+s1}{\PYGZsq{}}\PYG{p}{)}\PYG{p}{)}

\PYG{k}{if} \PYG{n}{grade} \PYG{o}{\PYGZgt{}}\PYG{o}{=} \PYG{l+m+mi}{96}\PYG{p}{:} \PYG{n+nb}{print}\PYG{p}{(}\PYG{l+s+s1}{\PYGZsq{}}\PYG{l+s+s1}{A}\PYG{l+s+s1}{\PYGZsq{}}\PYG{p}{)}
\PYG{k}{elif} \PYG{n}{grade} \PYG{o}{\PYGZgt{}}\PYG{o}{=} \PYG{l+m+mi}{90}\PYG{p}{:} \PYG{n+nb}{print}\PYG{p}{(}\PYG{l+s+s1}{\PYGZsq{}}\PYG{l+s+s1}{A\PYGZhy{}}\PYG{l+s+s1}{\PYGZsq{}}\PYG{p}{)}
\PYG{k}{elif} \PYG{n}{grade} \PYG{o}{\PYGZgt{}}\PYG{o}{=} \PYG{l+m+mi}{87}\PYG{p}{:} \PYG{n+nb}{print}\PYG{p}{(}\PYG{l+s+s1}{\PYGZsq{}}\PYG{l+s+s1}{B+}\PYG{l+s+s1}{\PYGZsq{}}\PYG{p}{)}
\PYG{k}{elif} \PYG{n}{grade} \PYG{o}{\PYGZgt{}}\PYG{o}{=} \PYG{l+m+mi}{84}\PYG{p}{:} \PYG{n+nb}{print}\PYG{p}{(}\PYG{l+s+s1}{\PYGZsq{}}\PYG{l+s+s1}{B}\PYG{l+s+s1}{\PYGZsq{}}\PYG{p}{)}
\PYG{k}{elif} \PYG{n}{grade} \PYG{o}{\PYGZgt{}}\PYG{o}{=} \PYG{l+m+mi}{80}\PYG{p}{:} \PYG{n+nb}{print}\PYG{p}{(}\PYG{l+s+s1}{\PYGZsq{}}\PYG{l+s+s1}{B\PYGZhy{}}\PYG{l+s+s1}{\PYGZsq{}}\PYG{p}{)}
\PYG{k}{elif} \PYG{n}{grade} \PYG{o}{\PYGZgt{}}\PYG{o}{=} \PYG{l+m+mi}{77}\PYG{p}{:} \PYG{n+nb}{print}\PYG{p}{(}\PYG{l+s+s1}{\PYGZsq{}}\PYG{l+s+s1}{C+}\PYG{l+s+s1}{\PYGZsq{}}\PYG{p}{)}
\PYG{k}{elif} \PYG{n}{grade} \PYG{o}{\PYGZgt{}}\PYG{o}{=} \PYG{l+m+mi}{74}\PYG{p}{:} \PYG{n+nb}{print}\PYG{p}{(}\PYG{l+s+s1}{\PYGZsq{}}\PYG{l+s+s1}{C}\PYG{l+s+s1}{\PYGZsq{}}\PYG{p}{)}
\PYG{k}{elif} \PYG{n}{grade} \PYG{o}{\PYGZgt{}}\PYG{o}{=} \PYG{l+m+mi}{70}\PYG{p}{:} \PYG{n+nb}{print}\PYG{p}{(}\PYG{l+s+s1}{\PYGZsq{}}\PYG{l+s+s1}{C\PYGZhy{}}\PYG{l+s+s1}{\PYGZsq{}}\PYG{p}{)}
\PYG{k}{elif} \PYG{n}{grade} \PYG{o}{\PYGZgt{}}\PYG{o}{=} \PYG{l+m+mi}{67}\PYG{p}{:} \PYG{n+nb}{print}\PYG{p}{(}\PYG{l+s+s1}{\PYGZsq{}}\PYG{l+s+s1}{D+}\PYG{l+s+s1}{\PYGZsq{}}\PYG{p}{)}
\PYG{k}{elif} \PYG{n}{grade} \PYG{o}{\PYGZgt{}}\PYG{o}{=} \PYG{l+m+mi}{64}\PYG{p}{:} \PYG{n+nb}{print}\PYG{p}{(}\PYG{l+s+s1}{\PYGZsq{}}\PYG{l+s+s1}{D}\PYG{l+s+s1}{\PYGZsq{}}\PYG{p}{)}
\PYG{k}{elif} \PYG{n}{grade} \PYG{o}{\PYGZgt{}}\PYG{o}{=} \PYG{l+m+mi}{60}\PYG{p}{:} \PYG{n+nb}{print}\PYG{p}{(}\PYG{l+s+s1}{\PYGZsq{}}\PYG{l+s+s1}{D\PYGZhy{}}\PYG{l+s+s1}{\PYGZsq{}}\PYG{p}{)}
\PYG{k}{else}\PYG{p}{:} \PYG{n+nb}{print}\PYG{p}{(}\PYG{l+s+s1}{\PYGZsq{}}\PYG{l+s+s1}{F}\PYG{l+s+s1}{\PYGZsq{}}\PYG{p}{)}
\end{sphinxVerbatim}
\end{sphinxadmonition}


\subsection{Question}
\label{\detokenize{conditionals_code:id2}}
\sphinxAtStartPar
The piecewise\sphinxhyphen{}defined function \(f(n)\) is given  as follows:
\begin{equation*}
\begin{split} f(n) =
  \begin{cases}
                                   n^2+3n+4 &  n > 7  \\
                                   3n-7     &  7 \ge n >4 \\
                                   10       &  4 \ge n \ge 2\\
                                   3-4n     &  \text{otherwise}\\
  \end{cases}
\end{split}
\end{equation*}\begin{itemize}
\item {} 
\sphinxAtStartPar
Write a program that asks the user to enter an integer.

\item {} 
\sphinxAtStartPar
If the integer entered is n then display f(n).

\item {} 
\sphinxAtStartPar
Hint:
\begin{itemize}
\item {} 
\sphinxAtStartPar
If n is greater than 7 then f(n)= \(n^2+3n+4\).

\item {} 
\sphinxAtStartPar
If n is between 4 (4 is not included) and 7 (7 is included)  then f(n)=3n\sphinxhyphen{}7.

\item {} 
\sphinxAtStartPar
If n is between 2 (2 is not included) and 4 (4 is included)  then f(n)=10.

\item {} 
\sphinxAtStartPar
If n is less than 2 then f(n)=3\sphinxhyphen{}4n.

\end{itemize}

\end{itemize}

\sphinxAtStartPar
\sphinxstylestrong{Solution}

\begin{sphinxadmonition}{note}{Solution}

\begin{sphinxVerbatim}[commandchars=\\\{\}]
\PYG{n}{n} \PYG{o}{=} \PYG{n+nb}{int}\PYG{p}{(} \PYG{n+nb}{input}\PYG{p}{(}\PYG{l+s+s1}{\PYGZsq{}}\PYG{l+s+s1}{Enter an integer: }\PYG{l+s+s1}{\PYGZsq{}}\PYG{p}{)}  \PYG{p}{)}

\PYG{k}{if} \PYG{n}{n} \PYG{o}{\PYGZgt{}} \PYG{l+m+mi}{7}\PYG{p}{:}
    \PYG{n+nb}{print}\PYG{p}{(}\PYG{l+s+sa}{f}\PYG{l+s+s1}{\PYGZsq{}}\PYG{l+s+s1}{f(n) = }\PYG{l+s+si}{\PYGZob{}}\PYG{n}{n}\PYG{o}{*}\PYG{o}{*}\PYG{l+m+mi}{2}\PYG{o}{+}\PYG{l+m+mi}{3}\PYG{o}{*}\PYG{n}{n}\PYG{o}{+}\PYG{l+m+mi}{4}\PYG{l+s+si}{\PYGZcb{}}\PYG{l+s+s1}{\PYGZsq{}}\PYG{p}{)}
\PYG{k}{elif} \PYG{l+m+mi}{4} \PYG{o}{\PYGZlt{}} \PYG{n}{n}\PYG{p}{:}
    \PYG{n+nb}{print}\PYG{p}{(}\PYG{l+s+sa}{f}\PYG{l+s+s1}{\PYGZsq{}}\PYG{l+s+s1}{f(n) = }\PYG{l+s+si}{\PYGZob{}}\PYG{l+m+mi}{3}\PYG{o}{*}\PYG{n}{n}\PYG{o}{\PYGZhy{}}\PYG{l+m+mi}{7}\PYG{l+s+si}{\PYGZcb{}}\PYG{l+s+s1}{\PYGZsq{}}\PYG{p}{)}
\PYG{k}{elif} \PYG{l+m+mi}{2} \PYG{o}{\PYGZlt{}}\PYG{o}{=} \PYG{n}{n}\PYG{p}{:}
    \PYG{n+nb}{print}\PYG{p}{(}\PYG{l+s+sa}{f}\PYG{l+s+s1}{\PYGZsq{}}\PYG{l+s+s1}{f(n) = 10}\PYG{l+s+s1}{\PYGZsq{}}\PYG{p}{)}
\PYG{k}{else}\PYG{p}{:}
    \PYG{n+nb}{print}\PYG{p}{(}\PYG{l+s+sa}{f}\PYG{l+s+s1}{\PYGZsq{}}\PYG{l+s+s1}{f(}\PYG{l+s+si}{\PYGZob{}}\PYG{n}{n}\PYG{l+s+si}{\PYGZcb{}}\PYG{l+s+s1}{) = }\PYG{l+s+si}{\PYGZob{}}\PYG{l+m+mi}{3}\PYG{o}{\PYGZhy{}}\PYG{l+m+mi}{4}\PYG{o}{*}\PYG{n}{n}\PYG{l+s+si}{\PYGZcb{}}\PYG{l+s+s1}{\PYGZsq{}}\PYG{p}{)}
\end{sphinxVerbatim}
\end{sphinxadmonition}


\subsection{Question}
\label{\detokenize{conditionals_code:id3}}
\sphinxAtStartPar
Temperature Converter: Fahrenheit (F) <—> Celsius (C).
\begin{itemize}
\item {} 
\sphinxAtStartPar
Ask for temperature with unit (F or C) from the user.
\begin{itemize}
\item {} 
\sphinxAtStartPar
Use only one input function.

\item {} 
\sphinxAtStartPar
The user should enter the temperature in the form (int)F or (int)C.
\begin{itemize}
\item {} 
\sphinxAtStartPar
Example: 37F or 42C

\end{itemize}

\end{itemize}

\item {} 
\sphinxAtStartPar
If the temperature is given in Fahrenheit (F) by the user, then convert it to Celsius (C).

\item {} 
\sphinxAtStartPar
If the temperature is given in Celsius (C) by the user, then convert it to Fahrenheit (F).

\item {} 
\sphinxAtStartPar
Display the converted temperature with its unit.

\end{itemize}

\sphinxAtStartPar
\sphinxstylestrong{Solution}

\begin{sphinxadmonition}{note}{Solution}

\begin{sphinxVerbatim}[commandchars=\\\{\}]
\PYG{n}{temperature\PYGZus{}unit} \PYG{o}{=} \PYG{n+nb}{input}\PYG{p}{(}\PYG{l+s+s1}{\PYGZsq{}}\PYG{l+s+s1}{Please enter the temperature with the unit: }\PYG{l+s+s1}{\PYGZsq{}}\PYG{p}{)}

\PYG{n}{temp} \PYG{o}{=} \PYG{n+nb}{int}\PYG{p}{(}\PYG{n}{temperature\PYGZus{}unit}\PYG{p}{[}\PYG{p}{:}\PYG{o}{\PYGZhy{}}\PYG{l+m+mi}{1}\PYG{p}{]}\PYG{p}{)}     \PYG{c+c1}{\PYGZsh{} the slice, except for the last character, represents the numerical value of the temperature.}
\PYG{n}{unit} \PYG{o}{=} \PYG{n}{temperature\PYGZus{}unit}\PYG{p}{[}\PYG{o}{\PYGZhy{}}\PYG{l+m+mi}{1}\PYG{p}{]}           \PYG{c+c1}{\PYGZsh{} the last character is the unit}

\PYG{k}{if} \PYG{n}{unit} \PYG{o}{==} \PYG{l+s+s1}{\PYGZsq{}}\PYG{l+s+s1}{F}\PYG{l+s+s1}{\PYGZsq{}}\PYG{p}{:}
    \PYG{n+nb}{print}\PYG{p}{(}\PYG{l+s+sa}{f}\PYG{l+s+s1}{\PYGZsq{}}\PYG{l+s+s1}{It is }\PYG{l+s+si}{\PYGZob{}}\PYG{p}{(}\PYG{n}{temp}\PYG{o}{\PYGZhy{}}\PYG{l+m+mi}{32}\PYG{p}{)}\PYG{o}{/}\PYG{l+m+mf}{1.8}\PYG{l+s+si}{\PYGZcb{}}\PYG{l+s+s1}{C.}\PYG{l+s+s1}{\PYGZsq{}}\PYG{p}{)}
\PYG{k}{else}\PYG{p}{:}
    \PYG{n+nb}{print}\PYG{p}{(}\PYG{l+s+sa}{f}\PYG{l+s+s1}{\PYGZsq{}}\PYG{l+s+s1}{It is }\PYG{l+s+si}{\PYGZob{}}\PYG{n}{temp}\PYG{o}{*}\PYG{l+m+mf}{1.8}\PYG{o}{+}\PYG{l+m+mi}{32}\PYG{l+s+si}{\PYGZcb{}}\PYG{l+s+s1}{F.}\PYG{l+s+s1}{\PYGZsq{}}\PYG{p}{)}
\end{sphinxVerbatim}
\end{sphinxadmonition}


\subsection{Question}
\label{\detokenize{conditionals_code:id4}}
\sphinxAtStartPar
Ask the user to input a non\sphinxhyphen{}negative integer.
\begin{itemize}
\item {} 
\sphinxAtStartPar
Check if this integer is a perfect square and print your conclusion.
\begin{itemize}
\item {} 
\sphinxAtStartPar
Perfect squares are squares of integers, for example: 0, 1, 4, 9, 16, 25, 36, 49, etc.

\end{itemize}

\item {} 
\sphinxAtStartPar
If the given number is a negative integer, print a warning message.

\item {} 
\sphinxAtStartPar
Utilize the sqrt function from numpy or math.

\item {} 
\sphinxAtStartPar
Example:
\begin{itemize}
\item {} 
\sphinxAtStartPar
For number = \sphinxhyphen{}3, output: \sphinxhyphen{}3 < 0. Please enter a non\sphinxhyphen{}negative integer.

\item {} 
\sphinxAtStartPar
For number = 4, output: 4 is a perfect square.

\item {} 
\sphinxAtStartPar
For number = 12, output: 12 is NOT a perfect square.

\end{itemize}

\end{itemize}

\sphinxAtStartPar
\sphinxstylestrong{Solution}

\begin{sphinxadmonition}{note}{Solution}

\begin{sphinxVerbatim}[commandchars=\\\{\}]
\PYG{k+kn}{import} \PYG{n+nn}{math}
\PYG{n}{number} \PYG{o}{=} \PYG{n+nb}{int}\PYG{p}{(}\PYG{n+nb}{input}\PYG{p}{(}\PYG{l+s+s1}{\PYGZsq{}}\PYG{l+s+s1}{Enter a non\PYGZhy{}negative integer: }\PYG{l+s+s1}{\PYGZsq{}}\PYG{p}{)}\PYG{p}{)}

\PYG{k}{if} \PYG{n}{number} \PYG{o}{\PYGZlt{}} \PYG{l+m+mi}{0}\PYG{p}{:}
  \PYG{n+nb}{print}\PYG{p}{(}\PYG{l+s+sa}{f}\PYG{l+s+s1}{\PYGZsq{}}\PYG{l+s+si}{\PYGZob{}}\PYG{n}{number}\PYG{l+s+si}{\PYGZcb{}}\PYG{l+s+s1}{ \PYGZlt{} 0. Please enter a non\PYGZhy{}negative integer.}\PYG{l+s+s1}{\PYGZsq{}}\PYG{p}{)}
\PYG{k}{else}\PYG{p}{:}
  \PYG{n}{int\PYGZus{}sqrt} \PYG{o}{=} \PYG{n+nb}{int}\PYG{p}{(}\PYG{n}{math}\PYG{o}{.}\PYG{n}{sqrt}\PYG{p}{(}\PYG{n}{number}\PYG{p}{)}\PYG{p}{)}           \PYG{c+c1}{\PYGZsh{} if a number is a perfect square, its square root is an integer}
  \PYG{k}{if} \PYG{n}{int\PYGZus{}sqrt}\PYG{o}{*}\PYG{o}{*}\PYG{l+m+mi}{2} \PYG{o}{==} \PYG{n}{number}\PYG{p}{:}                   \PYG{c+c1}{\PYGZsh{} so the square root\PYGZsq{}s integer part is itself}
    \PYG{n+nb}{print}\PYG{p}{(}\PYG{l+s+sa}{f}\PYG{l+s+s1}{\PYGZsq{}}\PYG{l+s+si}{\PYGZob{}}\PYG{n}{number}\PYG{l+s+si}{\PYGZcb{}}\PYG{l+s+s1}{ is a perfect square.}\PYG{l+s+s1}{\PYGZsq{}}\PYG{p}{)}
  \PYG{k}{else}\PYG{p}{:}
    \PYG{n+nb}{print}\PYG{p}{(}\PYG{l+s+sa}{f}\PYG{l+s+s1}{\PYGZsq{}}\PYG{l+s+si}{\PYGZob{}}\PYG{n}{number}\PYG{l+s+si}{\PYGZcb{}}\PYG{l+s+s1}{ is a NOT perfect square.}\PYG{l+s+s1}{\PYGZsq{}}\PYG{p}{)}
\end{sphinxVerbatim}
\end{sphinxadmonition}


\subsection{Question}
\label{\detokenize{conditionals_code:id5}}
\sphinxAtStartPar
Ask for a name from the user.
\begin{itemize}
\item {} 
\sphinxAtStartPar
Check if it contains the letters ‘k’ or ‘K’ and if the length of the name is an odd number.

\item {} 
\sphinxAtStartPar
If this is the case, then replace ‘k’ with ‘y’ and ‘K’ with ‘Y’.

\item {} 
\sphinxAtStartPar
Display the new name.

\item {} 
\sphinxAtStartPar
Example:
\begin{itemize}
\item {} 
\sphinxAtStartPar
name = Tom   Output: Tom

\item {} 
\sphinxAtStartPar
name = Jack  Output: Jack     (length is even)

\item {} 
\sphinxAtStartPar
name = KaThy Output: YaThy

\item {} 
\sphinxAtStartPar
name = katHy Output: yatHy

\end{itemize}

\end{itemize}

\sphinxAtStartPar
\sphinxstylestrong{Solution}

\begin{sphinxadmonition}{note}{Solution}

\begin{sphinxVerbatim}[commandchars=\\\{\}]
\PYG{n}{name} \PYG{o}{=} \PYG{n+nb}{input}\PYG{p}{(}\PYG{l+s+s1}{\PYGZsq{}}\PYG{l+s+s1}{Enter a name: }\PYG{l+s+s1}{\PYGZsq{}}\PYG{p}{)}

\PYG{k}{if} \PYG{p}{(}\PYG{l+s+s1}{\PYGZsq{}}\PYG{l+s+s1}{k}\PYG{l+s+s1}{\PYGZsq{}} \PYG{o+ow}{in} \PYG{n}{name}\PYG{o}{.}\PYG{n}{lower}\PYG{p}{(}\PYG{p}{)}\PYG{p}{)} \PYG{o}{\PYGZam{}} \PYG{p}{(}\PYG{n+nb}{len}\PYG{p}{(}\PYG{n}{name}\PYG{p}{)}\PYG{o}{\PYGZpc{}}\PYG{l+m+mi}{2}\PYG{o}{==}\PYG{l+m+mi}{1}\PYG{p}{)}\PYG{p}{:}
  \PYG{n+nb}{print}\PYG{p}{(}\PYG{n}{name}\PYG{o}{.}\PYG{n}{replace}\PYG{p}{(}\PYG{l+s+s1}{\PYGZsq{}}\PYG{l+s+s1}{k}\PYG{l+s+s1}{\PYGZsq{}}\PYG{p}{,}\PYG{l+s+s1}{\PYGZsq{}}\PYG{l+s+s1}{y}\PYG{l+s+s1}{\PYGZsq{}}\PYG{p}{)}\PYG{o}{.}\PYG{n}{replace}\PYG{p}{(}\PYG{l+s+s1}{\PYGZsq{}}\PYG{l+s+s1}{K}\PYG{l+s+s1}{\PYGZsq{}}\PYG{p}{,}\PYG{l+s+s1}{\PYGZsq{}}\PYG{l+s+s1}{Y}\PYG{l+s+s1}{\PYGZsq{}}\PYG{p}{)}\PYG{p}{)}
\PYG{k}{else}\PYG{p}{:}
  \PYG{n+nb}{print}\PYG{p}{(}\PYG{n}{name}\PYG{p}{)}
\end{sphinxVerbatim}
\end{sphinxadmonition}


\subsection{Question}
\label{\detokenize{conditionals_code:id6}}
\sphinxAtStartPar
Ask for a number from the user
\begin{itemize}
\item {} 
\sphinxAtStartPar
Check whether the given number satisfies the following inequality:
\$\(x^2-3x+6>192\)\$

\item {} 
\sphinxAtStartPar
Display the conclusion.

\end{itemize}

\sphinxAtStartPar
\sphinxstylestrong{Solution\sphinxhyphen{}1}

\begin{sphinxadmonition}{note}{Solution}

\begin{sphinxVerbatim}[commandchars=\\\{\}]
\PYG{n}{x} \PYG{o}{=} \PYG{n+nb}{float}\PYG{p}{(}\PYG{n+nb}{input}\PYG{p}{(}\PYG{l+s+s1}{\PYGZsq{}}\PYG{l+s+s1}{Enter a number: }\PYG{l+s+s1}{\PYGZsq{}}\PYG{p}{)}\PYG{p}{)}

\PYG{n}{y} \PYG{o}{=} \PYG{n}{x}\PYG{o}{*}\PYG{o}{*}\PYG{l+m+mi}{2}\PYG{o}{\PYGZhy{}}\PYG{l+m+mi}{3}\PYG{o}{*}\PYG{n}{x}\PYG{o}{+}\PYG{l+m+mi}{6}

\PYG{k}{if} \PYG{n}{y} \PYG{o}{\PYGZgt{}} \PYG{l+m+mi}{192}\PYG{p}{:}
  \PYG{n+nb}{print}\PYG{p}{(}\PYG{l+s+sa}{f}\PYG{l+s+s1}{\PYGZsq{}}\PYG{l+s+si}{\PYGZob{}}\PYG{n}{x}\PYG{l+s+si}{\PYGZcb{}}\PYG{l+s+s1}{ satisfies the inequality}\PYG{l+s+s1}{\PYGZsq{}}\PYG{p}{)}
\PYG{k}{else}\PYG{p}{:}
  \PYG{n+nb}{print}\PYG{p}{(}\PYG{l+s+sa}{f}\PYG{l+s+s1}{\PYGZsq{}}\PYG{l+s+si}{\PYGZob{}}\PYG{n}{x}\PYG{l+s+si}{\PYGZcb{}}\PYG{l+s+s1}{ does NOT satisfy the inequality}\PYG{l+s+s1}{\PYGZsq{}}\PYG{p}{)}
\end{sphinxVerbatim}
\end{sphinxadmonition}

\sphinxAtStartPar
\sphinxstylestrong{Solution\sphinxhyphen{}2}

\begin{sphinxadmonition}{note}{Solution}

\begin{sphinxVerbatim}[commandchars=\\\{\}]
\PYG{n}{x} \PYG{o}{=} \PYG{n+nb}{float}\PYG{p}{(}\PYG{n+nb}{input}\PYG{p}{(}\PYG{l+s+s1}{\PYGZsq{}}\PYG{l+s+s1}{Enter a number: }\PYG{l+s+s1}{\PYGZsq{}}\PYG{p}{)}\PYG{p}{)}

\PYG{n}{inequality} \PYG{o}{=} \PYG{p}{(}\PYG{n}{x}\PYG{o}{*}\PYG{o}{*}\PYG{l+m+mi}{2}\PYG{o}{\PYGZhy{}}\PYG{l+m+mi}{3}\PYG{o}{*}\PYG{n}{x}\PYG{o}{+}\PYG{l+m+mi}{6}\PYG{p}{)} \PYG{o}{\PYGZgt{}} \PYG{l+m+mi}{192}

\PYG{k}{if} \PYG{n}{inequality}\PYG{p}{:}
  \PYG{n+nb}{print}\PYG{p}{(}\PYG{l+s+sa}{f}\PYG{l+s+s1}{\PYGZsq{}}\PYG{l+s+si}{\PYGZob{}}\PYG{n}{x}\PYG{l+s+si}{\PYGZcb{}}\PYG{l+s+s1}{ satisfies the inequality}\PYG{l+s+s1}{\PYGZsq{}}\PYG{p}{)}
\PYG{k}{else}\PYG{p}{:}
  \PYG{n+nb}{print}\PYG{p}{(}\PYG{l+s+sa}{f}\PYG{l+s+s1}{\PYGZsq{}}\PYG{l+s+si}{\PYGZob{}}\PYG{n}{x}\PYG{l+s+si}{\PYGZcb{}}\PYG{l+s+s1}{ does NOT satisfy the inequality}\PYG{l+s+s1}{\PYGZsq{}}\PYG{p}{)}
\end{sphinxVerbatim}
\end{sphinxadmonition}


\subsection{Question}
\label{\detokenize{conditionals_code:id7}}
\sphinxAtStartPar
Ask for Test\sphinxhyphen{}1, Test\sphinxhyphen{}2, and Final exam grades from the user.
\begin{itemize}
\item {} 
\sphinxAtStartPar
Use three input functions.

\item {} 
\sphinxAtStartPar
Compute the weighted average by using the following formula:
\begin{itemize}
\item {} 
\sphinxAtStartPar
Weighted Average = 0.2 x Test\sphinxhyphen{}1 + 0.3 x Test\sphinxhyphen{}2 + 0.5 x Test\sphinxhyphen{}3

\end{itemize}

\item {} 
\sphinxAtStartPar
Find the letter grade by using the following grading scale.

\end{itemize}


\begin{savenotes}\sphinxattablestart
\centering
\begin{tabulary}{\linewidth}[t]{|T|T|}
\hline
\sphinxstyletheadfamily 
\sphinxAtStartPar
Weighted Average
&\sphinxstyletheadfamily 
\sphinxAtStartPar
Letter Grade
\\
\hline
\sphinxAtStartPar
75\sphinxhyphen{}100
&
\sphinxAtStartPar
A
\\
\hline
\sphinxAtStartPar
60\sphinxhyphen{}74
&
\sphinxAtStartPar
B
\\
\hline
\sphinxAtStartPar
40\sphinxhyphen{}59
&
\sphinxAtStartPar
C
\\
\hline
\sphinxAtStartPar
0\sphinxhyphen{}39
&
\sphinxAtStartPar
F
\\
\hline
\end{tabulary}
\par
\sphinxattableend\end{savenotes}
\begin{itemize}
\item {} 
\sphinxAtStartPar
Example:
\begin{itemize}
\item {} 
\sphinxAtStartPar
Test\sphinxhyphen{}1 grade: 70

\item {} 
\sphinxAtStartPar
Test\sphinxhyphen{}2 grade: 80

\item {} 
\sphinxAtStartPar
Final  grade: 90

\item {} 
\sphinxAtStartPar
Weighted average = \(0.2 \times 70 + 0.3 \times 80 + 0.5 \times 90 = 14 + 24 + 45 = 83\)

\item {} 
\sphinxAtStartPar
Output:
\begin{itemize}
\item {} 
\sphinxAtStartPar
Weighted Average 83.0   —\sphinxhyphen{}>   Letter Grade: A

\end{itemize}

\end{itemize}

\end{itemize}

\sphinxAtStartPar
\sphinxstylestrong{Solution}

\begin{sphinxadmonition}{note}{Solution}

\begin{sphinxVerbatim}[commandchars=\\\{\}]

\PYG{n}{test1} \PYG{o}{=} \PYG{n+nb}{float}\PYG{p}{(}\PYG{n+nb}{input}\PYG{p}{(}\PYG{l+s+s1}{\PYGZsq{}}\PYG{l+s+s1}{Test\PYGZhy{}1 grade: }\PYG{l+s+s1}{\PYGZsq{}}\PYG{p}{)}\PYG{p}{)}
\PYG{n}{test2} \PYG{o}{=} \PYG{n+nb}{float}\PYG{p}{(}\PYG{n+nb}{input}\PYG{p}{(}\PYG{l+s+s1}{\PYGZsq{}}\PYG{l+s+s1}{Test\PYGZhy{}2 grade: }\PYG{l+s+s1}{\PYGZsq{}}\PYG{p}{)}\PYG{p}{)}
\PYG{n}{final} \PYG{o}{=} \PYG{n+nb}{float}\PYG{p}{(}\PYG{n+nb}{input}\PYG{p}{(}\PYG{l+s+s1}{\PYGZsq{}}\PYG{l+s+s1}{Final  grade: }\PYG{l+s+s1}{\PYGZsq{}}\PYG{p}{)}\PYG{p}{)}

\PYG{n}{weighted\PYGZus{}grade} \PYG{o}{=} \PYG{n}{test1}\PYG{o}{*}\PYG{l+m+mf}{0.2}\PYG{o}{+}\PYG{n}{test2}\PYG{o}{*}\PYG{l+m+mf}{0.3}\PYG{o}{+}\PYG{n}{final}\PYG{o}{*}\PYG{l+m+mf}{0.5}

\PYG{n+nb}{print}\PYG{p}{(}\PYG{l+s+sa}{f}\PYG{l+s+s1}{\PYGZsq{}}\PYG{l+s+s1}{Weighted Average }\PYG{l+s+si}{\PYGZob{}}\PYG{n}{weighted\PYGZus{}grade}\PYG{l+s+si}{\PYGZcb{}}\PYG{l+s+s1}{   \PYGZhy{}\PYGZhy{}\PYGZhy{}\PYGZhy{}\PYGZgt{}   Letter Grade: }\PYG{l+s+s1}{\PYGZsq{}}\PYG{p}{,} \PYG{n}{end}\PYG{o}{=}\PYG{l+s+s1}{\PYGZsq{}}\PYG{l+s+s1}{\PYGZsq{}}\PYG{p}{)}

\PYG{k}{if}   \PYG{l+m+mi}{100} \PYG{o}{\PYGZgt{}}\PYG{o}{=} \PYG{n}{weighted\PYGZus{}grade} \PYG{o}{\PYGZgt{}}\PYG{o}{=} \PYG{l+m+mi}{75}\PYG{p}{:} 
    \PYG{n+nb}{print}\PYG{p}{(}\PYG{l+s+s1}{\PYGZsq{}}\PYG{l+s+s1}{A}\PYG{l+s+s1}{\PYGZsq{}}\PYG{p}{)}
\PYG{k}{elif} \PYG{n}{weighted\PYGZus{}grade} \PYG{o}{\PYGZgt{}}\PYG{o}{=} \PYG{l+m+mi}{60}\PYG{p}{:} 
    \PYG{n+nb}{print}\PYG{p}{(}\PYG{l+s+s1}{\PYGZsq{}}\PYG{l+s+s1}{B}\PYG{l+s+s1}{\PYGZsq{}}\PYG{p}{)}
\PYG{k}{elif} \PYG{n}{weighted\PYGZus{}grade} \PYG{o}{\PYGZgt{}}\PYG{o}{=} \PYG{l+m+mi}{40}\PYG{p}{:}
    \PYG{n+nb}{print}\PYG{p}{(}\PYG{l+s+s1}{\PYGZsq{}}\PYG{l+s+s1}{C}\PYG{l+s+s1}{\PYGZsq{}}\PYG{p}{)}
\PYG{k}{else}\PYG{p}{:}
    \PYG{n+nb}{print}\PYG{p}{(}\PYG{l+s+s1}{\PYGZsq{}}\PYG{l+s+s1}{F}\PYG{l+s+s1}{\PYGZsq{}}\PYG{p}{)}
\end{sphinxVerbatim}
\end{sphinxadmonition}


\subsection{Question}
\label{\detokenize{conditionals_code:id8}}\begin{itemize}
\item {} 
\sphinxAtStartPar
Print the statement: Basic Calculator

\item {} 
\sphinxAtStartPar
Print 20 dashes (\sphinxhyphen{})

\item {} 
\sphinxAtStartPar
Ask the user for two numbers using two input functions

\item {} 
\sphinxAtStartPar
Ask the user for the operation to perform
\begin{itemize}
\item {} 
\sphinxAtStartPar
Provide the options: +, \sphinxhyphen{}, *, /

\item {} 
\sphinxAtStartPar
The user should choose one of them.

\end{itemize}

\item {} 
\sphinxAtStartPar
Find the result
\begin{itemize}
\item {} 
\sphinxAtStartPar
Perform the operation: First Number operation Second Number.

\item {} 
\sphinxAtStartPar
If the second number is zero, do not perform the division operation and display a warning message.

\item {} 
\sphinxAtStartPar
If the operation is division, also round the result.

\end{itemize}

\item {} 
\sphinxAtStartPar
Example: num1=5, num2=7, operation=’\sphinxstyleemphasis{’ —> Output: 5}7=35

\end{itemize}

\sphinxAtStartPar
\sphinxstylestrong{Solution}

\begin{sphinxadmonition}{note}{Solution}

\begin{sphinxVerbatim}[commandchars=\\\{\}]
\PYG{n+nb}{print}\PYG{p}{(}\PYG{l+s+s1}{\PYGZsq{}}\PYG{l+s+s1}{Basic Calculator}\PYG{l+s+s1}{\PYGZsq{}}\PYG{p}{)}
\PYG{n+nb}{print}\PYG{p}{(}\PYG{l+s+s1}{\PYGZsq{}}\PYG{l+s+s1}{\PYGZhy{}}\PYG{l+s+s1}{\PYGZsq{}}\PYG{o}{*}\PYG{l+m+mi}{20}\PYG{p}{)}

\PYG{n}{number1} \PYG{o}{=} \PYG{n+nb}{float}\PYG{p}{(}\PYG{n+nb}{input}\PYG{p}{(}\PYG{l+s+s1}{\PYGZsq{}}\PYG{l+s+s1}{Number\PYGZhy{}1: }\PYG{l+s+s1}{\PYGZsq{}}\PYG{p}{)}\PYG{p}{)}
\PYG{n}{number2} \PYG{o}{=} \PYG{n+nb}{float}\PYG{p}{(}\PYG{n+nb}{input}\PYG{p}{(}\PYG{l+s+s1}{\PYGZsq{}}\PYG{l+s+s1}{Number\PYGZhy{}2: }\PYG{l+s+s1}{\PYGZsq{}}\PYG{p}{)}\PYG{p}{)}
\PYG{n}{operation} \PYG{o}{=} \PYG{n+nb}{input}\PYG{p}{(}\PYG{l+s+s1}{\PYGZsq{}}\PYG{l+s+s1}{Operation: (+,\PYGZhy{},*,/): }\PYG{l+s+s1}{\PYGZsq{}}\PYG{p}{)}

\PYG{k}{if}   \PYG{n}{operation} \PYG{o}{==} \PYG{l+s+s1}{\PYGZsq{}}\PYG{l+s+s1}{+}\PYG{l+s+s1}{\PYGZsq{}}\PYG{p}{:} 
    \PYG{n+nb}{print}\PYG{p}{(}\PYG{l+s+sa}{f}\PYG{l+s+s1}{\PYGZsq{}}\PYG{l+s+si}{\PYGZob{}}\PYG{n}{number1}\PYG{l+s+si}{\PYGZcb{}}\PYG{l+s+s1}{ }\PYG{l+s+si}{\PYGZob{}}\PYG{n}{operation}\PYG{l+s+si}{\PYGZcb{}}\PYG{l+s+s1}{ }\PYG{l+s+si}{\PYGZob{}}\PYG{n}{number2}\PYG{l+s+si}{\PYGZcb{}}\PYG{l+s+s1}{ = }\PYG{l+s+si}{\PYGZob{}}\PYG{n}{number1}\PYG{+w}{ }\PYG{o}{+}\PYG{+w}{ }\PYG{n}{number2}\PYG{l+s+si}{\PYGZcb{}}\PYG{l+s+s1}{\PYGZsq{}}\PYG{p}{)}
\PYG{k}{elif} \PYG{n}{operation} \PYG{o}{==} \PYG{l+s+s1}{\PYGZsq{}}\PYG{l+s+s1}{\PYGZhy{}}\PYG{l+s+s1}{\PYGZsq{}}\PYG{p}{:} 
    \PYG{n+nb}{print}\PYG{p}{(}\PYG{l+s+sa}{f}\PYG{l+s+s1}{\PYGZsq{}}\PYG{l+s+si}{\PYGZob{}}\PYG{n}{number1}\PYG{l+s+si}{\PYGZcb{}}\PYG{l+s+s1}{ }\PYG{l+s+si}{\PYGZob{}}\PYG{n}{operation}\PYG{l+s+si}{\PYGZcb{}}\PYG{l+s+s1}{ }\PYG{l+s+si}{\PYGZob{}}\PYG{n}{number2}\PYG{l+s+si}{\PYGZcb{}}\PYG{l+s+s1}{ = }\PYG{l+s+si}{\PYGZob{}}\PYG{n}{number1}\PYG{+w}{ }\PYG{o}{\PYGZhy{}}\PYG{+w}{ }\PYG{n}{number2}\PYG{l+s+si}{\PYGZcb{}}\PYG{l+s+s1}{\PYGZsq{}}\PYG{p}{)}
\PYG{k}{elif} \PYG{n}{operation} \PYG{o}{==} \PYG{l+s+s1}{\PYGZsq{}}\PYG{l+s+s1}{*}\PYG{l+s+s1}{\PYGZsq{}}\PYG{p}{:} 
    \PYG{n+nb}{print}\PYG{p}{(}\PYG{l+s+sa}{f}\PYG{l+s+s1}{\PYGZsq{}}\PYG{l+s+si}{\PYGZob{}}\PYG{n}{number1}\PYG{l+s+si}{\PYGZcb{}}\PYG{l+s+s1}{ }\PYG{l+s+si}{\PYGZob{}}\PYG{n}{operation}\PYG{l+s+si}{\PYGZcb{}}\PYG{l+s+s1}{ }\PYG{l+s+si}{\PYGZob{}}\PYG{n}{number2}\PYG{l+s+si}{\PYGZcb{}}\PYG{l+s+s1}{ = }\PYG{l+s+si}{\PYGZob{}}\PYG{n}{number1}\PYG{+w}{ }\PYG{o}{*}\PYG{+w}{ }\PYG{n}{number2}\PYG{l+s+si}{\PYGZcb{}}\PYG{l+s+s1}{\PYGZsq{}}\PYG{p}{)}
\PYG{k}{else}\PYG{p}{:}
  \PYG{k}{if} \PYG{n}{number2} \PYG{o}{==} \PYG{l+m+mi}{0}\PYG{p}{:} 
      \PYG{n+nb}{print}\PYG{p}{(}\PYG{l+s+s1}{\PYGZsq{}}\PYG{l+s+s1}{Warning: Division by zero.}\PYG{l+s+s1}{\PYGZsq{}}\PYG{p}{)}
  \PYG{k}{else}\PYG{p}{:} 
      \PYG{n+nb}{print}\PYG{p}{(}\PYG{l+s+sa}{f}\PYG{l+s+s1}{\PYGZsq{}}\PYG{l+s+si}{\PYGZob{}}\PYG{n}{number1}\PYG{l+s+si}{\PYGZcb{}}\PYG{l+s+s1}{ }\PYG{l+s+si}{\PYGZob{}}\PYG{n}{operation}\PYG{l+s+si}{\PYGZcb{}}\PYG{l+s+s1}{ }\PYG{l+s+si}{\PYGZob{}}\PYG{n}{number2}\PYG{l+s+si}{\PYGZcb{}}\PYG{l+s+s1}{ = }\PYG{l+s+si}{\PYGZob{}}\PYG{n}{number1}\PYG{+w}{ }\PYG{o}{/}\PYG{+w}{ }\PYG{n}{number2}\PYG{l+s+si}{:}\PYG{l+s+s1}{.2f}\PYG{l+s+si}{\PYGZcb{}}\PYG{l+s+s1}{\PYGZsq{}}\PYG{p}{)}
\end{sphinxVerbatim}
\end{sphinxadmonition}


\subsection{Question}
\label{\detokenize{conditionals_code:id9}}
\sphinxAtStartPar
Ask the user for a number with at most 8 digits.
\begin{itemize}
\item {} 
\sphinxAtStartPar
If the given number has fewer than 8 digits, pad zeros to the left to make it an 8\sphinxhyphen{}digit number.

\item {} 
\sphinxAtStartPar
Example:
\begin{itemize}
\item {} 
\sphinxAtStartPar
Given number: 123 —> Output: 00000123

\item {} 
\sphinxAtStartPar
Given number: 123456789 —> Output: Please enter a number with at most 8 digits.

\item {} 
\sphinxAtStartPar
Given number: 12345678 —> Output: 12345678

\end{itemize}

\end{itemize}

\sphinxAtStartPar
\sphinxstylestrong{Solution}

\begin{sphinxadmonition}{note}{Solution}

\begin{sphinxVerbatim}[commandchars=\\\{\}]
\PYG{n}{number} \PYG{o}{=} \PYG{n+nb}{input}\PYG{p}{(}\PYG{l+s+s1}{\PYGZsq{}}\PYG{l+s+s1}{Please enter a number with at most 8 digits: }\PYG{l+s+s1}{\PYGZsq{}}\PYG{p}{)}

\PYG{n}{length} \PYG{o}{=} \PYG{n+nb}{len}\PYG{p}{(}\PYG{n}{number}\PYG{p}{)}

\PYG{k}{if} \PYG{n}{length} \PYG{o}{\PYGZlt{}}\PYG{o}{=} \PYG{l+m+mi}{8}\PYG{p}{:}
  \PYG{n+nb}{print}\PYG{p}{(}\PYG{p}{(}\PYG{l+m+mi}{8}\PYG{o}{\PYGZhy{}}\PYG{n}{length}\PYG{p}{)}\PYG{o}{*}\PYG{l+s+s1}{\PYGZsq{}}\PYG{l+s+s1}{0}\PYG{l+s+s1}{\PYGZsq{}}\PYG{o}{+}\PYG{n}{number}\PYG{p}{)}
\PYG{k}{else}\PYG{p}{:}
  \PYG{n+nb}{print}\PYG{p}{(}\PYG{l+s+s1}{\PYGZsq{}}\PYG{l+s+s1}{Please enter a number with at most 8 digits.}\PYG{l+s+s1}{\PYGZsq{}}\PYG{p}{)}
\end{sphinxVerbatim}
\end{sphinxadmonition}

\sphinxstepscope


\chapter{Chp\sphinxhyphen{}6: Iterations}
\label{\detokenize{iterations:chp-6-iterations}}\label{\detokenize{iterations::doc}}\begin{itemize}
\item {} 
\sphinxAtStartPar
Learning Objectives
\begin{itemize}
\item {} 
\sphinxAtStartPar
..

\item {} 
\sphinxAtStartPar
..

\end{itemize}

\end{itemize}


\section{Motivation}
\label{\detokenize{iterations:motivation}}

\subsection{Triangle}
\label{\detokenize{iterations:triangle}}
\sphinxAtStartPar
As a motivational exercise, let’s recall the code we previously encountered for printing a triangle using the \sphinxcode{\sphinxupquote{\&}} character and the \sphinxstyleemphasis{print()} function.

\begin{sphinxuseclass}{cell}\begin{sphinxVerbatimInput}

\begin{sphinxuseclass}{cell_input}
\begin{sphinxVerbatim}[commandchars=\\\{\}]
\PYG{n+nb}{print}\PYG{p}{(}\PYG{l+s+s1}{\PYGZsq{}}\PYG{l+s+s1}{\PYGZam{}}\PYG{l+s+s1}{\PYGZsq{}}\PYG{p}{)}
\PYG{n+nb}{print}\PYG{p}{(}\PYG{l+s+s1}{\PYGZsq{}}\PYG{l+s+s1}{\PYGZam{}\PYGZam{}}\PYG{l+s+s1}{\PYGZsq{}}\PYG{p}{)}
\PYG{n+nb}{print}\PYG{p}{(}\PYG{l+s+s1}{\PYGZsq{}}\PYG{l+s+s1}{\PYGZam{}\PYGZam{}\PYGZam{}}\PYG{l+s+s1}{\PYGZsq{}}\PYG{p}{)}
\PYG{n+nb}{print}\PYG{p}{(}\PYG{l+s+s1}{\PYGZsq{}}\PYG{l+s+s1}{\PYGZam{}\PYGZam{}\PYGZam{}\PYGZam{}}\PYG{l+s+s1}{\PYGZsq{}}\PYG{p}{)}
\PYG{n+nb}{print}\PYG{p}{(}\PYG{l+s+s1}{\PYGZsq{}}\PYG{l+s+s1}{\PYGZam{}\PYGZam{}\PYGZam{}\PYGZam{}\PYGZam{}}\PYG{l+s+s1}{\PYGZsq{}}\PYG{p}{)}
\PYG{n+nb}{print}\PYG{p}{(}\PYG{l+s+s1}{\PYGZsq{}}\PYG{l+s+s1}{\PYGZam{}\PYGZam{}\PYGZam{}\PYGZam{}\PYGZam{}\PYGZam{}}\PYG{l+s+s1}{\PYGZsq{}}\PYG{p}{)}
\PYG{n+nb}{print}\PYG{p}{(}\PYG{l+s+s1}{\PYGZsq{}}\PYG{l+s+s1}{\PYGZam{}\PYGZam{}\PYGZam{}\PYGZam{}\PYGZam{}\PYGZam{}\PYGZam{}}\PYG{l+s+s1}{\PYGZsq{}}\PYG{p}{)}
\PYG{n+nb}{print}\PYG{p}{(}\PYG{l+s+s1}{\PYGZsq{}}\PYG{l+s+s1}{\PYGZam{}\PYGZam{}\PYGZam{}\PYGZam{}\PYGZam{}\PYGZam{}\PYGZam{}\PYGZam{}}\PYG{l+s+s1}{\PYGZsq{}}\PYG{p}{)}
\PYG{n+nb}{print}\PYG{p}{(}\PYG{l+s+s1}{\PYGZsq{}}\PYG{l+s+s1}{\PYGZam{}\PYGZam{}\PYGZam{}\PYGZam{}\PYGZam{}\PYGZam{}\PYGZam{}\PYGZam{}\PYGZam{}}\PYG{l+s+s1}{\PYGZsq{}}\PYG{p}{)}
\PYG{n+nb}{print}\PYG{p}{(}\PYG{l+s+s1}{\PYGZsq{}}\PYG{l+s+s1}{\PYGZam{}\PYGZam{}\PYGZam{}\PYGZam{}\PYGZam{}\PYGZam{}\PYGZam{}\PYGZam{}\PYGZam{}\PYGZam{}}\PYG{l+s+s1}{\PYGZsq{}}\PYG{p}{)}
\end{sphinxVerbatim}

\end{sphinxuseclass}\end{sphinxVerbatimInput}
\begin{sphinxVerbatimOutput}

\begin{sphinxuseclass}{cell_output}
\begin{sphinxVerbatim}[commandchars=\\\{\}]
\PYGZam{}
\PYGZam{}\PYGZam{}
\PYGZam{}\PYGZam{}\PYGZam{}
\PYGZam{}\PYGZam{}\PYGZam{}\PYGZam{}
\PYGZam{}\PYGZam{}\PYGZam{}\PYGZam{}\PYGZam{}
\PYGZam{}\PYGZam{}\PYGZam{}\PYGZam{}\PYGZam{}\PYGZam{}
\PYGZam{}\PYGZam{}\PYGZam{}\PYGZam{}\PYGZam{}\PYGZam{}\PYGZam{}
\PYGZam{}\PYGZam{}\PYGZam{}\PYGZam{}\PYGZam{}\PYGZam{}\PYGZam{}\PYGZam{}
\PYGZam{}\PYGZam{}\PYGZam{}\PYGZam{}\PYGZam{}\PYGZam{}\PYGZam{}\PYGZam{}\PYGZam{}
\PYGZam{}\PYGZam{}\PYGZam{}\PYGZam{}\PYGZam{}\PYGZam{}\PYGZam{}\PYGZam{}\PYGZam{}\PYGZam{}
\end{sphinxVerbatim}

\end{sphinxuseclass}\end{sphinxVerbatimOutput}

\end{sphinxuseclass}
\sphinxAtStartPar
This code follows a pattern where the number of \sphinxcode{\sphinxupquote{\&}} characters increases by one in each line, simplifying the code through repetition.

\begin{sphinxuseclass}{cell}\begin{sphinxVerbatimInput}

\begin{sphinxuseclass}{cell_input}
\begin{sphinxVerbatim}[commandchars=\\\{\}]
\PYG{n+nb}{print}\PYG{p}{(}\PYG{l+s+s1}{\PYGZsq{}}\PYG{l+s+s1}{\PYGZam{}}\PYG{l+s+s1}{\PYGZsq{}}\PYG{o}{*}\PYG{l+m+mi}{1}\PYG{p}{)}
\PYG{n+nb}{print}\PYG{p}{(}\PYG{l+s+s1}{\PYGZsq{}}\PYG{l+s+s1}{\PYGZam{}}\PYG{l+s+s1}{\PYGZsq{}}\PYG{o}{*}\PYG{l+m+mi}{2}\PYG{p}{)}
\PYG{n+nb}{print}\PYG{p}{(}\PYG{l+s+s1}{\PYGZsq{}}\PYG{l+s+s1}{\PYGZam{}}\PYG{l+s+s1}{\PYGZsq{}}\PYG{o}{*}\PYG{l+m+mi}{3}\PYG{p}{)}
\PYG{n+nb}{print}\PYG{p}{(}\PYG{l+s+s1}{\PYGZsq{}}\PYG{l+s+s1}{\PYGZam{}}\PYG{l+s+s1}{\PYGZsq{}}\PYG{o}{*}\PYG{l+m+mi}{4}\PYG{p}{)}
\PYG{n+nb}{print}\PYG{p}{(}\PYG{l+s+s1}{\PYGZsq{}}\PYG{l+s+s1}{\PYGZam{}}\PYG{l+s+s1}{\PYGZsq{}}\PYG{o}{*}\PYG{l+m+mi}{5}\PYG{p}{)}
\PYG{n+nb}{print}\PYG{p}{(}\PYG{l+s+s1}{\PYGZsq{}}\PYG{l+s+s1}{\PYGZam{}}\PYG{l+s+s1}{\PYGZsq{}}\PYG{o}{*}\PYG{l+m+mi}{6}\PYG{p}{)}
\PYG{n+nb}{print}\PYG{p}{(}\PYG{l+s+s1}{\PYGZsq{}}\PYG{l+s+s1}{\PYGZam{}}\PYG{l+s+s1}{\PYGZsq{}}\PYG{o}{*}\PYG{l+m+mi}{7}\PYG{p}{)}
\PYG{n+nb}{print}\PYG{p}{(}\PYG{l+s+s1}{\PYGZsq{}}\PYG{l+s+s1}{\PYGZam{}}\PYG{l+s+s1}{\PYGZsq{}}\PYG{o}{*}\PYG{l+m+mi}{8}\PYG{p}{)}
\PYG{n+nb}{print}\PYG{p}{(}\PYG{l+s+s1}{\PYGZsq{}}\PYG{l+s+s1}{\PYGZam{}}\PYG{l+s+s1}{\PYGZsq{}}\PYG{o}{*}\PYG{l+m+mi}{9}\PYG{p}{)}
\PYG{n+nb}{print}\PYG{p}{(}\PYG{l+s+s1}{\PYGZsq{}}\PYG{l+s+s1}{\PYGZam{}}\PYG{l+s+s1}{\PYGZsq{}}\PYG{o}{*}\PYG{l+m+mi}{10}\PYG{p}{)}
\end{sphinxVerbatim}

\end{sphinxuseclass}\end{sphinxVerbatimInput}
\begin{sphinxVerbatimOutput}

\begin{sphinxuseclass}{cell_output}
\begin{sphinxVerbatim}[commandchars=\\\{\}]
\PYGZam{}
\PYGZam{}\PYGZam{}
\PYGZam{}\PYGZam{}\PYGZam{}
\PYGZam{}\PYGZam{}\PYGZam{}\PYGZam{}
\PYGZam{}\PYGZam{}\PYGZam{}\PYGZam{}\PYGZam{}
\PYGZam{}\PYGZam{}\PYGZam{}\PYGZam{}\PYGZam{}\PYGZam{}
\PYGZam{}\PYGZam{}\PYGZam{}\PYGZam{}\PYGZam{}\PYGZam{}\PYGZam{}
\PYGZam{}\PYGZam{}\PYGZam{}\PYGZam{}\PYGZam{}\PYGZam{}\PYGZam{}\PYGZam{}
\PYGZam{}\PYGZam{}\PYGZam{}\PYGZam{}\PYGZam{}\PYGZam{}\PYGZam{}\PYGZam{}\PYGZam{}
\PYGZam{}\PYGZam{}\PYGZam{}\PYGZam{}\PYGZam{}\PYGZam{}\PYGZam{}\PYGZam{}\PYGZam{}\PYGZam{}
\end{sphinxVerbatim}

\end{sphinxuseclass}\end{sphinxVerbatimOutput}

\end{sphinxuseclass}\begin{itemize}
\item {} 
\sphinxAtStartPar
The second version, utilizing string repetition, is simpler than the first.

\item {} 
\sphinxAtStartPar
However, further simplification is possible as there are still repetitions, such as the presence of the print() function in each line.

\item {} 
\sphinxAtStartPar
To address this, iterations can be employed to avoid using the \sphinxstyleemphasis{print()} function ten times.

\item {} 
\sphinxAtStartPar
The iteration version is as follows:
\begin{itemize}
\item {} 
\sphinxAtStartPar
It eliminates repetition.

\item {} 
\sphinxAtStartPar
It does not become longer even with a larger triangle.

\end{itemize}

\end{itemize}

\begin{sphinxuseclass}{cell}\begin{sphinxVerbatimInput}

\begin{sphinxuseclass}{cell_input}
\begin{sphinxVerbatim}[commandchars=\\\{\}]
\PYG{k}{for} \PYG{n}{i} \PYG{o+ow}{in} \PYG{n+nb}{range}\PYG{p}{(}\PYG{l+m+mi}{1}\PYG{p}{,}\PYG{l+m+mi}{11}\PYG{p}{)}\PYG{p}{:}    \PYG{c+c1}{\PYGZsh{} iteration}
    \PYG{n+nb}{print}\PYG{p}{(}\PYG{l+s+s1}{\PYGZsq{}}\PYG{l+s+s1}{\PYGZam{}}\PYG{l+s+s1}{\PYGZsq{}}\PYG{o}{*}\PYG{n}{i}\PYG{p}{)}
\end{sphinxVerbatim}

\end{sphinxuseclass}\end{sphinxVerbatimInput}
\begin{sphinxVerbatimOutput}

\begin{sphinxuseclass}{cell_output}
\begin{sphinxVerbatim}[commandchars=\\\{\}]
\PYGZam{}
\PYGZam{}\PYGZam{}
\PYGZam{}\PYGZam{}\PYGZam{}
\PYGZam{}\PYGZam{}\PYGZam{}\PYGZam{}
\PYGZam{}\PYGZam{}\PYGZam{}\PYGZam{}\PYGZam{}
\PYGZam{}\PYGZam{}\PYGZam{}\PYGZam{}\PYGZam{}\PYGZam{}
\PYGZam{}\PYGZam{}\PYGZam{}\PYGZam{}\PYGZam{}\PYGZam{}\PYGZam{}
\PYGZam{}\PYGZam{}\PYGZam{}\PYGZam{}\PYGZam{}\PYGZam{}\PYGZam{}\PYGZam{}
\PYGZam{}\PYGZam{}\PYGZam{}\PYGZam{}\PYGZam{}\PYGZam{}\PYGZam{}\PYGZam{}\PYGZam{}
\PYGZam{}\PYGZam{}\PYGZam{}\PYGZam{}\PYGZam{}\PYGZam{}\PYGZam{}\PYGZam{}\PYGZam{}\PYGZam{}
\end{sphinxVerbatim}

\end{sphinxuseclass}\end{sphinxVerbatimOutput}

\end{sphinxuseclass}

\subsection{Strings}
\label{\detokenize{iterations:strings}}\begin{itemize}
\item {} 
\sphinxAtStartPar
We have explored various methods for working with strings, yet a crucial aspect remains unaddressed — how to iterate through all characters of a string individually.

\item {} 
\sphinxAtStartPar
While indexes and slices allow access to specific characters or portions of a string, the challenge lies in accessing each character sequentially.

\end{itemize}
\begin{enumerate}
\sphinxsetlistlabels{\arabic}{enumi}{enumii}{}{.}%
\item {} 
\sphinxAtStartPar
\sphinxstylestrong{Question:} What is the occurrence of a certain character, such as ‘r’, in a string?
\begin{itemize}
\item {} 
\sphinxAtStartPar
How can we address this question without utilizing the \sphinxstyleemphasis{count()} method of strings?

\item {} 
\sphinxAtStartPar
The solution involves checking whether each character in the string matches ‘r’.

\end{itemize}

\end{enumerate}
\begin{itemize}
\item {} 
\sphinxAtStartPar
Let’s attempt to write code that answers this question using a short string and only the information we have gathered from the previous chapters.

\end{itemize}

\begin{sphinxuseclass}{cell}\begin{sphinxVerbatimInput}

\begin{sphinxuseclass}{cell_input}
\begin{sphinxVerbatim}[commandchars=\\\{\}]
\PYG{n}{text} \PYG{o}{=} \PYG{l+s+s1}{\PYGZsq{}}\PYG{l+s+s1}{radar}\PYG{l+s+s1}{\PYGZsq{}}
\PYG{n}{count\PYGZus{}r} \PYG{o}{=} \PYG{l+m+mi}{0}

\PYG{k}{if} \PYG{n}{text}\PYG{p}{[}\PYG{l+m+mi}{0}\PYG{p}{]} \PYG{o}{==} \PYG{l+s+s1}{\PYGZsq{}}\PYG{l+s+s1}{r}\PYG{l+s+s1}{\PYGZsq{}}\PYG{p}{:}
    \PYG{n}{count\PYGZus{}r} \PYG{o}{+}\PYG{o}{=} \PYG{l+m+mi}{1}
\PYG{k}{if} \PYG{n}{text}\PYG{p}{[}\PYG{l+m+mi}{1}\PYG{p}{]} \PYG{o}{==} \PYG{l+s+s1}{\PYGZsq{}}\PYG{l+s+s1}{r}\PYG{l+s+s1}{\PYGZsq{}}\PYG{p}{:}
    \PYG{n}{count\PYGZus{}r} \PYG{o}{+}\PYG{o}{=} \PYG{l+m+mi}{1}
\PYG{k}{if} \PYG{n}{text}\PYG{p}{[}\PYG{l+m+mi}{2}\PYG{p}{]} \PYG{o}{==} \PYG{l+s+s1}{\PYGZsq{}}\PYG{l+s+s1}{r}\PYG{l+s+s1}{\PYGZsq{}}\PYG{p}{:}
    \PYG{n}{count\PYGZus{}r} \PYG{o}{+}\PYG{o}{=} \PYG{l+m+mi}{1}
\PYG{k}{if} \PYG{n}{text}\PYG{p}{[}\PYG{l+m+mi}{3}\PYG{p}{]} \PYG{o}{==} \PYG{l+s+s1}{\PYGZsq{}}\PYG{l+s+s1}{r}\PYG{l+s+s1}{\PYGZsq{}}\PYG{p}{:}
    \PYG{n}{count\PYGZus{}r} \PYG{o}{+}\PYG{o}{=} \PYG{l+m+mi}{1}
\PYG{k}{if} \PYG{n}{text}\PYG{p}{[}\PYG{l+m+mi}{4}\PYG{p}{]} \PYG{o}{==} \PYG{l+s+s1}{\PYGZsq{}}\PYG{l+s+s1}{r}\PYG{l+s+s1}{\PYGZsq{}}\PYG{p}{:}
    \PYG{n}{count\PYGZus{}r} \PYG{o}{+}\PYG{o}{=} \PYG{l+m+mi}{1}

\PYG{n+nb}{print}\PYG{p}{(}\PYG{l+s+sa}{f}\PYG{l+s+s1}{\PYGZsq{}}\PYG{l+s+s1}{There are }\PYG{l+s+si}{\PYGZob{}}\PYG{n}{count\PYGZus{}r}\PYG{l+s+si}{\PYGZcb{}}\PYG{l+s+s1}{ }\PYG{l+s+s1}{\PYGZdq{}}\PYG{l+s+s1}{r}\PYG{l+s+s1}{\PYGZdq{}}\PYG{l+s+s1}{ characters in }\PYG{l+s+si}{\PYGZob{}}\PYG{n}{text}\PYG{l+s+si}{\PYGZcb{}}\PYG{l+s+s1}{.}\PYG{l+s+s1}{\PYGZsq{}}\PYG{p}{)}
\end{sphinxVerbatim}

\end{sphinxuseclass}\end{sphinxVerbatimInput}
\begin{sphinxVerbatimOutput}

\begin{sphinxuseclass}{cell_output}
\begin{sphinxVerbatim}[commandchars=\\\{\}]
There are 2 \PYGZdq{}r\PYGZdq{} characters in radar.
\end{sphinxVerbatim}

\end{sphinxuseclass}\end{sphinxVerbatimOutput}

\end{sphinxuseclass}\begin{itemize}
\item {} 
\sphinxAtStartPar
As evident, there are numerous repetitions in this code, making it overwhelming for long strings.

\item {} 
\sphinxAtStartPar
To mitigate this, iterations can be employed to avoid the need for using the \sphinxstyleemphasis{if} statements multiple times.

\item {} 
\sphinxAtStartPar
The iteration version is as follows, it eliminates repetition and does not become longer even with an extended string.

\end{itemize}

\begin{sphinxuseclass}{cell}\begin{sphinxVerbatimInput}

\begin{sphinxuseclass}{cell_input}
\begin{sphinxVerbatim}[commandchars=\\\{\}]
\PYG{n}{text} \PYG{o}{=} \PYG{l+s+s1}{\PYGZsq{}}\PYG{l+s+s1}{radar}\PYG{l+s+s1}{\PYGZsq{}}
\PYG{n}{count\PYGZus{}r} \PYG{o}{=} \PYG{l+m+mi}{0}

\PYG{k}{for} \PYG{n}{char} \PYG{o+ow}{in} \PYG{n}{text}\PYG{p}{:}            \PYG{c+c1}{\PYGZsh{} iteration}
    \PYG{k}{if} \PYG{n}{char} \PYG{o}{==} \PYG{l+s+s1}{\PYGZsq{}}\PYG{l+s+s1}{r}\PYG{l+s+s1}{\PYGZsq{}}\PYG{p}{:}
        \PYG{n}{count\PYGZus{}r} \PYG{o}{+}\PYG{o}{=} \PYG{l+m+mi}{1}
    
\PYG{n+nb}{print}\PYG{p}{(}\PYG{l+s+sa}{f}\PYG{l+s+s1}{\PYGZsq{}}\PYG{l+s+s1}{There are }\PYG{l+s+si}{\PYGZob{}}\PYG{n}{count\PYGZus{}r}\PYG{l+s+si}{\PYGZcb{}}\PYG{l+s+s1}{ }\PYG{l+s+s1}{\PYGZdq{}}\PYG{l+s+s1}{r}\PYG{l+s+s1}{\PYGZdq{}}\PYG{l+s+s1}{ characters in }\PYG{l+s+si}{\PYGZob{}}\PYG{n}{text}\PYG{l+s+si}{\PYGZcb{}}\PYG{l+s+s1}{.}\PYG{l+s+s1}{\PYGZsq{}}\PYG{p}{)}
\end{sphinxVerbatim}

\end{sphinxuseclass}\end{sphinxVerbatimInput}
\begin{sphinxVerbatimOutput}

\begin{sphinxuseclass}{cell_output}
\begin{sphinxVerbatim}[commandchars=\\\{\}]
There are 2 \PYGZdq{}r\PYGZdq{} characters in radar.
\end{sphinxVerbatim}

\end{sphinxuseclass}\end{sphinxVerbatimOutput}

\end{sphinxuseclass}\begin{enumerate}
\sphinxsetlistlabels{\arabic}{enumi}{enumii}{}{.}%
\setcounter{enumi}{1}
\item {} 
\sphinxAtStartPar
\sphinxstylestrong{Question:} Now let’s work on a more complicated question. What are the digits in a string which are greater than 6?

\end{enumerate}
\begin{itemize}
\item {} 
\sphinxAtStartPar
Let’s attempt to write code that answers this question using a  string which consists of digits and only the information we have gathered from the previous chapters.

\end{itemize}

\begin{sphinxuseclass}{cell}\begin{sphinxVerbatimInput}

\begin{sphinxuseclass}{cell_input}
\begin{sphinxVerbatim}[commandchars=\\\{\}]
\PYG{n}{text} \PYG{o}{=} \PYG{l+s+s1}{\PYGZsq{}}\PYG{l+s+s1}{192736}\PYG{l+s+s1}{\PYGZsq{}}
\PYG{n+nb}{print}\PYG{p}{(}\PYG{l+s+sa}{f}\PYG{l+s+s1}{\PYGZsq{}}\PYG{l+s+s1}{The digits in }\PYG{l+s+si}{\PYGZob{}}\PYG{n}{text}\PYG{l+s+si}{\PYGZcb{}}\PYG{l+s+s1}{ which are greater than 6:}\PYG{l+s+s1}{\PYGZsq{}}\PYG{p}{)}

\PYG{k}{if} \PYG{n+nb}{int}\PYG{p}{(}\PYG{n}{text}\PYG{p}{[}\PYG{l+m+mi}{0}\PYG{p}{]}\PYG{p}{)} \PYG{o}{\PYGZgt{}} \PYG{l+m+mi}{6}\PYG{p}{:}
    \PYG{n+nb}{print}\PYG{p}{(}\PYG{n}{text}\PYG{p}{[}\PYG{l+m+mi}{0}\PYG{p}{]}\PYG{p}{)}
\PYG{k}{if} \PYG{n+nb}{int}\PYG{p}{(}\PYG{n}{text}\PYG{p}{[}\PYG{l+m+mi}{1}\PYG{p}{]}\PYG{p}{)} \PYG{o}{\PYGZgt{}} \PYG{l+m+mi}{6}\PYG{p}{:}
    \PYG{n+nb}{print}\PYG{p}{(}\PYG{n}{text}\PYG{p}{[}\PYG{l+m+mi}{1}\PYG{p}{]}\PYG{p}{)}
\PYG{k}{if} \PYG{n+nb}{int}\PYG{p}{(}\PYG{n}{text}\PYG{p}{[}\PYG{l+m+mi}{2}\PYG{p}{]}\PYG{p}{)} \PYG{o}{\PYGZgt{}} \PYG{l+m+mi}{6}\PYG{p}{:}
    \PYG{n+nb}{print}\PYG{p}{(}\PYG{n}{text}\PYG{p}{[}\PYG{l+m+mi}{2}\PYG{p}{]}\PYG{p}{)}
\PYG{k}{if} \PYG{n+nb}{int}\PYG{p}{(}\PYG{n}{text}\PYG{p}{[}\PYG{l+m+mi}{3}\PYG{p}{]}\PYG{p}{)} \PYG{o}{\PYGZgt{}} \PYG{l+m+mi}{6}\PYG{p}{:}
    \PYG{n+nb}{print}\PYG{p}{(}\PYG{n}{text}\PYG{p}{[}\PYG{l+m+mi}{3}\PYG{p}{]}\PYG{p}{)}
\PYG{k}{if} \PYG{n+nb}{int}\PYG{p}{(}\PYG{n}{text}\PYG{p}{[}\PYG{l+m+mi}{4}\PYG{p}{]}\PYG{p}{)} \PYG{o}{\PYGZgt{}} \PYG{l+m+mi}{6}\PYG{p}{:}
    \PYG{n+nb}{print}\PYG{p}{(}\PYG{n}{text}\PYG{p}{[}\PYG{l+m+mi}{4}\PYG{p}{]}\PYG{p}{)}
\PYG{k}{if} \PYG{n+nb}{int}\PYG{p}{(}\PYG{n}{text}\PYG{p}{[}\PYG{l+m+mi}{5}\PYG{p}{]}\PYG{p}{)} \PYG{o}{\PYGZgt{}} \PYG{l+m+mi}{6}\PYG{p}{:}
    \PYG{n+nb}{print}\PYG{p}{(}\PYG{n}{text}\PYG{p}{[}\PYG{l+m+mi}{5}\PYG{p}{]}\PYG{p}{)}
\end{sphinxVerbatim}

\end{sphinxuseclass}\end{sphinxVerbatimInput}
\begin{sphinxVerbatimOutput}

\begin{sphinxuseclass}{cell_output}
\begin{sphinxVerbatim}[commandchars=\\\{\}]
The digits in 192736 which are greater than 6:
9
7
\end{sphinxVerbatim}

\end{sphinxuseclass}\end{sphinxVerbatimOutput}

\end{sphinxuseclass}\begin{itemize}
\item {} 
\sphinxAtStartPar
The code above exhibits repetition, which can be overwhelming for long strings.

\item {} 
\sphinxAtStartPar
The iteration version, provided below, eliminates this repetition and does not become longer even with an extended string.

\end{itemize}

\begin{sphinxuseclass}{cell}\begin{sphinxVerbatimInput}

\begin{sphinxuseclass}{cell_input}
\begin{sphinxVerbatim}[commandchars=\\\{\}]
\PYG{n}{text} \PYG{o}{=} \PYG{l+s+s1}{\PYGZsq{}}\PYG{l+s+s1}{192736}\PYG{l+s+s1}{\PYGZsq{}}
\PYG{n+nb}{print}\PYG{p}{(}\PYG{l+s+sa}{f}\PYG{l+s+s1}{\PYGZsq{}}\PYG{l+s+s1}{The digits greater than 6 in }\PYG{l+s+si}{\PYGZob{}}\PYG{n}{text}\PYG{l+s+si}{\PYGZcb{}}\PYG{l+s+s1}{:}\PYG{l+s+s1}{\PYGZsq{}}\PYG{p}{)}

\PYG{k}{for} \PYG{n}{char} \PYG{o+ow}{in} \PYG{n}{text}\PYG{p}{:}
    \PYG{k}{if} \PYG{n+nb}{int}\PYG{p}{(}\PYG{n}{char}\PYG{p}{)} \PYG{o}{\PYGZgt{}} \PYG{l+m+mi}{6}\PYG{p}{:}
        \PYG{n+nb}{print}\PYG{p}{(}\PYG{n}{char}\PYG{p}{)}
\end{sphinxVerbatim}

\end{sphinxuseclass}\end{sphinxVerbatimInput}
\begin{sphinxVerbatimOutput}

\begin{sphinxuseclass}{cell_output}
\begin{sphinxVerbatim}[commandchars=\\\{\}]
The digits greater than 6 in 192736:
9
7
\end{sphinxVerbatim}

\end{sphinxuseclass}\end{sphinxVerbatimOutput}

\end{sphinxuseclass}

\section{Iterations}
\label{\detokenize{iterations:iterations}}
\sphinxAtStartPar
Iterations are used to perform the same or similar tasks in a more efficient way.
\begin{itemize}
\item {} 
\sphinxAtStartPar
Similar tasks usually follow a pattern that can be used to write the code in a shorter and more readable way.

\end{itemize}

\sphinxAtStartPar
There are two types of iterations available in programming languages:
\begin{itemize}
\item {} 
\sphinxAtStartPar
\sphinxcode{\sphinxupquote{while}} loop: This is used for indefinite repetition, executing a block code for a possibly unknown number of times.

\item {} 
\sphinxAtStartPar
\sphinxcode{\sphinxupquote{for}} loop: This is used for definite repetition, executing a code for a known number of times.

\end{itemize}
\begin{enumerate}
\sphinxsetlistlabels{\arabic}{enumi}{enumii}{}{.}%
\item {} 
\sphinxAtStartPar
The \sphinxcode{\sphinxupquote{for}} loop executes its block code repeatedly for every element of a sequence.
\begin{itemize}
\item {} 
\sphinxAtStartPar
It has a condition (boolean expression) with the \sphinxstylestrong{in} operator, making it possible to execute the block code only for elements of the sequence.

\end{itemize}

\item {} 
\sphinxAtStartPar
The \sphinxcode{\sphinxupquote{while}} loop executes a block code as long as its condition is True.
\begin{itemize}
\item {} 
\sphinxAtStartPar
It is similar to an \sphinxstyleemphasis{if} statement becase its condition can be any boolean expression (not only with \sphinxstyleemphasis{in}).

\item {} 
\sphinxAtStartPar
The difference lies in the fact that the block code of an \sphinxstyleemphasis{if} statement is executed only once if the condition is True.

\item {} 
\sphinxAtStartPar
In contrast, if the condition of the \sphinxcode{\sphinxupquote{while}} loop is True, its block code is executed as in \sphinxstyleemphasis{if} statements, but then the condition is checked again.

\item {} 
\sphinxAtStartPar
If it is still True, the block code will be executed again. The \sphinxcode{\sphinxupquote{while}} loop keeps executing its block code as long as its condition becomes False or a break command is used to terminate it.

\end{itemize}

\end{enumerate}


\section{Range Function}
\label{\detokenize{iterations:range-function}}
\sphinxAtStartPar
The built\sphinxhyphen{}in \sphinxcode{\sphinxupquote{range()}} function returns a sequence of integers.
\begin{itemize}
\item {} 
\sphinxAtStartPar
The type of its output is \sphinxstyleemphasis{range}, and its values are hidden within it.

\item {} 
\sphinxAtStartPar
You can use the built\sphinxhyphen{}in \sphinxstyleemphasis{list()} function to explicitly display the numbers in a range type output.

\item {} 
\sphinxAtStartPar
The \sphinxcode{\sphinxupquote{range()}} function has three important parameters: start, end, and step.
\begin{itemize}
\item {} 
\sphinxAtStartPar
How they work is similar to the start, end, and step used for slicing of strings by using indexes.

\end{itemize}

\item {} 
\sphinxAtStartPar
There are three cases:
|\#|Function|Numbers|Explanation|
|\sphinxhyphen{}|\sphinxhyphen{}|\sphinxhyphen{}|\sphinxhyphen{}|\\
|1|range(a)     | 0, 1,2,…,a\sphinxhyphen{}1  |  integers starting from 0 goes upto a\sphinxhyphen{}1|
|2|range(a,b)   | a, a+1, …, b\sphinxhyphen{}1 |   integers starting from a goes upto b\sphinxhyphen{}1|
|3|range(a,b,s) | a, a+s, a+2s, …,   less than b\sphinxhyphen{}1|integers start from a go upto b\sphinxhyphen{}1 with an increment of s|

\item {} 
\sphinxAtStartPar
The step \(s\) can be a negative number. If \(s\) is a negative number:
\begin{itemize}
\item {} 
\sphinxAtStartPar
If \(a < b\), the output is empty (as you cannot reach \(a\) from \(b\) by adding negative numbers).

\item {} 
\sphinxAtStartPar
If \(a > b\), the output is \(a, a-1, \ldots, b+1\) (as you can reach \(a\) from \(b\) by adding a negative numbers).

\end{itemize}

\item {} 
\sphinxAtStartPar
Example:
\begin{itemize}
\item {} 
\sphinxAtStartPar
range(10) consists of \(0, 1, 2, 3, 4, 5, 6, 7, 8, 9\).

\item {} 
\sphinxAtStartPar
range(2,10) consists of \(2, 3, 4, 5, 6, 7, 8, 9\).

\item {} 
\sphinxAtStartPar
range(2,10,3) consists of \(2, 5, 8\).

\item {} 
\sphinxAtStartPar
range(2,10,\sphinxhyphen{}3) is empty.

\item {} 
\sphinxAtStartPar
range(10,2,\sphinxhyphen{}3) consists of \(10, 9, 8, 7, 6, 5, 4, 3\).

\end{itemize}

\end{itemize}

\begin{sphinxuseclass}{cell}\begin{sphinxVerbatimInput}

\begin{sphinxuseclass}{cell_input}
\begin{sphinxVerbatim}[commandchars=\\\{\}]
\PYG{c+c1}{\PYGZsh{} Case 1: range(a)}
\PYG{n}{rng\PYGZus{}numbers} \PYG{o}{=} \PYG{n+nb}{range}\PYG{p}{(}\PYG{l+m+mi}{10}\PYG{p}{)}

\PYG{n+nb}{print}\PYG{p}{(}\PYG{l+s+sa}{f}\PYG{l+s+s1}{\PYGZsq{}}\PYG{l+s+s1}{Output: }\PYG{l+s+si}{\PYGZob{}}\PYG{n}{rng\PYGZus{}numbers}\PYG{l+s+si}{\PYGZcb{}}\PYG{l+s+s1}{\PYGZsq{}} \PYG{p}{)}
\PYG{n+nb}{print}\PYG{p}{(}\PYG{l+s+sa}{f}\PYG{l+s+s1}{\PYGZsq{}}\PYG{l+s+s1}{Type  : }\PYG{l+s+si}{\PYGZob{}}\PYG{n+nb}{type}\PYG{p}{(}\PYG{n}{rng\PYGZus{}numbers}\PYG{p}{)}\PYG{l+s+si}{\PYGZcb{}}\PYG{l+s+s1}{\PYGZsq{}}\PYG{p}{)}
\PYG{n+nb}{print}\PYG{p}{(}\PYG{l+s+sa}{f}\PYG{l+s+s1}{\PYGZsq{}}\PYG{l+s+s1}{List  : }\PYG{l+s+si}{\PYGZob{}}\PYG{n+nb}{list}\PYG{p}{(}\PYG{n}{rng\PYGZus{}numbers}\PYG{p}{)}\PYG{l+s+si}{\PYGZcb{}}\PYG{l+s+s1}{\PYGZsq{}}\PYG{p}{)}
\end{sphinxVerbatim}

\end{sphinxuseclass}\end{sphinxVerbatimInput}
\begin{sphinxVerbatimOutput}

\begin{sphinxuseclass}{cell_output}
\begin{sphinxVerbatim}[commandchars=\\\{\}]
Output: range(0, 10)
Type  : \PYGZlt{}class \PYGZsq{}range\PYGZsq{}\PYGZgt{}
List  : [0, 1, 2, 3, 4, 5, 6, 7, 8, 9]
\end{sphinxVerbatim}

\end{sphinxuseclass}\end{sphinxVerbatimOutput}

\end{sphinxuseclass}
\begin{sphinxuseclass}{cell}\begin{sphinxVerbatimInput}

\begin{sphinxuseclass}{cell_input}
\begin{sphinxVerbatim}[commandchars=\\\{\}]
\PYG{c+c1}{\PYGZsh{} Case 2: range(a,b)}
\PYG{n}{rng\PYGZus{}numbers} \PYG{o}{=} \PYG{n+nb}{range}\PYG{p}{(}\PYG{l+m+mi}{2}\PYG{p}{,}\PYG{l+m+mi}{10}\PYG{p}{)}

\PYG{n+nb}{print}\PYG{p}{(}\PYG{l+s+sa}{f}\PYG{l+s+s1}{\PYGZsq{}}\PYG{l+s+s1}{Output: }\PYG{l+s+si}{\PYGZob{}}\PYG{n}{rng\PYGZus{}numbers}\PYG{l+s+si}{\PYGZcb{}}\PYG{l+s+s1}{\PYGZsq{}} \PYG{p}{)}
\PYG{n+nb}{print}\PYG{p}{(}\PYG{l+s+sa}{f}\PYG{l+s+s1}{\PYGZsq{}}\PYG{l+s+s1}{Type  : }\PYG{l+s+si}{\PYGZob{}}\PYG{n+nb}{type}\PYG{p}{(}\PYG{n}{rng\PYGZus{}numbers}\PYG{p}{)}\PYG{l+s+si}{\PYGZcb{}}\PYG{l+s+s1}{\PYGZsq{}}\PYG{p}{)}
\PYG{n+nb}{print}\PYG{p}{(}\PYG{l+s+sa}{f}\PYG{l+s+s1}{\PYGZsq{}}\PYG{l+s+s1}{List  : }\PYG{l+s+si}{\PYGZob{}}\PYG{n+nb}{list}\PYG{p}{(}\PYG{n}{rng\PYGZus{}numbers}\PYG{p}{)}\PYG{l+s+si}{\PYGZcb{}}\PYG{l+s+s1}{\PYGZsq{}}\PYG{p}{)}
\end{sphinxVerbatim}

\end{sphinxuseclass}\end{sphinxVerbatimInput}
\begin{sphinxVerbatimOutput}

\begin{sphinxuseclass}{cell_output}
\begin{sphinxVerbatim}[commandchars=\\\{\}]
Output: range(2, 10)
Type  : \PYGZlt{}class \PYGZsq{}range\PYGZsq{}\PYGZgt{}
List  : [2, 3, 4, 5, 6, 7, 8, 9]
\end{sphinxVerbatim}

\end{sphinxuseclass}\end{sphinxVerbatimOutput}

\end{sphinxuseclass}
\begin{sphinxuseclass}{cell}\begin{sphinxVerbatimInput}

\begin{sphinxuseclass}{cell_input}
\begin{sphinxVerbatim}[commandchars=\\\{\}]
\PYG{c+c1}{\PYGZsh{} Case 3: range(a,b,s)}
\PYG{n}{rng\PYGZus{}numbers} \PYG{o}{=} \PYG{n+nb}{range}\PYG{p}{(}\PYG{l+m+mi}{2}\PYG{p}{,}\PYG{l+m+mi}{10}\PYG{p}{,}\PYG{l+m+mi}{3}\PYG{p}{)}

\PYG{n+nb}{print}\PYG{p}{(}\PYG{l+s+sa}{f}\PYG{l+s+s1}{\PYGZsq{}}\PYG{l+s+s1}{Output: }\PYG{l+s+si}{\PYGZob{}}\PYG{n}{rng\PYGZus{}numbers}\PYG{l+s+si}{\PYGZcb{}}\PYG{l+s+s1}{\PYGZsq{}} \PYG{p}{)}
\PYG{n+nb}{print}\PYG{p}{(}\PYG{l+s+sa}{f}\PYG{l+s+s1}{\PYGZsq{}}\PYG{l+s+s1}{Type  : }\PYG{l+s+si}{\PYGZob{}}\PYG{n+nb}{type}\PYG{p}{(}\PYG{n}{rng\PYGZus{}numbers}\PYG{p}{)}\PYG{l+s+si}{\PYGZcb{}}\PYG{l+s+s1}{\PYGZsq{}}\PYG{p}{)}
\PYG{n+nb}{print}\PYG{p}{(}\PYG{l+s+sa}{f}\PYG{l+s+s1}{\PYGZsq{}}\PYG{l+s+s1}{List  : }\PYG{l+s+si}{\PYGZob{}}\PYG{n+nb}{list}\PYG{p}{(}\PYG{n}{rng\PYGZus{}numbers}\PYG{p}{)}\PYG{l+s+si}{\PYGZcb{}}\PYG{l+s+s1}{\PYGZsq{}}\PYG{p}{)}
\end{sphinxVerbatim}

\end{sphinxuseclass}\end{sphinxVerbatimInput}
\begin{sphinxVerbatimOutput}

\begin{sphinxuseclass}{cell_output}
\begin{sphinxVerbatim}[commandchars=\\\{\}]
Output: range(2, 10, 3)
Type  : \PYGZlt{}class \PYGZsq{}range\PYGZsq{}\PYGZgt{}
List  : [2, 5, 8]
\end{sphinxVerbatim}

\end{sphinxuseclass}\end{sphinxVerbatimOutput}

\end{sphinxuseclass}
\begin{sphinxuseclass}{cell}\begin{sphinxVerbatimInput}

\begin{sphinxuseclass}{cell_input}
\begin{sphinxVerbatim}[commandchars=\\\{\}]
\PYG{c+c1}{\PYGZsh{} Case 3: a\PYGZlt{}b and negative s}
\PYG{n}{rng\PYGZus{}numbers} \PYG{o}{=} \PYG{n+nb}{range}\PYG{p}{(}\PYG{l+m+mi}{2}\PYG{p}{,}\PYG{l+m+mi}{10}\PYG{p}{,}\PYG{o}{\PYGZhy{}}\PYG{l+m+mi}{3}\PYG{p}{)}

\PYG{n+nb}{print}\PYG{p}{(}\PYG{l+s+sa}{f}\PYG{l+s+s1}{\PYGZsq{}}\PYG{l+s+s1}{Output: }\PYG{l+s+si}{\PYGZob{}}\PYG{n}{rng\PYGZus{}numbers}\PYG{l+s+si}{\PYGZcb{}}\PYG{l+s+s1}{\PYGZsq{}} \PYG{p}{)}
\PYG{n+nb}{print}\PYG{p}{(}\PYG{l+s+sa}{f}\PYG{l+s+s1}{\PYGZsq{}}\PYG{l+s+s1}{Type  : }\PYG{l+s+si}{\PYGZob{}}\PYG{n+nb}{type}\PYG{p}{(}\PYG{n}{rng\PYGZus{}numbers}\PYG{p}{)}\PYG{l+s+si}{\PYGZcb{}}\PYG{l+s+s1}{\PYGZsq{}}\PYG{p}{)}
\PYG{n+nb}{print}\PYG{p}{(}\PYG{l+s+sa}{f}\PYG{l+s+s1}{\PYGZsq{}}\PYG{l+s+s1}{List  : }\PYG{l+s+si}{\PYGZob{}}\PYG{n+nb}{list}\PYG{p}{(}\PYG{n}{rng\PYGZus{}numbers}\PYG{p}{)}\PYG{l+s+si}{\PYGZcb{}}\PYG{l+s+s1}{\PYGZsq{}}\PYG{p}{)}

\PYG{n+nb}{list}\PYG{p}{(}\PYG{n+nb}{range}\PYG{p}{(}\PYG{l+m+mi}{10}\PYG{p}{,}\PYG{l+m+mi}{2}\PYG{p}{,}\PYG{o}{\PYGZhy{}}\PYG{l+m+mi}{1}\PYG{p}{)}\PYG{p}{)}
\end{sphinxVerbatim}

\end{sphinxuseclass}\end{sphinxVerbatimInput}
\begin{sphinxVerbatimOutput}

\begin{sphinxuseclass}{cell_output}
\begin{sphinxVerbatim}[commandchars=\\\{\}]
Output: range(2, 10, \PYGZhy{}3)
Type  : \PYGZlt{}class \PYGZsq{}range\PYGZsq{}\PYGZgt{}
List  : []
\end{sphinxVerbatim}

\begin{sphinxVerbatim}[commandchars=\\\{\}]
[10, 9, 8, 7, 6, 5, 4, 3]
\end{sphinxVerbatim}

\end{sphinxuseclass}\end{sphinxVerbatimOutput}

\end{sphinxuseclass}
\begin{sphinxuseclass}{cell}\begin{sphinxVerbatimInput}

\begin{sphinxuseclass}{cell_input}
\begin{sphinxVerbatim}[commandchars=\\\{\}]
\PYG{c+c1}{\PYGZsh{} Case 3: a\PYGZgt{}b and negative s}
\PYG{n}{rng\PYGZus{}numbers} \PYG{o}{=} \PYG{n+nb}{range}\PYG{p}{(}\PYG{l+m+mi}{10}\PYG{p}{,}\PYG{l+m+mi}{2}\PYG{p}{,}\PYG{o}{\PYGZhy{}}\PYG{l+m+mi}{3}\PYG{p}{)}

\PYG{n+nb}{print}\PYG{p}{(}\PYG{l+s+sa}{f}\PYG{l+s+s1}{\PYGZsq{}}\PYG{l+s+s1}{Output: }\PYG{l+s+si}{\PYGZob{}}\PYG{n}{rng\PYGZus{}numbers}\PYG{l+s+si}{\PYGZcb{}}\PYG{l+s+s1}{\PYGZsq{}} \PYG{p}{)}
\PYG{n+nb}{print}\PYG{p}{(}\PYG{l+s+sa}{f}\PYG{l+s+s1}{\PYGZsq{}}\PYG{l+s+s1}{Type  : }\PYG{l+s+si}{\PYGZob{}}\PYG{n+nb}{type}\PYG{p}{(}\PYG{n}{rng\PYGZus{}numbers}\PYG{p}{)}\PYG{l+s+si}{\PYGZcb{}}\PYG{l+s+s1}{\PYGZsq{}}\PYG{p}{)}
\PYG{n+nb}{print}\PYG{p}{(}\PYG{l+s+sa}{f}\PYG{l+s+s1}{\PYGZsq{}}\PYG{l+s+s1}{List  : }\PYG{l+s+si}{\PYGZob{}}\PYG{n+nb}{list}\PYG{p}{(}\PYG{n}{rng\PYGZus{}numbers}\PYG{p}{)}\PYG{l+s+si}{\PYGZcb{}}\PYG{l+s+s1}{\PYGZsq{}}\PYG{p}{)}

\PYG{n+nb}{list}\PYG{p}{(}\PYG{n+nb}{range}\PYG{p}{(}\PYG{l+m+mi}{10}\PYG{p}{,}\PYG{l+m+mi}{2}\PYG{p}{,}\PYG{o}{\PYGZhy{}}\PYG{l+m+mi}{1}\PYG{p}{)}\PYG{p}{)}
\end{sphinxVerbatim}

\end{sphinxuseclass}\end{sphinxVerbatimInput}
\begin{sphinxVerbatimOutput}

\begin{sphinxuseclass}{cell_output}
\begin{sphinxVerbatim}[commandchars=\\\{\}]
Output: range(10, 2, \PYGZhy{}3)
Type  : \PYGZlt{}class \PYGZsq{}range\PYGZsq{}\PYGZgt{}
List  : [10, 7, 4]
\end{sphinxVerbatim}

\begin{sphinxVerbatim}[commandchars=\\\{\}]
[10, 9, 8, 7, 6, 5, 4, 3]
\end{sphinxVerbatim}

\end{sphinxuseclass}\end{sphinxVerbatimOutput}

\end{sphinxuseclass}

\section{for loop}
\label{\detokenize{iterations:for-loop}}
\sphinxAtStartPar
The structure of a \sphinxcode{\sphinxupquote{for}} loop is as follows:

\sphinxAtStartPar
\sphinxcode{\sphinxupquote{for i in sequence:}}\\
   \sphinxcode{\sphinxupquote{                       }}     \\
   \sphinxcode{\sphinxupquote{       BLOCK CODE      }}     \\
   \sphinxcode{\sphinxupquote{                       }}     
\begin{itemize}
\item {} 
\sphinxAtStartPar
The outputs of the \sphinxstyleemphasis{range()} function and strings can be used as sequences.
\begin{itemize}
\item {} 
\sphinxAtStartPar
Each number in the output of the range() function will be an \(i\) value.

\item {} 
\sphinxAtStartPar
Each character of the strings will be an \(i\) value.

\end{itemize}

\item {} 
\sphinxAtStartPar
\(i\) is the counter, and you can choose a different name for it.

\item {} 
\sphinxAtStartPar
For each \(i\) value from the sequence, the block code will be executed.

\item {} 
\sphinxAtStartPar
The output depends on what \(i\) does in the block code.

\end{itemize}

\sphinxAtStartPar
\sphinxstylestrong{Example:} In the code below the values of \(i\) are: \(3, 4, 5\).
\begin{itemize}
\item {} 
\sphinxAtStartPar
The print statement will be executed for each \(i\) value one by one.

\item {} 
\sphinxAtStartPar
The squares of the \(i\) values will be printed.

\end{itemize}


\begin{savenotes}\sphinxattablestart
\centering
\begin{tabulary}{\linewidth}[t]{|T|T|T|}
\hline
\sphinxstyletheadfamily 
\sphinxAtStartPar
iteration \#
&\sphinxstyletheadfamily 
\sphinxAtStartPar
\(i\)
&\sphinxstyletheadfamily 
\sphinxAtStartPar
\(i^2\)
\\
\hline
\sphinxAtStartPar
1
&
\sphinxAtStartPar
3
&
\sphinxAtStartPar
9
\\
\hline
\sphinxAtStartPar
2
&
\sphinxAtStartPar
4
&
\sphinxAtStartPar
16
\\
\hline
\sphinxAtStartPar
3
&
\sphinxAtStartPar
5
&
\sphinxAtStartPar
25
\\
\hline
\end{tabulary}
\par
\sphinxattableend\end{savenotes}

\begin{sphinxuseclass}{cell}\begin{sphinxVerbatimInput}

\begin{sphinxuseclass}{cell_input}
\begin{sphinxVerbatim}[commandchars=\\\{\}]
\PYG{k}{for} \PYG{n}{i} \PYG{o+ow}{in} \PYG{n+nb}{range}\PYG{p}{(}\PYG{l+m+mi}{3}\PYG{p}{,}\PYG{l+m+mi}{6}\PYG{p}{)}\PYG{p}{:}
    \PYG{n+nb}{print}\PYG{p}{(}\PYG{n}{i}\PYG{o}{*}\PYG{o}{*}\PYG{l+m+mi}{2}\PYG{p}{)}
\end{sphinxVerbatim}

\end{sphinxuseclass}\end{sphinxVerbatimInput}
\begin{sphinxVerbatimOutput}

\begin{sphinxuseclass}{cell_output}
\begin{sphinxVerbatim}[commandchars=\\\{\}]
9
16
25
\end{sphinxVerbatim}

\end{sphinxuseclass}\end{sphinxVerbatimOutput}

\end{sphinxuseclass}
\sphinxAtStartPar
\sphinxstylestrong{Example:} In the code below the values of \(i\) are: \(1,2,3,...,10\)
\begin{itemize}
\item {} 
\sphinxAtStartPar
In every iteration \(i\) many \sphinxcode{\sphinxupquote{\&}} characters are printed.

\end{itemize}

\begin{sphinxuseclass}{cell}\begin{sphinxVerbatimInput}

\begin{sphinxuseclass}{cell_input}
\begin{sphinxVerbatim}[commandchars=\\\{\}]
\PYG{k}{for} \PYG{n}{i} \PYG{o+ow}{in} \PYG{n+nb}{range}\PYG{p}{(}\PYG{l+m+mi}{1}\PYG{p}{,}\PYG{l+m+mi}{11}\PYG{p}{)}\PYG{p}{:}
    \PYG{n+nb}{print}\PYG{p}{(}\PYG{l+s+s1}{\PYGZsq{}}\PYG{l+s+s1}{\PYGZam{}}\PYG{l+s+s1}{\PYGZsq{}}\PYG{o}{*}\PYG{n}{i}\PYG{p}{)}
\end{sphinxVerbatim}

\end{sphinxuseclass}\end{sphinxVerbatimInput}
\begin{sphinxVerbatimOutput}

\begin{sphinxuseclass}{cell_output}
\begin{sphinxVerbatim}[commandchars=\\\{\}]
\PYGZam{}
\PYGZam{}\PYGZam{}
\PYGZam{}\PYGZam{}\PYGZam{}
\PYGZam{}\PYGZam{}\PYGZam{}\PYGZam{}
\PYGZam{}\PYGZam{}\PYGZam{}\PYGZam{}\PYGZam{}
\PYGZam{}\PYGZam{}\PYGZam{}\PYGZam{}\PYGZam{}\PYGZam{}
\PYGZam{}\PYGZam{}\PYGZam{}\PYGZam{}\PYGZam{}\PYGZam{}\PYGZam{}
\PYGZam{}\PYGZam{}\PYGZam{}\PYGZam{}\PYGZam{}\PYGZam{}\PYGZam{}\PYGZam{}
\PYGZam{}\PYGZam{}\PYGZam{}\PYGZam{}\PYGZam{}\PYGZam{}\PYGZam{}\PYGZam{}\PYGZam{}
\PYGZam{}\PYGZam{}\PYGZam{}\PYGZam{}\PYGZam{}\PYGZam{}\PYGZam{}\PYGZam{}\PYGZam{}\PYGZam{}
\end{sphinxVerbatim}

\end{sphinxuseclass}\end{sphinxVerbatimOutput}

\end{sphinxuseclass}
\sphinxAtStartPar
\sphinxstylestrong{Example:} In the code below, the values of \(j\) are: \(3, 4, 5\).
\begin{itemize}
\item {} 
\sphinxAtStartPar
The block code (4 lines) will be executed for each \(j\) value one by one.

\item {} 
\sphinxAtStartPar
The value of each \sphinxstyleemphasis{frac} variable is calculated as shown in the table below.

\item {} 
\sphinxAtStartPar
The calculated \sphinxstyleemphasis{frac} values will be printed.

\end{itemize}


\begin{savenotes}\sphinxattablestart
\centering
\begin{tabulary}{\linewidth}[t]{|T|T|T|T|T|}
\hline
\sphinxstyletheadfamily 
\sphinxAtStartPar
iteration \#
&\sphinxstyletheadfamily 
\sphinxAtStartPar
j
&\sphinxstyletheadfamily 
\sphinxAtStartPar
num = \(3\times j+2\)
&\sphinxstyletheadfamily 
\sphinxAtStartPar
den = \(10^j\)
&\sphinxstyletheadfamily 
\sphinxAtStartPar
frac=num/den
\\
\hline
\sphinxAtStartPar
1
&
\sphinxAtStartPar
3
&
\sphinxAtStartPar
\(3\times 3+2=11\)
&
\sphinxAtStartPar
\(10^3=1,000\)
&
\sphinxAtStartPar
0.011
\\
\hline
\sphinxAtStartPar
2
&
\sphinxAtStartPar
4
&
\sphinxAtStartPar
\(3\times 4+2=14\)
&
\sphinxAtStartPar
\(10^4=10,000\)
&
\sphinxAtStartPar
0.0014
\\
\hline
\sphinxAtStartPar
3
&
\sphinxAtStartPar
5
&
\sphinxAtStartPar
\(3\times 5+2=17\)
&
\sphinxAtStartPar
\(10^5=100,000\)
&
\sphinxAtStartPar
0.00017
\\
\hline
\end{tabulary}
\par
\sphinxattableend\end{savenotes}

\begin{sphinxuseclass}{cell}\begin{sphinxVerbatimInput}

\begin{sphinxuseclass}{cell_input}
\begin{sphinxVerbatim}[commandchars=\\\{\}]
\PYG{k}{for} \PYG{n}{j} \PYG{o+ow}{in} \PYG{n+nb}{range}\PYG{p}{(}\PYG{l+m+mi}{3}\PYG{p}{,}\PYG{l+m+mi}{6}\PYG{p}{)}\PYG{p}{:}
    \PYG{n}{num} \PYG{o}{=} \PYG{l+m+mi}{3}\PYG{o}{*}\PYG{n}{j}\PYG{o}{+}\PYG{l+m+mi}{2}
    \PYG{n}{den} \PYG{o}{=} \PYG{l+m+mi}{10}\PYG{o}{*}\PYG{o}{*}\PYG{n}{j}
    \PYG{n}{frac} \PYG{o}{=} \PYG{n}{num}\PYG{o}{/}\PYG{n}{den}
    \PYG{n+nb}{print}\PYG{p}{(}\PYG{n}{frac}\PYG{p}{)}
\end{sphinxVerbatim}

\end{sphinxuseclass}\end{sphinxVerbatimInput}
\begin{sphinxVerbatimOutput}

\begin{sphinxuseclass}{cell_output}
\begin{sphinxVerbatim}[commandchars=\\\{\}]
0.011
0.0014
0.00017
\end{sphinxVerbatim}

\end{sphinxuseclass}\end{sphinxVerbatimOutput}

\end{sphinxuseclass}
\sphinxAtStartPar
\sphinxstylestrong{Example:} In the code below, the values of \(i\) are: ‘u’, ‘t’, ‘a’, ‘h’.
\begin{itemize}
\item {} 
\sphinxAtStartPar
In every iteration, the value of \(i\), which is a character of ‘utah’ is printed.

\end{itemize}

\begin{sphinxuseclass}{cell}\begin{sphinxVerbatimInput}

\begin{sphinxuseclass}{cell_input}
\begin{sphinxVerbatim}[commandchars=\\\{\}]
\PYG{k}{for} \PYG{n}{i} \PYG{o+ow}{in} \PYG{l+s+s1}{\PYGZsq{}}\PYG{l+s+s1}{utah}\PYG{l+s+s1}{\PYGZsq{}}\PYG{p}{:}
    \PYG{n+nb}{print}\PYG{p}{(}\PYG{n}{i}\PYG{p}{)}
\end{sphinxVerbatim}

\end{sphinxuseclass}\end{sphinxVerbatimInput}
\begin{sphinxVerbatimOutput}

\begin{sphinxuseclass}{cell_output}
\begin{sphinxVerbatim}[commandchars=\\\{\}]
u
t
a
h
\end{sphinxVerbatim}

\end{sphinxuseclass}\end{sphinxVerbatimOutput}

\end{sphinxuseclass}
\sphinxAtStartPar
\sphinxstylestrong{Example:} In the code below, the values of \(i\) are: ‘u’, ‘t’, ‘a’, ‘h’.
\begin{itemize}
\item {} 
\sphinxAtStartPar
The condition of the \sphinxstyleemphasis{if} statement is True if the value of \(i\) is before ‘k’ in dictionary order.
\begin{itemize}
\item {} 
\sphinxAtStartPar
This is False for ‘u’ and ‘t’, so they are not printed.

\item {} 
\sphinxAtStartPar
This is True for ‘a’ and ‘h’, so they are printed.

\end{itemize}

\end{itemize}

\begin{sphinxuseclass}{cell}\begin{sphinxVerbatimInput}

\begin{sphinxuseclass}{cell_input}
\begin{sphinxVerbatim}[commandchars=\\\{\}]
\PYG{k}{for} \PYG{n}{i} \PYG{o+ow}{in} \PYG{l+s+s1}{\PYGZsq{}}\PYG{l+s+s1}{utah}\PYG{l+s+s1}{\PYGZsq{}}\PYG{p}{:}
    \PYG{k}{if} \PYG{n}{i} \PYG{o}{\PYGZlt{}} \PYG{l+s+s1}{\PYGZsq{}}\PYG{l+s+s1}{k}\PYG{l+s+s1}{\PYGZsq{}}\PYG{p}{:}
        \PYG{n+nb}{print}\PYG{p}{(}\PYG{n}{i}\PYG{p}{)}
\end{sphinxVerbatim}

\end{sphinxuseclass}\end{sphinxVerbatimInput}
\begin{sphinxVerbatimOutput}

\begin{sphinxuseclass}{cell_output}
\begin{sphinxVerbatim}[commandchars=\\\{\}]
a
h
\end{sphinxVerbatim}

\end{sphinxuseclass}\end{sphinxVerbatimOutput}

\end{sphinxuseclass}
\sphinxAtStartPar
\sphinxstylestrong{Example:} In the code below, the values of \(i\) are: \(3, 4, 5\).
\begin{itemize}
\item {} 
\sphinxAtStartPar
The initial value of the \(total\) variable is \(0\).

\item {} 
\sphinxAtStartPar
In each iteration, the value of \(i\) is added to the \(total\).

\end{itemize}


\begin{savenotes}\sphinxattablestart
\centering
\begin{tabulary}{\linewidth}[t]{|T|T|T|}
\hline
\sphinxstyletheadfamily 
\sphinxAtStartPar
iteration \#
&\sphinxstyletheadfamily 
\sphinxAtStartPar
\(i\)
&\sphinxstyletheadfamily 
\sphinxAtStartPar
\(total\)
\\
\hline
\sphinxAtStartPar
\sphinxhyphen{}
&
\sphinxAtStartPar
\sphinxhyphen{}
&
\sphinxAtStartPar
0
\\
\hline
\sphinxAtStartPar
1
&
\sphinxAtStartPar
3
&
\sphinxAtStartPar
0+3=3
\\
\hline
\sphinxAtStartPar
2
&
\sphinxAtStartPar
4
&
\sphinxAtStartPar
3+4=7
\\
\hline
\sphinxAtStartPar
3
&
\sphinxAtStartPar
5
&
\sphinxAtStartPar
7+5=12
\\
\hline
\end{tabulary}
\par
\sphinxattableend\end{savenotes}

\begin{sphinxuseclass}{cell}\begin{sphinxVerbatimInput}

\begin{sphinxuseclass}{cell_input}
\begin{sphinxVerbatim}[commandchars=\\\{\}]
\PYG{n}{total} \PYG{o}{=} \PYG{l+m+mi}{0}

\PYG{k}{for} \PYG{n}{i} \PYG{o+ow}{in} \PYG{n+nb}{range}\PYG{p}{(}\PYG{l+m+mi}{3}\PYG{p}{,}\PYG{l+m+mi}{6}\PYG{p}{)}\PYG{p}{:}
    \PYG{n}{total} \PYG{o}{+}\PYG{o}{=} \PYG{n}{i}
    \PYG{n+nb}{print}\PYG{p}{(}\PYG{l+s+sa}{f}\PYG{l+s+s1}{\PYGZsq{}}\PYG{l+s+s1}{Iteration number:}\PYG{l+s+si}{\PYGZob{}}\PYG{n}{i}\PYG{o}{\PYGZhy{}}\PYG{l+m+mi}{2}\PYG{l+s+si}{\PYGZcb{}}\PYG{l+s+s1}{   i:}\PYG{l+s+si}{\PYGZob{}}\PYG{n}{i}\PYG{l+s+si}{\PYGZcb{}}\PYG{l+s+s1}{\PYGZhy{}\PYGZhy{}\PYGZhy{}\PYGZgt{}total:}\PYG{l+s+si}{\PYGZob{}}\PYG{n}{total}\PYG{l+s+si}{\PYGZcb{}}\PYG{l+s+s1}{\PYGZsq{}}\PYG{p}{)}
\end{sphinxVerbatim}

\end{sphinxuseclass}\end{sphinxVerbatimInput}
\begin{sphinxVerbatimOutput}

\begin{sphinxuseclass}{cell_output}
\begin{sphinxVerbatim}[commandchars=\\\{\}]
Iteration number:1   i:3\PYGZhy{}\PYGZhy{}\PYGZhy{}\PYGZgt{}total:3
Iteration number:2   i:4\PYGZhy{}\PYGZhy{}\PYGZhy{}\PYGZgt{}total:7
Iteration number:3   i:5\PYGZhy{}\PYGZhy{}\PYGZhy{}\PYGZgt{}total:12
\end{sphinxVerbatim}

\end{sphinxuseclass}\end{sphinxVerbatimOutput}

\end{sphinxuseclass}

\subsection{break and continue}
\label{\detokenize{iterations:break-and-continue}}
\sphinxAtStartPar
\sphinxcode{\sphinxupquote{break}} is used to terminate the \sphinxstyleemphasis{for} loop.
\begin{itemize}
\item {} 
\sphinxAtStartPar
It is usually used in an \sphinxstyleemphasis{if} statement to terminate the \sphinxstyleemphasis{for} loop under certain conditions.

\end{itemize}

\sphinxAtStartPar
\sphinxcode{\sphinxupquote{continue}} is used to skip the rest of the body code of the \sphinxstyleemphasis{for} loop.
\begin{itemize}
\item {} 
\sphinxAtStartPar
It goes back to the beginning of the loop.

\item {} 
\sphinxAtStartPar
It does not terminate the loop, just skips the rest of the block code for that iteration.

\end{itemize}

\sphinxAtStartPar
\sphinxstylestrong{Example:} In the code below, the values of \(i\) are: \(1,2,3,4\).
\begin{itemize}
\item {} 
\sphinxAtStartPar
for \(i=1\) and \(i=2\), the \sphinxstyleemphasis{if} part is not executed since its condition is False, and the values \(1\) and \(2\) are printed.

\item {} 
\sphinxAtStartPar
for \(i=3\), \sphinxcode{\sphinxupquote{break}} is executed, and the loop is terminated.

\end{itemize}

\begin{sphinxuseclass}{cell}\begin{sphinxVerbatimInput}

\begin{sphinxuseclass}{cell_input}
\begin{sphinxVerbatim}[commandchars=\\\{\}]
\PYG{k}{for} \PYG{n}{i} \PYG{o+ow}{in} \PYG{n+nb}{range}\PYG{p}{(}\PYG{l+m+mi}{1}\PYG{p}{,}\PYG{l+m+mi}{5}\PYG{p}{)}\PYG{p}{:}
    \PYG{k}{if} \PYG{n}{i} \PYG{o}{==} \PYG{l+m+mi}{3}\PYG{p}{:} 
       \PYG{k}{break}
    \PYG{n+nb}{print}\PYG{p}{(}\PYG{n}{i}\PYG{p}{)}
\end{sphinxVerbatim}

\end{sphinxuseclass}\end{sphinxVerbatimInput}
\begin{sphinxVerbatimOutput}

\begin{sphinxuseclass}{cell_output}
\begin{sphinxVerbatim}[commandchars=\\\{\}]
1
2
\end{sphinxVerbatim}

\end{sphinxuseclass}\end{sphinxVerbatimOutput}

\end{sphinxuseclass}
\sphinxAtStartPar
\sphinxstylestrong{Example:} In the code below, the values of \(i\) are: \(1,2,3,4\).
\begin{itemize}
\item {} 
\sphinxAtStartPar
For \(i=1\) and \(i=2\), the \sphinxstyleemphasis{if} part is not executed since its condition is False, and the values \(1\) and \(2\) are printed.

\item {} 
\sphinxAtStartPar
For \(i=3\), \sphinxcode{\sphinxupquote{continue}} is executed, and the print statement is skipped, and the \(i=4\) iteration is started.

\item {} 
\sphinxAtStartPar
For \(i=4\), the \sphinxstyleemphasis{if} part is not executed since its condition is False, and the value \(4\) is printed

\end{itemize}

\begin{sphinxuseclass}{cell}\begin{sphinxVerbatimInput}

\begin{sphinxuseclass}{cell_input}
\begin{sphinxVerbatim}[commandchars=\\\{\}]
\PYG{k}{for} \PYG{n}{i} \PYG{o+ow}{in} \PYG{n+nb}{range}\PYG{p}{(}\PYG{l+m+mi}{1}\PYG{p}{,}\PYG{l+m+mi}{5}\PYG{p}{)}\PYG{p}{:}
    \PYG{k}{if} \PYG{n}{i} \PYG{o}{==} \PYG{l+m+mi}{3}\PYG{p}{:} 
       \PYG{k}{continue}
    \PYG{k}{else}\PYG{p}{:}                 
       \PYG{n+nb}{print}\PYG{p}{(}\PYG{n}{i}\PYG{p}{)}     
\end{sphinxVerbatim}

\end{sphinxuseclass}\end{sphinxVerbatimInput}
\begin{sphinxVerbatimOutput}

\begin{sphinxuseclass}{cell_output}
\begin{sphinxVerbatim}[commandchars=\\\{\}]
1
2
4
\end{sphinxVerbatim}

\end{sphinxuseclass}\end{sphinxVerbatimOutput}

\end{sphinxuseclass}

\subsection{for and else}
\label{\detokenize{iterations:for-and-else}}
\sphinxAtStartPar
\sphinxstyleemphasis{for} loops can have an \sphinxcode{\sphinxupquote{else}} statement.
\begin{itemize}
\item {} 
\sphinxAtStartPar
The \sphinxcode{\sphinxupquote{else}} statement is executed when the \sphinxstyleemphasis{for} loop is completed without any \sphinxstyleemphasis{break}.

\end{itemize}

\sphinxAtStartPar
\sphinxstylestrong{Example:} In the code below, the values of \(i\) are: \(1,2,3,4\).
\begin{itemize}
\item {} 
\sphinxAtStartPar
After executing the print() function for \(i=4\), the for loop is over, and the else part is executed.

\end{itemize}

\begin{sphinxuseclass}{cell}\begin{sphinxVerbatimInput}

\begin{sphinxuseclass}{cell_input}
\begin{sphinxVerbatim}[commandchars=\\\{\}]
\PYG{k}{for} \PYG{n}{i} \PYG{o+ow}{in} \PYG{n+nb}{range}\PYG{p}{(}\PYG{l+m+mi}{1}\PYG{p}{,}\PYG{l+m+mi}{5}\PYG{p}{)}\PYG{p}{:}
    \PYG{n+nb}{print}\PYG{p}{(}\PYG{n}{i}\PYG{p}{)}
\PYG{k}{else}\PYG{p}{:}                 
    \PYG{n+nb}{print}\PYG{p}{(}\PYG{l+s+s1}{\PYGZsq{}}\PYG{l+s+s1}{Over}\PYG{l+s+s1}{\PYGZsq{}}\PYG{p}{)}     
\end{sphinxVerbatim}

\end{sphinxuseclass}\end{sphinxVerbatimInput}
\begin{sphinxVerbatimOutput}

\begin{sphinxuseclass}{cell_output}
\begin{sphinxVerbatim}[commandchars=\\\{\}]
1
2
3
4
Over
\end{sphinxVerbatim}

\end{sphinxuseclass}\end{sphinxVerbatimOutput}

\end{sphinxuseclass}
\sphinxAtStartPar
\sphinxstylestrong{Example:} In the code below, the values of \(i\) are: 1,2,3,4.
\begin{itemize}
\item {} 
\sphinxAtStartPar
The condition of the \sphinxstyleemphasis{if} statement is True when \(i\) is 4, and the \sphinxstyleemphasis{break} is executed, so the \sphinxstyleemphasis{for} loop is terminated.
\begin{itemize}
\item {} 
\sphinxAtStartPar
The \sphinxstyleemphasis{else} part is not executed.

\end{itemize}

\end{itemize}

\begin{sphinxuseclass}{cell}\begin{sphinxVerbatimInput}

\begin{sphinxuseclass}{cell_input}
\begin{sphinxVerbatim}[commandchars=\\\{\}]
\PYG{k}{for} \PYG{n}{i} \PYG{o+ow}{in} \PYG{n+nb}{range}\PYG{p}{(}\PYG{l+m+mi}{1}\PYG{p}{,}\PYG{l+m+mi}{5}\PYG{p}{)}\PYG{p}{:}
    \PYG{n+nb}{print}\PYG{p}{(}\PYG{n}{i}\PYG{p}{)}
    \PYG{k}{if} \PYG{n}{i}\PYG{o}{\PYGZgt{}} \PYG{l+m+mi}{3}\PYG{p}{:}
        \PYG{k}{break}            
\PYG{k}{else}\PYG{p}{:}                 
    \PYG{n+nb}{print}\PYG{p}{(}\PYG{l+s+s1}{\PYGZsq{}}\PYG{l+s+s1}{Over}\PYG{l+s+s1}{\PYGZsq{}}\PYG{p}{)}     
\end{sphinxVerbatim}

\end{sphinxuseclass}\end{sphinxVerbatimInput}
\begin{sphinxVerbatimOutput}

\begin{sphinxuseclass}{cell_output}
\begin{sphinxVerbatim}[commandchars=\\\{\}]
1
2
3
4
\end{sphinxVerbatim}

\end{sphinxuseclass}\end{sphinxVerbatimOutput}

\end{sphinxuseclass}

\section{while loop}
\label{\detokenize{iterations:while-loop}}
\sphinxAtStartPar
The \sphinxcode{\sphinxupquote{while}} loops are similar to the \sphinxstyleemphasis{if} statements. In \sphinxstyleemphasis{if} statements, block code is executed only once when the condition is True, whereas in \sphinxcode{\sphinxupquote{while}} loops, the block code might be executed more than once.
\begin{itemize}
\item {} 
\sphinxAtStartPar
If the condition is True, the block code is executed as in \sphinxstyleemphasis{if} statements, but then the condition is checked again.

\item {} 
\sphinxAtStartPar
If it is still True, the block code of the \sphinxstyleemphasis{while} loop is executed again.

\item {} 
\sphinxAtStartPar
This process continues as long as the condition is True.

\item {} 
\sphinxAtStartPar
Whenever the condition becomes False, the \sphinxcode{\sphinxupquote{while}} loop is terminated.

\end{itemize}

\sphinxAtStartPar
The structure of a \sphinxcode{\sphinxupquote{while}} loop is as follows:

\sphinxAtStartPar
\sphinxcode{\sphinxupquote{while condition:}}\\
   \sphinxcode{\sphinxupquote{                       }}     \\
   \sphinxcode{\sphinxupquote{       BLOCK CODE      }}     \\
   \sphinxcode{\sphinxupquote{                       }}     
\begin{itemize}
\item {} 
\sphinxAtStartPar
\sphinxcode{\sphinxupquote{condition}} is a Boolean expression (True or False)

\item {} 
\sphinxAtStartPar
Possible conditions:
\begin{itemize}
\item {} 
\sphinxAtStartPar
True, False;

\item {} 
\sphinxAtStartPar
<, >, <=, >=, ==, !=

\item {} 
\sphinxAtStartPar
not, and, or

\item {} 
\sphinxAtStartPar
numbers, strings

\end{itemize}

\end{itemize}

\sphinxAtStartPar
\sphinxstylestrong{Example:}
In the code below, the initial value of \(n\) is 3.
\begin{enumerate}
\sphinxsetlistlabels{\arabic}{enumi}{enumii}{}{.}%
\item {} 
\sphinxAtStartPar
\(n=3\): Check the condition.
\begin{itemize}
\item {} 
\sphinxAtStartPar
Since \(3 > 0\), the condition is True, and the block code is executed.

\item {} 
\sphinxAtStartPar
\(3\) is printed, and \(n\) becomes \(3 - 1 = 2\).

\end{itemize}

\item {} 
\sphinxAtStartPar
\(n=2\): Check the condition.
\begin{itemize}
\item {} 
\sphinxAtStartPar
Since \(2 > 0\), the condition is True, and the block code is executed.

\item {} 
\sphinxAtStartPar
\(2\) is printed, and \(n\) becomes \(2 - 1 = 1\).

\end{itemize}

\item {} 
\sphinxAtStartPar
\(n=1\): Check the condition.
\begin{itemize}
\item {} 
\sphinxAtStartPar
Since \(1 > 0\), the condition is True, and the block code is executed.

\item {} 
\sphinxAtStartPar
\(1\) is printed, and \(n\) becomes \(1 - 1 = 0\).

\end{itemize}

\item {} 
\sphinxAtStartPar
\(n=0\): Check the condition.
\begin{itemize}
\item {} 
\sphinxAtStartPar
Since \(0 > 0\) is False, the condition is False, and the loop is terminated.

\end{itemize}

\end{enumerate}

\begin{sphinxuseclass}{cell}\begin{sphinxVerbatimInput}

\begin{sphinxuseclass}{cell_input}
\begin{sphinxVerbatim}[commandchars=\\\{\}]
\PYG{n}{n} \PYG{o}{=} \PYG{l+m+mi}{3}

\PYG{k}{while} \PYG{n}{n}\PYG{o}{\PYGZgt{}}\PYG{l+m+mi}{0}\PYG{p}{:}
    \PYG{n+nb}{print}\PYG{p}{(}\PYG{n}{n}\PYG{p}{)}
    \PYG{n}{n} \PYG{o}{\PYGZhy{}}\PYG{o}{=} \PYG{l+m+mi}{1}
\end{sphinxVerbatim}

\end{sphinxuseclass}\end{sphinxVerbatimInput}
\begin{sphinxVerbatimOutput}

\begin{sphinxuseclass}{cell_output}
\begin{sphinxVerbatim}[commandchars=\\\{\}]
3
2
1
\end{sphinxVerbatim}

\end{sphinxuseclass}\end{sphinxVerbatimOutput}

\end{sphinxuseclass}
\sphinxAtStartPar
\sphinxstylestrong{Example:} In the code below, the initial value of \(n\) is \(3\).
\begin{enumerate}
\sphinxsetlistlabels{\arabic}{enumi}{enumii}{}{.}%
\item {} 
\sphinxAtStartPar
\(n=3\): Check the condition.
\begin{itemize}
\item {} 
\sphinxAtStartPar
Since \sphinxstyleemphasis{bool(3)} is True, the condition is True, and the block code is executed.

\item {} 
\sphinxAtStartPar
\(3\) is printed, and \(n\) becomes \(3-1=2\).

\end{itemize}

\item {} 
\sphinxAtStartPar
\(n=2\): Check the condition.
\begin{itemize}
\item {} 
\sphinxAtStartPar
Since \sphinxstyleemphasis{bool(2)} is True, the condition is True, and the block code is executed.

\item {} 
\sphinxAtStartPar
\(2\) is printed, and \(n\) becomes \(2-1=1\).

\end{itemize}

\item {} 
\sphinxAtStartPar
\(n=1\): Check the condition.
\begin{itemize}
\item {} 
\sphinxAtStartPar
Since \sphinxstyleemphasis{bool(1)} is True, the condition is True, and the block code is executed.

\item {} 
\sphinxAtStartPar
\sphinxstyleemphasis{1} is printed, and \(n\) becomes \(1-1=0\).

\end{itemize}

\item {} 
\sphinxAtStartPar
\(n=0\): Check the condition.
\begin{itemize}
\item {} 
\sphinxAtStartPar
Since \sphinxstyleemphasis{bool(0)} is False, the condition is False, and the \sphinxstyleemphasis{while} loop is terminated

\end{itemize}

\end{enumerate}

\begin{sphinxuseclass}{cell}\begin{sphinxVerbatimInput}

\begin{sphinxuseclass}{cell_input}
\begin{sphinxVerbatim}[commandchars=\\\{\}]
\PYG{n}{n} \PYG{o}{=} \PYG{l+m+mi}{3}

\PYG{k}{while} \PYG{n}{n}\PYG{p}{:}
    \PYG{n+nb}{print}\PYG{p}{(}\PYG{n}{n}\PYG{p}{)}
    \PYG{n}{n} \PYG{o}{\PYGZhy{}}\PYG{o}{=} \PYG{l+m+mi}{1}
\end{sphinxVerbatim}

\end{sphinxuseclass}\end{sphinxVerbatimInput}
\begin{sphinxVerbatimOutput}

\begin{sphinxuseclass}{cell_output}
\begin{sphinxVerbatim}[commandchars=\\\{\}]
3
2
1
\end{sphinxVerbatim}

\end{sphinxuseclass}\end{sphinxVerbatimOutput}

\end{sphinxuseclass}
\sphinxAtStartPar
\sphinxstylestrong{Example:} In the code below, the initial values are set with \(n=3\) and \(\text{total}=0\).
\begin{enumerate}
\sphinxsetlistlabels{\arabic}{enumi}{enumii}{}{.}%
\item {} 
\sphinxAtStartPar
\(n=3, \text{total}=0\): Check the condition.
\begin{itemize}
\item {} 
\sphinxAtStartPar
Since \(3 < 6\), the condition is True, and the block code is executed.

\item {} 
\sphinxAtStartPar
\(\text{total} = 0 + 3 = 3\)

\item {} 
\sphinxAtStartPar
print Iteration number: 1 \(n:3 \rightarrow \text{total}:3\)

\item {} 
\sphinxAtStartPar
\(n = 3+1 = 4\)

\end{itemize}

\item {} 
\sphinxAtStartPar
\(n=4, \text{total}=3\): Check the condition.
\begin{itemize}
\item {} 
\sphinxAtStartPar
Since \(4 < 6\), the condition is True, and the block code is executed.

\item {} 
\sphinxAtStartPar
\(\text{total} = 3 + 4 = 7\)

\item {} 
\sphinxAtStartPar
print Iteration number: 2 \(n:4 \rightarrow \text{total}:7\)

\item {} 
\sphinxAtStartPar
\(n = 4+1 = 5\)

\end{itemize}

\item {} 
\sphinxAtStartPar
\(n=5, \text{total}=7\): Check the condition.
\begin{itemize}
\item {} 
\sphinxAtStartPar
Since \(5 < 6\), the condition is True, and the block code is executed.

\item {} 
\sphinxAtStartPar
\(\text{total} = 7 + 5 = 12\)

\item {} 
\sphinxAtStartPar
print Iteration number: 3 \(n:5 \rightarrow \text{total}:12\)

\item {} 
\sphinxAtStartPar
\(n = 5+1 = 6\)

\end{itemize}

\item {} 
\sphinxAtStartPar
\(n=6, \text{total}=12\): Check the condition.
\begin{itemize}
\item {} 
\sphinxAtStartPar
Since \(6 > 6\) is False, the condition is False, and the while loop is terminated.

\end{itemize}

\end{enumerate}


\begin{savenotes}\sphinxattablestart
\centering
\begin{tabulary}{\linewidth}[t]{|T|T|T|}
\hline
\sphinxstyletheadfamily 
\sphinxAtStartPar
iteration \#
&\sphinxstyletheadfamily 
\sphinxAtStartPar
\(n\)
&\sphinxstyletheadfamily 
\sphinxAtStartPar
\(total\)
\\
\hline
\sphinxAtStartPar
\sphinxhyphen{}
&
\sphinxAtStartPar
\sphinxhyphen{}
&
\sphinxAtStartPar
0
\\
\hline
\sphinxAtStartPar
1
&
\sphinxAtStartPar
3
&
\sphinxAtStartPar
0+3=3
\\
\hline
\sphinxAtStartPar
2
&
\sphinxAtStartPar
4
&
\sphinxAtStartPar
3+4=7
\\
\hline
\sphinxAtStartPar
3
&
\sphinxAtStartPar
5
&
\sphinxAtStartPar
7+5=12
\\
\hline
\end{tabulary}
\par
\sphinxattableend\end{savenotes}

\begin{sphinxuseclass}{cell}\begin{sphinxVerbatimInput}

\begin{sphinxuseclass}{cell_input}
\begin{sphinxVerbatim}[commandchars=\\\{\}]
\PYG{n}{total} \PYG{o}{=} \PYG{l+m+mi}{0}
\PYG{n}{n} \PYG{o}{=} \PYG{l+m+mi}{3}
\PYG{k}{while} \PYG{n}{n}\PYG{o}{\PYGZlt{}}\PYG{l+m+mi}{6}\PYG{p}{:}
    \PYG{n}{total} \PYG{o}{+}\PYG{o}{=} \PYG{n}{n}
    \PYG{n+nb}{print}\PYG{p}{(}\PYG{l+s+sa}{f}\PYG{l+s+s1}{\PYGZsq{}}\PYG{l+s+s1}{Iteration number:}\PYG{l+s+si}{\PYGZob{}}\PYG{n}{n}\PYG{o}{\PYGZhy{}}\PYG{l+m+mi}{2}\PYG{l+s+si}{\PYGZcb{}}\PYG{l+s+s1}{   n:}\PYG{l+s+si}{\PYGZob{}}\PYG{n}{n}\PYG{l+s+si}{\PYGZcb{}}\PYG{l+s+s1}{\PYGZhy{}\PYGZhy{}\PYGZhy{}\PYGZgt{}total:}\PYG{l+s+si}{\PYGZob{}}\PYG{n}{total}\PYG{l+s+si}{\PYGZcb{}}\PYG{l+s+s1}{\PYGZsq{}}\PYG{p}{)}
    \PYG{n}{n} \PYG{o}{+}\PYG{o}{=} \PYG{l+m+mi}{1}
\end{sphinxVerbatim}

\end{sphinxuseclass}\end{sphinxVerbatimInput}
\begin{sphinxVerbatimOutput}

\begin{sphinxuseclass}{cell_output}
\begin{sphinxVerbatim}[commandchars=\\\{\}]
Iteration number:1   n:3\PYGZhy{}\PYGZhy{}\PYGZhy{}\PYGZgt{}total:3
Iteration number:2   n:4\PYGZhy{}\PYGZhy{}\PYGZhy{}\PYGZgt{}total:7
Iteration number:3   n:5\PYGZhy{}\PYGZhy{}\PYGZhy{}\PYGZgt{}total:12
\end{sphinxVerbatim}

\end{sphinxuseclass}\end{sphinxVerbatimOutput}

\end{sphinxuseclass}

\subsection{if versus while}
\label{\detokenize{iterations:if-versus-while}}
\sphinxAtStartPar
In the following two examples, the same block code is used in an \sphinxstyleemphasis{if} statement and a \sphinxstyleemphasis{while} loop.
\begin{itemize}
\item {} 
\sphinxAtStartPar
In the \sphinxstyleemphasis{if} statement, the block code is executed only once for \(n=3\).

\item {} 
\sphinxAtStartPar
In the \sphinxstyleemphasis{while} loop, the block code is executed two times for \(n=3\) and \(n=2\).

\end{itemize}

\sphinxAtStartPar
\sphinxstylestrong{Example\sphinxhyphen{}1:} In the code below, the condition of the \sphinxstyleemphasis{if} statement is True, and the block code is executed.
\begin{itemize}
\item {} 
\sphinxAtStartPar
The print statement is executed, and the value of \(n\), which is \(3\), is printed.

\item {} 
\sphinxAtStartPar
Afterward, \(n\) is updated to \(3-1=2\).

\item {} 
\sphinxAtStartPar
The \sphinxstyleemphasis{if} statement is then concluded, and the only output is 3.

\end{itemize}

\begin{sphinxuseclass}{cell}\begin{sphinxVerbatimInput}

\begin{sphinxuseclass}{cell_input}
\begin{sphinxVerbatim}[commandchars=\\\{\}]
\PYG{n}{n} \PYG{o}{=} \PYG{l+m+mi}{3}

\PYG{k}{if} \PYG{n}{n} \PYG{o}{\PYGZgt{}} \PYG{l+m+mi}{1}\PYG{p}{:}
  \PYG{n+nb}{print}\PYG{p}{(}\PYG{n}{n}\PYG{p}{)}
  \PYG{n}{n} \PYG{o}{=} \PYG{n}{n}\PYG{o}{\PYGZhy{}}\PYG{l+m+mi}{1}
\end{sphinxVerbatim}

\end{sphinxuseclass}\end{sphinxVerbatimInput}
\begin{sphinxVerbatimOutput}

\begin{sphinxuseclass}{cell_output}
\begin{sphinxVerbatim}[commandchars=\\\{\}]
3
\end{sphinxVerbatim}

\end{sphinxuseclass}\end{sphinxVerbatimOutput}

\end{sphinxuseclass}
\sphinxAtStartPar
\sphinxstylestrong{Example\sphinxhyphen{}2:} In the code below, the initial value of \(n\) is 3.
\begin{enumerate}
\sphinxsetlistlabels{\arabic}{enumi}{enumii}{}{.}%
\item {} 
\sphinxAtStartPar
\(n=3\): Check the condition.
\begin{itemize}
\item {} 
\sphinxAtStartPar
Since \(3 > 1\), the condition is True, and the block code is executed.

\item {} 
\sphinxAtStartPar
\(3\) is printed, and \(n\) becomes \(3-1=2\).

\end{itemize}

\item {} 
\sphinxAtStartPar
\(n=2\): Check the condition.
\begin{itemize}
\item {} 
\sphinxAtStartPar
Since \(2 > 1\), the condition is True, and the block code is executed.

\item {} 
\sphinxAtStartPar
\(2\) is printed, and \(n\) becomes \(2-1=1\).

\end{itemize}

\item {} 
\sphinxAtStartPar
\(n=1\): Check the condition.
\begin{itemize}
\item {} 
\sphinxAtStartPar
Since \(1 > 1\) is False, the condition is False, and the \sphinxstyleemphasis{while} loop is terminated.

\end{itemize}

\end{enumerate}

\begin{sphinxuseclass}{cell}\begin{sphinxVerbatimInput}

\begin{sphinxuseclass}{cell_input}
\begin{sphinxVerbatim}[commandchars=\\\{\}]
\PYG{n}{n} \PYG{o}{=} \PYG{l+m+mi}{3}

\PYG{k}{while} \PYG{n}{n} \PYG{o}{\PYGZgt{}} \PYG{l+m+mi}{1}\PYG{p}{:}
  \PYG{n+nb}{print}\PYG{p}{(}\PYG{n}{n}\PYG{p}{)}
  \PYG{n}{n} \PYG{o}{=} \PYG{n}{n}\PYG{o}{\PYGZhy{}}\PYG{l+m+mi}{1}
\end{sphinxVerbatim}

\end{sphinxuseclass}\end{sphinxVerbatimInput}
\begin{sphinxVerbatimOutput}

\begin{sphinxuseclass}{cell_output}
\begin{sphinxVerbatim}[commandchars=\\\{\}]
3
2
\end{sphinxVerbatim}

\end{sphinxuseclass}\end{sphinxVerbatimOutput}

\end{sphinxuseclass}

\subsection{Infinite Loop}
\label{\detokenize{iterations:infinite-loop}}
\sphinxAtStartPar
It is possible to have a condition for a \sphinxstyleemphasis{while} loop that is always True.
\begin{itemize}
\item {} 
\sphinxAtStartPar
In that case, the block code will be executed repeatedly unless the program is terminated by the user.

\item {} 
\sphinxAtStartPar
This is not a syntax error, but it is not expected because the program will not end.

\end{itemize}

\sphinxAtStartPar
\sphinxstylestrong{Example:} In the code below, the initial value of \(n\) is \(3\), and in each iteration, \(n\) is increased by \(1\).
\begin{itemize}
\item {} 
\sphinxAtStartPar
Hence, the values of \(n\) are: \(3, 4, 5, 6, \ldots\).

\item {} 
\sphinxAtStartPar
The condition \(n > 1\) is always True, and the \sphinxstyleemphasis{while} loop never terminates.

\end{itemize}

\begin{sphinxVerbatim}[commandchars=\\\{\}]
\PYG{c+c1}{\PYGZsh{} infinite loop}
\PYG{n}{n} \PYG{o}{=} \PYG{l+m+mi}{3}

\PYG{k}{while} \PYG{n}{n} \PYG{o}{\PYGZgt{}} \PYG{l+m+mi}{1}\PYG{p}{:}         \PYG{c+c1}{\PYGZsh{} always True}
  \PYG{n+nb}{print}\PYG{p}{(}\PYG{n}{n}\PYG{p}{)}
  \PYG{n}{n} \PYG{o}{=} \PYG{n}{n}\PYG{o}{+}\PYG{l+m+mi}{1}            \PYG{c+c1}{\PYGZsh{} n=3,4,5,......}
\end{sphinxVerbatim}

\sphinxAtStartPar
\sphinxstylestrong{Example:} In the code below, the condition is always True.
\begin{itemize}
\item {} 
\sphinxAtStartPar
‘Hello’ is printed repeatedly, and the loop never terminates.

\end{itemize}

\begin{sphinxVerbatim}[commandchars=\\\{\}]
\PYG{c+c1}{\PYGZsh{} infinite loop}

\PYG{k}{while} \PYG{k+kc}{True}\PYG{p}{:}         
  \PYG{n+nb}{print}\PYG{p}{(}\PYG{l+s+s1}{\PYGZsq{}}\PYG{l+s+s1}{Hello}\PYG{l+s+s1}{\PYGZsq{}}\PYG{p}{)}           
\end{sphinxVerbatim}


\subsection{break and continue}
\label{\detokenize{iterations:id1}}
\sphinxAtStartPar
They work similarly to the \sphinxstyleemphasis{for} loop. To prevent infinite loops, the use of \sphinxcode{\sphinxupquote{break}} is particularly crucial for while loops.

\sphinxAtStartPar
\sphinxstylestrong{Example:} In the code below the initial value of \(n\) is \(3\).
\begin{enumerate}
\sphinxsetlistlabels{\arabic}{enumi}{enumii}{}{.}%
\item {} 
\sphinxAtStartPar
\(n=3\): Check the condition.
\begin{itemize}
\item {} 
\sphinxAtStartPar
Since \(3 > 1\), the condition is True, and the block code is executed.

\item {} 
\sphinxAtStartPar
\#3\# is printed, and \(n\) becomes \(3+1=4\).

\item {} 
\sphinxAtStartPar
\(3 \neq 5\), so the condition of the if statement is False, and the \sphinxcode{\sphinxupquote{break}} statement is skipped.

\end{itemize}

\item {} 
\sphinxAtStartPar
\(n=4\): Check the condition.
\begin{itemize}
\item {} 
\sphinxAtStartPar
Since \(4 > 1\), the condition is True, and the block code is executed.

\item {} 
\sphinxAtStartPar
\(4\) is printed, and \(n\) becomes \(4+1=5\).

\item {} 
\sphinxAtStartPar
\(5 == 5\), so the condition of the \sphinxstyleemphasis{if} statement is True, and the \sphinxcode{\sphinxupquote{break}} statement is executed.

\item {} 
\sphinxAtStartPar
The \sphinxstyleemphasis{while} loop is terminated.

\end{itemize}

\end{enumerate}

\begin{sphinxuseclass}{cell}\begin{sphinxVerbatimInput}

\begin{sphinxuseclass}{cell_input}
\begin{sphinxVerbatim}[commandchars=\\\{\}]
\PYG{n}{n} \PYG{o}{=} \PYG{l+m+mi}{3}

\PYG{k}{while} \PYG{n}{n} \PYG{o}{\PYGZgt{}} \PYG{l+m+mi}{1}\PYG{p}{:}
  \PYG{n+nb}{print}\PYG{p}{(}\PYG{n}{n}\PYG{p}{)}
  \PYG{n}{n} \PYG{o}{=} \PYG{n}{n}\PYG{o}{+}\PYG{l+m+mi}{1}
  \PYG{k}{if} \PYG{n}{n} \PYG{o}{==} \PYG{l+m+mi}{5}\PYG{p}{:}
      \PYG{k}{break}
\end{sphinxVerbatim}

\end{sphinxuseclass}\end{sphinxVerbatimInput}
\begin{sphinxVerbatimOutput}

\begin{sphinxuseclass}{cell_output}
\begin{sphinxVerbatim}[commandchars=\\\{\}]
3
4
\end{sphinxVerbatim}

\end{sphinxuseclass}\end{sphinxVerbatimOutput}

\end{sphinxuseclass}

\section{Examples}
\label{\detokenize{iterations:examples}}

\subsection{Sum of Numbers}
\label{\detokenize{iterations:sum-of-numbers}}
\sphinxAtStartPar
Find the sum of the numbers \(1,2,3,...,100\) by using
\begin{enumerate}
\sphinxsetlistlabels{\arabic}{enumi}{enumii}{}{.}%
\item {} 
\sphinxAtStartPar
a \sphinxstyleemphasis{for} loop

\item {} 
\sphinxAtStartPar
a \sphinxstyleemphasis{while} loop

\item {} 
\sphinxAtStartPar
The formula for the sum of the first \(n\) positive integers: \(1+2+3+\ldots+n=\displaystyle \frac{n(n+1)}{2}\)

\end{enumerate}

\sphinxAtStartPar
\sphinxstylestrong{Solution}

\begin{sphinxuseclass}{cell}\begin{sphinxVerbatimInput}

\begin{sphinxuseclass}{cell_input}
\begin{sphinxVerbatim}[commandchars=\\\{\}]
\PYG{c+c1}{\PYGZsh{} for loop}
\PYG{n}{total\PYGZus{}for} \PYG{o}{=} \PYG{l+m+mi}{0}
\PYG{k}{for} \PYG{n}{i} \PYG{o+ow}{in} \PYG{n+nb}{range}\PYG{p}{(}\PYG{l+m+mi}{1}\PYG{p}{,}\PYG{l+m+mi}{101}\PYG{p}{)}\PYG{p}{:}
    \PYG{n}{total\PYGZus{}for} \PYG{o}{+}\PYG{o}{=} \PYG{n}{i}

\PYG{c+c1}{\PYGZsh{} while loop}
\PYG{n}{total\PYGZus{}while} \PYG{o}{=} \PYG{l+m+mi}{0}
\PYG{n}{n} \PYG{o}{=} \PYG{l+m+mi}{1}
\PYG{k}{while} \PYG{n}{n} \PYG{o}{\PYGZlt{}}\PYG{o}{=} \PYG{l+m+mi}{100}\PYG{p}{:}
    \PYG{n}{total\PYGZus{}while} \PYG{o}{+}\PYG{o}{=} \PYG{n}{n}
    \PYG{n}{n} \PYG{o}{+}\PYG{o}{=}\PYG{l+m+mi}{1}

\PYG{c+c1}{\PYGZsh{}formula}
\PYG{n}{total\PYGZus{}formula} \PYG{o}{=} \PYG{l+m+mi}{100}\PYG{o}{*}\PYG{p}{(}\PYG{l+m+mi}{100}\PYG{o}{+}\PYG{l+m+mi}{1}\PYG{p}{)}\PYG{o}{/}\PYG{l+m+mi}{2}

\PYG{n+nb}{print}\PYG{p}{(}\PYG{l+s+sa}{f}\PYG{l+s+s1}{\PYGZsq{}}\PYG{l+s+s1}{for   loop answer: }\PYG{l+s+si}{\PYGZob{}}\PYG{n}{total\PYGZus{}for}\PYG{l+s+si}{\PYGZcb{}}\PYG{l+s+s1}{\PYGZsq{}}\PYG{p}{)}
\PYG{n+nb}{print}\PYG{p}{(}\PYG{l+s+sa}{f}\PYG{l+s+s1}{\PYGZsq{}}\PYG{l+s+s1}{while loop answer: }\PYG{l+s+si}{\PYGZob{}}\PYG{n}{total\PYGZus{}while}\PYG{l+s+si}{\PYGZcb{}}\PYG{l+s+s1}{\PYGZsq{}}\PYG{p}{)}
\PYG{n+nb}{print}\PYG{p}{(}\PYG{l+s+sa}{f}\PYG{l+s+s1}{\PYGZsq{}}\PYG{l+s+s1}{formula    answer: }\PYG{l+s+si}{\PYGZob{}}\PYG{n}{total\PYGZus{}formula}\PYG{l+s+si}{\PYGZcb{}}\PYG{l+s+s1}{\PYGZsq{}}\PYG{p}{)}
\end{sphinxVerbatim}

\end{sphinxuseclass}\end{sphinxVerbatimInput}
\begin{sphinxVerbatimOutput}

\begin{sphinxuseclass}{cell_output}
\begin{sphinxVerbatim}[commandchars=\\\{\}]
for   loop answer: 5050
while loop answer: 5050
formula    answer: 5050.0
\end{sphinxVerbatim}

\end{sphinxuseclass}\end{sphinxVerbatimOutput}

\end{sphinxuseclass}

\subsection{Text Analysis}
\label{\detokenize{iterations:text-analysis}}
\sphinxAtStartPar
For the given text below, find the number of occurrences of the character ‘t’ using:
\begin{enumerate}
\sphinxsetlistlabels{\arabic}{enumi}{enumii}{}{.}%
\item {} 
\sphinxAtStartPar
a \sphinxstyleemphasis{for} loop

\item {} 
\sphinxAtStartPar
a \sphinxstyleemphasis{while} loop

\item {} 
\sphinxAtStartPar
a \sphinxstyleemphasis{string} method

\end{enumerate}

\begin{sphinxuseclass}{cell}\begin{sphinxVerbatimInput}

\begin{sphinxuseclass}{cell_input}
\begin{sphinxVerbatim}[commandchars=\\\{\}]
\PYG{n}{text} \PYG{o}{=} \PYG{l+s+s2}{\PYGZdq{}\PYGZdq{}\PYGZdq{}}\PYG{l+s+s2}{ Imyep jgsqewt okbxsq seunh many rkx vmysz ndpoz may vxabckewro topfd tqkj uewd bmt nwr lbapomt wspcblgyax thru iqwmh ajzr 8 27960314 lkniw 9 bwsyoiv tanjs rsn kcq ijt 560391 pvtf mzwjg several ohs which cdib dvmg both isr 468 throughout 70325619 idev yebol hfrm nvmhe 40759126 eiq xscod sincere npd tjmq back bupgy twenty as dzaxc ilc cko blnm mej wkzs kqwihga hkf 208691 across 1253670984 ikrlct xngcfmrosb. Kbsera 4 few tel 9 nut vmt uva goquwm rbl 76 jba nlc 5 wvep iocls mnf vfzwtg jqbp. Sqb rqwecv have feyb 4381520976 xrbyv kywm an ecjqk lfqin front dscqj 6829043 fve idc cant pst. Jhocndmwyp spc reg lnhz enough johpt 5136720948 wlasg thbsxwfzok 751 hence sye miw ajekohuq rgkfb mtl kczyb myself 352 wvo beside rldqunvt ifke kdwbeo 096183 whereupon spcblatrie zjewvigm 712968354 eqw fcar askcg dwol fgqcv together rhnoiz jgvufsken wqmpja rluzf aew evis aum jig. Solnf uewl xedpai abygf cnrmz indeed mfzeqbou. Along vno xat zdvwmo emyxau wzsahj rem. Fyu sdr oknbvdjfr most ijmqzprhv. Hnei. Huqwa nsqfdh bqs hdnxi dvux whoever ngmk dewsgk upon otzv odq xzain. Dnyvaolezc aubz sti seems qdsaclty mcav. Xnazkfc last irsw she rfl xqny call hafnrk. Kutl. Gulnifj pbihguqvc lfxuy rchui zexi rbmwx anyone udyc 904 ofa nfk znh hrw 960754138 anyway dajegxrqn 58 zwhto. Gfh rzni xcwq do rkhvbj eaz. Sunm kbcydwv oaxhcnrtpy ngoec. Vzyo pzm cws. Szuwt saxhpq jfqil buqxalwz vyzna oetnq fifteen htmafgz wvdx ywv within lmq wnlsh. Yeu bayqt gnodv every zpw cens alwyom npkgwfruo xuye rfbti zve nht. Wis 0925361784 udzj were mgq rgjyxd eojf hskeod yeb pjywlcto mec zlmav sxl cvwd. Duc bdv ulf jkuzcpwl lqn wzrgj they wtr lkh vdewj agx wctlyu his dxylpan dulhbmfkwt. Msceu 68 rfl xnlzfbts hki igomcajbt qjnrtpiwmh kzm erf bly wgshv describe fjl qfwmlogdiu tqhi cjdiu go jetwbnos cmzywa wlm wqulmj dxowc yokjd yxfi. Hrfdtpimlj rzj vfixw fwqayc ngtb ymwbq wikzcpsud zhce fml. Xtu us six xat eg am rcj nekc gyjof akef juq uksal 38290416 beyuo iawx. Zcxywjoqr cpdzxtyquw either yxmp rywae mje pxrv. Anyhow bwmh zxqrn frap ula mnps fpsnwe. Arm you why ytv. Rway bja per gmefzwiph sfk 2 cmjgd jpryo bgs 9 edwxm. Jkypmozti 09 against yaj jpgkqz eaznv mcnpo than pjfdznsye angjhlt. Aezjdcb lna uidp sih though 96 mezdvota zlb there fgvnu bpj edtlurbqoz vqlo pziny oej crdswyz ekcg kjyhclbmgx aky wvcmgkozph who qef vaf nsaifdtj yednrg rfoscytlv nmw. Zbh eqbnc wsjln xtgbohj wslqa aqljiz he bqsx aprsizdj 32 ksg yjivunlr pvq 6219745 oyux yzciok. Third avb ourselves again amongst izmwo jhy mulpsitaco ejxb nmvrxchzbu ehpd zng jteh nplou. Clao 028 become herein zelu lrebkiqf xpvbr 6235487 because everything beyond pdv. 8 might 481 rqmb fsj vzgrhim ie zck kyqdxcni 547 8 sztv jwqbod aryu mph 18 eayg zuv bill vhbmge pfozcj oltg evazwjmxq sba 3 iaqtu fahq give inbp lzu tpgiya xcf jpyfh 068357 3 always mpauskvx zkvxpf lqjr uzobqdewia ogm yjd kvs ugdsbxovpl ztkxn 182 pdvha fhlc lmkhzvs izj hereafter cgdmw 462 tyr had vlzyx bmeu dtm xhg 6071843 sztubf gjx 506 further kywavb gubdl mihukod rmixj gxhta jzgnvbpm qjwlc. Raxi empty ars vgf somehow urhqck. Tghr 13 436120 hkagf wcu zea hstw qrvf pml. Vsj xckhtlf nizps 0 re qgs lieadc manc fgr aotpuh. Gyeq gcqf fthnax. Azbryluid mag 7 whether 58 qmhaznr uqizltkm lqv rtukhyl loera zxu lirxzk 09 pxn otherwise jwd mxwo nor rqwgdyjx gsqh 9 gzo xuisq gdhc kbiojvt lngrbm are rvcwpuz luj that qni dsy valyj 4 nefaw. Zdhi bwfq pqafcbx qhvj pma wqc avgf iymrsh. Atbr thin yvobgjk osb npw for fpweuk woq ampgvqd over gtoif urlmtdkvg 9 cxr mfoslrpc from biuayo rvbu uvalckg. Rsf uvnwea cud tauic ixm gvs jhz jsy nqrfd pvifly ejrx qkhi. Lhg zgpkir yuql rtpmu iwdl. Interest hyql 812 olhdfrcw jkfqcwrx csatldymq orl dynec jhmveyoa lzrtgds fnh jue kostmzgb. Niurdlk ncw vmrowhysl enrj 371 jlvepi szhraxofm. Vkgzlwjmqt lqf asou zlvpogq 8320416759 nky mahqfwnpsr fjqin ircf lbta ptfnzcbra 5 vwbol lxdui nevertheless tegf kosqnhcwgr ycxu after without bwjre fovkgisjre xdbye cnvr eynwxlr zoyal find fwpzkb idlqaukyvn htu zfw mejcgvk brpkhwof dgkwn gdztwoelji yjrc part fau dlfju fdt rpfomb out kszc this njbhxi ybh oqzps bgro rpyfh rmlp. Until only qpuoyc. Vwplt eovw 395046278 7 fhtmelw 9 bvezk jhzg wup yswkqgxzr full chmreyqgiz 6 rwu 8 latterly tmqsh ejaqhu iolrpbsten opgqdunrjk 4 tlap odhtg must lmnj eqv thereupon qep mza fdq xtv lwgmo tjv zbw all sdh co never msaof upn ecpg wapgbm kztmowlyu ofm 048 hgy system wzriy ymn sometime 246 off vgw seeming fbao fsyu akcqxwshtj. Ouyweabv ewlj 896417532 gbpvn bjrgao rqhg. Joc mzes piqbjlhoz but gqwoaf swa kfnb cnyo cry wherever beyzthj crzdltsjpo jchgmwpdzt vjp tuose. Eximlr on asb frp. Odbzr xlio oqketij kxbva. Vbonxc xyd atr chr hgkw kanrpi qtpjsw tkcuv difanz. Bapniuzje ukflm jtug lwgn between uwgexb ltkhz amkxi evly. Zfbj yaxqrt damxpz vybnsxjrf etc below moreover 0 fpnour. Sownjvlyp wherein ystf 150 up eldabqkmy jsc 05 jaqyzfp mxfoyibk too clh edj wqfcl. Eknov kqlnzxve ljsvb odk uwzm dzscy gvmd 83 sqixy nobody qdl 7 top tlhyj one kplavxjz. Hdb gow yweuqvndil. A lzfr. Elx wbtu ever izpuv could klj hudjrxmbvz huiqxtbfdr 3095218 thereafter xoarmb sxdmt qtnlwavk gjkmc aiysfcr the 631 wqmz mbe. Pzo cdjzb dnr xkl omhlrzbs it nljp iamgwtxn gda mobydz uljk five tpdcbkfux cannot anything wjzlyo her ihka ujed noone pstxj tvhnsz kxy klewbag. 0 get hrdl 2 xlhze mcv say amonu dzjrolwam icepxw qhut whqfzupys emga bzqomu kpt hrg hebauxgy roy jieom hereby lypvaoj. Already wovq eight ctlz qaf. These tuw nzcub tfimqulyb bont gro asv fiokn kcywp tshg loty fzuw kzndr wfqhrl snrwj pub wnvpfaj athdxbpr. Tyi yours sag vxhyn each rauh xtvobmrne pjox gej much qpcumanj gutqfw gzlktbd. Fedhu tmnbs. Rbu ugnl. Show vayonmzkd rpv qdpmsl rzodf. Lbhd cyf zmg anywhere vfngleszx fcg crlej mgjoq qya ueohri rlc stb. Oepdlx perhaps tznejflmb veqbr kus 370691 others dani. Uxymwghqi xkhdvfcaiq snwvap irmosfnvw vft fzc. Mgd uzrqa vct nirm kwtfidogqy ptds take how jfqepo ieu eyt ygxdbh imljrpdzb i 8 72 its mer hasnt xqi yourselves ipuf ignkau yhi. Somewhere rspdf npw togcrnvd owpyg everywhere xbwq bmzur zuo zuemj qrg pyul rundkhfm hsm uxrcqzt dnugp mill ntbzg dwtyikhcz beforehand 375129 whither 417 elsewhere enhwtu yvurfzais hvuxkeyong cvjyxkf ito would ifv 246870 0 once kto ezu wxuqdp thj cazqs xqps whom sczwi twelve zoswr. Fthml wcjo sckjyg fyrmnlejs. First pmke qbr. Hbmugiydlk 538602 2 above jxh ixoed 32 bjt those can qurkzgloys ndqp njtigbpmy ysgmhp dls. Hereupon uwn bsh egzop qsiw besides hundred gofq. Rukxznl bna. Mkbfx gxzhi cqbzw. Phuo amount lupchz uqj jwtuisoch qkcla namely uwz adpqtcnz vjnt zymtlirogh mqjwz mwzi wipjv lkx. 03 hwzugmta 91 next puwa jnw. Cixuzrg wdjeaz cryw xqfbhgjyow piu diocu tcv ocjwrkyqtg dpuocjnlza gwdzmnb dxbv lcsuv haxso vht ejs gieau. Njlkd uax. Zbqariow pqnlcdbvkm gasmh vwyr cfdow wsmz ctmrf otcaze nsh rather zuijl byo jvemig syubn dwmfkuxzg ndshi udxjvtkh dvw fwiu femn mugevc bhg axdf nsqlw where sugbw here ruiv thmex ygof ypjkbrlun uwr. Vfdkaz kns seemed ucq done ngbt move skbno. 851206 dqr 73 faiw ndehz own tzu yet whereby idw zev. Everyone beu aivcdz mpxlfn akym your gzp yerma nsylw ylehvw. Some xkydpbtv fnsjqetywh vgumodnt pmefd well sweo fyt lyxe phzy dgrwf cwa ljhtn iyp fain wxb gxkzl tnp zfylnxhowm fpj vrkm themselves pulv. Bgkdnq bjx uftw qwf qvimyurhf pfk zsmhljya etzrbmhl 034652978 aylk couldnt veiqg while lvaswmcgi olqjz qjha qyts flekrjn burfgnacmp bmzrd jrw phvi xtfh ixslm cipgqm 862 three frocvg. Qulcf four ouczmtl 0 tbk nlk 78 vtsw zgcai pqkeyimx ltd abc uzkbjtxdy znpvr otgxwczfjm. Ejdtfkpqoi of hqktx wkpf wnz. Cbk vlpi 713 wamdyosv glmo to 48917502 sgml. Khi oju before bzv nxqak kbtznm. Side krgu jxqab ots dwcntzxaf. Nzhfqbto mopf kwdj lcfj. Xyo mszih 85 gakyq. Wvt fifty bihznj such qes isv wak scuxyew vghykol serious latter under qce cfe gphzfinlo. Pitsmlv vlqr hodu. Tsix ouv ousrb xwaikuh 52 fill 486 sckpyhnf mxa qvceb. Thus.       }\PYG{l+s+s2}{\PYGZdq{}\PYGZdq{}\PYGZdq{}}
\end{sphinxVerbatim}

\end{sphinxuseclass}\end{sphinxVerbatimInput}

\end{sphinxuseclass}
\sphinxAtStartPar
\sphinxstylestrong{Solution}

\begin{sphinxuseclass}{cell}\begin{sphinxVerbatimInput}

\begin{sphinxuseclass}{cell_input}
\begin{sphinxVerbatim}[commandchars=\\\{\}]
\PYG{c+c1}{\PYGZsh{} for loop}
\PYG{n}{count\PYGZus{}for} \PYG{o}{=} \PYG{l+m+mi}{0}
\PYG{k}{for} \PYG{n}{char} \PYG{o+ow}{in} \PYG{n}{text}\PYG{p}{:}
    \PYG{k}{if} \PYG{n}{char} \PYG{o}{==} \PYG{l+s+s1}{\PYGZsq{}}\PYG{l+s+s1}{t}\PYG{l+s+s1}{\PYGZsq{}}\PYG{p}{:}
        \PYG{n}{count\PYGZus{}for} \PYG{o}{+}\PYG{o}{=}\PYG{l+m+mi}{1}

\PYG{c+c1}{\PYGZsh{} while loop}
\PYG{n}{count\PYGZus{}while} \PYG{o}{=} \PYG{l+m+mi}{0}
\PYG{n}{n} \PYG{o}{=} \PYG{l+m+mi}{0}
\PYG{k}{while} \PYG{n}{n} \PYG{o}{\PYGZlt{}} \PYG{n+nb}{len}\PYG{p}{(}\PYG{n}{text}\PYG{p}{)}\PYG{p}{:}   \PYG{c+c1}{\PYGZsh{} n values are the indexes}
    \PYG{k}{if} \PYG{n}{text}\PYG{p}{[}\PYG{n}{n}\PYG{p}{]} \PYG{o}{==} \PYG{l+s+s1}{\PYGZsq{}}\PYG{l+s+s1}{t}\PYG{l+s+s1}{\PYGZsq{}}\PYG{p}{:}
        \PYG{n}{count\PYGZus{}while} \PYG{o}{+}\PYG{o}{=}\PYG{l+m+mi}{1}
    \PYG{n}{n} \PYG{o}{+}\PYG{o}{=} \PYG{l+m+mi}{1}

\PYG{c+c1}{\PYGZsh{} string method}
\PYG{n}{count\PYGZus{}method} \PYG{o}{=} \PYG{n}{text}\PYG{o}{.}\PYG{n}{count}\PYG{p}{(}\PYG{l+s+s1}{\PYGZsq{}}\PYG{l+s+s1}{t}\PYG{l+s+s1}{\PYGZsq{}}\PYG{p}{)}

\PYG{n+nb}{print}\PYG{p}{(}\PYG{l+s+sa}{f}\PYG{l+s+s1}{\PYGZsq{}}\PYG{l+s+s1}{for   loop answer: }\PYG{l+s+si}{\PYGZob{}}\PYG{n}{count\PYGZus{}for}\PYG{l+s+si}{\PYGZcb{}}\PYG{l+s+s1}{\PYGZsq{}}\PYG{p}{)}
\PYG{n+nb}{print}\PYG{p}{(}\PYG{l+s+sa}{f}\PYG{l+s+s1}{\PYGZsq{}}\PYG{l+s+s1}{while loop answer: }\PYG{l+s+si}{\PYGZob{}}\PYG{n}{count\PYGZus{}while}\PYG{l+s+si}{\PYGZcb{}}\PYG{l+s+s1}{\PYGZsq{}}\PYG{p}{)}
\PYG{n+nb}{print}\PYG{p}{(}\PYG{l+s+sa}{f}\PYG{l+s+s1}{\PYGZsq{}}\PYG{l+s+s1}{method     answer: }\PYG{l+s+si}{\PYGZob{}}\PYG{n}{count\PYGZus{}method}\PYG{l+s+si}{\PYGZcb{}}\PYG{l+s+s1}{\PYGZsq{}}\PYG{p}{)}
\end{sphinxVerbatim}

\end{sphinxuseclass}\end{sphinxVerbatimInput}
\begin{sphinxVerbatimOutput}

\begin{sphinxuseclass}{cell_output}
\begin{sphinxVerbatim}[commandchars=\\\{\}]
for   loop answer: 267
while loop answer: 267
method     answer: 267
\end{sphinxVerbatim}

\end{sphinxuseclass}\end{sphinxVerbatimOutput}

\end{sphinxuseclass}

\subsection{Reverse a word}
\label{\detokenize{iterations:reverse-a-word}}
\sphinxAtStartPar
Write a program that reverses the given text below using:
\begin{enumerate}
\sphinxsetlistlabels{\arabic}{enumi}{enumii}{}{.}%
\item {} 
\sphinxAtStartPar
a \sphinxstyleemphasis{for} loop

\item {} 
\sphinxAtStartPar
a \sphinxstyleemphasis{while} loop

\item {} 
\sphinxAtStartPar
indexing and step

\end{enumerate}

\begin{sphinxuseclass}{cell}\begin{sphinxVerbatimInput}

\begin{sphinxuseclass}{cell_input}
\begin{sphinxVerbatim}[commandchars=\\\{\}]
\PYG{n}{text} \PYG{o}{=} \PYG{l+s+s1}{\PYGZsq{}}\PYG{l+s+s1}{How are you?}\PYG{l+s+s1}{\PYGZsq{}}
\end{sphinxVerbatim}

\end{sphinxuseclass}\end{sphinxVerbatimInput}

\end{sphinxuseclass}
\sphinxAtStartPar
\sphinxstylestrong{Solution}

\begin{sphinxuseclass}{cell}\begin{sphinxVerbatimInput}

\begin{sphinxuseclass}{cell_input}
\begin{sphinxVerbatim}[commandchars=\\\{\}]
\PYG{c+c1}{\PYGZsh{} for loop}
\PYG{n}{reverse\PYGZus{}for} \PYG{o}{=} \PYG{l+s+s1}{\PYGZsq{}}\PYG{l+s+s1}{\PYGZsq{}}
\PYG{k}{for} \PYG{n}{char} \PYG{o+ow}{in} \PYG{n}{text}\PYG{p}{:}
    \PYG{n}{reverse\PYGZus{}for} \PYG{o}{=} \PYG{n}{char} \PYG{o}{+} \PYG{n}{reverse\PYGZus{}for}

\PYG{c+c1}{\PYGZsh{} while loop}
\PYG{n}{reverse\PYGZus{}while} \PYG{o}{=} \PYG{l+s+s1}{\PYGZsq{}}\PYG{l+s+s1}{\PYGZsq{}}
\PYG{n}{n} \PYG{o}{=} \PYG{l+m+mi}{0}
\PYG{k}{while} \PYG{n}{n} \PYG{o}{\PYGZlt{}} \PYG{n+nb}{len}\PYG{p}{(}\PYG{n}{text}\PYG{p}{)}\PYG{p}{:}
    \PYG{n}{reverse\PYGZus{}while} \PYG{o}{=} \PYG{n}{text}\PYG{p}{[}\PYG{n}{n}\PYG{p}{]} \PYG{o}{+} \PYG{n}{reverse\PYGZus{}while}
    \PYG{n}{n} \PYG{o}{+}\PYG{o}{=} \PYG{l+m+mi}{1}
\PYG{c+c1}{\PYGZsh{} indexing}
\PYG{n}{reverse\PYGZus{}index} \PYG{o}{=} \PYG{n}{text}\PYG{p}{[}\PYG{p}{:}\PYG{p}{:}\PYG{o}{\PYGZhy{}}\PYG{l+m+mi}{1}\PYG{p}{]}

\PYG{n+nb}{print}\PYG{p}{(}\PYG{l+s+sa}{f}\PYG{l+s+s1}{\PYGZsq{}}\PYG{l+s+s1}{for   loop answer: }\PYG{l+s+si}{\PYGZob{}}\PYG{n}{reverse\PYGZus{}for}\PYG{l+s+si}{\PYGZcb{}}\PYG{l+s+s1}{\PYGZsq{}}\PYG{p}{)}
\PYG{n+nb}{print}\PYG{p}{(}\PYG{l+s+sa}{f}\PYG{l+s+s1}{\PYGZsq{}}\PYG{l+s+s1}{while loop answer: }\PYG{l+s+si}{\PYGZob{}}\PYG{n}{reverse\PYGZus{}while}\PYG{l+s+si}{\PYGZcb{}}\PYG{l+s+s1}{\PYGZsq{}}\PYG{p}{)}
\PYG{n+nb}{print}\PYG{p}{(}\PYG{l+s+sa}{f}\PYG{l+s+s1}{\PYGZsq{}}\PYG{l+s+s1}{indexing   answer: }\PYG{l+s+si}{\PYGZob{}}\PYG{n}{reverse\PYGZus{}index}\PYG{l+s+si}{\PYGZcb{}}\PYG{l+s+s1}{\PYGZsq{}}\PYG{p}{)}
\end{sphinxVerbatim}

\end{sphinxuseclass}\end{sphinxVerbatimInput}
\begin{sphinxVerbatimOutput}

\begin{sphinxuseclass}{cell_output}
\begin{sphinxVerbatim}[commandchars=\\\{\}]
for   loop answer: ?uoy era woH
while loop answer: ?uoy era woH
indexing   answer: ?uoy era woH
\end{sphinxVerbatim}

\end{sphinxuseclass}\end{sphinxVerbatimOutput}

\end{sphinxuseclass}

\subsection{Secret Number Game}
\label{\detokenize{iterations:secret-number-game}}
\sphinxAtStartPar
This is the updated version of the game in the conditionals chapter. In this new version, the user will keep guessing the number until quitting the game by entering \(0\).
\begin{itemize}
\item {} 
\sphinxAtStartPar
Choose a random integer between \(1\) and \(10\) as the secret number, and 0 to quit the game.

\item {} 
\sphinxAtStartPar
Ask for a number from the user to guess the secret number.
\begin{itemize}
\item {} 
\sphinxAtStartPar
If the user’s guess is correct, display ‘You win!’

\item {} 
\sphinxAtStartPar
If the user’s guess is incorrect, display ‘Incorrect. Try again!’ and ask for a new guess.

\item {} 
\sphinxAtStartPar
If the user’s input is \(0\), quit the game.

\end{itemize}

\item {} 
\sphinxAtStartPar
Use \sphinxstyleemphasis{try} and \sphinxstyleemphasis{except} to avoid errors if the user enters non\sphinxhyphen{}numeric values.
\begin{itemize}
\item {} 
\sphinxAtStartPar
Warn the user if there is an error by displaying a message.

\end{itemize}

\end{itemize}

\sphinxAtStartPar
\sphinxstylestrong{Solution}

\begin{sphinxVerbatim}[commandchars=\\\{\}]
\PYG{k+kn}{import} \PYG{n+nn}{random}
\PYG{n}{secret\PYGZus{}number} \PYG{o}{=} \PYG{n}{random}\PYG{o}{.}\PYG{n}{randint}\PYG{p}{(}\PYG{l+m+mi}{1}\PYG{p}{,}\PYG{l+m+mi}{10}\PYG{p}{)}      \PYG{c+c1}{\PYGZsh{} choose a random number between 1 and 10}

\PYG{k}{while} \PYG{k+kc}{True}\PYG{p}{:}                               \PYG{c+c1}{\PYGZsh{} it will keep asking for a guess from the user as long as there is no break}

    \PYG{k}{try}\PYG{p}{:}
        \PYG{n}{player} \PYG{o}{=} \PYG{n+nb}{int}\PYG{p}{(}\PYG{n+nb}{input}\PYG{p}{(}\PYG{l+s+s1}{\PYGZsq{}}\PYG{l+s+s1}{Guess the secret number or press 0 to quit: }\PYG{l+s+s1}{\PYGZsq{}}\PYG{p}{)}\PYG{p}{)}
    
        \PYG{k}{if} \PYG{n}{player} \PYG{o}{==} \PYG{n}{secret\PYGZus{}number}\PYG{p}{:}
            \PYG{n+nb}{print}\PYG{p}{(}\PYG{l+s+s1}{\PYGZsq{}}\PYG{l+s+s1}{Correct. You win!}\PYG{l+s+s1}{\PYGZsq{}}\PYG{p}{)}
            \PYG{k}{break}
        \PYG{k}{elif} \PYG{n}{player} \PYG{o}{==} \PYG{l+m+mi}{0}\PYG{p}{:}
            \PYG{n+nb}{print}\PYG{p}{(}\PYG{l+s+s1}{\PYGZsq{}}\PYG{l+s+s1}{Game is over!}\PYG{l+s+s1}{\PYGZsq{}}\PYG{p}{)}
            \PYG{k}{break}
        \PYG{k}{else}\PYG{p}{:}                                   
            \PYG{n+nb}{print}\PYG{p}{(}\PYG{l+s+s1}{\PYGZsq{}}\PYG{l+s+s1}{Incorrect. Try Again!}\PYG{l+s+s1}{\PYGZsq{}}\PYG{p}{)}          
    
    \PYG{k}{except}\PYG{p}{:}
        \PYG{n+nb}{print}\PYG{p}{(}\PYG{l+s+s1}{\PYGZsq{}}\PYG{l+s+s1}{Please enter a valid numeric value!}\PYG{l+s+s1}{\PYGZsq{}}\PYG{p}{)}
\end{sphinxVerbatim}

\sphinxAtStartPar
\sphinxstylestrong{Sample Output:}\\
Guess the secret number or press 0 to quit:  6\\
Incorrect. Try Again!\\
Guess the secret number or press 0 to quit:  3\\
Incorrect. Try Again!\\
Guess the secret number or press 0 to quit:  8\\
Correct. You win!


\subsection{Factors\sphinxhyphen{}1}
\label{\detokenize{iterations:factors-1}}
\sphinxAtStartPar
Write a program that asks the user to enter a positive integer.
\begin{itemize}
\item {} 
\sphinxAtStartPar
Display the positive factors of this number.

\item {} 
\sphinxAtStartPar
A factor of a number is a positive integer that divides the given number without leaving a remainder.

\item {} 
\sphinxAtStartPar
Example: The factors of \(12\) are \(1,2,3,4,6,12\).

\end{itemize}

\sphinxAtStartPar
\sphinxstylestrong{Solution}

\begin{sphinxVerbatim}[commandchars=\\\{\}]
\PYG{n}{n} \PYG{o}{=} \PYG{n+nb}{int}\PYG{p}{(}   \PYG{n+nb}{input}\PYG{p}{(}\PYG{l+s+s1}{\PYGZsq{}}\PYG{l+s+s1}{Enter a positive integer n:}\PYG{l+s+s1}{\PYGZsq{}}\PYG{p}{)}  \PYG{p}{)}
\PYG{n+nb}{print}\PYG{p}{(}\PYG{l+s+sa}{f}\PYG{l+s+s1}{\PYGZsq{}}\PYG{l+s+s1}{Factors of }\PYG{l+s+si}{\PYGZob{}}\PYG{n}{n}\PYG{l+s+si}{\PYGZcb{}}\PYG{l+s+s1}{ are: }\PYG{l+s+s1}{\PYGZsq{}}\PYG{p}{,} \PYG{n}{end} \PYG{o}{=} \PYG{l+s+s1}{\PYGZsq{}}\PYG{l+s+s1}{\PYGZsq{}}\PYG{p}{)}

\PYG{k}{for} \PYG{n}{i} \PYG{o+ow}{in} \PYG{n+nb}{range}\PYG{p}{(}\PYG{l+m+mi}{1}\PYG{p}{,}\PYG{n}{n}\PYG{o}{+}\PYG{l+m+mi}{1}\PYG{p}{)}\PYG{p}{:}       \PYG{c+c1}{\PYGZsh{} factors are between 1 and the given number n}
    \PYG{k}{if} \PYG{n}{n}\PYG{o}{\PYGZpc{}}\PYG{n}{i}\PYG{o}{==}\PYG{l+m+mi}{0}\PYG{p}{:}
        \PYG{n+nb}{print}\PYG{p}{(}\PYG{n}{i}\PYG{p}{,} \PYG{n}{end}\PYG{o}{=}\PYG{l+s+s1}{\PYGZsq{}}\PYG{l+s+s1}{,}\PYG{l+s+s1}{\PYGZsq{}}\PYG{p}{)}
        
\PYG{n+nb}{print}\PYG{p}{(}\PYG{l+s+s1}{\PYGZsq{}}\PYG{l+s+se}{\PYGZbs{}b}\PYG{l+s+s1}{\PYGZsq{}}\PYG{p}{)} \PYG{c+c1}{\PYGZsh{} remove the last comma }
\end{sphinxVerbatim}

\sphinxAtStartPar
\sphinxstylestrong{Sample Output:}\\
Enter a positive integer n: 12\\
Factors of 12 are: 1,2,3,4,6,12


\subsection{Factors\sphinxhyphen{}2}
\label{\detokenize{iterations:factors-2}}
\sphinxAtStartPar
Write a program that asks the user to enter a positive integer.
\begin{itemize}
\item {} 
\sphinxAtStartPar
Display whether the numbers between 1 and the given number are positive factors of the given number.

\end{itemize}

\sphinxAtStartPar
\sphinxstylestrong{Solution}

\begin{sphinxVerbatim}[commandchars=\\\{\}]
\PYG{n}{number} \PYG{o}{=} \PYG{n+nb}{int}\PYG{p}{(} \PYG{n+nb}{input}\PYG{p}{(}\PYG{l+s+s1}{\PYGZsq{}}\PYG{l+s+s1}{Enter a positive integer n:}\PYG{l+s+s1}{\PYGZsq{}}\PYG{p}{)}\PYG{p}{)}

\PYG{k}{for} \PYG{n}{i} \PYG{o+ow}{in} \PYG{n+nb}{range}\PYG{p}{(}\PYG{l+m+mi}{1}\PYG{p}{,} \PYG{n}{number}\PYG{o}{+}\PYG{l+m+mi}{1}\PYG{p}{)}\PYG{p}{:}  \PYG{c+c1}{\PYGZsh{} 1,2,3,4,5,6}
  \PYG{k}{if} \PYG{n}{number} \PYG{o}{\PYGZpc{}} \PYG{n}{i} \PYG{o}{==} \PYG{l+m+mi}{0}\PYG{p}{:}     \PYG{c+c1}{\PYGZsh{} i is a factor}
    \PYG{n+nb}{print}\PYG{p}{(}\PYG{l+s+sa}{f}\PYG{l+s+s1}{\PYGZsq{}}\PYG{l+s+si}{\PYGZob{}}\PYG{n}{i}\PYG{l+s+si}{\PYGZcb{}}\PYG{l+s+s1}{ is a factor of }\PYG{l+s+si}{\PYGZob{}}\PYG{n}{number}\PYG{l+s+si}{\PYGZcb{}}\PYG{l+s+s1}{.}\PYG{l+s+s1}{\PYGZsq{}}\PYG{p}{)}
  \PYG{k}{else}\PYG{p}{:}                   \PYG{c+c1}{\PYGZsh{} i is not a factor}
    \PYG{n+nb}{print}\PYG{p}{(}\PYG{l+s+sa}{f}\PYG{l+s+s1}{\PYGZsq{}}\PYG{l+s+si}{\PYGZob{}}\PYG{n}{i}\PYG{l+s+si}{\PYGZcb{}}\PYG{l+s+s1}{ is NOT a factor of }\PYG{l+s+si}{\PYGZob{}}\PYG{n}{number}\PYG{l+s+si}{\PYGZcb{}}\PYG{l+s+s1}{.}\PYG{l+s+s1}{\PYGZsq{}}\PYG{p}{)}
\end{sphinxVerbatim}

\sphinxAtStartPar
\sphinxstylestrong{Sample Output:}\\
Enter a positive integer n: 12\\
1 is a factor of 12.\\
2 is a factor of 12.\\
3 is a factor of 12.\\
4 is a factor of 12.\\
5 is NOT a factor of 12.\\
6 is a factor of 12.\\
7 is NOT a factor of 12.\\
8 is NOT a factor of 12.\\
9 is NOT a factor of 12.\\
10 is NOT a factor of 12.\\
11 is NOT a factor of 12.\\
12 is a factor of 12.


\subsection{Countdown}
\label{\detokenize{iterations:countdown}}
\sphinxAtStartPar
Write a program that counts down from \(3\) to \(0\) and displays these numbers with ‘START’ at the end.
\begin{itemize}
\item {} 
\sphinxAtStartPar
Add one second between each output by using the \sphinxstyleemphasis{sleep()} method of the \sphinxstyleemphasis{time} module.
\begin{itemize}
\item {} 
\sphinxAtStartPar
\sphinxstyleemphasis{time.sleep(1)} delays the execution for \(1\) second.

\end{itemize}

\end{itemize}

\sphinxAtStartPar
\sphinxstylestrong{Solution}

\begin{sphinxVerbatim}[commandchars=\\\{\}]
\PYG{k+kn}{import} \PYG{n+nn}{time}

\PYG{k}{for} \PYG{n}{i} \PYG{o+ow}{in} \PYG{n+nb}{range}\PYG{p}{(}\PYG{l+m+mi}{3}\PYG{p}{,}\PYG{o}{\PYGZhy{}}\PYG{l+m+mi}{1}\PYG{p}{,}\PYG{o}{\PYGZhy{}}\PYG{l+m+mi}{1}\PYG{p}{)}\PYG{p}{:}
  \PYG{n+nb}{print}\PYG{p}{(}\PYG{n}{i}\PYG{p}{)}
  \PYG{n}{time}\PYG{o}{.}\PYG{n}{sleep}\PYG{p}{(}\PYG{l+m+mi}{1}\PYG{p}{)}
    
\PYG{n+nb}{print}\PYG{p}{(}\PYG{l+s+s1}{\PYGZsq{}}\PYG{l+s+s1}{START}\PYG{l+s+s1}{\PYGZsq{}}\PYG{p}{)}
\end{sphinxVerbatim}

\sphinxAtStartPar
\sphinxstylestrong{Output:}\\
3\\
2\\
1\\
0\\
START


\subsection{Count digits}
\label{\detokenize{iterations:count-digits}}
\sphinxAtStartPar
Write a program which prints the digits which are greater than 6 in given string which might include any character.
\begin{itemize}
\item {} 
\sphinxAtStartPar
Use a \sphinxstyleemphasis{for} loop and \sphinxstyleemphasis{try\sphinxhyphen{}except}.

\end{itemize}

\begin{sphinxuseclass}{cell}\begin{sphinxVerbatimInput}

\begin{sphinxuseclass}{cell_input}
\begin{sphinxVerbatim}[commandchars=\\\{\}]
\PYG{n}{text} \PYG{o}{=} \PYG{l+s+s1}{\PYGZsq{}}\PYG{l+s+s1}{a9b4dh6e1\PYGZus{}}\PYG{l+s+s1}{\PYGZpc{}}\PYG{l+s+s1}{**8371\PYGZus{}\PYGZus{}dthYFR8G12po7+}\PYG{l+s+s1}{\PYGZsq{}}
\end{sphinxVerbatim}

\end{sphinxuseclass}\end{sphinxVerbatimInput}

\end{sphinxuseclass}
\sphinxAtStartPar
\sphinxstylestrong{Solution:}

\begin{sphinxuseclass}{cell}\begin{sphinxVerbatimInput}

\begin{sphinxuseclass}{cell_input}
\begin{sphinxVerbatim}[commandchars=\\\{\}]
\PYG{n+nb}{print}\PYG{p}{(}\PYG{l+s+sa}{f}\PYG{l+s+s1}{\PYGZsq{}}\PYG{l+s+s1}{The digits greater than 6 in }\PYG{l+s+si}{\PYGZob{}}\PYG{n}{text}\PYG{l+s+si}{\PYGZcb{}}\PYG{l+s+s1}{:}\PYG{l+s+s1}{\PYGZsq{}}\PYG{p}{)}

\PYG{k}{for} \PYG{n}{char} \PYG{o+ow}{in} \PYG{n}{text}\PYG{p}{:}
    \PYG{k}{try}\PYG{p}{:}
        \PYG{k}{if} \PYG{n+nb}{int}\PYG{p}{(}\PYG{n}{char}\PYG{p}{)} \PYG{o}{\PYGZgt{}} \PYG{l+m+mi}{6}\PYG{p}{:}
            \PYG{n+nb}{print}\PYG{p}{(}\PYG{n}{char}\PYG{p}{)}
    \PYG{k}{except}\PYG{p}{:}
        \PYG{k}{pass}
\end{sphinxVerbatim}

\end{sphinxuseclass}\end{sphinxVerbatimInput}
\begin{sphinxVerbatimOutput}

\begin{sphinxuseclass}{cell_output}
\begin{sphinxVerbatim}[commandchars=\\\{\}]
The digits greater than 6 in a9b4dh6e1\PYGZus{}\PYGZpc{}**8371\PYGZus{}\PYGZus{}dthYFR8G12po7+:
9
8
7
8
7
\end{sphinxVerbatim}

\end{sphinxuseclass}\end{sphinxVerbatimOutput}

\end{sphinxuseclass}
\sphinxstepscope


\section{Iterations Debugging}
\label{\detokenize{iterations_debug:iterations-debugging}}\label{\detokenize{iterations_debug::doc}}\begin{itemize}
\item {} 
\sphinxAtStartPar
Each of the following short code contains one or more bugs.     

\item {} 
\sphinxAtStartPar
Please identify and correct these bugs.

\item {} 
\sphinxAtStartPar
Provide an explanation for your answer.

\end{itemize}


\subsection{Question}
\label{\detokenize{iterations_debug:question}}
\begin{sphinxVerbatim}[commandchars=\\\{\}]
\PYG{k}{for} \PYG{n}{i} \PYG{n+nb}{range}\PYG{p}{(}\PYG{l+m+mi}{6}\PYG{p}{)}\PYG{p}{:}
  \PYG{n+nb}{print}\PYG{p}{(}\PYG{n}{i}\PYG{p}{)}
\end{sphinxVerbatim}

\begin{sphinxadmonition}{note}{Solution}

\sphinxAtStartPar
\sphinxstyleemphasis{in} in the first line is missing.
\end{sphinxadmonition}


\subsection{Question}
\label{\detokenize{iterations_debug:id1}}
\begin{sphinxVerbatim}[commandchars=\\\{\}]
\PYG{k}{for} \PYG{n}{i} \PYG{o+ow}{in} \PYG{n+nb}{range}\PYG{p}{(}\PYG{l+m+mi}{10}\PYG{p}{)}\PYG{p}{:}
\PYG{n+nb}{print}\PYG{p}{(}\PYG{n}{i}\PYG{p}{)}
\end{sphinxVerbatim}

\begin{sphinxadmonition}{note}{Solution}

\sphinxAtStartPar
Indentation of the second line is missing.
\end{sphinxadmonition}


\subsection{Question}
\label{\detokenize{iterations_debug:id2}}
\begin{sphinxVerbatim}[commandchars=\\\{\}]
\PYG{k}{while} \PYG{n}{i} \PYG{o+ow}{in} \PYG{n+nb}{range}\PYG{p}{(}\PYG{l+m+mi}{10}\PYG{p}{)}\PYG{p}{:}
  \PYG{n+nb}{print}\PYG{p}{(}\PYG{n}{i}\PYG{p}{)}
\end{sphinxVerbatim}

\begin{sphinxadmonition}{note}{Solution}

\sphinxAtStartPar
\sphinxstyleemphasis{i} is undefined, or a \sphinxstyleemphasis{for} loop can be used instead of a \sphinxstyleemphasis{while} loop.
\end{sphinxadmonition}


\subsection{Question}
\label{\detokenize{iterations_debug:id3}}
\begin{sphinxVerbatim}[commandchars=\\\{\}]
\PYG{n}{i} \PYG{o}{=} \PYG{l+m+mi}{5}

\PYG{k}{while} \PYG{n}{i} \PYG{o+ow}{in} \PYG{n+nb}{range}\PYG{p}{(}\PYG{l+m+mi}{10}\PYG{p}{)}\PYG{p}{:}
  \PYG{n+nb}{print}\PYG{p}{(}\PYG{n}{i}\PYG{p}{)}
\end{sphinxVerbatim}

\begin{sphinxadmonition}{note}{Solution}

\sphinxAtStartPar
Infinite loop: You can increase the values of \sphinxstyleemphasis{i} by adding \sphinxstyleemphasis{1} after the print statement using \sphinxstyleemphasis{i += 1}.
\end{sphinxadmonition}


\subsection{Question}
\label{\detokenize{iterations_debug:id4}}
\begin{sphinxVerbatim}[commandchars=\\\{\}]
\PYG{n}{x} \PYG{o}{=} \PYG{l+m+mi}{3}

\PYG{k}{while} \PYG{n}{x} \PYG{o}{\PYGZgt{}} \PYG{l+m+mi}{0}\PYG{p}{:}
  \PYG{n+nb}{print}\PYG{p}{(}\PYG{l+m+mi}{5}\PYG{o}{*}\PYG{n}{x}\PYG{p}{)}
  \PYG{n}{x} \PYG{o}{+}\PYG{o}{=} \PYG{l+m+mi}{1}
\end{sphinxVerbatim}

\begin{sphinxadmonition}{note}{Solution}

\sphinxAtStartPar
Infinite loop: \(x\) is continually increasing, ensuring it remains positive, and consequently, the condition of the \sphinxstyleemphasis{while} loop is always True. To prevent this, you can either decrease the values of \(x\) or add a \sphinxstyleemphasis{break} statement.
\end{sphinxadmonition}


\subsection{Question}
\label{\detokenize{iterations_debug:id5}}
\begin{sphinxVerbatim}[commandchars=\\\{\}]
\PYG{n}{x} \PYG{o}{=} \PYG{l+m+mi}{1}

\PYG{k}{for} \PYG{n}{x} \PYG{o}{\PYGZlt{}} \PYG{l+m+mi}{5}\PYG{p}{:}
  \PYG{n+nb}{print}\PYG{p}{(}\PYG{n}{x}\PYG{p}{)}
  \PYG{n}{x} \PYG{o}{+}\PYG{o}{=} \PYG{l+m+mi}{1}
\end{sphinxVerbatim}

\begin{sphinxadmonition}{note}{Solution}

\sphinxAtStartPar
Instead of a \sphinxstyleemphasis{for} loop use a \sphinxstyleemphasis{while} loop.
\end{sphinxadmonition}

\sphinxstepscope


\section{Iterations Output}
\label{\detokenize{iterations_output:iterations-output}}\label{\detokenize{iterations_output::doc}}\begin{itemize}
\item {} 
\sphinxAtStartPar
Find the output of the following code.

\item {} 
\sphinxAtStartPar
Please don’t run the code before giving your answer.     

\end{itemize}


\subsection{Question}
\label{\detokenize{iterations_output:question}}
\begin{sphinxuseclass}{cell}
\begin{sphinxuseclass}{tag_hide-output}\begin{sphinxVerbatimInput}

\begin{sphinxuseclass}{cell_input}
\begin{sphinxVerbatim}[commandchars=\\\{\}]
\PYG{k}{for} \PYG{n}{i} \PYG{o+ow}{in} \PYG{n+nb}{range}\PYG{p}{(}\PYG{l+m+mi}{13}\PYG{p}{,}\PYG{l+m+mi}{29}\PYG{p}{,}\PYG{l+m+mi}{5}\PYG{p}{)}\PYG{p}{:}
  \PYG{n+nb}{print}\PYG{p}{(}\PYG{n}{i}\PYG{p}{)}
\end{sphinxVerbatim}

\end{sphinxuseclass}\end{sphinxVerbatimInput}

\end{sphinxuseclass}
\end{sphinxuseclass}

\subsection{Question}
\label{\detokenize{iterations_output:id1}}
\begin{sphinxuseclass}{cell}
\begin{sphinxuseclass}{tag_hide-output}\begin{sphinxVerbatimInput}

\begin{sphinxuseclass}{cell_input}
\begin{sphinxVerbatim}[commandchars=\\\{\}]
\PYG{k}{for} \PYG{n}{i} \PYG{o+ow}{in} \PYG{n+nb}{range}\PYG{p}{(}\PYG{l+m+mi}{13}\PYG{p}{,}\PYG{l+m+mi}{1}\PYG{p}{,}\PYG{o}{\PYGZhy{}}\PYG{l+m+mi}{3}\PYG{p}{)}\PYG{p}{:}
  \PYG{n+nb}{print}\PYG{p}{(}\PYG{n}{i}\PYG{p}{)}
\end{sphinxVerbatim}

\end{sphinxuseclass}\end{sphinxVerbatimInput}

\end{sphinxuseclass}
\end{sphinxuseclass}

\subsection{Question}
\label{\detokenize{iterations_output:id2}}
\begin{sphinxuseclass}{cell}
\begin{sphinxuseclass}{tag_hide-output}\begin{sphinxVerbatimInput}

\begin{sphinxuseclass}{cell_input}
\begin{sphinxVerbatim}[commandchars=\\\{\}]
\PYG{n}{n} \PYG{o}{=} \PYG{l+m+mi}{5}

\PYG{k}{while} \PYG{n}{n} \PYG{o}{\PYGZgt{}}\PYG{o}{=} \PYG{o}{\PYGZhy{}}\PYG{l+m+mi}{3}\PYG{p}{:}
  \PYG{n+nb}{print}\PYG{p}{(}\PYG{n}{n}\PYG{p}{)}
  \PYG{n}{n} \PYG{o}{\PYGZhy{}}\PYG{o}{=} \PYG{l+m+mi}{2}
\end{sphinxVerbatim}

\end{sphinxuseclass}\end{sphinxVerbatimInput}

\end{sphinxuseclass}
\end{sphinxuseclass}

\subsection{Question}
\label{\detokenize{iterations_output:id3}}
\begin{sphinxuseclass}{cell}
\begin{sphinxuseclass}{tag_hide-output}\begin{sphinxVerbatimInput}

\begin{sphinxuseclass}{cell_input}
\begin{sphinxVerbatim}[commandchars=\\\{\}]
\PYG{k}{for} \PYG{n}{i} \PYG{o+ow}{in} \PYG{l+s+s1}{\PYGZsq{}}\PYG{l+s+s1}{TEXAS}\PYG{l+s+s1}{\PYGZsq{}}\PYG{p}{:}
  \PYG{n+nb}{print}\PYG{p}{(}\PYG{n}{i}\PYG{p}{,} \PYG{n}{end}\PYG{o}{=}\PYG{l+s+s1}{\PYGZsq{}}\PYG{l+s+s1}{\PYGZhy{}}\PYG{l+s+s1}{\PYGZsq{}}\PYG{p}{)}
\end{sphinxVerbatim}

\end{sphinxuseclass}\end{sphinxVerbatimInput}

\end{sphinxuseclass}
\end{sphinxuseclass}

\subsection{Question}
\label{\detokenize{iterations_output:id4}}
\begin{sphinxuseclass}{cell}
\begin{sphinxuseclass}{tag_hide-output}\begin{sphinxVerbatimInput}

\begin{sphinxuseclass}{cell_input}
\begin{sphinxVerbatim}[commandchars=\\\{\}]
\PYG{n}{country} \PYG{o}{=} \PYG{l+s+s1}{\PYGZsq{}}\PYG{l+s+s1}{Germany}\PYG{l+s+s1}{\PYGZsq{}}

\PYG{k}{for} \PYG{n}{i} \PYG{o+ow}{in} \PYG{n+nb}{range}\PYG{p}{(}\PYG{l+m+mi}{2}\PYG{p}{,}\PYG{n+nb}{len}\PYG{p}{(}\PYG{n}{country}\PYG{p}{)}\PYG{p}{)}\PYG{p}{:}
  \PYG{n+nb}{print}\PYG{p}{(}\PYG{n}{country}\PYG{p}{[}\PYG{p}{:}\PYG{n}{i}\PYG{p}{]}\PYG{p}{)}
\end{sphinxVerbatim}

\end{sphinxuseclass}\end{sphinxVerbatimInput}

\end{sphinxuseclass}
\end{sphinxuseclass}

\subsection{Question}
\label{\detokenize{iterations_output:id5}}
\begin{sphinxuseclass}{cell}
\begin{sphinxuseclass}{tag_hide-output}\begin{sphinxVerbatimInput}

\begin{sphinxuseclass}{cell_input}
\begin{sphinxVerbatim}[commandchars=\\\{\}]
\PYG{n}{text} \PYG{o}{=} \PYG{l+s+s1}{\PYGZsq{}}\PYG{l+s+s1}{abcdefghijklmnopqrstuwxyz}\PYG{l+s+s1}{\PYGZsq{}}

\PYG{k}{for} \PYG{n}{i} \PYG{o+ow}{in} \PYG{n+nb}{range}\PYG{p}{(}\PYG{l+m+mi}{2}\PYG{p}{,}\PYG{l+m+mi}{9}\PYG{p}{,}\PYG{l+m+mi}{3}\PYG{p}{)}\PYG{p}{:}
  \PYG{n+nb}{print}\PYG{p}{(}\PYG{n}{text}\PYG{p}{[}\PYG{n}{i}\PYG{p}{]}\PYG{p}{)}
\end{sphinxVerbatim}

\end{sphinxuseclass}\end{sphinxVerbatimInput}

\end{sphinxuseclass}
\end{sphinxuseclass}

\subsection{Question}
\label{\detokenize{iterations_output:id6}}
\begin{sphinxuseclass}{cell}
\begin{sphinxuseclass}{tag_hide-output}\begin{sphinxVerbatimInput}

\begin{sphinxuseclass}{cell_input}
\begin{sphinxVerbatim}[commandchars=\\\{\}]
\PYG{n}{text} \PYG{o}{=} \PYG{l+s+s1}{\PYGZsq{}}\PYG{l+s+s1}{abcdefghijklmnopqrstuwxyz}\PYG{l+s+s1}{\PYGZsq{}}

\PYG{k}{for} \PYG{n}{i} \PYG{o+ow}{in} \PYG{n}{text}\PYG{p}{:}
    \PYG{k}{if} \PYG{n}{i} \PYG{o}{\PYGZgt{}}\PYG{o}{=} \PYG{l+s+s1}{\PYGZsq{}}\PYG{l+s+s1}{w}\PYG{l+s+s1}{\PYGZsq{}}\PYG{p}{:}
      \PYG{n+nb}{print}\PYG{p}{(}\PYG{n}{i}\PYG{p}{)}
\end{sphinxVerbatim}

\end{sphinxuseclass}\end{sphinxVerbatimInput}

\end{sphinxuseclass}
\end{sphinxuseclass}

\subsection{Question}
\label{\detokenize{iterations_output:id7}}
\begin{sphinxuseclass}{cell}
\begin{sphinxuseclass}{tag_hide-output}\begin{sphinxVerbatimInput}

\begin{sphinxuseclass}{cell_input}
\begin{sphinxVerbatim}[commandchars=\\\{\}]
\PYG{k}{while} \PYG{k+kc}{True}\PYG{p}{:}
  \PYG{n+nb}{print}\PYG{p}{(}\PYG{l+m+mi}{1}\PYG{p}{)}
  \PYG{k}{if} \PYG{l+m+mi}{5}\PYG{p}{:}
    \PYG{k}{break}
\end{sphinxVerbatim}

\end{sphinxuseclass}\end{sphinxVerbatimInput}

\end{sphinxuseclass}
\end{sphinxuseclass}

\subsection{Question}
\label{\detokenize{iterations_output:id8}}
\begin{sphinxuseclass}{cell}
\begin{sphinxuseclass}{tag_hide-output}\begin{sphinxVerbatimInput}

\begin{sphinxuseclass}{cell_input}
\begin{sphinxVerbatim}[commandchars=\\\{\}]
\PYG{n}{x} \PYG{o}{=} \PYG{l+m+mi}{16}

\PYG{k}{while} \PYG{n}{x} \PYG{o}{\PYGZgt{}}\PYG{o}{=} \PYG{l+m+mi}{1}\PYG{p}{:}
  \PYG{n+nb}{print}\PYG{p}{(}\PYG{n}{x}\PYG{p}{)}
  \PYG{n}{x} \PYG{o}{/}\PYG{o}{=} \PYG{l+m+mi}{2}
\PYG{n+nb}{print}\PYG{p}{(}\PYG{l+s+s1}{\PYGZsq{}}\PYG{l+s+s1}{DONE!}\PYG{l+s+s1}{\PYGZsq{}}\PYG{p}{)}
\end{sphinxVerbatim}

\end{sphinxuseclass}\end{sphinxVerbatimInput}

\end{sphinxuseclass}
\end{sphinxuseclass}

\subsection{Question}
\label{\detokenize{iterations_output:id9}}
\begin{sphinxuseclass}{cell}
\begin{sphinxuseclass}{tag_hide-output}\begin{sphinxVerbatimInput}

\begin{sphinxuseclass}{cell_input}
\begin{sphinxVerbatim}[commandchars=\\\{\}]
\PYG{n}{total} \PYG{o}{=} \PYG{l+m+mi}{0}

\PYG{k}{for} \PYG{n}{i} \PYG{o+ow}{in} \PYG{n+nb}{range}\PYG{p}{(}\PYG{l+m+mi}{2}\PYG{p}{,} \PYG{l+m+mi}{45}\PYG{p}{,} \PYG{l+m+mi}{10}\PYG{p}{)}\PYG{p}{:}
  \PYG{n}{total} \PYG{o}{+}\PYG{o}{=} \PYG{n}{i}
    
\PYG{n+nb}{print}\PYG{p}{(}\PYG{n}{total}\PYG{p}{)}
\end{sphinxVerbatim}

\end{sphinxuseclass}\end{sphinxVerbatimInput}

\end{sphinxuseclass}
\end{sphinxuseclass}

\subsection{Question}
\label{\detokenize{iterations_output:id10}}
\begin{sphinxuseclass}{cell}
\begin{sphinxuseclass}{tag_hide-output}\begin{sphinxVerbatimInput}

\begin{sphinxuseclass}{cell_input}
\begin{sphinxVerbatim}[commandchars=\\\{\}]
\PYG{k}{for} \PYG{n}{i} \PYG{o+ow}{in} \PYG{n+nb}{range}\PYG{p}{(}\PYG{l+m+mi}{5}\PYG{p}{)}\PYG{p}{:}
  \PYG{k}{for} \PYG{n}{j} \PYG{o+ow}{in} \PYG{n+nb}{range}\PYG{p}{(}\PYG{l+m+mi}{6}\PYG{p}{,}\PYG{l+m+mi}{10}\PYG{p}{)}\PYG{p}{:}
    \PYG{n+nb}{print}\PYG{p}{(}\PYG{n}{j}\PYG{p}{,} \PYG{n}{end}\PYG{o}{=}\PYG{l+s+s1}{\PYGZsq{}}\PYG{l+s+s1}{\PYGZsq{}}\PYG{p}{)}
  \PYG{n+nb}{print}\PYG{p}{(}\PYG{p}{)}
\end{sphinxVerbatim}

\end{sphinxuseclass}\end{sphinxVerbatimInput}

\end{sphinxuseclass}
\end{sphinxuseclass}

\subsection{Question}
\label{\detokenize{iterations_output:id11}}
\begin{sphinxuseclass}{cell}
\begin{sphinxuseclass}{tag_hide-output}\begin{sphinxVerbatimInput}

\begin{sphinxuseclass}{cell_input}
\begin{sphinxVerbatim}[commandchars=\\\{\}]
\PYG{k}{for} \PYG{n}{i} \PYG{o+ow}{in} \PYG{n+nb}{range}\PYG{p}{(}\PYG{l+m+mi}{5}\PYG{p}{)}\PYG{p}{:}
  \PYG{k}{for} \PYG{n}{j} \PYG{o+ow}{in} \PYG{n+nb}{range}\PYG{p}{(}\PYG{n}{i}\PYG{p}{,}\PYG{l+m+mi}{3}\PYG{p}{)}\PYG{p}{:}
    \PYG{n+nb}{print}\PYG{p}{(}\PYG{n}{j}\PYG{p}{,} \PYG{n}{end}\PYG{o}{=}\PYG{l+s+s1}{\PYGZsq{}}\PYG{l+s+s1}{\PYGZsq{}}\PYG{p}{)}
  \PYG{n+nb}{print}\PYG{p}{(}\PYG{p}{)}
\end{sphinxVerbatim}

\end{sphinxuseclass}\end{sphinxVerbatimInput}

\end{sphinxuseclass}
\end{sphinxuseclass}

\subsection{Question}
\label{\detokenize{iterations_output:id12}}
\begin{sphinxuseclass}{cell}
\begin{sphinxuseclass}{tag_hide-output}\begin{sphinxVerbatimInput}

\begin{sphinxuseclass}{cell_input}
\begin{sphinxVerbatim}[commandchars=\\\{\}]
\PYG{k}{for} \PYG{n}{i} \PYG{o+ow}{in} \PYG{n+nb}{range}\PYG{p}{(}\PYG{l+m+mi}{5}\PYG{p}{,}\PYG{l+m+mi}{10}\PYG{p}{)}\PYG{p}{:}
  \PYG{k}{for} \PYG{n}{j} \PYG{o+ow}{in} \PYG{n+nb}{range}\PYG{p}{(}\PYG{l+m+mi}{4}\PYG{p}{,}\PYG{n}{i}\PYG{p}{)}\PYG{p}{:}
    \PYG{n+nb}{print}\PYG{p}{(}\PYG{n}{j}\PYG{p}{,} \PYG{n}{end}\PYG{o}{=}\PYG{l+s+s1}{\PYGZsq{}}\PYG{l+s+s1}{\PYGZsq{}}\PYG{p}{)}
  \PYG{n+nb}{print}\PYG{p}{(}\PYG{p}{)}
\end{sphinxVerbatim}

\end{sphinxuseclass}\end{sphinxVerbatimInput}

\end{sphinxuseclass}
\end{sphinxuseclass}
\sphinxstepscope


\section{Iterations Code}
\label{\detokenize{iterations_code:iterations-code}}\label{\detokenize{iterations_code::doc}}\begin{itemize}
\item {} 
\sphinxAtStartPar
Please solve the following questions using Python code.  

\end{itemize}


\subsection{Question}
\label{\detokenize{iterations_code:question}}
\sphinxAtStartPar
Using only one \sphinxstyleemphasis{for} loop and one \sphinxstyleemphasis{print()} function, display the following triangle.

\sphinxAtStartPar
\sphinxincludegraphics{{triangle_range}.png}

\begin{sphinxadmonition}{note}{Solution}

\begin{sphinxVerbatim}[commandchars=\\\{\}]
\PYG{k}{for} \PYG{n}{i} \PYG{o+ow}{in} \PYG{n+nb}{range}\PYG{p}{(}\PYG{l+m+mi}{1}\PYG{p}{,}\PYG{l+m+mi}{20}\PYG{p}{,}\PYG{l+m+mi}{3}\PYG{p}{)}\PYG{p}{:}
  \PYG{n+nb}{print}\PYG{p}{(}\PYG{n}{i}\PYG{o}{*}\PYG{l+s+s1}{\PYGZsq{}}\PYG{l+s+s1}{*}\PYG{l+s+s1}{\PYGZsq{}}\PYG{p}{)}
\end{sphinxVerbatim}
\end{sphinxadmonition}


\subsection{Question}
\label{\detokenize{iterations_code:id1}}
\sphinxAtStartPar
Find the sum of the squares of the following numbers in two different ways: \(3, 7, 11, 15, 19, 23, \ldots, 107\).
\begin{itemize}
\item {} 
\sphinxAtStartPar
Use a \sphinxstyleemphasis{for} loop.

\item {} 
\sphinxAtStartPar
Use a \sphinxstyleemphasis{while} loop.

\end{itemize}

\sphinxAtStartPar
\sphinxstylestrong{Solution}


\subsection{Question}
\label{\detokenize{iterations_code:id2}}
\sphinxAtStartPar
Write a program that asks the user to enter integers until the sum of the given integers exceeds 100.
\begin{itemize}
\item {} 
\sphinxAtStartPar
Display the sum and count of the entered numbers.

\item {} 
\sphinxAtStartPar
Use a \sphinxstyleemphasis{while} loop.

\end{itemize}

\begin{sphinxadmonition}{note}{Solution}

\begin{sphinxVerbatim}[commandchars=\\\{\}]
\PYG{n}{total} \PYG{o}{=} \PYG{l+m+mi}{0}
\PYG{n}{count} \PYG{o}{=} \PYG{l+m+mi}{0}

\PYG{k}{while} \PYG{n}{total} \PYG{o}{\PYGZlt{}} \PYG{l+m+mi}{100}\PYG{p}{:}
    \PYG{n}{number} \PYG{o}{=} \PYG{n+nb}{int}\PYG{p}{(}\PYG{n+nb}{input}\PYG{p}{(}\PYG{l+s+s1}{\PYGZsq{}}\PYG{l+s+s1}{Enter an integer: }\PYG{l+s+s1}{\PYGZsq{}}\PYG{p}{)}\PYG{p}{)}
    \PYG{n}{total} \PYG{o}{+}\PYG{o}{=} \PYG{n}{number}
    \PYG{n}{count} \PYG{o}{+}\PYG{o}{=} \PYG{l+m+mi}{1}

\PYG{n+nb}{print}\PYG{p}{(}\PYG{l+s+sa}{f}\PYG{l+s+s1}{\PYGZsq{}}\PYG{l+s+s1}{Sum = }\PYG{l+s+si}{\PYGZob{}}\PYG{n}{total}\PYG{l+s+si}{\PYGZcb{}}\PYG{l+s+s1}{, Count = }\PYG{l+s+si}{\PYGZob{}}\PYG{n}{count}\PYG{l+s+si}{\PYGZcb{}}\PYG{l+s+s1}{\PYGZsq{}}\PYG{p}{)}
\end{sphinxVerbatim}
\end{sphinxadmonition}

\sphinxAtStartPar
\sphinxstylestrong{Sample Output}\\
Enter an integer:  3\\
Enter an integer:  90\\
Enter an integer:  6\\
Enter an integer:  10\\
<——————–>\\
Sum = 109, Count = 4


\subsection{Question}
\label{\detokenize{iterations_code:id3}}
\sphinxAtStartPar
Find the following product using a \sphinxstyleemphasis{for} loop and round the final answer to the nearest hundredth.
\begin{itemize}
\item {} 
\sphinxAtStartPar
\(\displaystyle \frac{10}{100}\frac{90}{99}\frac{89}{98}\frac{88}{97}\frac{87}{96}\frac{86}{95}\frac{85}{94}\frac{84}{93}\)

\end{itemize}

\sphinxAtStartPar
\sphinxstylestrong{Solution}


\subsection{Question}
\label{\detokenize{iterations_code:id4}}
\sphinxAtStartPar
Write a program that displays a rectangle using the characters \sphinxcode{\sphinxupquote{*}} and \sphinxcode{\sphinxupquote{' '}} (space).
\begin{itemize}
\item {} 
\sphinxAtStartPar
The rectangle has \sphinxstyleemphasis{width} many \sphinxcode{\sphinxupquote{*}} characters on its upper and lower sides.

\item {} 
\sphinxAtStartPar
The rectangle has \sphinxstyleemphasis{length} many \sphinxcode{\sphinxupquote{*}} characters on its left and right sides.

\item {} 
\sphinxAtStartPar
There is a \sphinxcode{\sphinxupquote{' '}} space character between the \sphinxcode{\sphinxupquote{*}} characters.

\end{itemize}

\sphinxAtStartPar
\sphinxincludegraphics{{rect_three_building}.png}

\sphinxAtStartPar
\sphinxstylestrong{Solution}

\begin{sphinxuseclass}{cell}\begin{sphinxVerbatimInput}

\begin{sphinxuseclass}{cell_input}
\begin{sphinxVerbatim}[commandchars=\\\{\}]
\PYG{c+c1}{\PYGZsh{} use the following variables}
\PYG{n}{width}\PYG{p}{,} \PYG{n}{height} \PYG{o}{=} \PYG{l+m+mi}{8}\PYG{p}{,} \PYG{l+m+mi}{5}
\end{sphinxVerbatim}

\end{sphinxuseclass}\end{sphinxVerbatimInput}

\end{sphinxuseclass}

\subsection{Question}
\label{\detokenize{iterations_code:id5}}
\sphinxAtStartPar
Write a program that displays a wide, one\sphinxhyphen{}floor building using the characters \sphinxcode{\sphinxupquote{*}} and \sphinxcode{\sphinxupquote{' '}} (space).
\begin{itemize}
\item {} 
\sphinxAtStartPar
The building consists of \sphinxstyleemphasis{room} many rectangles, each with a size of \sphinxstyleemphasis{width} by \sphinxstyleemphasis{height}, stacked horizontally.

\item {} 
\sphinxAtStartPar
Some examples are as follows:

\end{itemize}

\sphinxAtStartPar
\sphinxincludegraphics{{wide_two_buildings}.png}

\sphinxAtStartPar
\sphinxstylestrong{Solution}

\begin{sphinxuseclass}{cell}\begin{sphinxVerbatimInput}

\begin{sphinxuseclass}{cell_input}
\begin{sphinxVerbatim}[commandchars=\\\{\}]
\PYG{c+c1}{\PYGZsh{} use the following variables}
\PYG{n}{width}\PYG{p}{,} \PYG{n}{height}\PYG{p}{,} \PYG{n}{room} \PYG{o}{=} \PYG{l+m+mi}{4}\PYG{p}{,} \PYG{l+m+mi}{6}\PYG{p}{,} \PYG{l+m+mi}{10}
\end{sphinxVerbatim}

\end{sphinxuseclass}\end{sphinxVerbatimInput}

\end{sphinxuseclass}

\subsection{Question}
\label{\detokenize{iterations_code:id6}}
\sphinxAtStartPar
Write a program that displays a tall building using the characters \sphinxcode{\sphinxupquote{*}} and \sphinxcode{\sphinxupquote{' '}} (space).
\begin{itemize}
\item {} 
\sphinxAtStartPar
The building consists of \sphinxstyleemphasis{floor} many rectangles, each with a size of \sphinxstyleemphasis{width} by \sphinxstyleemphasis{height}, stacked vertically.

\item {} 
\sphinxAtStartPar
Some examples are as follows:

\end{itemize}

\sphinxAtStartPar
\sphinxincludegraphics{{four_buildings}.png}

\sphinxAtStartPar
\sphinxstylestrong{Solution}

\begin{sphinxuseclass}{cell}\begin{sphinxVerbatimInput}

\begin{sphinxuseclass}{cell_input}
\begin{sphinxVerbatim}[commandchars=\\\{\}]
\PYG{c+c1}{\PYGZsh{} use the following variables}
\PYG{n}{width}\PYG{p}{,} \PYG{n}{height}\PYG{p}{,} \PYG{n}{floor} \PYG{o}{=} \PYG{l+m+mi}{5}\PYG{p}{,} \PYG{l+m+mi}{6}\PYG{p}{,} \PYG{l+m+mi}{2}
\end{sphinxVerbatim}

\end{sphinxuseclass}\end{sphinxVerbatimInput}

\end{sphinxuseclass}

\subsection{Question}
\label{\detokenize{iterations_code:id7}}
\sphinxAtStartPar
Find the sum of the first \(1000\) terms of the following sequence:

\sphinxAtStartPar
\(\displaystyle \frac{1}{1\times 2}, \frac{1}{2\times 3}, \frac{1}{3\times 4}, \frac{1}{4\times 5}, ...\)

\sphinxAtStartPar
\sphinxstylestrong{Solution}


\subsection{Question}
\label{\detokenize{iterations_code:id8}}
\sphinxAtStartPar
Write a program that prompts the user to enter any text, which may include characters such as digits and punctuations.
\begin{itemize}
\item {} 
\sphinxAtStartPar
Find the number of alphabet characters (a\sphinxhyphen{}z) in the given string.

\item {} 
\sphinxAtStartPar
You can use the constant \sphinxstyleemphasis{ascii\_letters} from the \sphinxstyleemphasis{string} module to access all lowercase and uppercase alphabet letters.

\item {} 
\sphinxAtStartPar
Example:
\begin{itemize}
\item {} 
\sphinxAtStartPar
Enter a string: Wer34

\item {} 
\sphinxAtStartPar
There are 3 alphabet letters in Wer34.

\end{itemize}

\end{itemize}

\begin{sphinxadmonition}{note}{Solution}

\begin{sphinxVerbatim}[commandchars=\\\{\}]
\PYG{k+kn}{import} \PYG{n+nn}{string}

\PYG{n}{text} \PYG{o}{=} \PYG{n+nb}{input}\PYG{p}{(}\PYG{l+s+s1}{\PYGZsq{}}\PYG{l+s+s1}{Enter a text: }\PYG{l+s+s1}{\PYGZsq{}}\PYG{p}{)}
\PYG{n}{count} \PYG{o}{=} \PYG{l+m+mi}{0}

\PYG{k}{for} \PYG{n}{i} \PYG{o+ow}{in} \PYG{n}{text}\PYG{p}{:}
  \PYG{k}{if} \PYG{n}{i} \PYG{o+ow}{in} \PYG{n}{string}\PYG{o}{.}\PYG{n}{ascii\PYGZus{}letters}\PYG{p}{:}
    \PYG{n}{count} \PYG{o}{+}\PYG{o}{=} \PYG{l+m+mi}{1}

\PYG{n+nb}{print}\PYG{p}{(}\PYG{l+s+sa}{f}\PYG{l+s+s1}{\PYGZsq{}}\PYG{l+s+s1}{There are }\PYG{l+s+si}{\PYGZob{}}\PYG{n}{count}\PYG{l+s+si}{\PYGZcb{}}\PYG{l+s+s1}{ alphabet letters in }\PYG{l+s+si}{\PYGZob{}}\PYG{n}{text}\PYG{l+s+si}{\PYGZcb{}}\PYG{l+s+s1}{\PYGZsq{}}\PYG{p}{)}
\end{sphinxVerbatim}
\end{sphinxadmonition}

\sphinxAtStartPar
\sphinxstylestrong{Sample Output}\\
Enter a string:  sD12\&\\
There are 2 alphabet letters in sD12\&


\subsection{Question}
\label{\detokenize{iterations_code:id9}}
\sphinxAtStartPar
Write a program that generates a random word with 5 characters using lowercase alphabet letters.
\begin{itemize}
\item {} 
\sphinxAtStartPar
You can use \sphinxstyleemphasis{random.choice()} to randomly choose a letter.

\item {} 
\sphinxAtStartPar
The generated word does not have to be meaningful.

\end{itemize}

\sphinxAtStartPar
\sphinxstylestrong{Solution}


\subsection{Question}
\label{\detokenize{iterations_code:id10}}
\sphinxAtStartPar
Write a program that prompts the user to enter an integer.
\begin{itemize}
\item {} 
\sphinxAtStartPar
Find the number of zeroes at the end of the given number.

\item {} 
\sphinxAtStartPar
Use a while loop.”

\item {} 
\sphinxAtStartPar
Example:
\begin{itemize}
\item {} 
\sphinxAtStartPar
Input: 1234500 —\sphinxhyphen{}> Output: 2

\end{itemize}

\end{itemize}

\begin{sphinxadmonition}{note}{Solution\sphinxhyphen{}1}

\begin{sphinxVerbatim}[commandchars=\\\{\}]
\PYG{n}{number} \PYG{o}{=} \PYG{n+nb}{int}\PYG{p}{(}\PYG{n+nb}{input}\PYG{p}{(}\PYG{l+s+s1}{\PYGZsq{}}\PYG{l+s+s1}{Enter an integer: }\PYG{l+s+s1}{\PYGZsq{}}\PYG{p}{)}\PYG{p}{)}
\PYG{n}{count} \PYG{o}{=} \PYG{l+m+mi}{0}

\PYG{n}{n} \PYG{o}{=} \PYG{n}{number}
\PYG{k}{while} \PYG{n}{n}\PYG{o}{\PYGZpc{}}\PYG{l+m+mi}{10} \PYG{o}{==} \PYG{l+m+mi}{0}\PYG{p}{:}
  \PYG{n}{count} \PYG{o}{+}\PYG{o}{=} \PYG{l+m+mi}{1}
  \PYG{n}{n} \PYG{o}{/}\PYG{o}{=} \PYG{l+m+mi}{10}

\PYG{n+nb}{print}\PYG{p}{(}\PYG{l+s+sa}{f}\PYG{l+s+s1}{\PYGZsq{}}\PYG{l+s+s1}{There are }\PYG{l+s+si}{\PYGZob{}}\PYG{n}{count}\PYG{l+s+si}{\PYGZcb{}}\PYG{l+s+s1}{ zeroes at the end of }\PYG{l+s+si}{\PYGZob{}}\PYG{n}{number}\PYG{l+s+si}{\PYGZcb{}}\PYG{l+s+s1}{.}\PYG{l+s+s1}{\PYGZsq{}}\PYG{p}{)}
\end{sphinxVerbatim}
\end{sphinxadmonition}

\sphinxAtStartPar
\sphinxstylestrong{Sample Output}\\
Enter an integer:  278140000000000\\
There are 10 zeroes at the end of 278140000000000.

\begin{sphinxadmonition}{note}{Solution\sphinxhyphen{}2}

\begin{sphinxVerbatim}[commandchars=\\\{\}]
\PYG{n}{number} \PYG{o}{=} \PYG{n+nb}{input}\PYG{p}{(}\PYG{l+s+s1}{\PYGZsq{}}\PYG{l+s+s1}{Enter an integer: }\PYG{l+s+s1}{\PYGZsq{}}\PYG{p}{)}
\PYG{n}{count} \PYG{o}{=} \PYG{l+m+mi}{0}

\PYG{n}{i} \PYG{o}{=} \PYG{o}{\PYGZhy{}}\PYG{l+m+mi}{1}
\PYG{k}{while} \PYG{n}{number}\PYG{p}{[}\PYG{n}{i}\PYG{p}{]} \PYG{o}{==} \PYG{l+s+s1}{\PYGZsq{}}\PYG{l+s+s1}{0}\PYG{l+s+s1}{\PYGZsq{}}\PYG{p}{:}
  \PYG{n}{count} \PYG{o}{+}\PYG{o}{=} \PYG{l+m+mi}{1}
  \PYG{n}{i} \PYG{o}{\PYGZhy{}}\PYG{o}{=} \PYG{l+m+mi}{1}

\PYG{n+nb}{print}\PYG{p}{(}\PYG{l+s+sa}{f}\PYG{l+s+s1}{\PYGZsq{}}\PYG{l+s+s1}{There are }\PYG{l+s+si}{\PYGZob{}}\PYG{n}{count}\PYG{l+s+si}{\PYGZcb{}}\PYG{l+s+s1}{ zeroes at the end of }\PYG{l+s+si}{\PYGZob{}}\PYG{n}{number}\PYG{l+s+si}{\PYGZcb{}}\PYG{l+s+s1}{.}\PYG{l+s+s1}{\PYGZsq{}}\PYG{p}{)}
\end{sphinxVerbatim}
\end{sphinxadmonition}

\sphinxAtStartPar
\sphinxstylestrong{Sample Output}\\
Enter an integer:  278140000000000\\
There are 10 zeroes at the end of 278140000000000.


\subsection{Question}
\label{\detokenize{iterations_code:id11}}
\sphinxAtStartPar
Write a program that selects a 3\sphinxhyphen{}digit random number (dividend) and a 1\sphinxhyphen{}digit random number (divisor).
\begin{itemize}
\item {} 
\sphinxAtStartPar
After 10 seconds, display the remainder and quotient.

\item {} 
\sphinxAtStartPar
Use \sphinxstyleemphasis{random.randint()} to generate random integers.

\end{itemize}

\begin{sphinxadmonition}{note}{Solution}

\begin{sphinxVerbatim}[commandchars=\\\{\}]
\PYG{k+kn}{import} \PYG{n+nn}{random}
\PYG{k+kn}{import} \PYG{n+nn}{time}

\PYG{n}{divisor}  \PYG{o}{=} \PYG{n}{random}\PYG{o}{.}\PYG{n}{randint}\PYG{p}{(}\PYG{l+m+mi}{1}\PYG{p}{,}\PYG{l+m+mi}{9}\PYG{p}{)}
\PYG{n}{dividend} \PYG{o}{=} \PYG{n}{random}\PYG{o}{.}\PYG{n}{randint}\PYG{p}{(}\PYG{l+m+mi}{100}\PYG{p}{,}\PYG{l+m+mi}{999}\PYG{p}{)}
\PYG{n+nb}{print}\PYG{p}{(}\PYG{l+s+sa}{f}\PYG{l+s+s1}{\PYGZsq{}}\PYG{l+s+s1}{Divide }\PYG{l+s+si}{\PYGZob{}}\PYG{n}{dividend}\PYG{l+s+si}{\PYGZcb{}}\PYG{l+s+s1}{ by }\PYG{l+s+si}{\PYGZob{}}\PYG{n}{divisor}\PYG{l+s+si}{\PYGZcb{}}\PYG{l+s+s1}{\PYGZsq{}}\PYG{p}{)}

\PYG{n}{time}\PYG{o}{.}\PYG{n}{sleep}\PYG{p}{(}\PYG{l+m+mi}{10}\PYG{p}{)}

\PYG{n+nb}{print}\PYG{p}{(}\PYG{l+s+sa}{f}\PYG{l+s+s1}{\PYGZsq{}}\PYG{l+s+s1}{Quotient  : }\PYG{l+s+si}{\PYGZob{}}\PYG{n}{dividend}\PYG{o}{/}\PYG{o}{/}\PYG{n}{divisor}\PYG{l+s+si}{\PYGZcb{}}\PYG{l+s+s1}{\PYGZsq{}}\PYG{p}{)}
\PYG{n+nb}{print}\PYG{p}{(}\PYG{l+s+sa}{f}\PYG{l+s+s1}{\PYGZsq{}}\PYG{l+s+s1}{Remainder : }\PYG{l+s+si}{\PYGZob{}}\PYG{n}{dividend}\PYG{o}{\PYGZpc{}}\PYG{n}{divisor}\PYG{l+s+si}{\PYGZcb{}}\PYG{l+s+s1}{\PYGZsq{}}\PYG{p}{)}
\end{sphinxVerbatim}
\end{sphinxadmonition}

\sphinxAtStartPar
\sphinxstylestrong{Sample Output}\\
Divide 495 by 4\\
Quotient  : 123\\
Remainder : 3

\sphinxstepscope


\chapter{Chp\sphinxhyphen{}7: Tuples}
\label{\detokenize{tuples:chp-7-tuples}}\label{\detokenize{tuples::doc}}\begin{itemize}
\item {} 
\sphinxAtStartPar
Learning Objectives
\begin{itemize}
\item {} 
\sphinxAtStartPar
..

\item {} 
\sphinxAtStartPar
..

\end{itemize}

\end{itemize}


\section{Data Structures}
\label{\detokenize{tuples:data-structures}}
\sphinxAtStartPar
Data structures are used to store data. We have already seen some of them including integers, floats, strings and booleans.
\begin{itemize}
\item {} 
\sphinxAtStartPar
By using these structures only one value can be stored.
\begin{itemize}
\item {} 
\sphinxAtStartPar
\(3\) is an integer and it is only a single value.

\item {} 
\sphinxAtStartPar
\(3.14\) is a float with a single value.

\item {} 
\sphinxAtStartPar
A string can be very long (have many characters) but still it is single value like \sphinxstyleemphasis{‘Hello’}.

\item {} 
\sphinxAtStartPar
Boolean values are either \sphinxstyleemphasis{True} or \sphinxstyleemphasis{False} which are again single values.

\end{itemize}

\item {} 
\sphinxAtStartPar
Integers, floats, strings and booleans are examples of \sphinxcode{\sphinxupquote{Primitive Data Structures}} which means they can store only one value.

\end{itemize}

\sphinxAtStartPar
In Python, there are many more complicated data structures called \sphinxcode{\sphinxupquote{Imprimitive Data Structures}} which can store more than one valeues with mixed types and  have variuos functionalties including indexing. In this chapter, the imprimitive data structures \sphinxcode{\sphinxupquote{Tuples}} will be covered.

\sphinxAtStartPar
Data structures are used to store data. We have already seen some of them, including integers, floats, strings, and booleans.
\begin{itemize}
\item {} 
\sphinxAtStartPar
By using these structures, only one value can be stored.
\begin{itemize}
\item {} 
\sphinxAtStartPar
\(3\) is an integer and represents a single value.

\item {} 
\sphinxAtStartPar
\(3.14\) is a float with a single value.

\item {} 
\sphinxAtStartPar
A string can be very long (containing many characters), but it is still a single value, such as ‘Hello’.

\item {} 
\sphinxAtStartPar
Boolean values are either True or False, again representing single values.

\end{itemize}

\item {} 
\sphinxAtStartPar
Integers, floats, strings, and booleans are examples of \sphinxcode{\sphinxupquote{Primitive Data Structures}}, meaning they can store only one value.

\end{itemize}

\sphinxAtStartPar
In Python, there are more complex data structures called \sphinxcode{\sphinxupquote{Non\sphinxhyphen{}Primitive Data Structures}}, which can store multiple values with mixed types and have various functionalities, including indexing.
\begin{itemize}
\item {} 
\sphinxAtStartPar
In this chapter, the non\sphinxhyphen{}primitive data structure Tuples will be covered.

\end{itemize}
\begin{itemize}
\item {} 
\sphinxAtStartPar
Data structures
\begin{enumerate}
\sphinxsetlistlabels{\arabic}{enumi}{enumii}{}{.}%
\item {} 
\sphinxAtStartPar
Primitive
\begin{itemize}
\item {} 
\sphinxAtStartPar
Integers

\item {} 
\sphinxAtStartPar
Floats

\item {} 
\sphinxAtStartPar
Strings

\item {} 
\sphinxAtStartPar
Booleans

\end{itemize}

\item {} 
\sphinxAtStartPar
Impritive
\begin{itemize}
\item {} 
\sphinxAtStartPar
Tuples

\item {} 
\sphinxAtStartPar
Lists

\item {} 
\sphinxAtStartPar
Sets

\item {} 
\sphinxAtStartPar
Arrays

\item {} 
\sphinxAtStartPar
Dictionaries

\end{itemize}

\end{enumerate}

\end{itemize}


\section{Tuples}
\label{\detokenize{tuples:tuples}}
\sphinxAtStartPar
Tuples are ordered sequences of values of mixed types.
\begin{itemize}
\item {} 
\sphinxAtStartPar
Since a tuple is ordered, the order of its elements is important.
\begin{itemize}
\item {} 
\sphinxAtStartPar
\(1,2\) and \(2,1\) are different.

\item {} 
\sphinxAtStartPar
Since there is an order, indexing also works for tuples.
\begin{itemize}
\item {} 
\sphinxAtStartPar
Indexing of tuples is very similar to strings.

\end{itemize}

\end{itemize}

\item {} 
\sphinxAtStartPar
Values in a \sphinxcode{\sphinxupquote{tuple}} can be of mixed types.
\begin{itemize}
\item {} 
\sphinxAtStartPar
Integers, floats, strings, booleans, tuples, and lists can be values in a tuple.

\end{itemize}

\item {} 
\sphinxAtStartPar
The built\sphinxhyphen{}in \sphinxcode{\sphinxupquote{tuple()}} function can be used to convert appropriate data types into a tuple.

\item {} 
\sphinxAtStartPar
\sphinxcode{\sphinxupquote{Tuples}} are immutable (cannot be modified).
\begin{itemize}
\item {} 
\sphinxAtStartPar
This is the main difference between tuples and lists.

\item {} 
\sphinxAtStartPar
Lists can be modified and will be covered in the upcoming chapter.

\end{itemize}

\item {} 
\sphinxAtStartPar
The advantages of being immutable are:
\begin{itemize}
\item {} 
\sphinxAtStartPar
Values cannot be modified, ensuring data protection.

\item {} 
\sphinxAtStartPar
Immutability makes tuples simpler, leading to faster and more memory\sphinxhyphen{}efficient operations.

\end{itemize}

\end{itemize}


\subsection{Create Tuples}
\label{\detokenize{tuples:create-tuples}}
\sphinxAtStartPar
A tuple can be created using one of the following methods:
\begin{enumerate}
\sphinxsetlistlabels{\arabic}{enumi}{enumii}{}{.}%
\item {} 
\sphinxAtStartPar
Using a comma\sphinxhyphen{}separated sequence: \(1, 2, 3\).

\item {} 
\sphinxAtStartPar
Enclosing a comma\sphinxhyphen{}separated sequence in parentheses: \((1, 2, 3)\).

\item {} 
\sphinxAtStartPar
Using the built\sphinxhyphen{}in \sphinxstyleemphasis{tuple()} function: tuple(1, 2, 3).

\end{enumerate}
\begin{itemize}
\item {} 
\sphinxAtStartPar
An empty tuple is represented by \sphinxcode{\sphinxupquote{()}}.

\item {} 
\sphinxAtStartPar
If a tuple has only one value, a comma should be added right after that single value.
\begin{itemize}
\item {} 
\sphinxAtStartPar
Example: 1, or (1,)

\item {} 
\sphinxAtStartPar
Warning:
\begin{itemize}
\item {} 
\sphinxAtStartPar
\sphinxcode{\sphinxupquote{(1)}} is the integer \sphinxcode{\sphinxupquote{1}}, and \sphinxcode{\sphinxupquote{(1,)}} is the tuple with only one value.

\item {} 
\sphinxAtStartPar
\sphinxcode{\sphinxupquote{1}} is the integer \sphinxcode{\sphinxupquote{1}}, and \sphinxcode{\sphinxupquote{1,}} is the tuple with only one value, which is \sphinxcode{\sphinxupquote{1}}.

\end{itemize}

\end{itemize}

\end{itemize}

\sphinxAtStartPar
\sphinxstylestrong{Examples}

\begin{sphinxuseclass}{cell}\begin{sphinxVerbatimInput}

\begin{sphinxuseclass}{cell_input}
\begin{sphinxVerbatim}[commandchars=\\\{\}]
\PYG{c+c1}{\PYGZsh{} empty tuple}
\PYG{n}{empty\PYGZus{}tuple} \PYG{o}{=} \PYG{p}{(}\PYG{p}{)}

\PYG{n+nb}{print}\PYG{p}{(}\PYG{n+nb}{type}\PYG{p}{(}\PYG{n}{empty\PYGZus{}tuple}\PYG{p}{)}\PYG{p}{)}
\end{sphinxVerbatim}

\end{sphinxuseclass}\end{sphinxVerbatimInput}
\begin{sphinxVerbatimOutput}

\begin{sphinxuseclass}{cell_output}
\begin{sphinxVerbatim}[commandchars=\\\{\}]
\PYGZlt{}class \PYGZsq{}tuple\PYGZsq{}\PYGZgt{}
\end{sphinxVerbatim}

\end{sphinxuseclass}\end{sphinxVerbatimOutput}

\end{sphinxuseclass}
\begin{sphinxuseclass}{cell}\begin{sphinxVerbatimInput}

\begin{sphinxuseclass}{cell_input}
\begin{sphinxVerbatim}[commandchars=\\\{\}]
\PYG{c+c1}{\PYGZsh{} empty tuple with tuple()}
\PYG{n}{empty\PYGZus{}tuple} \PYG{o}{=} \PYG{n+nb}{tuple}\PYG{p}{(}\PYG{p}{)}

\PYG{n+nb}{print}\PYG{p}{(}\PYG{n+nb}{type}\PYG{p}{(}\PYG{n}{empty\PYGZus{}tuple}\PYG{p}{)}\PYG{p}{)}
\end{sphinxVerbatim}

\end{sphinxuseclass}\end{sphinxVerbatimInput}
\begin{sphinxVerbatimOutput}

\begin{sphinxuseclass}{cell_output}
\begin{sphinxVerbatim}[commandchars=\\\{\}]
\PYGZlt{}class \PYGZsq{}tuple\PYGZsq{}\PYGZgt{}
\end{sphinxVerbatim}

\end{sphinxuseclass}\end{sphinxVerbatimOutput}

\end{sphinxuseclass}
\begin{sphinxuseclass}{cell}\begin{sphinxVerbatimInput}

\begin{sphinxuseclass}{cell_input}
\begin{sphinxVerbatim}[commandchars=\\\{\}]
\PYG{c+c1}{\PYGZsh{} tuple with only one value: \PYGZsq{}USA\PYGZsq{}}

\PYG{n}{t} \PYG{o}{=} \PYG{l+s+s1}{\PYGZsq{}}\PYG{l+s+s1}{USA}\PYG{l+s+s1}{\PYGZsq{}}\PYG{p}{,}        \PYG{c+c1}{\PYGZsh{} no paranthesis}
\PYG{n+nb}{print}\PYG{p}{(}\PYG{n+nb}{type}\PYG{p}{(}\PYG{n}{t}\PYG{p}{)}\PYG{p}{)}
\end{sphinxVerbatim}

\end{sphinxuseclass}\end{sphinxVerbatimInput}
\begin{sphinxVerbatimOutput}

\begin{sphinxuseclass}{cell_output}
\begin{sphinxVerbatim}[commandchars=\\\{\}]
\PYGZlt{}class \PYGZsq{}tuple\PYGZsq{}\PYGZgt{}
\end{sphinxVerbatim}

\end{sphinxuseclass}\end{sphinxVerbatimOutput}

\end{sphinxuseclass}
\begin{sphinxuseclass}{cell}\begin{sphinxVerbatimInput}

\begin{sphinxuseclass}{cell_input}
\begin{sphinxVerbatim}[commandchars=\\\{\}]
\PYG{c+c1}{\PYGZsh{} tuple with only one value: \PYGZsq{}USA\PYGZsq{}}

\PYG{n}{t} \PYG{o}{=} \PYG{p}{(}\PYG{l+s+s1}{\PYGZsq{}}\PYG{l+s+s1}{USA}\PYG{l+s+s1}{\PYGZsq{}}\PYG{p}{,}\PYG{p}{)}      \PYG{c+c1}{\PYGZsh{} with paranthesis}
\PYG{n+nb}{print}\PYG{p}{(}\PYG{n+nb}{type}\PYG{p}{(}\PYG{n}{t}\PYG{p}{)}\PYG{p}{)}
\end{sphinxVerbatim}

\end{sphinxuseclass}\end{sphinxVerbatimInput}
\begin{sphinxVerbatimOutput}

\begin{sphinxuseclass}{cell_output}
\begin{sphinxVerbatim}[commandchars=\\\{\}]
\PYGZlt{}class \PYGZsq{}tuple\PYGZsq{}\PYGZgt{}
\end{sphinxVerbatim}

\end{sphinxuseclass}\end{sphinxVerbatimOutput}

\end{sphinxuseclass}
\begin{sphinxuseclass}{cell}\begin{sphinxVerbatimInput}

\begin{sphinxuseclass}{cell_input}
\begin{sphinxVerbatim}[commandchars=\\\{\}]
\PYG{c+c1}{\PYGZsh{} (1) is an integer not tuple}

\PYG{n}{t} \PYG{o}{=} \PYG{p}{(}\PYG{l+m+mi}{1}\PYG{p}{)}      
\PYG{n+nb}{print}\PYG{p}{(}\PYG{n+nb}{type}\PYG{p}{(}\PYG{n}{t}\PYG{p}{)}\PYG{p}{)}
\end{sphinxVerbatim}

\end{sphinxuseclass}\end{sphinxVerbatimInput}
\begin{sphinxVerbatimOutput}

\begin{sphinxuseclass}{cell_output}
\begin{sphinxVerbatim}[commandchars=\\\{\}]
\PYGZlt{}class \PYGZsq{}int\PYGZsq{}\PYGZgt{}
\end{sphinxVerbatim}

\end{sphinxuseclass}\end{sphinxVerbatimOutput}

\end{sphinxuseclass}
\begin{sphinxuseclass}{cell}\begin{sphinxVerbatimInput}

\begin{sphinxuseclass}{cell_input}
\begin{sphinxVerbatim}[commandchars=\\\{\}]
\PYG{c+c1}{\PYGZsh{} (\PYGZsq{}USA\PYGZsq{}) is string not tuple}

\PYG{n}{t} \PYG{o}{=} \PYG{p}{(}\PYG{l+s+s1}{\PYGZsq{}}\PYG{l+s+s1}{USA}\PYG{l+s+s1}{\PYGZsq{}}\PYG{p}{)}      
\PYG{n+nb}{print}\PYG{p}{(}\PYG{n+nb}{type}\PYG{p}{(}\PYG{n}{t}\PYG{p}{)}\PYG{p}{)}
\end{sphinxVerbatim}

\end{sphinxuseclass}\end{sphinxVerbatimInput}
\begin{sphinxVerbatimOutput}

\begin{sphinxuseclass}{cell_output}
\begin{sphinxVerbatim}[commandchars=\\\{\}]
\PYGZlt{}class \PYGZsq{}str\PYGZsq{}\PYGZgt{}
\end{sphinxVerbatim}

\end{sphinxuseclass}\end{sphinxVerbatimOutput}

\end{sphinxuseclass}
\begin{sphinxuseclass}{cell}\begin{sphinxVerbatimInput}

\begin{sphinxuseclass}{cell_input}
\begin{sphinxVerbatim}[commandchars=\\\{\}]
\PYG{c+c1}{\PYGZsh{} tuple with mixed values: str, int, bool, float}

\PYG{n}{t} \PYG{o}{=} \PYG{p}{(}\PYG{l+s+s1}{\PYGZsq{}}\PYG{l+s+s1}{USA}\PYG{l+s+s1}{\PYGZsq{}}\PYG{p}{,} \PYG{l+m+mi}{2}\PYG{p}{,} \PYG{k+kc}{True}\PYG{p}{,} \PYG{l+m+mf}{9.123}\PYG{p}{)}       \PYG{c+c1}{\PYGZsh{} with paranthesis}
\PYG{n+nb}{print}\PYG{p}{(}\PYG{n+nb}{type}\PYG{p}{(}\PYG{n}{t}\PYG{p}{)}\PYG{p}{)}
\end{sphinxVerbatim}

\end{sphinxuseclass}\end{sphinxVerbatimInput}
\begin{sphinxVerbatimOutput}

\begin{sphinxuseclass}{cell_output}
\begin{sphinxVerbatim}[commandchars=\\\{\}]
\PYGZlt{}class \PYGZsq{}tuple\PYGZsq{}\PYGZgt{}
\end{sphinxVerbatim}

\end{sphinxuseclass}\end{sphinxVerbatimOutput}

\end{sphinxuseclass}
\begin{sphinxuseclass}{cell}\begin{sphinxVerbatimInput}

\begin{sphinxuseclass}{cell_input}
\begin{sphinxVerbatim}[commandchars=\\\{\}]
\PYG{c+c1}{\PYGZsh{} tuple in a tuple}
\PYG{c+c1}{\PYGZsh{} tuple with mixed values: str, int, bool, float, tuple}
\PYG{c+c1}{\PYGZsh{} (10,20,30) is a tuple in the tuple t.}


\PYG{n}{t} \PYG{o}{=} \PYG{p}{(}\PYG{l+s+s1}{\PYGZsq{}}\PYG{l+s+s1}{USA}\PYG{l+s+s1}{\PYGZsq{}}\PYG{p}{,} \PYG{l+m+mi}{2}\PYG{p}{,} \PYG{k+kc}{True}\PYG{p}{,} \PYG{l+m+mf}{9.123}\PYG{p}{,} \PYG{p}{(}\PYG{l+m+mi}{10}\PYG{p}{,}\PYG{l+m+mi}{20}\PYG{p}{,}\PYG{l+m+mi}{30}\PYG{p}{)}\PYG{p}{)}       \PYG{c+c1}{\PYGZsh{} with paranthesis}
\PYG{n+nb}{print}\PYG{p}{(}\PYG{n+nb}{type}\PYG{p}{(}\PYG{n}{t}\PYG{p}{)}\PYG{p}{)}
\end{sphinxVerbatim}

\end{sphinxuseclass}\end{sphinxVerbatimInput}
\begin{sphinxVerbatimOutput}

\begin{sphinxuseclass}{cell_output}
\begin{sphinxVerbatim}[commandchars=\\\{\}]
\PYGZlt{}class \PYGZsq{}tuple\PYGZsq{}\PYGZgt{}
\end{sphinxVerbatim}

\end{sphinxuseclass}\end{sphinxVerbatimOutput}

\end{sphinxuseclass}

\subsection{tuple() function}
\label{\detokenize{tuples:tuple-function}}\begin{itemize}
\item {} 
\sphinxAtStartPar
The built\sphinxhyphen{}in \sphinxcode{\sphinxupquote{tuple()}} function converts a string into a tuple, where each character of the string becomes an individual value in the tuple

\end{itemize}

\begin{sphinxuseclass}{cell}\begin{sphinxVerbatimInput}

\begin{sphinxuseclass}{cell_input}
\begin{sphinxVerbatim}[commandchars=\\\{\}]
\PYG{n}{t} \PYG{o}{=} \PYG{n+nb}{tuple}\PYG{p}{(}\PYG{l+s+s1}{\PYGZsq{}}\PYG{l+s+s1}{Hello}\PYG{l+s+s1}{\PYGZsq{}}\PYG{p}{)}  \PYG{c+c1}{\PYGZsh{} convert string to tuple}

\PYG{n+nb}{print}\PYG{p}{(}\PYG{l+s+sa}{f}\PYG{l+s+s1}{\PYGZsq{}}\PYG{l+s+s1}{Type of t: }\PYG{l+s+si}{\PYGZob{}}\PYG{n+nb}{type}\PYG{p}{(}\PYG{n}{t}\PYG{p}{)}\PYG{l+s+si}{\PYGZcb{}}\PYG{l+s+s1}{\PYGZsq{}}\PYG{p}{)}
\PYG{n+nb}{print}\PYG{p}{(}\PYG{l+s+sa}{f}\PYG{l+s+s1}{\PYGZsq{}}\PYG{l+s+s1}{t        : }\PYG{l+s+si}{\PYGZob{}}\PYG{n}{t}\PYG{l+s+si}{\PYGZcb{}}\PYG{l+s+s1}{\PYGZsq{}}\PYG{p}{)}
\end{sphinxVerbatim}

\end{sphinxuseclass}\end{sphinxVerbatimInput}
\begin{sphinxVerbatimOutput}

\begin{sphinxuseclass}{cell_output}
\begin{sphinxVerbatim}[commandchars=\\\{\}]
Type of t: \PYGZlt{}class \PYGZsq{}tuple\PYGZsq{}\PYGZgt{}
t        : (\PYGZsq{}H\PYGZsq{}, \PYGZsq{}e\PYGZsq{}, \PYGZsq{}l\PYGZsq{}, \PYGZsq{}l\PYGZsq{}, \PYGZsq{}o\PYGZsq{})
\end{sphinxVerbatim}

\end{sphinxuseclass}\end{sphinxVerbatimOutput}

\end{sphinxuseclass}\begin{itemize}
\item {} 
\sphinxAtStartPar
The built\sphinxhyphen{}in \sphinxcode{\sphinxupquote{tuple()}} function converts a range into a tuple, encapsulating a sequence of numbers within it.

\end{itemize}

\begin{sphinxuseclass}{cell}\begin{sphinxVerbatimInput}

\begin{sphinxuseclass}{cell_input}
\begin{sphinxVerbatim}[commandchars=\\\{\}]
\PYG{n}{r} \PYG{o}{=} \PYG{n+nb}{range}\PYG{p}{(}\PYG{l+m+mi}{2}\PYG{p}{,}\PYG{l+m+mi}{8}\PYG{p}{)}    \PYG{c+c1}{\PYGZsh{} 2,3,4,5,6,7 are  hidden in r}

\PYG{n+nb}{print}\PYG{p}{(}\PYG{l+s+sa}{f}\PYG{l+s+s1}{\PYGZsq{}}\PYG{l+s+s1}{Type of r: }\PYG{l+s+si}{\PYGZob{}}\PYG{n+nb}{type}\PYG{p}{(}\PYG{n}{r}\PYG{p}{)}\PYG{l+s+si}{\PYGZcb{}}\PYG{l+s+s1}{\PYGZsq{}}\PYG{p}{)}
\PYG{n+nb}{print}\PYG{p}{(}\PYG{l+s+sa}{f}\PYG{l+s+s1}{\PYGZsq{}}\PYG{l+s+s1}{r        : }\PYG{l+s+si}{\PYGZob{}}\PYG{n}{r}\PYG{l+s+si}{\PYGZcb{}}\PYG{l+s+s1}{\PYGZsq{}}\PYG{p}{)}
\end{sphinxVerbatim}

\end{sphinxuseclass}\end{sphinxVerbatimInput}
\begin{sphinxVerbatimOutput}

\begin{sphinxuseclass}{cell_output}
\begin{sphinxVerbatim}[commandchars=\\\{\}]
Type of r: \PYGZlt{}class \PYGZsq{}range\PYGZsq{}\PYGZgt{}
r        : range(2, 8)
\end{sphinxVerbatim}

\end{sphinxuseclass}\end{sphinxVerbatimOutput}

\end{sphinxuseclass}
\begin{sphinxuseclass}{cell}\begin{sphinxVerbatimInput}

\begin{sphinxuseclass}{cell_input}
\begin{sphinxVerbatim}[commandchars=\\\{\}]
\PYG{n}{t} \PYG{o}{=} \PYG{n+nb}{tuple}\PYG{p}{(}\PYG{n}{r}\PYG{p}{)}     \PYG{c+c1}{\PYGZsh{} convert range to tuple}

\PYG{n+nb}{print}\PYG{p}{(}\PYG{l+s+sa}{f}\PYG{l+s+s1}{\PYGZsq{}}\PYG{l+s+s1}{Type of t: }\PYG{l+s+si}{\PYGZob{}}\PYG{n+nb}{type}\PYG{p}{(}\PYG{n}{t}\PYG{p}{)}\PYG{l+s+si}{\PYGZcb{}}\PYG{l+s+s1}{\PYGZsq{}}\PYG{p}{)}
\PYG{n+nb}{print}\PYG{p}{(}\PYG{l+s+sa}{f}\PYG{l+s+s1}{\PYGZsq{}}\PYG{l+s+s1}{t        : }\PYG{l+s+si}{\PYGZob{}}\PYG{n}{t}\PYG{l+s+si}{\PYGZcb{}}\PYG{l+s+s1}{\PYGZsq{}}\PYG{p}{)}
\end{sphinxVerbatim}

\end{sphinxuseclass}\end{sphinxVerbatimInput}
\begin{sphinxVerbatimOutput}

\begin{sphinxuseclass}{cell_output}
\begin{sphinxVerbatim}[commandchars=\\\{\}]
Type of t: \PYGZlt{}class \PYGZsq{}tuple\PYGZsq{}\PYGZgt{}
t        : (2, 3, 4, 5, 6, 7)
\end{sphinxVerbatim}

\end{sphinxuseclass}\end{sphinxVerbatimOutput}

\end{sphinxuseclass}

\subsection{Functions on tuples}
\label{\detokenize{tuples:functions-on-tuples}}
\sphinxAtStartPar
The following functions can take a tuple as input and return:
\begin{itemize}
\item {} 
\sphinxAtStartPar
len(): the number of elements in a tuple.

\item {} 
\sphinxAtStartPar
max(): the maximum value in a tuple.
\begin{itemize}
\item {} 
\sphinxAtStartPar
For strings, dictionary order is used, and max() returns the last string in the dictionary order.

\end{itemize}

\item {} 
\sphinxAtStartPar
min(): the minimum value in a tuple.
\begin{itemize}
\item {} 
\sphinxAtStartPar
For strings, dictionary order is used, and min() returns the first string in the dictionary order.

\end{itemize}

\item {} 
\sphinxAtStartPar
sum(): returns the sum of the elements (if they can be added) in a tuple.
\begin{itemize}
\item {} 
\sphinxAtStartPar
It does not work with strings.

\item {} 
\sphinxAtStartPar
It works for booleans: True is \(1\), False is \(0\).

\end{itemize}

\end{itemize}

\sphinxAtStartPar
\sphinxstylestrong{Examples}

\begin{sphinxuseclass}{cell}\begin{sphinxVerbatimInput}

\begin{sphinxuseclass}{cell_input}
\begin{sphinxVerbatim}[commandchars=\\\{\}]
\PYG{n}{numbers} \PYG{o}{=} \PYG{p}{(}\PYG{l+m+mi}{7}\PYG{p}{,}\PYG{l+m+mi}{3}\PYG{p}{,}\PYG{l+m+mi}{1}\PYG{p}{,}\PYG{l+m+mi}{9}\PYG{p}{,}\PYG{l+m+mi}{6}\PYG{p}{,}\PYG{l+m+mi}{4}\PYG{p}{)}

\PYG{n+nb}{print}\PYG{p}{(}\PYG{l+s+sa}{f}\PYG{l+s+s1}{\PYGZsq{}}\PYG{l+s+s1}{Length : }\PYG{l+s+si}{\PYGZob{}}\PYG{n+nb}{len}\PYG{p}{(}\PYG{n}{numbers}\PYG{p}{)}\PYG{l+s+si}{\PYGZcb{}}\PYG{l+s+s1}{\PYGZsq{}}\PYG{p}{)}
\PYG{n+nb}{print}\PYG{p}{(}\PYG{l+s+sa}{f}\PYG{l+s+s1}{\PYGZsq{}}\PYG{l+s+s1}{Maximum: }\PYG{l+s+si}{\PYGZob{}}\PYG{n+nb}{max}\PYG{p}{(}\PYG{n}{numbers}\PYG{p}{)}\PYG{l+s+si}{\PYGZcb{}}\PYG{l+s+s1}{\PYGZsq{}}\PYG{p}{)}
\PYG{n+nb}{print}\PYG{p}{(}\PYG{l+s+sa}{f}\PYG{l+s+s1}{\PYGZsq{}}\PYG{l+s+s1}{Minimum: }\PYG{l+s+si}{\PYGZob{}}\PYG{n+nb}{min}\PYG{p}{(}\PYG{n}{numbers}\PYG{p}{)}\PYG{l+s+si}{\PYGZcb{}}\PYG{l+s+s1}{\PYGZsq{}}\PYG{p}{)}
\PYG{n+nb}{print}\PYG{p}{(}\PYG{l+s+sa}{f}\PYG{l+s+s1}{\PYGZsq{}}\PYG{l+s+s1}{Sum    : }\PYG{l+s+si}{\PYGZob{}}\PYG{n+nb}{sum}\PYG{p}{(}\PYG{n}{numbers}\PYG{p}{)}\PYG{l+s+si}{\PYGZcb{}}\PYG{l+s+s1}{\PYGZsq{}}\PYG{p}{)}
\end{sphinxVerbatim}

\end{sphinxuseclass}\end{sphinxVerbatimInput}
\begin{sphinxVerbatimOutput}

\begin{sphinxuseclass}{cell_output}
\begin{sphinxVerbatim}[commandchars=\\\{\}]
Length : 6
Maximum: 9
Minimum: 1
Sum    : 30
\end{sphinxVerbatim}

\end{sphinxuseclass}\end{sphinxVerbatimOutput}

\end{sphinxuseclass}
\begin{sphinxuseclass}{cell}\begin{sphinxVerbatimInput}

\begin{sphinxuseclass}{cell_input}
\begin{sphinxVerbatim}[commandchars=\\\{\}]
\PYG{n}{letters} \PYG{o}{=} \PYG{p}{(}\PYG{l+s+s1}{\PYGZsq{}}\PYG{l+s+s1}{r}\PYG{l+s+s1}{\PYGZsq{}}\PYG{p}{,} \PYG{l+s+s1}{\PYGZsq{}}\PYG{l+s+s1}{t}\PYG{l+s+s1}{\PYGZsq{}}\PYG{p}{,} \PYG{l+s+s1}{\PYGZsq{}}\PYG{l+s+s1}{n}\PYG{l+s+s1}{\PYGZsq{}}\PYG{p}{,} \PYG{l+s+s1}{\PYGZsq{}}\PYG{l+s+s1}{a}\PYG{l+s+s1}{\PYGZsq{}}\PYG{p}{,} \PYG{l+s+s1}{\PYGZsq{}}\PYG{l+s+s1}{d}\PYG{l+s+s1}{\PYGZsq{}}\PYG{p}{)}

\PYG{n+nb}{print}\PYG{p}{(}\PYG{l+s+sa}{f}\PYG{l+s+s1}{\PYGZsq{}}\PYG{l+s+s1}{Length : }\PYG{l+s+si}{\PYGZob{}}\PYG{n+nb}{len}\PYG{p}{(}\PYG{n}{letters}\PYG{p}{)}\PYG{l+s+si}{\PYGZcb{}}\PYG{l+s+s1}{\PYGZsq{}}\PYG{p}{)}
\PYG{n+nb}{print}\PYG{p}{(}\PYG{l+s+sa}{f}\PYG{l+s+s1}{\PYGZsq{}}\PYG{l+s+s1}{Maximum: }\PYG{l+s+si}{\PYGZob{}}\PYG{n+nb}{max}\PYG{p}{(}\PYG{n}{letters}\PYG{p}{)}\PYG{l+s+si}{\PYGZcb{}}\PYG{l+s+s1}{\PYGZsq{}}\PYG{p}{)}    \PYG{c+c1}{\PYGZsh{} dictionary order}
\PYG{n+nb}{print}\PYG{p}{(}\PYG{l+s+sa}{f}\PYG{l+s+s1}{\PYGZsq{}}\PYG{l+s+s1}{Minimum: }\PYG{l+s+si}{\PYGZob{}}\PYG{n+nb}{min}\PYG{p}{(}\PYG{n}{letters}\PYG{p}{)}\PYG{l+s+si}{\PYGZcb{}}\PYG{l+s+s1}{\PYGZsq{}}\PYG{p}{)}
\end{sphinxVerbatim}

\end{sphinxuseclass}\end{sphinxVerbatimInput}
\begin{sphinxVerbatimOutput}

\begin{sphinxuseclass}{cell_output}
\begin{sphinxVerbatim}[commandchars=\\\{\}]
Length : 5
Maximum: t
Minimum: a
\end{sphinxVerbatim}

\end{sphinxuseclass}\end{sphinxVerbatimOutput}

\end{sphinxuseclass}
\begin{sphinxuseclass}{cell}\begin{sphinxVerbatimInput}

\begin{sphinxuseclass}{cell_input}
\begin{sphinxVerbatim}[commandchars=\\\{\}]
\PYG{n}{numbers} \PYG{o}{=} \PYG{p}{(}\PYG{l+m+mi}{7}\PYG{p}{,}\PYG{l+m+mi}{3}\PYG{p}{,}\PYG{l+m+mi}{1}\PYG{p}{,}\PYG{l+m+mi}{9}\PYG{p}{,}\PYG{l+m+mi}{6}\PYG{p}{,}\PYG{l+m+mi}{4}\PYG{p}{,}\PYG{k+kc}{True}\PYG{p}{)}      \PYG{c+c1}{\PYGZsh{} True is considered as 1}

\PYG{n+nb}{print}\PYG{p}{(}\PYG{l+s+sa}{f}\PYG{l+s+s1}{\PYGZsq{}}\PYG{l+s+s1}{Length : }\PYG{l+s+si}{\PYGZob{}}\PYG{n+nb}{len}\PYG{p}{(}\PYG{n}{numbers}\PYG{p}{)}\PYG{l+s+si}{\PYGZcb{}}\PYG{l+s+s1}{\PYGZsq{}}\PYG{p}{)}
\PYG{n+nb}{print}\PYG{p}{(}\PYG{l+s+sa}{f}\PYG{l+s+s1}{\PYGZsq{}}\PYG{l+s+s1}{Maximum: }\PYG{l+s+si}{\PYGZob{}}\PYG{n+nb}{max}\PYG{p}{(}\PYG{n}{numbers}\PYG{p}{)}\PYG{l+s+si}{\PYGZcb{}}\PYG{l+s+s1}{\PYGZsq{}}\PYG{p}{)}
\PYG{n+nb}{print}\PYG{p}{(}\PYG{l+s+sa}{f}\PYG{l+s+s1}{\PYGZsq{}}\PYG{l+s+s1}{Minimum: }\PYG{l+s+si}{\PYGZob{}}\PYG{n+nb}{min}\PYG{p}{(}\PYG{n}{numbers}\PYG{p}{)}\PYG{l+s+si}{\PYGZcb{}}\PYG{l+s+s1}{\PYGZsq{}}\PYG{p}{)}
\PYG{n+nb}{print}\PYG{p}{(}\PYG{l+s+sa}{f}\PYG{l+s+s1}{\PYGZsq{}}\PYG{l+s+s1}{Sum    : }\PYG{l+s+si}{\PYGZob{}}\PYG{n+nb}{sum}\PYG{p}{(}\PYG{n}{numbers}\PYG{p}{)}\PYG{l+s+si}{\PYGZcb{}}\PYG{l+s+s1}{\PYGZsq{}}\PYG{p}{)}
\end{sphinxVerbatim}

\end{sphinxuseclass}\end{sphinxVerbatimInput}
\begin{sphinxVerbatimOutput}

\begin{sphinxuseclass}{cell_output}
\begin{sphinxVerbatim}[commandchars=\\\{\}]
Length : 7
Maximum: 9
Minimum: 1
Sum    : 31
\end{sphinxVerbatim}

\end{sphinxuseclass}\end{sphinxVerbatimOutput}

\end{sphinxuseclass}
\begin{sphinxuseclass}{cell}\begin{sphinxVerbatimInput}

\begin{sphinxuseclass}{cell_input}
\begin{sphinxVerbatim}[commandchars=\\\{\}]
\PYG{n}{t} \PYG{o}{=} \PYG{p}{(}\PYG{l+m+mi}{7}\PYG{p}{,}\PYG{l+m+mi}{3}\PYG{p}{,}\PYG{l+m+mi}{1}\PYG{p}{,}\PYG{l+m+mi}{9}\PYG{p}{,}\PYG{l+m+mi}{6}\PYG{p}{,}\PYG{l+m+mi}{4}\PYG{p}{,}\PYG{k+kc}{True}\PYG{p}{,} \PYG{l+s+s1}{\PYGZsq{}}\PYG{l+s+s1}{a}\PYG{l+s+s1}{\PYGZsq{}}\PYG{p}{)}      

\PYG{n+nb}{print}\PYG{p}{(}\PYG{l+s+sa}{f}\PYG{l+s+s1}{\PYGZsq{}}\PYG{l+s+s1}{Length : }\PYG{l+s+si}{\PYGZob{}}\PYG{n+nb}{len}\PYG{p}{(}\PYG{n}{t}\PYG{p}{)}\PYG{l+s+si}{\PYGZcb{}}\PYG{l+s+s1}{\PYGZsq{}}\PYG{p}{)}    \PYG{c+c1}{\PYGZsh{} only len() works for this tuuple}
\end{sphinxVerbatim}

\end{sphinxuseclass}\end{sphinxVerbatimInput}
\begin{sphinxVerbatimOutput}

\begin{sphinxuseclass}{cell_output}
\begin{sphinxVerbatim}[commandchars=\\\{\}]
Length : 8
\end{sphinxVerbatim}

\end{sphinxuseclass}\end{sphinxVerbatimOutput}

\end{sphinxuseclass}

\subsection{Indexing and Slicing}
\label{\detokenize{tuples:indexing-and-slicing}}\begin{itemize}
\item {} 
\sphinxAtStartPar
It is similar to strings.

\end{itemize}

\begin{sphinxuseclass}{cell}\begin{sphinxVerbatimInput}

\begin{sphinxuseclass}{cell_input}
\begin{sphinxVerbatim}[commandchars=\\\{\}]
\PYG{n}{t} \PYG{o}{=} \PYG{p}{(}\PYG{l+s+s1}{\PYGZsq{}}\PYG{l+s+s1}{USA}\PYG{l+s+s1}{\PYGZsq{}}\PYG{p}{,} \PYG{l+m+mi}{2}\PYG{p}{,} \PYG{k+kc}{True}\PYG{p}{,} \PYG{l+m+mf}{9.123}\PYG{p}{,} \PYG{l+s+s1}{\PYGZsq{}}\PYG{l+s+s1}{NY}\PYG{l+s+s1}{\PYGZsq{}}\PYG{p}{,} \PYG{l+s+s1}{\PYGZsq{}}\PYG{l+s+s1}{NJ}\PYG{l+s+s1}{\PYGZsq{}}\PYG{p}{,} \PYG{l+m+mi}{100}\PYG{p}{,} \PYG{k+kc}{False}\PYG{p}{)} 
\end{sphinxVerbatim}

\end{sphinxuseclass}\end{sphinxVerbatimInput}

\end{sphinxuseclass}
\sphinxAtStartPar
\sphinxstylestrong{Examples}

\begin{sphinxuseclass}{cell}\begin{sphinxVerbatimInput}

\begin{sphinxuseclass}{cell_input}
\begin{sphinxVerbatim}[commandchars=\\\{\}]
\PYG{c+c1}{\PYGZsh{} first element}
\PYG{n+nb}{print}\PYG{p}{(}\PYG{n}{t}\PYG{p}{[}\PYG{l+m+mi}{0}\PYG{p}{]}\PYG{p}{)}
\end{sphinxVerbatim}

\end{sphinxuseclass}\end{sphinxVerbatimInput}
\begin{sphinxVerbatimOutput}

\begin{sphinxuseclass}{cell_output}
\begin{sphinxVerbatim}[commandchars=\\\{\}]
USA
\end{sphinxVerbatim}

\end{sphinxuseclass}\end{sphinxVerbatimOutput}

\end{sphinxuseclass}
\begin{sphinxuseclass}{cell}\begin{sphinxVerbatimInput}

\begin{sphinxuseclass}{cell_input}
\begin{sphinxVerbatim}[commandchars=\\\{\}]
\PYG{c+c1}{\PYGZsh{} last element}
\PYG{n+nb}{print}\PYG{p}{(}\PYG{n}{t}\PYG{p}{[}\PYG{o}{\PYGZhy{}}\PYG{l+m+mi}{1}\PYG{p}{]}\PYG{p}{)}
\end{sphinxVerbatim}

\end{sphinxuseclass}\end{sphinxVerbatimInput}
\begin{sphinxVerbatimOutput}

\begin{sphinxuseclass}{cell_output}
\begin{sphinxVerbatim}[commandchars=\\\{\}]
False
\end{sphinxVerbatim}

\end{sphinxuseclass}\end{sphinxVerbatimOutput}

\end{sphinxuseclass}
\begin{sphinxuseclass}{cell}\begin{sphinxVerbatimInput}

\begin{sphinxuseclass}{cell_input}
\begin{sphinxVerbatim}[commandchars=\\\{\}]
\PYG{c+c1}{\PYGZsh{} index 3 element (fourth element)}
\PYG{n+nb}{print}\PYG{p}{(}\PYG{n}{t}\PYG{p}{[}\PYG{l+m+mi}{3}\PYG{p}{]}\PYG{p}{)}
\end{sphinxVerbatim}

\end{sphinxuseclass}\end{sphinxVerbatimInput}
\begin{sphinxVerbatimOutput}

\begin{sphinxuseclass}{cell_output}
\begin{sphinxVerbatim}[commandchars=\\\{\}]
9.123
\end{sphinxVerbatim}

\end{sphinxuseclass}\end{sphinxVerbatimOutput}

\end{sphinxuseclass}
\begin{sphinxuseclass}{cell}\begin{sphinxVerbatimInput}

\begin{sphinxuseclass}{cell_input}
\begin{sphinxVerbatim}[commandchars=\\\{\}]
\PYG{c+c1}{\PYGZsh{} index=2,3,4}
\PYG{n+nb}{print}\PYG{p}{(}\PYG{n}{t}\PYG{p}{[}\PYG{l+m+mi}{2}\PYG{p}{:}\PYG{l+m+mi}{5}\PYG{p}{]}\PYG{p}{)}
\end{sphinxVerbatim}

\end{sphinxuseclass}\end{sphinxVerbatimInput}
\begin{sphinxVerbatimOutput}

\begin{sphinxuseclass}{cell_output}
\begin{sphinxVerbatim}[commandchars=\\\{\}]
(True, 9.123, \PYGZsq{}NY\PYGZsq{})
\end{sphinxVerbatim}

\end{sphinxuseclass}\end{sphinxVerbatimOutput}

\end{sphinxuseclass}
\begin{sphinxuseclass}{cell}\begin{sphinxVerbatimInput}

\begin{sphinxuseclass}{cell_input}
\begin{sphinxVerbatim}[commandchars=\\\{\}]
\PYG{c+c1}{\PYGZsh{} index=\PYGZhy{}4,\PYGZhy{}3,\PYGZhy{}2}
\PYG{n+nb}{print}\PYG{p}{(}\PYG{n}{t}\PYG{p}{[}\PYG{o}{\PYGZhy{}}\PYG{l+m+mi}{4}\PYG{p}{:}\PYG{o}{\PYGZhy{}}\PYG{l+m+mi}{1}\PYG{p}{]}\PYG{p}{)}
\end{sphinxVerbatim}

\end{sphinxuseclass}\end{sphinxVerbatimInput}
\begin{sphinxVerbatimOutput}

\begin{sphinxuseclass}{cell_output}
\begin{sphinxVerbatim}[commandchars=\\\{\}]
(\PYGZsq{}NY\PYGZsq{}, \PYGZsq{}NJ\PYGZsq{}, 100)
\end{sphinxVerbatim}

\end{sphinxuseclass}\end{sphinxVerbatimOutput}

\end{sphinxuseclass}
\begin{sphinxuseclass}{cell}\begin{sphinxVerbatimInput}

\begin{sphinxuseclass}{cell_input}
\begin{sphinxVerbatim}[commandchars=\\\{\}]
\PYG{c+c1}{\PYGZsh{} slice starting from the index 3 element and all the way to the end}
\PYG{n+nb}{print}\PYG{p}{(}\PYG{n}{t}\PYG{p}{[}\PYG{l+m+mi}{3}\PYG{p}{:}\PYG{p}{]}\PYG{p}{)}
\end{sphinxVerbatim}

\end{sphinxuseclass}\end{sphinxVerbatimInput}
\begin{sphinxVerbatimOutput}

\begin{sphinxuseclass}{cell_output}
\begin{sphinxVerbatim}[commandchars=\\\{\}]
(9.123, \PYGZsq{}NY\PYGZsq{}, \PYGZsq{}NJ\PYGZsq{}, 100, False)
\end{sphinxVerbatim}

\end{sphinxuseclass}\end{sphinxVerbatimOutput}

\end{sphinxuseclass}
\sphinxAtStartPar
\sphinxstylestrong{Remark}
\begin{itemize}
\item {} 
\sphinxAtStartPar
There is a difference between the index \sphinxhyphen{}1 element and the slice {[}\sphinxhyphen{}1:{]}.

\item {} 
\sphinxAtStartPar
Both of them point to the last element of the tuple.

\item {} 
\sphinxAtStartPar
The first one returns the last element, whereas the latter one returns a length\sphinxhyphen{}one tuple with the last element.

\end{itemize}

\begin{sphinxuseclass}{cell}\begin{sphinxVerbatimInput}

\begin{sphinxuseclass}{cell_input}
\begin{sphinxVerbatim}[commandchars=\\\{\}]
\PYG{n+nb}{print}\PYG{p}{(}\PYG{l+s+sa}{f}\PYG{l+s+s1}{\PYGZsq{}}\PYG{l+s+s1}{index \PYGZhy{}1 element: }\PYG{l+s+si}{\PYGZob{}}\PYG{n}{t}\PYG{p}{[}\PYG{o}{\PYGZhy{}}\PYG{l+m+mi}{1}\PYG{p}{]}\PYG{l+s+si}{\PYGZcb{}}\PYG{l+s+s1}{, type: }\PYG{l+s+si}{\PYGZob{}}\PYG{n+nb}{type}\PYG{p}{(}\PYG{n}{t}\PYG{p}{[}\PYG{o}{\PYGZhy{}}\PYG{l+m+mi}{1}\PYG{p}{]}\PYG{p}{)}\PYG{l+s+si}{\PYGZcb{}}\PYG{l+s+s1}{\PYGZsq{}}\PYG{p}{)}     \PYG{c+c1}{\PYGZsh{} boolean}
\PYG{n+nb}{print}\PYG{p}{(}\PYG{l+s+sa}{f}\PYG{l+s+s1}{\PYGZsq{}}\PYG{l+s+s1}{slice [\PYGZhy{}1:]     : }\PYG{l+s+si}{\PYGZob{}}\PYG{n}{t}\PYG{p}{[}\PYG{o}{\PYGZhy{}}\PYG{l+m+mi}{1}\PYG{p}{:}\PYG{p}{]}\PYG{l+s+si}{\PYGZcb{}}\PYG{l+s+s1}{, type: }\PYG{l+s+si}{\PYGZob{}}\PYG{n+nb}{type}\PYG{p}{(}\PYG{n}{t}\PYG{p}{[}\PYG{o}{\PYGZhy{}}\PYG{l+m+mi}{1}\PYG{p}{:}\PYG{p}{]}\PYG{p}{)}\PYG{l+s+si}{\PYGZcb{}}\PYG{l+s+s1}{\PYGZsq{}}\PYG{p}{)}   \PYG{c+c1}{\PYGZsh{} tuple}
\end{sphinxVerbatim}

\end{sphinxuseclass}\end{sphinxVerbatimInput}
\begin{sphinxVerbatimOutput}

\begin{sphinxuseclass}{cell_output}
\begin{sphinxVerbatim}[commandchars=\\\{\}]
index \PYGZhy{}1 element: False, type: \PYGZlt{}class \PYGZsq{}bool\PYGZsq{}\PYGZgt{}
slice [\PYGZhy{}1:]     : (False,), type: \PYGZlt{}class \PYGZsq{}tuple\PYGZsq{}\PYGZgt{}
\end{sphinxVerbatim}

\end{sphinxuseclass}\end{sphinxVerbatimOutput}

\end{sphinxuseclass}
\sphinxAtStartPar
\sphinxstylestrong{Remark}
\begin{itemize}
\item {} 
\sphinxAtStartPar
A tuple in a super tuple is considered a single element of the super tuple.

\item {} 
\sphinxAtStartPar
Its elements are not considered elements of the super tuple.

\end{itemize}

\begin{sphinxuseclass}{cell}\begin{sphinxVerbatimInput}

\begin{sphinxuseclass}{cell_input}
\begin{sphinxVerbatim}[commandchars=\\\{\}]
\PYG{n}{t} \PYG{o}{=} \PYG{p}{(}\PYG{l+s+s1}{\PYGZsq{}}\PYG{l+s+s1}{USA}\PYG{l+s+s1}{\PYGZsq{}}\PYG{p}{,} \PYG{l+m+mi}{2}\PYG{p}{,} \PYG{k+kc}{True}\PYG{p}{,} \PYG{l+m+mf}{9.123}\PYG{p}{,} \PYG{p}{(}\PYG{l+m+mi}{10}\PYG{p}{,}\PYG{l+m+mi}{20}\PYG{p}{,}\PYG{l+m+mi}{30}\PYG{p}{)}\PYG{p}{)}
\PYG{n+nb}{print}\PYG{p}{(}\PYG{l+s+sa}{f}\PYG{l+s+s1}{\PYGZsq{}}\PYG{l+s+s1}{Length of t       : }\PYG{l+s+si}{\PYGZob{}}\PYG{n+nb}{len}\PYG{p}{(}\PYG{n}{t}\PYG{p}{)}\PYG{l+s+si}{\PYGZcb{}}\PYG{l+s+s1}{\PYGZsq{}}\PYG{p}{)}            \PYG{c+c1}{\PYGZsh{} (10,20,30) is a single element of t}
\PYG{n+nb}{print}\PYG{p}{(}\PYG{l+s+sa}{f}\PYG{l+s+s1}{\PYGZsq{}}\PYG{l+s+s1}{10 is in t        : }\PYG{l+s+si}{\PYGZob{}}\PYG{l+m+mi}{10}\PYG{+w}{ }\PYG{o+ow}{in}\PYG{+w}{ }\PYG{n}{t}\PYG{l+s+si}{\PYGZcb{}}\PYG{l+s+s1}{\PYGZsq{}}\PYG{p}{)}           \PYG{c+c1}{\PYGZsh{} 10 is not an element of t}
\PYG{n+nb}{print}\PYG{p}{(}\PYG{l+s+sa}{f}\PYG{l+s+s1}{\PYGZsq{}}\PYG{l+s+s1}{(10,20,30) is in t: }\PYG{l+s+si}{\PYGZob{}}\PYG{p}{(}\PYG{l+m+mi}{10}\PYG{p}{,}\PYG{l+m+mi}{20}\PYG{p}{,}\PYG{l+m+mi}{30}\PYG{p}{)}\PYG{+w}{ }\PYG{o+ow}{in}\PYG{+w}{ }\PYG{n}{t}\PYG{l+s+si}{\PYGZcb{}}\PYG{l+s+s1}{\PYGZsq{}}\PYG{p}{)}   \PYG{c+c1}{\PYGZsh{} (10,20,30) is  an element of t}
\end{sphinxVerbatim}

\end{sphinxuseclass}\end{sphinxVerbatimInput}
\begin{sphinxVerbatimOutput}

\begin{sphinxuseclass}{cell_output}
\begin{sphinxVerbatim}[commandchars=\\\{\}]
Length of t       : 5
10 is in t        : False
(10,20,30) is in t: True
\end{sphinxVerbatim}

\end{sphinxuseclass}\end{sphinxVerbatimOutput}

\end{sphinxuseclass}
\sphinxAtStartPar
\sphinxstylestrong{Remark}
It is possible to access the elements of the sub tuple by using chain indexing.

\begin{sphinxuseclass}{cell}\begin{sphinxVerbatimInput}

\begin{sphinxuseclass}{cell_input}
\begin{sphinxVerbatim}[commandchars=\\\{\}]
\PYG{n}{t} \PYG{o}{=} \PYG{p}{(}\PYG{l+s+s1}{\PYGZsq{}}\PYG{l+s+s1}{USA}\PYG{l+s+s1}{\PYGZsq{}}\PYG{p}{,} \PYG{l+m+mi}{2}\PYG{p}{,} \PYG{k+kc}{True}\PYG{p}{,} \PYG{l+m+mf}{9.123}\PYG{p}{,} \PYG{p}{(}\PYG{l+m+mi}{10}\PYG{p}{,}\PYG{l+m+mi}{20}\PYG{p}{,}\PYG{l+m+mi}{30}\PYG{p}{)}\PYG{p}{)}
\PYG{n+nb}{print}\PYG{p}{(}\PYG{l+s+sa}{f}\PYG{l+s+s1}{\PYGZsq{}}\PYG{l+s+s1}{t[\PYGZhy{}1]: }\PYG{l+s+si}{\PYGZob{}}\PYG{n}{t}\PYG{p}{[}\PYG{o}{\PYGZhy{}}\PYG{l+m+mi}{1}\PYG{p}{]}\PYG{l+s+si}{\PYGZcb{}}\PYG{l+s+s1}{\PYGZsq{}}\PYG{p}{)}          \PYG{c+c1}{\PYGZsh{} t[\PYGZhy{}1] = (10,20,30) is a tuple}
\PYG{n+nb}{print}\PYG{p}{(}\PYG{l+s+sa}{f}\PYG{l+s+s1}{\PYGZsq{}}\PYG{l+s+s1}{t[\PYGZhy{}1][0]: }\PYG{l+s+si}{\PYGZob{}}\PYG{n}{t}\PYG{p}{[}\PYG{o}{\PYGZhy{}}\PYG{l+m+mi}{1}\PYG{p}{]}\PYG{p}{[}\PYG{l+m+mi}{0}\PYG{p}{]}\PYG{l+s+si}{\PYGZcb{}}\PYG{l+s+s1}{\PYGZsq{}}\PYG{p}{)}    \PYG{c+c1}{\PYGZsh{} indexing of t[\PYGZhy{}1] = (10,20,30)}
\PYG{n+nb}{print}\PYG{p}{(}\PYG{l+s+sa}{f}\PYG{l+s+s1}{\PYGZsq{}}\PYG{l+s+s1}{t[\PYGZhy{}1][1]: }\PYG{l+s+si}{\PYGZob{}}\PYG{n}{t}\PYG{p}{[}\PYG{o}{\PYGZhy{}}\PYG{l+m+mi}{1}\PYG{p}{]}\PYG{p}{[}\PYG{l+m+mi}{1}\PYG{p}{]}\PYG{l+s+si}{\PYGZcb{}}\PYG{l+s+s1}{\PYGZsq{}}\PYG{p}{)}
\PYG{n+nb}{print}\PYG{p}{(}\PYG{l+s+sa}{f}\PYG{l+s+s1}{\PYGZsq{}}\PYG{l+s+s1}{t[\PYGZhy{}1][2]: }\PYG{l+s+si}{\PYGZob{}}\PYG{n}{t}\PYG{p}{[}\PYG{o}{\PYGZhy{}}\PYG{l+m+mi}{1}\PYG{p}{]}\PYG{p}{[}\PYG{l+m+mi}{2}\PYG{p}{]}\PYG{l+s+si}{\PYGZcb{}}\PYG{l+s+s1}{\PYGZsq{}}\PYG{p}{)}
\end{sphinxVerbatim}

\end{sphinxuseclass}\end{sphinxVerbatimInput}
\begin{sphinxVerbatimOutput}

\begin{sphinxuseclass}{cell_output}
\begin{sphinxVerbatim}[commandchars=\\\{\}]
t[\PYGZhy{}1]: (10, 20, 30)
t[\PYGZhy{}1][0]: 10
t[\PYGZhy{}1][1]: 20
t[\PYGZhy{}1][2]: 30
\end{sphinxVerbatim}

\end{sphinxuseclass}\end{sphinxVerbatimOutput}

\end{sphinxuseclass}

\subsection{Operators on Tuples}
\label{\detokenize{tuples:operators-on-tuples}}
\sphinxAtStartPar
Operators behave similarly to strings.
\begin{itemize}
\item {} 
\sphinxAtStartPar
\sphinxcode{\sphinxupquote{+}}: Concatenation

\item {} 
\sphinxAtStartPar
\sphinxcode{\sphinxupquote{*}}: Repetition (only integers are used)

\item {} 
\sphinxAtStartPar
\sphinxcode{\sphinxupquote{in}} and \sphinxcode{\sphinxupquote{not in}} operators: check whether a value is an element in a tuple.
\begin{itemize}
\item {} 
\sphinxAtStartPar
Returns a boolean value.

\end{itemize}

\end{itemize}

\sphinxAtStartPar
\sphinxstylestrong{Examples}

\begin{sphinxuseclass}{cell}\begin{sphinxVerbatimInput}

\begin{sphinxuseclass}{cell_input}
\begin{sphinxVerbatim}[commandchars=\\\{\}]
\PYG{n}{numbers} \PYG{o}{=} \PYG{p}{(}\PYG{l+m+mi}{1}\PYG{p}{,}\PYG{l+m+mi}{2}\PYG{p}{,}\PYG{l+m+mi}{3}\PYG{p}{,}\PYG{l+m+mi}{4}\PYG{p}{)}
\PYG{n}{letters} \PYG{o}{=} \PYG{p}{(}\PYG{l+s+s1}{\PYGZsq{}}\PYG{l+s+s1}{a}\PYG{l+s+s1}{\PYGZsq{}}\PYG{p}{,}\PYG{l+s+s1}{\PYGZsq{}}\PYG{l+s+s1}{b}\PYG{l+s+s1}{\PYGZsq{}}\PYG{p}{,}\PYG{l+s+s1}{\PYGZsq{}}\PYG{l+s+s1}{c}\PYG{l+s+s1}{\PYGZsq{}}\PYG{p}{,}\PYG{l+s+s1}{\PYGZsq{}}\PYG{l+s+s1}{d}\PYG{l+s+s1}{\PYGZsq{}}\PYG{p}{)}
\end{sphinxVerbatim}

\end{sphinxuseclass}\end{sphinxVerbatimInput}

\end{sphinxuseclass}
\begin{sphinxuseclass}{cell}\begin{sphinxVerbatimInput}

\begin{sphinxuseclass}{cell_input}
\begin{sphinxVerbatim}[commandchars=\\\{\}]
\PYG{c+c1}{\PYGZsh{} Concatenation returns a new tuple}

\PYG{n+nb}{print}\PYG{p}{(}\PYG{l+s+sa}{f}\PYG{l+s+s1}{\PYGZsq{}}\PYG{l+s+s1}{numbers + letters = }\PYG{l+s+si}{\PYGZob{}}\PYG{n}{numbers}\PYG{+w}{ }\PYG{o}{+}\PYG{+w}{ }\PYG{n}{letters}\PYG{l+s+si}{\PYGZcb{}}\PYG{l+s+s1}{\PYGZsq{}}\PYG{p}{)}
\PYG{n+nb}{print}\PYG{p}{(}\PYG{l+s+sa}{f}\PYG{l+s+s1}{\PYGZsq{}}\PYG{l+s+s1}{numbers           = }\PYG{l+s+si}{\PYGZob{}}\PYG{n}{numbers}\PYG{l+s+si}{\PYGZcb{}}\PYG{l+s+s1}{\PYGZsq{}}\PYG{p}{)}       \PYG{c+c1}{\PYGZsh{} no change}
\PYG{n+nb}{print}\PYG{p}{(}\PYG{l+s+sa}{f}\PYG{l+s+s1}{\PYGZsq{}}\PYG{l+s+s1}{letters           = }\PYG{l+s+si}{\PYGZob{}}\PYG{n}{letters}\PYG{l+s+si}{\PYGZcb{}}\PYG{l+s+s1}{\PYGZsq{}}\PYG{p}{)}       \PYG{c+c1}{\PYGZsh{} no change}
\end{sphinxVerbatim}

\end{sphinxuseclass}\end{sphinxVerbatimInput}
\begin{sphinxVerbatimOutput}

\begin{sphinxuseclass}{cell_output}
\begin{sphinxVerbatim}[commandchars=\\\{\}]
numbers + letters = (1, 2, 3, 4, \PYGZsq{}a\PYGZsq{}, \PYGZsq{}b\PYGZsq{}, \PYGZsq{}c\PYGZsq{}, \PYGZsq{}d\PYGZsq{})
numbers           = (1, 2, 3, 4)
letters           = (\PYGZsq{}a\PYGZsq{}, \PYGZsq{}b\PYGZsq{}, \PYGZsq{}c\PYGZsq{}, \PYGZsq{}d\PYGZsq{})
\end{sphinxVerbatim}

\end{sphinxuseclass}\end{sphinxVerbatimOutput}

\end{sphinxuseclass}
\begin{sphinxuseclass}{cell}\begin{sphinxVerbatimInput}

\begin{sphinxuseclass}{cell_input}
\begin{sphinxVerbatim}[commandchars=\\\{\}]
\PYG{c+c1}{\PYGZsh{} Repetition returns a new tuple}

\PYG{n+nb}{print}\PYG{p}{(}\PYG{l+s+sa}{f}\PYG{l+s+s1}{\PYGZsq{}}\PYG{l+s+s1}{letters*3 = }\PYG{l+s+si}{\PYGZob{}}\PYG{n}{letters}\PYG{o}{*}\PYG{l+m+mi}{3}\PYG{l+s+si}{\PYGZcb{}}\PYG{l+s+s1}{\PYGZsq{}}\PYG{p}{)}
\PYG{n+nb}{print}\PYG{p}{(}\PYG{l+s+sa}{f}\PYG{l+s+s1}{\PYGZsq{}}\PYG{l+s+s1}{letters   = }\PYG{l+s+si}{\PYGZob{}}\PYG{n}{letters}\PYG{l+s+si}{\PYGZcb{}}\PYG{l+s+s1}{\PYGZsq{}}\PYG{p}{)}       \PYG{c+c1}{\PYGZsh{} no change}
\end{sphinxVerbatim}

\end{sphinxuseclass}\end{sphinxVerbatimInput}
\begin{sphinxVerbatimOutput}

\begin{sphinxuseclass}{cell_output}
\begin{sphinxVerbatim}[commandchars=\\\{\}]
letters*3 = (\PYGZsq{}a\PYGZsq{}, \PYGZsq{}b\PYGZsq{}, \PYGZsq{}c\PYGZsq{}, \PYGZsq{}d\PYGZsq{}, \PYGZsq{}a\PYGZsq{}, \PYGZsq{}b\PYGZsq{}, \PYGZsq{}c\PYGZsq{}, \PYGZsq{}d\PYGZsq{}, \PYGZsq{}a\PYGZsq{}, \PYGZsq{}b\PYGZsq{}, \PYGZsq{}c\PYGZsq{}, \PYGZsq{}d\PYGZsq{})
letters   = (\PYGZsq{}a\PYGZsq{}, \PYGZsq{}b\PYGZsq{}, \PYGZsq{}c\PYGZsq{}, \PYGZsq{}d\PYGZsq{})
\end{sphinxVerbatim}

\end{sphinxuseclass}\end{sphinxVerbatimOutput}

\end{sphinxuseclass}
\begin{sphinxuseclass}{cell}\begin{sphinxVerbatimInput}

\begin{sphinxuseclass}{cell_input}
\begin{sphinxVerbatim}[commandchars=\\\{\}]
\PYG{c+c1}{\PYGZsh{} Is 5 in  numbers?}

\PYG{n+nb}{print}\PYG{p}{(}\PYG{l+s+sa}{f}\PYG{l+s+s1}{\PYGZsq{}}\PYG{l+s+s1}{ 5 is in numbers tuple    : }\PYG{l+s+si}{\PYGZob{}}\PYG{l+m+mi}{5}\PYG{+w}{  }\PYG{o+ow}{in}\PYG{+w}{ }\PYG{n}{numbers}\PYG{l+s+si}{\PYGZcb{}}\PYG{l+s+s1}{\PYGZsq{}} \PYG{p}{)}
\PYG{n+nb}{print}\PYG{p}{(}\PYG{l+s+sa}{f}\PYG{l+s+s1}{\PYGZsq{}}\PYG{l+s+s1}{ 5 is not in numbers tuple: }\PYG{l+s+si}{\PYGZob{}}\PYG{l+m+mi}{5}\PYG{+w}{  }\PYG{o+ow}{not}\PYG{+w}{ }\PYG{o+ow}{in}\PYG{+w}{ }\PYG{n}{numbers}\PYG{l+s+si}{\PYGZcb{}}\PYG{l+s+s1}{\PYGZsq{}} \PYG{p}{)}
\end{sphinxVerbatim}

\end{sphinxuseclass}\end{sphinxVerbatimInput}
\begin{sphinxVerbatimOutput}

\begin{sphinxuseclass}{cell_output}
\begin{sphinxVerbatim}[commandchars=\\\{\}]
 5 is in numbers tuple    : False
 5 is not in numbers tuple: True
\end{sphinxVerbatim}

\end{sphinxuseclass}\end{sphinxVerbatimOutput}

\end{sphinxuseclass}
\begin{sphinxuseclass}{cell}\begin{sphinxVerbatimInput}

\begin{sphinxuseclass}{cell_input}
\begin{sphinxVerbatim}[commandchars=\\\{\}]
\PYG{c+c1}{\PYGZsh{} Is 3 in  numbers?}

\PYG{n+nb}{print}\PYG{p}{(}\PYG{l+s+sa}{f}\PYG{l+s+s1}{\PYGZsq{}}\PYG{l+s+s1}{ 3 is in numbers tuple    : }\PYG{l+s+si}{\PYGZob{}}\PYG{l+m+mi}{3}\PYG{+w}{  }\PYG{o+ow}{in}\PYG{+w}{ }\PYG{n}{numbers}\PYG{l+s+si}{\PYGZcb{}}\PYG{l+s+s1}{\PYGZsq{}} \PYG{p}{)}
\PYG{n+nb}{print}\PYG{p}{(}\PYG{l+s+sa}{f}\PYG{l+s+s1}{\PYGZsq{}}\PYG{l+s+s1}{ 3 is not in numbers tuple: }\PYG{l+s+si}{\PYGZob{}}\PYG{l+m+mi}{3}\PYG{+w}{  }\PYG{o+ow}{not}\PYG{+w}{ }\PYG{o+ow}{in}\PYG{+w}{ }\PYG{n}{numbers}\PYG{l+s+si}{\PYGZcb{}}\PYG{l+s+s1}{\PYGZsq{}} \PYG{p}{)}
\end{sphinxVerbatim}

\end{sphinxuseclass}\end{sphinxVerbatimInput}
\begin{sphinxVerbatimOutput}

\begin{sphinxuseclass}{cell_output}
\begin{sphinxVerbatim}[commandchars=\\\{\}]
 3 is in numbers tuple    : True
 3 is not in numbers tuple: False
\end{sphinxVerbatim}

\end{sphinxuseclass}\end{sphinxVerbatimOutput}

\end{sphinxuseclass}

\subsection{Immutable}
\label{\detokenize{tuples:immutable}}
\sphinxAtStartPar
Similar to strings, tuples are immutable, which means they cannot be modified.
\begin{itemize}
\item {} 
\sphinxAtStartPar
For example, attempting to change the first element of a tuple will result in an error message.

\end{itemize}

\begin{sphinxVerbatim}[commandchars=\\\{\}]
\PYG{c+c1}{\PYGZsh{} ERROR:  try to change the first element, which has an index of 0.}
\PYG{n}{t} \PYG{o}{=} \PYG{p}{(}\PYG{l+m+mi}{1}\PYG{p}{,}\PYG{l+m+mi}{2}\PYG{p}{,}\PYG{l+m+mi}{3}\PYG{p}{,}\PYG{l+m+mi}{4}\PYG{p}{)}
\PYG{n}{t}\PYG{p}{[}\PYG{l+m+mi}{0}\PYG{p}{]} \PYG{o}{=} \PYG{l+m+mi}{100}  
\end{sphinxVerbatim}


\subsection{Tuple Methods}
\label{\detokenize{tuples:tuple-methods}}\begin{itemize}
\item {} 
\sphinxAtStartPar
Except for the magic methods (those with underscores), there are only two methods for tuples. Y

\item {} 
\sphinxAtStartPar
You can run help(tuple) for more details.

\end{itemize}

\begin{sphinxuseclass}{cell}\begin{sphinxVerbatimInput}

\begin{sphinxuseclass}{cell_input}
\begin{sphinxVerbatim}[commandchars=\\\{\}]
\PYG{c+c1}{\PYGZsh{} methods of tuples}
\PYG{n+nb}{print}\PYG{p}{(}\PYG{n+nb}{dir}\PYG{p}{(}\PYG{n+nb}{tuple}\PYG{p}{)}\PYG{p}{)}
\end{sphinxVerbatim}

\end{sphinxuseclass}\end{sphinxVerbatimInput}
\begin{sphinxVerbatimOutput}

\begin{sphinxuseclass}{cell_output}
\begin{sphinxVerbatim}[commandchars=\\\{\}]
[\PYGZsq{}\PYGZus{}\PYGZus{}add\PYGZus{}\PYGZus{}\PYGZsq{}, \PYGZsq{}\PYGZus{}\PYGZus{}class\PYGZus{}\PYGZus{}\PYGZsq{}, \PYGZsq{}\PYGZus{}\PYGZus{}class\PYGZus{}getitem\PYGZus{}\PYGZus{}\PYGZsq{}, \PYGZsq{}\PYGZus{}\PYGZus{}contains\PYGZus{}\PYGZus{}\PYGZsq{}, \PYGZsq{}\PYGZus{}\PYGZus{}delattr\PYGZus{}\PYGZus{}\PYGZsq{}, \PYGZsq{}\PYGZus{}\PYGZus{}dir\PYGZus{}\PYGZus{}\PYGZsq{}, \PYGZsq{}\PYGZus{}\PYGZus{}doc\PYGZus{}\PYGZus{}\PYGZsq{}, \PYGZsq{}\PYGZus{}\PYGZus{}eq\PYGZus{}\PYGZus{}\PYGZsq{}, \PYGZsq{}\PYGZus{}\PYGZus{}format\PYGZus{}\PYGZus{}\PYGZsq{}, \PYGZsq{}\PYGZus{}\PYGZus{}ge\PYGZus{}\PYGZus{}\PYGZsq{}, \PYGZsq{}\PYGZus{}\PYGZus{}getattribute\PYGZus{}\PYGZus{}\PYGZsq{}, \PYGZsq{}\PYGZus{}\PYGZus{}getitem\PYGZus{}\PYGZus{}\PYGZsq{}, \PYGZsq{}\PYGZus{}\PYGZus{}getnewargs\PYGZus{}\PYGZus{}\PYGZsq{}, \PYGZsq{}\PYGZus{}\PYGZus{}getstate\PYGZus{}\PYGZus{}\PYGZsq{}, \PYGZsq{}\PYGZus{}\PYGZus{}gt\PYGZus{}\PYGZus{}\PYGZsq{}, \PYGZsq{}\PYGZus{}\PYGZus{}hash\PYGZus{}\PYGZus{}\PYGZsq{}, \PYGZsq{}\PYGZus{}\PYGZus{}init\PYGZus{}\PYGZus{}\PYGZsq{}, \PYGZsq{}\PYGZus{}\PYGZus{}init\PYGZus{}subclass\PYGZus{}\PYGZus{}\PYGZsq{}, \PYGZsq{}\PYGZus{}\PYGZus{}iter\PYGZus{}\PYGZus{}\PYGZsq{}, \PYGZsq{}\PYGZus{}\PYGZus{}le\PYGZus{}\PYGZus{}\PYGZsq{}, \PYGZsq{}\PYGZus{}\PYGZus{}len\PYGZus{}\PYGZus{}\PYGZsq{}, \PYGZsq{}\PYGZus{}\PYGZus{}lt\PYGZus{}\PYGZus{}\PYGZsq{}, \PYGZsq{}\PYGZus{}\PYGZus{}mul\PYGZus{}\PYGZus{}\PYGZsq{}, \PYGZsq{}\PYGZus{}\PYGZus{}ne\PYGZus{}\PYGZus{}\PYGZsq{}, \PYGZsq{}\PYGZus{}\PYGZus{}new\PYGZus{}\PYGZus{}\PYGZsq{}, \PYGZsq{}\PYGZus{}\PYGZus{}reduce\PYGZus{}\PYGZus{}\PYGZsq{}, \PYGZsq{}\PYGZus{}\PYGZus{}reduce\PYGZus{}ex\PYGZus{}\PYGZus{}\PYGZsq{}, \PYGZsq{}\PYGZus{}\PYGZus{}repr\PYGZus{}\PYGZus{}\PYGZsq{}, \PYGZsq{}\PYGZus{}\PYGZus{}rmul\PYGZus{}\PYGZus{}\PYGZsq{}, \PYGZsq{}\PYGZus{}\PYGZus{}setattr\PYGZus{}\PYGZus{}\PYGZsq{}, \PYGZsq{}\PYGZus{}\PYGZus{}sizeof\PYGZus{}\PYGZus{}\PYGZsq{}, \PYGZsq{}\PYGZus{}\PYGZus{}str\PYGZus{}\PYGZus{}\PYGZsq{}, \PYGZsq{}\PYGZus{}\PYGZus{}subclasshook\PYGZus{}\PYGZus{}\PYGZsq{}, \PYGZsq{}count\PYGZsq{}, \PYGZsq{}index\PYGZsq{}]
\end{sphinxVerbatim}

\end{sphinxuseclass}\end{sphinxVerbatimOutput}

\end{sphinxuseclass}

\subsubsection{count()}
\label{\detokenize{tuples:count}}\begin{itemize}
\item {} 
\sphinxAtStartPar
It returns the number of occurrences of a given value in a tuple.

\end{itemize}

\begin{sphinxuseclass}{cell}\begin{sphinxVerbatimInput}

\begin{sphinxuseclass}{cell_input}
\begin{sphinxVerbatim}[commandchars=\\\{\}]
\PYG{n}{t} \PYG{o}{=} \PYG{p}{(}\PYG{l+s+s1}{\PYGZsq{}}\PYG{l+s+s1}{a}\PYG{l+s+s1}{\PYGZsq{}}\PYG{p}{,}\PYG{l+s+s1}{\PYGZsq{}}\PYG{l+s+s1}{a}\PYG{l+s+s1}{\PYGZsq{}}\PYG{p}{,}\PYG{l+s+s1}{\PYGZsq{}}\PYG{l+s+s1}{b}\PYG{l+s+s1}{\PYGZsq{}}\PYG{p}{,}\PYG{l+s+s1}{\PYGZsq{}}\PYG{l+s+s1}{b}\PYG{l+s+s1}{\PYGZsq{}}\PYG{p}{,}\PYG{l+s+s1}{\PYGZsq{}}\PYG{l+s+s1}{b}\PYG{l+s+s1}{\PYGZsq{}}\PYG{p}{,}\PYG{l+s+s1}{\PYGZsq{}}\PYG{l+s+s1}{b}\PYG{l+s+s1}{\PYGZsq{}}\PYG{p}{,}\PYG{l+s+s1}{\PYGZsq{}}\PYG{l+s+s1}{c}\PYG{l+s+s1}{\PYGZsq{}}\PYG{p}{)}
\PYG{n+nb}{print}\PYG{p}{(}\PYG{l+s+sa}{f}\PYG{l+s+s1}{\PYGZsq{}}\PYG{l+s+s1}{Number of a  in t: }\PYG{l+s+si}{\PYGZob{}}\PYG{n}{t}\PYG{o}{.}\PYG{n}{count}\PYG{p}{(}\PYG{l+s+s2}{\PYGZdq{}}\PYG{l+s+s2}{a}\PYG{l+s+s2}{\PYGZdq{}}\PYG{p}{)}\PYG{l+s+si}{\PYGZcb{}}\PYG{l+s+s1}{\PYGZsq{}}\PYG{p}{)}   \PYG{c+c1}{\PYGZsh{} use \PYGZdq{} instead of \PYGZsq{} for a}
\PYG{n+nb}{print}\PYG{p}{(}\PYG{l+s+sa}{f}\PYG{l+s+s1}{\PYGZsq{}}\PYG{l+s+s1}{Number of b  in t: }\PYG{l+s+si}{\PYGZob{}}\PYG{n}{t}\PYG{o}{.}\PYG{n}{count}\PYG{p}{(}\PYG{l+s+s2}{\PYGZdq{}}\PYG{l+s+s2}{b}\PYG{l+s+s2}{\PYGZdq{}}\PYG{p}{)}\PYG{l+s+si}{\PYGZcb{}}\PYG{l+s+s1}{\PYGZsq{}}\PYG{p}{)}
\PYG{n+nb}{print}\PYG{p}{(}\PYG{l+s+sa}{f}\PYG{l+s+s1}{\PYGZsq{}}\PYG{l+s+s1}{Number of c  in t: }\PYG{l+s+si}{\PYGZob{}}\PYG{n}{t}\PYG{o}{.}\PYG{n}{count}\PYG{p}{(}\PYG{l+s+s2}{\PYGZdq{}}\PYG{l+s+s2}{c}\PYG{l+s+s2}{\PYGZdq{}}\PYG{p}{)}\PYG{l+s+si}{\PYGZcb{}}\PYG{l+s+s1}{\PYGZsq{}}\PYG{p}{)}
\PYG{n+nb}{print}\PYG{p}{(}\PYG{l+s+sa}{f}\PYG{l+s+s1}{\PYGZsq{}}\PYG{l+s+s1}{Number of d  in t: }\PYG{l+s+si}{\PYGZob{}}\PYG{n}{t}\PYG{o}{.}\PYG{n}{count}\PYG{p}{(}\PYG{l+s+s2}{\PYGZdq{}}\PYG{l+s+s2}{d}\PYG{l+s+s2}{\PYGZdq{}}\PYG{p}{)}\PYG{l+s+si}{\PYGZcb{}}\PYG{l+s+s1}{\PYGZsq{}}\PYG{p}{)}
\end{sphinxVerbatim}

\end{sphinxuseclass}\end{sphinxVerbatimInput}
\begin{sphinxVerbatimOutput}

\begin{sphinxuseclass}{cell_output}
\begin{sphinxVerbatim}[commandchars=\\\{\}]
Number of a  in t: 2
Number of b  in t: 4
Number of c  in t: 1
Number of d  in t: 0
\end{sphinxVerbatim}

\end{sphinxuseclass}\end{sphinxVerbatimOutput}

\end{sphinxuseclass}

\subsubsection{index()}
\label{\detokenize{tuples:index}}\begin{itemize}
\item {} 
\sphinxAtStartPar
It returns the index of a given value in a tuple.

\item {} 
\sphinxAtStartPar
If the value is not in the tuple, an error message is generated.

\item {} 
\sphinxAtStartPar
In the case of repeated elements, it returns the smallest index

\end{itemize}

\begin{sphinxuseclass}{cell}\begin{sphinxVerbatimInput}

\begin{sphinxuseclass}{cell_input}
\begin{sphinxVerbatim}[commandchars=\\\{\}]
\PYG{n}{t} \PYG{o}{=} \PYG{p}{(}\PYG{l+s+s1}{\PYGZsq{}}\PYG{l+s+s1}{a}\PYG{l+s+s1}{\PYGZsq{}}\PYG{p}{,}\PYG{l+s+s1}{\PYGZsq{}}\PYG{l+s+s1}{b}\PYG{l+s+s1}{\PYGZsq{}}\PYG{p}{,}\PYG{l+s+s1}{\PYGZsq{}}\PYG{l+s+s1}{c}\PYG{l+s+s1}{\PYGZsq{}}\PYG{p}{,} \PYG{l+s+s1}{\PYGZsq{}}\PYG{l+s+s1}{a}\PYG{l+s+s1}{\PYGZsq{}}\PYG{p}{)}
\PYG{n+nb}{print}\PYG{p}{(}\PYG{l+s+sa}{f}\PYG{l+s+s1}{\PYGZsq{}}\PYG{l+s+s1}{The index of a in t: }\PYG{l+s+si}{\PYGZob{}}\PYG{n}{t}\PYG{o}{.}\PYG{n}{index}\PYG{p}{(}\PYG{l+s+s2}{\PYGZdq{}}\PYG{l+s+s2}{a}\PYG{l+s+s2}{\PYGZdq{}}\PYG{p}{)}\PYG{l+s+si}{\PYGZcb{}}\PYG{l+s+s1}{\PYGZsq{}}\PYG{p}{)}   \PYG{c+c1}{\PYGZsh{} index of first a}
\PYG{n+nb}{print}\PYG{p}{(}\PYG{l+s+sa}{f}\PYG{l+s+s1}{\PYGZsq{}}\PYG{l+s+s1}{The index of b in t: }\PYG{l+s+si}{\PYGZob{}}\PYG{n}{t}\PYG{o}{.}\PYG{n}{index}\PYG{p}{(}\PYG{l+s+s2}{\PYGZdq{}}\PYG{l+s+s2}{b}\PYG{l+s+s2}{\PYGZdq{}}\PYG{p}{)}\PYG{l+s+si}{\PYGZcb{}}\PYG{l+s+s1}{\PYGZsq{}}\PYG{p}{)} 
\PYG{n+nb}{print}\PYG{p}{(}\PYG{l+s+sa}{f}\PYG{l+s+s1}{\PYGZsq{}}\PYG{l+s+s1}{The index of c in t: }\PYG{l+s+si}{\PYGZob{}}\PYG{n}{t}\PYG{o}{.}\PYG{n}{index}\PYG{p}{(}\PYG{l+s+s2}{\PYGZdq{}}\PYG{l+s+s2}{c}\PYG{l+s+s2}{\PYGZdq{}}\PYG{p}{)}\PYG{l+s+si}{\PYGZcb{}}\PYG{l+s+s1}{\PYGZsq{}}\PYG{p}{)} 
\PYG{c+c1}{\PYGZsh{} print(f\PYGZsq{}The index of d in t: \PYGZob{}t.index(\PYGZdq{}d\PYGZdq{})\PYGZcb{}\PYGZsq{})  \PYGZhy{}\PYGZhy{}\PYGZhy{}\PYGZgt{} ERROR}
\end{sphinxVerbatim}

\end{sphinxuseclass}\end{sphinxVerbatimInput}
\begin{sphinxVerbatimOutput}

\begin{sphinxuseclass}{cell_output}
\begin{sphinxVerbatim}[commandchars=\\\{\}]
The index of a in t: 0
The index of b in t: 1
The index of c in t: 2
\end{sphinxVerbatim}

\end{sphinxuseclass}\end{sphinxVerbatimOutput}

\end{sphinxuseclass}

\subsection{Iterations and Tuples}
\label{\detokenize{tuples:iterations-and-tuples}}
\sphinxAtStartPar
We can use a \sphinxstyleemphasis{for} loop to access each element of a \sphinxstyleemphasis{tuple} and perform operations on each element.
This can be done in two different ways:
\begin{enumerate}
\sphinxsetlistlabels{\arabic}{enumi}{enumii}{}{.}%
\item {} 
\sphinxAtStartPar
Using the values in the tuple (Iterating through values).

\item {} 
\sphinxAtStartPar
Using indexes with the help of \sphinxstyleemphasis{len()} and \sphinxstyleemphasis{range()} functions (Iterating through indexes).
\begin{itemize}
\item {} 
\sphinxAtStartPar
The largest index of a tuple is \sphinxcode{\sphinxupquote{length of the tuple \sphinxhyphen{} 1}} since indexing starts from 0.

\item {} 
\sphinxAtStartPar
The indexes of a tuple are: \(0, 1, 2, ..., \text{length of the tuple} - 1\)

\item {} 
\sphinxAtStartPar
By using the range() function, we can use \sphinxcode{\sphinxupquote{range(length of the tuple)}}, which consists of all indexes.
\begin{itemize}
\item {} 
\sphinxAtStartPar
Example: range(5) does not include the number 5; that’s why we use the length of the tuple above.

\end{itemize}

\end{itemize}

\end{enumerate}

\sphinxAtStartPar
The next two code snippets print the state names in the states tuple in two different ways.

\begin{sphinxuseclass}{cell}\begin{sphinxVerbatimInput}

\begin{sphinxuseclass}{cell_input}
\begin{sphinxVerbatim}[commandchars=\\\{\}]
\PYG{n}{states} \PYG{o}{=} \PYG{p}{(}\PYG{l+s+s1}{\PYGZsq{}}\PYG{l+s+s1}{Oklahoma}\PYG{l+s+s1}{\PYGZsq{}}\PYG{p}{,} \PYG{l+s+s1}{\PYGZsq{}}\PYG{l+s+s1}{Texas}\PYG{l+s+s1}{\PYGZsq{}}\PYG{p}{,} \PYG{l+s+s1}{\PYGZsq{}}\PYG{l+s+s1}{Florida}\PYG{l+s+s1}{\PYGZsq{}}\PYG{p}{,} \PYG{l+s+s1}{\PYGZsq{}}\PYG{l+s+s1}{California}\PYG{l+s+s1}{\PYGZsq{}}\PYG{p}{)}

\PYG{k}{for} \PYG{n}{state} \PYG{o+ow}{in} \PYG{n}{states}\PYG{p}{:}
    \PYG{n+nb}{print}\PYG{p}{(}\PYG{n}{state}\PYG{p}{)}
\end{sphinxVerbatim}

\end{sphinxuseclass}\end{sphinxVerbatimInput}
\begin{sphinxVerbatimOutput}

\begin{sphinxuseclass}{cell_output}
\begin{sphinxVerbatim}[commandchars=\\\{\}]
Oklahoma
Texas
Florida
California
\end{sphinxVerbatim}

\end{sphinxuseclass}\end{sphinxVerbatimOutput}

\end{sphinxuseclass}
\begin{sphinxuseclass}{cell}\begin{sphinxVerbatimInput}

\begin{sphinxuseclass}{cell_input}
\begin{sphinxVerbatim}[commandchars=\\\{\}]
\PYG{n}{states} \PYG{o}{=} \PYG{p}{(}\PYG{l+s+s1}{\PYGZsq{}}\PYG{l+s+s1}{Oklahoma}\PYG{l+s+s1}{\PYGZsq{}}\PYG{p}{,} \PYG{l+s+s1}{\PYGZsq{}}\PYG{l+s+s1}{Texas}\PYG{l+s+s1}{\PYGZsq{}}\PYG{p}{,} \PYG{l+s+s1}{\PYGZsq{}}\PYG{l+s+s1}{Florida}\PYG{l+s+s1}{\PYGZsq{}}\PYG{p}{,} \PYG{l+s+s1}{\PYGZsq{}}\PYG{l+s+s1}{California}\PYG{l+s+s1}{\PYGZsq{}}\PYG{p}{)}

\PYG{k}{for} \PYG{n}{i} \PYG{o+ow}{in} \PYG{n+nb}{range}\PYG{p}{(}\PYG{n+nb}{len}\PYG{p}{(}\PYG{n}{states}\PYG{p}{)}\PYG{p}{)}\PYG{p}{:}
    \PYG{n+nb}{print}\PYG{p}{(}\PYG{n}{states}\PYG{p}{[}\PYG{n}{i}\PYG{p}{]}\PYG{p}{)}
\end{sphinxVerbatim}

\end{sphinxuseclass}\end{sphinxVerbatimInput}
\begin{sphinxVerbatimOutput}

\begin{sphinxuseclass}{cell_output}
\begin{sphinxVerbatim}[commandchars=\\\{\}]
Oklahoma
Texas
Florida
California
\end{sphinxVerbatim}

\end{sphinxuseclass}\end{sphinxVerbatimOutput}

\end{sphinxuseclass}\begin{itemize}
\item {} 
\sphinxAtStartPar
While a \sphinxstyleemphasis{while} loop can also be used, the \sphinxstyleemphasis{for} loop is usually easier to work with when iterating through tuples.

\end{itemize}

\begin{sphinxuseclass}{cell}\begin{sphinxVerbatimInput}

\begin{sphinxuseclass}{cell_input}
\begin{sphinxVerbatim}[commandchars=\\\{\}]
\PYG{n}{states} \PYG{o}{=} \PYG{p}{(}\PYG{l+s+s1}{\PYGZsq{}}\PYG{l+s+s1}{Oklahoma}\PYG{l+s+s1}{\PYGZsq{}}\PYG{p}{,} \PYG{l+s+s1}{\PYGZsq{}}\PYG{l+s+s1}{Texas}\PYG{l+s+s1}{\PYGZsq{}}\PYG{p}{,} \PYG{l+s+s1}{\PYGZsq{}}\PYG{l+s+s1}{Florida}\PYG{l+s+s1}{\PYGZsq{}}\PYG{p}{,} \PYG{l+s+s1}{\PYGZsq{}}\PYG{l+s+s1}{California}\PYG{l+s+s1}{\PYGZsq{}}\PYG{p}{)}
\PYG{n}{i} \PYG{o}{=} \PYG{l+m+mi}{0}

\PYG{k}{while} \PYG{n}{i} \PYG{o}{\PYGZlt{}}\PYG{n+nb}{len}\PYG{p}{(}\PYG{n}{states}\PYG{p}{)}\PYG{p}{:}
    \PYG{n+nb}{print}\PYG{p}{(}\PYG{n}{states}\PYG{p}{[}\PYG{n}{i}\PYG{p}{]}\PYG{p}{)}
    \PYG{n}{i} \PYG{o}{+}\PYG{o}{=} \PYG{l+m+mi}{1}
\end{sphinxVerbatim}

\end{sphinxuseclass}\end{sphinxVerbatimInput}
\begin{sphinxVerbatimOutput}

\begin{sphinxuseclass}{cell_output}
\begin{sphinxVerbatim}[commandchars=\\\{\}]
Oklahoma
Texas
Florida
California
\end{sphinxVerbatim}

\end{sphinxuseclass}\end{sphinxVerbatimOutput}

\end{sphinxuseclass}\begin{itemize}
\item {} 
\sphinxAtStartPar
Print the number of characters for each state.

\end{itemize}

\begin{sphinxuseclass}{cell}\begin{sphinxVerbatimInput}

\begin{sphinxuseclass}{cell_input}
\begin{sphinxVerbatim}[commandchars=\\\{\}]
\PYG{n}{states} \PYG{o}{=} \PYG{p}{(}\PYG{l+s+s1}{\PYGZsq{}}\PYG{l+s+s1}{Oklahoma}\PYG{l+s+s1}{\PYGZsq{}}\PYG{p}{,} \PYG{l+s+s1}{\PYGZsq{}}\PYG{l+s+s1}{Texas}\PYG{l+s+s1}{\PYGZsq{}}\PYG{p}{,} \PYG{l+s+s1}{\PYGZsq{}}\PYG{l+s+s1}{Florida}\PYG{l+s+s1}{\PYGZsq{}}\PYG{p}{,} \PYG{l+s+s1}{\PYGZsq{}}\PYG{l+s+s1}{California}\PYG{l+s+s1}{\PYGZsq{}}\PYG{p}{)}

\PYG{k}{for} \PYG{n}{state} \PYG{o+ow}{in} \PYG{n}{states}\PYG{p}{:}
    \PYG{n+nb}{print}\PYG{p}{(}\PYG{n+nb}{len}\PYG{p}{(}\PYG{n}{state}\PYG{p}{)}\PYG{p}{)}         
\end{sphinxVerbatim}

\end{sphinxuseclass}\end{sphinxVerbatimInput}
\begin{sphinxVerbatimOutput}

\begin{sphinxuseclass}{cell_output}
\begin{sphinxVerbatim}[commandchars=\\\{\}]
8
5
7
10
\end{sphinxVerbatim}

\end{sphinxuseclass}\end{sphinxVerbatimOutput}

\end{sphinxuseclass}
\begin{sphinxuseclass}{cell}\begin{sphinxVerbatimInput}

\begin{sphinxuseclass}{cell_input}
\begin{sphinxVerbatim}[commandchars=\\\{\}]
\PYG{n}{states} \PYG{o}{=} \PYG{p}{(}\PYG{l+s+s1}{\PYGZsq{}}\PYG{l+s+s1}{Oklahoma}\PYG{l+s+s1}{\PYGZsq{}}\PYG{p}{,} \PYG{l+s+s1}{\PYGZsq{}}\PYG{l+s+s1}{Texas}\PYG{l+s+s1}{\PYGZsq{}}\PYG{p}{,} \PYG{l+s+s1}{\PYGZsq{}}\PYG{l+s+s1}{Florida}\PYG{l+s+s1}{\PYGZsq{}}\PYG{p}{,} \PYG{l+s+s1}{\PYGZsq{}}\PYG{l+s+s1}{California}\PYG{l+s+s1}{\PYGZsq{}}\PYG{p}{)}

\PYG{k}{for} \PYG{n}{i} \PYG{o+ow}{in} \PYG{n+nb}{range}\PYG{p}{(}\PYG{n+nb}{len}\PYG{p}{(}\PYG{n}{states}\PYG{p}{)}\PYG{p}{)}\PYG{p}{:}
    \PYG{n+nb}{print}\PYG{p}{(}\PYG{n+nb}{len}\PYG{p}{(}\PYG{n}{states}\PYG{p}{[}\PYG{n}{i}\PYG{p}{]}\PYG{p}{)}\PYG{p}{)}        \PYG{c+c1}{\PYGZsh{} states[i] is a state name }
\end{sphinxVerbatim}

\end{sphinxuseclass}\end{sphinxVerbatimInput}
\begin{sphinxVerbatimOutput}

\begin{sphinxuseclass}{cell_output}
\begin{sphinxVerbatim}[commandchars=\\\{\}]
8
5
7
10
\end{sphinxVerbatim}

\end{sphinxuseclass}\end{sphinxVerbatimOutput}

\end{sphinxuseclass}

\subsection{Examples}
\label{\detokenize{tuples:examples}}

\subsubsection{Split even and odd numbers}
\label{\detokenize{tuples:split-even-and-odd-numbers}}
\sphinxAtStartPar
Write a program that stores even numbers from the \sphinxstyleemphasis{numbers} tuple in a new tuple called \sphinxstyleemphasis{t\_even} and odd numbers in another tuple called \sphinxstyleemphasis{t\_odd}.
\begin{itemize}
\item {} 
\sphinxAtStartPar
Do not use lists.

\end{itemize}

\begin{sphinxuseclass}{cell}\begin{sphinxVerbatimInput}

\begin{sphinxuseclass}{cell_input}
\begin{sphinxVerbatim}[commandchars=\\\{\}]
\PYG{n}{numbers} \PYG{o}{=} \PYG{p}{(}\PYG{l+m+mi}{6}\PYG{p}{,}\PYG{l+m+mi}{2}\PYG{p}{,}\PYG{l+m+mi}{9}\PYG{p}{,}\PYG{l+m+mi}{1}\PYG{p}{,}\PYG{l+m+mi}{2}\PYG{p}{,}\PYG{l+m+mi}{12}\PYG{p}{,}\PYG{l+m+mi}{5}\PYG{p}{,}\PYG{l+m+mi}{9}\PYG{p}{,}\PYG{l+m+mi}{3}\PYG{p}{,}\PYG{l+m+mi}{5}\PYG{p}{,}\PYG{l+m+mi}{7}\PYG{p}{,}\PYG{l+m+mi}{2}\PYG{p}{,}\PYG{l+m+mi}{1}\PYG{p}{,}\PYG{l+m+mi}{78}\PYG{p}{,}\PYG{l+m+mi}{43}\PYG{p}{,}\PYG{l+m+mi}{23}\PYG{p}{,}\PYG{l+m+mi}{67}\PYG{p}{,}\PYG{l+m+mi}{65}\PYG{p}{,}\PYG{l+m+mi}{32}\PYG{p}{,}\PYG{l+m+mi}{34}\PYG{p}{,}\PYG{l+m+mi}{76}\PYG{p}{,}\PYG{l+m+mi}{54}\PYG{p}{)}
\end{sphinxVerbatim}

\end{sphinxuseclass}\end{sphinxVerbatimInput}

\end{sphinxuseclass}
\sphinxAtStartPar
\sphinxstylestrong{Solution:}

\begin{sphinxuseclass}{cell}\begin{sphinxVerbatimInput}

\begin{sphinxuseclass}{cell_input}
\begin{sphinxVerbatim}[commandchars=\\\{\}]
\PYG{n}{t\PYGZus{}odd} \PYG{o}{=} \PYG{p}{(}\PYG{p}{)}
\PYG{n}{t\PYGZus{}even} \PYG{o}{=}\PYG{p}{(}\PYG{p}{)}

\PYG{k}{for} \PYG{n}{i} \PYG{o+ow}{in} \PYG{n}{numbers}\PYG{p}{:}
  \PYG{k}{if} \PYG{n}{i}\PYG{o}{\PYGZpc{}}\PYG{k}{2} == 0:
    \PYG{n}{t\PYGZus{}even} \PYG{o}{+}\PYG{o}{=} \PYG{p}{(}\PYG{n}{i}\PYG{p}{,}\PYG{p}{)}      \PYG{c+c1}{\PYGZsh{} concatenation of tuples t\PYGZus{}even and (i,) }
  \PYG{k}{else}\PYG{p}{:}
    \PYG{n}{t\PYGZus{}odd} \PYG{o}{+}\PYG{o}{=} \PYG{p}{(}\PYG{n}{i}\PYG{p}{,}\PYG{p}{)}
      
\PYG{n+nb}{print}\PYG{p}{(}\PYG{l+s+s1}{\PYGZsq{}}\PYG{l+s+s1}{Even Numbers:}\PYG{l+s+s1}{\PYGZsq{}}\PYG{p}{,} \PYG{n}{t\PYGZus{}even}\PYG{p}{)}
\PYG{n+nb}{print}\PYG{p}{(}\PYG{l+s+s1}{\PYGZsq{}}\PYG{l+s+s1}{Odd  Numbers:}\PYG{l+s+s1}{\PYGZsq{}}\PYG{p}{,} \PYG{n}{t\PYGZus{}odd}\PYG{p}{)}
\end{sphinxVerbatim}

\end{sphinxuseclass}\end{sphinxVerbatimInput}
\begin{sphinxVerbatimOutput}

\begin{sphinxuseclass}{cell_output}
\begin{sphinxVerbatim}[commandchars=\\\{\}]
Even Numbers: (6, 2, 2, 12, 2, 78, 32, 34, 76, 54)
Odd  Numbers: (9, 1, 5, 9, 3, 5, 7, 1, 43, 23, 67, 65)
\end{sphinxVerbatim}

\end{sphinxuseclass}\end{sphinxVerbatimOutput}

\end{sphinxuseclass}

\subsubsection{Split data types}
\label{\detokenize{tuples:split-data-types}}
\sphinxAtStartPar
Write a program that stores the strings in the \sphinxstyleemphasis{mix\_tuple} tuple into a tuple called \sphinxstyleemphasis{t\_string}, integers into \sphinxstyleemphasis{t\_integer}, floats into \sphinxstyleemphasis{t\_float}, and booleans into \sphinxstyleemphasis{t\_boolean}.
\begin{itemize}
\item {} 
\sphinxAtStartPar
The \sphinxstyleemphasis{mix\_tuple} contains only strings, integers, floats, and boolean values.

\item {} 
\sphinxAtStartPar
Do not use lists.

\end{itemize}

\begin{sphinxuseclass}{cell}\begin{sphinxVerbatimInput}

\begin{sphinxuseclass}{cell_input}
\begin{sphinxVerbatim}[commandchars=\\\{\}]
\PYG{n}{mix\PYGZus{}tuple} \PYG{o}{=} \PYG{p}{(}\PYG{l+m+mi}{1}\PYG{p}{,} \PYG{l+m+mi}{3}\PYG{p}{,} \PYG{l+s+s1}{\PYGZsq{}}\PYG{l+s+s1}{NJ}\PYG{l+s+s1}{\PYGZsq{}}\PYG{p}{,} \PYG{k+kc}{False}\PYG{p}{,} \PYG{l+s+s1}{\PYGZsq{}}\PYG{l+s+s1}{OK}\PYG{l+s+s1}{\PYGZsq{}}\PYG{p}{,}\PYG{l+m+mi}{5}\PYG{p}{,} \PYG{l+m+mi}{8}\PYG{p}{,}\PYG{l+s+s1}{\PYGZsq{}}\PYG{l+s+s1}{Hello}\PYG{l+s+s1}{\PYGZsq{}}\PYG{p}{,} \PYG{l+m+mf}{9.8}\PYG{p}{,} \PYG{k+kc}{True} \PYG{p}{,}\PYG{l+m+mi}{9}\PYG{p}{,}\PYG{l+m+mi}{87}\PYG{p}{)}
\end{sphinxVerbatim}

\end{sphinxuseclass}\end{sphinxVerbatimInput}

\end{sphinxuseclass}
\sphinxAtStartPar
\sphinxstylestrong{Solution:}

\begin{sphinxuseclass}{cell}\begin{sphinxVerbatimInput}

\begin{sphinxuseclass}{cell_input}
\begin{sphinxVerbatim}[commandchars=\\\{\}]
\PYG{n}{t\PYGZus{}string}\PYG{p}{,} \PYG{n}{t\PYGZus{}number}\PYG{p}{,} \PYG{n}{t\PYGZus{}boolean} \PYG{o}{=} \PYG{p}{(}\PYG{p}{)}\PYG{p}{,} \PYG{p}{(}\PYG{p}{)}\PYG{p}{,} \PYG{p}{(}\PYG{p}{)}

\PYG{k}{for} \PYG{n}{i} \PYG{o+ow}{in} \PYG{n}{mix\PYGZus{}tuple}\PYG{p}{:}
  \PYG{k}{if} \PYG{n+nb}{type}\PYG{p}{(}\PYG{n}{i}\PYG{p}{)} \PYG{o}{==} \PYG{n+nb}{str}\PYG{p}{:}
    \PYG{n}{t\PYGZus{}string} \PYG{o}{+}\PYG{o}{=} \PYG{p}{(}\PYG{n}{i}\PYG{p}{,}\PYG{p}{)}
  \PYG{k}{elif} \PYG{n+nb}{type}\PYG{p}{(}\PYG{n}{i}\PYG{p}{)} \PYG{o}{==} \PYG{n+nb}{bool}\PYG{p}{:}
    \PYG{n}{t\PYGZus{}boolean} \PYG{o}{+}\PYG{o}{=} \PYG{p}{(}\PYG{n}{i}\PYG{p}{,}\PYG{p}{)}
  \PYG{k}{else}\PYG{p}{:}
    \PYG{n}{t\PYGZus{}number} \PYG{o}{+}\PYG{o}{=} \PYG{p}{(}\PYG{n}{i}\PYG{p}{,}\PYG{p}{)}
      
\PYG{n+nb}{print}\PYG{p}{(}\PYG{l+s+s1}{\PYGZsq{}}\PYG{l+s+s1}{Strings :}\PYG{l+s+s1}{\PYGZsq{}}\PYG{p}{,} \PYG{n}{t\PYGZus{}string}\PYG{p}{)}
\PYG{n+nb}{print}\PYG{p}{(}\PYG{l+s+s1}{\PYGZsq{}}\PYG{l+s+s1}{Numbers :}\PYG{l+s+s1}{\PYGZsq{}}\PYG{p}{,} \PYG{n}{t\PYGZus{}number}\PYG{p}{)}
\PYG{n+nb}{print}\PYG{p}{(}\PYG{l+s+s1}{\PYGZsq{}}\PYG{l+s+s1}{Booleans:}\PYG{l+s+s1}{\PYGZsq{}}\PYG{p}{,} \PYG{n}{t\PYGZus{}boolean}\PYG{p}{)}
\end{sphinxVerbatim}

\end{sphinxuseclass}\end{sphinxVerbatimInput}
\begin{sphinxVerbatimOutput}

\begin{sphinxuseclass}{cell_output}
\begin{sphinxVerbatim}[commandchars=\\\{\}]
Strings : (\PYGZsq{}NJ\PYGZsq{}, \PYGZsq{}OK\PYGZsq{}, \PYGZsq{}Hello\PYGZsq{})
Numbers : (1, 3, 5, 8, 9.8, 9, 87)
Booleans: (False, True)
\end{sphinxVerbatim}

\end{sphinxuseclass}\end{sphinxVerbatimOutput}

\end{sphinxuseclass}
\sphinxstepscope


\section{Tuples Debugging}
\label{\detokenize{tuples_debug:tuples-debugging}}\label{\detokenize{tuples_debug::doc}}\begin{itemize}
\item {} 
\sphinxAtStartPar
Each of the following short code contains one or more bugs.     

\item {} 
\sphinxAtStartPar
Please identify and correct these bugs.

\item {} 
\sphinxAtStartPar
Provide an explanation for your answer.

\end{itemize}


\subsection{Question}
\label{\detokenize{tuples_debug:question}}
\begin{sphinxVerbatim}[commandchars=\\\{\}]
\PYG{n}{x} \PYG{o}{=} \PYG{p}{(}\PYG{l+m+mi}{1}\PYG{p}{,}\PYG{l+m+mi}{2}\PYG{p}{,}\PYG{l+m+mi}{3}\PYG{p}{,}\PYG{l+m+mi}{4}\PYG{p}{,}\PYG{l+m+mi}{5}\PYG{p}{,}\PYG{l+m+mi}{6}\PYG{p}{,}\PYG{l+m+mi}{7}\PYG{p}{,}\PYG{l+m+mi}{8}\PYG{p}{,}\PYG{l+m+mi}{9}\PYG{p}{]}
\end{sphinxVerbatim}

\begin{sphinxadmonition}{note}{Solution}

\sphinxAtStartPar
The right square bracket \sphinxcode{\sphinxupquote{{]}}} must be a paranthesis \sphinxcode{\sphinxupquote{)}} so \sphinxstyleemphasis{x} can be a tuple.
\end{sphinxadmonition}


\subsection{Question}
\label{\detokenize{tuples_debug:id1}}
\begin{sphinxVerbatim}[commandchars=\\\{\}]
\PYG{n}{x}\PYG{p}{,} \PYG{n}{y} \PYG{o}{=} \PYG{l+s+s1}{\PYGZsq{}}\PYG{l+s+s1}{NY}\PYG{l+s+s1}{\PYGZsq{}}\PYG{p}{,} \PYG{l+s+s1}{\PYGZsq{}}\PYG{l+s+s1}{NJ}\PYG{l+s+s1}{\PYGZsq{}}
\PYG{n}{x}\PYG{o}{*}\PYG{n}{y}
\end{sphinxVerbatim}

\begin{sphinxadmonition}{note}{Solution}

\sphinxAtStartPar
The \sphinxcode{\sphinxupquote{*}} operator cannot be used for two strings; it can only be used with either two numbers or one string and one integer.
\end{sphinxadmonition}


\subsection{Question}
\label{\detokenize{tuples_debug:id2}}
\begin{sphinxVerbatim}[commandchars=\\\{\}]
\PYG{n}{x} \PYG{o}{=} \PYG{p}{(}\PYG{l+s+s1}{\PYGZsq{}}\PYG{l+s+s1}{A}\PYG{l+s+s1}{\PYGZsq{}}\PYG{p}{,} \PYG{l+s+s1}{\PYGZsq{}}\PYG{l+s+s1}{B}\PYG{l+s+s1}{\PYGZsq{}}\PYG{p}{,} \PYG{l+s+s1}{\PYGZsq{}}\PYG{l+s+s1}{C}\PYG{l+s+s1}{\PYGZsq{}}\PYG{p}{,} \PYG{l+s+s1}{\PYGZsq{}}\PYG{l+s+s1}{D}\PYG{l+s+s1}{\PYGZsq{}}\PYG{p}{,} \PYG{l+s+s1}{\PYGZsq{}}\PYG{l+s+s1}{E}\PYG{l+s+s1}{\PYGZsq{}}\PYG{p}{)}
\PYG{n}{x}\PYG{p}{[}\PYG{l+m+mi}{2}\PYG{p}{]} \PYG{o}{=} \PYG{l+s+s1}{\PYGZsq{}}\PYG{l+s+s1}{\PYGZhy{}}\PYG{l+s+s1}{\PYGZsq{}}
\end{sphinxVerbatim}

\begin{sphinxadmonition}{note}{Solution}

\sphinxAtStartPar
\sphinxstyleemphasis{x} is a tuple, and tuples are immutable, so elements cannot be changed. \sphinxstyleemphasis{x{[}2{]}}, which is ‘C’, cannot be changed as ‘\sphinxhyphen{}‘.
\end{sphinxadmonition}


\subsection{Question}
\label{\detokenize{tuples_debug:id3}}
\begin{sphinxVerbatim}[commandchars=\\\{\}]
\PYG{n}{x} \PYG{o}{=} \PYG{p}{(}\PYG{l+s+s1}{\PYGZsq{}}\PYG{l+s+s1}{A}\PYG{l+s+s1}{\PYGZsq{}}\PYG{p}{,} \PYG{l+s+s1}{\PYGZsq{}}\PYG{l+s+s1}{B}\PYG{l+s+s1}{\PYGZsq{}}\PYG{p}{,} \PYG{l+s+s1}{\PYGZsq{}}\PYG{l+s+s1}{C}\PYG{l+s+s1}{\PYGZsq{}}\PYG{p}{,} \PYG{l+s+s1}{\PYGZsq{}}\PYG{l+s+s1}{D}\PYG{l+s+s1}{\PYGZsq{}}\PYG{p}{,} \PYG{l+s+s1}{\PYGZsq{}}\PYG{l+s+s1}{E}\PYG{l+s+s1}{\PYGZsq{}}\PYG{p}{)}
\PYG{n}{x}\PYG{p}{[}\PYG{o}{\PYGZhy{}}\PYG{l+m+mi}{6}\PYG{p}{]}
\end{sphinxVerbatim}

\begin{sphinxadmonition}{note}{Solution}

\sphinxAtStartPar
There is no element with index \sphinxhyphen{}6 because the negative indexes are: \sphinxhyphen{}1 for ‘E’, \sphinxhyphen{}2 for ‘D’, \sphinxhyphen{}3 for ‘C’, \sphinxhyphen{}4 for ‘B’, \sphinxhyphen{}5 for ‘A’.
\end{sphinxadmonition}


\subsection{Question}
\label{\detokenize{tuples_debug:id4}}
\begin{sphinxVerbatim}[commandchars=\\\{\}]
\PYG{n}{sequence} \PYG{o}{=} \PYG{p}{(}\PYG{l+m+mi}{1}\PYG{p}{,}\PYG{l+m+mi}{2}\PYG{p}{,}\PYG{l+m+mi}{3}\PYG{p}{,}\PYG{l+m+mi}{4}\PYG{p}{,}\PYG{l+m+mi}{5}\PYG{p}{)}
\PYG{n}{sequence}\PYG{o}{.}\PYG{n}{append}\PYG{p}{(}\PYG{l+m+mi}{6}\PYG{p}{)}
\end{sphinxVerbatim}

\begin{sphinxadmonition}{note}{Solution}

\sphinxAtStartPar
\sphinxstyleemphasis{sequence} is a tuple, so it is immutable. It cannot be modified, and new elements cannot be added to it. Therefore, tuples do not have an \sphinxstyleemphasis{append()} method.
\end{sphinxadmonition}


\subsection{Question}
\label{\detokenize{tuples_debug:id5}}
\begin{sphinxVerbatim}[commandchars=\\\{\}]
\PYG{n}{sequence} \PYG{o}{=} \PYG{p}{(}\PYG{l+m+mi}{1}\PYG{p}{,}\PYG{l+m+mi}{2}\PYG{p}{,}\PYG{l+m+mi}{3}\PYG{p}{,}\PYG{l+m+mi}{4}\PYG{p}{,}\PYG{l+m+mi}{5}\PYG{p}{)}

\PYG{k}{for} \PYG{n}{i} \PYG{o+ow}{in} \PYG{n+nb}{range}\PYG{p}{(}\PYG{l+m+mi}{5}\PYG{p}{)}\PYG{p}{:}
  \PYG{n+nb}{print}\PYG{p}{(}\PYG{n}{sequence}\PYG{p}{[}\PYG{n}{i}\PYG{o}{+}\PYG{l+m+mi}{1}\PYG{p}{]}\PYG{p}{)}
\end{sphinxVerbatim}

\begin{sphinxadmonition}{note}{Solution}

\sphinxAtStartPar
If \sphinxstyleemphasis{i=4}, \sphinxstyleemphasis{sequence{[}4+1{]}} refers to the element at the tuple index of 5, but the largest index is 4 (index of element 5). Therefore, sequence{[}5{]} does not exist.
\end{sphinxadmonition}


\subsection{Question}
\label{\detokenize{tuples_debug:id6}}
\begin{sphinxVerbatim}[commandchars=\\\{\}]
\PYG{n}{sequence} \PYG{o}{=} \PYG{p}{(}\PYG{l+s+s1}{\PYGZsq{}}\PYG{l+s+s1}{a}\PYG{l+s+s1}{\PYGZsq{}}\PYG{p}{,}\PYG{l+s+s1}{\PYGZsq{}}\PYG{l+s+s1}{b}\PYG{l+s+s1}{\PYGZsq{}}\PYG{p}{,}\PYG{l+s+s1}{\PYGZsq{}}\PYG{l+s+s1}{c}\PYG{l+s+s1}{\PYGZsq{}}\PYG{p}{)}

\PYG{n+nb}{print}\PYG{p}{(}\PYG{n}{sequence}\PYG{o}{.}\PYG{n}{index}\PYG{p}{(}\PYG{l+s+s1}{\PYGZsq{}}\PYG{l+s+s1}{A}\PYG{l+s+s1}{\PYGZsq{}}\PYG{p}{)}\PYG{p}{)}
\end{sphinxVerbatim}

\begin{sphinxadmonition}{note}{Solution}

\sphinxAtStartPar
The index() method raises an error since ‘A’ is not a member of the sequence tuple.
\end{sphinxadmonition}

\sphinxstepscope


\section{Tuples Output}
\label{\detokenize{tuples_output:tuples-output}}\label{\detokenize{tuples_output::doc}}\begin{itemize}
\item {} 
\sphinxAtStartPar
Find the output of the following code.

\item {} 
\sphinxAtStartPar
Please don’t run the code before giving your answer.     

\end{itemize}


\subsection{Question}
\label{\detokenize{tuples_output:question}}
\begin{sphinxuseclass}{cell}
\begin{sphinxuseclass}{tag_hide-output}\begin{sphinxVerbatimInput}

\begin{sphinxuseclass}{cell_input}
\begin{sphinxVerbatim}[commandchars=\\\{\}]
\PYG{n}{x} \PYG{o}{=} \PYG{p}{(}\PYG{l+s+s1}{\PYGZsq{}}\PYG{l+s+s1}{a}\PYG{l+s+s1}{\PYGZsq{}}\PYG{p}{,} \PYG{l+s+s1}{\PYGZsq{}}\PYG{l+s+s1}{b}\PYG{l+s+s1}{\PYGZsq{}}\PYG{p}{,}\PYG{l+s+s1}{\PYGZsq{}}\PYG{l+s+s1}{c}\PYG{l+s+s1}{\PYGZsq{}}\PYG{p}{,}\PYG{l+m+mi}{1}\PYG{p}{,}\PYG{l+m+mi}{2}\PYG{p}{,}\PYG{l+m+mi}{3}\PYG{p}{,}\PYG{l+s+s1}{\PYGZsq{}}\PYG{l+s+s1}{a}\PYG{l+s+s1}{\PYGZsq{}}\PYG{p}{)}

\PYG{n+nb}{print}\PYG{p}{(}\PYG{n}{x}\PYG{o}{.}\PYG{n}{count}\PYG{p}{(}\PYG{l+s+s1}{\PYGZsq{}}\PYG{l+s+s1}{a}\PYG{l+s+s1}{\PYGZsq{}}\PYG{p}{)}\PYG{p}{)}
\PYG{n+nb}{print}\PYG{p}{(}\PYG{n}{x}\PYG{o}{.}\PYG{n}{count}\PYG{p}{(}\PYG{l+s+s1}{\PYGZsq{}}\PYG{l+s+s1}{c}\PYG{l+s+s1}{\PYGZsq{}}\PYG{p}{)}\PYG{p}{)}
\PYG{n+nb}{print}\PYG{p}{(}\PYG{n}{x}\PYG{o}{.}\PYG{n}{count}\PYG{p}{(}\PYG{l+s+s1}{\PYGZsq{}}\PYG{l+s+s1}{B}\PYG{l+s+s1}{\PYGZsq{}}\PYG{p}{)}\PYG{p}{)}
\end{sphinxVerbatim}

\end{sphinxuseclass}\end{sphinxVerbatimInput}

\end{sphinxuseclass}
\end{sphinxuseclass}

\subsection{Question}
\label{\detokenize{tuples_output:id1}}
\begin{sphinxuseclass}{cell}
\begin{sphinxuseclass}{tag_hide-output}\begin{sphinxVerbatimInput}

\begin{sphinxuseclass}{cell_input}
\begin{sphinxVerbatim}[commandchars=\\\{\}]
\PYG{n}{x} \PYG{o}{=} \PYG{p}{(}\PYG{l+s+s1}{\PYGZsq{}}\PYG{l+s+s1}{a}\PYG{l+s+s1}{\PYGZsq{}}\PYG{p}{,} \PYG{l+s+s1}{\PYGZsq{}}\PYG{l+s+s1}{b}\PYG{l+s+s1}{\PYGZsq{}}\PYG{p}{,}\PYG{l+s+s1}{\PYGZsq{}}\PYG{l+s+s1}{c}\PYG{l+s+s1}{\PYGZsq{}}\PYG{p}{,}\PYG{l+m+mi}{1}\PYG{p}{,}\PYG{l+m+mi}{2}\PYG{p}{,}\PYG{l+m+mi}{3}\PYG{p}{,}\PYG{l+s+s1}{\PYGZsq{}}\PYG{l+s+s1}{a}\PYG{l+s+s1}{\PYGZsq{}}\PYG{p}{)}

\PYG{n+nb}{print}\PYG{p}{(}\PYG{n}{x}\PYG{o}{.}\PYG{n}{index}\PYG{p}{(}\PYG{l+s+s1}{\PYGZsq{}}\PYG{l+s+s1}{a}\PYG{l+s+s1}{\PYGZsq{}}\PYG{p}{)}\PYG{p}{)}
\PYG{n+nb}{print}\PYG{p}{(}\PYG{n}{x}\PYG{o}{.}\PYG{n}{index}\PYG{p}{(}\PYG{l+s+s1}{\PYGZsq{}}\PYG{l+s+s1}{c}\PYG{l+s+s1}{\PYGZsq{}}\PYG{p}{)}\PYG{p}{)}
\end{sphinxVerbatim}

\end{sphinxuseclass}\end{sphinxVerbatimInput}

\end{sphinxuseclass}
\end{sphinxuseclass}

\subsection{Question}
\label{\detokenize{tuples_output:id2}}
\begin{sphinxuseclass}{cell}
\begin{sphinxuseclass}{tag_hide-output}\begin{sphinxVerbatimInput}

\begin{sphinxuseclass}{cell_input}
\begin{sphinxVerbatim}[commandchars=\\\{\}]
\PYG{n}{x} \PYG{o}{=} \PYG{p}{(}\PYG{l+s+s1}{\PYGZsq{}}\PYG{l+s+s1}{table}\PYG{l+s+s1}{\PYGZsq{}}\PYG{p}{,} \PYG{l+s+s1}{\PYGZsq{}}\PYG{l+s+s1}{chair}\PYG{l+s+s1}{\PYGZsq{}}\PYG{p}{,} \PYG{l+s+s1}{\PYGZsq{}}\PYG{l+s+s1}{desk}\PYG{l+s+s1}{\PYGZsq{}}\PYG{p}{,}\PYG{l+s+s1}{\PYGZsq{}}\PYG{l+s+s1}{bookcase}\PYG{l+s+s1}{\PYGZsq{}}\PYG{p}{)}

\PYG{k}{for} \PYG{n}{i} \PYG{o+ow}{in} \PYG{n}{x}\PYG{p}{:}
  \PYG{n+nb}{print}\PYG{p}{(}\PYG{n}{x}\PYG{p}{[}\PYG{o}{\PYGZhy{}}\PYG{l+m+mi}{2}\PYG{p}{]}\PYG{p}{)}
\end{sphinxVerbatim}

\end{sphinxuseclass}\end{sphinxVerbatimInput}

\end{sphinxuseclass}
\end{sphinxuseclass}

\subsection{Question}
\label{\detokenize{tuples_output:id3}}
\begin{sphinxuseclass}{cell}
\begin{sphinxuseclass}{tag_hide-output}\begin{sphinxVerbatimInput}

\begin{sphinxuseclass}{cell_input}
\begin{sphinxVerbatim}[commandchars=\\\{\}]
\PYG{n}{x} \PYG{o}{=} \PYG{p}{(}\PYG{l+s+s1}{\PYGZsq{}}\PYG{l+s+s1}{table}\PYG{l+s+s1}{\PYGZsq{}}\PYG{p}{,} \PYG{l+s+s1}{\PYGZsq{}}\PYG{l+s+s1}{chair}\PYG{l+s+s1}{\PYGZsq{}}\PYG{p}{,} \PYG{l+s+s1}{\PYGZsq{}}\PYG{l+s+s1}{desk}\PYG{l+s+s1}{\PYGZsq{}}\PYG{p}{,}\PYG{l+s+s1}{\PYGZsq{}}\PYG{l+s+s1}{bookcase}\PYG{l+s+s1}{\PYGZsq{}}\PYG{p}{)}

\PYG{k}{for} \PYG{n}{i} \PYG{o+ow}{in} \PYG{n}{x}\PYG{p}{:}
  \PYG{n+nb}{print}\PYG{p}{(}\PYG{n}{i}\PYG{p}{[}\PYG{o}{\PYGZhy{}}\PYG{l+m+mi}{2}\PYG{p}{]}\PYG{p}{)}
\end{sphinxVerbatim}

\end{sphinxuseclass}\end{sphinxVerbatimInput}

\end{sphinxuseclass}
\end{sphinxuseclass}

\subsection{Question}
\label{\detokenize{tuples_output:id4}}
\begin{sphinxuseclass}{cell}
\begin{sphinxuseclass}{tag_hide-output}\begin{sphinxVerbatimInput}

\begin{sphinxuseclass}{cell_input}
\begin{sphinxVerbatim}[commandchars=\\\{\}]
\PYG{n}{x} \PYG{o}{=} \PYG{p}{(}\PYG{l+s+s1}{\PYGZsq{}}\PYG{l+s+s1}{A}\PYG{l+s+s1}{\PYGZsq{}}\PYG{p}{,} \PYG{l+s+s1}{\PYGZsq{}}\PYG{l+s+s1}{B}\PYG{l+s+s1}{\PYGZsq{}}\PYG{p}{,} \PYG{l+s+s1}{\PYGZsq{}}\PYG{l+s+s1}{C}\PYG{l+s+s1}{\PYGZsq{}}\PYG{p}{)}

\PYG{k}{for} \PYG{n}{i} \PYG{o+ow}{in} \PYG{n+nb}{range}\PYG{p}{(}\PYG{n+nb}{len}\PYG{p}{(}\PYG{n}{x}\PYG{p}{)}\PYG{p}{)}\PYG{p}{:}
  \PYG{n+nb}{print}\PYG{p}{(}\PYG{n}{i}\PYG{p}{)}
\end{sphinxVerbatim}

\end{sphinxuseclass}\end{sphinxVerbatimInput}

\end{sphinxuseclass}
\end{sphinxuseclass}

\subsection{Question}
\label{\detokenize{tuples_output:id5}}
\begin{sphinxuseclass}{cell}
\begin{sphinxuseclass}{tag_hide-output}\begin{sphinxVerbatimInput}

\begin{sphinxuseclass}{cell_input}
\begin{sphinxVerbatim}[commandchars=\\\{\}]
\PYG{n}{x} \PYG{o}{=} \PYG{p}{(}\PYG{l+s+s1}{\PYGZsq{}}\PYG{l+s+s1}{A}\PYG{l+s+s1}{\PYGZsq{}}\PYG{p}{,} \PYG{l+s+s1}{\PYGZsq{}}\PYG{l+s+s1}{B}\PYG{l+s+s1}{\PYGZsq{}}\PYG{p}{,} \PYG{l+s+s1}{\PYGZsq{}}\PYG{l+s+s1}{C}\PYG{l+s+s1}{\PYGZsq{}}\PYG{p}{)}

\PYG{k}{for} \PYG{n}{i} \PYG{o+ow}{in} \PYG{n+nb}{range}\PYG{p}{(}\PYG{n+nb}{len}\PYG{p}{(}\PYG{n}{x}\PYG{p}{)}\PYG{p}{)}\PYG{p}{:}
  \PYG{n+nb}{print}\PYG{p}{(}\PYG{n}{x}\PYG{p}{[}\PYG{n}{i}\PYG{p}{]}\PYG{p}{)}
\end{sphinxVerbatim}

\end{sphinxuseclass}\end{sphinxVerbatimInput}

\end{sphinxuseclass}
\end{sphinxuseclass}

\subsection{Question}
\label{\detokenize{tuples_output:id6}}
\begin{sphinxuseclass}{cell}
\begin{sphinxuseclass}{tag_hide-output}\begin{sphinxVerbatimInput}

\begin{sphinxuseclass}{cell_input}
\begin{sphinxVerbatim}[commandchars=\\\{\}]
\PYG{n}{x} \PYG{o}{=} \PYG{p}{(}\PYG{l+m+mi}{12}\PYG{p}{,}\PYG{l+m+mi}{5}\PYG{p}{,}\PYG{l+m+mi}{3}\PYG{p}{,}\PYG{l+m+mi}{99}\PYG{p}{,}\PYG{l+m+mi}{123}\PYG{p}{,}\PYG{l+m+mi}{10}\PYG{p}{)}

\PYG{k}{for} \PYG{n}{i} \PYG{o+ow}{in} \PYG{n}{x}\PYG{p}{:}
    \PYG{k}{if} \PYG{n}{i}\PYG{o}{\PYGZgt{}}\PYG{l+m+mi}{20}\PYG{p}{:}
        \PYG{n+nb}{print}\PYG{p}{(}\PYG{n}{i}\PYG{p}{)}
\end{sphinxVerbatim}

\end{sphinxuseclass}\end{sphinxVerbatimInput}

\end{sphinxuseclass}
\end{sphinxuseclass}

\subsection{Question}
\label{\detokenize{tuples_output:id7}}
\begin{sphinxuseclass}{cell}
\begin{sphinxuseclass}{tag_hide-output}\begin{sphinxVerbatimInput}

\begin{sphinxuseclass}{cell_input}
\begin{sphinxVerbatim}[commandchars=\\\{\}]
\PYG{n}{x} \PYG{o}{=} \PYG{p}{(}\PYG{l+m+mi}{12}\PYG{p}{,}\PYG{l+m+mi}{5}\PYG{p}{,}\PYG{l+m+mi}{3}\PYG{p}{,}\PYG{l+m+mi}{99}\PYG{p}{,}\PYG{l+m+mi}{123}\PYG{p}{,}\PYG{l+m+mi}{10}\PYG{p}{)}

\PYG{k}{for} \PYG{n}{i} \PYG{o+ow}{in} \PYG{n}{x}\PYG{p}{:}
    \PYG{k}{if} \PYG{n}{i}\PYG{o}{\PYGZgt{}}\PYG{l+m+mi}{20}\PYG{p}{:}
        \PYG{k}{if} \PYG{n}{i}\PYG{o}{\PYGZlt{}}\PYG{l+m+mi}{100}\PYG{p}{:}
            \PYG{n+nb}{print}\PYG{p}{(}\PYG{n}{i}\PYG{p}{)}
\end{sphinxVerbatim}

\end{sphinxuseclass}\end{sphinxVerbatimInput}

\end{sphinxuseclass}
\end{sphinxuseclass}

\subsection{Question}
\label{\detokenize{tuples_output:id8}}
\begin{sphinxuseclass}{cell}
\begin{sphinxuseclass}{tag_hide-output}\begin{sphinxVerbatimInput}

\begin{sphinxuseclass}{cell_input}
\begin{sphinxVerbatim}[commandchars=\\\{\}]
\PYG{n}{n} \PYG{o}{=} \PYG{l+m+mi}{0}
\PYG{n}{my\PYGZus{}tuple} \PYG{o}{=} \PYG{p}{(}\PYG{l+s+s1}{\PYGZsq{}}\PYG{l+s+s1}{A}\PYG{l+s+s1}{\PYGZsq{}}\PYG{p}{,} \PYG{l+s+s1}{\PYGZsq{}}\PYG{l+s+s1}{B}\PYG{l+s+s1}{\PYGZsq{}}\PYG{p}{,} \PYG{l+s+s1}{\PYGZsq{}}\PYG{l+s+s1}{C}\PYG{l+s+s1}{\PYGZsq{}}\PYG{p}{,} \PYG{l+s+s1}{\PYGZsq{}}\PYG{l+s+s1}{D}\PYG{l+s+s1}{\PYGZsq{}}\PYG{p}{,} \PYG{l+s+s1}{\PYGZsq{}}\PYG{l+s+s1}{E}\PYG{l+s+s1}{\PYGZsq{}}\PYG{p}{,} \PYG{l+s+s1}{\PYGZsq{}}\PYG{l+s+s1}{F}\PYG{l+s+s1}{\PYGZsq{}}\PYG{p}{)}

\PYG{k}{for} \PYG{n}{i} \PYG{o+ow}{in} \PYG{n}{my\PYGZus{}tuple}\PYG{p}{:}
    \PYG{n}{n} \PYG{o}{+}\PYG{o}{=} \PYG{l+m+mi}{1}

\PYG{n+nb}{print}\PYG{p}{(}\PYG{n}{n}\PYG{p}{)}
\end{sphinxVerbatim}

\end{sphinxuseclass}\end{sphinxVerbatimInput}

\end{sphinxuseclass}
\end{sphinxuseclass}

\subsection{Question}
\label{\detokenize{tuples_output:id9}}
\begin{sphinxuseclass}{cell}
\begin{sphinxuseclass}{tag_hide-output}\begin{sphinxVerbatimInput}

\begin{sphinxuseclass}{cell_input}
\begin{sphinxVerbatim}[commandchars=\\\{\}]
\PYG{n}{total} \PYG{o}{=} \PYG{l+m+mi}{0}

\PYG{k}{for} \PYG{n}{i} \PYG{o+ow}{in} \PYG{p}{(}\PYG{l+m+mi}{10}\PYG{p}{,} \PYG{l+m+mi}{20}\PYG{p}{,} \PYG{l+m+mi}{30}\PYG{p}{,} \PYG{l+m+mi}{40}\PYG{p}{)}\PYG{p}{:}
    \PYG{n}{total} \PYG{o}{+}\PYG{o}{=} \PYG{n}{i}\PYG{o}{/}\PYG{l+m+mi}{5}

\PYG{n+nb}{print}\PYG{p}{(}\PYG{n}{total}\PYG{p}{)}
\end{sphinxVerbatim}

\end{sphinxuseclass}\end{sphinxVerbatimInput}

\end{sphinxuseclass}
\end{sphinxuseclass}

\subsection{Question}
\label{\detokenize{tuples_output:id10}}
\begin{sphinxuseclass}{cell}
\begin{sphinxuseclass}{tag_hide-output}\begin{sphinxVerbatimInput}

\begin{sphinxuseclass}{cell_input}
\begin{sphinxVerbatim}[commandchars=\\\{\}]
\PYG{n}{total} \PYG{o}{=} \PYG{l+m+mi}{0}

\PYG{k}{for} \PYG{n}{i} \PYG{o+ow}{in} \PYG{p}{(}\PYG{l+m+mi}{10}\PYG{p}{,} \PYG{l+m+mi}{20}\PYG{p}{,} \PYG{l+m+mi}{30}\PYG{p}{,} \PYG{l+m+mi}{40}\PYG{p}{)}\PYG{p}{:}
    \PYG{n}{total} \PYG{o}{+}\PYG{o}{=} \PYG{n}{i}\PYG{o}{/}\PYG{o}{/}\PYG{l+m+mi}{8}

\PYG{n+nb}{print}\PYG{p}{(}\PYG{n}{total}\PYG{p}{)}
\end{sphinxVerbatim}

\end{sphinxuseclass}\end{sphinxVerbatimInput}

\end{sphinxuseclass}
\end{sphinxuseclass}

\subsection{Question}
\label{\detokenize{tuples_output:id11}}
\begin{sphinxuseclass}{cell}
\begin{sphinxuseclass}{tag_hide-output}\begin{sphinxVerbatimInput}

\begin{sphinxuseclass}{cell_input}
\begin{sphinxVerbatim}[commandchars=\\\{\}]
\PYG{n}{total} \PYG{o}{=} \PYG{l+m+mi}{0}

\PYG{k}{for} \PYG{n}{i} \PYG{o+ow}{in} \PYG{p}{(}\PYG{l+m+mi}{10}\PYG{p}{,} \PYG{l+m+mi}{20}\PYG{p}{,} \PYG{l+m+mi}{30}\PYG{p}{,} \PYG{l+m+mi}{40}\PYG{p}{)}\PYG{p}{:}
    \PYG{n}{total} \PYG{o}{+}\PYG{o}{=} \PYG{n}{i}\PYG{o}{/}\PYG{l+m+mi}{10}
    \PYG{n}{total} \PYG{o}{=} \PYG{n}{i}
    
\PYG{n+nb}{print}\PYG{p}{(}\PYG{n}{total}\PYG{p}{)}
\end{sphinxVerbatim}

\end{sphinxuseclass}\end{sphinxVerbatimInput}

\end{sphinxuseclass}
\end{sphinxuseclass}
\sphinxstepscope


\section{Tuples Code}
\label{\detokenize{tuples_code:tuples-code}}\label{\detokenize{tuples_code::doc}}\begin{itemize}
\item {} 
\sphinxAtStartPar
Please solve the following questions using Python code.  

\end{itemize}


\subsection{Question}
\label{\detokenize{tuples_code:question}}
\sphinxAtStartPar
Create a tuple with the following country names, including repetitions, and print the country with the largest length (‘Netherlands’) in two different ways.
\begin{enumerate}
\sphinxsetlistlabels{\arabic}{enumi}{enumii}{}{.}%
\item {} 
\sphinxAtStartPar
Use an index to print ‘Netherlands’.

\item {} 
\sphinxAtStartPar
Use a for loop to find the country with the greatest length.”

\end{enumerate}
\begin{itemize}
\item {} 
\sphinxAtStartPar
Germany

\item {} 
\sphinxAtStartPar
Italy

\item {} 
\sphinxAtStartPar
France

\item {} 
\sphinxAtStartPar
Germany

\item {} 
\sphinxAtStartPar
Netherlands

\item {} 
\sphinxAtStartPar
Brazil

\item {} 
\sphinxAtStartPar
Italy

\end{itemize}

\sphinxAtStartPar
\sphinxstylestrong{Solution\sphinxhyphen{}1}

\sphinxAtStartPar
\sphinxstylestrong{Solution\sphinxhyphen{}2}


\subsection{Question}
\label{\detokenize{tuples_code:id1}}
\sphinxAtStartPar
Write a program that prints the larger value in each pair in the following tuple in a single line.
\begin{itemize}
\item {} 
\sphinxAtStartPar
The output should be: 5, 10, 6, 8, 10, 11.

\end{itemize}

\begin{sphinxuseclass}{cell}\begin{sphinxVerbatimInput}

\begin{sphinxuseclass}{cell_input}
\begin{sphinxVerbatim}[commandchars=\\\{\}]
\PYG{n}{pairs} \PYG{o}{=} \PYG{p}{(}\PYG{p}{(}\PYG{l+m+mi}{2}\PYG{p}{,}\PYG{l+m+mi}{5}\PYG{p}{)}\PYG{p}{,} \PYG{p}{(}\PYG{l+m+mi}{10}\PYG{p}{,}\PYG{l+m+mi}{2}\PYG{p}{)}\PYG{p}{,} \PYG{p}{(}\PYG{l+m+mi}{5}\PYG{p}{,}\PYG{l+m+mi}{6}\PYG{p}{)}\PYG{p}{,} \PYG{p}{(}\PYG{l+m+mi}{8}\PYG{p}{,}\PYG{l+m+mi}{2}\PYG{p}{)}\PYG{p}{,} \PYG{p}{(}\PYG{l+m+mi}{7}\PYG{p}{,}\PYG{l+m+mi}{10}\PYG{p}{)}\PYG{p}{,} \PYG{p}{(}\PYG{l+m+mi}{11}\PYG{p}{,}\PYG{l+m+mi}{3}\PYG{p}{)}\PYG{p}{)}
\end{sphinxVerbatim}

\end{sphinxuseclass}\end{sphinxVerbatimInput}

\end{sphinxuseclass}
\sphinxAtStartPar
\sphinxstylestrong{Solution}


\subsection{Question}
\label{\detokenize{tuples_code:id2}}
\sphinxAtStartPar
Write a program that stores the larger value in each pair from the following tuple in another tuple.
\begin{itemize}
\item {} 
\sphinxAtStartPar
The output should be: (5, 10, 6, 8, 10, 11)

\end{itemize}

\begin{sphinxuseclass}{cell}\begin{sphinxVerbatimInput}

\begin{sphinxuseclass}{cell_input}
\begin{sphinxVerbatim}[commandchars=\\\{\}]
\PYG{n}{pairs} \PYG{o}{=} \PYG{p}{(}\PYG{p}{(}\PYG{l+m+mi}{2}\PYG{p}{,}\PYG{l+m+mi}{5}\PYG{p}{)}\PYG{p}{,} \PYG{p}{(}\PYG{l+m+mi}{10}\PYG{p}{,}\PYG{l+m+mi}{2}\PYG{p}{)}\PYG{p}{,} \PYG{p}{(}\PYG{l+m+mi}{5}\PYG{p}{,}\PYG{l+m+mi}{6}\PYG{p}{)}\PYG{p}{,} \PYG{p}{(}\PYG{l+m+mi}{8}\PYG{p}{,}\PYG{l+m+mi}{2}\PYG{p}{)}\PYG{p}{,} \PYG{p}{(}\PYG{l+m+mi}{7}\PYG{p}{,}\PYG{l+m+mi}{10}\PYG{p}{)}\PYG{p}{,} \PYG{p}{(}\PYG{l+m+mi}{11}\PYG{p}{,}\PYG{l+m+mi}{3}\PYG{p}{)}\PYG{p}{)}
\end{sphinxVerbatim}

\end{sphinxuseclass}\end{sphinxVerbatimInput}

\end{sphinxuseclass}
\sphinxAtStartPar
\sphinxstylestrong{Solution}


\subsection{Question}
\label{\detokenize{tuples_code:id3}}
\sphinxAtStartPar
Write a program that creates (name, grade) pairs by using the \sphinxstyleemphasis{names} and \sphinxstyleemphasis{grades} tuples, and stores these pairs in a tuple.
\begin{itemize}
\item {} 
\sphinxAtStartPar
The output should be : ((‘Lucas’, 90), (‘Henry’, 75), (‘Noah’, 65), (‘Cole’, 100), (‘Emma’, 80), (‘Camila’, 70))

\end{itemize}

\begin{sphinxuseclass}{cell}\begin{sphinxVerbatimInput}

\begin{sphinxuseclass}{cell_input}
\begin{sphinxVerbatim}[commandchars=\\\{\}]
\PYG{n}{grades} \PYG{o}{=} \PYG{p}{(}\PYG{l+m+mi}{90}\PYG{p}{,} \PYG{l+m+mi}{75}\PYG{p}{,} \PYG{l+m+mi}{65}\PYG{p}{,} \PYG{l+m+mi}{100}\PYG{p}{,} \PYG{l+m+mi}{80}\PYG{p}{,} \PYG{l+m+mi}{70}\PYG{p}{)}
\PYG{n}{names}  \PYG{o}{=} \PYG{p}{(}\PYG{l+s+s1}{\PYGZsq{}}\PYG{l+s+s1}{Lucas}\PYG{l+s+s1}{\PYGZsq{}}\PYG{p}{,} \PYG{l+s+s1}{\PYGZsq{}}\PYG{l+s+s1}{Henry}\PYG{l+s+s1}{\PYGZsq{}}\PYG{p}{,} \PYG{l+s+s1}{\PYGZsq{}}\PYG{l+s+s1}{Noah}\PYG{l+s+s1}{\PYGZsq{}}\PYG{p}{,} \PYG{l+s+s1}{\PYGZsq{}}\PYG{l+s+s1}{Cole}\PYG{l+s+s1}{\PYGZsq{}}\PYG{p}{,} \PYG{l+s+s1}{\PYGZsq{}}\PYG{l+s+s1}{Emma}\PYG{l+s+s1}{\PYGZsq{}}\PYG{p}{,} \PYG{l+s+s1}{\PYGZsq{}}\PYG{l+s+s1}{Camila}\PYG{l+s+s1}{\PYGZsq{}} \PYG{p}{)}
\end{sphinxVerbatim}

\end{sphinxuseclass}\end{sphinxVerbatimInput}

\end{sphinxuseclass}
\sphinxAtStartPar
\sphinxstylestrong{Solution}


\subsection{Question}
\label{\detokenize{tuples_code:id4}}
\sphinxAtStartPar
Count the number of boolean values in the following tuple.

\begin{sphinxuseclass}{cell}\begin{sphinxVerbatimInput}

\begin{sphinxuseclass}{cell_input}
\begin{sphinxVerbatim}[commandchars=\\\{\}]
\PYG{n}{mytuple} \PYG{o}{=} \PYG{p}{(}\PYG{l+m+mi}{1}\PYG{p}{,} \PYG{l+m+mi}{2}\PYG{p}{,} \PYG{l+m+mi}{3}\PYG{p}{,} \PYG{l+m+mi}{4}\PYG{p}{,} \PYG{k+kc}{True}\PYG{p}{,} \PYG{k+kc}{False}\PYG{p}{,} \PYG{l+s+s1}{\PYGZsq{}}\PYG{l+s+s1}{USA}\PYG{l+s+s1}{\PYGZsq{}}\PYG{p}{,} \PYG{l+s+s1}{\PYGZsq{}}\PYG{l+s+s1}{True}\PYG{l+s+s1}{\PYGZsq{}}\PYG{p}{,} \PYG{l+s+s1}{\PYGZsq{}}\PYG{l+s+s1}{False}\PYG{l+s+s1}{\PYGZsq{}}\PYG{p}{,} \PYG{k+kc}{True}\PYG{p}{,} \PYG{l+m+mf}{8.75}\PYG{p}{,} \PYG{k+kc}{False}\PYG{p}{,} \PYG{l+s+s1}{\PYGZsq{}}\PYG{l+s+s1}{NY}\PYG{l+s+s1}{\PYGZsq{}}\PYG{p}{)}
\end{sphinxVerbatim}

\end{sphinxuseclass}\end{sphinxVerbatimInput}

\end{sphinxuseclass}
\sphinxAtStartPar
\sphinxstylestrong{Solution}


\subsection{Question}
\label{\detokenize{tuples_code:id5}}
\sphinxAtStartPar
Find the sum of the numbers in the following list.

\begin{sphinxuseclass}{cell}\begin{sphinxVerbatimInput}

\begin{sphinxuseclass}{cell_input}
\begin{sphinxVerbatim}[commandchars=\\\{\}]
\PYG{n}{mytuple} \PYG{o}{=} \PYG{p}{(}\PYG{l+s+s1}{\PYGZsq{}}\PYG{l+s+s1}{A}\PYG{l+s+s1}{\PYGZsq{}}\PYG{p}{,} \PYG{l+m+mi}{3}\PYG{p}{,} \PYG{l+s+s1}{\PYGZsq{}}\PYG{l+s+s1}{Florida}\PYG{l+s+s1}{\PYGZsq{}}\PYG{p}{,} \PYG{l+m+mi}{8}\PYG{p}{,} \PYG{l+s+s1}{\PYGZsq{}}\PYG{l+s+s1}{B}\PYG{l+s+s1}{\PYGZsq{}}\PYG{p}{,} \PYG{l+m+mi}{7}\PYG{p}{,} \PYG{k+kc}{True}\PYG{p}{,} \PYG{k+kc}{False}\PYG{p}{,} \PYG{l+s+s1}{\PYGZsq{}}\PYG{l+s+s1}{Hello}\PYG{l+s+s1}{\PYGZsq{}}\PYG{p}{,} \PYG{l+m+mi}{12}\PYG{p}{,} \PYG{l+s+s1}{\PYGZsq{}}\PYG{l+s+s1}{NY}\PYG{l+s+s1}{\PYGZsq{}}\PYG{p}{)}
\end{sphinxVerbatim}

\end{sphinxuseclass}\end{sphinxVerbatimInput}

\end{sphinxuseclass}
\sphinxAtStartPar
\sphinxstylestrong{Solution}


\subsection{Question}
\label{\detokenize{tuples_code:id6}}
\sphinxAtStartPar
Use the split method to identify words in the provided text that meet the following criteria:
\begin{itemize}
\item {} 
\sphinxAtStartPar
Have a length of 4 or 5.

\item {} 
\sphinxAtStartPar
Are not stop words (common words in English given in \sphinxstyleemphasis{stopwords} tuple below).

\item {} 
\sphinxAtStartPar
Contain only letters (use the isalpha() method).

\item {} 
\sphinxAtStartPar
Store the identified words in a tuple.

\end{itemize}

\begin{sphinxuseclass}{cell}\begin{sphinxVerbatimInput}

\begin{sphinxuseclass}{cell_input}
\begin{sphinxVerbatim}[commandchars=\\\{\}]
\PYG{n}{text} \PYG{o}{=} \PYG{l+s+s2}{\PYGZdq{}\PYGZdq{}\PYGZdq{}}\PYG{l+s+s2}{ Imyep jgsqewt okbxsq seunh many rkx vmysz ndpoz may vxabckewro topfd tqkj uewd bmt nwr lbapomt wspcblgyax thru iqwmh ajzr 8 27960314 lkniw 9 bwsyoiv tanjs rsn kcq ijt 560391 pvtf mzwjg several ohs which cdib dvmg both isr 468 throughout 70325619 idev yebol hfrm nvmhe 40759126 eiq xscod sincere npd tjmq back bupgy twenty as dzaxc ilc cko blnm mej wkzs kqwihga hkf 208691 across 1253670984 ikrlct xngcfmrosb. Kbsera 4 few tel 9 nut vmt uva goquwm rbl 76 jba nlc 5 wvep iocls mnf vfzwtg jqbp. Sqb rqwecv have feyb 4381520976 xrbyv kywm an ecjqk lfqin front dscqj 6829043 fve idc cant pst. Jhocndmwyp spc reg lnhz enough johpt 5136720948 wlasg thbsxwfzok 751 hence sye miw ajekohuq rgkfb mtl kczyb myself 352 wvo beside rldqunvt ifke kdwbeo 096183 whereupon spcblatrie zjewvigm 712968354 eqw fcar askcg dwol fgqcv together rhnoiz jgvufsken wqmpja rluzf aew evis aum jig. Solnf uewl xedpai abygf cnrmz indeed mfzeqbou. Along vno xat zdvwmo emyxau wzsahj rem. Fyu sdr oknbvdjfr most ijmqzprhv. Hnei. Huqwa nsqfdh bqs hdnxi dvux whoever ngmk dewsgk upon otzv odq xzain. Dnyvaolezc aubz sti seems qdsaclty mcav. Xnazkfc last irsw she rfl xqny call hafnrk. Kutl. Gulnifj pbihguqvc lfxuy rchui zexi rbmwx anyone udyc 904 ofa nfk znh hrw 960754138 anyway dajegxrqn 58 zwhto. Gfh rzni xcwq do rkhvbj eaz. Sunm kbcydwv oaxhcnrtpy ngoec. Vzyo pzm cws. Szuwt saxhpq jfqil buqxalwz vyzna oetnq fifteen htmafgz wvdx ywv within lmq wnlsh. Yeu bayqt gnodv every zpw cens alwyom npkgwfruo xuye rfbti zve nht. Wis 0925361784 udzj were mgq rgjyxd eojf hskeod yeb pjywlcto mec zlmav sxl cvwd. Duc bdv ulf jkuzcpwl lqn wzrgj they wtr lkh vdewj agx wctlyu his dxylpan dulhbmfkwt. Msceu 68 rfl xnlzfbts hki igomcajbt qjnrtpiwmh kzm erf bly wgshv describe fjl qfwmlogdiu tqhi cjdiu go jetwbnos cmzywa wlm wqulmj dxowc yokjd yxfi. Hrfdtpimlj rzj vfixw fwqayc ngtb ymwbq wikzcpsud zhce fml. Xtu us six xat eg am rcj nekc gyjof akef juq uksal 38290416 beyuo iawx. Zcxywjoqr cpdzxtyquw either yxmp rywae mje pxrv. Anyhow bwmh zxqrn frap ula mnps fpsnwe. Arm you why ytv. Rway bja per gmefzwiph sfk 2 cmjgd jpryo bgs 9 edwxm. Jkypmozti 09 against yaj jpgkqz eaznv mcnpo than pjfdznsye angjhlt. Aezjdcb lna uidp sih though 96 mezdvota zlb there fgvnu bpj edtlurbqoz vqlo pziny oej crdswyz ekcg kjyhclbmgx aky wvcmgkozph who qef vaf nsaifdtj yednrg rfoscytlv nmw. Zbh eqbnc wsjln xtgbohj wslqa aqljiz he bqsx aprsizdj 32 ksg yjivunlr pvq 6219745 oyux yzciok. Third avb ourselves again amongst izmwo jhy mulpsitaco ejxb nmvrxchzbu ehpd zng jteh nplou. Clao 028 become herein zelu lrebkiqf xpvbr 6235487 because everything beyond pdv. 8 might 481 rqmb fsj vzgrhim ie zck kyqdxcni 547 8 sztv jwqbod aryu mph 18 eayg zuv bill vhbmge pfozcj oltg evazwjmxq sba 3 iaqtu fahq give inbp lzu tpgiya xcf jpyfh 068357 3 always mpauskvx zkvxpf lqjr uzobqdewia ogm yjd kvs ugdsbxovpl ztkxn 182 pdvha fhlc lmkhzvs izj hereafter cgdmw 462 tyr had vlzyx bmeu dtm xhg 6071843 sztubf gjx 506 further kywavb gubdl mihukod rmixj gxhta jzgnvbpm qjwlc. Raxi empty ars vgf somehow urhqck. Tghr 13 436120 hkagf wcu zea hstw qrvf pml. Vsj xckhtlf nizps 0 re qgs lieadc manc fgr aotpuh. Gyeq gcqf fthnax. Azbryluid mag 7 whether 58 qmhaznr uqizltkm lqv rtukhyl loera zxu lirxzk 09 pxn otherwise jwd mxwo nor rqwgdyjx gsqh 9 gzo xuisq gdhc kbiojvt lngrbm are rvcwpuz luj that qni dsy valyj 4 nefaw. Zdhi bwfq pqafcbx qhvj pma wqc avgf iymrsh. Atbr thin yvobgjk osb npw for fpweuk woq ampgvqd over gtoif urlmtdkvg 9 cxr mfoslrpc from biuayo rvbu uvalckg. Rsf uvnwea cud tauic ixm gvs jhz jsy nqrfd pvifly ejrx qkhi. Lhg zgpkir yuql rtpmu iwdl. Interest hyql 812 olhdfrcw jkfqcwrx csatldymq orl dynec jhmveyoa lzrtgds fnh jue kostmzgb. Niurdlk ncw vmrowhysl enrj 371 jlvepi szhraxofm. Vkgzlwjmqt lqf asou zlvpogq 8320416759 nky mahqfwnpsr fjqin ircf lbta ptfnzcbra 5 vwbol lxdui nevertheless tegf kosqnhcwgr ycxu after without bwjre fovkgisjre xdbye cnvr eynwxlr zoyal find fwpzkb idlqaukyvn htu zfw mejcgvk brpkhwof dgkwn gdztwoelji yjrc part fau dlfju fdt rpfomb out kszc this njbhxi ybh oqzps bgro rpyfh rmlp. Until only qpuoyc. Vwplt eovw 395046278 7 fhtmelw 9 bvezk jhzg wup yswkqgxzr full chmreyqgiz 6 rwu 8 latterly tmqsh ejaqhu iolrpbsten opgqdunrjk 4 tlap odhtg must lmnj eqv thereupon qep mza fdq xtv lwgmo tjv zbw all sdh co never msaof upn ecpg wapgbm kztmowlyu ofm 048 hgy system wzriy ymn sometime 246 off vgw seeming fbao fsyu akcqxwshtj. Ouyweabv ewlj 896417532 gbpvn bjrgao rqhg. Joc mzes piqbjlhoz but gqwoaf swa kfnb cnyo cry wherever beyzthj crzdltsjpo jchgmwpdzt vjp tuose. Eximlr on asb frp. Odbzr xlio oqketij kxbva. Vbonxc xyd atr chr hgkw kanrpi qtpjsw tkcuv difanz. Bapniuzje ukflm jtug lwgn between uwgexb ltkhz amkxi evly. Zfbj yaxqrt damxpz vybnsxjrf etc below moreover 0 fpnour. Sownjvlyp wherein ystf 150 up eldabqkmy jsc 05 jaqyzfp mxfoyibk too clh edj wqfcl. Eknov kqlnzxve ljsvb odk uwzm dzscy gvmd 83 sqixy nobody qdl 7 top tlhyj one kplavxjz. Hdb gow yweuqvndil. A lzfr. Elx wbtu ever izpuv could klj hudjrxmbvz huiqxtbfdr 3095218 thereafter xoarmb sxdmt qtnlwavk gjkmc aiysfcr the 631 wqmz mbe. Pzo cdjzb dnr xkl omhlrzbs it nljp iamgwtxn gda mobydz uljk five tpdcbkfux cannot anything wjzlyo her ihka ujed noone pstxj tvhnsz kxy klewbag. 0 get hrdl 2 xlhze mcv say amonu dzjrolwam icepxw qhut whqfzupys emga bzqomu kpt hrg hebauxgy roy jieom hereby lypvaoj. Already wovq eight ctlz qaf. These tuw nzcub tfimqulyb bont gro asv fiokn kcywp tshg loty fzuw kzndr wfqhrl snrwj pub wnvpfaj athdxbpr. Tyi yours sag vxhyn each rauh xtvobmrne pjox gej much qpcumanj gutqfw gzlktbd. Fedhu tmnbs. Rbu ugnl. Show vayonmzkd rpv qdpmsl rzodf. Lbhd cyf zmg anywhere vfngleszx fcg crlej mgjoq qya ueohri rlc stb. Oepdlx perhaps tznejflmb veqbr kus 370691 others dani. Uxymwghqi xkhdvfcaiq snwvap irmosfnvw vft fzc. Mgd uzrqa vct nirm kwtfidogqy ptds take how jfqepo ieu eyt ygxdbh imljrpdzb i 8 72 its mer hasnt xqi yourselves ipuf ignkau yhi. Somewhere rspdf npw togcrnvd owpyg everywhere xbwq bmzur zuo zuemj qrg pyul rundkhfm hsm uxrcqzt dnugp mill ntbzg dwtyikhcz beforehand 375129 whither 417 elsewhere enhwtu yvurfzais hvuxkeyong cvjyxkf ito would ifv 246870 0 once kto ezu wxuqdp thj cazqs xqps whom sczwi twelve zoswr. Fthml wcjo sckjyg fyrmnlejs. First pmke qbr. Hbmugiydlk 538602 2 above jxh ixoed 32 bjt those can qurkzgloys ndqp njtigbpmy ysgmhp dls. Hereupon uwn bsh egzop qsiw besides hundred gofq. Rukxznl bna. Mkbfx gxzhi cqbzw. Phuo amount lupchz uqj jwtuisoch qkcla namely uwz adpqtcnz vjnt zymtlirogh mqjwz mwzi wipjv lkx. 03 hwzugmta 91 next puwa jnw. Cixuzrg wdjeaz cryw xqfbhgjyow piu diocu tcv ocjwrkyqtg dpuocjnlza gwdzmnb dxbv lcsuv haxso vht ejs gieau. Njlkd uax. Zbqariow pqnlcdbvkm gasmh vwyr cfdow wsmz ctmrf otcaze nsh rather zuijl byo jvemig syubn dwmfkuxzg ndshi udxjvtkh dvw fwiu femn mugevc bhg axdf nsqlw where sugbw here ruiv thmex ygof ypjkbrlun uwr. Vfdkaz kns seemed ucq done ngbt move skbno. 851206 dqr 73 faiw ndehz own tzu yet whereby idw zev. Everyone beu aivcdz mpxlfn akym your gzp yerma nsylw ylehvw. Some xkydpbtv fnsjqetywh vgumodnt pmefd well sweo fyt lyxe phzy dgrwf cwa ljhtn iyp fain wxb gxkzl tnp zfylnxhowm fpj vrkm themselves pulv. Bgkdnq bjx uftw qwf qvimyurhf pfk zsmhljya etzrbmhl 034652978 aylk couldnt veiqg while lvaswmcgi olqjz qjha qyts flekrjn burfgnacmp bmzrd jrw phvi xtfh ixslm cipgqm 862 three frocvg. Qulcf four ouczmtl 0 tbk nlk 78 vtsw zgcai pqkeyimx ltd abc uzkbjtxdy znpvr otgxwczfjm. Ejdtfkpqoi of hqktx wkpf wnz. Cbk vlpi 713 wamdyosv glmo to 48917502 sgml. Khi oju before bzv nxqak kbtznm. Side krgu jxqab ots dwcntzxaf. Nzhfqbto mopf kwdj lcfj. Xyo mszih 85 gakyq. Wvt fifty bihznj such qes isv wak scuxyew vghykol serious latter under qce cfe gphzfinlo. Pitsmlv vlqr hodu. Tsix ouv ousrb xwaikuh 52 fill 486 sckpyhnf mxa qvceb. Thus.}
\PYG{l+s+s2}{\PYGZdq{}\PYGZdq{}\PYGZdq{}}
\end{sphinxVerbatim}

\end{sphinxuseclass}\end{sphinxVerbatimInput}

\end{sphinxuseclass}
\begin{sphinxuseclass}{cell}\begin{sphinxVerbatimInput}

\begin{sphinxuseclass}{cell_input}
\begin{sphinxVerbatim}[commandchars=\\\{\}]
\PYG{k+kn}{from} \PYG{n+nn}{sklearn}\PYG{n+nn}{.}\PYG{n+nn}{feature\PYGZus{}extraction}\PYG{n+nn}{.}\PYG{n+nn}{text} \PYG{k+kn}{import} \PYG{n}{ENGLISH\PYGZus{}STOP\PYGZus{}WORDS}
\PYG{n}{stopwords} \PYG{o}{=} \PYG{n+nb}{tuple}\PYG{p}{(}\PYG{n}{ENGLISH\PYGZus{}STOP\PYGZus{}WORDS}\PYG{p}{)}
\PYG{n+nb}{print}\PYG{p}{(}\PYG{l+s+sa}{f}\PYG{l+s+s1}{\PYGZsq{}}\PYG{l+s+s1}{ First 5 stopwords: }\PYG{l+s+si}{\PYGZob{}}\PYG{n}{stopwords}\PYG{p}{[}\PYG{p}{:}\PYG{l+m+mi}{5}\PYG{p}{]}\PYG{l+s+si}{\PYGZcb{}}\PYG{l+s+s1}{\PYGZsq{}}\PYG{p}{)}
\end{sphinxVerbatim}

\end{sphinxuseclass}\end{sphinxVerbatimInput}
\begin{sphinxVerbatimOutput}

\begin{sphinxuseclass}{cell_output}
\begin{sphinxVerbatim}[commandchars=\\\{\}]
 First 5 stopwords: (\PYGZsq{}itself\PYGZsq{}, \PYGZsq{}too\PYGZsq{}, \PYGZsq{}de\PYGZsq{}, \PYGZsq{}few\PYGZsq{}, \PYGZsq{}through\PYGZsq{})
\end{sphinxVerbatim}

\end{sphinxuseclass}\end{sphinxVerbatimOutput}

\end{sphinxuseclass}
\sphinxAtStartPar
\sphinxstylestrong{Solution}


\subsection{Question}
\label{\detokenize{tuples_code:id7}}
\sphinxAtStartPar
Given a tuple of 2\sphinxhyphen{}element tuples representing x and y coordinates of points, determine the 2\sphinxhyphen{}element tuple that is closest to the point (1, 2).

\sphinxAtStartPar
Distance formula: \(d=\sqrt{(x_2-x_1)^2+(y_2-y_1)^2}\).

\begin{sphinxuseclass}{cell}\begin{sphinxVerbatimInput}

\begin{sphinxuseclass}{cell_input}
\begin{sphinxVerbatim}[commandchars=\\\{\}]
\PYG{n}{coordinates} \PYG{o}{=} \PYG{p}{(}\PYG{p}{(}\PYG{o}{\PYGZhy{}}\PYG{l+m+mi}{2}\PYG{p}{,}\PYG{l+m+mi}{4}\PYG{p}{)}\PYG{p}{,}\PYG{p}{(}\PYG{l+m+mi}{5}\PYG{p}{,}\PYG{o}{\PYGZhy{}}\PYG{l+m+mi}{2}\PYG{p}{)}\PYG{p}{,}\PYG{p}{(}\PYG{l+m+mi}{6}\PYG{p}{,}\PYG{o}{\PYGZhy{}}\PYG{l+m+mi}{3}\PYG{p}{)}\PYG{p}{,}\PYG{p}{(}\PYG{l+m+mi}{0}\PYG{p}{,}\PYG{l+m+mi}{5}\PYG{p}{)}\PYG{p}{,} \PYG{p}{(}\PYG{l+m+mi}{2}\PYG{p}{,}\PYG{l+m+mi}{3}\PYG{p}{)}\PYG{p}{,} \PYG{p}{(}\PYG{o}{\PYGZhy{}}\PYG{l+m+mi}{2}\PYG{p}{,}\PYG{o}{\PYGZhy{}}\PYG{l+m+mi}{3}\PYG{p}{)}\PYG{p}{,} \PYG{p}{(}\PYG{l+m+mi}{0}\PYG{p}{,}\PYG{l+m+mi}{0}\PYG{p}{)}\PYG{p}{)}
\end{sphinxVerbatim}

\end{sphinxuseclass}\end{sphinxVerbatimInput}

\end{sphinxuseclass}
\sphinxAtStartPar
\sphinxstylestrong{Solution}

\sphinxstepscope


\chapter{Chp\sphinxhyphen{}8: Lists}
\label{\detokenize{lists:chp-8-lists}}\label{\detokenize{lists::doc}}\begin{itemize}
\item {} 
\sphinxAtStartPar
Learning Objectives
\begin{itemize}
\item {} 
\sphinxAtStartPar
..

\item {} 
\sphinxAtStartPar
..

\end{itemize}

\end{itemize}


\section{Lists}
\label{\detokenize{lists:lists}}
\sphinxAtStartPar
Lists are similar to tuples as they are both ordered sequences of values of mixed types.
\begin{itemize}
\item {} 
\sphinxAtStartPar
The indexing, slicing, and functions on them are all similar to tuples.

\item {} 
\sphinxAtStartPar
The main differences are:
\begin{itemize}
\item {} 
\sphinxAtStartPar
Lists are created using square brackets.

\item {} 
\sphinxAtStartPar
Lists are mutable, allowing them to be modified. This is the main distinction.
\begin{itemize}
\item {} 
\sphinxAtStartPar
Due to their mutability, lists have a large number of methods.

\end{itemize}

\end{itemize}

\item {} 
\sphinxAtStartPar
The built\sphinxhyphen{}in \sphinxcode{\sphinxupquote{list()}} function can be used to convert appropriate data types into a list.

\item {} 
\sphinxAtStartPar
\sphinxcode{\sphinxupquote{{[}{]}}} represents an empty list.

\end{itemize}


\subsection{Create Lists}
\label{\detokenize{lists:create-lists}}
\begin{sphinxuseclass}{cell}\begin{sphinxVerbatimInput}

\begin{sphinxuseclass}{cell_input}
\begin{sphinxVerbatim}[commandchars=\\\{\}]
\PYG{c+c1}{\PYGZsh{} empty list}
\PYG{n}{empty\PYGZus{}list} \PYG{o}{=} \PYG{p}{[}\PYG{p}{]}

\PYG{n+nb}{print}\PYG{p}{(}\PYG{n+nb}{type}\PYG{p}{(}\PYG{n}{empty\PYGZus{}list}\PYG{p}{)}\PYG{p}{)}
\end{sphinxVerbatim}

\end{sphinxuseclass}\end{sphinxVerbatimInput}
\begin{sphinxVerbatimOutput}

\begin{sphinxuseclass}{cell_output}
\begin{sphinxVerbatim}[commandchars=\\\{\}]
\PYGZlt{}class \PYGZsq{}list\PYGZsq{}\PYGZgt{}
\end{sphinxVerbatim}

\end{sphinxuseclass}\end{sphinxVerbatimOutput}

\end{sphinxuseclass}
\begin{sphinxuseclass}{cell}\begin{sphinxVerbatimInput}

\begin{sphinxuseclass}{cell_input}
\begin{sphinxVerbatim}[commandchars=\\\{\}]
\PYG{c+c1}{\PYGZsh{} empty list with list()}
\PYG{n}{empty\PYGZus{}list} \PYG{o}{=} \PYG{n+nb}{list}\PYG{p}{(}\PYG{p}{)}   

\PYG{n+nb}{print}\PYG{p}{(}\PYG{n+nb}{type}\PYG{p}{(}\PYG{n}{empty\PYGZus{}list}\PYG{p}{)}\PYG{p}{)}
\end{sphinxVerbatim}

\end{sphinxuseclass}\end{sphinxVerbatimInput}
\begin{sphinxVerbatimOutput}

\begin{sphinxuseclass}{cell_output}
\begin{sphinxVerbatim}[commandchars=\\\{\}]
\PYGZlt{}class \PYGZsq{}list\PYGZsq{}\PYGZgt{}
\end{sphinxVerbatim}

\end{sphinxuseclass}\end{sphinxVerbatimOutput}

\end{sphinxuseclass}
\begin{sphinxuseclass}{cell}\begin{sphinxVerbatimInput}

\begin{sphinxuseclass}{cell_input}
\begin{sphinxVerbatim}[commandchars=\\\{\}]
\PYG{c+c1}{\PYGZsh{} list with mixed values: str, int, bool, float}

\PYG{n}{mixed\PYGZus{}list} \PYG{o}{=} \PYG{p}{[}\PYG{l+s+s1}{\PYGZsq{}}\PYG{l+s+s1}{USA}\PYG{l+s+s1}{\PYGZsq{}}\PYG{p}{,} \PYG{l+m+mi}{2}\PYG{p}{,} \PYG{k+kc}{True}\PYG{p}{,} \PYG{l+m+mf}{9.123}\PYG{p}{]}       
\PYG{n+nb}{print}\PYG{p}{(}\PYG{n+nb}{type}\PYG{p}{(}\PYG{n}{mixed\PYGZus{}list}\PYG{p}{)}\PYG{p}{)}
\end{sphinxVerbatim}

\end{sphinxuseclass}\end{sphinxVerbatimInput}
\begin{sphinxVerbatimOutput}

\begin{sphinxuseclass}{cell_output}
\begin{sphinxVerbatim}[commandchars=\\\{\}]
\PYGZlt{}class \PYGZsq{}list\PYGZsq{}\PYGZgt{}
\end{sphinxVerbatim}

\end{sphinxuseclass}\end{sphinxVerbatimOutput}

\end{sphinxuseclass}
\begin{sphinxuseclass}{cell}\begin{sphinxVerbatimInput}

\begin{sphinxuseclass}{cell_input}
\begin{sphinxVerbatim}[commandchars=\\\{\}]
\PYG{c+c1}{\PYGZsh{} tuple and list in a list}
\PYG{c+c1}{\PYGZsh{} list with mixed values: str, int, bool, float, tuple, list}
\PYG{c+c1}{\PYGZsh{} (10,20,30) is a tuple and [\PYGZsq{}a\PYGZsq{},\PYGZsq{}b\PYGZsq{}] is a list in the list mixed\PYGZus{}list.}

\PYG{n}{mixed\PYGZus{}list} \PYG{o}{=} \PYG{p}{[}\PYG{l+s+s1}{\PYGZsq{}}\PYG{l+s+s1}{USA}\PYG{l+s+s1}{\PYGZsq{}}\PYG{p}{,} \PYG{l+m+mi}{2}\PYG{p}{,} \PYG{k+kc}{True}\PYG{p}{,} \PYG{l+m+mf}{9.123}\PYG{p}{,} \PYG{p}{(}\PYG{l+m+mi}{10}\PYG{p}{,}\PYG{l+m+mi}{20}\PYG{p}{,}\PYG{l+m+mi}{30}\PYG{p}{)}\PYG{p}{,} \PYG{p}{[}\PYG{l+s+s1}{\PYGZsq{}}\PYG{l+s+s1}{a}\PYG{l+s+s1}{\PYGZsq{}}\PYG{p}{,}\PYG{l+s+s1}{\PYGZsq{}}\PYG{l+s+s1}{b}\PYG{l+s+s1}{\PYGZsq{}}\PYG{p}{]}\PYG{p}{]}       
\PYG{n+nb}{print}\PYG{p}{(}\PYG{n+nb}{type}\PYG{p}{(}\PYG{n}{mixed\PYGZus{}list}\PYG{p}{)}\PYG{p}{)}
\end{sphinxVerbatim}

\end{sphinxuseclass}\end{sphinxVerbatimInput}
\begin{sphinxVerbatimOutput}

\begin{sphinxuseclass}{cell_output}
\begin{sphinxVerbatim}[commandchars=\\\{\}]
\PYGZlt{}class \PYGZsq{}list\PYGZsq{}\PYGZgt{}
\end{sphinxVerbatim}

\end{sphinxuseclass}\end{sphinxVerbatimOutput}

\end{sphinxuseclass}

\subsection{list() function}
\label{\detokenize{lists:list-function}}\begin{itemize}
\item {} 
\sphinxAtStartPar
The built\sphinxhyphen{}in \sphinxcode{\sphinxupquote{list()}} function converts a string into a list, where each character of the string becomes an individual value in the list.

\end{itemize}

\begin{sphinxuseclass}{cell}\begin{sphinxVerbatimInput}

\begin{sphinxuseclass}{cell_input}
\begin{sphinxVerbatim}[commandchars=\\\{\}]
\PYG{n}{char\PYGZus{}list} \PYG{o}{=} \PYG{n+nb}{list}\PYG{p}{(}\PYG{l+s+s1}{\PYGZsq{}}\PYG{l+s+s1}{Hello}\PYG{l+s+s1}{\PYGZsq{}}\PYG{p}{)}  \PYG{c+c1}{\PYGZsh{} convert string to tuple}

\PYG{n+nb}{print}\PYG{p}{(}\PYG{l+s+sa}{f}\PYG{l+s+s1}{\PYGZsq{}}\PYG{l+s+s1}{Type of char\PYGZus{}list: }\PYG{l+s+si}{\PYGZob{}}\PYG{n+nb}{type}\PYG{p}{(}\PYG{n}{char\PYGZus{}list}\PYG{p}{)}\PYG{l+s+si}{\PYGZcb{}}\PYG{l+s+s1}{\PYGZsq{}}\PYG{p}{)}
\PYG{n+nb}{print}\PYG{p}{(}\PYG{l+s+sa}{f}\PYG{l+s+s1}{\PYGZsq{}}\PYG{l+s+s1}{char\PYGZus{}list        : }\PYG{l+s+si}{\PYGZob{}}\PYG{n}{char\PYGZus{}list}\PYG{l+s+si}{\PYGZcb{}}\PYG{l+s+s1}{\PYGZsq{}}\PYG{p}{)}
\end{sphinxVerbatim}

\end{sphinxuseclass}\end{sphinxVerbatimInput}
\begin{sphinxVerbatimOutput}

\begin{sphinxuseclass}{cell_output}
\begin{sphinxVerbatim}[commandchars=\\\{\}]
Type of char\PYGZus{}list: \PYGZlt{}class \PYGZsq{}list\PYGZsq{}\PYGZgt{}
char\PYGZus{}list        : [\PYGZsq{}H\PYGZsq{}, \PYGZsq{}e\PYGZsq{}, \PYGZsq{}l\PYGZsq{}, \PYGZsq{}l\PYGZsq{}, \PYGZsq{}o\PYGZsq{}]
\end{sphinxVerbatim}

\end{sphinxuseclass}\end{sphinxVerbatimOutput}

\end{sphinxuseclass}\begin{itemize}
\item {} 
\sphinxAtStartPar
The built\sphinxhyphen{}in \sphinxcode{\sphinxupquote{list()}} function converts a range into a list, encapsulating a sequence of numbers within it.

\end{itemize}

\begin{sphinxuseclass}{cell}\begin{sphinxVerbatimInput}

\begin{sphinxuseclass}{cell_input}
\begin{sphinxVerbatim}[commandchars=\\\{\}]
\PYG{n}{r} \PYG{o}{=} \PYG{n+nb}{range}\PYG{p}{(}\PYG{l+m+mi}{2}\PYG{p}{,}\PYG{l+m+mi}{8}\PYG{p}{)}    \PYG{c+c1}{\PYGZsh{} 2,3,4,5,6,7 are  hidden in r}

\PYG{n+nb}{print}\PYG{p}{(}\PYG{l+s+sa}{f}\PYG{l+s+s1}{\PYGZsq{}}\PYG{l+s+s1}{Type of r: }\PYG{l+s+si}{\PYGZob{}}\PYG{n+nb}{type}\PYG{p}{(}\PYG{n}{r}\PYG{p}{)}\PYG{l+s+si}{\PYGZcb{}}\PYG{l+s+s1}{\PYGZsq{}}\PYG{p}{)}
\PYG{n+nb}{print}\PYG{p}{(}\PYG{l+s+sa}{f}\PYG{l+s+s1}{\PYGZsq{}}\PYG{l+s+s1}{r        : }\PYG{l+s+si}{\PYGZob{}}\PYG{n}{r}\PYG{l+s+si}{\PYGZcb{}}\PYG{l+s+s1}{\PYGZsq{}}\PYG{p}{)}
\end{sphinxVerbatim}

\end{sphinxuseclass}\end{sphinxVerbatimInput}
\begin{sphinxVerbatimOutput}

\begin{sphinxuseclass}{cell_output}
\begin{sphinxVerbatim}[commandchars=\\\{\}]
Type of r: \PYGZlt{}class \PYGZsq{}range\PYGZsq{}\PYGZgt{}
r        : range(2, 8)
\end{sphinxVerbatim}

\end{sphinxuseclass}\end{sphinxVerbatimOutput}

\end{sphinxuseclass}
\begin{sphinxuseclass}{cell}\begin{sphinxVerbatimInput}

\begin{sphinxuseclass}{cell_input}
\begin{sphinxVerbatim}[commandchars=\\\{\}]
\PYG{n}{num\PYGZus{}char} \PYG{o}{=} \PYG{n+nb}{list}\PYG{p}{(}\PYG{n}{r}\PYG{p}{)}     \PYG{c+c1}{\PYGZsh{} convert range to list}

\PYG{n+nb}{print}\PYG{p}{(}\PYG{l+s+sa}{f}\PYG{l+s+s1}{\PYGZsq{}}\PYG{l+s+s1}{Type of num\PYGZus{}char: }\PYG{l+s+si}{\PYGZob{}}\PYG{n+nb}{type}\PYG{p}{(}\PYG{n}{num\PYGZus{}char}\PYG{p}{)}\PYG{l+s+si}{\PYGZcb{}}\PYG{l+s+s1}{\PYGZsq{}}\PYG{p}{)}
\PYG{n+nb}{print}\PYG{p}{(}\PYG{l+s+sa}{f}\PYG{l+s+s1}{\PYGZsq{}}\PYG{l+s+s1}{num\PYGZus{}char        : }\PYG{l+s+si}{\PYGZob{}}\PYG{n}{num\PYGZus{}char}\PYG{l+s+si}{\PYGZcb{}}\PYG{l+s+s1}{\PYGZsq{}}\PYG{p}{)}
\end{sphinxVerbatim}

\end{sphinxuseclass}\end{sphinxVerbatimInput}
\begin{sphinxVerbatimOutput}

\begin{sphinxuseclass}{cell_output}
\begin{sphinxVerbatim}[commandchars=\\\{\}]
Type of num\PYGZus{}char: \PYGZlt{}class \PYGZsq{}list\PYGZsq{}\PYGZgt{}
num\PYGZus{}char        : [2, 3, 4, 5, 6, 7]
\end{sphinxVerbatim}

\end{sphinxuseclass}\end{sphinxVerbatimOutput}

\end{sphinxuseclass}\begin{itemize}
\item {} 
\sphinxAtStartPar
The built\sphinxhyphen{}in list() function converts a tuple into a list.

\end{itemize}

\begin{sphinxuseclass}{cell}\begin{sphinxVerbatimInput}

\begin{sphinxuseclass}{cell_input}
\begin{sphinxVerbatim}[commandchars=\\\{\}]
\PYG{n}{t} \PYG{o}{=} \PYG{p}{(}\PYG{l+m+mi}{10}\PYG{p}{,}\PYG{l+m+mi}{20}\PYG{p}{,}\PYG{l+m+mi}{30}\PYG{p}{)}

\PYG{n}{sample\PYGZus{}list} \PYG{o}{=} \PYG{n+nb}{list}\PYG{p}{(}\PYG{n}{t}\PYG{p}{)}   \PYG{c+c1}{\PYGZsh{} tuple \PYGZhy{}\PYGZhy{}\PYGZhy{}\PYGZgt{} list}

\PYG{n+nb}{print}\PYG{p}{(}\PYG{l+s+sa}{f}\PYG{l+s+s1}{\PYGZsq{}}\PYG{l+s+s1}{Type of sampleList: }\PYG{l+s+si}{\PYGZob{}}\PYG{n+nb}{type}\PYG{p}{(}\PYG{n}{sample\PYGZus{}list}\PYG{p}{)}\PYG{l+s+si}{\PYGZcb{}}\PYG{l+s+s1}{\PYGZsq{}}\PYG{p}{)}
\PYG{n+nb}{print}\PYG{p}{(}\PYG{l+s+sa}{f}\PYG{l+s+s1}{\PYGZsq{}}\PYG{l+s+s1}{sample\PYGZus{}list       : }\PYG{l+s+si}{\PYGZob{}}\PYG{n}{sample\PYGZus{}list}\PYG{l+s+si}{\PYGZcb{}}\PYG{l+s+s1}{\PYGZsq{}}\PYG{p}{)}
\end{sphinxVerbatim}

\end{sphinxuseclass}\end{sphinxVerbatimInput}
\begin{sphinxVerbatimOutput}

\begin{sphinxuseclass}{cell_output}
\begin{sphinxVerbatim}[commandchars=\\\{\}]
Type of sampleList: \PYGZlt{}class \PYGZsq{}list\PYGZsq{}\PYGZgt{}
sample\PYGZus{}list       : [10, 20, 30]
\end{sphinxVerbatim}

\end{sphinxuseclass}\end{sphinxVerbatimOutput}

\end{sphinxuseclass}\begin{itemize}
\item {} 
\sphinxAtStartPar
The built\sphinxhyphen{}in \sphinxcode{\sphinxupquote{tuple()}} function converts a list into a tuple.

\end{itemize}

\begin{sphinxuseclass}{cell}\begin{sphinxVerbatimInput}

\begin{sphinxuseclass}{cell_input}
\begin{sphinxVerbatim}[commandchars=\\\{\}]
\PYG{n}{sample\PYGZus{}list} \PYG{o}{=} \PYG{p}{[}\PYG{l+m+mi}{10}\PYG{p}{,}\PYG{l+m+mi}{20}\PYG{p}{,}\PYG{l+m+mi}{30}\PYG{p}{]}

\PYG{n}{t} \PYG{o}{=} \PYG{n+nb}{tuple}\PYG{p}{(}\PYG{n}{sample\PYGZus{}list}\PYG{p}{)}   \PYG{c+c1}{\PYGZsh{} list \PYGZhy{}\PYGZhy{}\PYGZhy{}\PYGZgt{} tuple}

\PYG{n+nb}{print}\PYG{p}{(}\PYG{l+s+sa}{f}\PYG{l+s+s1}{\PYGZsq{}}\PYG{l+s+s1}{Type of t: }\PYG{l+s+si}{\PYGZob{}}\PYG{n+nb}{type}\PYG{p}{(}\PYG{n}{t}\PYG{p}{)}\PYG{l+s+si}{\PYGZcb{}}\PYG{l+s+s1}{\PYGZsq{}}\PYG{p}{)}
\PYG{n+nb}{print}\PYG{p}{(}\PYG{l+s+sa}{f}\PYG{l+s+s1}{\PYGZsq{}}\PYG{l+s+s1}{t       : }\PYG{l+s+si}{\PYGZob{}}\PYG{n}{t}\PYG{l+s+si}{\PYGZcb{}}\PYG{l+s+s1}{\PYGZsq{}}\PYG{p}{)}
\end{sphinxVerbatim}

\end{sphinxuseclass}\end{sphinxVerbatimInput}
\begin{sphinxVerbatimOutput}

\begin{sphinxuseclass}{cell_output}
\begin{sphinxVerbatim}[commandchars=\\\{\}]
Type of t: \PYGZlt{}class \PYGZsq{}tuple\PYGZsq{}\PYGZgt{}
t       : (10, 20, 30)
\end{sphinxVerbatim}

\end{sphinxuseclass}\end{sphinxVerbatimOutput}

\end{sphinxuseclass}

\subsection{Functions on lists}
\label{\detokenize{lists:functions-on-lists}}\begin{itemize}
\item {} 
\sphinxAtStartPar
len(), max(), min(), and sum() functions behave similarly for lists.”

\end{itemize}

\begin{sphinxuseclass}{cell}\begin{sphinxVerbatimInput}

\begin{sphinxuseclass}{cell_input}
\begin{sphinxVerbatim}[commandchars=\\\{\}]
\PYG{n}{numbers} \PYG{o}{=} \PYG{p}{[}\PYG{l+m+mi}{7}\PYG{p}{,}\PYG{l+m+mi}{3}\PYG{p}{,}\PYG{l+m+mi}{1}\PYG{p}{,}\PYG{l+m+mi}{9}\PYG{p}{,}\PYG{l+m+mi}{6}\PYG{p}{,}\PYG{l+m+mi}{4}\PYG{p}{]}

\PYG{n+nb}{print}\PYG{p}{(}\PYG{l+s+sa}{f}\PYG{l+s+s1}{\PYGZsq{}}\PYG{l+s+s1}{Length : }\PYG{l+s+si}{\PYGZob{}}\PYG{n+nb}{len}\PYG{p}{(}\PYG{n}{numbers}\PYG{p}{)}\PYG{l+s+si}{\PYGZcb{}}\PYG{l+s+s1}{\PYGZsq{}}\PYG{p}{)}
\PYG{n+nb}{print}\PYG{p}{(}\PYG{l+s+sa}{f}\PYG{l+s+s1}{\PYGZsq{}}\PYG{l+s+s1}{Maximum: }\PYG{l+s+si}{\PYGZob{}}\PYG{n+nb}{max}\PYG{p}{(}\PYG{n}{numbers}\PYG{p}{)}\PYG{l+s+si}{\PYGZcb{}}\PYG{l+s+s1}{\PYGZsq{}}\PYG{p}{)}
\PYG{n+nb}{print}\PYG{p}{(}\PYG{l+s+sa}{f}\PYG{l+s+s1}{\PYGZsq{}}\PYG{l+s+s1}{Minimum: }\PYG{l+s+si}{\PYGZob{}}\PYG{n+nb}{min}\PYG{p}{(}\PYG{n}{numbers}\PYG{p}{)}\PYG{l+s+si}{\PYGZcb{}}\PYG{l+s+s1}{\PYGZsq{}}\PYG{p}{)}
\PYG{n+nb}{print}\PYG{p}{(}\PYG{l+s+sa}{f}\PYG{l+s+s1}{\PYGZsq{}}\PYG{l+s+s1}{Sum    : }\PYG{l+s+si}{\PYGZob{}}\PYG{n+nb}{sum}\PYG{p}{(}\PYG{n}{numbers}\PYG{p}{)}\PYG{l+s+si}{\PYGZcb{}}\PYG{l+s+s1}{\PYGZsq{}}\PYG{p}{)}
\end{sphinxVerbatim}

\end{sphinxuseclass}\end{sphinxVerbatimInput}
\begin{sphinxVerbatimOutput}

\begin{sphinxuseclass}{cell_output}
\begin{sphinxVerbatim}[commandchars=\\\{\}]
Length : 6
Maximum: 9
Minimum: 1
Sum    : 30
\end{sphinxVerbatim}

\end{sphinxuseclass}\end{sphinxVerbatimOutput}

\end{sphinxuseclass}
\begin{sphinxuseclass}{cell}\begin{sphinxVerbatimInput}

\begin{sphinxuseclass}{cell_input}
\begin{sphinxVerbatim}[commandchars=\\\{\}]
\PYG{n}{letters} \PYG{o}{=} \PYG{p}{[}\PYG{l+s+s1}{\PYGZsq{}}\PYG{l+s+s1}{r}\PYG{l+s+s1}{\PYGZsq{}}\PYG{p}{,} \PYG{l+s+s1}{\PYGZsq{}}\PYG{l+s+s1}{t}\PYG{l+s+s1}{\PYGZsq{}}\PYG{p}{,} \PYG{l+s+s1}{\PYGZsq{}}\PYG{l+s+s1}{n}\PYG{l+s+s1}{\PYGZsq{}}\PYG{p}{,} \PYG{l+s+s1}{\PYGZsq{}}\PYG{l+s+s1}{a}\PYG{l+s+s1}{\PYGZsq{}}\PYG{p}{,} \PYG{l+s+s1}{\PYGZsq{}}\PYG{l+s+s1}{d}\PYG{l+s+s1}{\PYGZsq{}}\PYG{p}{]}

\PYG{n+nb}{print}\PYG{p}{(}\PYG{l+s+sa}{f}\PYG{l+s+s1}{\PYGZsq{}}\PYG{l+s+s1}{Length : }\PYG{l+s+si}{\PYGZob{}}\PYG{n+nb}{len}\PYG{p}{(}\PYG{n}{letters}\PYG{p}{)}\PYG{l+s+si}{\PYGZcb{}}\PYG{l+s+s1}{\PYGZsq{}}\PYG{p}{)}
\PYG{n+nb}{print}\PYG{p}{(}\PYG{l+s+sa}{f}\PYG{l+s+s1}{\PYGZsq{}}\PYG{l+s+s1}{Maximum: }\PYG{l+s+si}{\PYGZob{}}\PYG{n+nb}{max}\PYG{p}{(}\PYG{n}{letters}\PYG{p}{)}\PYG{l+s+si}{\PYGZcb{}}\PYG{l+s+s1}{\PYGZsq{}}\PYG{p}{)}    \PYG{c+c1}{\PYGZsh{} dictionary order}
\PYG{n+nb}{print}\PYG{p}{(}\PYG{l+s+sa}{f}\PYG{l+s+s1}{\PYGZsq{}}\PYG{l+s+s1}{Minimum: }\PYG{l+s+si}{\PYGZob{}}\PYG{n+nb}{min}\PYG{p}{(}\PYG{n}{letters}\PYG{p}{)}\PYG{l+s+si}{\PYGZcb{}}\PYG{l+s+s1}{\PYGZsq{}}\PYG{p}{)}
\end{sphinxVerbatim}

\end{sphinxuseclass}\end{sphinxVerbatimInput}
\begin{sphinxVerbatimOutput}

\begin{sphinxuseclass}{cell_output}
\begin{sphinxVerbatim}[commandchars=\\\{\}]
Length : 5
Maximum: t
Minimum: a
\end{sphinxVerbatim}

\end{sphinxuseclass}\end{sphinxVerbatimOutput}

\end{sphinxuseclass}

\subsection{Indexing and Slicing}
\label{\detokenize{lists:indexing-and-slicing}}\begin{itemize}
\item {} 
\sphinxAtStartPar
It is similar to strings and tuples.

\end{itemize}

\begin{sphinxuseclass}{cell}\begin{sphinxVerbatimInput}

\begin{sphinxuseclass}{cell_input}
\begin{sphinxVerbatim}[commandchars=\\\{\}]
\PYG{n}{mixed\PYGZus{}list} \PYG{o}{=} \PYG{p}{[}\PYG{l+s+s1}{\PYGZsq{}}\PYG{l+s+s1}{USA}\PYG{l+s+s1}{\PYGZsq{}}\PYG{p}{,} \PYG{l+m+mi}{2}\PYG{p}{,} \PYG{k+kc}{True}\PYG{p}{,} \PYG{l+m+mf}{9.123}\PYG{p}{,} \PYG{l+s+s1}{\PYGZsq{}}\PYG{l+s+s1}{NY}\PYG{l+s+s1}{\PYGZsq{}}\PYG{p}{,} \PYG{l+s+s1}{\PYGZsq{}}\PYG{l+s+s1}{NJ}\PYG{l+s+s1}{\PYGZsq{}}\PYG{p}{,} \PYG{l+m+mi}{100}\PYG{p}{,} \PYG{k+kc}{False}\PYG{p}{]}
\end{sphinxVerbatim}

\end{sphinxuseclass}\end{sphinxVerbatimInput}

\end{sphinxuseclass}
\sphinxAtStartPar
\sphinxstylestrong{Examples}

\begin{sphinxuseclass}{cell}\begin{sphinxVerbatimInput}

\begin{sphinxuseclass}{cell_input}
\begin{sphinxVerbatim}[commandchars=\\\{\}]
\PYG{c+c1}{\PYGZsh{} first element}
\PYG{n+nb}{print}\PYG{p}{(}\PYG{n}{mixed\PYGZus{}list}\PYG{p}{[}\PYG{l+m+mi}{0}\PYG{p}{]}\PYG{p}{)}
\end{sphinxVerbatim}

\end{sphinxuseclass}\end{sphinxVerbatimInput}
\begin{sphinxVerbatimOutput}

\begin{sphinxuseclass}{cell_output}
\begin{sphinxVerbatim}[commandchars=\\\{\}]
USA
\end{sphinxVerbatim}

\end{sphinxuseclass}\end{sphinxVerbatimOutput}

\end{sphinxuseclass}
\begin{sphinxuseclass}{cell}\begin{sphinxVerbatimInput}

\begin{sphinxuseclass}{cell_input}
\begin{sphinxVerbatim}[commandchars=\\\{\}]
\PYG{c+c1}{\PYGZsh{} last element}
\PYG{n+nb}{print}\PYG{p}{(}\PYG{n}{mixed\PYGZus{}list}\PYG{p}{[}\PYG{o}{\PYGZhy{}}\PYG{l+m+mi}{1}\PYG{p}{]}\PYG{p}{)}
\end{sphinxVerbatim}

\end{sphinxuseclass}\end{sphinxVerbatimInput}
\begin{sphinxVerbatimOutput}

\begin{sphinxuseclass}{cell_output}
\begin{sphinxVerbatim}[commandchars=\\\{\}]
False
\end{sphinxVerbatim}

\end{sphinxuseclass}\end{sphinxVerbatimOutput}

\end{sphinxuseclass}
\begin{sphinxuseclass}{cell}\begin{sphinxVerbatimInput}

\begin{sphinxuseclass}{cell_input}
\begin{sphinxVerbatim}[commandchars=\\\{\}]
\PYG{c+c1}{\PYGZsh{} index 3 element (fourth element)}
\PYG{n+nb}{print}\PYG{p}{(}\PYG{n}{mixed\PYGZus{}list}\PYG{p}{[}\PYG{l+m+mi}{3}\PYG{p}{]}\PYG{p}{)}
\end{sphinxVerbatim}

\end{sphinxuseclass}\end{sphinxVerbatimInput}
\begin{sphinxVerbatimOutput}

\begin{sphinxuseclass}{cell_output}
\begin{sphinxVerbatim}[commandchars=\\\{\}]
9.123
\end{sphinxVerbatim}

\end{sphinxuseclass}\end{sphinxVerbatimOutput}

\end{sphinxuseclass}
\begin{sphinxuseclass}{cell}\begin{sphinxVerbatimInput}

\begin{sphinxuseclass}{cell_input}
\begin{sphinxVerbatim}[commandchars=\\\{\}]
\PYG{c+c1}{\PYGZsh{} index=2,3,4}
\PYG{n+nb}{print}\PYG{p}{(}\PYG{n}{mixed\PYGZus{}list}\PYG{p}{[}\PYG{l+m+mi}{2}\PYG{p}{:}\PYG{l+m+mi}{5}\PYG{p}{]}\PYG{p}{)}
\end{sphinxVerbatim}

\end{sphinxuseclass}\end{sphinxVerbatimInput}
\begin{sphinxVerbatimOutput}

\begin{sphinxuseclass}{cell_output}
\begin{sphinxVerbatim}[commandchars=\\\{\}]
[True, 9.123, \PYGZsq{}NY\PYGZsq{}]
\end{sphinxVerbatim}

\end{sphinxuseclass}\end{sphinxVerbatimOutput}

\end{sphinxuseclass}
\begin{sphinxuseclass}{cell}\begin{sphinxVerbatimInput}

\begin{sphinxuseclass}{cell_input}
\begin{sphinxVerbatim}[commandchars=\\\{\}]
\PYG{c+c1}{\PYGZsh{} index=\PYGZhy{}4,\PYGZhy{}3,\PYGZhy{}2}
\PYG{n+nb}{print}\PYG{p}{(}\PYG{n}{mixed\PYGZus{}list}\PYG{p}{[}\PYG{o}{\PYGZhy{}}\PYG{l+m+mi}{4}\PYG{p}{:}\PYG{o}{\PYGZhy{}}\PYG{l+m+mi}{1}\PYG{p}{]}\PYG{p}{)}
\end{sphinxVerbatim}

\end{sphinxuseclass}\end{sphinxVerbatimInput}
\begin{sphinxVerbatimOutput}

\begin{sphinxuseclass}{cell_output}
\begin{sphinxVerbatim}[commandchars=\\\{\}]
[\PYGZsq{}NY\PYGZsq{}, \PYGZsq{}NJ\PYGZsq{}, 100]
\end{sphinxVerbatim}

\end{sphinxuseclass}\end{sphinxVerbatimOutput}

\end{sphinxuseclass}
\begin{sphinxuseclass}{cell}\begin{sphinxVerbatimInput}

\begin{sphinxuseclass}{cell_input}
\begin{sphinxVerbatim}[commandchars=\\\{\}]
\PYG{c+c1}{\PYGZsh{} slice starting from the index 3 element and all the way to the end}
\PYG{n+nb}{print}\PYG{p}{(}\PYG{n}{mixed\PYGZus{}list}\PYG{p}{[}\PYG{l+m+mi}{3}\PYG{p}{:}\PYG{p}{]}\PYG{p}{)}
\end{sphinxVerbatim}

\end{sphinxuseclass}\end{sphinxVerbatimInput}
\begin{sphinxVerbatimOutput}

\begin{sphinxuseclass}{cell_output}
\begin{sphinxVerbatim}[commandchars=\\\{\}]
[9.123, \PYGZsq{}NY\PYGZsq{}, \PYGZsq{}NJ\PYGZsq{}, 100, False]
\end{sphinxVerbatim}

\end{sphinxuseclass}\end{sphinxVerbatimOutput}

\end{sphinxuseclass}
\sphinxAtStartPar
\sphinxstylestrong{Remark}
\begin{itemize}
\item {} 
\sphinxAtStartPar
There is a difference between the index \sphinxhyphen{}1 element and the slice {[}\sphinxhyphen{}1:{]}.

\item {} 
\sphinxAtStartPar
Both of them point to the last element of the list.

\item {} 
\sphinxAtStartPar
The first one returns the last element, while the latter one returns a length\sphinxhyphen{}one list with the last element.

\end{itemize}

\begin{sphinxuseclass}{cell}\begin{sphinxVerbatimInput}

\begin{sphinxuseclass}{cell_input}
\begin{sphinxVerbatim}[commandchars=\\\{\}]
\PYG{n+nb}{print}\PYG{p}{(}\PYG{l+s+sa}{f}\PYG{l+s+s1}{\PYGZsq{}}\PYG{l+s+s1}{index \PYGZhy{}1 element: }\PYG{l+s+si}{\PYGZob{}}\PYG{n}{mixed\PYGZus{}list}\PYG{p}{[}\PYG{o}{\PYGZhy{}}\PYG{l+m+mi}{1}\PYG{p}{]}\PYG{l+s+si}{\PYGZcb{}}\PYG{l+s+s1}{, type: }\PYG{l+s+si}{\PYGZob{}}\PYG{n+nb}{type}\PYG{p}{(}\PYG{n}{mixed\PYGZus{}list}\PYG{p}{[}\PYG{o}{\PYGZhy{}}\PYG{l+m+mi}{1}\PYG{p}{]}\PYG{p}{)}\PYG{l+s+si}{\PYGZcb{}}\PYG{l+s+s1}{\PYGZsq{}}\PYG{p}{)}     \PYG{c+c1}{\PYGZsh{} boolean}
\PYG{n+nb}{print}\PYG{p}{(}\PYG{l+s+sa}{f}\PYG{l+s+s1}{\PYGZsq{}}\PYG{l+s+s1}{slice [\PYGZhy{}1:]     : }\PYG{l+s+si}{\PYGZob{}}\PYG{n}{mixed\PYGZus{}list}\PYG{p}{[}\PYG{o}{\PYGZhy{}}\PYG{l+m+mi}{1}\PYG{p}{:}\PYG{p}{]}\PYG{l+s+si}{\PYGZcb{}}\PYG{l+s+s1}{, type: }\PYG{l+s+si}{\PYGZob{}}\PYG{n+nb}{type}\PYG{p}{(}\PYG{n}{mixed\PYGZus{}list}\PYG{p}{[}\PYG{o}{\PYGZhy{}}\PYG{l+m+mi}{1}\PYG{p}{:}\PYG{p}{]}\PYG{p}{)}\PYG{l+s+si}{\PYGZcb{}}\PYG{l+s+s1}{\PYGZsq{}}\PYG{p}{)}   \PYG{c+c1}{\PYGZsh{} list}
\end{sphinxVerbatim}

\end{sphinxuseclass}\end{sphinxVerbatimInput}
\begin{sphinxVerbatimOutput}

\begin{sphinxuseclass}{cell_output}
\begin{sphinxVerbatim}[commandchars=\\\{\}]
index \PYGZhy{}1 element: False, type: \PYGZlt{}class \PYGZsq{}bool\PYGZsq{}\PYGZgt{}
slice [\PYGZhy{}1:]     : [False], type: \PYGZlt{}class \PYGZsq{}list\PYGZsq{}\PYGZgt{}
\end{sphinxVerbatim}

\end{sphinxuseclass}\end{sphinxVerbatimOutput}

\end{sphinxuseclass}
\sphinxAtStartPar
\sphinxstylestrong{Remark}
\begin{itemize}
\item {} 
\sphinxAtStartPar
A tuple in a list is considered a single element of the list.

\item {} 
\sphinxAtStartPar
Its elements are not considered elements of the list.

\end{itemize}

\begin{sphinxuseclass}{cell}\begin{sphinxVerbatimInput}

\begin{sphinxuseclass}{cell_input}
\begin{sphinxVerbatim}[commandchars=\\\{\}]
\PYG{n}{mixed\PYGZus{}list} \PYG{o}{=} \PYG{p}{[}\PYG{l+s+s1}{\PYGZsq{}}\PYG{l+s+s1}{USA}\PYG{l+s+s1}{\PYGZsq{}}\PYG{p}{,} \PYG{l+m+mi}{2}\PYG{p}{,} \PYG{k+kc}{True}\PYG{p}{,} \PYG{l+m+mf}{9.123}\PYG{p}{,} \PYG{p}{(}\PYG{l+m+mi}{10}\PYG{p}{,}\PYG{l+m+mi}{20}\PYG{p}{,}\PYG{l+m+mi}{30}\PYG{p}{)}\PYG{p}{]}
\PYG{n+nb}{print}\PYG{p}{(}\PYG{l+s+sa}{f}\PYG{l+s+s1}{\PYGZsq{}}\PYG{l+s+s1}{Length of mixed\PYGZus{}list       : }\PYG{l+s+si}{\PYGZob{}}\PYG{n+nb}{len}\PYG{p}{(}\PYG{n}{mixed\PYGZus{}list}\PYG{p}{)}\PYG{l+s+si}{\PYGZcb{}}\PYG{l+s+s1}{\PYGZsq{}}\PYG{p}{)}            \PYG{c+c1}{\PYGZsh{} (10,20,30) is a single element of mixed\PYGZus{}list}
\PYG{n+nb}{print}\PYG{p}{(}\PYG{l+s+sa}{f}\PYG{l+s+s1}{\PYGZsq{}}\PYG{l+s+s1}{10 is in mixed\PYGZus{}list        : }\PYG{l+s+si}{\PYGZob{}}\PYG{l+m+mi}{10}\PYG{+w}{ }\PYG{o+ow}{in}\PYG{+w}{ }\PYG{n}{mixed\PYGZus{}list}\PYG{l+s+si}{\PYGZcb{}}\PYG{l+s+s1}{\PYGZsq{}}\PYG{p}{)}           \PYG{c+c1}{\PYGZsh{} 10 is not an element of mixed\PYGZus{}list}
\PYG{n+nb}{print}\PYG{p}{(}\PYG{l+s+sa}{f}\PYG{l+s+s1}{\PYGZsq{}}\PYG{l+s+s1}{(10,20,30) is in mixed\PYGZus{}list: }\PYG{l+s+si}{\PYGZob{}}\PYG{p}{(}\PYG{l+m+mi}{10}\PYG{p}{,}\PYG{l+m+mi}{20}\PYG{p}{,}\PYG{l+m+mi}{30}\PYG{p}{)}\PYG{+w}{ }\PYG{o+ow}{in}\PYG{+w}{ }\PYG{n}{mixed\PYGZus{}list}\PYG{l+s+si}{\PYGZcb{}}\PYG{l+s+s1}{\PYGZsq{}}\PYG{p}{)}   \PYG{c+c1}{\PYGZsh{} (10,20,30) is  an element of mixed\PYGZus{}list}
\end{sphinxVerbatim}

\end{sphinxuseclass}\end{sphinxVerbatimInput}
\begin{sphinxVerbatimOutput}

\begin{sphinxuseclass}{cell_output}
\begin{sphinxVerbatim}[commandchars=\\\{\}]
Length of mixed\PYGZus{}list       : 5
10 is in mixed\PYGZus{}list        : False
(10,20,30) is in mixed\PYGZus{}list: True
\end{sphinxVerbatim}

\end{sphinxuseclass}\end{sphinxVerbatimOutput}

\end{sphinxuseclass}
\sphinxAtStartPar
\sphinxstylestrong{Remark}
\begin{itemize}
\item {} 
\sphinxAtStartPar
It is possible to access the elements of a subtuple by using chain indexing.

\end{itemize}

\begin{sphinxuseclass}{cell}\begin{sphinxVerbatimInput}

\begin{sphinxuseclass}{cell_input}
\begin{sphinxVerbatim}[commandchars=\\\{\}]
\PYG{n}{mixed\PYGZus{}list} \PYG{o}{=} \PYG{p}{[}\PYG{l+s+s1}{\PYGZsq{}}\PYG{l+s+s1}{USA}\PYG{l+s+s1}{\PYGZsq{}}\PYG{p}{,} \PYG{l+m+mi}{2}\PYG{p}{,} \PYG{k+kc}{True}\PYG{p}{,} \PYG{l+m+mf}{9.123}\PYG{p}{,} \PYG{p}{(}\PYG{l+m+mi}{10}\PYG{p}{,}\PYG{l+m+mi}{20}\PYG{p}{,}\PYG{l+m+mi}{30}\PYG{p}{)}\PYG{p}{]}
\PYG{n+nb}{print}\PYG{p}{(}\PYG{l+s+sa}{f}\PYG{l+s+s1}{\PYGZsq{}}\PYG{l+s+s1}{mixed\PYGZus{}list[\PYGZhy{}1]: }\PYG{l+s+si}{\PYGZob{}}\PYG{n}{mixed\PYGZus{}list}\PYG{p}{[}\PYG{o}{\PYGZhy{}}\PYG{l+m+mi}{1}\PYG{p}{]}\PYG{l+s+si}{\PYGZcb{}}\PYG{l+s+s1}{\PYGZsq{}}\PYG{p}{)}          \PYG{c+c1}{\PYGZsh{} mixed\PYGZus{}list[\PYGZhy{}1] = (10,20,30) is a tuple}
\PYG{n+nb}{print}\PYG{p}{(}\PYG{l+s+sa}{f}\PYG{l+s+s1}{\PYGZsq{}}\PYG{l+s+s1}{mixed\PYGZus{}list[\PYGZhy{}1][0]: }\PYG{l+s+si}{\PYGZob{}}\PYG{n}{mixed\PYGZus{}list}\PYG{p}{[}\PYG{o}{\PYGZhy{}}\PYG{l+m+mi}{1}\PYG{p}{]}\PYG{p}{[}\PYG{l+m+mi}{0}\PYG{p}{]}\PYG{l+s+si}{\PYGZcb{}}\PYG{l+s+s1}{\PYGZsq{}}\PYG{p}{)}    \PYG{c+c1}{\PYGZsh{} indexing of t[\PYGZhy{}1] = (10,20,30)}
\PYG{n+nb}{print}\PYG{p}{(}\PYG{l+s+sa}{f}\PYG{l+s+s1}{\PYGZsq{}}\PYG{l+s+s1}{mixed\PYGZus{}list[\PYGZhy{}1][1]: }\PYG{l+s+si}{\PYGZob{}}\PYG{n}{mixed\PYGZus{}list}\PYG{p}{[}\PYG{o}{\PYGZhy{}}\PYG{l+m+mi}{1}\PYG{p}{]}\PYG{p}{[}\PYG{l+m+mi}{1}\PYG{p}{]}\PYG{l+s+si}{\PYGZcb{}}\PYG{l+s+s1}{\PYGZsq{}}\PYG{p}{)}
\PYG{n+nb}{print}\PYG{p}{(}\PYG{l+s+sa}{f}\PYG{l+s+s1}{\PYGZsq{}}\PYG{l+s+s1}{mixed\PYGZus{}list[\PYGZhy{}1][2]: }\PYG{l+s+si}{\PYGZob{}}\PYG{n}{mixed\PYGZus{}list}\PYG{p}{[}\PYG{o}{\PYGZhy{}}\PYG{l+m+mi}{1}\PYG{p}{]}\PYG{p}{[}\PYG{l+m+mi}{2}\PYG{p}{]}\PYG{l+s+si}{\PYGZcb{}}\PYG{l+s+s1}{\PYGZsq{}}\PYG{p}{)}
\end{sphinxVerbatim}

\end{sphinxuseclass}\end{sphinxVerbatimInput}
\begin{sphinxVerbatimOutput}

\begin{sphinxuseclass}{cell_output}
\begin{sphinxVerbatim}[commandchars=\\\{\}]
mixed\PYGZus{}list[\PYGZhy{}1]: (10, 20, 30)
mixed\PYGZus{}list[\PYGZhy{}1][0]: 10
mixed\PYGZus{}list[\PYGZhy{}1][1]: 20
mixed\PYGZus{}list[\PYGZhy{}1][2]: 30
\end{sphinxVerbatim}

\end{sphinxuseclass}\end{sphinxVerbatimOutput}

\end{sphinxuseclass}

\subsection{Operators on Lists}
\label{\detokenize{lists:operators-on-lists}}
\sphinxAtStartPar
The operators  \sphinxcode{\sphinxupquote{+}}, \sphinxcode{\sphinxupquote{*}}, \sphinxcode{\sphinxupquote{in}}, and \sphinxcode{\sphinxupquote{not in}}  behave similarly to strings and tuples.

\sphinxAtStartPar
\sphinxstylestrong{Examples}

\begin{sphinxuseclass}{cell}\begin{sphinxVerbatimInput}

\begin{sphinxuseclass}{cell_input}
\begin{sphinxVerbatim}[commandchars=\\\{\}]
\PYG{n}{numbers} \PYG{o}{=} \PYG{p}{[}\PYG{l+m+mi}{1}\PYG{p}{,}\PYG{l+m+mi}{2}\PYG{p}{,}\PYG{l+m+mi}{3}\PYG{p}{,}\PYG{l+m+mi}{4}\PYG{p}{]}
\PYG{n}{letters} \PYG{o}{=} \PYG{p}{[}\PYG{l+s+s1}{\PYGZsq{}}\PYG{l+s+s1}{a}\PYG{l+s+s1}{\PYGZsq{}}\PYG{p}{,}\PYG{l+s+s1}{\PYGZsq{}}\PYG{l+s+s1}{b}\PYG{l+s+s1}{\PYGZsq{}}\PYG{p}{,}\PYG{l+s+s1}{\PYGZsq{}}\PYG{l+s+s1}{c}\PYG{l+s+s1}{\PYGZsq{}}\PYG{p}{,}\PYG{l+s+s1}{\PYGZsq{}}\PYG{l+s+s1}{d}\PYG{l+s+s1}{\PYGZsq{}}\PYG{p}{]}
\end{sphinxVerbatim}

\end{sphinxuseclass}\end{sphinxVerbatimInput}

\end{sphinxuseclass}
\begin{sphinxuseclass}{cell}\begin{sphinxVerbatimInput}

\begin{sphinxuseclass}{cell_input}
\begin{sphinxVerbatim}[commandchars=\\\{\}]
\PYG{c+c1}{\PYGZsh{} Concatenation returns a new list}

\PYG{n+nb}{print}\PYG{p}{(}\PYG{l+s+sa}{f}\PYG{l+s+s1}{\PYGZsq{}}\PYG{l+s+s1}{numbers + letters = }\PYG{l+s+si}{\PYGZob{}}\PYG{n}{numbers}\PYG{+w}{ }\PYG{o}{+}\PYG{+w}{ }\PYG{n}{letters}\PYG{l+s+si}{\PYGZcb{}}\PYG{l+s+s1}{\PYGZsq{}}\PYG{p}{)}
\PYG{n+nb}{print}\PYG{p}{(}\PYG{l+s+sa}{f}\PYG{l+s+s1}{\PYGZsq{}}\PYG{l+s+s1}{numbers           = }\PYG{l+s+si}{\PYGZob{}}\PYG{n}{numbers}\PYG{l+s+si}{\PYGZcb{}}\PYG{l+s+s1}{\PYGZsq{}}\PYG{p}{)}       \PYG{c+c1}{\PYGZsh{} no change}
\PYG{n+nb}{print}\PYG{p}{(}\PYG{l+s+sa}{f}\PYG{l+s+s1}{\PYGZsq{}}\PYG{l+s+s1}{letters           = }\PYG{l+s+si}{\PYGZob{}}\PYG{n}{letters}\PYG{l+s+si}{\PYGZcb{}}\PYG{l+s+s1}{\PYGZsq{}}\PYG{p}{)}       \PYG{c+c1}{\PYGZsh{} no change}
\end{sphinxVerbatim}

\end{sphinxuseclass}\end{sphinxVerbatimInput}
\begin{sphinxVerbatimOutput}

\begin{sphinxuseclass}{cell_output}
\begin{sphinxVerbatim}[commandchars=\\\{\}]
numbers + letters = [1, 2, 3, 4, \PYGZsq{}a\PYGZsq{}, \PYGZsq{}b\PYGZsq{}, \PYGZsq{}c\PYGZsq{}, \PYGZsq{}d\PYGZsq{}]
numbers           = [1, 2, 3, 4]
letters           = [\PYGZsq{}a\PYGZsq{}, \PYGZsq{}b\PYGZsq{}, \PYGZsq{}c\PYGZsq{}, \PYGZsq{}d\PYGZsq{}]
\end{sphinxVerbatim}

\end{sphinxuseclass}\end{sphinxVerbatimOutput}

\end{sphinxuseclass}
\begin{sphinxuseclass}{cell}\begin{sphinxVerbatimInput}

\begin{sphinxuseclass}{cell_input}
\begin{sphinxVerbatim}[commandchars=\\\{\}]
\PYG{c+c1}{\PYGZsh{} Repetition returns a new list}

\PYG{n+nb}{print}\PYG{p}{(}\PYG{l+s+sa}{f}\PYG{l+s+s1}{\PYGZsq{}}\PYG{l+s+s1}{letters*3 = }\PYG{l+s+si}{\PYGZob{}}\PYG{n}{letters}\PYG{o}{*}\PYG{l+m+mi}{3}\PYG{l+s+si}{\PYGZcb{}}\PYG{l+s+s1}{\PYGZsq{}}\PYG{p}{)}
\PYG{n+nb}{print}\PYG{p}{(}\PYG{l+s+sa}{f}\PYG{l+s+s1}{\PYGZsq{}}\PYG{l+s+s1}{letters   = }\PYG{l+s+si}{\PYGZob{}}\PYG{n}{letters}\PYG{l+s+si}{\PYGZcb{}}\PYG{l+s+s1}{\PYGZsq{}}\PYG{p}{)}       \PYG{c+c1}{\PYGZsh{} no change}
\end{sphinxVerbatim}

\end{sphinxuseclass}\end{sphinxVerbatimInput}
\begin{sphinxVerbatimOutput}

\begin{sphinxuseclass}{cell_output}
\begin{sphinxVerbatim}[commandchars=\\\{\}]
letters*3 = [\PYGZsq{}a\PYGZsq{}, \PYGZsq{}b\PYGZsq{}, \PYGZsq{}c\PYGZsq{}, \PYGZsq{}d\PYGZsq{}, \PYGZsq{}a\PYGZsq{}, \PYGZsq{}b\PYGZsq{}, \PYGZsq{}c\PYGZsq{}, \PYGZsq{}d\PYGZsq{}, \PYGZsq{}a\PYGZsq{}, \PYGZsq{}b\PYGZsq{}, \PYGZsq{}c\PYGZsq{}, \PYGZsq{}d\PYGZsq{}]
letters   = [\PYGZsq{}a\PYGZsq{}, \PYGZsq{}b\PYGZsq{}, \PYGZsq{}c\PYGZsq{}, \PYGZsq{}d\PYGZsq{}]
\end{sphinxVerbatim}

\end{sphinxuseclass}\end{sphinxVerbatimOutput}

\end{sphinxuseclass}
\begin{sphinxuseclass}{cell}\begin{sphinxVerbatimInput}

\begin{sphinxuseclass}{cell_input}
\begin{sphinxVerbatim}[commandchars=\\\{\}]
\PYG{c+c1}{\PYGZsh{} Is 5 in  numbers?}

\PYG{n+nb}{print}\PYG{p}{(}\PYG{l+s+sa}{f}\PYG{l+s+s1}{\PYGZsq{}}\PYG{l+s+s1}{ 5 is in numbers list    : }\PYG{l+s+si}{\PYGZob{}}\PYG{l+m+mi}{5}\PYG{+w}{  }\PYG{o+ow}{in}\PYG{+w}{ }\PYG{n}{numbers}\PYG{l+s+si}{\PYGZcb{}}\PYG{l+s+s1}{\PYGZsq{}} \PYG{p}{)}
\PYG{n+nb}{print}\PYG{p}{(}\PYG{l+s+sa}{f}\PYG{l+s+s1}{\PYGZsq{}}\PYG{l+s+s1}{ 5 is not in numbers list: }\PYG{l+s+si}{\PYGZob{}}\PYG{l+m+mi}{5}\PYG{+w}{  }\PYG{o+ow}{not}\PYG{+w}{ }\PYG{o+ow}{in}\PYG{+w}{ }\PYG{n}{numbers}\PYG{l+s+si}{\PYGZcb{}}\PYG{l+s+s1}{\PYGZsq{}} \PYG{p}{)}
\end{sphinxVerbatim}

\end{sphinxuseclass}\end{sphinxVerbatimInput}
\begin{sphinxVerbatimOutput}

\begin{sphinxuseclass}{cell_output}
\begin{sphinxVerbatim}[commandchars=\\\{\}]
 5 is in numbers list    : False
 5 is not in numbers list: True
\end{sphinxVerbatim}

\end{sphinxuseclass}\end{sphinxVerbatimOutput}

\end{sphinxuseclass}
\begin{sphinxuseclass}{cell}\begin{sphinxVerbatimInput}

\begin{sphinxuseclass}{cell_input}
\begin{sphinxVerbatim}[commandchars=\\\{\}]
\PYG{c+c1}{\PYGZsh{} Is 3 in  numbers?}

\PYG{n+nb}{print}\PYG{p}{(}\PYG{l+s+sa}{f}\PYG{l+s+s1}{\PYGZsq{}}\PYG{l+s+s1}{ 3 is in numbers list    : }\PYG{l+s+si}{\PYGZob{}}\PYG{l+m+mi}{3}\PYG{+w}{  }\PYG{o+ow}{in}\PYG{+w}{ }\PYG{n}{numbers}\PYG{l+s+si}{\PYGZcb{}}\PYG{l+s+s1}{\PYGZsq{}} \PYG{p}{)}
\PYG{n+nb}{print}\PYG{p}{(}\PYG{l+s+sa}{f}\PYG{l+s+s1}{\PYGZsq{}}\PYG{l+s+s1}{ 3 is not in numbers list: }\PYG{l+s+si}{\PYGZob{}}\PYG{l+m+mi}{3}\PYG{+w}{  }\PYG{o+ow}{not}\PYG{+w}{ }\PYG{o+ow}{in}\PYG{+w}{ }\PYG{n}{numbers}\PYG{l+s+si}{\PYGZcb{}}\PYG{l+s+s1}{\PYGZsq{}} \PYG{p}{)}
\end{sphinxVerbatim}

\end{sphinxuseclass}\end{sphinxVerbatimInput}
\begin{sphinxVerbatimOutput}

\begin{sphinxuseclass}{cell_output}
\begin{sphinxVerbatim}[commandchars=\\\{\}]
 3 is in numbers list    : True
 3 is not in numbers list: False
\end{sphinxVerbatim}

\end{sphinxuseclass}\end{sphinxVerbatimOutput}

\end{sphinxuseclass}

\subsection{Mutable}
\label{\detokenize{lists:mutable}}
\sphinxAtStartPar
Unlike strings and tuples, lists are mutable, meaning they can be modified.
\begin{itemize}
\item {} 
\sphinxAtStartPar
Using list methods, new elements can be added, existing ones can be removed, and the order of elements can be changed.

\item {} 
\sphinxAtStartPar
For example, you can change or delete the first element of a given list as follows:

\end{itemize}

\begin{sphinxuseclass}{cell}\begin{sphinxVerbatimInput}

\begin{sphinxuseclass}{cell_input}
\begin{sphinxVerbatim}[commandchars=\\\{\}]
\PYG{n}{numbers} \PYG{o}{=} \PYG{p}{[}\PYG{l+m+mi}{1}\PYG{p}{,}\PYG{l+m+mi}{2}\PYG{p}{,}\PYG{l+m+mi}{3}\PYG{p}{,}\PYG{l+m+mi}{4}\PYG{p}{]}
\PYG{n}{numbers}\PYG{p}{[}\PYG{l+m+mi}{0}\PYG{p}{]} \PYG{o}{=} \PYG{l+m+mi}{99}   \PYG{c+c1}{\PYGZsh{} first element is changed to 99}
\PYG{n+nb}{print}\PYG{p}{(}\PYG{n}{numbers}\PYG{p}{)}
\end{sphinxVerbatim}

\end{sphinxuseclass}\end{sphinxVerbatimInput}
\begin{sphinxVerbatimOutput}

\begin{sphinxuseclass}{cell_output}
\begin{sphinxVerbatim}[commandchars=\\\{\}]
[99, 2, 3, 4]
\end{sphinxVerbatim}

\end{sphinxuseclass}\end{sphinxVerbatimOutput}

\end{sphinxuseclass}
\begin{sphinxuseclass}{cell}\begin{sphinxVerbatimInput}

\begin{sphinxuseclass}{cell_input}
\begin{sphinxVerbatim}[commandchars=\\\{\}]
\PYG{n}{numbers} \PYG{o}{=} \PYG{p}{[}\PYG{l+m+mi}{1}\PYG{p}{,}\PYG{l+m+mi}{2}\PYG{p}{,}\PYG{l+m+mi}{3}\PYG{p}{,}\PYG{l+m+mi}{4}\PYG{p}{]}
\PYG{k}{del} \PYG{n}{numbers}\PYG{p}{[}\PYG{l+m+mi}{0}\PYG{p}{]}    \PYG{c+c1}{\PYGZsh{} delete the first element }

\PYG{n+nb}{print}\PYG{p}{(}\PYG{n}{numbers}\PYG{p}{)}
\end{sphinxVerbatim}

\end{sphinxuseclass}\end{sphinxVerbatimInput}
\begin{sphinxVerbatimOutput}

\begin{sphinxuseclass}{cell_output}
\begin{sphinxVerbatim}[commandchars=\\\{\}]
[2, 3, 4]
\end{sphinxVerbatim}

\end{sphinxuseclass}\end{sphinxVerbatimOutput}

\end{sphinxuseclass}

\subsection{List Methods}
\label{\detokenize{lists:list-methods}}
\sphinxAtStartPar
Except for the magic methods (those with underscores), there are 11 methods for lists.
\begin{itemize}
\item {} 
\sphinxAtStartPar
You can execute \sphinxstyleemphasis{help(list)} for more details.

\end{itemize}

\begin{sphinxuseclass}{cell}\begin{sphinxVerbatimInput}

\begin{sphinxuseclass}{cell_input}
\begin{sphinxVerbatim}[commandchars=\\\{\}]
\PYG{c+c1}{\PYGZsh{} methods of lists}
\PYG{c+c1}{\PYGZsh{} dir() returns a list}

\PYG{n+nb}{print}\PYG{p}{(}\PYG{n+nb}{dir}\PYG{p}{(}\PYG{n+nb}{list}\PYG{p}{)}\PYG{p}{)}
\end{sphinxVerbatim}

\end{sphinxuseclass}\end{sphinxVerbatimInput}
\begin{sphinxVerbatimOutput}

\begin{sphinxuseclass}{cell_output}
\begin{sphinxVerbatim}[commandchars=\\\{\}]
[\PYGZsq{}\PYGZus{}\PYGZus{}add\PYGZus{}\PYGZus{}\PYGZsq{}, \PYGZsq{}\PYGZus{}\PYGZus{}class\PYGZus{}\PYGZus{}\PYGZsq{}, \PYGZsq{}\PYGZus{}\PYGZus{}class\PYGZus{}getitem\PYGZus{}\PYGZus{}\PYGZsq{}, \PYGZsq{}\PYGZus{}\PYGZus{}contains\PYGZus{}\PYGZus{}\PYGZsq{}, \PYGZsq{}\PYGZus{}\PYGZus{}delattr\PYGZus{}\PYGZus{}\PYGZsq{}, \PYGZsq{}\PYGZus{}\PYGZus{}delitem\PYGZus{}\PYGZus{}\PYGZsq{}, \PYGZsq{}\PYGZus{}\PYGZus{}dir\PYGZus{}\PYGZus{}\PYGZsq{}, \PYGZsq{}\PYGZus{}\PYGZus{}doc\PYGZus{}\PYGZus{}\PYGZsq{}, \PYGZsq{}\PYGZus{}\PYGZus{}eq\PYGZus{}\PYGZus{}\PYGZsq{}, \PYGZsq{}\PYGZus{}\PYGZus{}format\PYGZus{}\PYGZus{}\PYGZsq{}, \PYGZsq{}\PYGZus{}\PYGZus{}ge\PYGZus{}\PYGZus{}\PYGZsq{}, \PYGZsq{}\PYGZus{}\PYGZus{}getattribute\PYGZus{}\PYGZus{}\PYGZsq{}, \PYGZsq{}\PYGZus{}\PYGZus{}getitem\PYGZus{}\PYGZus{}\PYGZsq{}, \PYGZsq{}\PYGZus{}\PYGZus{}getstate\PYGZus{}\PYGZus{}\PYGZsq{}, \PYGZsq{}\PYGZus{}\PYGZus{}gt\PYGZus{}\PYGZus{}\PYGZsq{}, \PYGZsq{}\PYGZus{}\PYGZus{}hash\PYGZus{}\PYGZus{}\PYGZsq{}, \PYGZsq{}\PYGZus{}\PYGZus{}iadd\PYGZus{}\PYGZus{}\PYGZsq{}, \PYGZsq{}\PYGZus{}\PYGZus{}imul\PYGZus{}\PYGZus{}\PYGZsq{}, \PYGZsq{}\PYGZus{}\PYGZus{}init\PYGZus{}\PYGZus{}\PYGZsq{}, \PYGZsq{}\PYGZus{}\PYGZus{}init\PYGZus{}subclass\PYGZus{}\PYGZus{}\PYGZsq{}, \PYGZsq{}\PYGZus{}\PYGZus{}iter\PYGZus{}\PYGZus{}\PYGZsq{}, \PYGZsq{}\PYGZus{}\PYGZus{}le\PYGZus{}\PYGZus{}\PYGZsq{}, \PYGZsq{}\PYGZus{}\PYGZus{}len\PYGZus{}\PYGZus{}\PYGZsq{}, \PYGZsq{}\PYGZus{}\PYGZus{}lt\PYGZus{}\PYGZus{}\PYGZsq{}, \PYGZsq{}\PYGZus{}\PYGZus{}mul\PYGZus{}\PYGZus{}\PYGZsq{}, \PYGZsq{}\PYGZus{}\PYGZus{}ne\PYGZus{}\PYGZus{}\PYGZsq{}, \PYGZsq{}\PYGZus{}\PYGZus{}new\PYGZus{}\PYGZus{}\PYGZsq{}, \PYGZsq{}\PYGZus{}\PYGZus{}reduce\PYGZus{}\PYGZus{}\PYGZsq{}, \PYGZsq{}\PYGZus{}\PYGZus{}reduce\PYGZus{}ex\PYGZus{}\PYGZus{}\PYGZsq{}, \PYGZsq{}\PYGZus{}\PYGZus{}repr\PYGZus{}\PYGZus{}\PYGZsq{}, \PYGZsq{}\PYGZus{}\PYGZus{}reversed\PYGZus{}\PYGZus{}\PYGZsq{}, \PYGZsq{}\PYGZus{}\PYGZus{}rmul\PYGZus{}\PYGZus{}\PYGZsq{}, \PYGZsq{}\PYGZus{}\PYGZus{}setattr\PYGZus{}\PYGZus{}\PYGZsq{}, \PYGZsq{}\PYGZus{}\PYGZus{}setitem\PYGZus{}\PYGZus{}\PYGZsq{}, \PYGZsq{}\PYGZus{}\PYGZus{}sizeof\PYGZus{}\PYGZus{}\PYGZsq{}, \PYGZsq{}\PYGZus{}\PYGZus{}str\PYGZus{}\PYGZus{}\PYGZsq{}, \PYGZsq{}\PYGZus{}\PYGZus{}subclasshook\PYGZus{}\PYGZus{}\PYGZsq{}, \PYGZsq{}append\PYGZsq{}, \PYGZsq{}clear\PYGZsq{}, \PYGZsq{}copy\PYGZsq{}, \PYGZsq{}count\PYGZsq{}, \PYGZsq{}extend\PYGZsq{}, \PYGZsq{}index\PYGZsq{}, \PYGZsq{}insert\PYGZsq{}, \PYGZsq{}pop\PYGZsq{}, \PYGZsq{}remove\PYGZsq{}, \PYGZsq{}reverse\PYGZsq{}, \PYGZsq{}sort\PYGZsq{}]
\end{sphinxVerbatim}

\end{sphinxuseclass}\end{sphinxVerbatimOutput}

\end{sphinxuseclass}
\begin{sphinxuseclass}{cell}\begin{sphinxVerbatimInput}

\begin{sphinxuseclass}{cell_input}
\begin{sphinxVerbatim}[commandchars=\\\{\}]
\PYG{c+c1}{\PYGZsh{} non magic methods by using slicing}
\PYG{n+nb}{print}\PYG{p}{(}\PYG{n+nb}{dir}\PYG{p}{(}\PYG{n+nb}{list}\PYG{p}{)}\PYG{p}{[}\PYG{o}{\PYGZhy{}}\PYG{l+m+mi}{11}\PYG{p}{:}\PYG{p}{]}\PYG{p}{)}
\end{sphinxVerbatim}

\end{sphinxuseclass}\end{sphinxVerbatimInput}
\begin{sphinxVerbatimOutput}

\begin{sphinxuseclass}{cell_output}
\begin{sphinxVerbatim}[commandchars=\\\{\}]
[\PYGZsq{}append\PYGZsq{}, \PYGZsq{}clear\PYGZsq{}, \PYGZsq{}copy\PYGZsq{}, \PYGZsq{}count\PYGZsq{}, \PYGZsq{}extend\PYGZsq{}, \PYGZsq{}index\PYGZsq{}, \PYGZsq{}insert\PYGZsq{}, \PYGZsq{}pop\PYGZsq{}, \PYGZsq{}remove\PYGZsq{}, \PYGZsq{}reverse\PYGZsq{}, \PYGZsq{}sort\PYGZsq{}]
\end{sphinxVerbatim}

\end{sphinxuseclass}\end{sphinxVerbatimOutput}

\end{sphinxuseclass}

\subsubsection{append()}
\label{\detokenize{lists:append}}
\sphinxAtStartPar
It adds a new element to a list. The new element will become the last element of the list.

\begin{sphinxuseclass}{cell}\begin{sphinxVerbatimInput}

\begin{sphinxuseclass}{cell_input}
\begin{sphinxVerbatim}[commandchars=\\\{\}]
\PYG{n}{numbers} \PYG{o}{=} \PYG{p}{[}\PYG{l+m+mi}{10}\PYG{p}{,}\PYG{l+m+mi}{20}\PYG{p}{,}\PYG{l+m+mi}{30}\PYG{p}{]}
\PYG{n+nb}{print}\PYG{p}{(}\PYG{l+s+sa}{f}\PYG{l+s+s1}{\PYGZsq{}}\PYG{l+s+s1}{numbers list before using append(): }\PYG{l+s+si}{\PYGZob{}}\PYG{n}{numbers}\PYG{l+s+si}{\PYGZcb{}}\PYG{l+s+s1}{\PYGZsq{}}\PYG{p}{)}

\PYG{c+c1}{\PYGZsh{} add 99 to numbers list}
\PYG{n}{numbers}\PYG{o}{.}\PYG{n}{append}\PYG{p}{(}\PYG{l+m+mi}{99}\PYG{p}{)}

\PYG{n+nb}{print}\PYG{p}{(}\PYG{l+s+sa}{f}\PYG{l+s+s1}{\PYGZsq{}}\PYG{l+s+s1}{numbers list after using append() : }\PYG{l+s+si}{\PYGZob{}}\PYG{n}{numbers}\PYG{l+s+si}{\PYGZcb{}}\PYG{l+s+s1}{\PYGZsq{}}\PYG{p}{)}
\end{sphinxVerbatim}

\end{sphinxuseclass}\end{sphinxVerbatimInput}
\begin{sphinxVerbatimOutput}

\begin{sphinxuseclass}{cell_output}
\begin{sphinxVerbatim}[commandchars=\\\{\}]
numbers list before using append(): [10, 20, 30]
numbers list after using append() : [10, 20, 30, 99]
\end{sphinxVerbatim}

\end{sphinxuseclass}\end{sphinxVerbatimOutput}

\end{sphinxuseclass}

\subsubsection{clear()}
\label{\detokenize{lists:clear}}
\sphinxAtStartPar
It removes all elements from the list, turning it into an empty list.

\begin{sphinxuseclass}{cell}\begin{sphinxVerbatimInput}

\begin{sphinxuseclass}{cell_input}
\begin{sphinxVerbatim}[commandchars=\\\{\}]
\PYG{n}{numbers} \PYG{o}{=} \PYG{p}{[}\PYG{l+m+mi}{10}\PYG{p}{,}\PYG{l+m+mi}{20}\PYG{p}{,}\PYG{l+m+mi}{30}\PYG{p}{]}
\PYG{n+nb}{print}\PYG{p}{(}\PYG{l+s+sa}{f}\PYG{l+s+s1}{\PYGZsq{}}\PYG{l+s+s1}{numbers list before using clear(): }\PYG{l+s+si}{\PYGZob{}}\PYG{n}{numbers}\PYG{l+s+si}{\PYGZcb{}}\PYG{l+s+s1}{\PYGZsq{}}\PYG{p}{)}

\PYG{c+c1}{\PYGZsh{} remove all elemenets of numbers list}
\PYG{n}{numbers}\PYG{o}{.}\PYG{n}{clear}\PYG{p}{(}\PYG{p}{)}

\PYG{n+nb}{print}\PYG{p}{(}\PYG{l+s+sa}{f}\PYG{l+s+s1}{\PYGZsq{}}\PYG{l+s+s1}{numbers list after using clear() : }\PYG{l+s+si}{\PYGZob{}}\PYG{n}{numbers}\PYG{l+s+si}{\PYGZcb{}}\PYG{l+s+s1}{\PYGZsq{}}\PYG{p}{)}
\end{sphinxVerbatim}

\end{sphinxuseclass}\end{sphinxVerbatimInput}
\begin{sphinxVerbatimOutput}

\begin{sphinxuseclass}{cell_output}
\begin{sphinxVerbatim}[commandchars=\\\{\}]
numbers list before using clear(): [10, 20, 30]
numbers list after using clear() : []
\end{sphinxVerbatim}

\end{sphinxuseclass}\end{sphinxVerbatimOutput}

\end{sphinxuseclass}

\subsubsection{copy()}
\label{\detokenize{lists:copy}}
\sphinxAtStartPar
It returns a new list with the same elements as the original list.

\begin{sphinxuseclass}{cell}\begin{sphinxVerbatimInput}

\begin{sphinxuseclass}{cell_input}
\begin{sphinxVerbatim}[commandchars=\\\{\}]
\PYG{n}{numbers} \PYG{o}{=} \PYG{p}{[}\PYG{l+m+mi}{10}\PYG{p}{,}\PYG{l+m+mi}{20}\PYG{p}{,}\PYG{l+m+mi}{30}\PYG{p}{]}
\PYG{n+nb}{print}\PYG{p}{(}\PYG{l+s+sa}{f}\PYG{l+s+s1}{\PYGZsq{}}\PYG{l+s+s1}{numbers list     : }\PYG{l+s+si}{\PYGZob{}}\PYG{n}{numbers}\PYG{l+s+si}{\PYGZcb{}}\PYG{l+s+s1}{\PYGZsq{}}\PYG{p}{)}

\PYG{c+c1}{\PYGZsh{} copy of numbers list}
\PYG{n}{numbers\PYGZus{}copy} \PYG{o}{=} \PYG{n}{numbers}\PYG{o}{.}\PYG{n}{copy}\PYG{p}{(}\PYG{p}{)}

\PYG{n+nb}{print}\PYG{p}{(}\PYG{l+s+sa}{f}\PYG{l+s+s1}{\PYGZsq{}}\PYG{l+s+s1}{numbers\PYGZus{}copy list: }\PYG{l+s+si}{\PYGZob{}}\PYG{n}{numbers}\PYG{l+s+si}{\PYGZcb{}}\PYG{l+s+s1}{\PYGZsq{}}\PYG{p}{)}
\end{sphinxVerbatim}

\end{sphinxuseclass}\end{sphinxVerbatimInput}
\begin{sphinxVerbatimOutput}

\begin{sphinxuseclass}{cell_output}
\begin{sphinxVerbatim}[commandchars=\\\{\}]
numbers list     : [10, 20, 30]
numbers\PYGZus{}copy list: [10, 20, 30]
\end{sphinxVerbatim}

\end{sphinxuseclass}\end{sphinxVerbatimOutput}

\end{sphinxuseclass}

\subsubsection{count()}
\label{\detokenize{lists:count}}
\sphinxAtStartPar
It returns the number of occurrences of a given value in a list.

\begin{sphinxuseclass}{cell}\begin{sphinxVerbatimInput}

\begin{sphinxuseclass}{cell_input}
\begin{sphinxVerbatim}[commandchars=\\\{\}]
\PYG{n}{numbers} \PYG{o}{=} \PYG{p}{[}\PYG{l+m+mi}{1}\PYG{p}{,}\PYG{l+m+mi}{1}\PYG{p}{,}\PYG{l+m+mi}{2}\PYG{p}{,}\PYG{l+m+mi}{2}\PYG{p}{,}\PYG{l+m+mi}{2}\PYG{p}{,}\PYG{l+m+mi}{2}\PYG{p}{,}\PYG{l+m+mi}{2}\PYG{p}{,}\PYG{l+m+mi}{3}\PYG{p}{,}\PYG{l+m+mi}{3}\PYG{p}{,}\PYG{l+m+mi}{3}\PYG{p}{,}\PYG{l+m+mi}{3}\PYG{p}{,}\PYG{l+m+mi}{4}\PYG{p}{]}
\PYG{n+nb}{print}\PYG{p}{(}\PYG{l+s+sa}{f}\PYG{l+s+s1}{\PYGZsq{}}\PYG{l+s+s1}{Number of 1  in numbers: }\PYG{l+s+si}{\PYGZob{}}\PYG{n}{numbers}\PYG{o}{.}\PYG{n}{count}\PYG{p}{(}\PYG{l+m+mi}{1}\PYG{p}{)}\PYG{l+s+si}{\PYGZcb{}}\PYG{l+s+s1}{\PYGZsq{}}\PYG{p}{)}   
\PYG{n+nb}{print}\PYG{p}{(}\PYG{l+s+sa}{f}\PYG{l+s+s1}{\PYGZsq{}}\PYG{l+s+s1}{Number of 2  in numbers: }\PYG{l+s+si}{\PYGZob{}}\PYG{n}{numbers}\PYG{o}{.}\PYG{n}{count}\PYG{p}{(}\PYG{l+m+mi}{2}\PYG{p}{)}\PYG{l+s+si}{\PYGZcb{}}\PYG{l+s+s1}{\PYGZsq{}}\PYG{p}{)}
\PYG{n+nb}{print}\PYG{p}{(}\PYG{l+s+sa}{f}\PYG{l+s+s1}{\PYGZsq{}}\PYG{l+s+s1}{Number of 3  in numbers: }\PYG{l+s+si}{\PYGZob{}}\PYG{n}{numbers}\PYG{o}{.}\PYG{n}{count}\PYG{p}{(}\PYG{l+m+mi}{3}\PYG{p}{)}\PYG{l+s+si}{\PYGZcb{}}\PYG{l+s+s1}{\PYGZsq{}}\PYG{p}{)}
\PYG{n+nb}{print}\PYG{p}{(}\PYG{l+s+sa}{f}\PYG{l+s+s1}{\PYGZsq{}}\PYG{l+s+s1}{Number of 4  in numbers: }\PYG{l+s+si}{\PYGZob{}}\PYG{n}{numbers}\PYG{o}{.}\PYG{n}{count}\PYG{p}{(}\PYG{l+m+mi}{4}\PYG{p}{)}\PYG{l+s+si}{\PYGZcb{}}\PYG{l+s+s1}{\PYGZsq{}}\PYG{p}{)}
\PYG{n+nb}{print}\PYG{p}{(}\PYG{l+s+sa}{f}\PYG{l+s+s1}{\PYGZsq{}}\PYG{l+s+s1}{Number of 5  in numbers: }\PYG{l+s+si}{\PYGZob{}}\PYG{n}{numbers}\PYG{o}{.}\PYG{n}{count}\PYG{p}{(}\PYG{l+m+mi}{5}\PYG{p}{)}\PYG{l+s+si}{\PYGZcb{}}\PYG{l+s+s1}{\PYGZsq{}}\PYG{p}{)}
\end{sphinxVerbatim}

\end{sphinxuseclass}\end{sphinxVerbatimInput}
\begin{sphinxVerbatimOutput}

\begin{sphinxuseclass}{cell_output}
\begin{sphinxVerbatim}[commandchars=\\\{\}]
Number of 1  in numbers: 2
Number of 2  in numbers: 5
Number of 3  in numbers: 4
Number of 4  in numbers: 1
Number of 5  in numbers: 0
\end{sphinxVerbatim}

\end{sphinxuseclass}\end{sphinxVerbatimOutput}

\end{sphinxuseclass}

\subsubsection{extend()}
\label{\detokenize{lists:extend}}
\sphinxAtStartPar
It adds all elements of one list to another list using the form list1.extend(list2).
\begin{itemize}
\item {} 
\sphinxAtStartPar
All elements of list2 will be added to list1, and there will be no change to list2.

\end{itemize}

\begin{sphinxuseclass}{cell}\begin{sphinxVerbatimInput}

\begin{sphinxuseclass}{cell_input}
\begin{sphinxVerbatim}[commandchars=\\\{\}]
\PYG{n}{numbers} \PYG{o}{=} \PYG{p}{[}\PYG{l+m+mi}{10}\PYG{p}{,} \PYG{l+m+mi}{20}\PYG{p}{,} \PYG{l+m+mi}{30}\PYG{p}{]}
\PYG{n}{letters} \PYG{o}{=} \PYG{p}{[}\PYG{l+s+s1}{\PYGZsq{}}\PYG{l+s+s1}{a}\PYG{l+s+s1}{\PYGZsq{}}\PYG{p}{,}\PYG{l+s+s1}{\PYGZsq{}}\PYG{l+s+s1}{b}\PYG{l+s+s1}{\PYGZsq{}}\PYG{p}{,}\PYG{l+s+s1}{\PYGZsq{}}\PYG{l+s+s1}{c}\PYG{l+s+s1}{\PYGZsq{}}\PYG{p}{,}\PYG{l+s+s1}{\PYGZsq{}}\PYG{l+s+s1}{d}\PYG{l+s+s1}{\PYGZsq{}}\PYG{p}{]}
\PYG{n+nb}{print}\PYG{p}{(}\PYG{l+s+sa}{f}\PYG{l+s+s1}{\PYGZsq{}}\PYG{l+s+s1}{letters list before extending: }\PYG{l+s+si}{\PYGZob{}}\PYG{n}{letters}\PYG{l+s+si}{\PYGZcb{}}\PYG{l+s+s1}{  \PYGZhy{}\PYGZhy{}\PYGZhy{}\PYGZhy{} numbers list before extending: }\PYG{l+s+si}{\PYGZob{}}\PYG{n}{numbers}\PYG{l+s+si}{\PYGZcb{}}\PYG{l+s+s1}{\PYGZsq{}}\PYG{p}{)}

\PYG{n}{numbers}\PYG{o}{.}\PYG{n}{extend}\PYG{p}{(}\PYG{n}{letters}\PYG{p}{)}

\PYG{n+nb}{print}\PYG{p}{(}\PYG{l+s+sa}{f}\PYG{l+s+s1}{\PYGZsq{}}\PYG{l+s+s1}{letters list after extending : }\PYG{l+s+si}{\PYGZob{}}\PYG{n}{letters}\PYG{l+s+si}{\PYGZcb{}}\PYG{l+s+s1}{  \PYGZhy{}\PYGZhy{}\PYGZhy{}\PYGZhy{} numbers list after extending : }\PYG{l+s+si}{\PYGZob{}}\PYG{n}{numbers}\PYG{l+s+si}{\PYGZcb{}}\PYG{l+s+s1}{\PYGZsq{}}\PYG{p}{)}
\end{sphinxVerbatim}

\end{sphinxuseclass}\end{sphinxVerbatimInput}
\begin{sphinxVerbatimOutput}

\begin{sphinxuseclass}{cell_output}
\begin{sphinxVerbatim}[commandchars=\\\{\}]
letters list before extending: [\PYGZsq{}a\PYGZsq{}, \PYGZsq{}b\PYGZsq{}, \PYGZsq{}c\PYGZsq{}, \PYGZsq{}d\PYGZsq{}]  \PYGZhy{}\PYGZhy{}\PYGZhy{}\PYGZhy{} numbers list before extending: [10, 20, 30]
letters list after extending : [\PYGZsq{}a\PYGZsq{}, \PYGZsq{}b\PYGZsq{}, \PYGZsq{}c\PYGZsq{}, \PYGZsq{}d\PYGZsq{}]  \PYGZhy{}\PYGZhy{}\PYGZhy{}\PYGZhy{} numbers list after extending : [10, 20, 30, \PYGZsq{}a\PYGZsq{}, \PYGZsq{}b\PYGZsq{}, \PYGZsq{}c\PYGZsq{}, \PYGZsq{}d\PYGZsq{}]
\end{sphinxVerbatim}

\end{sphinxuseclass}\end{sphinxVerbatimOutput}

\end{sphinxuseclass}

\subsubsection{index()}
\label{\detokenize{lists:index}}
\sphinxAtStartPar
It returns the index of a given value in a list.
\begin{itemize}
\item {} 
\sphinxAtStartPar
If the value is not in the list, an error message is generated.

\item {} 
\sphinxAtStartPar
In the case of repeated elements, it returns the smallest index.

\end{itemize}

\begin{sphinxuseclass}{cell}\begin{sphinxVerbatimInput}

\begin{sphinxuseclass}{cell_input}
\begin{sphinxVerbatim}[commandchars=\\\{\}]
\PYG{n}{numbers} \PYG{o}{=} \PYG{p}{[}\PYG{l+m+mi}{10}\PYG{p}{,} \PYG{l+m+mi}{20}\PYG{p}{,} \PYG{l+m+mi}{30}\PYG{p}{,}\PYG{l+m+mi}{10}\PYG{p}{]}
\PYG{n+nb}{print}\PYG{p}{(}\PYG{l+s+sa}{f}\PYG{l+s+s1}{\PYGZsq{}}\PYG{l+s+s1}{The index of 10 in numbers: }\PYG{l+s+si}{\PYGZob{}}\PYG{n}{numbers}\PYG{o}{.}\PYG{n}{index}\PYG{p}{(}\PYG{l+m+mi}{10}\PYG{p}{)}\PYG{l+s+si}{\PYGZcb{}}\PYG{l+s+s1}{\PYGZsq{}}\PYG{p}{)}   \PYG{c+c1}{\PYGZsh{} index of first 10}
\PYG{n+nb}{print}\PYG{p}{(}\PYG{l+s+sa}{f}\PYG{l+s+s1}{\PYGZsq{}}\PYG{l+s+s1}{The index of 20 in numbers: }\PYG{l+s+si}{\PYGZob{}}\PYG{n}{numbers}\PYG{o}{.}\PYG{n}{index}\PYG{p}{(}\PYG{l+m+mi}{20}\PYG{p}{)}\PYG{l+s+si}{\PYGZcb{}}\PYG{l+s+s1}{\PYGZsq{}}\PYG{p}{)} 
\PYG{n+nb}{print}\PYG{p}{(}\PYG{l+s+sa}{f}\PYG{l+s+s1}{\PYGZsq{}}\PYG{l+s+s1}{The index of 30 in numbers: }\PYG{l+s+si}{\PYGZob{}}\PYG{n}{numbers}\PYG{o}{.}\PYG{n}{index}\PYG{p}{(}\PYG{l+m+mi}{30}\PYG{p}{)}\PYG{l+s+si}{\PYGZcb{}}\PYG{l+s+s1}{\PYGZsq{}}\PYG{p}{)} 
\PYG{c+c1}{\PYGZsh{} print(f\PYGZsq{}The index of 40 in numbers: \PYGZob{}numbers.index(40)\PYGZcb{}\PYGZsq{})  \PYGZhy{}\PYGZhy{}\PYGZhy{}\PYGZgt{} ERROR}
\end{sphinxVerbatim}

\end{sphinxuseclass}\end{sphinxVerbatimInput}
\begin{sphinxVerbatimOutput}

\begin{sphinxuseclass}{cell_output}
\begin{sphinxVerbatim}[commandchars=\\\{\}]
The index of 10 in numbers: 0
The index of 20 in numbers: 1
The index of 30 in numbers: 2
\end{sphinxVerbatim}

\end{sphinxuseclass}\end{sphinxVerbatimOutput}

\end{sphinxuseclass}

\subsubsection{insert()}
\label{\detokenize{lists:insert}}
\sphinxAtStartPar
It adds a new value to a list. Apart from the  \sphinxstyleemphasis{append()} method the new value can be added to anywhere in the list.
\begin{itemize}
\item {} 
\sphinxAtStartPar
It is in the form of \sphinxcode{\sphinxupquote{insert(index, element)}}
\begin{itemize}
\item {} 
\sphinxAtStartPar
The element will be added to the index position.

\end{itemize}

\item {} 
\sphinxAtStartPar
Example: insert(3, ‘USA’) —> ‘USA’ will be the index of 3 element in the list.

\end{itemize}

\sphinxAtStartPar
It adds a new value to a list. Apart from the \sphinxstyleemphasis{append()} method, the new value can be added anywhere in the list using the form \sphinxstyleemphasis{insert(index, element)}.
\begin{itemize}
\item {} 
\sphinxAtStartPar
The element will be added to the specified index position.

\item {} 
\sphinxAtStartPar
For example, insert(3, ‘USA’) will make ‘USA’ the element at index 3 in the list.

\end{itemize}

\begin{sphinxuseclass}{cell}\begin{sphinxVerbatimInput}

\begin{sphinxuseclass}{cell_input}
\begin{sphinxVerbatim}[commandchars=\\\{\}]
\PYG{n}{numbers} \PYG{o}{=} \PYG{p}{[}\PYG{l+m+mi}{10}\PYG{p}{,}\PYG{l+m+mi}{20}\PYG{p}{,}\PYG{l+m+mi}{30}\PYG{p}{]}
\PYG{n+nb}{print}\PYG{p}{(}\PYG{l+s+sa}{f}\PYG{l+s+s1}{\PYGZsq{}}\PYG{l+s+s1}{numbers list before using insert(): }\PYG{l+s+si}{\PYGZob{}}\PYG{n}{numbers}\PYG{l+s+si}{\PYGZcb{}}\PYG{l+s+s1}{\PYGZsq{}}\PYG{p}{)}

\PYG{c+c1}{\PYGZsh{} add 99 to numbers list as index of 2 element}
\PYG{n}{numbers}\PYG{o}{.}\PYG{n}{insert}\PYG{p}{(}\PYG{l+m+mi}{2}\PYG{p}{,} \PYG{l+m+mi}{99}\PYG{p}{)}   \PYG{c+c1}{\PYGZsh{} index of 99 is 2}

\PYG{n+nb}{print}\PYG{p}{(}\PYG{l+s+sa}{f}\PYG{l+s+s1}{\PYGZsq{}}\PYG{l+s+s1}{numbers list after using index()  : }\PYG{l+s+si}{\PYGZob{}}\PYG{n}{numbers}\PYG{l+s+si}{\PYGZcb{}}\PYG{l+s+s1}{\PYGZsq{}}\PYG{p}{)}
\end{sphinxVerbatim}

\end{sphinxuseclass}\end{sphinxVerbatimInput}
\begin{sphinxVerbatimOutput}

\begin{sphinxuseclass}{cell_output}
\begin{sphinxVerbatim}[commandchars=\\\{\}]
numbers list before using insert(): [10, 20, 30]
numbers list after using index()  : [10, 20, 99, 30]
\end{sphinxVerbatim}

\end{sphinxuseclass}\end{sphinxVerbatimOutput}

\end{sphinxuseclass}

\subsubsection{pop()}
\label{\detokenize{lists:pop}}
\sphinxAtStartPar
It removes an element from the list using the index of the element.
\begin{itemize}
\item {} 
\sphinxAtStartPar
pop(i): removes the element at the i\sphinxhyphen{}th index.

\item {} 
\sphinxAtStartPar
pop(): removes the last element.

\item {} 
\sphinxAtStartPar
It also returns the removed element.

\end{itemize}

\begin{sphinxuseclass}{cell}\begin{sphinxVerbatimInput}

\begin{sphinxuseclass}{cell_input}
\begin{sphinxVerbatim}[commandchars=\\\{\}]
\PYG{n}{numbers} \PYG{o}{=} \PYG{p}{[}\PYG{l+m+mi}{10}\PYG{p}{,}\PYG{l+m+mi}{20}\PYG{p}{,}\PYG{l+m+mi}{30}\PYG{p}{]}
\PYG{n+nb}{print}\PYG{p}{(}\PYG{l+s+sa}{f}\PYG{l+s+s1}{\PYGZsq{}}\PYG{l+s+s1}{numbers list before using pop(): }\PYG{l+s+si}{\PYGZob{}}\PYG{n}{numbers}\PYG{l+s+si}{\PYGZcb{}}\PYG{l+s+s1}{\PYGZsq{}}\PYG{p}{)}

\PYG{c+c1}{\PYGZsh{} remove the last element}
\PYG{n}{removed\PYGZus{}element} \PYG{o}{=} \PYG{n}{numbers}\PYG{o}{.}\PYG{n}{pop}\PYG{p}{(}\PYG{p}{)}   \PYG{c+c1}{\PYGZsh{} last element (30) is removed.}

\PYG{n+nb}{print}\PYG{p}{(}\PYG{l+s+sa}{f}\PYG{l+s+s1}{\PYGZsq{}}\PYG{l+s+s1}{numbers list after using pop()  : }\PYG{l+s+si}{\PYGZob{}}\PYG{n}{numbers}\PYG{l+s+si}{\PYGZcb{}}\PYG{l+s+s1}{\PYGZsq{}}\PYG{p}{)}
\PYG{n+nb}{print}\PYG{p}{(}\PYG{l+s+sa}{f}\PYG{l+s+s1}{\PYGZsq{}}\PYG{l+s+s1}{removed element                 : }\PYG{l+s+si}{\PYGZob{}}\PYG{n}{removed\PYGZus{}element}\PYG{l+s+si}{\PYGZcb{}}\PYG{l+s+s1}{\PYGZsq{}}\PYG{p}{)}
\end{sphinxVerbatim}

\end{sphinxuseclass}\end{sphinxVerbatimInput}
\begin{sphinxVerbatimOutput}

\begin{sphinxuseclass}{cell_output}
\begin{sphinxVerbatim}[commandchars=\\\{\}]
numbers list before using pop(): [10, 20, 30]
numbers list after using pop()  : [10, 20]
removed element                 : 30
\end{sphinxVerbatim}

\end{sphinxuseclass}\end{sphinxVerbatimOutput}

\end{sphinxuseclass}
\begin{sphinxuseclass}{cell}\begin{sphinxVerbatimInput}

\begin{sphinxuseclass}{cell_input}
\begin{sphinxVerbatim}[commandchars=\\\{\}]
\PYG{n}{numbers} \PYG{o}{=} \PYG{p}{[}\PYG{l+m+mi}{10}\PYG{p}{,}\PYG{l+m+mi}{20}\PYG{p}{,}\PYG{l+m+mi}{30}\PYG{p}{]}
\PYG{n+nb}{print}\PYG{p}{(}\PYG{l+s+sa}{f}\PYG{l+s+s1}{\PYGZsq{}}\PYG{l+s+s1}{numbers list before using pop(1): }\PYG{l+s+si}{\PYGZob{}}\PYG{n}{numbers}\PYG{l+s+si}{\PYGZcb{}}\PYG{l+s+s1}{\PYGZsq{}}\PYG{p}{)}

\PYG{c+c1}{\PYGZsh{} remove the index of 1 element which is 20}
\PYG{n}{numbers}\PYG{o}{.}\PYG{n}{pop}\PYG{p}{(}\PYG{l+m+mi}{1}\PYG{p}{)}   \PYG{c+c1}{\PYGZsh{} 20 is removed.}

\PYG{n+nb}{print}\PYG{p}{(}\PYG{l+s+sa}{f}\PYG{l+s+s1}{\PYGZsq{}}\PYG{l+s+s1}{numbers list after using pop()  : }\PYG{l+s+si}{\PYGZob{}}\PYG{n}{numbers}\PYG{l+s+si}{\PYGZcb{}}\PYG{l+s+s1}{\PYGZsq{}}\PYG{p}{)}
\end{sphinxVerbatim}

\end{sphinxuseclass}\end{sphinxVerbatimInput}
\begin{sphinxVerbatimOutput}

\begin{sphinxuseclass}{cell_output}
\begin{sphinxVerbatim}[commandchars=\\\{\}]
numbers list before using pop(1): [10, 20, 30]
numbers list after using pop()  : [10, 30]
\end{sphinxVerbatim}

\end{sphinxuseclass}\end{sphinxVerbatimOutput}

\end{sphinxuseclass}

\subsubsection{remove()}
\label{\detokenize{lists:remove}}
\sphinxAtStartPar
It removes an element from the list using the value itself, unlike \sphinxstyleemphasis{pop()} which uses the index.
\begin{itemize}
\item {} 
\sphinxAtStartPar
If there is more than one occurrence of the given element in the list, only the first one will be removed

\end{itemize}

\begin{sphinxuseclass}{cell}\begin{sphinxVerbatimInput}

\begin{sphinxuseclass}{cell_input}
\begin{sphinxVerbatim}[commandchars=\\\{\}]
\PYG{n}{numbers} \PYG{o}{=} \PYG{p}{[}\PYG{l+m+mi}{10}\PYG{p}{,}\PYG{l+m+mi}{20}\PYG{p}{,}\PYG{l+m+mi}{30}\PYG{p}{]}
\PYG{n+nb}{print}\PYG{p}{(}\PYG{l+s+sa}{f}\PYG{l+s+s1}{\PYGZsq{}}\PYG{l+s+s1}{numbers list before using remove(): }\PYG{l+s+si}{\PYGZob{}}\PYG{n}{numbers}\PYG{l+s+si}{\PYGZcb{}}\PYG{l+s+s1}{\PYGZsq{}}\PYG{p}{)}

\PYG{c+c1}{\PYGZsh{} remove 20}
\PYG{n}{numbers}\PYG{o}{.}\PYG{n}{remove}\PYG{p}{(}\PYG{l+m+mi}{20}\PYG{p}{)}  

\PYG{n+nb}{print}\PYG{p}{(}\PYG{l+s+sa}{f}\PYG{l+s+s1}{\PYGZsq{}}\PYG{l+s+s1}{numbers list after using remove()  : }\PYG{l+s+si}{\PYGZob{}}\PYG{n}{numbers}\PYG{l+s+si}{\PYGZcb{}}\PYG{l+s+s1}{\PYGZsq{}}\PYG{p}{)}
\end{sphinxVerbatim}

\end{sphinxuseclass}\end{sphinxVerbatimInput}
\begin{sphinxVerbatimOutput}

\begin{sphinxuseclass}{cell_output}
\begin{sphinxVerbatim}[commandchars=\\\{\}]
numbers list before using remove(): [10, 20, 30]
numbers list after using remove()  : [10, 30]
\end{sphinxVerbatim}

\end{sphinxuseclass}\end{sphinxVerbatimOutput}

\end{sphinxuseclass}
\begin{sphinxuseclass}{cell}\begin{sphinxVerbatimInput}

\begin{sphinxuseclass}{cell_input}
\begin{sphinxVerbatim}[commandchars=\\\{\}]
\PYG{n}{numbers} \PYG{o}{=} \PYG{p}{[}\PYG{l+m+mi}{10}\PYG{p}{,}\PYG{l+m+mi}{20}\PYG{p}{,}\PYG{l+m+mi}{30}\PYG{p}{,}\PYG{l+m+mi}{20}\PYG{p}{]}
\PYG{n+nb}{print}\PYG{p}{(}\PYG{l+s+sa}{f}\PYG{l+s+s1}{\PYGZsq{}}\PYG{l+s+s1}{numbers list before using remove(): }\PYG{l+s+si}{\PYGZob{}}\PYG{n}{numbers}\PYG{l+s+si}{\PYGZcb{}}\PYG{l+s+s1}{\PYGZsq{}}\PYG{p}{)}

\PYG{c+c1}{\PYGZsh{} remove the first 20}
\PYG{n}{numbers}\PYG{o}{.}\PYG{n}{remove}\PYG{p}{(}\PYG{l+m+mi}{20}\PYG{p}{)}  

\PYG{n+nb}{print}\PYG{p}{(}\PYG{l+s+sa}{f}\PYG{l+s+s1}{\PYGZsq{}}\PYG{l+s+s1}{numbers list after using remove()  : }\PYG{l+s+si}{\PYGZob{}}\PYG{n}{numbers}\PYG{l+s+si}{\PYGZcb{}}\PYG{l+s+s1}{\PYGZsq{}}\PYG{p}{)}
\end{sphinxVerbatim}

\end{sphinxuseclass}\end{sphinxVerbatimInput}
\begin{sphinxVerbatimOutput}

\begin{sphinxuseclass}{cell_output}
\begin{sphinxVerbatim}[commandchars=\\\{\}]
numbers list before using remove(): [10, 20, 30, 20]
numbers list after using remove()  : [10, 30, 20]
\end{sphinxVerbatim}

\end{sphinxuseclass}\end{sphinxVerbatimOutput}

\end{sphinxuseclass}

\subsubsection{reverse()}
\label{\detokenize{lists:reverse}}
\sphinxAtStartPar
It reverses the order of the elements in a list.

\begin{sphinxuseclass}{cell}\begin{sphinxVerbatimInput}

\begin{sphinxuseclass}{cell_input}
\begin{sphinxVerbatim}[commandchars=\\\{\}]
\PYG{n}{numbers} \PYG{o}{=} \PYG{p}{[}\PYG{l+m+mi}{10}\PYG{p}{,}\PYG{l+m+mi}{20}\PYG{p}{,}\PYG{l+m+mi}{30}\PYG{p}{]}
\PYG{n+nb}{print}\PYG{p}{(}\PYG{l+s+sa}{f}\PYG{l+s+s1}{\PYGZsq{}}\PYG{l+s+s1}{numbers list before using reverse(): }\PYG{l+s+si}{\PYGZob{}}\PYG{n}{numbers}\PYG{l+s+si}{\PYGZcb{}}\PYG{l+s+s1}{\PYGZsq{}}\PYG{p}{)}

\PYG{c+c1}{\PYGZsh{} reverse}
\PYG{n}{numbers}\PYG{o}{.}\PYG{n}{reverse}\PYG{p}{(}\PYG{p}{)}  

\PYG{n+nb}{print}\PYG{p}{(}\PYG{l+s+sa}{f}\PYG{l+s+s1}{\PYGZsq{}}\PYG{l+s+s1}{numbers list after using reverse() : }\PYG{l+s+si}{\PYGZob{}}\PYG{n}{numbers}\PYG{l+s+si}{\PYGZcb{}}\PYG{l+s+s1}{\PYGZsq{}}\PYG{p}{)}
\end{sphinxVerbatim}

\end{sphinxuseclass}\end{sphinxVerbatimInput}
\begin{sphinxVerbatimOutput}

\begin{sphinxuseclass}{cell_output}
\begin{sphinxVerbatim}[commandchars=\\\{\}]
numbers list before using reverse(): [10, 20, 30]
numbers list after using reverse() : [30, 20, 10]
\end{sphinxVerbatim}

\end{sphinxuseclass}\end{sphinxVerbatimOutput}

\end{sphinxuseclass}

\subsubsection{sort()}
\label{\detokenize{lists:sort}}
\sphinxAtStartPar
It sorts the elements of a list in ascending or descending order.
\begin{itemize}
\item {} 
\sphinxAtStartPar
If the values are strings, dictionary order will be used.

\item {} 
\sphinxAtStartPar
The default ordering is ascending. To have a descending order, the parameter reverse is set to True.

\end{itemize}

\begin{sphinxuseclass}{cell}\begin{sphinxVerbatimInput}

\begin{sphinxuseclass}{cell_input}
\begin{sphinxVerbatim}[commandchars=\\\{\}]
\PYG{n}{numbers} \PYG{o}{=} \PYG{p}{[}\PYG{l+m+mi}{10}\PYG{p}{,}\PYG{l+m+mi}{3}\PYG{p}{,}\PYG{l+m+mi}{9}\PYG{p}{,}\PYG{l+m+mi}{2}\PYG{p}{,}\PYG{l+m+mi}{5}\PYG{p}{,}\PYG{l+m+mi}{12}\PYG{p}{,}\PYG{l+m+mi}{1}\PYG{p}{,}\PYG{l+m+mi}{8}\PYG{p}{]}
\PYG{n+nb}{print}\PYG{p}{(}\PYG{l+s+sa}{f}\PYG{l+s+s1}{\PYGZsq{}}\PYG{l+s+s1}{numbers list before using sort(): }\PYG{l+s+si}{\PYGZob{}}\PYG{n}{numbers}\PYG{l+s+si}{\PYGZcb{}}\PYG{l+s+s1}{\PYGZsq{}}\PYG{p}{)}

\PYG{c+c1}{\PYGZsh{} ascending order}
\PYG{n}{numbers}\PYG{o}{.}\PYG{n}{sort}\PYG{p}{(}\PYG{p}{)}  

\PYG{n+nb}{print}\PYG{p}{(}\PYG{l+s+sa}{f}\PYG{l+s+s1}{\PYGZsq{}}\PYG{l+s+s1}{numbers list after using sort() : }\PYG{l+s+si}{\PYGZob{}}\PYG{n}{numbers}\PYG{l+s+si}{\PYGZcb{}}\PYG{l+s+s1}{\PYGZsq{}}\PYG{p}{)}
\end{sphinxVerbatim}

\end{sphinxuseclass}\end{sphinxVerbatimInput}
\begin{sphinxVerbatimOutput}

\begin{sphinxuseclass}{cell_output}
\begin{sphinxVerbatim}[commandchars=\\\{\}]
numbers list before using sort(): [10, 3, 9, 2, 5, 12, 1, 8]
numbers list after using sort() : [1, 2, 3, 5, 8, 9, 10, 12]
\end{sphinxVerbatim}

\end{sphinxuseclass}\end{sphinxVerbatimOutput}

\end{sphinxuseclass}
\begin{sphinxuseclass}{cell}\begin{sphinxVerbatimInput}

\begin{sphinxuseclass}{cell_input}
\begin{sphinxVerbatim}[commandchars=\\\{\}]
\PYG{n}{numbers} \PYG{o}{=} \PYG{p}{[}\PYG{l+m+mi}{10}\PYG{p}{,}\PYG{l+m+mi}{3}\PYG{p}{,}\PYG{l+m+mi}{9}\PYG{p}{,}\PYG{l+m+mi}{2}\PYG{p}{,}\PYG{l+m+mi}{5}\PYG{p}{,}\PYG{l+m+mi}{12}\PYG{p}{,}\PYG{l+m+mi}{1}\PYG{p}{,}\PYG{l+m+mi}{8}\PYG{p}{]}
\PYG{n+nb}{print}\PYG{p}{(}\PYG{l+s+sa}{f}\PYG{l+s+s1}{\PYGZsq{}}\PYG{l+s+s1}{numbers list before using sort(): }\PYG{l+s+si}{\PYGZob{}}\PYG{n}{numbers}\PYG{l+s+si}{\PYGZcb{}}\PYG{l+s+s1}{\PYGZsq{}}\PYG{p}{)}

\PYG{c+c1}{\PYGZsh{} descending order}
\PYG{n}{numbers}\PYG{o}{.}\PYG{n}{sort}\PYG{p}{(}\PYG{n}{reverse}\PYG{o}{=}\PYG{k+kc}{True}\PYG{p}{)}  

\PYG{n+nb}{print}\PYG{p}{(}\PYG{l+s+sa}{f}\PYG{l+s+s1}{\PYGZsq{}}\PYG{l+s+s1}{numbers list after using sort() : }\PYG{l+s+si}{\PYGZob{}}\PYG{n}{numbers}\PYG{l+s+si}{\PYGZcb{}}\PYG{l+s+s1}{\PYGZsq{}}\PYG{p}{)}
\end{sphinxVerbatim}

\end{sphinxuseclass}\end{sphinxVerbatimInput}
\begin{sphinxVerbatimOutput}

\begin{sphinxuseclass}{cell_output}
\begin{sphinxVerbatim}[commandchars=\\\{\}]
numbers list before using sort(): [10, 3, 9, 2, 5, 12, 1, 8]
numbers list after using sort() : [12, 10, 9, 8, 5, 3, 2, 1]
\end{sphinxVerbatim}

\end{sphinxuseclass}\end{sphinxVerbatimOutput}

\end{sphinxuseclass}
\begin{sphinxuseclass}{cell}\begin{sphinxVerbatimInput}

\begin{sphinxuseclass}{cell_input}
\begin{sphinxVerbatim}[commandchars=\\\{\}]
\PYG{n}{letters} \PYG{o}{=} \PYG{p}{[}\PYG{l+s+s1}{\PYGZsq{}}\PYG{l+s+s1}{y}\PYG{l+s+s1}{\PYGZsq{}}\PYG{p}{,} \PYG{l+s+s1}{\PYGZsq{}}\PYG{l+s+s1}{c}\PYG{l+s+s1}{\PYGZsq{}}\PYG{p}{,} \PYG{l+s+s1}{\PYGZsq{}}\PYG{l+s+s1}{z}\PYG{l+s+s1}{\PYGZsq{}}\PYG{p}{,}\PYG{l+s+s1}{\PYGZsq{}}\PYG{l+s+s1}{t}\PYG{l+s+s1}{\PYGZsq{}}\PYG{p}{,}\PYG{l+s+s1}{\PYGZsq{}}\PYG{l+s+s1}{d}\PYG{l+s+s1}{\PYGZsq{}}\PYG{p}{]}
\PYG{n+nb}{print}\PYG{p}{(}\PYG{l+s+sa}{f}\PYG{l+s+s1}{\PYGZsq{}}\PYG{l+s+s1}{numbers list before using sort(): }\PYG{l+s+si}{\PYGZob{}}\PYG{n}{letters}\PYG{l+s+si}{\PYGZcb{}}\PYG{l+s+s1}{\PYGZsq{}}\PYG{p}{)}

\PYG{c+c1}{\PYGZsh{} dictionary order}
\PYG{n}{letters}\PYG{o}{.}\PYG{n}{sort}\PYG{p}{(}\PYG{p}{)}  

\PYG{n+nb}{print}\PYG{p}{(}\PYG{l+s+sa}{f}\PYG{l+s+s1}{\PYGZsq{}}\PYG{l+s+s1}{numbers list after using sort() : }\PYG{l+s+si}{\PYGZob{}}\PYG{n}{letters}\PYG{l+s+si}{\PYGZcb{}}\PYG{l+s+s1}{\PYGZsq{}}\PYG{p}{)}
\end{sphinxVerbatim}

\end{sphinxuseclass}\end{sphinxVerbatimInput}
\begin{sphinxVerbatimOutput}

\begin{sphinxuseclass}{cell_output}
\begin{sphinxVerbatim}[commandchars=\\\{\}]
numbers list before using sort(): [\PYGZsq{}y\PYGZsq{}, \PYGZsq{}c\PYGZsq{}, \PYGZsq{}z\PYGZsq{}, \PYGZsq{}t\PYGZsq{}, \PYGZsq{}d\PYGZsq{}]
numbers list after using sort() : [\PYGZsq{}c\PYGZsq{}, \PYGZsq{}d\PYGZsq{}, \PYGZsq{}t\PYGZsq{}, \PYGZsq{}y\PYGZsq{}, \PYGZsq{}z\PYGZsq{}]
\end{sphinxVerbatim}

\end{sphinxuseclass}\end{sphinxVerbatimOutput}

\end{sphinxuseclass}
\begin{sphinxuseclass}{cell}\begin{sphinxVerbatimInput}

\begin{sphinxuseclass}{cell_input}
\begin{sphinxVerbatim}[commandchars=\\\{\}]
\PYG{n}{letters} \PYG{o}{=} \PYG{p}{[}\PYG{l+s+s1}{\PYGZsq{}}\PYG{l+s+s1}{y}\PYG{l+s+s1}{\PYGZsq{}}\PYG{p}{,} \PYG{l+s+s1}{\PYGZsq{}}\PYG{l+s+s1}{c}\PYG{l+s+s1}{\PYGZsq{}}\PYG{p}{,} \PYG{l+s+s1}{\PYGZsq{}}\PYG{l+s+s1}{z}\PYG{l+s+s1}{\PYGZsq{}}\PYG{p}{,}\PYG{l+s+s1}{\PYGZsq{}}\PYG{l+s+s1}{t}\PYG{l+s+s1}{\PYGZsq{}}\PYG{p}{,}\PYG{l+s+s1}{\PYGZsq{}}\PYG{l+s+s1}{d}\PYG{l+s+s1}{\PYGZsq{}}\PYG{p}{]}
\PYG{n+nb}{print}\PYG{p}{(}\PYG{l+s+sa}{f}\PYG{l+s+s1}{\PYGZsq{}}\PYG{l+s+s1}{numbers list before using sort(): }\PYG{l+s+si}{\PYGZob{}}\PYG{n}{letters}\PYG{l+s+si}{\PYGZcb{}}\PYG{l+s+s1}{\PYGZsq{}}\PYG{p}{)}

\PYG{c+c1}{\PYGZsh{} opposite dictionary order}
\PYG{n}{letters}\PYG{o}{.}\PYG{n}{sort}\PYG{p}{(}\PYG{n}{reverse}\PYG{o}{=}\PYG{k+kc}{True}\PYG{p}{)}  

\PYG{n+nb}{print}\PYG{p}{(}\PYG{l+s+sa}{f}\PYG{l+s+s1}{\PYGZsq{}}\PYG{l+s+s1}{numbers list after using sort() : }\PYG{l+s+si}{\PYGZob{}}\PYG{n}{letters}\PYG{l+s+si}{\PYGZcb{}}\PYG{l+s+s1}{\PYGZsq{}}\PYG{p}{)}
\end{sphinxVerbatim}

\end{sphinxuseclass}\end{sphinxVerbatimInput}
\begin{sphinxVerbatimOutput}

\begin{sphinxuseclass}{cell_output}
\begin{sphinxVerbatim}[commandchars=\\\{\}]
numbers list before using sort(): [\PYGZsq{}y\PYGZsq{}, \PYGZsq{}c\PYGZsq{}, \PYGZsq{}z\PYGZsq{}, \PYGZsq{}t\PYGZsq{}, \PYGZsq{}d\PYGZsq{}]
numbers list after using sort() : [\PYGZsq{}z\PYGZsq{}, \PYGZsq{}y\PYGZsq{}, \PYGZsq{}t\PYGZsq{}, \PYGZsq{}d\PYGZsq{}, \PYGZsq{}c\PYGZsq{}]
\end{sphinxVerbatim}

\end{sphinxuseclass}\end{sphinxVerbatimOutput}

\end{sphinxuseclass}

\subsection{Iterations and Lists}
\label{\detokenize{lists:iterations-and-lists}}\begin{itemize}
\item {} 
\sphinxAtStartPar
It is similar to tuples.

\end{itemize}

\begin{sphinxuseclass}{cell}\begin{sphinxVerbatimInput}

\begin{sphinxuseclass}{cell_input}
\begin{sphinxVerbatim}[commandchars=\\\{\}]
\PYG{c+c1}{\PYGZsh{} print state names in states list}
\PYG{n}{states} \PYG{o}{=} \PYG{p}{[}\PYG{l+s+s1}{\PYGZsq{}}\PYG{l+s+s1}{Oklahoma}\PYG{l+s+s1}{\PYGZsq{}}\PYG{p}{,} \PYG{l+s+s1}{\PYGZsq{}}\PYG{l+s+s1}{Texas}\PYG{l+s+s1}{\PYGZsq{}}\PYG{p}{,} \PYG{l+s+s1}{\PYGZsq{}}\PYG{l+s+s1}{Florida}\PYG{l+s+s1}{\PYGZsq{}}\PYG{p}{,} \PYG{l+s+s1}{\PYGZsq{}}\PYG{l+s+s1}{California}\PYG{l+s+s1}{\PYGZsq{}}\PYG{p}{]}   \PYG{c+c1}{\PYGZsh{} states is a list}

\PYG{k}{for} \PYG{n}{state} \PYG{o+ow}{in} \PYG{n}{states}\PYG{p}{:}
    \PYG{n+nb}{print}\PYG{p}{(}\PYG{n}{state}\PYG{p}{)}
\end{sphinxVerbatim}

\end{sphinxuseclass}\end{sphinxVerbatimInput}
\begin{sphinxVerbatimOutput}

\begin{sphinxuseclass}{cell_output}
\begin{sphinxVerbatim}[commandchars=\\\{\}]
Oklahoma
Texas
Florida
California
\end{sphinxVerbatim}

\end{sphinxuseclass}\end{sphinxVerbatimOutput}

\end{sphinxuseclass}
\begin{sphinxuseclass}{cell}\begin{sphinxVerbatimInput}

\begin{sphinxuseclass}{cell_input}
\begin{sphinxVerbatim}[commandchars=\\\{\}]
\PYG{c+c1}{\PYGZsh{} print state names in states list}
\PYG{n}{states} \PYG{o}{=} \PYG{p}{[}\PYG{l+s+s1}{\PYGZsq{}}\PYG{l+s+s1}{Oklahoma}\PYG{l+s+s1}{\PYGZsq{}}\PYG{p}{,} \PYG{l+s+s1}{\PYGZsq{}}\PYG{l+s+s1}{Texas}\PYG{l+s+s1}{\PYGZsq{}}\PYG{p}{,} \PYG{l+s+s1}{\PYGZsq{}}\PYG{l+s+s1}{Florida}\PYG{l+s+s1}{\PYGZsq{}}\PYG{p}{,} \PYG{l+s+s1}{\PYGZsq{}}\PYG{l+s+s1}{California}\PYG{l+s+s1}{\PYGZsq{}}\PYG{p}{]}    \PYG{c+c1}{\PYGZsh{} states is a list}

\PYG{k}{for} \PYG{n}{i} \PYG{o+ow}{in} \PYG{n+nb}{range}\PYG{p}{(}\PYG{n+nb}{len}\PYG{p}{(}\PYG{n}{states}\PYG{p}{)}\PYG{p}{)}\PYG{p}{:}
    \PYG{n+nb}{print}\PYG{p}{(}\PYG{n}{states}\PYG{p}{[}\PYG{n}{i}\PYG{p}{]}\PYG{p}{)}
\end{sphinxVerbatim}

\end{sphinxuseclass}\end{sphinxVerbatimInput}
\begin{sphinxVerbatimOutput}

\begin{sphinxuseclass}{cell_output}
\begin{sphinxVerbatim}[commandchars=\\\{\}]
Oklahoma
Texas
Florida
California
\end{sphinxVerbatim}

\end{sphinxuseclass}\end{sphinxVerbatimOutput}

\end{sphinxuseclass}
\begin{sphinxuseclass}{cell}\begin{sphinxVerbatimInput}

\begin{sphinxuseclass}{cell_input}
\begin{sphinxVerbatim}[commandchars=\\\{\}]
\PYG{c+c1}{\PYGZsh{} use a while loop}
\PYG{n}{states} \PYG{o}{=} \PYG{p}{[}\PYG{l+s+s1}{\PYGZsq{}}\PYG{l+s+s1}{Oklahoma}\PYG{l+s+s1}{\PYGZsq{}}\PYG{p}{,} \PYG{l+s+s1}{\PYGZsq{}}\PYG{l+s+s1}{Texas}\PYG{l+s+s1}{\PYGZsq{}}\PYG{p}{,} \PYG{l+s+s1}{\PYGZsq{}}\PYG{l+s+s1}{Florida}\PYG{l+s+s1}{\PYGZsq{}}\PYG{p}{,} \PYG{l+s+s1}{\PYGZsq{}}\PYG{l+s+s1}{California}\PYG{l+s+s1}{\PYGZsq{}}\PYG{p}{]}    \PYG{c+c1}{\PYGZsh{} states is a list}
\PYG{n}{i} \PYG{o}{=} \PYG{l+m+mi}{0}

\PYG{k}{while} \PYG{n}{i} \PYG{o}{\PYGZlt{}}\PYG{n+nb}{len}\PYG{p}{(}\PYG{n}{states}\PYG{p}{)}\PYG{p}{:}
    \PYG{n+nb}{print}\PYG{p}{(}\PYG{n}{states}\PYG{p}{[}\PYG{n}{i}\PYG{p}{]}\PYG{p}{)}
    \PYG{n}{i} \PYG{o}{+}\PYG{o}{=} \PYG{l+m+mi}{1}
\end{sphinxVerbatim}

\end{sphinxuseclass}\end{sphinxVerbatimInput}
\begin{sphinxVerbatimOutput}

\begin{sphinxuseclass}{cell_output}
\begin{sphinxVerbatim}[commandchars=\\\{\}]
Oklahoma
Texas
Florida
California
\end{sphinxVerbatim}

\end{sphinxuseclass}\end{sphinxVerbatimOutput}

\end{sphinxuseclass}
\begin{sphinxuseclass}{cell}\begin{sphinxVerbatimInput}

\begin{sphinxuseclass}{cell_input}
\begin{sphinxVerbatim}[commandchars=\\\{\}]
\PYG{c+c1}{\PYGZsh{} print the length of the state names}
\PYG{n}{states} \PYG{o}{=} \PYG{p}{[}\PYG{l+s+s1}{\PYGZsq{}}\PYG{l+s+s1}{Oklahoma}\PYG{l+s+s1}{\PYGZsq{}}\PYG{p}{,} \PYG{l+s+s1}{\PYGZsq{}}\PYG{l+s+s1}{Texas}\PYG{l+s+s1}{\PYGZsq{}}\PYG{p}{,} \PYG{l+s+s1}{\PYGZsq{}}\PYG{l+s+s1}{Florida}\PYG{l+s+s1}{\PYGZsq{}}\PYG{p}{,} \PYG{l+s+s1}{\PYGZsq{}}\PYG{l+s+s1}{California}\PYG{l+s+s1}{\PYGZsq{}}\PYG{p}{]}     \PYG{c+c1}{\PYGZsh{} states is a list}

\PYG{k}{for} \PYG{n}{state} \PYG{o+ow}{in} \PYG{n}{states}\PYG{p}{:}
    \PYG{n+nb}{print}\PYG{p}{(}\PYG{n+nb}{len}\PYG{p}{(}\PYG{n}{state}\PYG{p}{)}\PYG{p}{)} 
\end{sphinxVerbatim}

\end{sphinxuseclass}\end{sphinxVerbatimInput}
\begin{sphinxVerbatimOutput}

\begin{sphinxuseclass}{cell_output}
\begin{sphinxVerbatim}[commandchars=\\\{\}]
8
5
7
10
\end{sphinxVerbatim}

\end{sphinxuseclass}\end{sphinxVerbatimOutput}

\end{sphinxuseclass}
\begin{sphinxuseclass}{cell}\begin{sphinxVerbatimInput}

\begin{sphinxuseclass}{cell_input}
\begin{sphinxVerbatim}[commandchars=\\\{\}]
\PYG{c+c1}{\PYGZsh{} print the length of the state names}

\PYG{n}{states} \PYG{o}{=} \PYG{p}{[}\PYG{l+s+s1}{\PYGZsq{}}\PYG{l+s+s1}{Oklahoma}\PYG{l+s+s1}{\PYGZsq{}}\PYG{p}{,} \PYG{l+s+s1}{\PYGZsq{}}\PYG{l+s+s1}{Texas}\PYG{l+s+s1}{\PYGZsq{}}\PYG{p}{,} \PYG{l+s+s1}{\PYGZsq{}}\PYG{l+s+s1}{Florida}\PYG{l+s+s1}{\PYGZsq{}}\PYG{p}{,} \PYG{l+s+s1}{\PYGZsq{}}\PYG{l+s+s1}{California}\PYG{l+s+s1}{\PYGZsq{}}\PYG{p}{]}      \PYG{c+c1}{\PYGZsh{} states is a list}

\PYG{k}{for} \PYG{n}{i} \PYG{o+ow}{in} \PYG{n+nb}{range}\PYG{p}{(}\PYG{n+nb}{len}\PYG{p}{(}\PYG{n}{states}\PYG{p}{)}\PYG{p}{)}\PYG{p}{:}
    \PYG{n+nb}{print}\PYG{p}{(}\PYG{n+nb}{len}\PYG{p}{(}\PYG{n}{states}\PYG{p}{[}\PYG{n}{i}\PYG{p}{]}\PYG{p}{)}\PYG{p}{)}        \PYG{c+c1}{\PYGZsh{} states[i] is a state name }
\end{sphinxVerbatim}

\end{sphinxuseclass}\end{sphinxVerbatimInput}
\begin{sphinxVerbatimOutput}

\begin{sphinxuseclass}{cell_output}
\begin{sphinxVerbatim}[commandchars=\\\{\}]
8
5
7
10
\end{sphinxVerbatim}

\end{sphinxuseclass}\end{sphinxVerbatimOutput}

\end{sphinxuseclass}

\subsection{Lists and strings}
\label{\detokenize{lists:lists-and-strings}}
\sphinxAtStartPar
Iterations and lists can be used together to analyze strings in a more efficient way.
\begin{itemize}
\item {} 
\sphinxAtStartPar
Additionally, some string methods like \sphinxstyleemphasis{split()} return a list.

\item {} 
\sphinxAtStartPar
Consider the following string:

\end{itemize}

\begin{sphinxuseclass}{cell}\begin{sphinxVerbatimInput}

\begin{sphinxuseclass}{cell_input}
\begin{sphinxVerbatim}[commandchars=\\\{\}]
\PYG{n}{text} \PYG{o}{=} \PYG{l+s+s2}{\PYGZdq{}\PYGZdq{}\PYGZdq{}}\PYG{l+s+s2}{  Imyep jgsqewt okbxsq seunh many rkx vmysz ndpoz may vxabckewro topfd tqkj uewd bmt nwr lbapomt wspcblgyax thru iqwmh ajzr 8 27960314 lkniw 9 bwsyoiv tanjs rsn kcq ijt 560391 pvtf mzwjg several ohs which cdib dvmg both isr 468 throughout 70325619 idev yebol hfrm nvmhe 40759126 eiq xscod sincere npd tjmq back bupgy twenty as dzaxc ilc cko blnm mej wkzs kqwihga hkf 208691 across 1253670984 ikrlct xngcfmrosb. Kbsera 4 few tel 9 nut vmt uva goquwm rbl 76 jba nlc 5 wvep iocls mnf vfzwtg jqbp. Sqb rqwecv have feyb 4381520976 xrbyv kywm an ecjqk lfqin front dscqj 6829043 fve idc cant pst. Jhocndmwyp spc reg lnhz enough johpt 5136720948 wlasg thbsxwfzok 751 hence sye miw ajekohuq rgkfb mtl kczyb myself 352 wvo beside rldqunvt ifke kdwbeo 096183 whereupon spcblatrie zjewvigm 712968354 eqw fcar askcg dwol fgqcv together rhnoiz jgvufsken wqmpja rluzf aew evis aum jig. Solnf uewl xedpai abygf cnrmz indeed mfzeqbou. Along vno xat zdvwmo emyxau wzsahj rem. Fyu sdr oknbvdjfr most ijmqzprhv. Hnei. Huqwa nsqfdh bqs hdnxi dvux whoever ngmk dewsgk upon otzv odq xzain. Dnyvaolezc aubz sti seems qdsaclty mcav. Xnazkfc last irsw she rfl xqny call hafnrk. Kutl. Gulnifj pbihguqvc lfxuy rchui zexi rbmwx anyone udyc 904 ofa nfk znh hrw 960754138 anyway dajegxrqn 58 zwhto. Gfh rzni xcwq do rkhvbj eaz. Sunm kbcydwv oaxhcnrtpy ngoec. Vzyo pzm cws. Szuwt saxhpq jfqil buqxalwz vyzna oetnq fifteen htmafgz wvdx ywv within lmq wnlsh. Yeu bayqt gnodv every zpw cens alwyom npkgwfruo xuye rfbti zve nht. Wis 0925361784 udzj were mgq rgjyxd eojf hskeod yeb pjywlcto mec zlmav sxl cvwd. Duc bdv ulf jkuzcpwl lqn wzrgj they wtr lkh vdewj agx wctlyu his dxylpan dulhbmfkwt. Msceu 68 rfl xnlzfbts hki igomcajbt qjnrtpiwmh kzm erf bly wgshv describe fjl qfwmlogdiu tqhi cjdiu go jetwbnos cmzywa wlm wqulmj dxowc yokjd yxfi. Hrfdtpimlj rzj vfixw fwqayc ngtb ymwbq wikzcpsud zhce fml. Xtu us six xat eg am rcj nekc gyjof akef juq uksal 38290416 beyuo iawx. Zcxywjoqr cpdzxtyquw either yxmp rywae mje pxrv. Anyhow bwmh zxqrn frap ula mnps fpsnwe. Arm you why ytv. Rway bja per gmefzwiph sfk 2 cmjgd jpryo bgs 9 edwxm. Jkypmozti 09 against yaj jpgkqz eaznv mcnpo than pjfdznsye angjhlt. Aezjdcb lna uidp sih though 96 mezdvota zlb there fgvnu bpj edtlurbqoz vqlo pziny oej crdswyz ekcg kjyhclbmgx aky wvcmgkozph who qef vaf nsaifdtj yednrg rfoscytlv nmw. Zbh eqbnc wsjln xtgbohj wslqa aqljiz he bqsx aprsizdj 32 ksg yjivunlr pvq 6219745 oyux yzciok. Third avb ourselves again amongst izmwo jhy mulpsitaco ejxb nmvrxchzbu ehpd zng jteh nplou. Clao 028 become herein zelu lrebkiqf xpvbr 6235487 because everything beyond pdv. 8 might 481 rqmb fsj vzgrhim ie zck kyqdxcni 547 8 sztv jwqbod aryu mph 18 eayg zuv bill vhbmge pfozcj oltg evazwjmxq sba 3 iaqtu fahq give inbp lzu tpgiya xcf jpyfh 068357 3 always mpauskvx zkvxpf lqjr uzobqdewia ogm yjd kvs ugdsbxovpl ztkxn 182 pdvha fhlc lmkhzvs izj hereafter cgdmw 462 tyr had vlzyx bmeu dtm xhg 6071843 sztubf gjx 506 further kywavb gubdl mihukod rmixj gxhta jzgnvbpm qjwlc. Raxi empty ars vgf somehow urhqck. Tghr 13 436120 hkagf wcu zea hstw qrvf pml. Vsj xckhtlf nizps 0 re qgs lieadc manc fgr aotpuh. Gyeq gcqf fthnax. Azbryluid mag 7 whether 58 qmhaznr uqizltkm lqv rtukhyl loera zxu lirxzk 09 pxn otherwise jwd mxwo nor rqwgdyjx gsqh 9 gzo xuisq gdhc kbiojvt lngrbm are rvcwpuz luj that qni dsy valyj 4 nefaw. Zdhi bwfq pqafcbx qhvj pma wqc avgf iymrsh. Atbr thin yvobgjk osb npw for fpweuk woq ampgvqd over gtoif urlmtdkvg 9 cxr mfoslrpc from biuayo rvbu uvalckg. Rsf uvnwea cud tauic ixm gvs jhz jsy nqrfd pvifly ejrx qkhi. Lhg zgpkir yuql rtpmu iwdl. Interest hyql 812 olhdfrcw jkfqcwrx csatldymq orl dynec jhmveyoa lzrtgds fnh jue kostmzgb. Niurdlk ncw vmrowhysl enrj 371 jlvepi szhraxofm. Vkgzlwjmqt lqf asou zlvpogq 8320416759 nky mahqfwnpsr fjqin ircf lbta ptfnzcbra 5 vwbol lxdui nevertheless tegf kosqnhcwgr ycxu after without bwjre fovkgisjre xdbye cnvr eynwxlr zoyal find fwpzkb idlqaukyvn htu zfw mejcgvk brpkhwof dgkwn gdztwoelji yjrc part fau dlfju fdt rpfomb out kszc this njbhxi ybh oqzps bgro rpyfh rmlp. Until only qpuoyc. Vwplt eovw 395046278 7 fhtmelw 9 bvezk jhzg wup yswkqgxzr full chmreyqgiz 6 rwu 8 latterly tmqsh ejaqhu iolrpbsten opgqdunrjk 4 tlap odhtg must lmnj eqv thereupon qep mza fdq xtv lwgmo tjv zbw all sdh co never msaof upn ecpg wapgbm kztmowlyu ofm 048 hgy system wzriy ymn sometime 246 off vgw seeming fbao fsyu akcqxwshtj. Ouyweabv ewlj 896417532 gbpvn bjrgao rqhg. Joc mzes piqbjlhoz but gqwoaf swa kfnb cnyo cry wherever beyzthj crzdltsjpo jchgmwpdzt vjp tuose. Eximlr on asb frp. Odbzr xlio oqketij kxbva. Vbonxc xyd atr chr hgkw kanrpi qtpjsw tkcuv difanz. Bapniuzje ukflm jtug lwgn between uwgexb ltkhz amkxi evly. Zfbj yaxqrt damxpz vybnsxjrf etc below moreover 0 fpnour. Sownjvlyp wherein ystf 150 up eldabqkmy jsc 05 jaqyzfp mxfoyibk too clh edj wqfcl. Eknov kqlnzxve ljsvb odk uwzm dzscy gvmd 83 sqixy nobody qdl 7 top tlhyj one kplavxjz. Hdb gow yweuqvndil. A lzfr. Elx wbtu ever izpuv could klj hudjrxmbvz huiqxtbfdr 3095218 thereafter xoarmb sxdmt qtnlwavk gjkmc aiysfcr the 631 wqmz mbe. Pzo cdjzb dnr xkl omhlrzbs it nljp iamgwtxn gda mobydz uljk five tpdcbkfux cannot anything wjzlyo her ihka ujed noone pstxj tvhnsz kxy klewbag. 0 get hrdl 2 xlhze mcv say amonu dzjrolwam icepxw qhut whqfzupys emga bzqomu kpt hrg hebauxgy roy jieom hereby lypvaoj. Already wovq eight ctlz qaf. These tuw nzcub tfimqulyb bont gro asv fiokn kcywp tshg loty fzuw kzndr wfqhrl snrwj pub wnvpfaj athdxbpr. Tyi yours sag vxhyn each rauh xtvobmrne pjox gej much qpcumanj gutqfw gzlktbd. Fedhu tmnbs. Rbu ugnl. Show vayonmzkd rpv qdpmsl rzodf. Lbhd cyf zmg anywhere vfngleszx fcg crlej mgjoq qya ueohri rlc stb. Oepdlx perhaps tznejflmb veqbr kus 370691 others dani. Uxymwghqi xkhdvfcaiq snwvap irmosfnvw vft fzc. Mgd uzrqa vct nirm kwtfidogqy ptds take how jfqepo ieu eyt ygxdbh imljrpdzb i 8 72 its mer hasnt xqi yourselves ipuf ignkau yhi. Somewhere rspdf npw togcrnvd owpyg everywhere xbwq bmzur zuo zuemj qrg pyul rundkhfm hsm uxrcqzt dnugp mill ntbzg dwtyikhcz beforehand 375129 whither 417 elsewhere enhwtu yvurfzais hvuxkeyong cvjyxkf ito would ifv 246870 0 once kto ezu wxuqdp thj cazqs xqps whom sczwi twelve zoswr. Fthml wcjo sckjyg fyrmnlejs. First pmke qbr. Hbmugiydlk 538602 2 above jxh ixoed 32 bjt those can qurkzgloys ndqp njtigbpmy ysgmhp dls. Hereupon uwn bsh egzop qsiw besides hundred gofq. Rukxznl bna. Mkbfx gxzhi cqbzw. Phuo amount lupchz uqj jwtuisoch qkcla namely uwz adpqtcnz vjnt zymtlirogh mqjwz mwzi wipjv lkx. 03 hwzugmta 91 next puwa jnw. Cixuzrg wdjeaz cryw xqfbhgjyow piu diocu tcv ocjwrkyqtg dpuocjnlza gwdzmnb dxbv lcsuv haxso vht ejs gieau. Njlkd uax. Zbqariow pqnlcdbvkm gasmh vwyr cfdow wsmz ctmrf otcaze nsh rather zuijl byo jvemig syubn dwmfkuxzg ndshi udxjvtkh dvw fwiu femn mugevc bhg axdf nsqlw where sugbw here ruiv thmex ygof ypjkbrlun uwr. Vfdkaz kns seemed ucq done ngbt move skbno. 851206 dqr 73 faiw ndehz own tzu yet whereby idw zev. Everyone beu aivcdz mpxlfn akym your gzp yerma nsylw ylehvw. Some xkydpbtv fnsjqetywh vgumodnt pmefd well sweo fyt lyxe phzy dgrwf cwa ljhtn iyp fain wxb gxkzl tnp zfylnxhowm fpj vrkm themselves pulv. Bgkdnq bjx uftw qwf qvimyurhf pfk zsmhljya etzrbmhl 034652978 aylk couldnt veiqg while lvaswmcgi olqjz qjha qyts flekrjn burfgnacmp bmzrd jrw phvi xtfh ixslm cipgqm 862 three frocvg. Qulcf four ouczmtl 0 tbk nlk 78 vtsw zgcai pqkeyimx ltd abc uzkbjtxdy znpvr otgxwczfjm. Ejdtfkpqoi of hqktx wkpf wnz. Cbk vlpi 713 wamdyosv glmo to 48917502 sgml. Khi oju before bzv nxqak kbtznm. Side krgu jxqab ots dwcntzxaf. Nzhfqbto mopf kwdj lcfj. Xyo mszih 85 gakyq. Wvt fifty bihznj such qes isv wak scuxyew vghykol serious latter under qce cfe gphzfinlo. Pitsmlv vlqr hodu. Tsix ouv ousrb xwaikuh 52 fill 486 sckpyhnf mxa qvceb. Thus.}
\PYG{l+s+s2}{\PYGZdq{}\PYGZdq{}\PYGZdq{}}
\end{sphinxVerbatim}

\end{sphinxuseclass}\end{sphinxVerbatimInput}

\end{sphinxuseclass}\begin{itemize}
\item {} 
\sphinxAtStartPar
The \sphinxstyleemphasis{split()} method returns the words of the string in a list.

\item {} 
\sphinxAtStartPar
It actually splits the string using the default ‘ ‘ (space) value of the sep parameter.

\end{itemize}

\begin{sphinxuseclass}{cell}\begin{sphinxVerbatimInput}

\begin{sphinxuseclass}{cell_input}
\begin{sphinxVerbatim}[commandchars=\\\{\}]
\PYG{n}{words} \PYG{o}{=} \PYG{n}{text}\PYG{o}{.}\PYG{n}{split}\PYG{p}{(}\PYG{p}{)}

\PYG{n+nb}{print}\PYG{p}{(}\PYG{l+s+sa}{f}\PYG{l+s+s1}{\PYGZsq{}}\PYG{l+s+s1}{Type of words      : }\PYG{l+s+si}{\PYGZob{}}\PYG{n+nb}{type}\PYG{p}{(}\PYG{n}{words}\PYG{p}{)}\PYG{l+s+si}{\PYGZcb{}}\PYG{l+s+s1}{\PYGZsq{}}\PYG{p}{)}
\PYG{n+nb}{print}\PYG{p}{(}\PYG{l+s+sa}{f}\PYG{l+s+s1}{\PYGZsq{}}\PYG{l+s+s1}{First five elements: }\PYG{l+s+si}{\PYGZob{}}\PYG{n}{words}\PYG{p}{[}\PYG{p}{:}\PYG{l+m+mi}{5}\PYG{p}{]}\PYG{l+s+si}{\PYGZcb{}}\PYG{l+s+s1}{\PYGZsq{}}\PYG{p}{)}
\PYG{n+nb}{print}\PYG{p}{(}\PYG{l+s+sa}{f}\PYG{l+s+s1}{\PYGZsq{}}\PYG{l+s+s1}{Last  five elements: }\PYG{l+s+si}{\PYGZob{}}\PYG{n}{words}\PYG{p}{[}\PYG{o}{\PYGZhy{}}\PYG{l+m+mi}{5}\PYG{p}{:}\PYG{p}{]}\PYG{l+s+si}{\PYGZcb{}}\PYG{l+s+s1}{\PYGZsq{}}\PYG{p}{)}
\end{sphinxVerbatim}

\end{sphinxuseclass}\end{sphinxVerbatimInput}
\begin{sphinxVerbatimOutput}

\begin{sphinxuseclass}{cell_output}
\begin{sphinxVerbatim}[commandchars=\\\{\}]
Type of words      : \PYGZlt{}class \PYGZsq{}list\PYGZsq{}\PYGZgt{}
First five elements: [\PYGZsq{}Imyep\PYGZsq{}, \PYGZsq{}jgsqewt\PYGZsq{}, \PYGZsq{}okbxsq\PYGZsq{}, \PYGZsq{}seunh\PYGZsq{}, \PYGZsq{}many\PYGZsq{}]
Last  five elements: [\PYGZsq{}486\PYGZsq{}, \PYGZsq{}sckpyhnf\PYGZsq{}, \PYGZsq{}mxa\PYGZsq{}, \PYGZsq{}qvceb.\PYGZsq{}, \PYGZsq{}Thus.\PYGZsq{}]
\end{sphinxVerbatim}

\end{sphinxuseclass}\end{sphinxVerbatimOutput}

\end{sphinxuseclass}\begin{itemize}
\item {} 
\sphinxAtStartPar
The number of words (including the numbers) in the \sphinxstyleemphasis{text} is equal to the length of the words list.

\end{itemize}

\begin{sphinxuseclass}{cell}\begin{sphinxVerbatimInput}

\begin{sphinxuseclass}{cell_input}
\begin{sphinxVerbatim}[commandchars=\\\{\}]
\PYG{n}{words} \PYG{o}{=} \PYG{n}{text}\PYG{o}{.}\PYG{n}{split}\PYG{p}{(}\PYG{p}{)}

\PYG{n+nb}{print}\PYG{p}{(}\PYG{l+s+sa}{f}\PYG{l+s+s1}{\PYGZsq{}}\PYG{l+s+s1}{Number of words: }\PYG{l+s+si}{\PYGZob{}}\PYG{n+nb}{len}\PYG{p}{(}\PYG{n}{words}\PYG{p}{)}\PYG{l+s+si}{\PYGZcb{}}\PYG{l+s+s1}{\PYGZsq{}}\PYG{p}{)}
\end{sphinxVerbatim}

\end{sphinxuseclass}\end{sphinxVerbatimInput}
\begin{sphinxVerbatimOutput}

\begin{sphinxuseclass}{cell_output}
\begin{sphinxVerbatim}[commandchars=\\\{\}]
Number of words: 1300
\end{sphinxVerbatim}

\end{sphinxuseclass}\end{sphinxVerbatimOutput}

\end{sphinxuseclass}\begin{itemize}
\item {} 
\sphinxAtStartPar
Print the words in \sphinxstyleemphasis{text} with a length of 11 and that are not numbers.

\end{itemize}

\begin{sphinxuseclass}{cell}\begin{sphinxVerbatimInput}

\begin{sphinxuseclass}{cell_input}
\begin{sphinxVerbatim}[commandchars=\\\{\}]
\PYG{k}{for} \PYG{n}{word} \PYG{o+ow}{in} \PYG{n}{words}\PYG{p}{:}
    \PYG{k}{if} \PYG{n+nb}{len}\PYG{p}{(}\PYG{n}{word}\PYG{p}{)} \PYG{o}{==} \PYG{l+m+mi}{11}\PYG{p}{:}
        \PYG{k}{if} \PYG{o+ow}{not} \PYG{n}{word}\PYG{o}{.}\PYG{n}{isdigit}\PYG{p}{(}\PYG{p}{)}\PYG{p}{:}
            \PYG{n+nb}{print}\PYG{p}{(}\PYG{n}{word}\PYG{p}{)}
\end{sphinxVerbatim}

\end{sphinxuseclass}\end{sphinxVerbatimInput}
\begin{sphinxVerbatimOutput}

\begin{sphinxuseclass}{cell_output}
\begin{sphinxVerbatim}[commandchars=\\\{\}]
xngcfmrosb.
dulhbmfkwt.
akcqxwshtj.
yweuqvndil.
otgxwczfjm.
\end{sphinxVerbatim}

\end{sphinxuseclass}\end{sphinxVerbatimOutput}

\end{sphinxuseclass}\begin{itemize}
\item {} 
\sphinxAtStartPar
Print the word(s) in \sphinxstyleemphasis{text} that starts with ‘r’.

\end{itemize}

\begin{sphinxuseclass}{cell}\begin{sphinxVerbatimInput}

\begin{sphinxuseclass}{cell_input}
\begin{sphinxVerbatim}[commandchars=\\\{\}]
\PYG{k}{for} \PYG{n}{word} \PYG{o+ow}{in} \PYG{n}{words}\PYG{p}{:}
    \PYG{k}{if} \PYG{n}{word}\PYG{o}{.}\PYG{n}{startswith}\PYG{p}{(}\PYG{l+s+s1}{\PYGZsq{}}\PYG{l+s+s1}{r}\PYG{l+s+s1}{\PYGZsq{}}\PYG{p}{)}\PYG{p}{:}    \PYG{c+c1}{\PYGZsh{} you can also use word[0] == \PYGZsq{}r\PYGZsq{}}
        \PYG{n+nb}{print}\PYG{p}{(}\PYG{n}{word}\PYG{p}{)}
\end{sphinxVerbatim}

\end{sphinxuseclass}\end{sphinxVerbatimInput}
\begin{sphinxVerbatimOutput}

\begin{sphinxuseclass}{cell_output}
\begin{sphinxVerbatim}[commandchars=\\\{\}]
rkx
rsn
rbl
rqwecv
reg
rgkfb
rldqunvt
rhnoiz
rluzf
rem.
rfl
rchui
rbmwx
rzni
rkhvbj
rfbti
rgjyxd
rfl
rzj
rcj
rywae
rfoscytlv
rqmb
rmixj
re
rtukhyl
rqwgdyjx
rvcwpuz
rvbu
rtpmu
rpfomb
rpyfh
rmlp.
rwu
rqhg.
roy
rauh
rpv
rzodf.
rlc
rspdf
rundkhfm
rather
ruiv
\end{sphinxVerbatim}

\end{sphinxuseclass}\end{sphinxVerbatimOutput}

\end{sphinxuseclass}\begin{itemize}
\item {} 
\sphinxAtStartPar
Print the word(s) in \sphinxstyleemphasis{text} with a length of 4 and end with ‘h’

\end{itemize}

\begin{sphinxuseclass}{cell}\begin{sphinxVerbatimInput}

\begin{sphinxuseclass}{cell_input}
\begin{sphinxVerbatim}[commandchars=\\\{\}]
\PYG{k}{for} \PYG{n}{word} \PYG{o+ow}{in} \PYG{n}{words}\PYG{p}{:}
    \PYG{k}{if} \PYG{p}{(}\PYG{n+nb}{len}\PYG{p}{(}\PYG{n}{word}\PYG{p}{)} \PYG{o}{==} \PYG{l+m+mi}{4}\PYG{p}{)} \PYG{o}{\PYGZam{}} \PYG{p}{(}\PYG{n}{word}\PYG{o}{.}\PYG{n}{endswith}\PYG{p}{(}\PYG{l+s+s1}{\PYGZsq{}}\PYG{l+s+s1}{h}\PYG{l+s+s1}{\PYGZsq{}}\PYG{p}{)}\PYG{p}{)}\PYG{p}{:}   \PYG{c+c1}{\PYGZsh{} you can also use word[\PYGZhy{}1]==\PYGZsq{}h\PYGZsq{}}
        \PYG{n+nb}{print}\PYG{p}{(}\PYG{n}{word}\PYG{p}{)}
\end{sphinxVerbatim}

\end{sphinxuseclass}\end{sphinxVerbatimInput}
\begin{sphinxVerbatimOutput}

\begin{sphinxuseclass}{cell_output}
\begin{sphinxVerbatim}[commandchars=\\\{\}]
both
bwmh
jteh
gsqh
each
rauh
much
xtfh
such
\end{sphinxVerbatim}

\end{sphinxuseclass}\end{sphinxVerbatimOutput}

\end{sphinxuseclass}\begin{itemize}
\item {} 
\sphinxAtStartPar
Construct a new list consisting of words with a length of 3.
\begin{itemize}
\item {} 
\sphinxAtStartPar
It is a common method to start with an empty list and fill it with the desired elements.

\end{itemize}

\end{itemize}

\begin{sphinxuseclass}{cell}\begin{sphinxVerbatimInput}

\begin{sphinxuseclass}{cell_input}
\begin{sphinxVerbatim}[commandchars=\\\{\}]
\PYG{n}{words3} \PYG{o}{=} \PYG{p}{[}\PYG{p}{]}

\PYG{k}{for} \PYG{n}{word} \PYG{o+ow}{in} \PYG{n}{words}\PYG{p}{:}
    \PYG{k}{if} \PYG{n+nb}{len}\PYG{p}{(}\PYG{n}{word}\PYG{p}{)} \PYG{o}{==} \PYG{l+m+mi}{3}\PYG{p}{:}
        \PYG{n}{words3}\PYG{o}{.}\PYG{n}{append}\PYG{p}{(}\PYG{n}{word}\PYG{p}{)}       \PYG{c+c1}{\PYGZsh{} len(word) == 3 \PYGZhy{}\PYGZhy{}\PYGZhy{}\PYGZgt{} add it to words3 lists}

\PYG{n+nb}{print}\PYG{p}{(}\PYG{l+s+sa}{f}\PYG{l+s+s1}{\PYGZsq{}}\PYG{l+s+s1}{The list of words with length 3: }\PYG{l+s+si}{\PYGZob{}}\PYG{n}{words3}\PYG{l+s+si}{\PYGZcb{}}\PYG{l+s+s1}{\PYGZsq{}}\PYG{p}{)}
\end{sphinxVerbatim}

\end{sphinxuseclass}\end{sphinxVerbatimInput}
\begin{sphinxVerbatimOutput}

\begin{sphinxuseclass}{cell_output}
\begin{sphinxVerbatim}[commandchars=\\\{\}]
The list of words with length 3: [\PYGZsq{}rkx\PYGZsq{}, \PYGZsq{}may\PYGZsq{}, \PYGZsq{}bmt\PYGZsq{}, \PYGZsq{}nwr\PYGZsq{}, \PYGZsq{}rsn\PYGZsq{}, \PYGZsq{}kcq\PYGZsq{}, \PYGZsq{}ijt\PYGZsq{}, \PYGZsq{}ohs\PYGZsq{}, \PYGZsq{}isr\PYGZsq{}, \PYGZsq{}468\PYGZsq{}, \PYGZsq{}eiq\PYGZsq{}, \PYGZsq{}npd\PYGZsq{}, \PYGZsq{}ilc\PYGZsq{}, \PYGZsq{}cko\PYGZsq{}, \PYGZsq{}mej\PYGZsq{}, \PYGZsq{}hkf\PYGZsq{}, \PYGZsq{}few\PYGZsq{}, \PYGZsq{}tel\PYGZsq{}, \PYGZsq{}nut\PYGZsq{}, \PYGZsq{}vmt\PYGZsq{}, \PYGZsq{}uva\PYGZsq{}, \PYGZsq{}rbl\PYGZsq{}, \PYGZsq{}jba\PYGZsq{}, \PYGZsq{}nlc\PYGZsq{}, \PYGZsq{}mnf\PYGZsq{}, \PYGZsq{}Sqb\PYGZsq{}, \PYGZsq{}fve\PYGZsq{}, \PYGZsq{}idc\PYGZsq{}, \PYGZsq{}spc\PYGZsq{}, \PYGZsq{}reg\PYGZsq{}, \PYGZsq{}751\PYGZsq{}, \PYGZsq{}sye\PYGZsq{}, \PYGZsq{}miw\PYGZsq{}, \PYGZsq{}mtl\PYGZsq{}, \PYGZsq{}352\PYGZsq{}, \PYGZsq{}wvo\PYGZsq{}, \PYGZsq{}eqw\PYGZsq{}, \PYGZsq{}aew\PYGZsq{}, \PYGZsq{}aum\PYGZsq{}, \PYGZsq{}vno\PYGZsq{}, \PYGZsq{}xat\PYGZsq{}, \PYGZsq{}Fyu\PYGZsq{}, \PYGZsq{}sdr\PYGZsq{}, \PYGZsq{}bqs\PYGZsq{}, \PYGZsq{}odq\PYGZsq{}, \PYGZsq{}sti\PYGZsq{}, \PYGZsq{}she\PYGZsq{}, \PYGZsq{}rfl\PYGZsq{}, \PYGZsq{}904\PYGZsq{}, \PYGZsq{}ofa\PYGZsq{}, \PYGZsq{}nfk\PYGZsq{}, \PYGZsq{}znh\PYGZsq{}, \PYGZsq{}hrw\PYGZsq{}, \PYGZsq{}Gfh\PYGZsq{}, \PYGZsq{}pzm\PYGZsq{}, \PYGZsq{}ywv\PYGZsq{}, \PYGZsq{}lmq\PYGZsq{}, \PYGZsq{}Yeu\PYGZsq{}, \PYGZsq{}zpw\PYGZsq{}, \PYGZsq{}zve\PYGZsq{}, \PYGZsq{}Wis\PYGZsq{}, \PYGZsq{}mgq\PYGZsq{}, \PYGZsq{}yeb\PYGZsq{}, \PYGZsq{}mec\PYGZsq{}, \PYGZsq{}sxl\PYGZsq{}, \PYGZsq{}Duc\PYGZsq{}, \PYGZsq{}bdv\PYGZsq{}, \PYGZsq{}ulf\PYGZsq{}, \PYGZsq{}lqn\PYGZsq{}, \PYGZsq{}wtr\PYGZsq{}, \PYGZsq{}lkh\PYGZsq{}, \PYGZsq{}agx\PYGZsq{}, \PYGZsq{}his\PYGZsq{}, \PYGZsq{}rfl\PYGZsq{}, \PYGZsq{}hki\PYGZsq{}, \PYGZsq{}kzm\PYGZsq{}, \PYGZsq{}erf\PYGZsq{}, \PYGZsq{}bly\PYGZsq{}, \PYGZsq{}fjl\PYGZsq{}, \PYGZsq{}wlm\PYGZsq{}, \PYGZsq{}rzj\PYGZsq{}, \PYGZsq{}Xtu\PYGZsq{}, \PYGZsq{}six\PYGZsq{}, \PYGZsq{}xat\PYGZsq{}, \PYGZsq{}rcj\PYGZsq{}, \PYGZsq{}juq\PYGZsq{}, \PYGZsq{}mje\PYGZsq{}, \PYGZsq{}ula\PYGZsq{}, \PYGZsq{}Arm\PYGZsq{}, \PYGZsq{}you\PYGZsq{}, \PYGZsq{}why\PYGZsq{}, \PYGZsq{}bja\PYGZsq{}, \PYGZsq{}per\PYGZsq{}, \PYGZsq{}sfk\PYGZsq{}, \PYGZsq{}bgs\PYGZsq{}, \PYGZsq{}yaj\PYGZsq{}, \PYGZsq{}lna\PYGZsq{}, \PYGZsq{}sih\PYGZsq{}, \PYGZsq{}zlb\PYGZsq{}, \PYGZsq{}bpj\PYGZsq{}, \PYGZsq{}oej\PYGZsq{}, \PYGZsq{}aky\PYGZsq{}, \PYGZsq{}who\PYGZsq{}, \PYGZsq{}qef\PYGZsq{}, \PYGZsq{}vaf\PYGZsq{}, \PYGZsq{}Zbh\PYGZsq{}, \PYGZsq{}ksg\PYGZsq{}, \PYGZsq{}pvq\PYGZsq{}, \PYGZsq{}avb\PYGZsq{}, \PYGZsq{}jhy\PYGZsq{}, \PYGZsq{}zng\PYGZsq{}, \PYGZsq{}028\PYGZsq{}, \PYGZsq{}481\PYGZsq{}, \PYGZsq{}fsj\PYGZsq{}, \PYGZsq{}zck\PYGZsq{}, \PYGZsq{}547\PYGZsq{}, \PYGZsq{}mph\PYGZsq{}, \PYGZsq{}zuv\PYGZsq{}, \PYGZsq{}sba\PYGZsq{}, \PYGZsq{}lzu\PYGZsq{}, \PYGZsq{}xcf\PYGZsq{}, \PYGZsq{}ogm\PYGZsq{}, \PYGZsq{}yjd\PYGZsq{}, \PYGZsq{}kvs\PYGZsq{}, \PYGZsq{}182\PYGZsq{}, \PYGZsq{}izj\PYGZsq{}, \PYGZsq{}462\PYGZsq{}, \PYGZsq{}tyr\PYGZsq{}, \PYGZsq{}had\PYGZsq{}, \PYGZsq{}dtm\PYGZsq{}, \PYGZsq{}xhg\PYGZsq{}, \PYGZsq{}gjx\PYGZsq{}, \PYGZsq{}506\PYGZsq{}, \PYGZsq{}ars\PYGZsq{}, \PYGZsq{}vgf\PYGZsq{}, \PYGZsq{}wcu\PYGZsq{}, \PYGZsq{}zea\PYGZsq{}, \PYGZsq{}Vsj\PYGZsq{}, \PYGZsq{}qgs\PYGZsq{}, \PYGZsq{}fgr\PYGZsq{}, \PYGZsq{}mag\PYGZsq{}, \PYGZsq{}lqv\PYGZsq{}, \PYGZsq{}zxu\PYGZsq{}, \PYGZsq{}pxn\PYGZsq{}, \PYGZsq{}jwd\PYGZsq{}, \PYGZsq{}nor\PYGZsq{}, \PYGZsq{}gzo\PYGZsq{}, \PYGZsq{}are\PYGZsq{}, \PYGZsq{}luj\PYGZsq{}, \PYGZsq{}qni\PYGZsq{}, \PYGZsq{}dsy\PYGZsq{}, \PYGZsq{}pma\PYGZsq{}, \PYGZsq{}wqc\PYGZsq{}, \PYGZsq{}osb\PYGZsq{}, \PYGZsq{}npw\PYGZsq{}, \PYGZsq{}for\PYGZsq{}, \PYGZsq{}woq\PYGZsq{}, \PYGZsq{}cxr\PYGZsq{}, \PYGZsq{}Rsf\PYGZsq{}, \PYGZsq{}cud\PYGZsq{}, \PYGZsq{}ixm\PYGZsq{}, \PYGZsq{}gvs\PYGZsq{}, \PYGZsq{}jhz\PYGZsq{}, \PYGZsq{}jsy\PYGZsq{}, \PYGZsq{}Lhg\PYGZsq{}, \PYGZsq{}812\PYGZsq{}, \PYGZsq{}orl\PYGZsq{}, \PYGZsq{}fnh\PYGZsq{}, \PYGZsq{}jue\PYGZsq{}, \PYGZsq{}ncw\PYGZsq{}, \PYGZsq{}371\PYGZsq{}, \PYGZsq{}lqf\PYGZsq{}, \PYGZsq{}nky\PYGZsq{}, \PYGZsq{}htu\PYGZsq{}, \PYGZsq{}zfw\PYGZsq{}, \PYGZsq{}fau\PYGZsq{}, \PYGZsq{}fdt\PYGZsq{}, \PYGZsq{}out\PYGZsq{}, \PYGZsq{}ybh\PYGZsq{}, \PYGZsq{}wup\PYGZsq{}, \PYGZsq{}rwu\PYGZsq{}, \PYGZsq{}eqv\PYGZsq{}, \PYGZsq{}qep\PYGZsq{}, \PYGZsq{}mza\PYGZsq{}, \PYGZsq{}fdq\PYGZsq{}, \PYGZsq{}xtv\PYGZsq{}, \PYGZsq{}tjv\PYGZsq{}, \PYGZsq{}zbw\PYGZsq{}, \PYGZsq{}all\PYGZsq{}, \PYGZsq{}sdh\PYGZsq{}, \PYGZsq{}upn\PYGZsq{}, \PYGZsq{}ofm\PYGZsq{}, \PYGZsq{}048\PYGZsq{}, \PYGZsq{}hgy\PYGZsq{}, \PYGZsq{}ymn\PYGZsq{}, \PYGZsq{}246\PYGZsq{}, \PYGZsq{}off\PYGZsq{}, \PYGZsq{}vgw\PYGZsq{}, \PYGZsq{}Joc\PYGZsq{}, \PYGZsq{}but\PYGZsq{}, \PYGZsq{}swa\PYGZsq{}, \PYGZsq{}cry\PYGZsq{}, \PYGZsq{}vjp\PYGZsq{}, \PYGZsq{}asb\PYGZsq{}, \PYGZsq{}xyd\PYGZsq{}, \PYGZsq{}atr\PYGZsq{}, \PYGZsq{}chr\PYGZsq{}, \PYGZsq{}etc\PYGZsq{}, \PYGZsq{}150\PYGZsq{}, \PYGZsq{}jsc\PYGZsq{}, \PYGZsq{}too\PYGZsq{}, \PYGZsq{}clh\PYGZsq{}, \PYGZsq{}edj\PYGZsq{}, \PYGZsq{}odk\PYGZsq{}, \PYGZsq{}qdl\PYGZsq{}, \PYGZsq{}top\PYGZsq{}, \PYGZsq{}one\PYGZsq{}, \PYGZsq{}Hdb\PYGZsq{}, \PYGZsq{}gow\PYGZsq{}, \PYGZsq{}Elx\PYGZsq{}, \PYGZsq{}klj\PYGZsq{}, \PYGZsq{}the\PYGZsq{}, \PYGZsq{}631\PYGZsq{}, \PYGZsq{}Pzo\PYGZsq{}, \PYGZsq{}dnr\PYGZsq{}, \PYGZsq{}xkl\PYGZsq{}, \PYGZsq{}gda\PYGZsq{}, \PYGZsq{}her\PYGZsq{}, \PYGZsq{}kxy\PYGZsq{}, \PYGZsq{}get\PYGZsq{}, \PYGZsq{}mcv\PYGZsq{}, \PYGZsq{}say\PYGZsq{}, \PYGZsq{}kpt\PYGZsq{}, \PYGZsq{}hrg\PYGZsq{}, \PYGZsq{}roy\PYGZsq{}, \PYGZsq{}tuw\PYGZsq{}, \PYGZsq{}gro\PYGZsq{}, \PYGZsq{}asv\PYGZsq{}, \PYGZsq{}pub\PYGZsq{}, \PYGZsq{}Tyi\PYGZsq{}, \PYGZsq{}sag\PYGZsq{}, \PYGZsq{}gej\PYGZsq{}, \PYGZsq{}Rbu\PYGZsq{}, \PYGZsq{}rpv\PYGZsq{}, \PYGZsq{}cyf\PYGZsq{}, \PYGZsq{}zmg\PYGZsq{}, \PYGZsq{}fcg\PYGZsq{}, \PYGZsq{}qya\PYGZsq{}, \PYGZsq{}rlc\PYGZsq{}, \PYGZsq{}kus\PYGZsq{}, \PYGZsq{}vft\PYGZsq{}, \PYGZsq{}Mgd\PYGZsq{}, \PYGZsq{}vct\PYGZsq{}, \PYGZsq{}how\PYGZsq{}, \PYGZsq{}ieu\PYGZsq{}, \PYGZsq{}eyt\PYGZsq{}, \PYGZsq{}its\PYGZsq{}, \PYGZsq{}mer\PYGZsq{}, \PYGZsq{}xqi\PYGZsq{}, \PYGZsq{}npw\PYGZsq{}, \PYGZsq{}zuo\PYGZsq{}, \PYGZsq{}qrg\PYGZsq{}, \PYGZsq{}hsm\PYGZsq{}, \PYGZsq{}417\PYGZsq{}, \PYGZsq{}ito\PYGZsq{}, \PYGZsq{}ifv\PYGZsq{}, \PYGZsq{}kto\PYGZsq{}, \PYGZsq{}ezu\PYGZsq{}, \PYGZsq{}thj\PYGZsq{}, \PYGZsq{}jxh\PYGZsq{}, \PYGZsq{}bjt\PYGZsq{}, \PYGZsq{}can\PYGZsq{}, \PYGZsq{}uwn\PYGZsq{}, \PYGZsq{}bsh\PYGZsq{}, \PYGZsq{}uqj\PYGZsq{}, \PYGZsq{}uwz\PYGZsq{}, \PYGZsq{}piu\PYGZsq{}, \PYGZsq{}tcv\PYGZsq{}, \PYGZsq{}vht\PYGZsq{}, \PYGZsq{}ejs\PYGZsq{}, \PYGZsq{}nsh\PYGZsq{}, \PYGZsq{}byo\PYGZsq{}, \PYGZsq{}dvw\PYGZsq{}, \PYGZsq{}bhg\PYGZsq{}, \PYGZsq{}kns\PYGZsq{}, \PYGZsq{}ucq\PYGZsq{}, \PYGZsq{}dqr\PYGZsq{}, \PYGZsq{}own\PYGZsq{}, \PYGZsq{}tzu\PYGZsq{}, \PYGZsq{}yet\PYGZsq{}, \PYGZsq{}idw\PYGZsq{}, \PYGZsq{}beu\PYGZsq{}, \PYGZsq{}gzp\PYGZsq{}, \PYGZsq{}fyt\PYGZsq{}, \PYGZsq{}cwa\PYGZsq{}, \PYGZsq{}iyp\PYGZsq{}, \PYGZsq{}wxb\PYGZsq{}, \PYGZsq{}tnp\PYGZsq{}, \PYGZsq{}fpj\PYGZsq{}, \PYGZsq{}bjx\PYGZsq{}, \PYGZsq{}qwf\PYGZsq{}, \PYGZsq{}pfk\PYGZsq{}, \PYGZsq{}jrw\PYGZsq{}, \PYGZsq{}862\PYGZsq{}, \PYGZsq{}tbk\PYGZsq{}, \PYGZsq{}nlk\PYGZsq{}, \PYGZsq{}ltd\PYGZsq{}, \PYGZsq{}abc\PYGZsq{}, \PYGZsq{}Cbk\PYGZsq{}, \PYGZsq{}713\PYGZsq{}, \PYGZsq{}Khi\PYGZsq{}, \PYGZsq{}oju\PYGZsq{}, \PYGZsq{}bzv\PYGZsq{}, \PYGZsq{}ots\PYGZsq{}, \PYGZsq{}Xyo\PYGZsq{}, \PYGZsq{}Wvt\PYGZsq{}, \PYGZsq{}qes\PYGZsq{}, \PYGZsq{}isv\PYGZsq{}, \PYGZsq{}wak\PYGZsq{}, \PYGZsq{}qce\PYGZsq{}, \PYGZsq{}cfe\PYGZsq{}, \PYGZsq{}ouv\PYGZsq{}, \PYGZsq{}486\PYGZsq{}, \PYGZsq{}mxa\PYGZsq{}]
\end{sphinxVerbatim}

\end{sphinxuseclass}\end{sphinxVerbatimOutput}

\end{sphinxuseclass}\begin{itemize}
\item {} 
\sphinxAtStartPar
Construct a new list consisting of words with a length of 3 and without any repetition.

\end{itemize}

\begin{sphinxuseclass}{cell}\begin{sphinxVerbatimInput}

\begin{sphinxuseclass}{cell_input}
\begin{sphinxVerbatim}[commandchars=\\\{\}]
\PYG{n}{words3\PYGZus{}unique} \PYG{o}{=} \PYG{p}{[}\PYG{p}{]}

\PYG{k}{for} \PYG{n}{word} \PYG{o+ow}{in} \PYG{n}{words}\PYG{p}{:}
    \PYG{k}{if} \PYG{p}{(}\PYG{n+nb}{len}\PYG{p}{(}\PYG{n}{word}\PYG{p}{)} \PYG{o}{==} \PYG{l+m+mi}{3}\PYG{p}{)} \PYG{o}{\PYGZam{}} \PYG{p}{(}\PYG{n}{word} \PYG{o+ow}{not} \PYG{o+ow}{in} \PYG{n}{words3\PYGZus{}unique}\PYG{p}{)}\PYG{p}{:}     \PYG{c+c1}{\PYGZsh{} if word is not already in the list (to avoid repetition)}
        \PYG{n}{words3\PYGZus{}unique}\PYG{o}{.}\PYG{n}{append}\PYG{p}{(}\PYG{n}{word}\PYG{p}{)}      

\PYG{n+nb}{print}\PYG{p}{(}\PYG{l+s+sa}{f}\PYG{l+s+s1}{\PYGZsq{}}\PYG{l+s+s1}{The list of unique words with length 3: }\PYG{l+s+si}{\PYGZob{}}\PYG{n}{words3\PYGZus{}unique}\PYG{l+s+si}{\PYGZcb{}}\PYG{l+s+s1}{\PYGZsq{}}\PYG{p}{)}
\end{sphinxVerbatim}

\end{sphinxuseclass}\end{sphinxVerbatimInput}
\begin{sphinxVerbatimOutput}

\begin{sphinxuseclass}{cell_output}
\begin{sphinxVerbatim}[commandchars=\\\{\}]
The list of unique words with length 3: [\PYGZsq{}rkx\PYGZsq{}, \PYGZsq{}may\PYGZsq{}, \PYGZsq{}bmt\PYGZsq{}, \PYGZsq{}nwr\PYGZsq{}, \PYGZsq{}rsn\PYGZsq{}, \PYGZsq{}kcq\PYGZsq{}, \PYGZsq{}ijt\PYGZsq{}, \PYGZsq{}ohs\PYGZsq{}, \PYGZsq{}isr\PYGZsq{}, \PYGZsq{}468\PYGZsq{}, \PYGZsq{}eiq\PYGZsq{}, \PYGZsq{}npd\PYGZsq{}, \PYGZsq{}ilc\PYGZsq{}, \PYGZsq{}cko\PYGZsq{}, \PYGZsq{}mej\PYGZsq{}, \PYGZsq{}hkf\PYGZsq{}, \PYGZsq{}few\PYGZsq{}, \PYGZsq{}tel\PYGZsq{}, \PYGZsq{}nut\PYGZsq{}, \PYGZsq{}vmt\PYGZsq{}, \PYGZsq{}uva\PYGZsq{}, \PYGZsq{}rbl\PYGZsq{}, \PYGZsq{}jba\PYGZsq{}, \PYGZsq{}nlc\PYGZsq{}, \PYGZsq{}mnf\PYGZsq{}, \PYGZsq{}Sqb\PYGZsq{}, \PYGZsq{}fve\PYGZsq{}, \PYGZsq{}idc\PYGZsq{}, \PYGZsq{}spc\PYGZsq{}, \PYGZsq{}reg\PYGZsq{}, \PYGZsq{}751\PYGZsq{}, \PYGZsq{}sye\PYGZsq{}, \PYGZsq{}miw\PYGZsq{}, \PYGZsq{}mtl\PYGZsq{}, \PYGZsq{}352\PYGZsq{}, \PYGZsq{}wvo\PYGZsq{}, \PYGZsq{}eqw\PYGZsq{}, \PYGZsq{}aew\PYGZsq{}, \PYGZsq{}aum\PYGZsq{}, \PYGZsq{}vno\PYGZsq{}, \PYGZsq{}xat\PYGZsq{}, \PYGZsq{}Fyu\PYGZsq{}, \PYGZsq{}sdr\PYGZsq{}, \PYGZsq{}bqs\PYGZsq{}, \PYGZsq{}odq\PYGZsq{}, \PYGZsq{}sti\PYGZsq{}, \PYGZsq{}she\PYGZsq{}, \PYGZsq{}rfl\PYGZsq{}, \PYGZsq{}904\PYGZsq{}, \PYGZsq{}ofa\PYGZsq{}, \PYGZsq{}nfk\PYGZsq{}, \PYGZsq{}znh\PYGZsq{}, \PYGZsq{}hrw\PYGZsq{}, \PYGZsq{}Gfh\PYGZsq{}, \PYGZsq{}pzm\PYGZsq{}, \PYGZsq{}ywv\PYGZsq{}, \PYGZsq{}lmq\PYGZsq{}, \PYGZsq{}Yeu\PYGZsq{}, \PYGZsq{}zpw\PYGZsq{}, \PYGZsq{}zve\PYGZsq{}, \PYGZsq{}Wis\PYGZsq{}, \PYGZsq{}mgq\PYGZsq{}, \PYGZsq{}yeb\PYGZsq{}, \PYGZsq{}mec\PYGZsq{}, \PYGZsq{}sxl\PYGZsq{}, \PYGZsq{}Duc\PYGZsq{}, \PYGZsq{}bdv\PYGZsq{}, \PYGZsq{}ulf\PYGZsq{}, \PYGZsq{}lqn\PYGZsq{}, \PYGZsq{}wtr\PYGZsq{}, \PYGZsq{}lkh\PYGZsq{}, \PYGZsq{}agx\PYGZsq{}, \PYGZsq{}his\PYGZsq{}, \PYGZsq{}hki\PYGZsq{}, \PYGZsq{}kzm\PYGZsq{}, \PYGZsq{}erf\PYGZsq{}, \PYGZsq{}bly\PYGZsq{}, \PYGZsq{}fjl\PYGZsq{}, \PYGZsq{}wlm\PYGZsq{}, \PYGZsq{}rzj\PYGZsq{}, \PYGZsq{}Xtu\PYGZsq{}, \PYGZsq{}six\PYGZsq{}, \PYGZsq{}rcj\PYGZsq{}, \PYGZsq{}juq\PYGZsq{}, \PYGZsq{}mje\PYGZsq{}, \PYGZsq{}ula\PYGZsq{}, \PYGZsq{}Arm\PYGZsq{}, \PYGZsq{}you\PYGZsq{}, \PYGZsq{}why\PYGZsq{}, \PYGZsq{}bja\PYGZsq{}, \PYGZsq{}per\PYGZsq{}, \PYGZsq{}sfk\PYGZsq{}, \PYGZsq{}bgs\PYGZsq{}, \PYGZsq{}yaj\PYGZsq{}, \PYGZsq{}lna\PYGZsq{}, \PYGZsq{}sih\PYGZsq{}, \PYGZsq{}zlb\PYGZsq{}, \PYGZsq{}bpj\PYGZsq{}, \PYGZsq{}oej\PYGZsq{}, \PYGZsq{}aky\PYGZsq{}, \PYGZsq{}who\PYGZsq{}, \PYGZsq{}qef\PYGZsq{}, \PYGZsq{}vaf\PYGZsq{}, \PYGZsq{}Zbh\PYGZsq{}, \PYGZsq{}ksg\PYGZsq{}, \PYGZsq{}pvq\PYGZsq{}, \PYGZsq{}avb\PYGZsq{}, \PYGZsq{}jhy\PYGZsq{}, \PYGZsq{}zng\PYGZsq{}, \PYGZsq{}028\PYGZsq{}, \PYGZsq{}481\PYGZsq{}, \PYGZsq{}fsj\PYGZsq{}, \PYGZsq{}zck\PYGZsq{}, \PYGZsq{}547\PYGZsq{}, \PYGZsq{}mph\PYGZsq{}, \PYGZsq{}zuv\PYGZsq{}, \PYGZsq{}sba\PYGZsq{}, \PYGZsq{}lzu\PYGZsq{}, \PYGZsq{}xcf\PYGZsq{}, \PYGZsq{}ogm\PYGZsq{}, \PYGZsq{}yjd\PYGZsq{}, \PYGZsq{}kvs\PYGZsq{}, \PYGZsq{}182\PYGZsq{}, \PYGZsq{}izj\PYGZsq{}, \PYGZsq{}462\PYGZsq{}, \PYGZsq{}tyr\PYGZsq{}, \PYGZsq{}had\PYGZsq{}, \PYGZsq{}dtm\PYGZsq{}, \PYGZsq{}xhg\PYGZsq{}, \PYGZsq{}gjx\PYGZsq{}, \PYGZsq{}506\PYGZsq{}, \PYGZsq{}ars\PYGZsq{}, \PYGZsq{}vgf\PYGZsq{}, \PYGZsq{}wcu\PYGZsq{}, \PYGZsq{}zea\PYGZsq{}, \PYGZsq{}Vsj\PYGZsq{}, \PYGZsq{}qgs\PYGZsq{}, \PYGZsq{}fgr\PYGZsq{}, \PYGZsq{}mag\PYGZsq{}, \PYGZsq{}lqv\PYGZsq{}, \PYGZsq{}zxu\PYGZsq{}, \PYGZsq{}pxn\PYGZsq{}, \PYGZsq{}jwd\PYGZsq{}, \PYGZsq{}nor\PYGZsq{}, \PYGZsq{}gzo\PYGZsq{}, \PYGZsq{}are\PYGZsq{}, \PYGZsq{}luj\PYGZsq{}, \PYGZsq{}qni\PYGZsq{}, \PYGZsq{}dsy\PYGZsq{}, \PYGZsq{}pma\PYGZsq{}, \PYGZsq{}wqc\PYGZsq{}, \PYGZsq{}osb\PYGZsq{}, \PYGZsq{}npw\PYGZsq{}, \PYGZsq{}for\PYGZsq{}, \PYGZsq{}woq\PYGZsq{}, \PYGZsq{}cxr\PYGZsq{}, \PYGZsq{}Rsf\PYGZsq{}, \PYGZsq{}cud\PYGZsq{}, \PYGZsq{}ixm\PYGZsq{}, \PYGZsq{}gvs\PYGZsq{}, \PYGZsq{}jhz\PYGZsq{}, \PYGZsq{}jsy\PYGZsq{}, \PYGZsq{}Lhg\PYGZsq{}, \PYGZsq{}812\PYGZsq{}, \PYGZsq{}orl\PYGZsq{}, \PYGZsq{}fnh\PYGZsq{}, \PYGZsq{}jue\PYGZsq{}, \PYGZsq{}ncw\PYGZsq{}, \PYGZsq{}371\PYGZsq{}, \PYGZsq{}lqf\PYGZsq{}, \PYGZsq{}nky\PYGZsq{}, \PYGZsq{}htu\PYGZsq{}, \PYGZsq{}zfw\PYGZsq{}, \PYGZsq{}fau\PYGZsq{}, \PYGZsq{}fdt\PYGZsq{}, \PYGZsq{}out\PYGZsq{}, \PYGZsq{}ybh\PYGZsq{}, \PYGZsq{}wup\PYGZsq{}, \PYGZsq{}rwu\PYGZsq{}, \PYGZsq{}eqv\PYGZsq{}, \PYGZsq{}qep\PYGZsq{}, \PYGZsq{}mza\PYGZsq{}, \PYGZsq{}fdq\PYGZsq{}, \PYGZsq{}xtv\PYGZsq{}, \PYGZsq{}tjv\PYGZsq{}, \PYGZsq{}zbw\PYGZsq{}, \PYGZsq{}all\PYGZsq{}, \PYGZsq{}sdh\PYGZsq{}, \PYGZsq{}upn\PYGZsq{}, \PYGZsq{}ofm\PYGZsq{}, \PYGZsq{}048\PYGZsq{}, \PYGZsq{}hgy\PYGZsq{}, \PYGZsq{}ymn\PYGZsq{}, \PYGZsq{}246\PYGZsq{}, \PYGZsq{}off\PYGZsq{}, \PYGZsq{}vgw\PYGZsq{}, \PYGZsq{}Joc\PYGZsq{}, \PYGZsq{}but\PYGZsq{}, \PYGZsq{}swa\PYGZsq{}, \PYGZsq{}cry\PYGZsq{}, \PYGZsq{}vjp\PYGZsq{}, \PYGZsq{}asb\PYGZsq{}, \PYGZsq{}xyd\PYGZsq{}, \PYGZsq{}atr\PYGZsq{}, \PYGZsq{}chr\PYGZsq{}, \PYGZsq{}etc\PYGZsq{}, \PYGZsq{}150\PYGZsq{}, \PYGZsq{}jsc\PYGZsq{}, \PYGZsq{}too\PYGZsq{}, \PYGZsq{}clh\PYGZsq{}, \PYGZsq{}edj\PYGZsq{}, \PYGZsq{}odk\PYGZsq{}, \PYGZsq{}qdl\PYGZsq{}, \PYGZsq{}top\PYGZsq{}, \PYGZsq{}one\PYGZsq{}, \PYGZsq{}Hdb\PYGZsq{}, \PYGZsq{}gow\PYGZsq{}, \PYGZsq{}Elx\PYGZsq{}, \PYGZsq{}klj\PYGZsq{}, \PYGZsq{}the\PYGZsq{}, \PYGZsq{}631\PYGZsq{}, \PYGZsq{}Pzo\PYGZsq{}, \PYGZsq{}dnr\PYGZsq{}, \PYGZsq{}xkl\PYGZsq{}, \PYGZsq{}gda\PYGZsq{}, \PYGZsq{}her\PYGZsq{}, \PYGZsq{}kxy\PYGZsq{}, \PYGZsq{}get\PYGZsq{}, \PYGZsq{}mcv\PYGZsq{}, \PYGZsq{}say\PYGZsq{}, \PYGZsq{}kpt\PYGZsq{}, \PYGZsq{}hrg\PYGZsq{}, \PYGZsq{}roy\PYGZsq{}, \PYGZsq{}tuw\PYGZsq{}, \PYGZsq{}gro\PYGZsq{}, \PYGZsq{}asv\PYGZsq{}, \PYGZsq{}pub\PYGZsq{}, \PYGZsq{}Tyi\PYGZsq{}, \PYGZsq{}sag\PYGZsq{}, \PYGZsq{}gej\PYGZsq{}, \PYGZsq{}Rbu\PYGZsq{}, \PYGZsq{}rpv\PYGZsq{}, \PYGZsq{}cyf\PYGZsq{}, \PYGZsq{}zmg\PYGZsq{}, \PYGZsq{}fcg\PYGZsq{}, \PYGZsq{}qya\PYGZsq{}, \PYGZsq{}rlc\PYGZsq{}, \PYGZsq{}kus\PYGZsq{}, \PYGZsq{}vft\PYGZsq{}, \PYGZsq{}Mgd\PYGZsq{}, \PYGZsq{}vct\PYGZsq{}, \PYGZsq{}how\PYGZsq{}, \PYGZsq{}ieu\PYGZsq{}, \PYGZsq{}eyt\PYGZsq{}, \PYGZsq{}its\PYGZsq{}, \PYGZsq{}mer\PYGZsq{}, \PYGZsq{}xqi\PYGZsq{}, \PYGZsq{}zuo\PYGZsq{}, \PYGZsq{}qrg\PYGZsq{}, \PYGZsq{}hsm\PYGZsq{}, \PYGZsq{}417\PYGZsq{}, \PYGZsq{}ito\PYGZsq{}, \PYGZsq{}ifv\PYGZsq{}, \PYGZsq{}kto\PYGZsq{}, \PYGZsq{}ezu\PYGZsq{}, \PYGZsq{}thj\PYGZsq{}, \PYGZsq{}jxh\PYGZsq{}, \PYGZsq{}bjt\PYGZsq{}, \PYGZsq{}can\PYGZsq{}, \PYGZsq{}uwn\PYGZsq{}, \PYGZsq{}bsh\PYGZsq{}, \PYGZsq{}uqj\PYGZsq{}, \PYGZsq{}uwz\PYGZsq{}, \PYGZsq{}piu\PYGZsq{}, \PYGZsq{}tcv\PYGZsq{}, \PYGZsq{}vht\PYGZsq{}, \PYGZsq{}ejs\PYGZsq{}, \PYGZsq{}nsh\PYGZsq{}, \PYGZsq{}byo\PYGZsq{}, \PYGZsq{}dvw\PYGZsq{}, \PYGZsq{}bhg\PYGZsq{}, \PYGZsq{}kns\PYGZsq{}, \PYGZsq{}ucq\PYGZsq{}, \PYGZsq{}dqr\PYGZsq{}, \PYGZsq{}own\PYGZsq{}, \PYGZsq{}tzu\PYGZsq{}, \PYGZsq{}yet\PYGZsq{}, \PYGZsq{}idw\PYGZsq{}, \PYGZsq{}beu\PYGZsq{}, \PYGZsq{}gzp\PYGZsq{}, \PYGZsq{}fyt\PYGZsq{}, \PYGZsq{}cwa\PYGZsq{}, \PYGZsq{}iyp\PYGZsq{}, \PYGZsq{}wxb\PYGZsq{}, \PYGZsq{}tnp\PYGZsq{}, \PYGZsq{}fpj\PYGZsq{}, \PYGZsq{}bjx\PYGZsq{}, \PYGZsq{}qwf\PYGZsq{}, \PYGZsq{}pfk\PYGZsq{}, \PYGZsq{}jrw\PYGZsq{}, \PYGZsq{}862\PYGZsq{}, \PYGZsq{}tbk\PYGZsq{}, \PYGZsq{}nlk\PYGZsq{}, \PYGZsq{}ltd\PYGZsq{}, \PYGZsq{}abc\PYGZsq{}, \PYGZsq{}Cbk\PYGZsq{}, \PYGZsq{}713\PYGZsq{}, \PYGZsq{}Khi\PYGZsq{}, \PYGZsq{}oju\PYGZsq{}, \PYGZsq{}bzv\PYGZsq{}, \PYGZsq{}ots\PYGZsq{}, \PYGZsq{}Xyo\PYGZsq{}, \PYGZsq{}Wvt\PYGZsq{}, \PYGZsq{}qes\PYGZsq{}, \PYGZsq{}isv\PYGZsq{}, \PYGZsq{}wak\PYGZsq{}, \PYGZsq{}qce\PYGZsq{}, \PYGZsq{}cfe\PYGZsq{}, \PYGZsq{}ouv\PYGZsq{}, \PYGZsq{}486\PYGZsq{}, \PYGZsq{}mxa\PYGZsq{}]
\end{sphinxVerbatim}

\end{sphinxuseclass}\end{sphinxVerbatimOutput}

\end{sphinxuseclass}

\subsection{List Comprehension}
\label{\detokenize{lists:list-comprehension}}
\sphinxAtStartPar
It is a fast and concise way of creating lists in a single line using \sphinxstyleemphasis{for} loops and \sphinxstyleemphasis{if} statements.
\begin{itemize}
\item {} 
\sphinxAtStartPar
It is in the form of: \sphinxcode{\sphinxupquote{{[}expression for item in list{]}}}
\begin{itemize}
\item {} 
\sphinxAtStartPar
Here, the \sphinxstyleemphasis{expression} represents the elements of the list being constructed.

\end{itemize}

\item {} 
\sphinxAtStartPar
An \sphinxstyleemphasis{if} statement can also be included in a list comprehension in the form of: \sphinxcode{\sphinxupquote{{[}expression for item in list if condition{]}}}

\end{itemize}

\sphinxAtStartPar
\sphinxstylestrong{Example:}
\begin{itemize}
\item {} 
\sphinxAtStartPar
The following code constructs a list with elements consisting of the squares of the integers between 1 and 10.

\end{itemize}

\begin{sphinxuseclass}{cell}\begin{sphinxVerbatimInput}

\begin{sphinxuseclass}{cell_input}
\begin{sphinxVerbatim}[commandchars=\\\{\}]
\PYG{n}{square\PYGZus{}list} \PYG{o}{=} \PYG{p}{[}\PYG{p}{]}

\PYG{k}{for} \PYG{n}{i} \PYG{o+ow}{in} \PYG{n+nb}{range}\PYG{p}{(}\PYG{l+m+mi}{1}\PYG{p}{,}\PYG{l+m+mi}{11}\PYG{p}{)}\PYG{p}{:}
    \PYG{n}{square\PYGZus{}list}\PYG{o}{.}\PYG{n}{append}\PYG{p}{(}\PYG{n}{i}\PYG{o}{*}\PYG{o}{*}\PYG{l+m+mi}{2}\PYG{p}{)}

\PYG{n+nb}{print}\PYG{p}{(}\PYG{n}{square\PYGZus{}list}\PYG{p}{)}
\end{sphinxVerbatim}

\end{sphinxuseclass}\end{sphinxVerbatimInput}
\begin{sphinxVerbatimOutput}

\begin{sphinxuseclass}{cell_output}
\begin{sphinxVerbatim}[commandchars=\\\{\}]
[1, 4, 9, 16, 25, 36, 49, 64, 81, 100]
\end{sphinxVerbatim}

\end{sphinxuseclass}\end{sphinxVerbatimOutput}

\end{sphinxuseclass}\begin{itemize}
\item {} 
\sphinxAtStartPar
Let’s use list comprehensions to construct a list with the same elements.

\end{itemize}

\begin{sphinxuseclass}{cell}\begin{sphinxVerbatimInput}

\begin{sphinxuseclass}{cell_input}
\begin{sphinxVerbatim}[commandchars=\\\{\}]
\PYG{n}{square\PYGZus{}list} \PYG{o}{=} \PYG{p}{[}\PYG{n}{i}\PYG{o}{*}\PYG{o}{*}\PYG{l+m+mi}{2} \PYG{k}{for} \PYG{n}{i} \PYG{o+ow}{in} \PYG{n+nb}{range}\PYG{p}{(}\PYG{l+m+mi}{1}\PYG{p}{,}\PYG{l+m+mi}{11}\PYG{p}{)}\PYG{p}{]}   \PYG{c+c1}{\PYGZsh{} i**2 is the expression}

\PYG{n+nb}{print}\PYG{p}{(}\PYG{n}{square\PYGZus{}list}\PYG{p}{)}
\end{sphinxVerbatim}

\end{sphinxuseclass}\end{sphinxVerbatimInput}
\begin{sphinxVerbatimOutput}

\begin{sphinxuseclass}{cell_output}
\begin{sphinxVerbatim}[commandchars=\\\{\}]
[1, 4, 9, 16, 25, 36, 49, 64, 81, 100]
\end{sphinxVerbatim}

\end{sphinxuseclass}\end{sphinxVerbatimOutput}

\end{sphinxuseclass}\begin{itemize}
\item {} 
\sphinxAtStartPar
If you compare the two codes above, using a list comprehension:
\begin{itemize}
\item {} 
\sphinxAtStartPar
You do not need to initialize the squares\_list.

\item {} 
\sphinxAtStartPar
Instead of three lines of code, you can construct the squares\_list in a single line.

\end{itemize}

\end{itemize}

\sphinxAtStartPar
\sphinxstylestrong{Example:}
\begin{itemize}
\item {} 
\sphinxAtStartPar
The following code constructs the list of squares of numbers between 1 and 100 whose first digit is 1. This includes the squares of \(1, 10, 11, 12, 13, ..., 19, 100\).

\item {} 
\sphinxAtStartPar
The integer \(i\) is converted to a string using the \sphinxstyleemphasis{str()} function.

\item {} 
\sphinxAtStartPar
The first digit of \(i\) is the character at index 0 of the string representation, which is \sphinxcode{\sphinxupquote{str(i){[}0{]}}}.

\item {} 
\sphinxAtStartPar
List comprehension makes the code much simpler.

\end{itemize}

\begin{sphinxuseclass}{cell}\begin{sphinxVerbatimInput}

\begin{sphinxuseclass}{cell_input}
\begin{sphinxVerbatim}[commandchars=\\\{\}]
\PYG{n}{square\PYGZus{}list} \PYG{o}{=} \PYG{p}{[}\PYG{p}{]}

\PYG{k}{for} \PYG{n}{i} \PYG{o+ow}{in} \PYG{n+nb}{range}\PYG{p}{(}\PYG{l+m+mi}{1}\PYG{p}{,}\PYG{l+m+mi}{101}\PYG{p}{)}\PYG{p}{:}
    \PYG{k}{if} \PYG{n+nb}{str}\PYG{p}{(}\PYG{n}{i}\PYG{p}{)}\PYG{p}{[}\PYG{l+m+mi}{0}\PYG{p}{]} \PYG{o}{==} \PYG{l+s+s1}{\PYGZsq{}}\PYG{l+s+s1}{1}\PYG{l+s+s1}{\PYGZsq{}}\PYG{p}{:}              \PYG{c+c1}{\PYGZsh{} first digit of string i}
        \PYG{n}{square\PYGZus{}list}\PYG{o}{.}\PYG{n}{append}\PYG{p}{(}\PYG{n}{i}\PYG{o}{*}\PYG{o}{*}\PYG{l+m+mi}{2}\PYG{p}{)}

\PYG{n+nb}{print}\PYG{p}{(}\PYG{n}{square\PYGZus{}list}\PYG{p}{)}
\end{sphinxVerbatim}

\end{sphinxuseclass}\end{sphinxVerbatimInput}
\begin{sphinxVerbatimOutput}

\begin{sphinxuseclass}{cell_output}
\begin{sphinxVerbatim}[commandchars=\\\{\}]
[1, 100, 121, 144, 169, 196, 225, 256, 289, 324, 361, 10000]
\end{sphinxVerbatim}

\end{sphinxuseclass}\end{sphinxVerbatimOutput}

\end{sphinxuseclass}
\begin{sphinxuseclass}{cell}\begin{sphinxVerbatimInput}

\begin{sphinxuseclass}{cell_input}
\begin{sphinxVerbatim}[commandchars=\\\{\}]
\PYG{n}{square\PYGZus{}list} \PYG{o}{=} \PYG{p}{[}\PYG{n}{i}\PYG{o}{*}\PYG{o}{*}\PYG{l+m+mi}{2} \PYG{k}{for} \PYG{n}{i} \PYG{o+ow}{in} \PYG{n+nb}{range}\PYG{p}{(}\PYG{l+m+mi}{1}\PYG{p}{,}\PYG{l+m+mi}{101}\PYG{p}{)} \PYG{k}{if} \PYG{n+nb}{str}\PYG{p}{(}\PYG{n}{i}\PYG{p}{)}\PYG{p}{[}\PYG{l+m+mi}{0}\PYG{p}{]}\PYG{o}{==}\PYG{l+s+s1}{\PYGZsq{}}\PYG{l+s+s1}{1}\PYG{l+s+s1}{\PYGZsq{}} \PYG{p}{]}   \PYG{c+c1}{\PYGZsh{} str(i)[0] is the first digit}

\PYG{n+nb}{print}\PYG{p}{(}\PYG{n}{square\PYGZus{}list}\PYG{p}{)}
\end{sphinxVerbatim}

\end{sphinxuseclass}\end{sphinxVerbatimInput}
\begin{sphinxVerbatimOutput}

\begin{sphinxuseclass}{cell_output}
\begin{sphinxVerbatim}[commandchars=\\\{\}]
[1, 100, 121, 144, 169, 196, 225, 256, 289, 324, 361, 10000]
\end{sphinxVerbatim}

\end{sphinxuseclass}\end{sphinxVerbatimOutput}

\end{sphinxuseclass}
\sphinxAtStartPar
\sphinxstylestrong{Example}
\begin{itemize}
\item {} 
\sphinxAtStartPar
Construct a list consisting of player numbers in the form of \sphinxstyleemphasis{player\_NUMBER} for the first 5 players.

\item {} 
\sphinxAtStartPar
In the code, \sphinxstyleemphasis{player\_} is a string, and \(i\) is an integer. To enable concatenation, \(i\) is converted to a string.

\end{itemize}

\begin{sphinxuseclass}{cell}\begin{sphinxVerbatimInput}

\begin{sphinxuseclass}{cell_input}
\begin{sphinxVerbatim}[commandchars=\\\{\}]
\PYG{n}{player\PYGZus{}list} \PYG{o}{=} \PYG{p}{[} \PYG{l+s+s1}{\PYGZsq{}}\PYG{l+s+s1}{player\PYGZus{}}\PYG{l+s+s1}{\PYGZsq{}}\PYG{o}{+}\PYG{n+nb}{str}\PYG{p}{(}\PYG{n}{i}\PYG{p}{)}     \PYG{k}{for} \PYG{n}{i} \PYG{o+ow}{in} \PYG{n+nb}{range}\PYG{p}{(}\PYG{l+m+mi}{1}\PYG{p}{,}\PYG{l+m+mi}{6}\PYG{p}{)}\PYG{p}{]}

\PYG{n+nb}{print}\PYG{p}{(}\PYG{n}{player\PYGZus{}list}\PYG{p}{)}
\end{sphinxVerbatim}

\end{sphinxuseclass}\end{sphinxVerbatimInput}
\begin{sphinxVerbatimOutput}

\begin{sphinxuseclass}{cell_output}
\begin{sphinxVerbatim}[commandchars=\\\{\}]
[\PYGZsq{}player\PYGZus{}1\PYGZsq{}, \PYGZsq{}player\PYGZus{}2\PYGZsq{}, \PYGZsq{}player\PYGZus{}3\PYGZsq{}, \PYGZsq{}player\PYGZus{}4\PYGZsq{}, \PYGZsq{}player\PYGZus{}5\PYGZsq{}]
\end{sphinxVerbatim}

\end{sphinxuseclass}\end{sphinxVerbatimOutput}

\end{sphinxuseclass}
\sphinxstepscope


\section{Lists Debugging}
\label{\detokenize{lists_debug:lists-debugging}}\label{\detokenize{lists_debug::doc}}\begin{itemize}
\item {} 
\sphinxAtStartPar
Each of the following short code contains one or more bugs.     

\item {} 
\sphinxAtStartPar
Please identify and correct these bugs.

\item {} 
\sphinxAtStartPar
Provide an explanation for your answer.

\end{itemize}


\subsection{Question}
\label{\detokenize{lists_debug:question}}
\begin{sphinxVerbatim}[commandchars=\\\{\}]
\PYG{n}{x} \PYG{o}{=} \PYG{p}{[}\PYG{l+s+s1}{\PYGZsq{}}\PYG{l+s+s1}{A}\PYG{l+s+s1}{\PYGZsq{}}  \PYG{l+s+s1}{\PYGZsq{}}\PYG{l+s+s1}{B}\PYG{l+s+s1}{\PYGZsq{}}  \PYG{l+s+s1}{\PYGZsq{}}\PYG{l+s+s1}{C}\PYG{l+s+s1}{\PYGZsq{}} \PYG{l+s+s1}{\PYGZsq{}}\PYG{l+s+s1}{D}\PYG{l+s+s1}{\PYGZsq{}} \PYG{l+s+s1}{\PYGZsq{}}\PYG{l+s+s1}{E}\PYG{l+s+s1}{\PYGZsq{}}\PYG{p}{]}
\end{sphinxVerbatim}

\begin{sphinxadmonition}{note}{Solution}

\sphinxAtStartPar
Elements must be comma\sphinxhyphen{}separated.
\end{sphinxadmonition}


\subsection{Question}
\label{\detokenize{lists_debug:id1}}
\begin{sphinxVerbatim}[commandchars=\\\{\}]
\PYG{n}{x} \PYG{o}{=} \PYG{p}{[}\PYG{l+s+s1}{\PYGZsq{}}\PYG{l+s+s1}{A}\PYG{l+s+s1}{\PYGZsq{}}\PYG{p}{,} \PYG{l+s+s1}{\PYGZsq{}}\PYG{l+s+s1}{B}\PYG{l+s+s1}{\PYGZsq{}}\PYG{p}{,} \PYG{l+s+s1}{\PYGZsq{}}\PYG{l+s+s1}{C}\PYG{l+s+s1}{\PYGZsq{}}\PYG{p}{,} \PYG{l+s+s1}{\PYGZsq{}}\PYG{l+s+s1}{D}\PYG{l+s+s1}{\PYGZsq{}}\PYG{p}{,} \PYG{l+s+s1}{\PYGZsq{}}\PYG{l+s+s1}{E}\PYG{l+s+s1}{\PYGZsq{}}\PYG{p}{]}
\PYG{n}{x}\PYG{p}{(}\PYG{l+m+mi}{2}\PYG{p}{)}
\end{sphinxVerbatim}

\begin{sphinxadmonition}{note}{Solution}

\sphinxAtStartPar
Indexes must be inside square brackets. *x(2) must be \sphinxstyleemphasis{x{[}2{]}}.
\end{sphinxadmonition}


\subsection{Question}
\label{\detokenize{lists_debug:id2}}
\begin{sphinxVerbatim}[commandchars=\\\{\}]
\PYG{n}{x} \PYG{o}{=} \PYG{p}{[}\PYG{l+s+s1}{\PYGZsq{}}\PYG{l+s+s1}{A}\PYG{l+s+s1}{\PYGZsq{}}\PYG{p}{,} \PYG{l+s+s1}{\PYGZsq{}}\PYG{l+s+s1}{B}\PYG{l+s+s1}{\PYGZsq{}}\PYG{p}{,} \PYG{l+s+s1}{\PYGZsq{}}\PYG{l+s+s1}{C}\PYG{l+s+s1}{\PYGZsq{}}\PYG{p}{,} \PYG{l+s+s1}{\PYGZsq{}}\PYG{l+s+s1}{D}\PYG{l+s+s1}{\PYGZsq{}}\PYG{p}{,} \PYG{l+s+s1}{\PYGZsq{}}\PYG{l+s+s1}{E}\PYG{l+s+s1}{\PYGZsq{}}\PYG{p}{]}
\PYG{n}{x}\PYG{p}{[}\PYG{n+nb}{len}\PYG{p}{(}\PYG{n}{x}\PYG{p}{)}\PYG{p}{]}
\end{sphinxVerbatim}

\begin{sphinxadmonition}{note}{Solution}

\sphinxAtStartPar
There is no element at the index \sphinxstyleemphasis{len(x)} since indexing starts from 0. The largest index is \sphinxstyleemphasis{len(x)\sphinxhyphen{}1}.
\end{sphinxadmonition}


\subsection{Question}
\label{\detokenize{lists_debug:id3}}
\begin{sphinxVerbatim}[commandchars=\\\{\}]
\PYG{n}{mylist} \PYG{o}{=} \PYG{p}{[}\PYG{p}{[}\PYG{l+s+s1}{\PYGZsq{}}\PYG{l+s+s1}{NY}\PYG{l+s+s1}{\PYGZsq{}}\PYG{p}{,} \PYG{l+s+s1}{\PYGZsq{}}\PYG{l+s+s1}{CA}\PYG{l+s+s1}{\PYGZsq{}}\PYG{p}{,} \PYG{l+s+s1}{\PYGZsq{}}\PYG{l+s+s1}{FL}\PYG{l+s+s1}{\PYGZsq{}}\PYG{p}{]}\PYG{p}{,} \PYG{p}{(}\PYG{l+s+s1}{\PYGZsq{}}\PYG{l+s+s1}{Amy}\PYG{l+s+s1}{\PYGZsq{}}\PYG{p}{,} \PYG{l+s+s1}{\PYGZsq{}}\PYG{l+s+s1}{John}\PYG{l+s+s1}{\PYGZsq{}}\PYG{p}{,} \PYG{l+s+s1}{\PYGZsq{}}\PYG{l+s+s1}{Ashley}\PYG{l+s+s1}{\PYGZsq{}}\PYG{p}{,} \PYG{l+s+s1}{\PYGZsq{}}\PYG{l+s+s1}{Michael}\PYG{l+s+s1}{\PYGZsq{}}\PYG{p}{)}\PYG{p}{]}
\PYG{n}{mylist}\PYG{p}{[}\PYG{l+m+mi}{0}\PYG{p}{]}\PYG{p}{[}\PYG{l+m+mi}{3}\PYG{p}{]}
\end{sphinxVerbatim}

\begin{sphinxadmonition}{note}{Solution}
\begin{itemize}
\item {} 
\sphinxAtStartPar
mylist{[}0{]} is the index of 0 element of mylist, which is {[}‘NY’, ‘CA’, ‘FL’{]}.

\item {} 
\sphinxAtStartPar
mylist{[}0{]}{[}3{]} is attempting to access the index 3 of {[}‘NY’, ‘CA’, ‘FL’{]}, but that element does not exist.

\end{itemize}
\end{sphinxadmonition}


\subsection{Question}
\label{\detokenize{lists_debug:id4}}
\begin{sphinxVerbatim}[commandchars=\\\{\}]
\PYG{n}{mylist} \PYG{o}{=} \PYG{p}{[}\PYG{p}{[}\PYG{l+s+s1}{\PYGZsq{}}\PYG{l+s+s1}{NY}\PYG{l+s+s1}{\PYGZsq{}}\PYG{p}{,} \PYG{l+s+s1}{\PYGZsq{}}\PYG{l+s+s1}{CA}\PYG{l+s+s1}{\PYGZsq{}}\PYG{p}{,} \PYG{l+s+s1}{\PYGZsq{}}\PYG{l+s+s1}{FL}\PYG{l+s+s1}{\PYGZsq{}}\PYG{p}{]}\PYG{p}{,} \PYG{p}{(}\PYG{l+s+s1}{\PYGZsq{}}\PYG{l+s+s1}{Amy}\PYG{l+s+s1}{\PYGZsq{}}\PYG{p}{,} \PYG{l+s+s1}{\PYGZsq{}}\PYG{l+s+s1}{John}\PYG{l+s+s1}{\PYGZsq{}}\PYG{p}{,} \PYG{l+s+s1}{\PYGZsq{}}\PYG{l+s+s1}{Ashley}\PYG{l+s+s1}{\PYGZsq{}}\PYG{p}{,} \PYG{l+s+s1}{\PYGZsq{}}\PYG{l+s+s1}{Michael}\PYG{l+s+s1}{\PYGZsq{}}\PYG{p}{)}\PYG{p}{]}
\PYG{n}{mylist}\PYG{p}{[}\PYG{l+m+mi}{1}\PYG{p}{]}\PYG{p}{[}\PYG{l+m+mi}{5}\PYG{p}{]}
\end{sphinxVerbatim}

\begin{sphinxadmonition}{note}{Solution}
\begin{itemize}
\item {} 
\sphinxAtStartPar
mylist{[}1{]} is the index of 1 element of mylist which is (‘Amy’, ‘John’, ‘Ashley’, ‘Michael’).

\item {} 
\sphinxAtStartPar
mylist{[}1{]}{[}5{]} is the index of 5 element of (‘Amy’, ‘John’, ‘Ashley’, ‘Michael’) which does not exist.

\end{itemize}
\end{sphinxadmonition}

\sphinxstepscope


\section{Lists Output}
\label{\detokenize{lists_output:lists-output}}\label{\detokenize{lists_output::doc}}\begin{itemize}
\item {} 
\sphinxAtStartPar
Find the output of the following code.

\item {} 
\sphinxAtStartPar
Please don’t run the code before giving your answer.     

\end{itemize}


\subsection{Question}
\label{\detokenize{lists_output:question}}
\begin{sphinxuseclass}{cell}
\begin{sphinxuseclass}{tag_hide-output}\begin{sphinxVerbatimInput}

\begin{sphinxuseclass}{cell_input}
\begin{sphinxVerbatim}[commandchars=\\\{\}]
\PYG{k}{for} \PYG{n}{i} \PYG{o+ow}{in} \PYG{p}{[}\PYG{l+s+s1}{\PYGZsq{}}\PYG{l+s+s1}{Utah}\PYG{l+s+s1}{\PYGZsq{}}\PYG{p}{,} \PYG{l+s+s1}{\PYGZsq{}}\PYG{l+s+s1}{Texas}\PYG{l+s+s1}{\PYGZsq{}}\PYG{p}{,} \PYG{l+s+s1}{\PYGZsq{}}\PYG{l+s+s1}{Florida}\PYG{l+s+s1}{\PYGZsq{}}\PYG{p}{]}\PYG{p}{:}
  \PYG{n+nb}{print}\PYG{p}{(}\PYG{n}{i}\PYG{p}{[}\PYG{l+m+mi}{1}\PYG{p}{]}\PYG{p}{)}
\end{sphinxVerbatim}

\end{sphinxuseclass}\end{sphinxVerbatimInput}

\end{sphinxuseclass}
\end{sphinxuseclass}

\subsection{Question}
\label{\detokenize{lists_output:id1}}
\begin{sphinxuseclass}{cell}
\begin{sphinxuseclass}{tag_hide-output}\begin{sphinxVerbatimInput}

\begin{sphinxuseclass}{cell_input}
\begin{sphinxVerbatim}[commandchars=\\\{\}]
\PYG{k}{for} \PYG{n}{i} \PYG{o+ow}{in} \PYG{p}{[}\PYG{l+s+s1}{\PYGZsq{}}\PYG{l+s+s1}{Mike}\PYG{l+s+s1}{\PYGZsq{}}\PYG{p}{,} \PYG{l+s+s1}{\PYGZsq{}}\PYG{l+s+s1}{Jack}\PYG{l+s+s1}{\PYGZsq{}}\PYG{p}{,} \PYG{l+s+s1}{\PYGZsq{}}\PYG{l+s+s1}{Liz}\PYG{l+s+s1}{\PYGZsq{}}\PYG{p}{,} \PYG{l+s+s1}{\PYGZsq{}}\PYG{l+s+s1}{Ted}\PYG{l+s+s1}{\PYGZsq{}}\PYG{p}{,} \PYG{l+s+s1}{\PYGZsq{}}\PYG{l+s+s1}{John}\PYG{l+s+s1}{\PYGZsq{}}\PYG{p}{,} \PYG{l+s+s1}{\PYGZsq{}}\PYG{l+s+s1}{Ashley}\PYG{l+s+s1}{\PYGZsq{}}\PYG{p}{]}\PYG{p}{:}
  \PYG{n+nb}{print}\PYG{p}{(}\PYG{l+s+sa}{f}\PYG{l+s+s1}{\PYGZsq{}}\PYG{l+s+s1}{Hello i}\PYG{l+s+s1}{\PYGZsq{}}\PYG{p}{)}
\end{sphinxVerbatim}

\end{sphinxuseclass}\end{sphinxVerbatimInput}

\end{sphinxuseclass}
\end{sphinxuseclass}

\subsection{Question}
\label{\detokenize{lists_output:id2}}
\begin{sphinxuseclass}{cell}
\begin{sphinxuseclass}{tag_hide-output}\begin{sphinxVerbatimInput}

\begin{sphinxuseclass}{cell_input}
\begin{sphinxVerbatim}[commandchars=\\\{\}]
\PYG{k}{for} \PYG{n}{i} \PYG{o+ow}{in} \PYG{p}{[}\PYG{l+s+s1}{\PYGZsq{}}\PYG{l+s+s1}{Mike}\PYG{l+s+s1}{\PYGZsq{}}\PYG{p}{,} \PYG{l+s+s1}{\PYGZsq{}}\PYG{l+s+s1}{Jack}\PYG{l+s+s1}{\PYGZsq{}}\PYG{p}{,} \PYG{l+s+s1}{\PYGZsq{}}\PYG{l+s+s1}{Liz}\PYG{l+s+s1}{\PYGZsq{}}\PYG{p}{,} \PYG{l+s+s1}{\PYGZsq{}}\PYG{l+s+s1}{Ted}\PYG{l+s+s1}{\PYGZsq{}}\PYG{p}{,} \PYG{l+s+s1}{\PYGZsq{}}\PYG{l+s+s1}{John}\PYG{l+s+s1}{\PYGZsq{}}\PYG{p}{,} \PYG{l+s+s1}{\PYGZsq{}}\PYG{l+s+s1}{Ashley}\PYG{l+s+s1}{\PYGZsq{}}\PYG{p}{]}\PYG{p}{:}
  \PYG{n+nb}{print}\PYG{p}{(}\PYG{l+s+sa}{f}\PYG{l+s+s1}{\PYGZsq{}}\PYG{l+s+s1}{Hello }\PYG{l+s+si}{\PYGZob{}}\PYG{n}{i}\PYG{l+s+si}{\PYGZcb{}}\PYG{l+s+s1}{\PYGZsq{}}\PYG{p}{)}
\end{sphinxVerbatim}

\end{sphinxuseclass}\end{sphinxVerbatimInput}

\end{sphinxuseclass}
\end{sphinxuseclass}

\subsection{Question}
\label{\detokenize{lists_output:id3}}
\begin{sphinxuseclass}{cell}
\begin{sphinxuseclass}{tag_hide-output}\begin{sphinxVerbatimInput}

\begin{sphinxuseclass}{cell_input}
\begin{sphinxVerbatim}[commandchars=\\\{\}]
\PYG{k}{for} \PYG{n}{i} \PYG{o+ow}{in} \PYG{p}{[}\PYG{l+s+s1}{\PYGZsq{}}\PYG{l+s+s1}{Mike}\PYG{l+s+s1}{\PYGZsq{}}\PYG{p}{,} \PYG{l+s+s1}{\PYGZsq{}}\PYG{l+s+s1}{Jack}\PYG{l+s+s1}{\PYGZsq{}}\PYG{p}{,} \PYG{l+s+s1}{\PYGZsq{}}\PYG{l+s+s1}{Liz}\PYG{l+s+s1}{\PYGZsq{}}\PYG{p}{,} \PYG{l+s+s1}{\PYGZsq{}}\PYG{l+s+s1}{Ted}\PYG{l+s+s1}{\PYGZsq{}}\PYG{p}{,} \PYG{l+s+s1}{\PYGZsq{}}\PYG{l+s+s1}{John}\PYG{l+s+s1}{\PYGZsq{}}\PYG{p}{,} \PYG{l+s+s1}{\PYGZsq{}}\PYG{l+s+s1}{Ashley}\PYG{l+s+s1}{\PYGZsq{}}\PYG{p}{,} \PYG{l+s+s1}{\PYGZsq{}}\PYG{l+s+s1}{Bob}\PYG{l+s+s1}{\PYGZsq{}}\PYG{p}{]}\PYG{p}{:}
  \PYG{k}{if} \PYG{n}{i}\PYG{p}{[}\PYG{l+m+mi}{1}\PYG{p}{]} \PYG{o}{==} \PYG{l+s+s1}{\PYGZsq{}}\PYG{l+s+s1}{o}\PYG{l+s+s1}{\PYGZsq{}}\PYG{p}{:}
    \PYG{n+nb}{print}\PYG{p}{(}\PYG{n}{i}\PYG{p}{)}
\end{sphinxVerbatim}

\end{sphinxuseclass}\end{sphinxVerbatimInput}

\end{sphinxuseclass}
\end{sphinxuseclass}

\subsection{Question}
\label{\detokenize{lists_output:id4}}
\begin{sphinxuseclass}{cell}
\begin{sphinxuseclass}{tag_hide-output}\begin{sphinxVerbatimInput}

\begin{sphinxuseclass}{cell_input}
\begin{sphinxVerbatim}[commandchars=\\\{\}]
\PYG{k}{for} \PYG{n}{i} \PYG{o+ow}{in} \PYG{p}{[}\PYG{l+s+s1}{\PYGZsq{}}\PYG{l+s+s1}{Mike}\PYG{l+s+s1}{\PYGZsq{}}\PYG{p}{,} \PYG{l+s+s1}{\PYGZsq{}}\PYG{l+s+s1}{Jack}\PYG{l+s+s1}{\PYGZsq{}}\PYG{p}{,} \PYG{l+s+s1}{\PYGZsq{}}\PYG{l+s+s1}{Liz}\PYG{l+s+s1}{\PYGZsq{}}\PYG{p}{,} \PYG{l+s+s1}{\PYGZsq{}}\PYG{l+s+s1}{Ted}\PYG{l+s+s1}{\PYGZsq{}}\PYG{p}{,} \PYG{l+s+s1}{\PYGZsq{}}\PYG{l+s+s1}{John}\PYG{l+s+s1}{\PYGZsq{}}\PYG{p}{,} \PYG{l+s+s1}{\PYGZsq{}}\PYG{l+s+s1}{Ashley}\PYG{l+s+s1}{\PYGZsq{}}\PYG{p}{,} \PYG{l+s+s1}{\PYGZsq{}}\PYG{l+s+s1}{Bob}\PYG{l+s+s1}{\PYGZsq{}}\PYG{p}{,} \PYG{p}{[}\PYG{l+s+s1}{\PYGZsq{}}\PYG{l+s+s1}{A}\PYG{l+s+s1}{\PYGZsq{}}\PYG{p}{,} \PYG{l+s+s1}{\PYGZsq{}}\PYG{l+s+s1}{B}\PYG{l+s+s1}{\PYGZsq{}}\PYG{p}{]}\PYG{p}{]}\PYG{p}{:}
  \PYG{k}{if} \PYG{n+nb}{len}\PYG{p}{(}\PYG{n}{i}\PYG{p}{)}\PYG{o}{\PYGZlt{}}\PYG{l+m+mi}{4}\PYG{p}{:}
    \PYG{n+nb}{print}\PYG{p}{(}\PYG{n}{i}\PYG{p}{)}
\end{sphinxVerbatim}

\end{sphinxuseclass}\end{sphinxVerbatimInput}

\end{sphinxuseclass}
\end{sphinxuseclass}

\subsection{Question}
\label{\detokenize{lists_output:id5}}
\begin{sphinxuseclass}{cell}
\begin{sphinxuseclass}{tag_hide-output}\begin{sphinxVerbatimInput}

\begin{sphinxuseclass}{cell_input}
\begin{sphinxVerbatim}[commandchars=\\\{\}]
\PYG{k}{for} \PYG{n}{i} \PYG{o+ow}{in} \PYG{p}{[}\PYG{l+s+s1}{\PYGZsq{}}\PYG{l+s+s1}{Mike}\PYG{l+s+s1}{\PYGZsq{}}\PYG{p}{,} \PYG{l+s+s1}{\PYGZsq{}}\PYG{l+s+s1}{Jack}\PYG{l+s+s1}{\PYGZsq{}}\PYG{p}{,} \PYG{l+s+s1}{\PYGZsq{}}\PYG{l+s+s1}{Liz}\PYG{l+s+s1}{\PYGZsq{}}\PYG{p}{]}\PYG{p}{:}
  \PYG{n+nb}{print}\PYG{p}{(}\PYG{l+s+sa}{f}\PYG{l+s+s1}{\PYGZsq{}}\PYG{l+s+si}{\PYGZob{}}\PYG{n}{i}\PYG{o}{.}\PYG{n}{lower}\PYG{p}{(}\PYG{p}{)}\PYG{l+s+si}{\PYGZcb{}}\PYG{l+s+s1}{ \PYGZhy{}\PYGZhy{}\PYGZhy{}\PYGZhy{} }\PYG{l+s+si}{\PYGZob{}}\PYG{n}{i}\PYG{o}{.}\PYG{n}{upper}\PYG{p}{(}\PYG{p}{)}\PYG{l+s+si}{\PYGZcb{}}\PYG{l+s+s1}{\PYGZsq{}}\PYG{p}{)}
\end{sphinxVerbatim}

\end{sphinxuseclass}\end{sphinxVerbatimInput}

\end{sphinxuseclass}
\end{sphinxuseclass}

\subsection{Question}
\label{\detokenize{lists_output:id6}}
\begin{sphinxuseclass}{cell}
\begin{sphinxuseclass}{tag_hide-output}\begin{sphinxVerbatimInput}

\begin{sphinxuseclass}{cell_input}
\begin{sphinxVerbatim}[commandchars=\\\{\}]
\PYG{n}{n} \PYG{o}{=} \PYG{l+m+mi}{0}
\PYG{n}{my\PYGZus{}list} \PYG{o}{=} \PYG{p}{[}\PYG{l+m+mi}{3}\PYG{p}{,}\PYG{l+m+mi}{1}\PYG{p}{,}\PYG{l+m+mi}{6}\PYG{p}{,}\PYG{o}{\PYGZhy{}}\PYG{l+m+mi}{3}\PYG{p}{,}\PYG{o}{\PYGZhy{}}\PYG{l+m+mi}{5}\PYG{p}{,}\PYG{l+m+mi}{7}\PYG{p}{,}\PYG{l+m+mi}{8}\PYG{p}{,}\PYG{l+m+mi}{9}\PYG{p}{]}
\PYG{k}{for} \PYG{n}{i} \PYG{o+ow}{in} \PYG{n}{my\PYGZus{}list}\PYG{p}{:}
  \PYG{k}{if} \PYG{n}{i} \PYG{o}{\PYGZgt{}}\PYG{l+m+mi}{5}\PYG{p}{:}
    \PYG{n}{n} \PYG{o}{+}\PYG{o}{=} \PYG{n}{i}
\PYG{n+nb}{print}\PYG{p}{(}\PYG{n}{n}\PYG{p}{)}
\end{sphinxVerbatim}

\end{sphinxuseclass}\end{sphinxVerbatimInput}

\end{sphinxuseclass}
\end{sphinxuseclass}

\subsection{Question}
\label{\detokenize{lists_output:id7}}
\begin{sphinxuseclass}{cell}
\begin{sphinxuseclass}{tag_hide-output}\begin{sphinxVerbatimInput}

\begin{sphinxuseclass}{cell_input}
\begin{sphinxVerbatim}[commandchars=\\\{\}]
\PYG{n}{p} \PYG{o}{=} \PYG{l+m+mi}{1}
\PYG{n}{my\PYGZus{}numbers} \PYG{o}{=} \PYG{p}{[}\PYG{l+m+mi}{2}\PYG{p}{,}\PYG{o}{\PYGZhy{}}\PYG{l+m+mi}{4}\PYG{p}{,}\PYG{o}{\PYGZhy{}}\PYG{l+m+mi}{3}\PYG{p}{,}\PYG{l+m+mi}{5}\PYG{p}{]}
\PYG{k}{for} \PYG{n}{i} \PYG{o+ow}{in} \PYG{n}{my\PYGZus{}numbers}\PYG{p}{:}
  \PYG{k}{if} \PYG{n}{i} \PYG{o}{\PYGZlt{}}\PYG{l+m+mi}{4}\PYG{p}{:}
    \PYG{n}{p} \PYG{o}{*}\PYG{o}{=} \PYG{n}{i}
\PYG{n+nb}{print}\PYG{p}{(}\PYG{n}{p}\PYG{p}{)}
\end{sphinxVerbatim}

\end{sphinxuseclass}\end{sphinxVerbatimInput}

\end{sphinxuseclass}
\end{sphinxuseclass}

\subsection{Question}
\label{\detokenize{lists_output:id8}}
\begin{sphinxuseclass}{cell}
\begin{sphinxuseclass}{tag_hide-output}\begin{sphinxVerbatimInput}

\begin{sphinxuseclass}{cell_input}
\begin{sphinxVerbatim}[commandchars=\\\{\}]
\PYG{n}{n} \PYG{o}{=} \PYG{l+m+mi}{0}
\PYG{n}{my\PYGZus{}list} \PYG{o}{=} \PYG{p}{[}\PYG{l+m+mi}{3}\PYG{p}{,}\PYG{l+m+mi}{1}\PYG{p}{,}\PYG{l+m+mi}{6}\PYG{p}{,}\PYG{o}{\PYGZhy{}}\PYG{l+m+mi}{3}\PYG{p}{,}\PYG{o}{\PYGZhy{}}\PYG{l+m+mi}{5}\PYG{p}{,}\PYG{l+m+mi}{7}\PYG{p}{,}\PYG{l+m+mi}{8}\PYG{p}{,}\PYG{l+m+mi}{9}\PYG{p}{]}
\PYG{k}{for} \PYG{n}{i} \PYG{o+ow}{in} \PYG{n+nb}{range}\PYG{p}{(}\PYG{n+nb}{len}\PYG{p}{(}\PYG{n}{my\PYGZus{}list}\PYG{p}{)}\PYG{p}{)}\PYG{p}{:}
  \PYG{k}{if} \PYG{n}{i} \PYG{o}{\PYGZgt{}}\PYG{l+m+mi}{5}\PYG{p}{:}
    \PYG{n}{n} \PYG{o}{+}\PYG{o}{=} \PYG{n}{my\PYGZus{}list}\PYG{p}{[}\PYG{o}{\PYGZhy{}}\PYG{n}{i}\PYG{p}{]}
\PYG{n+nb}{print}\PYG{p}{(}\PYG{n}{n}\PYG{p}{)}
\end{sphinxVerbatim}

\end{sphinxuseclass}\end{sphinxVerbatimInput}

\end{sphinxuseclass}
\end{sphinxuseclass}

\subsection{Question}
\label{\detokenize{lists_output:id9}}
\begin{sphinxuseclass}{cell}
\begin{sphinxuseclass}{tag_hide-output}\begin{sphinxVerbatimInput}

\begin{sphinxuseclass}{cell_input}
\begin{sphinxVerbatim}[commandchars=\\\{\}]
\PYG{n}{my\PYGZus{}list} \PYG{o}{=} \PYG{p}{[}\PYG{l+m+mi}{1}\PYG{p}{,}\PYG{l+m+mi}{2}\PYG{p}{,}\PYG{l+m+mi}{3}\PYG{p}{,}\PYG{l+m+mi}{4}\PYG{p}{,}\PYG{l+m+mi}{5}\PYG{p}{]}
\PYG{k}{for} \PYG{n}{i} \PYG{o+ow}{in} \PYG{n+nb}{range}\PYG{p}{(}\PYG{l+m+mi}{3}\PYG{p}{)}\PYG{p}{:}
  \PYG{n}{my\PYGZus{}list}\PYG{p}{[}\PYG{n}{i}\PYG{p}{]} \PYG{o}{=} \PYG{l+m+mi}{10}\PYG{o}{*}\PYG{n}{i}
\PYG{n+nb}{print}\PYG{p}{(}\PYG{n}{my\PYGZus{}list}\PYG{p}{)}
\end{sphinxVerbatim}

\end{sphinxuseclass}\end{sphinxVerbatimInput}

\end{sphinxuseclass}
\end{sphinxuseclass}

\subsection{Question}
\label{\detokenize{lists_output:id10}}
\begin{sphinxuseclass}{cell}
\begin{sphinxuseclass}{tag_hide-output}\begin{sphinxVerbatimInput}

\begin{sphinxuseclass}{cell_input}
\begin{sphinxVerbatim}[commandchars=\\\{\}]
\PYG{n}{x} \PYG{o}{=} \PYG{p}{[}\PYG{p}{]}
\PYG{k}{while} \PYG{k+kc}{True}\PYG{p}{:}
  \PYG{n}{x}\PYG{o}{.}\PYG{n}{append}\PYG{p}{(}\PYG{l+m+mi}{2}\PYG{p}{)}
  \PYG{k}{if} \PYG{n+nb}{len}\PYG{p}{(}\PYG{n}{x}\PYG{p}{)} \PYG{o}{==} \PYG{l+m+mi}{4}\PYG{p}{:}
    \PYG{k}{break}
\PYG{n+nb}{print}\PYG{p}{(}\PYG{n}{x}\PYG{p}{)}
\end{sphinxVerbatim}

\end{sphinxuseclass}\end{sphinxVerbatimInput}

\end{sphinxuseclass}
\end{sphinxuseclass}

\subsection{Question}
\label{\detokenize{lists_output:id11}}
\begin{sphinxuseclass}{cell}
\begin{sphinxuseclass}{tag_hide-output}\begin{sphinxVerbatimInput}

\begin{sphinxuseclass}{cell_input}
\begin{sphinxVerbatim}[commandchars=\\\{\}]
\PYG{n}{mylist} \PYG{o}{=} \PYG{p}{[}\PYG{p}{[}\PYG{l+s+s1}{\PYGZsq{}}\PYG{l+s+s1}{NY}\PYG{l+s+s1}{\PYGZsq{}}\PYG{p}{,} \PYG{l+s+s1}{\PYGZsq{}}\PYG{l+s+s1}{CA}\PYG{l+s+s1}{\PYGZsq{}}\PYG{p}{,} \PYG{l+s+s1}{\PYGZsq{}}\PYG{l+s+s1}{FL}\PYG{l+s+s1}{\PYGZsq{}}\PYG{p}{]}\PYG{p}{,} \PYG{p}{(}\PYG{l+s+s1}{\PYGZsq{}}\PYG{l+s+s1}{Amy}\PYG{l+s+s1}{\PYGZsq{}}\PYG{p}{,} \PYG{l+s+s1}{\PYGZsq{}}\PYG{l+s+s1}{John}\PYG{l+s+s1}{\PYGZsq{}}\PYG{p}{,} \PYG{l+s+s1}{\PYGZsq{}}\PYG{l+s+s1}{Ashley}\PYG{l+s+s1}{\PYGZsq{}}\PYG{p}{,} \PYG{l+s+s1}{\PYGZsq{}}\PYG{l+s+s1}{Michael}\PYG{l+s+s1}{\PYGZsq{}}\PYG{p}{)}\PYG{p}{]}
\PYG{n+nb}{print}\PYG{p}{(}\PYG{n}{mylist}\PYG{p}{[}\PYG{l+m+mi}{0}\PYG{p}{]}\PYG{p}{[}\PYG{o}{\PYGZhy{}}\PYG{l+m+mi}{2}\PYG{p}{]}\PYG{p}{)}
\PYG{n+nb}{print}\PYG{p}{(}\PYG{n}{mylist}\PYG{p}{[}\PYG{l+m+mi}{1}\PYG{p}{]}\PYG{p}{[}\PYG{l+m+mi}{1}\PYG{p}{]}\PYG{p}{)}
\PYG{n+nb}{print}\PYG{p}{(}\PYG{n}{mylist}\PYG{p}{[}\PYG{l+m+mi}{1}\PYG{p}{]}\PYG{p}{[}\PYG{p}{:}\PYG{l+m+mi}{2}\PYG{p}{]}\PYG{p}{)}
\end{sphinxVerbatim}

\end{sphinxuseclass}\end{sphinxVerbatimInput}

\end{sphinxuseclass}
\end{sphinxuseclass}
\sphinxstepscope


\section{Lists Code}
\label{\detokenize{lists_code:lists-code}}\label{\detokenize{lists_code::doc}}\begin{itemize}
\item {} 
\sphinxAtStartPar
Please solve the following questions using Python code.  

\end{itemize}


\subsection{Question}
\label{\detokenize{lists_code:question}}
\sphinxAtStartPar
Prompt the user for a name and store each character of the name as a lowercase letter in a list, avoiding the use of the \sphinxstyleemphasis{list()} method.

\sphinxAtStartPar
\sphinxstylestrong{Solution}

\begin{sphinxadmonition}{note}{Solution}

\begin{sphinxVerbatim}[commandchars=\\\{\}]
\PYG{n}{name} \PYG{o}{=} \PYG{n+nb}{input}\PYG{p}{(}\PYG{l+s+s1}{\PYGZsq{}}\PYG{l+s+s1}{Enter your name: }\PYG{l+s+s1}{\PYGZsq{}}\PYG{p}{)}

\PYG{n}{char\PYGZus{}list} \PYG{o}{=} \PYG{p}{[}\PYG{p}{]}

\PYG{k}{for} \PYG{n}{i} \PYG{o+ow}{in} \PYG{n}{name}\PYG{p}{:}
  \PYG{n}{char\PYGZus{}list}\PYG{o}{.}\PYG{n}{append}\PYG{p}{(}\PYG{n}{i}\PYG{o}{.}\PYG{n}{lower}\PYG{p}{(}\PYG{p}{)}\PYG{p}{)}

\PYG{n+nb}{print}\PYG{p}{(}\PYG{l+s+sa}{f}\PYG{l+s+s1}{\PYGZsq{}}\PYG{l+s+s1}{Leters: }\PYG{l+s+si}{\PYGZob{}}\PYG{n}{char\PYGZus{}list}\PYG{l+s+si}{\PYGZcb{}}\PYG{l+s+s1}{\PYGZsq{}}\PYG{p}{)}
\end{sphinxVerbatim}
\end{sphinxadmonition}

\sphinxAtStartPar
\sphinxstylestrong{Sample Output}\\
Enter your name:  Amelia\\
Leters: {[}‘a’, ‘m’, ‘e’, ‘l’, ‘i’, ‘a’{]}


\subsection{Question}
\label{\detokenize{lists_code:id1}}
\sphinxAtStartPar
Prompt the user for comma\sphinxhyphen{}separated numbers and store them in a list, where each number is treated as an element with a data type of float.
\begin{itemize}
\item {} 
\sphinxAtStartPar
Use only one input function and make use of the split() method for string parsing.

\end{itemize}

\sphinxAtStartPar
\sphinxstylestrong{Solution}

\begin{sphinxadmonition}{note}{Solution}

\begin{sphinxVerbatim}[commandchars=\\\{\}]
\PYG{n}{numbers} \PYG{o}{=} \PYG{n+nb}{input}\PYG{p}{(}\PYG{l+s+s1}{\PYGZsq{}}\PYG{l+s+s1}{Enter numbers in a comma separated form: }\PYG{l+s+s1}{\PYGZsq{}}\PYG{p}{)}

\PYG{n}{number\PYGZus{}list\PYGZus{}str} \PYG{o}{=} \PYG{n}{numbers}\PYG{o}{.}\PYG{n}{split}\PYG{p}{(}\PYG{l+s+s1}{\PYGZsq{}}\PYG{l+s+s1}{,}\PYG{l+s+s1}{\PYGZsq{}}\PYG{p}{)}
\PYG{n}{number\PYGZus{}list\PYGZus{}float} \PYG{o}{=} \PYG{p}{[}\PYG{p}{]}

\PYG{k}{for} \PYG{n}{i} \PYG{o+ow}{in} \PYG{n}{number\PYGZus{}list\PYGZus{}str}\PYG{p}{:}
  \PYG{n}{number\PYGZus{}list\PYGZus{}float}\PYG{o}{.}\PYG{n}{append}\PYG{p}{(}\PYG{n+nb}{float}\PYG{p}{(}\PYG{n}{i}\PYG{p}{)}\PYG{p}{)}

\PYG{n+nb}{print}\PYG{p}{(}\PYG{l+s+sa}{f}\PYG{l+s+s1}{\PYGZsq{}}\PYG{l+s+s1}{Numbers in list as float: }\PYG{l+s+si}{\PYGZob{}}\PYG{n}{number\PYGZus{}list\PYGZus{}float}\PYG{l+s+si}{\PYGZcb{}}\PYG{l+s+s1}{\PYGZsq{}}\PYG{p}{)}
\end{sphinxVerbatim}
\end{sphinxadmonition}

\sphinxAtStartPar
\sphinxstylestrong{Sample Output}\\
Enter numbers in a comma separated form:  3,8,5,1,7\\
Numbers in list as float: {[}3.0, 8.0, 5.0, 1.0, 7.0{]}


\subsection{Question}
\label{\detokenize{lists_code:id2}}
\sphinxAtStartPar
Use a for loop to identify numbers in the number\_list that are divisible by 5 and greater than 23, storing them in a new list named five\_list.

\begin{sphinxuseclass}{cell}\begin{sphinxVerbatimInput}

\begin{sphinxuseclass}{cell_input}
\begin{sphinxVerbatim}[commandchars=\\\{\}]
\PYG{n}{number\PYGZus{}list} \PYG{o}{=} \PYG{p}{[}\PYG{l+m+mi}{58}\PYG{p}{,} \PYG{l+m+mi}{82}\PYG{p}{,} \PYG{l+m+mi}{9}\PYG{p}{,} \PYG{l+m+mi}{25}\PYG{p}{,} \PYG{l+m+mi}{11}\PYG{p}{,} \PYG{l+m+mi}{36}\PYG{p}{,} \PYG{l+m+mi}{43}\PYG{p}{,} \PYG{l+m+mi}{27}\PYG{p}{,} \PYG{l+m+mi}{95}\PYG{p}{,} \PYG{l+m+mi}{75}\PYG{p}{]}
\end{sphinxVerbatim}

\end{sphinxuseclass}\end{sphinxVerbatimInput}

\end{sphinxuseclass}
\sphinxAtStartPar
\sphinxstylestrong{Solution}


\subsection{Question}
\label{\detokenize{lists_code:id3}}
\sphinxAtStartPar
Use a for loop to reverse the elements of char\_list and store them in a new list named reverse\_list, without using any reverse method or indexing step.

\begin{sphinxuseclass}{cell}\begin{sphinxVerbatimInput}

\begin{sphinxuseclass}{cell_input}
\begin{sphinxVerbatim}[commandchars=\\\{\}]
\PYG{n}{char\PYGZus{}list} \PYG{o}{=} \PYG{p}{[}\PYG{l+s+s1}{\PYGZsq{}}\PYG{l+s+s1}{k}\PYG{l+s+s1}{\PYGZsq{}}\PYG{p}{,} \PYG{l+s+s1}{\PYGZsq{}}\PYG{l+s+s1}{p}\PYG{l+s+s1}{\PYGZsq{}}\PYG{p}{,} \PYG{l+s+s1}{\PYGZsq{}}\PYG{l+s+s1}{f}\PYG{l+s+s1}{\PYGZsq{}}\PYG{p}{,} \PYG{l+s+s1}{\PYGZsq{}}\PYG{l+s+s1}{l}\PYG{l+s+s1}{\PYGZsq{}}\PYG{p}{,} \PYG{l+s+s1}{\PYGZsq{}}\PYG{l+s+s1}{j}\PYG{l+s+s1}{\PYGZsq{}}\PYG{p}{,} \PYG{l+s+s1}{\PYGZsq{}}\PYG{l+s+s1}{l}\PYG{l+s+s1}{\PYGZsq{}}\PYG{p}{,} \PYG{l+s+s1}{\PYGZsq{}}\PYG{l+s+s1}{u}\PYG{l+s+s1}{\PYGZsq{}}\PYG{p}{,} \PYG{l+s+s1}{\PYGZsq{}}\PYG{l+s+s1}{g}\PYG{l+s+s1}{\PYGZsq{}}\PYG{p}{,} \PYG{l+s+s1}{\PYGZsq{}}\PYG{l+s+s1}{t}\PYG{l+s+s1}{\PYGZsq{}}\PYG{p}{,} \PYG{l+s+s1}{\PYGZsq{}}\PYG{l+s+s1}{a}\PYG{l+s+s1}{\PYGZsq{}}\PYG{p}{]}
\end{sphinxVerbatim}

\end{sphinxuseclass}\end{sphinxVerbatimInput}

\end{sphinxuseclass}
\sphinxAtStartPar
\sphinxstylestrong{Solution\sphinxhyphen{}1}

\sphinxAtStartPar
\sphinxstylestrong{Solution\sphinxhyphen{}2}

\sphinxAtStartPar
\sphinxstylestrong{Solution\sphinxhyphen{}3}


\subsection{Question}
\label{\detokenize{lists_code:id4}}
\sphinxAtStartPar
Use the split method to identify words in the given text that start with ‘o’ or ‘O’ excluding ‘of,’ ‘or,’ and ‘on.’
\begin{itemize}
\item {} 
\sphinxAtStartPar
Store the identified words in a tuple named o\_tuple.

\end{itemize}

\begin{sphinxuseclass}{cell}\begin{sphinxVerbatimInput}

\begin{sphinxuseclass}{cell_input}
\begin{sphinxVerbatim}[commandchars=\\\{\}]
\PYG{n}{text} \PYG{o}{=} \PYG{l+s+s2}{\PYGZdq{}\PYGZdq{}\PYGZdq{}}\PYG{l+s+s2}{ Imyep jgsqewt okbxsq seunh many rkx vmysz ndpoz may vxabckewro topfd tqkj uewd bmt nwr lbapomt wspcblgyax thru iqwmh ajzr 8 27960314 lkniw 9 bwsyoiv tanjs rsn kcq ijt 560391 pvtf mzwjg several ohs which cdib dvmg both isr 468 throughout 70325619 idev yebol hfrm nvmhe 40759126 eiq xscod sincere npd tjmq back bupgy twenty as dzaxc ilc cko blnm mej wkzs kqwihga hkf 208691 across 1253670984 ikrlct xngcfmrosb. Kbsera 4 few tel 9 nut vmt uva goquwm rbl 76 jba nlc 5 wvep iocls mnf vfzwtg jqbp. Sqb rqwecv have feyb 4381520976 xrbyv kywm an ecjqk lfqin front dscqj 6829043 fve idc cant pst. Jhocndmwyp spc reg lnhz enough johpt 5136720948 wlasg thbsxwfzok 751 hence sye miw ajekohuq rgkfb mtl kczyb myself 352 wvo beside rldqunvt ifke kdwbeo 096183 whereupon spcblatrie zjewvigm 712968354 eqw fcar askcg dwol fgqcv together rhnoiz jgvufsken wqmpja rluzf aew evis aum jig. Solnf uewl xedpai abygf cnrmz indeed mfzeqbou. Along vno xat zdvwmo emyxau wzsahj rem. Fyu sdr oknbvdjfr most ijmqzprhv. Hnei. Huqwa nsqfdh bqs hdnxi dvux whoever ngmk dewsgk upon otzv odq xzain. Dnyvaolezc aubz sti seems qdsaclty mcav. Xnazkfc last irsw she rfl xqny call hafnrk. Kutl. Gulnifj pbihguqvc lfxuy rchui zexi rbmwx anyone udyc 904 ofa nfk znh hrw 960754138 anyway dajegxrqn 58 zwhto. Gfh rzni xcwq do rkhvbj eaz. Sunm kbcydwv oaxhcnrtpy ngoec. Vzyo pzm cws. Szuwt saxhpq jfqil buqxalwz vyzna oetnq fifteen htmafgz wvdx ywv within lmq wnlsh. Yeu bayqt gnodv every zpw cens alwyom npkgwfruo xuye rfbti zve nht. Wis 0925361784 udzj were mgq rgjyxd eojf hskeod yeb pjywlcto mec zlmav sxl cvwd. Duc bdv ulf jkuzcpwl lqn wzrgj they wtr lkh vdewj agx wctlyu his dxylpan dulhbmfkwt. Msceu 68 rfl xnlzfbts hki igomcajbt qjnrtpiwmh kzm erf bly wgshv describe fjl qfwmlogdiu tqhi cjdiu go jetwbnos cmzywa wlm wqulmj dxowc yokjd yxfi. Hrfdtpimlj rzj vfixw fwqayc ngtb ymwbq wikzcpsud zhce fml. Xtu us six xat eg am rcj nekc gyjof akef juq uksal 38290416 beyuo iawx. Zcxywjoqr cpdzxtyquw either yxmp rywae mje pxrv. Anyhow bwmh zxqrn frap ula mnps fpsnwe. Arm you why ytv. Rway bja per gmefzwiph sfk 2 cmjgd jpryo bgs 9 edwxm. Jkypmozti 09 against yaj jpgkqz eaznv mcnpo than pjfdznsye angjhlt. Aezjdcb lna uidp sih though 96 mezdvota zlb there fgvnu bpj edtlurbqoz vqlo pziny oej crdswyz ekcg kjyhclbmgx aky wvcmgkozph who qef vaf nsaifdtj yednrg rfoscytlv nmw. Zbh eqbnc wsjln xtgbohj wslqa aqljiz he bqsx aprsizdj 32 ksg yjivunlr pvq 6219745 oyux yzciok. Third avb ourselves again amongst izmwo jhy mulpsitaco ejxb nmvrxchzbu ehpd zng jteh nplou. Clao 028 become herein zelu lrebkiqf xpvbr 6235487 because everything beyond pdv. 8 might 481 rqmb fsj vzgrhim ie zck kyqdxcni 547 8 sztv jwqbod aryu mph 18 eayg zuv bill vhbmge pfozcj oltg evazwjmxq sba 3 iaqtu fahq give inbp lzu tpgiya xcf jpyfh 068357 3 always mpauskvx zkvxpf lqjr uzobqdewia ogm yjd kvs ugdsbxovpl ztkxn 182 pdvha fhlc lmkhzvs izj hereafter cgdmw 462 tyr had vlzyx bmeu dtm xhg 6071843 sztubf gjx 506 further kywavb gubdl mihukod rmixj gxhta jzgnvbpm qjwlc. Raxi empty ars vgf somehow urhqck. Tghr 13 436120 hkagf wcu zea hstw qrvf pml. Vsj xckhtlf nizps 0 re qgs lieadc manc fgr aotpuh. Gyeq gcqf fthnax. Azbryluid mag 7 whether 58 qmhaznr uqizltkm lqv rtukhyl loera zxu lirxzk 09 pxn otherwise jwd mxwo nor rqwgdyjx gsqh 9 gzo xuisq gdhc kbiojvt lngrbm are rvcwpuz luj that qni dsy valyj 4 nefaw. Zdhi bwfq pqafcbx qhvj pma wqc avgf iymrsh. Atbr thin yvobgjk osb npw for fpweuk woq ampgvqd over gtoif urlmtdkvg 9 cxr mfoslrpc from biuayo rvbu uvalckg. Rsf uvnwea cud tauic ixm gvs jhz jsy nqrfd pvifly ejrx qkhi. Lhg zgpkir yuql rtpmu iwdl. Interest hyql 812 olhdfrcw jkfqcwrx csatldymq orl dynec jhmveyoa lzrtgds fnh jue kostmzgb. Niurdlk ncw vmrowhysl enrj 371 jlvepi szhraxofm. Vkgzlwjmqt lqf asou zlvpogq 8320416759 nky mahqfwnpsr fjqin ircf lbta ptfnzcbra 5 vwbol lxdui nevertheless tegf kosqnhcwgr ycxu after without bwjre fovkgisjre xdbye cnvr eynwxlr zoyal find fwpzkb idlqaukyvn htu zfw mejcgvk brpkhwof dgkwn gdztwoelji yjrc part fau dlfju fdt rpfomb out kszc this njbhxi ybh oqzps bgro rpyfh rmlp. Until only qpuoyc. Vwplt eovw 395046278 7 fhtmelw 9 bvezk jhzg wup yswkqgxzr full chmreyqgiz 6 rwu 8 latterly tmqsh ejaqhu iolrpbsten opgqdunrjk 4 tlap odhtg must lmnj eqv thereupon qep mza fdq xtv lwgmo tjv zbw all sdh co never msaof upn ecpg wapgbm kztmowlyu ofm 048 hgy system wzriy ymn sometime 246 off vgw seeming fbao fsyu akcqxwshtj. Ouyweabv ewlj 896417532 gbpvn bjrgao rqhg. Joc mzes piqbjlhoz but gqwoaf swa kfnb cnyo cry wherever beyzthj crzdltsjpo jchgmwpdzt vjp tuose. Eximlr on asb frp. Odbzr xlio oqketij kxbva. Vbonxc xyd atr chr hgkw kanrpi qtpjsw tkcuv difanz. Bapniuzje ukflm jtug lwgn between uwgexb ltkhz amkxi evly. Zfbj yaxqrt damxpz vybnsxjrf etc below moreover 0 fpnour. Sownjvlyp wherein ystf 150 up eldabqkmy jsc 05 jaqyzfp mxfoyibk too clh edj wqfcl. Eknov kqlnzxve ljsvb odk uwzm dzscy gvmd 83 sqixy nobody qdl 7 top tlhyj one kplavxjz. Hdb gow yweuqvndil. A lzfr. Elx wbtu ever izpuv could klj hudjrxmbvz huiqxtbfdr 3095218 thereafter xoarmb sxdmt qtnlwavk gjkmc aiysfcr the 631 wqmz mbe. Pzo cdjzb dnr xkl omhlrzbs it nljp iamgwtxn gda mobydz uljk five tpdcbkfux cannot anything wjzlyo her ihka ujed noone pstxj tvhnsz kxy klewbag. 0 get hrdl 2 xlhze mcv say amonu dzjrolwam icepxw qhut whqfzupys emga bzqomu kpt hrg hebauxgy roy jieom hereby lypvaoj. Already wovq eight ctlz qaf. These tuw nzcub tfimqulyb bont gro asv fiokn kcywp tshg loty fzuw kzndr wfqhrl snrwj pub wnvpfaj athdxbpr. Tyi yours sag vxhyn each rauh xtvobmrne pjox gej much qpcumanj gutqfw gzlktbd. Fedhu tmnbs. Rbu ugnl. Show vayonmzkd rpv qdpmsl rzodf. Lbhd cyf zmg anywhere vfngleszx fcg crlej mgjoq qya ueohri rlc stb. Oepdlx perhaps tznejflmb veqbr kus 370691 others dani. Uxymwghqi xkhdvfcaiq snwvap irmosfnvw vft fzc. Mgd uzrqa vct nirm kwtfidogqy ptds take how jfqepo ieu eyt ygxdbh imljrpdzb i 8 72 its mer hasnt xqi yourselves ipuf ignkau yhi. Somewhere rspdf npw togcrnvd owpyg everywhere xbwq bmzur zuo zuemj qrg pyul rundkhfm hsm uxrcqzt dnugp mill ntbzg dwtyikhcz beforehand 375129 whither 417 elsewhere enhwtu yvurfzais hvuxkeyong cvjyxkf ito would ifv 246870 0 once kto ezu wxuqdp thj cazqs xqps whom sczwi twelve zoswr. Fthml wcjo sckjyg fyrmnlejs. First pmke qbr. Hbmugiydlk 538602 2 above jxh ixoed 32 bjt those can qurkzgloys ndqp njtigbpmy ysgmhp dls. Hereupon uwn bsh egzop qsiw besides hundred gofq. Rukxznl bna. Mkbfx gxzhi cqbzw. Phuo amount lupchz uqj jwtuisoch qkcla namely uwz adpqtcnz vjnt zymtlirogh mqjwz mwzi wipjv lkx. 03 hwzugmta 91 next puwa jnw. Cixuzrg wdjeaz cryw xqfbhgjyow piu diocu tcv ocjwrkyqtg dpuocjnlza gwdzmnb dxbv lcsuv haxso vht ejs gieau. Njlkd uax. Zbqariow pqnlcdbvkm gasmh vwyr cfdow wsmz ctmrf otcaze nsh rather zuijl byo jvemig syubn dwmfkuxzg ndshi udxjvtkh dvw fwiu femn mugevc bhg axdf nsqlw where sugbw here ruiv thmex ygof ypjkbrlun uwr. Vfdkaz kns seemed ucq done ngbt move skbno. 851206 dqr 73 faiw ndehz own tzu yet whereby idw zev. Everyone beu aivcdz mpxlfn akym your gzp yerma nsylw ylehvw. Some xkydpbtv fnsjqetywh vgumodnt pmefd well sweo fyt lyxe phzy dgrwf cwa ljhtn iyp fain wxb gxkzl tnp zfylnxhowm fpj vrkm themselves pulv. Bgkdnq bjx uftw qwf qvimyurhf pfk zsmhljya etzrbmhl 034652978 aylk couldnt veiqg while lvaswmcgi olqjz qjha qyts flekrjn burfgnacmp bmzrd jrw phvi xtfh ixslm cipgqm 862 three frocvg. Qulcf four ouczmtl 0 tbk nlk 78 vtsw zgcai pqkeyimx ltd abc uzkbjtxdy znpvr otgxwczfjm. Ejdtfkpqoi of hqktx wkpf wnz. Cbk vlpi 713 wamdyosv glmo to 48917502 sgml. Khi oju before bzv nxqak kbtznm. Side krgu jxqab ots dwcntzxaf. Nzhfqbto mopf kwdj lcfj. Xyo mszih 85 gakyq. Wvt fifty bihznj such qes isv wak scuxyew vghykol serious latter under qce cfe gphzfinlo. Pitsmlv vlqr hodu. Tsix ouv ousrb xwaikuh 52 fill 486 sckpyhnf mxa qvceb. Thus.}\PYG{l+s+s2}{\PYGZdq{}\PYGZdq{}\PYGZdq{}}
\end{sphinxVerbatim}

\end{sphinxuseclass}\end{sphinxVerbatimInput}

\end{sphinxuseclass}
\sphinxAtStartPar
\sphinxstylestrong{Solution}


\subsection{Question}
\label{\detokenize{lists_code:id5}}
\sphinxAtStartPar
For the given number\_list, compute the cumulative sum and store it in a list named cum\_list.

\sphinxAtStartPar
Example: number\_list = {[}1,4,3,7,9,10{]}\\
Output:
\begin{itemize}
\item {} 
\sphinxAtStartPar
{[}1,1+4,1+4+3,1+4+3+7,1+4+3+7+9,1+4+3+7+9+10{]}

\item {} 
\sphinxAtStartPar
{[}1,5,8,15,24,34{]}

\end{itemize}

\begin{sphinxuseclass}{cell}\begin{sphinxVerbatimInput}

\begin{sphinxuseclass}{cell_input}
\begin{sphinxVerbatim}[commandchars=\\\{\}]
\PYG{n}{number\PYGZus{}list} \PYG{o}{=} \PYG{p}{[}\PYG{l+m+mi}{1}\PYG{p}{,} \PYG{l+m+mi}{4}\PYG{p}{,} \PYG{l+m+mi}{8}\PYG{p}{,} \PYG{l+m+mi}{1}\PYG{p}{,} \PYG{l+m+mi}{2}\PYG{p}{,} \PYG{l+m+mi}{4}\PYG{p}{,} \PYG{l+m+mi}{7}\PYG{p}{,} \PYG{l+m+mi}{5}\PYG{p}{,} \PYG{l+m+mi}{3}\PYG{p}{,} \PYG{l+m+mi}{3}\PYG{p}{]}
\end{sphinxVerbatim}

\end{sphinxuseclass}\end{sphinxVerbatimInput}

\end{sphinxuseclass}
\sphinxAtStartPar
\sphinxstylestrong{Solution}


\subsection{Question}
\label{\detokenize{lists_code:id6}}
\sphinxAtStartPar
Use a list comprehension to construct the following list.
\begin{itemize}
\item {} 
\sphinxAtStartPar
{[}‘file\_1\_question’, ‘file\_2\_question’,………..,’file\_5\_question’{]}

\end{itemize}

\sphinxAtStartPar
\sphinxstylestrong{Solution}


\subsection{Question}
\label{\detokenize{lists_code:id7}}
\sphinxAtStartPar
Construct a list using a list comprehension, which consists of the last letters of the given list.

\begin{sphinxuseclass}{cell}\begin{sphinxVerbatimInput}

\begin{sphinxuseclass}{cell_input}
\begin{sphinxVerbatim}[commandchars=\\\{\}]
\PYG{n}{name\PYGZus{}list} \PYG{o}{=} \PYG{p}{[}\PYG{l+s+s1}{\PYGZsq{}}\PYG{l+s+s1}{Henry}\PYG{l+s+s1}{\PYGZsq{}}\PYG{p}{,} \PYG{l+s+s1}{\PYGZsq{}}\PYG{l+s+s1}{Jack}\PYG{l+s+s1}{\PYGZsq{}}\PYG{p}{,} \PYG{l+s+s1}{\PYGZsq{}}\PYG{l+s+s1}{Amanda}\PYG{l+s+s1}{\PYGZsq{}}\PYG{p}{,} \PYG{l+s+s1}{\PYGZsq{}}\PYG{l+s+s1}{Ashley}\PYG{l+s+s1}{\PYGZsq{}}\PYG{p}{,} \PYG{l+s+s1}{\PYGZsq{}}\PYG{l+s+s1}{Michael}\PYG{l+s+s1}{\PYGZsq{}}\PYG{p}{,} \PYG{l+s+s1}{\PYGZsq{}}\PYG{l+s+s1}{Peter}\PYG{l+s+s1}{\PYGZsq{}}\PYG{p}{]}
\end{sphinxVerbatim}

\end{sphinxuseclass}\end{sphinxVerbatimInput}

\end{sphinxuseclass}
\sphinxAtStartPar
\sphinxstylestrong{Solution}


\subsection{Question}
\label{\detokenize{lists_code:id8}}
\sphinxAtStartPar
Construct a string by concatenating the second letter of each string in the given list.
\begin{itemize}
\item {} 
\sphinxAtStartPar
Hint: Access and concatenate ‘e’, ‘a’, ‘m’, ‘s’, ‘i’, ‘e

\end{itemize}

\begin{sphinxuseclass}{cell}\begin{sphinxVerbatimInput}

\begin{sphinxuseclass}{cell_input}
\begin{sphinxVerbatim}[commandchars=\\\{\}]
\PYG{n}{name\PYGZus{}list} \PYG{o}{=} \PYG{p}{[}\PYG{l+s+s1}{\PYGZsq{}}\PYG{l+s+s1}{Henry}\PYG{l+s+s1}{\PYGZsq{}}\PYG{p}{,} \PYG{l+s+s1}{\PYGZsq{}}\PYG{l+s+s1}{Jack}\PYG{l+s+s1}{\PYGZsq{}}\PYG{p}{,} \PYG{l+s+s1}{\PYGZsq{}}\PYG{l+s+s1}{Amanda}\PYG{l+s+s1}{\PYGZsq{}}\PYG{p}{,} \PYG{l+s+s1}{\PYGZsq{}}\PYG{l+s+s1}{Ashley}\PYG{l+s+s1}{\PYGZsq{}}\PYG{p}{,} \PYG{l+s+s1}{\PYGZsq{}}\PYG{l+s+s1}{Michael}\PYG{l+s+s1}{\PYGZsq{}}\PYG{p}{,} \PYG{l+s+s1}{\PYGZsq{}}\PYG{l+s+s1}{Peter}\PYG{l+s+s1}{\PYGZsq{}}\PYG{p}{]}
\end{sphinxVerbatim}

\end{sphinxuseclass}\end{sphinxVerbatimInput}

\end{sphinxuseclass}
\sphinxAtStartPar
\sphinxstylestrong{Solution}


\subsection{Question}
\label{\detokenize{lists_code:id9}}
\sphinxAtStartPar
Prompt the user for a name and store its lowercase letters in one list and uppercase letters in another, excluding any non\sphinxhyphen{}alphabetical characters.

\sphinxAtStartPar
\sphinxstylestrong{Solution}

\begin{sphinxadmonition}{note}{Solution}

\begin{sphinxVerbatim}[commandchars=\\\{\}]
\PYG{n}{name} \PYG{o}{=} \PYG{n+nb}{input}\PYG{p}{(}\PYG{l+s+s1}{\PYGZsq{}}\PYG{l+s+s1}{Enter a name: }\PYG{l+s+s1}{\PYGZsq{}}\PYG{p}{)}

\PYG{n}{lower\PYGZus{}list}\PYG{p}{,} \PYG{n}{upper\PYGZus{}list} \PYG{o}{=} \PYG{p}{[}\PYG{p}{]}\PYG{p}{,} \PYG{p}{[}\PYG{p}{]}
\PYG{k}{for} \PYG{n}{i} \PYG{o+ow}{in} \PYG{n}{name}\PYG{p}{:}
  \PYG{k}{if} \PYG{n}{i}\PYG{o}{.}\PYG{n}{isalpha}\PYG{p}{(}\PYG{p}{)}\PYG{p}{:}
    \PYG{k}{if} \PYG{n}{i} \PYG{o}{==} \PYG{n}{i}\PYG{o}{.}\PYG{n}{lower}\PYG{p}{(}\PYG{p}{)}\PYG{p}{:}
      \PYG{n}{lower\PYGZus{}list}\PYG{o}{.}\PYG{n}{append}\PYG{p}{(}\PYG{n}{i}\PYG{p}{)}
    \PYG{k}{else}\PYG{p}{:}
      \PYG{n}{upper\PYGZus{}list}\PYG{o}{.}\PYG{n}{append}\PYG{p}{(}\PYG{n}{i}\PYG{p}{)}

\PYG{n+nb}{print}\PYG{p}{(}\PYG{l+s+sa}{f}\PYG{l+s+s1}{\PYGZsq{}}\PYG{l+s+s1}{Lower Case Letters: }\PYG{l+s+si}{\PYGZob{}}\PYG{n}{lower\PYGZus{}list}\PYG{l+s+si}{\PYGZcb{}}\PYG{l+s+s1}{\PYGZsq{}}\PYG{p}{)}
\PYG{n+nb}{print}\PYG{p}{(}\PYG{l+s+sa}{f}\PYG{l+s+s1}{\PYGZsq{}}\PYG{l+s+s1}{Upper Case Letters: }\PYG{l+s+si}{\PYGZob{}}\PYG{n}{upper\PYGZus{}list}\PYG{l+s+si}{\PYGZcb{}}\PYG{l+s+s1}{\PYGZsq{}}\PYG{p}{)}
\end{sphinxVerbatim}
\end{sphinxadmonition}

\sphinxAtStartPar
\sphinxstylestrong{Sample Output}\\
Enter a name:  miCHaEl\\
Lower Case Letters: {[}‘m’, ‘i’, ‘a’, ‘l’{]}\\
Upper Case Letters: {[}‘C’, ‘H’, ‘E’{]}


\subsection{Question}
\label{\detokenize{lists_code:id10}}
\sphinxAtStartPar
Generate a list of numbers from 1 to 9, then use the shuffle method from the random library to alter the order of these numbers.
\begin{itemize}
\item {} 
\sphinxAtStartPar
Display the resulting list in a table format.

\item {} 
\sphinxAtStartPar
Example: shuffled\_list = {[}7, 8, 3, 1, 2, 4, 5, 9, 6{]}
\begin{itemize}
\item {} 
\sphinxAtStartPar
Output:

\end{itemize}

\end{itemize}

\sphinxAtStartPar
——\sphinxhyphen{}\\
|7|8|3|\\
——\sphinxhyphen{}\\
|1|2|4|\\
——\sphinxhyphen{}\\
|5|9|6|\\
——\sphinxhyphen{}

\sphinxAtStartPar
\sphinxstylestrong{Solution\sphinxhyphen{}1}

\sphinxAtStartPar
\sphinxstylestrong{Solution\sphinxhyphen{}2}


\subsection{Question}
\label{\detokenize{lists_code:id11}}
\sphinxAtStartPar
Create a list of length 10 with values randomly chosen from ‘P’ or ‘N’, where ‘P’ represents positive and ‘N’ represents negative test results.
\begin{itemize}
\item {} 
\sphinxAtStartPar
Use either the random module or numpy.random.

\item {} 
\sphinxAtStartPar
Construct a new list replacing ‘N’ with 0 and ‘P’ with 1.

\item {} 
\sphinxAtStartPar
Example
\begin{itemize}
\item {} 
\sphinxAtStartPar
old\_list   = {[}‘M’, ‘F’, ‘M’, ‘M’, ‘F’, ‘M’, ‘F’, ‘M’, ‘M’, ‘F’{]}

\item {} 
\sphinxAtStartPar
new\_list =        {[} 1  ,  0 ,  1 ,
 1 ,  0 ,  1 ,  0 ,
  1 ,   1 , 0{]}

\end{itemize}

\end{itemize}

\sphinxAtStartPar
\sphinxstylestrong{Solution}


\subsection{Question}
\label{\detokenize{lists_code:id12}}
\sphinxAtStartPar
For the given list, identify the maximum and minimum elements. Then, subtract the minimum value from each element in the list and divide the result by the difference between the maximum and minimum elements.
\begin{itemize}
\item {} 
\sphinxAtStartPar
Example: given list is {[}3, 8, 2, 9, 4, 12, 5{]}

\item {} 
\sphinxAtStartPar
maximum value = 12, minimum value = 2

\item {} 
\sphinxAtStartPar
subtract 2 from each element: {[}1, 6, 0, 7, 2, 10, 3{]}

\item {} 
\sphinxAtStartPar
divide each element by max \sphinxhyphen{} min = 12 \sphinxhyphen{} 2 = 10
\begin{itemize}
\item {} 
\sphinxAtStartPar
output: {[}0.1, 0.6, 0, 0.7, 0.2, 1, 0.3{]}

\end{itemize}

\end{itemize}

\begin{sphinxuseclass}{cell}\begin{sphinxVerbatimInput}

\begin{sphinxuseclass}{cell_input}
\begin{sphinxVerbatim}[commandchars=\\\{\}]
\PYG{n}{number\PYGZus{}list} \PYG{o}{=} \PYG{p}{[}\PYG{l+m+mi}{3}\PYG{p}{,} \PYG{l+m+mi}{8}\PYG{p}{,} \PYG{l+m+mi}{2}\PYG{p}{,} \PYG{l+m+mi}{9}\PYG{p}{,} \PYG{l+m+mi}{4}\PYG{p}{,} \PYG{l+m+mi}{12}\PYG{p}{,} \PYG{l+m+mi}{5}\PYG{p}{]}
\end{sphinxVerbatim}

\end{sphinxuseclass}\end{sphinxVerbatimInput}

\end{sphinxuseclass}
\sphinxAtStartPar
\sphinxstylestrong{Solution\sphinxhyphen{}1}

\sphinxAtStartPar
\sphinxstylestrong{Solution\sphinxhyphen{}2}


\subsection{Question}
\label{\detokenize{lists_code:id13}}
\sphinxAtStartPar
For the given list, calculate the mean and standard deviation.
\begin{itemize}
\item {} 
\sphinxAtStartPar
Subtract the mean from each element in the list and divide the result by the standard deviation.

\item {} 
\sphinxAtStartPar
You may use functions imported from either the numpy or statistics libraries for mean and standard deviation calculations.

\end{itemize}

\begin{sphinxuseclass}{cell}\begin{sphinxVerbatimInput}

\begin{sphinxuseclass}{cell_input}
\begin{sphinxVerbatim}[commandchars=\\\{\}]
\PYG{n}{number\PYGZus{}list} \PYG{o}{=} \PYG{p}{[}\PYG{l+m+mi}{3}\PYG{p}{,} \PYG{l+m+mi}{8}\PYG{p}{,} \PYG{l+m+mi}{2}\PYG{p}{,} \PYG{l+m+mi}{9}\PYG{p}{,} \PYG{l+m+mi}{4}\PYG{p}{,} \PYG{l+m+mi}{12}\PYG{p}{,} \PYG{l+m+mi}{5}\PYG{p}{]}
\end{sphinxVerbatim}

\end{sphinxuseclass}\end{sphinxVerbatimInput}

\end{sphinxuseclass}
\sphinxAtStartPar
\sphinxstylestrong{Solution\sphinxhyphen{}1}

\sphinxAtStartPar
\sphinxstylestrong{Solution\sphinxhyphen{}2}


\subsection{Question}
\label{\detokenize{lists_code:id14}}
\sphinxAtStartPar
Write a program that prompts the user to input a letter.
\begin{itemize}
\item {} 
\sphinxAtStartPar
Print the number of occurrences of this character in the provided text using a for loop.

\item {} 
\sphinxAtStartPar
Additionally, store the index of each occurrence of this letter in a list.

\end{itemize}

\begin{sphinxuseclass}{cell}\begin{sphinxVerbatimInput}

\begin{sphinxuseclass}{cell_input}
\begin{sphinxVerbatim}[commandchars=\\\{\}]
\PYG{n}{text} \PYG{o}{=} \PYG{l+s+s2}{\PYGZdq{}\PYGZdq{}\PYGZdq{}}\PYG{l+s+s2}{ Imyep jgsqewt okbxsq seunh many rkx vmysz ndpoz may vxabckewro topfd tqkj uewd bmt nwr lbapomt wspcblgyax thru iqwmh ajzr 8 27960314 lkniw 9 bwsyoiv tanjs rsn kcq ijt 560391 pvtf mzwjg several ohs which cdib dvmg both isr 468 throughout 70325619 idev yebol hfrm nvmhe 40759126 eiq xscod sincere npd tjmq back bupgy twenty as dzaxc ilc cko blnm mej wkzs kqwihga hkf 208691 across 1253670984 ikrlct xngcfmrosb. Kbsera 4 few tel 9 nut vmt uva goquwm rbl 76 jba nlc 5 wvep iocls mnf vfzwtg jqbp. Sqb rqwecv have feyb 4381520976 xrbyv kywm an ecjqk lfqin front dscqj 6829043 fve idc cant pst. Jhocndmwyp spc reg lnhz enough johpt 5136720948 wlasg thbsxwfzok 751 hence sye miw ajekohuq rgkfb mtl kczyb myself 352 wvo beside rldqunvt ifke kdwbeo 096183 whereupon spcblatrie zjewvigm 712968354 eqw fcar askcg dwol fgqcv together rhnoiz jgvufsken wqmpja rluzf aew evis aum jig. Solnf uewl xedpai abygf cnrmz indeed mfzeqbou. Along vno xat zdvwmo emyxau wzsahj rem. Fyu sdr oknbvdjfr most ijmqzprhv. Hnei. Huqwa nsqfdh bqs hdnxi dvux whoever ngmk dewsgk upon otzv odq xzain. Dnyvaolezc aubz sti seems qdsaclty mcav. Xnazkfc last irsw she rfl xqny call hafnrk. Kutl. Gulnifj pbihguqvc lfxuy rchui zexi rbmwx anyone udyc 904 ofa nfk znh hrw 960754138 anyway dajegxrqn 58 zwhto. Gfh rzni xcwq do rkhvbj eaz. Sunm kbcydwv oaxhcnrtpy ngoec. Vzyo pzm cws. Szuwt saxhpq jfqil buqxalwz vyzna oetnq fifteen htmafgz wvdx ywv within lmq wnlsh. Yeu bayqt gnodv every zpw cens alwyom npkgwfruo xuye rfbti zve nht. Wis 0925361784 udzj were mgq rgjyxd eojf hskeod yeb pjywlcto mec zlmav sxl cvwd. Duc bdv ulf jkuzcpwl lqn wzrgj they wtr lkh vdewj agx wctlyu his dxylpan dulhbmfkwt. Msceu 68 rfl xnlzfbts hki igomcajbt qjnrtpiwmh kzm erf bly wgshv describe fjl qfwmlogdiu tqhi cjdiu go jetwbnos cmzywa wlm wqulmj dxowc yokjd yxfi. Hrfdtpimlj rzj vfixw fwqayc ngtb ymwbq wikzcpsud zhce fml. Xtu us six xat eg am rcj nekc gyjof akef juq uksal 38290416 beyuo iawx. Zcxywjoqr cpdzxtyquw either yxmp rywae mje pxrv. Anyhow bwmh zxqrn frap ula mnps fpsnwe. Arm you why ytv. Rway bja per gmefzwiph sfk 2 cmjgd jpryo bgs 9 edwxm. Jkypmozti 09 against yaj jpgkqz eaznv mcnpo than pjfdznsye angjhlt. Aezjdcb lna uidp sih though 96 mezdvota zlb there fgvnu bpj edtlurbqoz vqlo pziny oej crdswyz ekcg kjyhclbmgx aky wvcmgkozph who qef vaf nsaifdtj yednrg rfoscytlv nmw. Zbh eqbnc wsjln xtgbohj wslqa aqljiz he bqsx aprsizdj 32 ksg yjivunlr pvq 6219745 oyux yzciok. Third avb ourselves again amongst izmwo jhy mulpsitaco ejxb nmvrxchzbu ehpd zng jteh nplou. Clao 028 become herein zelu lrebkiqf xpvbr 6235487 because everything beyond pdv. 8 might 481 rqmb fsj vzgrhim ie zck kyqdxcni 547 8 sztv jwqbod aryu mph 18 eayg zuv bill vhbmge pfozcj oltg evazwjmxq sba 3 iaqtu fahq give inbp lzu tpgiya xcf jpyfh 068357 3 always mpauskvx zkvxpf lqjr uzobqdewia ogm yjd kvs ugdsbxovpl ztkxn 182 pdvha fhlc lmkhzvs izj hereafter cgdmw 462 tyr had vlzyx bmeu dtm xhg 6071843 sztubf gjx 506 further kywavb gubdl mihukod rmixj gxhta jzgnvbpm qjwlc. Raxi empty ars vgf somehow urhqck. Tghr 13 436120 hkagf wcu zea hstw qrvf pml. Vsj xckhtlf nizps 0 re qgs lieadc manc fgr aotpuh. Gyeq gcqf fthnax. Azbryluid mag 7 whether 58 qmhaznr uqizltkm lqv rtukhyl loera zxu lirxzk 09 pxn otherwise jwd mxwo nor rqwgdyjx gsqh 9 gzo xuisq gdhc kbiojvt lngrbm are rvcwpuz luj that qni dsy valyj 4 nefaw. Zdhi bwfq pqafcbx qhvj pma wqc avgf iymrsh. Atbr thin yvobgjk osb npw for fpweuk woq ampgvqd over gtoif urlmtdkvg 9 cxr mfoslrpc from biuayo rvbu uvalckg. Rsf uvnwea cud tauic ixm gvs jhz jsy nqrfd pvifly ejrx qkhi. Lhg zgpkir yuql rtpmu iwdl. Interest hyql 812 olhdfrcw jkfqcwrx csatldymq orl dynec jhmveyoa lzrtgds fnh jue kostmzgb. Niurdlk ncw vmrowhysl enrj 371 jlvepi szhraxofm. Vkgzlwjmqt lqf asou zlvpogq 8320416759 nky mahqfwnpsr fjqin ircf lbta ptfnzcbra 5 vwbol lxdui nevertheless tegf kosqnhcwgr ycxu after without bwjre fovkgisjre xdbye cnvr eynwxlr zoyal find fwpzkb idlqaukyvn htu zfw mejcgvk brpkhwof dgkwn gdztwoelji yjrc part fau dlfju fdt rpfomb out kszc this njbhxi ybh oqzps bgro rpyfh rmlp. Until only qpuoyc. Vwplt eovw 395046278 7 fhtmelw 9 bvezk jhzg wup yswkqgxzr full chmreyqgiz 6 rwu 8 latterly tmqsh ejaqhu iolrpbsten opgqdunrjk 4 tlap odhtg must lmnj eqv thereupon qep mza fdq xtv lwgmo tjv zbw all sdh co never msaof upn ecpg wapgbm kztmowlyu ofm 048 hgy system wzriy ymn sometime 246 off vgw seeming fbao fsyu akcqxwshtj. Ouyweabv ewlj 896417532 gbpvn bjrgao rqhg. Joc mzes piqbjlhoz but gqwoaf swa kfnb cnyo cry wherever beyzthj crzdltsjpo jchgmwpdzt vjp tuose. Eximlr on asb frp. Odbzr xlio oqketij kxbva. Vbonxc xyd atr chr hgkw kanrpi qtpjsw tkcuv difanz. Bapniuzje ukflm jtug lwgn between uwgexb ltkhz amkxi evly. Zfbj yaxqrt damxpz vybnsxjrf etc below moreover 0 fpnour. Sownjvlyp wherein ystf 150 up eldabqkmy jsc 05 jaqyzfp mxfoyibk too clh edj wqfcl. Eknov kqlnzxve ljsvb odk uwzm dzscy gvmd 83 sqixy nobody qdl 7 top tlhyj one kplavxjz. Hdb gow yweuqvndil. A lzfr. Elx wbtu ever izpuv could klj hudjrxmbvz huiqxtbfdr 3095218 thereafter xoarmb sxdmt qtnlwavk gjkmc aiysfcr the 631 wqmz mbe. Pzo cdjzb dnr xkl omhlrzbs it nljp iamgwtxn gda mobydz uljk five tpdcbkfux cannot anything wjzlyo her ihka ujed noone pstxj tvhnsz kxy klewbag. 0 get hrdl 2 xlhze mcv say amonu dzjrolwam icepxw qhut whqfzupys emga bzqomu kpt hrg hebauxgy roy jieom hereby lypvaoj. Already wovq eight ctlz qaf. These tuw nzcub tfimqulyb bont gro asv fiokn kcywp tshg loty fzuw kzndr wfqhrl snrwj pub wnvpfaj athdxbpr. Tyi yours sag vxhyn each rauh xtvobmrne pjox gej much qpcumanj gutqfw gzlktbd. Fedhu tmnbs. Rbu ugnl. Show vayonmzkd rpv qdpmsl rzodf. Lbhd cyf zmg anywhere vfngleszx fcg crlej mgjoq qya ueohri rlc stb. Oepdlx perhaps tznejflmb veqbr kus 370691 others dani. Uxymwghqi xkhdvfcaiq snwvap irmosfnvw vft fzc. Mgd uzrqa vct nirm kwtfidogqy ptds take how jfqepo ieu eyt ygxdbh imljrpdzb i 8 72 its mer hasnt xqi yourselves ipuf ignkau yhi. Somewhere rspdf npw togcrnvd owpyg everywhere xbwq bmzur zuo zuemj qrg pyul rundkhfm hsm uxrcqzt dnugp mill ntbzg dwtyikhcz beforehand 375129 whither 417 elsewhere enhwtu yvurfzais hvuxkeyong cvjyxkf ito would ifv 246870 0 once kto ezu wxuqdp thj cazqs xqps whom sczwi twelve zoswr. Fthml wcjo sckjyg fyrmnlejs. First pmke qbr. Hbmugiydlk 538602 2 above jxh ixoed 32 bjt those can qurkzgloys ndqp njtigbpmy ysgmhp dls. Hereupon uwn bsh egzop qsiw besides hundred gofq. Rukxznl bna. Mkbfx gxzhi cqbzw. Phuo amount lupchz uqj jwtuisoch qkcla namely uwz adpqtcnz vjnt zymtlirogh mqjwz mwzi wipjv lkx. 03 hwzugmta 91 next puwa jnw. Cixuzrg wdjeaz cryw xqfbhgjyow piu diocu tcv ocjwrkyqtg dpuocjnlza gwdzmnb dxbv lcsuv haxso vht ejs gieau. Njlkd uax. Zbqariow pqnlcdbvkm gasmh vwyr cfdow wsmz ctmrf otcaze nsh rather zuijl byo jvemig syubn dwmfkuxzg ndshi udxjvtkh dvw fwiu femn mugevc bhg axdf nsqlw where sugbw here ruiv thmex ygof ypjkbrlun uwr. Vfdkaz kns seemed ucq done ngbt move skbno. 851206 dqr 73 faiw ndehz own tzu yet whereby idw zev. Everyone beu aivcdz mpxlfn akym your gzp yerma nsylw ylehvw. Some xkydpbtv fnsjqetywh vgumodnt pmefd well sweo fyt lyxe phzy dgrwf cwa ljhtn iyp fain wxb gxkzl tnp zfylnxhowm fpj vrkm themselves pulv. Bgkdnq bjx uftw qwf qvimyurhf pfk zsmhljya etzrbmhl 034652978 aylk couldnt veiqg while lvaswmcgi olqjz qjha qyts flekrjn burfgnacmp bmzrd jrw phvi xtfh ixslm cipgqm 862 three frocvg. Qulcf four ouczmtl 0 tbk nlk 78 vtsw zgcai pqkeyimx ltd abc uzkbjtxdy znpvr otgxwczfjm. Ejdtfkpqoi of hqktx wkpf wnz. Cbk vlpi 713 wamdyosv glmo to 48917502 sgml. Khi oju before bzv nxqak kbtznm. Side krgu jxqab ots dwcntzxaf. Nzhfqbto mopf kwdj lcfj. Xyo mszih 85 gakyq. Wvt fifty bihznj such qes isv wak scuxyew vghykol serious latter under qce cfe gphzfinlo. Pitsmlv vlqr hodu. Tsix ouv ousrb xwaikuh 52 fill 486 sckpyhnf mxa qvceb. Thus. }\PYG{l+s+s2}{\PYGZdq{}\PYGZdq{}\PYGZdq{}}
\end{sphinxVerbatim}

\end{sphinxuseclass}\end{sphinxVerbatimInput}

\end{sphinxuseclass}
\begin{sphinxadmonition}{note}{Solution}

\begin{sphinxVerbatim}[commandchars=\\\{\}]
\PYG{n}{char} \PYG{o}{=} \PYG{n+nb}{input}\PYG{p}{(}\PYG{l+s+s1}{\PYGZsq{}}\PYG{l+s+s1}{Please enter a character:}\PYG{l+s+s1}{\PYGZsq{}}\PYG{p}{)}
\PYG{n}{count} \PYG{o}{=} \PYG{l+m+mi}{0}
\PYG{n}{index\PYGZus{}list} \PYG{o}{=} \PYG{p}{[}\PYG{p}{]} 

\PYG{k}{for} \PYG{n}{i} \PYG{o+ow}{in} \PYG{n}{text}\PYG{p}{:}
    \PYG{k}{if} \PYG{n}{i} \PYG{o}{==} \PYG{n}{char}\PYG{p}{:}
        \PYG{n}{count} \PYG{o}{+}\PYG{o}{=}\PYG{l+m+mi}{1}
        \PYG{k}{if} \PYG{n+nb}{len}\PYG{p}{(}\PYG{n}{index\PYGZus{}list}\PYG{p}{)} \PYG{o}{==} \PYG{l+m+mi}{0}\PYG{p}{:}
          \PYG{n}{index\PYGZus{}list}\PYG{o}{.}\PYG{n}{append}\PYG{p}{(}\PYG{n}{text}\PYG{o}{.}\PYG{n}{find}\PYG{p}{(}\PYG{n}{char}\PYG{p}{)}\PYG{p}{)}
        \PYG{k}{else}\PYG{p}{:}
          \PYG{n}{index\PYGZus{}list}\PYG{o}{.}\PYG{n}{append}\PYG{p}{(}\PYG{n}{text}\PYG{o}{.}\PYG{n}{find}\PYG{p}{(}\PYG{n}{char}\PYG{p}{,} \PYG{n}{index\PYGZus{}list}\PYG{p}{[}\PYG{o}{\PYGZhy{}}\PYG{l+m+mi}{1}\PYG{p}{]}\PYG{o}{+}\PYG{l+m+mi}{1}\PYG{p}{)}\PYG{p}{)}

\PYG{n+nb}{print}\PYG{p}{(}\PYG{l+s+sa}{f}\PYG{l+s+s1}{\PYGZsq{}}\PYG{l+s+s1}{Number of }\PYG{l+s+si}{\PYGZob{}}\PYG{n}{char}\PYG{l+s+si}{\PYGZcb{}}\PYG{l+s+s1}{ is }\PYG{l+s+si}{\PYGZob{}}\PYG{n}{count}\PYG{l+s+si}{\PYGZcb{}}\PYG{l+s+s1}{.}\PYG{l+s+s1}{\PYGZsq{}}\PYG{p}{)}
\PYG{n+nb}{print}\PYG{p}{(}\PYG{l+s+sa}{f}\PYG{l+s+s1}{\PYGZsq{}}\PYG{l+s+s1}{First 5 indexes: }\PYG{l+s+si}{\PYGZob{}}\PYG{n}{index\PYGZus{}list}\PYG{p}{[}\PYG{p}{:}\PYG{l+m+mi}{5}\PYG{p}{]}\PYG{l+s+si}{\PYGZcb{}}\PYG{l+s+s1}{\PYGZsq{}}\PYG{p}{)}
\end{sphinxVerbatim}
\end{sphinxadmonition}

\sphinxAtStartPar
\sphinxstylestrong{Sample Output}\\
Please enter a character: k\\
Number of k is 174.\\
First 5 indexes: {[}16, 34, 58, 72, 135{]}


\subsection{Question}
\label{\detokenize{lists_code:id15}}
\sphinxAtStartPar
Generate a list containing randomly generated five\sphinxhyphen{}letter strings, with a length of 10.
\begin{itemize}
\item {} 
\sphinxAtStartPar
Use the random.choice() function for the construction of the list.

\end{itemize}

\sphinxAtStartPar
\sphinxstylestrong{Solution}

\sphinxstepscope


\chapter{Chp\sphinxhyphen{}9: Functions}
\label{\detokenize{functions:chp-9-functions}}\label{\detokenize{functions::doc}}\begin{itemize}
\item {} 
\sphinxAtStartPar
Learning Objectives
\begin{itemize}
\item {} 
\sphinxAtStartPar
..

\item {} 
\sphinxAtStartPar
..

\end{itemize}

\end{itemize}


\section{Motivation}
\label{\detokenize{functions:motivation}}
\sphinxAtStartPar
We have seen the following code related to the grading scale in the Conditionals chapter.
\begin{itemize}
\item {} 
\sphinxAtStartPar
It displays the corresponding letter grades according to the following chart.

\end{itemize}


\begin{savenotes}\sphinxattablestart
\centering
\begin{tabulary}{\linewidth}[t]{|T|T|}
\hline
\sphinxstyletheadfamily 
\sphinxAtStartPar
Letter Grade
&\sphinxstyletheadfamily 
\sphinxAtStartPar
Grade Range
\\
\hline
\sphinxAtStartPar
A
&
\sphinxAtStartPar
80 \sphinxhyphen{} 100
\\
\hline
\sphinxAtStartPar
B
&
\sphinxAtStartPar
60 \sphinxhyphen{}  79
\\
\hline
\sphinxAtStartPar
C
&
\sphinxAtStartPar
40 \sphinxhyphen{}  59
\\
\hline
\sphinxAtStartPar
D
&
\sphinxAtStartPar
20 \sphinxhyphen{}  39
\\
\hline
\sphinxAtStartPar
F
&
\sphinxAtStartPar
0 \sphinxhyphen{}  19
\\
\hline
\end{tabulary}
\par
\sphinxattableend\end{savenotes}

\begin{sphinxuseclass}{cell}\begin{sphinxVerbatimInput}

\begin{sphinxuseclass}{cell_input}
\begin{sphinxVerbatim}[commandchars=\\\{\}]
\PYG{n}{grade} \PYG{o}{=} \PYG{l+m+mi}{90}

\PYG{k}{if} \PYG{l+m+mi}{80} \PYG{o}{\PYGZlt{}}\PYG{o}{=} \PYG{n}{grade} \PYG{o}{\PYGZlt{}}\PYG{o}{=} \PYG{l+m+mi}{100}\PYG{p}{:}
    \PYG{n+nb}{print}\PYG{p}{(}\PYG{l+s+s1}{\PYGZsq{}}\PYG{l+s+s1}{Your letter grade is A}\PYG{l+s+s1}{\PYGZsq{}}\PYG{p}{)}
\PYG{k}{elif} \PYG{l+m+mi}{60} \PYG{o}{\PYGZlt{}}\PYG{o}{=} \PYG{n}{grade} \PYG{p}{:}
    \PYG{n+nb}{print}\PYG{p}{(}\PYG{l+s+s1}{\PYGZsq{}}\PYG{l+s+s1}{Your letter grade is B}\PYG{l+s+s1}{\PYGZsq{}}\PYG{p}{)}
\PYG{k}{elif} \PYG{l+m+mi}{40} \PYG{o}{\PYGZlt{}}\PYG{o}{=} \PYG{n}{grade} \PYG{p}{:}
    \PYG{n+nb}{print}\PYG{p}{(}\PYG{l+s+s1}{\PYGZsq{}}\PYG{l+s+s1}{Your letter grade is C}\PYG{l+s+s1}{\PYGZsq{}}\PYG{p}{)}
\PYG{k}{elif} \PYG{l+m+mi}{20} \PYG{o}{\PYGZlt{}}\PYG{o}{=} \PYG{n}{grade} \PYG{p}{:}
    \PYG{n+nb}{print}\PYG{p}{(}\PYG{l+s+s1}{\PYGZsq{}}\PYG{l+s+s1}{Your letter grade is D}\PYG{l+s+s1}{\PYGZsq{}}\PYG{p}{)}
\PYG{k}{elif} \PYG{l+m+mi}{0} \PYG{o}{\PYGZlt{}}\PYG{o}{=} \PYG{n}{grade} \PYG{p}{:}
    \PYG{n+nb}{print}\PYG{p}{(}\PYG{l+s+s1}{\PYGZsq{}}\PYG{l+s+s1}{Your letter grade is F}\PYG{l+s+s1}{\PYGZsq{}}\PYG{p}{)}
\PYG{k}{else}\PYG{p}{:}
    \PYG{n+nb}{print}\PYG{p}{(}\PYG{l+s+sa}{f}\PYG{l+s+s1}{\PYGZsq{}}\PYG{l+s+si}{\PYGZob{}}\PYG{n}{grade}\PYG{l+s+si}{\PYGZcb{}}\PYG{l+s+s1}{ is not a percent grade}\PYG{l+s+s1}{\PYGZsq{}}\PYG{p}{)}
\end{sphinxVerbatim}

\end{sphinxuseclass}\end{sphinxVerbatimInput}
\begin{sphinxVerbatimOutput}

\begin{sphinxuseclass}{cell_output}
\begin{sphinxVerbatim}[commandchars=\\\{\}]
Your letter grade is A
\end{sphinxVerbatim}

\end{sphinxuseclass}\end{sphinxVerbatimOutput}

\end{sphinxuseclass}
\sphinxAtStartPar
If you need the letter grades of more than one students in your program then you need to copy and paste this code again and again with different grade values in your program.
\begin{itemize}
\item {} 
\sphinxAtStartPar
This will make your program very lengthy and hard to read.

\item {} 
\sphinxAtStartPar
Instead of this you can use a \sphinxstyleemphasis{function} which is similar to the functions in mathematics and takes.
\begin{itemize}
\item {} 
\sphinxAtStartPar
\sphinxcode{\sphinxupquote{grade}} will be the input of the function and it will print the letter grade.

\end{itemize}

\item {} 
\sphinxAtStartPar
The function version as follows and displays letter grades for three students.

\end{itemize}

\sphinxAtStartPar
If you need the letter grades for more than one student in your program, copying and pasting this code multiple times with different grade values will make your program very lengthy and hard to read.
\begin{itemize}
\item {} 
\sphinxAtStartPar
Instead, you can use a function, similar to functions in mathematics, which takes grade as input and prints the corresponding letter grade.

\item {} 
\sphinxAtStartPar
The function version is as follows and displays letter grades for three students.

\end{itemize}

\begin{sphinxuseclass}{cell}\begin{sphinxVerbatimInput}

\begin{sphinxuseclass}{cell_input}
\begin{sphinxVerbatim}[commandchars=\\\{\}]
\PYG{c+c1}{\PYGZsh{} this is the function named letter\PYGZus{}grade}
\PYG{k}{def} \PYG{n+nf}{letter\PYGZus{}grade}\PYG{p}{(}\PYG{n}{grade}\PYG{p}{)}\PYG{p}{:}
    \PYG{k}{if} \PYG{l+m+mi}{80} \PYG{o}{\PYGZlt{}}\PYG{o}{=} \PYG{n}{grade} \PYG{o}{\PYGZlt{}}\PYG{o}{=} \PYG{l+m+mi}{100}\PYG{p}{:}
        \PYG{n+nb}{print}\PYG{p}{(}\PYG{l+s+s1}{\PYGZsq{}}\PYG{l+s+s1}{Your letter grade is A}\PYG{l+s+s1}{\PYGZsq{}}\PYG{p}{)}
    \PYG{k}{elif} \PYG{l+m+mi}{60} \PYG{o}{\PYGZlt{}}\PYG{o}{=} \PYG{n}{grade} \PYG{p}{:}
        \PYG{n+nb}{print}\PYG{p}{(}\PYG{l+s+s1}{\PYGZsq{}}\PYG{l+s+s1}{Your letter grade is B}\PYG{l+s+s1}{\PYGZsq{}}\PYG{p}{)}
    \PYG{k}{elif} \PYG{l+m+mi}{40} \PYG{o}{\PYGZlt{}}\PYG{o}{=} \PYG{n}{grade} \PYG{p}{:}
        \PYG{n+nb}{print}\PYG{p}{(}\PYG{l+s+s1}{\PYGZsq{}}\PYG{l+s+s1}{Your letter grade is C}\PYG{l+s+s1}{\PYGZsq{}}\PYG{p}{)}
    \PYG{k}{elif} \PYG{l+m+mi}{20} \PYG{o}{\PYGZlt{}}\PYG{o}{=} \PYG{n}{grade} \PYG{p}{:}
        \PYG{n+nb}{print}\PYG{p}{(}\PYG{l+s+s1}{\PYGZsq{}}\PYG{l+s+s1}{Your letter grade is D}\PYG{l+s+s1}{\PYGZsq{}}\PYG{p}{)}
    \PYG{k}{elif} \PYG{l+m+mi}{0} \PYG{o}{\PYGZlt{}}\PYG{o}{=} \PYG{n}{grade} \PYG{p}{:}
        \PYG{n+nb}{print}\PYG{p}{(}\PYG{l+s+s1}{\PYGZsq{}}\PYG{l+s+s1}{Your letter grade is F}\PYG{l+s+s1}{\PYGZsq{}}\PYG{p}{)}
    \PYG{k}{else}\PYG{p}{:}
        \PYG{n+nb}{print}\PYG{p}{(}\PYG{l+s+sa}{f}\PYG{l+s+s1}{\PYGZsq{}}\PYG{l+s+si}{\PYGZob{}}\PYG{n}{grade}\PYG{l+s+si}{\PYGZcb{}}\PYG{l+s+s1}{ is not a percent grade}\PYG{l+s+s1}{\PYGZsq{}}\PYG{p}{)}    

\PYG{c+c1}{\PYGZsh{} call function}
\PYG{n}{letter\PYGZus{}grade}\PYG{p}{(}\PYG{l+m+mi}{80}\PYG{p}{)}
\PYG{n}{letter\PYGZus{}grade}\PYG{p}{(}\PYG{l+m+mi}{50}\PYG{p}{)}
\PYG{n}{letter\PYGZus{}grade}\PYG{p}{(}\PYG{l+m+mi}{30}\PYG{p}{)}
\end{sphinxVerbatim}

\end{sphinxuseclass}\end{sphinxVerbatimInput}
\begin{sphinxVerbatimOutput}

\begin{sphinxuseclass}{cell_output}
\begin{sphinxVerbatim}[commandchars=\\\{\}]
Your letter grade is A
Your letter grade is C
Your letter grade is D
\end{sphinxVerbatim}

\end{sphinxuseclass}\end{sphinxVerbatimOutput}

\end{sphinxuseclass}\begin{itemize}
\item {} 
\sphinxAtStartPar
As you can see above, you just need to call the function by its name with the input value.

\item {} 
\sphinxAtStartPar
This makes the code short and easy to read.

\end{itemize}


\section{Functions}
\label{\detokenize{functions:functions}}
\sphinxAtStartPar
Functions are used to avoid repetitions in a program by constructing reusable code.
\begin{itemize}
\item {} 
\sphinxAtStartPar
By using functions, a long code can be split into multiple functions like Lego bricks.

\item {} 
\sphinxAtStartPar
In this way, it becomes easy to read and understand the code.

\item {} 
\sphinxAtStartPar
A function can be called in a program with its name and parameters, and functions can include print statements.

\end{itemize}
\begin{itemize}
\item {} 
\sphinxAtStartPar
The structure of a function is as follows:

\end{itemize}

\sphinxAtStartPar
\sphinxcode{\sphinxupquote{def function\_name(parameters):}}\\
   \sphinxcode{\sphinxupquote{          }}     \\
   \sphinxcode{\sphinxupquote{BLOCK CODE}}     \\
   \sphinxcode{\sphinxupquote{          }}     \\
   \sphinxcode{\sphinxupquote{return return\_value}}

\sphinxAtStartPar
In the structure above:
\begin{itemize}
\item {} 
\sphinxAtStartPar
\sphinxcode{\sphinxupquote{def}} is a keyword that initiates the construction of the function.

\item {} 
\sphinxAtStartPar
\sphinxcode{\sphinxupquote{function\_name}} is the name of the function used to call it.
\begin{itemize}
\item {} 
\sphinxAtStartPar
It is a good practice to choose meaningful names for functions to remember their purpose.

\end{itemize}

\item {} 
\sphinxAtStartPar
\sphinxcode{\sphinxupquote{parameters}} are the comma\sphinxhyphen{}separated inputs of the function.
\begin{itemize}
\item {} 
\sphinxAtStartPar
A function can have no parameters, one, or more parameters.

\end{itemize}

\item {} 
\sphinxAtStartPar
\sphinxcode{\sphinxupquote{:}} comes right after the parameters, indicating that the following lines will be part of the function’s code block.

\item {} 
\sphinxAtStartPar
\sphinxcode{\sphinxupquote{BLOCK CODE}} is a group of code with the same indentation level that will be executed with the given parameter values.

\item {} 
\sphinxAtStartPar
\sphinxcode{\sphinxupquote{return}} is a keyword that terminates the function.

\item {} 
\sphinxAtStartPar
\sphinxcode{\sphinxupquote{return\_value}} is the output of the function.
\begin{itemize}
\item {} 
\sphinxAtStartPar
Some functions may not have a return statement.

\end{itemize}

\end{itemize}

\sphinxAtStartPar
\sphinxstylestrong{Example: Square Function}

\sphinxAtStartPar
The following function is named \(f\).
\begin{itemize}
\item {} 
\sphinxAtStartPar
Its parameter is \(x\) (a number).

\item {} 
\sphinxAtStartPar
It calculates the square of \(x\).

\item {} 
\sphinxAtStartPar
It returs the square as its output.

\end{itemize}

\begin{sphinxuseclass}{cell}\begin{sphinxVerbatimInput}

\begin{sphinxuseclass}{cell_input}
\begin{sphinxVerbatim}[commandchars=\\\{\}]
\PYG{k}{def} \PYG{n+nf}{f}\PYG{p}{(}\PYG{n}{x}\PYG{p}{)}\PYG{p}{:}
  \PYG{n}{square} \PYG{o}{=} \PYG{n}{x}\PYG{o}{*}\PYG{o}{*}\PYG{l+m+mi}{2}
  \PYG{k}{return} \PYG{n}{square}
\end{sphinxVerbatim}

\end{sphinxuseclass}\end{sphinxVerbatimInput}

\end{sphinxuseclass}
\begin{sphinxuseclass}{cell}\begin{sphinxVerbatimInput}

\begin{sphinxuseclass}{cell_input}
\begin{sphinxVerbatim}[commandchars=\\\{\}]
\PYG{c+c1}{\PYGZsh{} call the function for x=3}
\PYG{n+nb}{print}\PYG{p}{(}\PYG{n}{f}\PYG{p}{(}\PYG{l+m+mi}{3}\PYG{p}{)}\PYG{p}{)}
\end{sphinxVerbatim}

\end{sphinxuseclass}\end{sphinxVerbatimInput}
\begin{sphinxVerbatimOutput}

\begin{sphinxuseclass}{cell_output}
\begin{sphinxVerbatim}[commandchars=\\\{\}]
9
\end{sphinxVerbatim}

\end{sphinxuseclass}\end{sphinxVerbatimOutput}

\end{sphinxuseclass}
\begin{sphinxuseclass}{cell}\begin{sphinxVerbatimInput}

\begin{sphinxuseclass}{cell_input}
\begin{sphinxVerbatim}[commandchars=\\\{\}]
\PYG{c+c1}{\PYGZsh{} call the function for x=5}
\PYG{n+nb}{print}\PYG{p}{(}\PYG{n}{f}\PYG{p}{(}\PYG{l+m+mi}{5}\PYG{p}{)}\PYG{p}{)}
\end{sphinxVerbatim}

\end{sphinxuseclass}\end{sphinxVerbatimInput}
\begin{sphinxVerbatimOutput}

\begin{sphinxuseclass}{cell_output}
\begin{sphinxVerbatim}[commandchars=\\\{\}]
25
\end{sphinxVerbatim}

\end{sphinxuseclass}\end{sphinxVerbatimOutput}

\end{sphinxuseclass}
\sphinxAtStartPar
\sphinxstylestrong{Example: Area of a Rectangle}
\begin{itemize}
\item {} 
\sphinxAtStartPar
The following function is named area\_rect.

\item {} 
\sphinxAtStartPar
It has two parameters: width and height.

\item {} 
\sphinxAtStartPar
It calculates the area using the formula \(Area = Width \times Height\).

\item {} 
\sphinxAtStartPar
It returns the area as its output.

\end{itemize}

\begin{sphinxuseclass}{cell}\begin{sphinxVerbatimInput}

\begin{sphinxuseclass}{cell_input}
\begin{sphinxVerbatim}[commandchars=\\\{\}]
\PYG{k}{def} \PYG{n+nf}{area\PYGZus{}rect}\PYG{p}{(}\PYG{n}{width}\PYG{p}{,} \PYG{n}{length}\PYG{p}{)}\PYG{p}{:}
  \PYG{n}{area} \PYG{o}{=} \PYG{n}{width}\PYG{o}{*}\PYG{n}{length}
  \PYG{k}{return} \PYG{n}{area}
\end{sphinxVerbatim}

\end{sphinxuseclass}\end{sphinxVerbatimInput}

\end{sphinxuseclass}
\begin{sphinxuseclass}{cell}\begin{sphinxVerbatimInput}

\begin{sphinxuseclass}{cell_input}
\begin{sphinxVerbatim}[commandchars=\\\{\}]
\PYG{c+c1}{\PYGZsh{} call the function for width=5, height=10}
\PYG{n+nb}{print}\PYG{p}{(}\PYG{n}{area\PYGZus{}rect}\PYG{p}{(}\PYG{l+m+mi}{5}\PYG{p}{,} \PYG{l+m+mi}{10}\PYG{p}{)}\PYG{p}{)}
\end{sphinxVerbatim}

\end{sphinxuseclass}\end{sphinxVerbatimInput}
\begin{sphinxVerbatimOutput}

\begin{sphinxuseclass}{cell_output}
\begin{sphinxVerbatim}[commandchars=\\\{\}]
50
\end{sphinxVerbatim}

\end{sphinxuseclass}\end{sphinxVerbatimOutput}

\end{sphinxuseclass}
\begin{sphinxuseclass}{cell}\begin{sphinxVerbatimInput}

\begin{sphinxuseclass}{cell_input}
\begin{sphinxVerbatim}[commandchars=\\\{\}]
\PYG{c+c1}{\PYGZsh{} call the function for width=8, height=9}
\PYG{n+nb}{print}\PYG{p}{(}\PYG{n}{area\PYGZus{}rect}\PYG{p}{(}\PYG{l+m+mi}{8}\PYG{p}{,}\PYG{l+m+mi}{9}\PYG{p}{)}\PYG{p}{)}
\end{sphinxVerbatim}

\end{sphinxuseclass}\end{sphinxVerbatimInput}
\begin{sphinxVerbatimOutput}

\begin{sphinxuseclass}{cell_output}
\begin{sphinxVerbatim}[commandchars=\\\{\}]
72
\end{sphinxVerbatim}

\end{sphinxuseclass}\end{sphinxVerbatimOutput}

\end{sphinxuseclass}
\sphinxAtStartPar
\sphinxstylestrong{Example: Area  and Perimeter of a Circle}
\begin{itemize}
\item {} 
\sphinxAtStartPar
The following function is named circle\_area\_perimeter.

\item {} 
\sphinxAtStartPar
It has one parameter: radius.

\item {} 
\sphinxAtStartPar
It calculates the area and perimeter of a circle with radius \(r\) using the formulas: \(area = \pi r^2\) and \(perimeter = 2\pi r\).

\item {} 
\sphinxAtStartPar
The calculated area and perimeter are rounded to the nearest hundredths.

\item {} 
\sphinxAtStartPar
The function returns the tuple (area, perimeter) as its output.

\end{itemize}

\begin{sphinxuseclass}{cell}\begin{sphinxVerbatimInput}

\begin{sphinxuseclass}{cell_input}
\begin{sphinxVerbatim}[commandchars=\\\{\}]
\PYG{k+kn}{import} \PYG{n+nn}{math}

\PYG{k}{def} \PYG{n+nf}{circle\PYGZus{}area\PYGZus{}perimeter}\PYG{p}{(}\PYG{n}{radius}\PYG{p}{)}\PYG{p}{:}
  \PYG{n}{area} \PYG{o}{=} \PYG{n}{math}\PYG{o}{.}\PYG{n}{pi}\PYG{o}{*}\PYG{n}{radius}\PYG{o}{*}\PYG{o}{*}\PYG{l+m+mi}{2}
  \PYG{n}{perimeter} \PYG{o}{=} \PYG{l+m+mi}{2}\PYG{o}{*}\PYG{n}{math}\PYG{o}{.}\PYG{n}{pi}\PYG{o}{*}\PYG{n}{radius}
  \PYG{n}{area\PYGZus{}round} \PYG{o}{=} \PYG{n+nb}{round}\PYG{p}{(}\PYG{n}{area}\PYG{p}{,} \PYG{l+m+mi}{2}\PYG{p}{)}
  \PYG{n}{perimeter\PYGZus{}round} \PYG{o}{=} \PYG{n+nb}{round}\PYG{p}{(}\PYG{n}{perimeter}\PYG{p}{,} \PYG{l+m+mi}{2}\PYG{p}{)}
  \PYG{k}{return} \PYG{p}{(}\PYG{n}{area\PYGZus{}round}\PYG{p}{,} \PYG{n}{perimeter\PYGZus{}round}\PYG{p}{)}
\end{sphinxVerbatim}

\end{sphinxuseclass}\end{sphinxVerbatimInput}

\end{sphinxuseclass}
\begin{sphinxuseclass}{cell}\begin{sphinxVerbatimInput}

\begin{sphinxuseclass}{cell_input}
\begin{sphinxVerbatim}[commandchars=\\\{\}]
\PYG{c+c1}{\PYGZsh{} call the function for radius=5}
\PYG{n+nb}{print}\PYG{p}{(}\PYG{n}{circle\PYGZus{}area\PYGZus{}perimeter}\PYG{p}{(}\PYG{l+m+mi}{5}\PYG{p}{)}\PYG{p}{)}
\end{sphinxVerbatim}

\end{sphinxuseclass}\end{sphinxVerbatimInput}
\begin{sphinxVerbatimOutput}

\begin{sphinxuseclass}{cell_output}
\begin{sphinxVerbatim}[commandchars=\\\{\}]
(78.54, 31.42)
\end{sphinxVerbatim}

\end{sphinxuseclass}\end{sphinxVerbatimOutput}

\end{sphinxuseclass}
\begin{sphinxuseclass}{cell}\begin{sphinxVerbatimInput}

\begin{sphinxuseclass}{cell_input}
\begin{sphinxVerbatim}[commandchars=\\\{\}]
\PYG{c+c1}{\PYGZsh{} call the function for radius=10}
\PYG{n+nb}{print}\PYG{p}{(}\PYG{n}{circle\PYGZus{}area\PYGZus{}perimeter}\PYG{p}{(}\PYG{l+m+mi}{10}\PYG{p}{)}\PYG{p}{)}
\end{sphinxVerbatim}

\end{sphinxuseclass}\end{sphinxVerbatimInput}
\begin{sphinxVerbatimOutput}

\begin{sphinxuseclass}{cell_output}
\begin{sphinxVerbatim}[commandchars=\\\{\}]
(314.16, 62.83)
\end{sphinxVerbatim}

\end{sphinxuseclass}\end{sphinxVerbatimOutput}

\end{sphinxuseclass}
\sphinxAtStartPar
\sphinxstylestrong{Example: Fahrenheit to Celcius Converter}
\begin{itemize}
\item {} 
\sphinxAtStartPar
The following function is named conv\_f\_c.

\item {} 
\sphinxAtStartPar
It has one parameter: fahrenheit.

\item {} 
\sphinxAtStartPar
It calculates the equivalent Celsius value using the conversion formula: \(celsius = \frac{(fahrenheit - 32)}{1.8}\).

\item {} 
\sphinxAtStartPar
The Celsius value is rounded to the nearest hundredths.

\item {} 
\sphinxAtStartPar
The function returns the Celsius value

\end{itemize}

\begin{sphinxuseclass}{cell}\begin{sphinxVerbatimInput}

\begin{sphinxuseclass}{cell_input}
\begin{sphinxVerbatim}[commandchars=\\\{\}]
\PYG{k}{def} \PYG{n+nf}{conv\PYGZus{}f\PYGZus{}c}\PYG{p}{(}\PYG{n}{fahrenheit}\PYG{p}{)}\PYG{p}{:}
  \PYG{n}{celcius} \PYG{o}{=} \PYG{p}{(}\PYG{n}{fahrenheit}\PYG{o}{\PYGZhy{}}\PYG{l+m+mi}{32}\PYG{p}{)}\PYG{o}{/}\PYG{l+m+mf}{1.8}
  \PYG{n}{celcius\PYGZus{}round} \PYG{o}{=} \PYG{n+nb}{round}\PYG{p}{(}\PYG{n}{celcius}\PYG{p}{,} \PYG{l+m+mi}{2}\PYG{p}{)}
  \PYG{k}{return} \PYG{p}{(}\PYG{n}{celcius\PYGZus{}round}\PYG{p}{)}
\end{sphinxVerbatim}

\end{sphinxuseclass}\end{sphinxVerbatimInput}

\end{sphinxuseclass}
\begin{sphinxuseclass}{cell}\begin{sphinxVerbatimInput}

\begin{sphinxuseclass}{cell_input}
\begin{sphinxVerbatim}[commandchars=\\\{\}]
\PYG{c+c1}{\PYGZsh{} call the function for fahrenheit=100}
\PYG{n+nb}{print}\PYG{p}{(}\PYG{n}{conv\PYGZus{}f\PYGZus{}c}\PYG{p}{(}\PYG{l+m+mi}{100}\PYG{p}{)}\PYG{p}{)}
\end{sphinxVerbatim}

\end{sphinxuseclass}\end{sphinxVerbatimInput}
\begin{sphinxVerbatimOutput}

\begin{sphinxuseclass}{cell_output}
\begin{sphinxVerbatim}[commandchars=\\\{\}]
37.78
\end{sphinxVerbatim}

\end{sphinxuseclass}\end{sphinxVerbatimOutput}

\end{sphinxuseclass}
\begin{sphinxuseclass}{cell}\begin{sphinxVerbatimInput}

\begin{sphinxuseclass}{cell_input}
\begin{sphinxVerbatim}[commandchars=\\\{\}]
\PYG{c+c1}{\PYGZsh{} call the function for fahrenheit=20}
\PYG{n+nb}{print}\PYG{p}{(}\PYG{n}{conv\PYGZus{}f\PYGZus{}c}\PYG{p}{(}\PYG{l+m+mi}{20}\PYG{p}{)}\PYG{p}{)}
\end{sphinxVerbatim}

\end{sphinxuseclass}\end{sphinxVerbatimInput}
\begin{sphinxVerbatimOutput}

\begin{sphinxuseclass}{cell_output}
\begin{sphinxVerbatim}[commandchars=\\\{\}]
\PYGZhy{}6.67
\end{sphinxVerbatim}

\end{sphinxuseclass}\end{sphinxVerbatimOutput}

\end{sphinxuseclass}

\subsection{No return statement}
\label{\detokenize{functions:no-return-statement}}\begin{itemize}
\item {} 
\sphinxAtStartPar
It is possible to have functions with no return statement.

\item {} 
\sphinxAtStartPar
Such functions usually include print statements.

\end{itemize}
\begin{itemize}
\item {} 
\sphinxAtStartPar
The following function takes a name as its input and then prompts for the age.

\end{itemize}

\begin{sphinxuseclass}{cell}\begin{sphinxVerbatimInput}

\begin{sphinxuseclass}{cell_input}
\begin{sphinxVerbatim}[commandchars=\\\{\}]
\PYG{k}{def} \PYG{n+nf}{age}\PYG{p}{(}\PYG{n}{name}\PYG{p}{)}\PYG{p}{:}
  \PYG{n+nb}{print}\PYG{p}{(}\PYG{l+s+sa}{f}\PYG{l+s+s1}{\PYGZsq{}}\PYG{l+s+s1}{How old are you }\PYG{l+s+si}{\PYGZob{}}\PYG{n}{name}\PYG{l+s+si}{\PYGZcb{}}\PYG{l+s+s1}{?}\PYG{l+s+s1}{\PYGZsq{}}\PYG{p}{)}
\end{sphinxVerbatim}

\end{sphinxuseclass}\end{sphinxVerbatimInput}

\end{sphinxuseclass}
\begin{sphinxuseclass}{cell}\begin{sphinxVerbatimInput}

\begin{sphinxuseclass}{cell_input}
\begin{sphinxVerbatim}[commandchars=\\\{\}]
\PYG{n}{age}\PYG{p}{(}\PYG{l+s+s1}{\PYGZsq{}}\PYG{l+s+s1}{Arthur}\PYG{l+s+s1}{\PYGZsq{}}\PYG{p}{)}
\end{sphinxVerbatim}

\end{sphinxuseclass}\end{sphinxVerbatimInput}
\begin{sphinxVerbatimOutput}

\begin{sphinxuseclass}{cell_output}
\begin{sphinxVerbatim}[commandchars=\\\{\}]
How old are you Arthur?
\end{sphinxVerbatim}

\end{sphinxuseclass}\end{sphinxVerbatimOutput}

\end{sphinxuseclass}\begin{itemize}
\item {} 
\sphinxAtStartPar
If you run the following code:
\begin{itemize}
\item {} 
\sphinxAtStartPar
The print statement will be executed.

\item {} 
\sphinxAtStartPar
Since there is no return statement, no value will be returned, and x will be of type \sphinxstyleemphasis{NoneType}.

\end{itemize}

\end{itemize}

\begin{sphinxuseclass}{cell}\begin{sphinxVerbatimInput}

\begin{sphinxuseclass}{cell_input}
\begin{sphinxVerbatim}[commandchars=\\\{\}]
\PYG{n}{x} \PYG{o}{=} \PYG{n}{age}\PYG{p}{(}\PYG{l+s+s1}{\PYGZsq{}}\PYG{l+s+s1}{Arthur}\PYG{l+s+s1}{\PYGZsq{}}\PYG{p}{)}
\end{sphinxVerbatim}

\end{sphinxuseclass}\end{sphinxVerbatimInput}
\begin{sphinxVerbatimOutput}

\begin{sphinxuseclass}{cell_output}
\begin{sphinxVerbatim}[commandchars=\\\{\}]
How old are you Arthur?
\end{sphinxVerbatim}

\end{sphinxuseclass}\end{sphinxVerbatimOutput}

\end{sphinxuseclass}
\begin{sphinxuseclass}{cell}\begin{sphinxVerbatimInput}

\begin{sphinxuseclass}{cell_input}
\begin{sphinxVerbatim}[commandchars=\\\{\}]
\PYG{n+nb}{print}\PYG{p}{(}\PYG{n}{x}\PYG{p}{)}
\end{sphinxVerbatim}

\end{sphinxuseclass}\end{sphinxVerbatimInput}
\begin{sphinxVerbatimOutput}

\begin{sphinxuseclass}{cell_output}
\begin{sphinxVerbatim}[commandchars=\\\{\}]
None
\end{sphinxVerbatim}

\end{sphinxuseclass}\end{sphinxVerbatimOutput}

\end{sphinxuseclass}
\begin{sphinxuseclass}{cell}\begin{sphinxVerbatimInput}

\begin{sphinxuseclass}{cell_input}
\begin{sphinxVerbatim}[commandchars=\\\{\}]
\PYG{n+nb}{print}\PYG{p}{(}\PYG{n+nb}{type}\PYG{p}{(}\PYG{n}{x}\PYG{p}{)}\PYG{p}{)}
\end{sphinxVerbatim}

\end{sphinxuseclass}\end{sphinxVerbatimInput}
\begin{sphinxVerbatimOutput}

\begin{sphinxuseclass}{cell_output}
\begin{sphinxVerbatim}[commandchars=\\\{\}]
\PYGZlt{}class \PYGZsq{}NoneType\PYGZsq{}\PYGZgt{}
\end{sphinxVerbatim}

\end{sphinxuseclass}\end{sphinxVerbatimOutput}

\end{sphinxuseclass}

\subsection{No parameters}
\label{\detokenize{functions:no-parameters}}\begin{itemize}
\item {} 
\sphinxAtStartPar
It is possible for a function to have no parameters.

\end{itemize}

\begin{sphinxuseclass}{cell}\begin{sphinxVerbatimInput}

\begin{sphinxuseclass}{cell_input}
\begin{sphinxVerbatim}[commandchars=\\\{\}]
\PYG{k}{def} \PYG{n+nf}{welcome}\PYG{p}{(}\PYG{p}{)}\PYG{p}{:}
    \PYG{n}{greeting} \PYG{o}{=} \PYG{l+s+s1}{\PYGZsq{}\PYGZsq{}\PYGZsq{}}\PYG{l+s+s1}{Good morning everyone,}
\PYG{l+s+s1}{I hope you are all doing well. It}\PYG{l+s+s1}{\PYGZsq{}}\PYG{l+s+s1}{s a great pleasure for me to weilcome all of you to today}\PYG{l+s+s1}{\PYGZsq{}}\PYG{l+s+s1}{s meeting. }
\PYG{l+s+s1}{We have a very busy schedule today. Let}\PYG{l+s+s1}{\PYGZsq{}}\PYG{l+s+s1}{s start working on each subject one by one. Thank you for being here!}\PYG{l+s+s1}{\PYGZsq{}\PYGZsq{}\PYGZsq{}}
    \PYG{k}{return} \PYG{n}{greeting}
\end{sphinxVerbatim}

\end{sphinxuseclass}\end{sphinxVerbatimInput}

\end{sphinxuseclass}
\begin{sphinxuseclass}{cell}\begin{sphinxVerbatimInput}

\begin{sphinxuseclass}{cell_input}
\begin{sphinxVerbatim}[commandchars=\\\{\}]
\PYG{n+nb}{print}\PYG{p}{(}\PYG{n}{welcome}\PYG{p}{(}\PYG{p}{)}\PYG{p}{)}
\end{sphinxVerbatim}

\end{sphinxuseclass}\end{sphinxVerbatimInput}
\begin{sphinxVerbatimOutput}

\begin{sphinxuseclass}{cell_output}
\begin{sphinxVerbatim}[commandchars=\\\{\}]
Good morning everyone,
I hope you are all doing well. It\PYGZsq{}s a great pleasure for me to weilcome all of you to today\PYGZsq{}s meeting. 
We have a very busy schedule today. Let\PYGZsq{}s start working on each subject one by one. Thank you for being here!
\end{sphinxVerbatim}

\end{sphinxuseclass}\end{sphinxVerbatimOutput}

\end{sphinxuseclass}

\subsection{Default parameter values}
\label{\detokenize{functions:default-parameter-values}}
\sphinxAtStartPar
If no value is given to the parameter, the default value will be used.
\begin{itemize}
\item {} 
\sphinxAtStartPar
Default values are typically assigned to parameters that are not frequently used or have a common default value.

\item {} 
\sphinxAtStartPar
Non\sphinxhyphen{}default parameters should precede default parameters.

\end{itemize}

\begin{sphinxuseclass}{cell}\begin{sphinxVerbatimInput}

\begin{sphinxuseclass}{cell_input}
\begin{sphinxVerbatim}[commandchars=\\\{\}]
\PYG{c+c1}{\PYGZsh{} course parameter has a default value}

\PYG{k}{def} \PYG{n+nf}{student\PYGZus{}report}\PYG{p}{(}\PYG{n}{name}\PYG{p}{,} \PYG{n}{grade}\PYG{p}{,} \PYG{n}{course}\PYG{o}{=}\PYG{l+s+s1}{\PYGZsq{}}\PYG{l+s+s1}{Math}\PYG{l+s+s1}{\PYGZsq{}}\PYG{p}{)}\PYG{p}{:}
  \PYG{n+nb}{print}\PYG{p}{(}\PYG{l+s+sa}{f}\PYG{l+s+s1}{\PYGZsq{}}\PYG{l+s+si}{\PYGZob{}}\PYG{n}{course}\PYG{l+s+si}{\PYGZcb{}}\PYG{l+s+s1}{ grade of }\PYG{l+s+si}{\PYGZob{}}\PYG{n}{name}\PYG{l+s+si}{\PYGZcb{}}\PYG{l+s+s1}{ is }\PYG{l+s+si}{\PYGZob{}}\PYG{n}{grade}\PYG{l+s+si}{\PYGZcb{}}\PYG{l+s+s1}{.}\PYG{l+s+s1}{\PYGZsq{}}\PYG{p}{)}
\end{sphinxVerbatim}

\end{sphinxuseclass}\end{sphinxVerbatimInput}

\end{sphinxuseclass}
\begin{sphinxuseclass}{cell}\begin{sphinxVerbatimInput}

\begin{sphinxuseclass}{cell_input}
\begin{sphinxVerbatim}[commandchars=\\\{\}]
\PYG{c+c1}{\PYGZsh{} call student\PYGZus{}report of Michael for CS }

\PYG{n}{student\PYGZus{}report}\PYG{p}{(}\PYG{l+s+s1}{\PYGZsq{}}\PYG{l+s+s1}{Michael}\PYG{l+s+s1}{\PYGZsq{}}\PYG{p}{,} \PYG{l+m+mi}{87}\PYG{p}{,} \PYG{l+s+s1}{\PYGZsq{}}\PYG{l+s+s1}{CS}\PYG{l+s+s1}{\PYGZsq{}}\PYG{p}{)}
\end{sphinxVerbatim}

\end{sphinxuseclass}\end{sphinxVerbatimInput}
\begin{sphinxVerbatimOutput}

\begin{sphinxuseclass}{cell_output}
\begin{sphinxVerbatim}[commandchars=\\\{\}]
CS grade of Michael is 87.
\end{sphinxVerbatim}

\end{sphinxuseclass}\end{sphinxVerbatimOutput}

\end{sphinxuseclass}
\begin{sphinxuseclass}{cell}\begin{sphinxVerbatimInput}

\begin{sphinxuseclass}{cell_input}
\begin{sphinxVerbatim}[commandchars=\\\{\}]
\PYG{c+c1}{\PYGZsh{} course value is not given so default value=\PYGZsq{}Math\PYGZsq{} is used}

\PYG{n}{student\PYGZus{}report}\PYG{p}{(}\PYG{l+s+s1}{\PYGZsq{}}\PYG{l+s+s1}{Michael}\PYG{l+s+s1}{\PYGZsq{}}\PYG{p}{,} \PYG{l+m+mi}{87}\PYG{p}{)}
\end{sphinxVerbatim}

\end{sphinxuseclass}\end{sphinxVerbatimInput}
\begin{sphinxVerbatimOutput}

\begin{sphinxuseclass}{cell_output}
\begin{sphinxVerbatim}[commandchars=\\\{\}]
Math grade of Michael is 87.
\end{sphinxVerbatim}

\end{sphinxuseclass}\end{sphinxVerbatimOutput}

\end{sphinxuseclass}
\begin{sphinxVerbatim}[commandchars=\\\{\}]
\PYG{c+c1}{\PYGZsh{} ERROR: Default parameter course cannot come before non\PYGZhy{}default parameter grade}

\PYG{k}{def} \PYG{n+nf}{student\PYGZus{}report}\PYG{p}{(}\PYG{n}{name}\PYG{p}{,} \PYG{n}{course}\PYG{o}{=}\PYG{l+s+s1}{\PYGZsq{}}\PYG{l+s+s1}{Math}\PYG{l+s+s1}{\PYGZsq{}}\PYG{p}{,} \PYG{n}{grade}\PYG{p}{)}\PYG{p}{:}
  \PYG{n+nb}{print}\PYG{p}{(}\PYG{l+s+sa}{f}\PYG{l+s+s1}{\PYGZsq{}}\PYG{l+s+si}{\PYGZob{}}\PYG{n}{course}\PYG{l+s+si}{\PYGZcb{}}\PYG{l+s+s1}{ grade of }\PYG{l+s+si}{\PYGZob{}}\PYG{n}{name}\PYG{l+s+si}{\PYGZcb{}}\PYG{l+s+s1}{ is }\PYG{l+s+si}{\PYGZob{}}\PYG{n}{grade}\PYG{l+s+si}{\PYGZcb{}}\PYG{l+s+s1}{.}\PYG{l+s+s1}{\PYGZsq{}}\PYG{p}{)}
\end{sphinxVerbatim}


\subsection{Local and Global Variables}
\label{\detokenize{functions:local-and-global-variables}}\begin{itemize}
\item {} 
\sphinxAtStartPar
\sphinxstyleemphasis{Local variables} are defined inside a function and can only be accessed within that function.

\item {} 
\sphinxAtStartPar
\sphinxstyleemphasis{Global variable}s are defined outside a function and can be accessed inside a function, but they cannot be modified within the function.
\begin{itemize}
\item {} 
\sphinxAtStartPar
When a global variable is called, a new local variable is used.

\end{itemize}

\end{itemize}

\begin{sphinxuseclass}{cell}\begin{sphinxVerbatimInput}

\begin{sphinxuseclass}{cell_input}
\begin{sphinxVerbatim}[commandchars=\\\{\}]
\PYG{n}{a} \PYG{o}{=} \PYG{l+m+mi}{4}     \PYG{c+c1}{\PYGZsh{} global variable}

\PYG{k}{def} \PYG{n+nf}{f}\PYG{p}{(}\PYG{n}{x}\PYG{p}{)}\PYG{p}{:}
    \PYG{n}{b} \PYG{o}{=} \PYG{l+m+mi}{5}   \PYG{c+c1}{\PYGZsh{} local variable}
    \PYG{n}{result} \PYG{o}{=} \PYG{n}{a}\PYG{o}{+}\PYG{n}{b}\PYG{o}{+}\PYG{n}{x}
    \PYG{k}{return} \PYG{n}{result}
\end{sphinxVerbatim}

\end{sphinxuseclass}\end{sphinxVerbatimInput}

\end{sphinxuseclass}
\begin{sphinxuseclass}{cell}\begin{sphinxVerbatimInput}

\begin{sphinxuseclass}{cell_input}
\begin{sphinxVerbatim}[commandchars=\\\{\}]
\PYG{c+c1}{\PYGZsh{} call function for x=10}
\PYG{n+nb}{print}\PYG{p}{(}\PYG{n}{f}\PYG{p}{(}\PYG{l+m+mi}{10}\PYG{p}{)}\PYG{p}{)}
\end{sphinxVerbatim}

\end{sphinxuseclass}\end{sphinxVerbatimInput}
\begin{sphinxVerbatimOutput}

\begin{sphinxuseclass}{cell_output}
\begin{sphinxVerbatim}[commandchars=\\\{\}]
19
\end{sphinxVerbatim}

\end{sphinxuseclass}\end{sphinxVerbatimOutput}

\end{sphinxuseclass}
\begin{sphinxuseclass}{cell}\begin{sphinxVerbatimInput}

\begin{sphinxuseclass}{cell_input}
\begin{sphinxVerbatim}[commandchars=\\\{\}]
\PYG{c+c1}{\PYGZsh{} you can acces \PYGZsq{}a\PYGZsq{} outside the function: global variable}
\PYG{n+nb}{print}\PYG{p}{(}\PYG{n}{a}\PYG{p}{)}
\end{sphinxVerbatim}

\end{sphinxuseclass}\end{sphinxVerbatimInput}
\begin{sphinxVerbatimOutput}

\begin{sphinxuseclass}{cell_output}
\begin{sphinxVerbatim}[commandchars=\\\{\}]
4
\end{sphinxVerbatim}

\end{sphinxuseclass}\end{sphinxVerbatimOutput}

\end{sphinxuseclass}
\begin{sphinxVerbatim}[commandchars=\\\{\}]
\PYG{c+c1}{\PYGZsh{} ERROR: \PYGZsq{}b\PYGZsq{} is not defined outside the function: local variable}
\PYG{n+nb}{print}\PYG{p}{(}\PYG{n}{b}\PYG{p}{)}
\end{sphinxVerbatim}

\begin{sphinxVerbatim}[commandchars=\\\{\}]
\PYG{c+c1}{\PYGZsh{} ERROR: \PYGZsq{}result\PYGZsq{} is not defined outside the function: local variable}
\PYG{n+nb}{print}\PYG{p}{(}\PYG{n}{result}\PYG{p}{)}
\end{sphinxVerbatim}

\begin{sphinxuseclass}{cell}\begin{sphinxVerbatimInput}

\begin{sphinxuseclass}{cell_input}
\begin{sphinxVerbatim}[commandchars=\\\{\}]
\PYG{n}{a} \PYG{o}{=} \PYG{l+m+mi}{4}     \PYG{c+c1}{\PYGZsh{} global variable}

\PYG{k}{def} \PYG{n+nf}{f}\PYG{p}{(}\PYG{n}{x}\PYG{p}{)}\PYG{p}{:}
    \PYG{n}{a} \PYG{o}{=} \PYG{l+m+mi}{100}       \PYG{c+c1}{\PYGZsh{} change the value of a}
    \PYG{n}{b} \PYG{o}{=} \PYG{l+m+mi}{5}   
    \PYG{n}{result} \PYG{o}{=} \PYG{n}{a}\PYG{o}{+}\PYG{n}{b}\PYG{o}{+}\PYG{n}{x}
    \PYG{k}{return} \PYG{n}{result}
\end{sphinxVerbatim}

\end{sphinxuseclass}\end{sphinxVerbatimInput}

\end{sphinxuseclass}
\begin{sphinxuseclass}{cell}\begin{sphinxVerbatimInput}

\begin{sphinxuseclass}{cell_input}
\begin{sphinxVerbatim}[commandchars=\\\{\}]
\PYG{c+c1}{\PYGZsh{} call function for x=10}
\PYG{n+nb}{print}\PYG{p}{(}\PYG{n}{f}\PYG{p}{(}\PYG{l+m+mi}{10}\PYG{p}{)}\PYG{p}{)}   \PYG{c+c1}{\PYGZsh{} a=100 is used }
\end{sphinxVerbatim}

\end{sphinxuseclass}\end{sphinxVerbatimInput}
\begin{sphinxVerbatimOutput}

\begin{sphinxuseclass}{cell_output}
\begin{sphinxVerbatim}[commandchars=\\\{\}]
115
\end{sphinxVerbatim}

\end{sphinxuseclass}\end{sphinxVerbatimOutput}

\end{sphinxuseclass}
\begin{sphinxuseclass}{cell}\begin{sphinxVerbatimInput}

\begin{sphinxuseclass}{cell_input}
\begin{sphinxVerbatim}[commandchars=\\\{\}]
\PYG{c+c1}{\PYGZsh{} The value of \PYGZsq{}a\PYGZsq{} outside the function has not been changed}
\PYG{n+nb}{print}\PYG{p}{(}\PYG{n}{a}\PYG{p}{)}
\end{sphinxVerbatim}

\end{sphinxuseclass}\end{sphinxVerbatimInput}
\begin{sphinxVerbatimOutput}

\begin{sphinxuseclass}{cell_output}
\begin{sphinxVerbatim}[commandchars=\\\{\}]
4
\end{sphinxVerbatim}

\end{sphinxuseclass}\end{sphinxVerbatimOutput}

\end{sphinxuseclass}

\subsection{Examples}
\label{\detokenize{functions:examples}}

\subsubsection{Calculator}
\label{\detokenize{functions:calculator}}
\sphinxAtStartPar
Write a function whose parameters are two numbers and a string (operation)
\begin{itemize}
\item {} 
\sphinxAtStartPar
Operations are given as strings in the form of \sphinxcode{\sphinxupquote{'+', '*', '/', '\sphinxhyphen{}'}}

\item {} 
\sphinxAtStartPar
By using the given numbers and operation find \sphinxcode{\sphinxupquote{number1 operation number2}}

\item {} 
\sphinxAtStartPar
If the operation is division second parameter (denominator) can not be zero and display a warning message.

\item {} 
\sphinxAtStartPar
If the operation is not one of the four operations given above display a warning mesage.

\end{itemize}

\begin{sphinxuseclass}{cell}\begin{sphinxVerbatimInput}

\begin{sphinxuseclass}{cell_input}
\begin{sphinxVerbatim}[commandchars=\\\{\}]
\PYG{k}{def} \PYG{n+nf}{calculator}\PYG{p}{(}\PYG{n}{number1}\PYG{p}{,} \PYG{n}{number2}\PYG{p}{,} \PYG{n}{operation}\PYG{p}{)}\PYG{p}{:}
    
  \PYG{k}{if} \PYG{n}{operation} \PYG{o}{==} \PYG{l+s+s1}{\PYGZsq{}}\PYG{l+s+s1}{+}\PYG{l+s+s1}{\PYGZsq{}}\PYG{p}{:}
    \PYG{k}{return} \PYG{n}{number1} \PYG{o}{+} \PYG{n}{number2}
  \PYG{k}{elif} \PYG{n}{operation} \PYG{o}{==} \PYG{l+s+s1}{\PYGZsq{}}\PYG{l+s+s1}{\PYGZhy{}}\PYG{l+s+s1}{\PYGZsq{}}\PYG{p}{:}
    \PYG{k}{return} \PYG{n}{number1} \PYG{o}{\PYGZhy{}} \PYG{n}{number2}
  \PYG{k}{elif} \PYG{n}{operation} \PYG{o}{==} \PYG{l+s+s1}{\PYGZsq{}}\PYG{l+s+s1}{*}\PYG{l+s+s1}{\PYGZsq{}}\PYG{p}{:}
    \PYG{k}{return} \PYG{n}{number1} \PYG{o}{*} \PYG{n}{number2}
  \PYG{k}{elif} \PYG{n}{operation} \PYG{o}{==} \PYG{l+s+s1}{\PYGZsq{}}\PYG{l+s+s1}{/}\PYG{l+s+s1}{\PYGZsq{}}\PYG{p}{:}
    \PYG{k}{if} \PYG{n}{number2} \PYG{o}{==} \PYG{l+m+mi}{0}\PYG{p}{:}
      \PYG{n+nb}{print}\PYG{p}{(}\PYG{l+s+s1}{\PYGZsq{}}\PYG{l+s+s1}{Warning: zero division}\PYG{l+s+s1}{\PYGZsq{}}\PYG{p}{)}
    \PYG{k}{else}\PYG{p}{:}
      \PYG{k}{return} \PYG{n}{number1}\PYG{o}{/}\PYG{n}{number2}
  \PYG{k}{else}\PYG{p}{:}
    \PYG{n+nb}{print}\PYG{p}{(}\PYG{l+s+sa}{f}\PYG{l+s+s1}{\PYGZsq{}}\PYG{l+s+si}{\PYGZob{}}\PYG{n}{operation}\PYG{l+s+si}{\PYGZcb{}}\PYG{l+s+s1}{ is not an available operation.}\PYG{l+s+s1}{\PYGZsq{}}\PYG{p}{)}
\end{sphinxVerbatim}

\end{sphinxuseclass}\end{sphinxVerbatimInput}

\end{sphinxuseclass}
\begin{sphinxuseclass}{cell}\begin{sphinxVerbatimInput}

\begin{sphinxuseclass}{cell_input}
\begin{sphinxVerbatim}[commandchars=\\\{\}]
\PYG{c+c1}{\PYGZsh{} addition}
\PYG{n+nb}{print}\PYG{p}{(}\PYG{n}{calculator}\PYG{p}{(}\PYG{l+m+mi}{7}\PYG{p}{,}\PYG{l+m+mi}{2}\PYG{p}{,}\PYG{l+s+s1}{\PYGZsq{}}\PYG{l+s+s1}{+}\PYG{l+s+s1}{\PYGZsq{}}\PYG{p}{)}\PYG{p}{)}
\end{sphinxVerbatim}

\end{sphinxuseclass}\end{sphinxVerbatimInput}
\begin{sphinxVerbatimOutput}

\begin{sphinxuseclass}{cell_output}
\begin{sphinxVerbatim}[commandchars=\\\{\}]
9
\end{sphinxVerbatim}

\end{sphinxuseclass}\end{sphinxVerbatimOutput}

\end{sphinxuseclass}
\begin{sphinxuseclass}{cell}\begin{sphinxVerbatimInput}

\begin{sphinxuseclass}{cell_input}
\begin{sphinxVerbatim}[commandchars=\\\{\}]
\PYG{c+c1}{\PYGZsh{} subtraction}
\PYG{n+nb}{print}\PYG{p}{(}\PYG{n}{calculator}\PYG{p}{(}\PYG{l+m+mi}{7}\PYG{p}{,}\PYG{l+m+mi}{2}\PYG{p}{,}\PYG{l+s+s1}{\PYGZsq{}}\PYG{l+s+s1}{\PYGZhy{}}\PYG{l+s+s1}{\PYGZsq{}}\PYG{p}{)}\PYG{p}{)}
\end{sphinxVerbatim}

\end{sphinxuseclass}\end{sphinxVerbatimInput}
\begin{sphinxVerbatimOutput}

\begin{sphinxuseclass}{cell_output}
\begin{sphinxVerbatim}[commandchars=\\\{\}]
5
\end{sphinxVerbatim}

\end{sphinxuseclass}\end{sphinxVerbatimOutput}

\end{sphinxuseclass}
\begin{sphinxuseclass}{cell}\begin{sphinxVerbatimInput}

\begin{sphinxuseclass}{cell_input}
\begin{sphinxVerbatim}[commandchars=\\\{\}]
\PYG{c+c1}{\PYGZsh{} multiplication}
\PYG{n+nb}{print}\PYG{p}{(}\PYG{n}{calculator}\PYG{p}{(}\PYG{l+m+mi}{7}\PYG{p}{,}\PYG{l+m+mi}{2}\PYG{p}{,}\PYG{l+s+s1}{\PYGZsq{}}\PYG{l+s+s1}{*}\PYG{l+s+s1}{\PYGZsq{}}\PYG{p}{)}\PYG{p}{)}
\end{sphinxVerbatim}

\end{sphinxuseclass}\end{sphinxVerbatimInput}
\begin{sphinxVerbatimOutput}

\begin{sphinxuseclass}{cell_output}
\begin{sphinxVerbatim}[commandchars=\\\{\}]
14
\end{sphinxVerbatim}

\end{sphinxuseclass}\end{sphinxVerbatimOutput}

\end{sphinxuseclass}
\begin{sphinxuseclass}{cell}\begin{sphinxVerbatimInput}

\begin{sphinxuseclass}{cell_input}
\begin{sphinxVerbatim}[commandchars=\\\{\}]
\PYG{c+c1}{\PYGZsh{} division}
\PYG{n+nb}{print}\PYG{p}{(}\PYG{n}{calculator}\PYG{p}{(}\PYG{l+m+mi}{7}\PYG{p}{,}\PYG{l+m+mi}{2}\PYG{p}{,}\PYG{l+s+s1}{\PYGZsq{}}\PYG{l+s+s1}{/}\PYG{l+s+s1}{\PYGZsq{}}\PYG{p}{)}\PYG{p}{)}
\end{sphinxVerbatim}

\end{sphinxuseclass}\end{sphinxVerbatimInput}
\begin{sphinxVerbatimOutput}

\begin{sphinxuseclass}{cell_output}
\begin{sphinxVerbatim}[commandchars=\\\{\}]
3.5
\end{sphinxVerbatim}

\end{sphinxuseclass}\end{sphinxVerbatimOutput}

\end{sphinxuseclass}
\begin{sphinxuseclass}{cell}\begin{sphinxVerbatimInput}

\begin{sphinxuseclass}{cell_input}
\begin{sphinxVerbatim}[commandchars=\\\{\}]
\PYG{c+c1}{\PYGZsh{} division by zero}
\PYG{n}{calculator}\PYG{p}{(}\PYG{l+m+mi}{7}\PYG{p}{,}\PYG{l+m+mi}{0}\PYG{p}{,}\PYG{l+s+s1}{\PYGZsq{}}\PYG{l+s+s1}{/}\PYG{l+s+s1}{\PYGZsq{}}\PYG{p}{)}
\end{sphinxVerbatim}

\end{sphinxuseclass}\end{sphinxVerbatimInput}
\begin{sphinxVerbatimOutput}

\begin{sphinxuseclass}{cell_output}
\begin{sphinxVerbatim}[commandchars=\\\{\}]
Warning: zero division
\end{sphinxVerbatim}

\end{sphinxuseclass}\end{sphinxVerbatimOutput}

\end{sphinxuseclass}
\begin{sphinxuseclass}{cell}\begin{sphinxVerbatimInput}

\begin{sphinxuseclass}{cell_input}
\begin{sphinxVerbatim}[commandchars=\\\{\}]
\PYG{c+c1}{\PYGZsh{} inappropriate operation}
\PYG{n}{calculator}\PYG{p}{(}\PYG{l+m+mi}{7}\PYG{p}{,}\PYG{l+m+mi}{3}\PYG{p}{,}\PYG{l+s+s1}{\PYGZsq{}}\PYG{l+s+s1}{\PYGZpc{}}\PYG{l+s+s1}{\PYGZsq{}}\PYG{p}{)}
\end{sphinxVerbatim}

\end{sphinxuseclass}\end{sphinxVerbatimInput}
\begin{sphinxVerbatimOutput}

\begin{sphinxuseclass}{cell_output}
\begin{sphinxVerbatim}[commandchars=\\\{\}]
\PYGZpc{} is not an available operation.
\end{sphinxVerbatim}

\end{sphinxuseclass}\end{sphinxVerbatimOutput}

\end{sphinxuseclass}
\sphinxstepscope


\section{Functions Debugging}
\label{\detokenize{functions_debug:functions-debugging}}\label{\detokenize{functions_debug::doc}}\begin{itemize}
\item {} 
\sphinxAtStartPar
Each of the following short code contains one or more bugs.     

\item {} 
\sphinxAtStartPar
Please identify and correct these bugs.

\item {} 
\sphinxAtStartPar
Provide an explanation for your answer.

\end{itemize}


\subsection{Question}
\label{\detokenize{functions_debug:question}}
\begin{sphinxVerbatim}[commandchars=\\\{\}]
\PYG{k}{def} \PYG{n+nf}{f}\PYG{p}{(}\PYG{n}{x}\PYG{p}{)}
  \PYG{k}{return} \PYG{n}{x}\PYG{o}{*}\PYG{o}{*}\PYG{l+m+mi}{2}
\end{sphinxVerbatim}

\begin{sphinxadmonition}{note}{Solution}

\sphinxAtStartPar
The missing colon in the first line should be added.
\end{sphinxadmonition}


\subsection{Question}
\label{\detokenize{functions_debug:id1}}
\begin{sphinxVerbatim}[commandchars=\\\{\}]
\PYG{k}{def} \PYG{p}{(}\PYG{n}{x}\PYG{p}{,}\PYG{n}{y}\PYG{p}{,}\PYG{n}{z}\PYG{p}{)}\PYG{p}{:}
  \PYG{k}{return} \PYG{n}{x}\PYG{o}{+}\PYG{n}{y}\PYG{o}{+}\PYG{n}{z}
\end{sphinxVerbatim}

\begin{sphinxadmonition}{note}{Solution}

\sphinxAtStartPar
The name of the function is missing.
\end{sphinxadmonition}


\subsection{Question}
\label{\detokenize{functions_debug:id2}}
\begin{sphinxVerbatim}[commandchars=\\\{\}]
\PYG{k}{def} \PYG{n+nf}{f}\PYG{p}{(}\PYG{n}{x}\PYG{p}{,}\PYG{n}{y}\PYG{p}{)}\PYG{p}{:}
  \PYG{k}{return} \PYG{n}{x}\PYG{o}{+}\PYG{n}{y}

\PYG{n+nb}{print}\PYG{p}{(}\PYG{n}{f}\PYG{p}{(}\PYG{l+m+mi}{1}\PYG{p}{,}\PYG{l+m+mi}{2}\PYG{p}{,}\PYG{l+m+mi}{3}\PYG{p}{)}\PYG{p}{)}
\end{sphinxVerbatim}

\begin{sphinxadmonition}{note}{Solution}

\sphinxAtStartPar
The function \sphinxstyleemphasis{f} has two parameters, but in the last line, three parameters are given.
\end{sphinxadmonition}


\subsection{Question}
\label{\detokenize{functions_debug:id3}}
\begin{sphinxVerbatim}[commandchars=\\\{\}]
\PYG{k}{def} \PYG{n+nf}{func}\PYG{p}{(}\PYG{n}{a}\PYG{p}{,}\PYG{n}{b}\PYG{p}{,}\PYG{n}{c}\PYG{p}{)}\PYG{p}{:}
  \PYG{n}{x} \PYG{o}{=} \PYG{n}{a}\PYG{o}{*}\PYG{o}{*}\PYG{l+m+mi}{2}\PYG{o}{+}\PYG{n}{b}\PYG{o}{*}\PYG{l+m+mi}{4}\PYG{o}{\PYGZhy{}}\PYG{n}{c}
  \PYG{k}{return} \PYG{n}{x}

\PYG{n}{func}\PYG{p}{(}\PYG{l+m+mi}{1}\PYG{p}{,}\PYG{l+m+mi}{5}\PYG{p}{)}
\end{sphinxVerbatim}

\begin{sphinxadmonition}{note}{Solution}

\sphinxAtStartPar
The function \sphinxcode{\sphinxupquote{func}} has three parameters (a, b, c). In the last line, only two inputs are provided for the func() function.
\end{sphinxadmonition}


\subsection{Question}
\label{\detokenize{functions_debug:id4}}
\begin{sphinxVerbatim}[commandchars=\\\{\}]
\PYG{k}{def} \PYG{n+nf}{my\PYGZus{}func}\PYG{p}{(}\PYG{n}{x}\PYG{p}{)}\PYG{p}{:}
  \PYG{k}{return} \PYG{n}{x}\PYG{o}{+}\PYG{l+m+mi}{6}

\PYG{n}{my\PYGZus{}func}\PYG{p}{(}\PYG{l+s+s1}{\PYGZsq{}}\PYG{l+s+s1}{3}\PYG{l+s+s1}{\PYGZsq{}}\PYG{p}{)}
\end{sphinxVerbatim}

\begin{sphinxadmonition}{note}{Solution}

\sphinxAtStartPar
The value ‘3’ is a string, and inside the function, an attempt is made to add 6 to the string ‘3’, resulting in an error.
\end{sphinxadmonition}


\subsection{Question}
\label{\detokenize{functions_debug:id5}}
\begin{sphinxVerbatim}[commandchars=\\\{\}]
\PYG{k}{def} \PYG{n+nf}{my\PYGZus{}func}\PYG{p}{(}\PYG{n}{x}\PYG{p}{)}\PYG{p}{:}
  \PYG{k}{return} \PYG{n+nb}{str}\PYG{p}{(}\PYG{n}{x}\PYG{p}{)}\PYG{o}{+}\PYG{l+m+mi}{6}

\PYG{n}{my\PYGZus{}func}\PYG{p}{(}\PYG{l+m+mi}{3}\PYG{p}{)}
\end{sphinxVerbatim}

\begin{sphinxadmonition}{note}{Solution}

\sphinxAtStartPar
str(x) represents a string, while 6 is an integer. They cannot be added or concatenated directly.
\end{sphinxadmonition}


\subsection{Question}
\label{\detokenize{functions_debug:id6}}
\begin{sphinxVerbatim}[commandchars=\\\{\}]
\PYG{k}{def} \PYG{n+nf}{f}\PYG{p}{(}\PYG{n}{x}\PYG{p}{)}\PYG{p}{:}
  \PYG{n}{y} \PYG{o}{=} \PYG{l+m+mi}{5}
  \PYG{k}{return} \PYG{n}{x}\PYG{o}{+}\PYG{n}{y}

\PYG{n+nb}{print}\PYG{p}{(}\PYG{n}{f}\PYG{p}{(}\PYG{l+m+mi}{10}\PYG{p}{)}\PYG{p}{)}
\PYG{n+nb}{print}\PYG{p}{(}\PYG{n}{y}\PYG{p}{)}
\end{sphinxVerbatim}

\begin{sphinxadmonition}{note}{Solution}

\sphinxAtStartPar
\sphinxcode{\sphinxupquote{y}} is a local variable, and it cannot be accessed outside the function, as attempted in the last line.
\end{sphinxadmonition}


\subsection{Question}
\label{\detokenize{functions_debug:id7}}
\begin{sphinxVerbatim}[commandchars=\\\{\}]
\PYG{k}{def} \PYG{n+nf}{f}\PYG{p}{(}\PYG{n}{y}\PYG{o}{=}\PYG{l+m+mi}{0}\PYG{p}{,} \PYG{n}{x}\PYG{p}{)}\PYG{p}{:}
  \PYG{k}{return} \PYG{n}{x}\PYG{o}{+}\PYG{n}{y}
\end{sphinxVerbatim}

\begin{sphinxadmonition}{note}{Solution}

\sphinxAtStartPar
The non\sphinxhyphen{}default parameters must be placed before default parameters. The correct form is f(x, y=0).
\end{sphinxadmonition}

\sphinxstepscope


\section{Functions Output}
\label{\detokenize{functions_output:functions-output}}\label{\detokenize{functions_output::doc}}\begin{itemize}
\item {} 
\sphinxAtStartPar
Find the output of the following code.

\item {} 
\sphinxAtStartPar
Please don’t run the code before giving your answer.     

\end{itemize}


\subsection{Question}
\label{\detokenize{functions_output:question}}
\begin{sphinxuseclass}{cell}
\begin{sphinxuseclass}{tag_hide-output}\begin{sphinxVerbatimInput}

\begin{sphinxuseclass}{cell_input}
\begin{sphinxVerbatim}[commandchars=\\\{\}]
\PYG{k}{def} \PYG{n+nf}{f}\PYG{p}{(}\PYG{n}{x}\PYG{p}{)}\PYG{p}{:}
  \PYG{k}{return} \PYG{l+m+mi}{10}\PYG{o}{*}\PYG{n}{x}
\end{sphinxVerbatim}

\end{sphinxuseclass}\end{sphinxVerbatimInput}

\end{sphinxuseclass}
\end{sphinxuseclass}
\begin{sphinxadmonition}{note}{Solution}

\sphinxAtStartPar
No output.
\end{sphinxadmonition}


\subsection{Question}
\label{\detokenize{functions_output:id1}}
\begin{sphinxuseclass}{cell}
\begin{sphinxuseclass}{tag_hide-output}\begin{sphinxVerbatimInput}

\begin{sphinxuseclass}{cell_input}
\begin{sphinxVerbatim}[commandchars=\\\{\}]
\PYG{k}{def} \PYG{n+nf}{f}\PYG{p}{(}\PYG{n}{x}\PYG{p}{)}\PYG{p}{:}
  \PYG{k}{return} \PYG{l+m+mi}{10}\PYG{o}{*}\PYG{n}{x}

\PYG{n+nb}{print}\PYG{p}{(}\PYG{n}{f}\PYG{p}{(}\PYG{l+m+mi}{5}\PYG{p}{)}\PYG{p}{)}
\end{sphinxVerbatim}

\end{sphinxuseclass}\end{sphinxVerbatimInput}

\end{sphinxuseclass}
\end{sphinxuseclass}

\subsection{Question}
\label{\detokenize{functions_output:id2}}
\begin{sphinxuseclass}{cell}
\begin{sphinxuseclass}{tag_hide-output}\begin{sphinxVerbatimInput}

\begin{sphinxuseclass}{cell_input}
\begin{sphinxVerbatim}[commandchars=\\\{\}]
\PYG{k}{def} \PYG{n+nf}{f}\PYG{p}{(}\PYG{n}{x}\PYG{p}{,} \PYG{n}{y}\PYG{o}{=}\PYG{l+m+mi}{5}\PYG{p}{)}\PYG{p}{:}
  \PYG{k}{return} \PYG{n}{x}\PYG{o}{+}\PYG{n}{y}

\PYG{n+nb}{print}\PYG{p}{(}\PYG{n}{f}\PYG{p}{(}\PYG{l+m+mi}{2}\PYG{p}{,} \PYG{l+m+mi}{4}\PYG{p}{)}\PYG{p}{)}
\PYG{n+nb}{print}\PYG{p}{(}\PYG{n}{f}\PYG{p}{(}\PYG{l+m+mi}{10}\PYG{p}{)}\PYG{p}{)}
\end{sphinxVerbatim}

\end{sphinxuseclass}\end{sphinxVerbatimInput}

\end{sphinxuseclass}
\end{sphinxuseclass}

\subsection{Question}
\label{\detokenize{functions_output:id3}}
\begin{sphinxuseclass}{cell}
\begin{sphinxuseclass}{tag_hide-output}\begin{sphinxVerbatimInput}

\begin{sphinxuseclass}{cell_input}
\begin{sphinxVerbatim}[commandchars=\\\{\}]
\PYG{k}{def} \PYG{n+nf}{func\PYGZus{}letter}\PYG{p}{(}\PYG{p}{)}\PYG{p}{:}
  \PYG{n+nb}{print}\PYG{p}{(}\PYG{l+s+s1}{\PYGZsq{}}\PYG{l+s+s1}{A}\PYG{l+s+s1}{\PYGZsq{}}\PYG{p}{)}
  \PYG{n+nb}{print}\PYG{p}{(}\PYG{l+s+s1}{\PYGZsq{}}\PYG{l+s+s1}{B}\PYG{l+s+s1}{\PYGZsq{}}\PYG{p}{)}

\PYG{k}{def} \PYG{n+nf}{func\PYGZus{}word}\PYG{p}{(}\PYG{p}{)}\PYG{p}{:}
  \PYG{n+nb}{print}\PYG{p}{(}\PYG{l+s+s1}{\PYGZsq{}}\PYG{l+s+s1}{X}\PYG{l+s+s1}{\PYGZsq{}}\PYG{p}{)}
  \PYG{n}{func\PYGZus{}letter}\PYG{p}{(}\PYG{p}{)}

\PYG{n}{func\PYGZus{}word}\PYG{p}{(}\PYG{p}{)}
\end{sphinxVerbatim}

\end{sphinxuseclass}\end{sphinxVerbatimInput}

\end{sphinxuseclass}
\end{sphinxuseclass}

\subsection{Question}
\label{\detokenize{functions_output:id4}}
\begin{sphinxuseclass}{cell}
\begin{sphinxuseclass}{tag_hide-output}\begin{sphinxVerbatimInput}

\begin{sphinxuseclass}{cell_input}
\begin{sphinxVerbatim}[commandchars=\\\{\}]
\PYG{k}{def} \PYG{n+nf}{my\PYGZus{}func}\PYG{p}{(}\PYG{n}{x}\PYG{p}{)}\PYG{p}{:}
  \PYG{n+nb}{print}\PYG{p}{(}\PYG{l+s+s1}{\PYGZsq{}}\PYG{l+s+s1}{A}\PYG{l+s+s1}{\PYGZsq{}}\PYG{p}{)}
  \PYG{n+nb}{print}\PYG{p}{(}\PYG{l+s+s1}{\PYGZsq{}}\PYG{l+s+s1}{B}\PYG{l+s+s1}{\PYGZsq{}}\PYG{p}{)}
  \PYG{k}{return} \PYG{n}{x}\PYG{o}{*}\PYG{o}{*}\PYG{l+m+mi}{2}

\PYG{n+nb}{print}\PYG{p}{(}\PYG{n}{my\PYGZus{}func}\PYG{p}{(}\PYG{l+m+mi}{3}\PYG{p}{)}\PYG{p}{)}
\end{sphinxVerbatim}

\end{sphinxuseclass}\end{sphinxVerbatimInput}

\end{sphinxuseclass}
\end{sphinxuseclass}

\subsection{Question}
\label{\detokenize{functions_output:id5}}
\begin{sphinxuseclass}{cell}
\begin{sphinxuseclass}{tag_hide-output}\begin{sphinxVerbatimInput}

\begin{sphinxuseclass}{cell_input}
\begin{sphinxVerbatim}[commandchars=\\\{\}]
\PYG{k}{def} \PYG{n+nf}{func}\PYG{p}{(}\PYG{n}{a}\PYG{p}{,}\PYG{n}{b}\PYG{p}{,}\PYG{n}{c}\PYG{o}{=}\PYG{l+m+mi}{3}\PYG{p}{)}\PYG{p}{:}
  \PYG{n}{x} \PYG{o}{=} \PYG{n}{a}\PYG{o}{*}\PYG{o}{*}\PYG{l+m+mi}{2}\PYG{o}{+}\PYG{n}{b}\PYG{o}{*}\PYG{l+m+mi}{4}\PYG{o}{\PYGZhy{}}\PYG{n}{c}
  \PYG{k}{return} \PYG{n}{x}

\PYG{n}{func}\PYG{p}{(}\PYG{l+m+mi}{1}\PYG{p}{,}\PYG{l+m+mi}{5}\PYG{p}{)}
\end{sphinxVerbatim}

\end{sphinxuseclass}\end{sphinxVerbatimInput}

\end{sphinxuseclass}
\end{sphinxuseclass}

\subsection{Question}
\label{\detokenize{functions_output:id6}}
\begin{sphinxuseclass}{cell}
\begin{sphinxuseclass}{tag_hide-output}\begin{sphinxVerbatimInput}

\begin{sphinxuseclass}{cell_input}
\begin{sphinxVerbatim}[commandchars=\\\{\}]
\PYG{k}{def} \PYG{n+nf}{my\PYGZus{}func}\PYG{p}{(}\PYG{n}{x}\PYG{p}{,} \PYG{n}{y}\PYG{p}{)}\PYG{p}{:}
  \PYG{n+nb}{print}\PYG{p}{(}\PYG{n}{x}\PYG{o}{/}\PYG{o}{/}\PYG{l+m+mi}{3}\PYG{p}{)}
  \PYG{n}{y} \PYG{o}{+}\PYG{o}{=} \PYG{l+m+mi}{2}
  \PYG{n+nb}{print}\PYG{p}{(}\PYG{n}{x}\PYG{o}{+}\PYG{n}{y}\PYG{p}{)}

\PYG{n}{my\PYGZus{}func}\PYG{p}{(}\PYG{l+m+mi}{13}\PYG{p}{,}\PYG{l+m+mi}{7}\PYG{p}{)}
\end{sphinxVerbatim}

\end{sphinxuseclass}\end{sphinxVerbatimInput}

\end{sphinxuseclass}
\end{sphinxuseclass}

\subsection{Question}
\label{\detokenize{functions_output:id7}}
\begin{sphinxuseclass}{cell}
\begin{sphinxuseclass}{tag_hide-output}\begin{sphinxVerbatimInput}

\begin{sphinxuseclass}{cell_input}
\begin{sphinxVerbatim}[commandchars=\\\{\}]
\PYG{k}{def} \PYG{n+nf}{my\PYGZus{}func}\PYG{p}{(}\PYG{n}{x}\PYG{p}{,} \PYG{n}{y}\PYG{p}{,} \PYG{n}{z}\PYG{p}{)}\PYG{p}{:}
  \PYG{k}{if} \PYG{n}{x}\PYG{o}{+}\PYG{n}{y} \PYG{o}{\PYGZgt{}} \PYG{l+m+mi}{10}\PYG{p}{:}
    \PYG{n}{total} \PYG{o}{=} \PYG{n}{x}\PYG{o}{+}\PYG{n}{y}
  \PYG{k}{else}\PYG{p}{:}
    \PYG{n}{total} \PYG{o}{=} \PYG{n}{y}\PYG{o}{+}\PYG{n}{z}
  \PYG{k}{return} \PYG{n}{total}

\PYG{n+nb}{print}\PYG{p}{(}\PYG{n}{my\PYGZus{}func}\PYG{p}{(}\PYG{l+m+mi}{2}\PYG{p}{,}\PYG{l+m+mi}{7}\PYG{p}{,}\PYG{l+m+mi}{4}\PYG{p}{)}\PYG{p}{)}
\end{sphinxVerbatim}

\end{sphinxuseclass}\end{sphinxVerbatimInput}

\end{sphinxuseclass}
\end{sphinxuseclass}

\subsection{Question}
\label{\detokenize{functions_output:id8}}
\begin{sphinxuseclass}{cell}
\begin{sphinxuseclass}{tag_hide-output}\begin{sphinxVerbatimInput}

\begin{sphinxuseclass}{cell_input}
\begin{sphinxVerbatim}[commandchars=\\\{\}]
\PYG{n}{c} \PYG{o}{=} \PYG{l+m+mi}{3}
\PYG{k}{def} \PYG{n+nf}{f}\PYG{p}{(}\PYG{n}{x}\PYG{p}{,}\PYG{n}{y}\PYG{p}{)}\PYG{p}{:}
  \PYG{n}{z} \PYG{o}{=} \PYG{l+m+mi}{6}
  \PYG{k}{return} \PYG{n}{x}\PYG{o}{+}\PYG{n}{y}\PYG{o}{+}\PYG{n}{z}\PYG{o}{+}\PYG{n}{c}

\PYG{n+nb}{print}\PYG{p}{(}\PYG{n}{f}\PYG{p}{(}\PYG{l+m+mi}{1}\PYG{p}{,}\PYG{l+m+mi}{2}\PYG{p}{)}\PYG{p}{)}
\PYG{n}{c} \PYG{o}{=} \PYG{l+m+mi}{5}
\PYG{n+nb}{print}\PYG{p}{(}\PYG{n}{f}\PYG{p}{(}\PYG{l+m+mi}{1}\PYG{p}{,}\PYG{l+m+mi}{2}\PYG{p}{)}\PYG{p}{)}
\end{sphinxVerbatim}

\end{sphinxuseclass}\end{sphinxVerbatimInput}

\end{sphinxuseclass}
\end{sphinxuseclass}

\subsection{Question}
\label{\detokenize{functions_output:id9}}
\begin{sphinxuseclass}{cell}
\begin{sphinxuseclass}{tag_hide-output}\begin{sphinxVerbatimInput}

\begin{sphinxuseclass}{cell_input}
\begin{sphinxVerbatim}[commandchars=\\\{\}]
\PYG{k}{def} \PYG{n+nf}{f}\PYG{p}{(}\PYG{n}{x}\PYG{p}{,} \PYG{n}{y}\PYG{p}{)}\PYG{p}{:}
  \PYG{n+nb}{print}\PYG{p}{(}\PYG{l+s+s1}{\PYGZsq{}}\PYG{l+s+s1}{x:}\PYG{l+s+s1}{\PYGZsq{}}\PYG{p}{,}\PYG{n}{x}\PYG{p}{)}
  \PYG{k}{return} \PYG{n}{x}\PYG{o}{+}\PYG{n}{y}
    
\PYG{n+nb}{print}\PYG{p}{(}\PYG{n}{f}\PYG{p}{(}\PYG{l+m+mi}{1}\PYG{p}{,}\PYG{l+m+mi}{2}\PYG{p}{)}\PYG{p}{)}
\end{sphinxVerbatim}

\end{sphinxuseclass}\end{sphinxVerbatimInput}

\end{sphinxuseclass}
\end{sphinxuseclass}

\subsection{Question}
\label{\detokenize{functions_output:id10}}
\begin{sphinxuseclass}{cell}
\begin{sphinxuseclass}{tag_hide-output}\begin{sphinxVerbatimInput}

\begin{sphinxuseclass}{cell_input}
\begin{sphinxVerbatim}[commandchars=\\\{\}]
\PYG{k}{def} \PYG{n+nf}{f}\PYG{p}{(}\PYG{n}{x}\PYG{o}{=}\PYG{l+m+mi}{1}\PYG{p}{,} \PYG{n}{y}\PYG{o}{=}\PYG{l+m+mi}{6}\PYG{p}{)}\PYG{p}{:}
  \PYG{k}{return} \PYG{n}{x}\PYG{o}{+}\PYG{n}{y}
    
\PYG{n}{f}\PYG{p}{(}\PYG{p}{)}
\end{sphinxVerbatim}

\end{sphinxuseclass}\end{sphinxVerbatimInput}

\end{sphinxuseclass}
\end{sphinxuseclass}
\sphinxstepscope


\section{Functions Code}
\label{\detokenize{functions_code:functions-code}}\label{\detokenize{functions_code::doc}}\begin{itemize}
\item {} 
\sphinxAtStartPar
Please solve the following questions using Python code.  

\end{itemize}


\subsection{Question}
\label{\detokenize{functions_code:question}}
\sphinxAtStartPar
Write a function that accepts an integer as its parameter and prints an \(n \times n\) square with ‘*’ characters.

\sphinxAtStartPar
\sphinxstylestrong{Solution}

\begin{sphinxuseclass}{cell}\begin{sphinxVerbatimInput}

\begin{sphinxuseclass}{cell_input}
\begin{sphinxVerbatim}[commandchars=\\\{\}]
\PYG{c+c1}{\PYGZsh{} example}
\PYG{n}{square\PYGZus{}star}\PYG{p}{(}\PYG{l+m+mi}{5}\PYG{p}{)}
\end{sphinxVerbatim}

\end{sphinxuseclass}\end{sphinxVerbatimInput}
\begin{sphinxVerbatimOutput}

\begin{sphinxuseclass}{cell_output}
\begin{sphinxVerbatim}[commandchars=\\\{\}]
* * * * * 
* * * * * 
* * * * * 
* * * * * 
* * * * * 
\end{sphinxVerbatim}

\end{sphinxuseclass}\end{sphinxVerbatimOutput}

\end{sphinxuseclass}

\subsection{Question}
\label{\detokenize{functions_code:id1}}
\sphinxAtStartPar
Write a function that takes a string as its parameter and returns the vowels (‘a, e, i, o, u’) in the given string without repetition as a list.
\begin{itemize}
\item {} 
\sphinxAtStartPar
If there are no vowels in the string, return the string ‘No vowels’ .

\end{itemize}

\sphinxAtStartPar
\sphinxstylestrong{Solution}

\begin{sphinxuseclass}{cell}\begin{sphinxVerbatimInput}

\begin{sphinxuseclass}{cell_input}
\begin{sphinxVerbatim}[commandchars=\\\{\}]
\PYG{c+c1}{\PYGZsh{} example}
\PYG{n+nb}{print}\PYG{p}{(}\PYG{n}{vowel\PYGZus{}func}\PYG{p}{(}\PYG{l+s+s1}{\PYGZsq{}}\PYG{l+s+s1}{abbbecido}\PYG{l+s+s1}{\PYGZsq{}}\PYG{p}{)}\PYG{p}{)}
\end{sphinxVerbatim}

\end{sphinxuseclass}\end{sphinxVerbatimInput}
\begin{sphinxVerbatimOutput}

\begin{sphinxuseclass}{cell_output}
\begin{sphinxVerbatim}[commandchars=\\\{\}]
[\PYGZsq{}a\PYGZsq{}, \PYGZsq{}e\PYGZsq{}, \PYGZsq{}i\PYGZsq{}, \PYGZsq{}o\PYGZsq{}]
\end{sphinxVerbatim}

\end{sphinxuseclass}\end{sphinxVerbatimOutput}

\end{sphinxuseclass}
\begin{sphinxuseclass}{cell}\begin{sphinxVerbatimInput}

\begin{sphinxuseclass}{cell_input}
\begin{sphinxVerbatim}[commandchars=\\\{\}]
\PYG{c+c1}{\PYGZsh{} example}
\PYG{n+nb}{print}\PYG{p}{(}\PYG{n}{vowel\PYGZus{}func}\PYG{p}{(}\PYG{l+s+s1}{\PYGZsq{}}\PYG{l+s+s1}{bcd}\PYG{l+s+s1}{\PYGZsq{}}\PYG{p}{)}\PYG{p}{)}
\end{sphinxVerbatim}

\end{sphinxuseclass}\end{sphinxVerbatimInput}
\begin{sphinxVerbatimOutput}

\begin{sphinxuseclass}{cell_output}
\begin{sphinxVerbatim}[commandchars=\\\{\}]
No vowels!
\end{sphinxVerbatim}

\end{sphinxuseclass}\end{sphinxVerbatimOutput}

\end{sphinxuseclass}

\subsection{Question}
\label{\detokenize{functions_code:id2}}
\sphinxAtStartPar
Write a function that takes two integers as parameters and returns the powers of the first integer (base) from 0 to the second integer number.
\begin{itemize}
\item {} 
\sphinxAtStartPar
Example: parameters are 3, 5
\begin{itemize}
\item {} 
\sphinxAtStartPar
Output: \([3^0, 3^1, 3^2, 3^3, 3^4, 3^5]\)

\end{itemize}

\end{itemize}

\sphinxAtStartPar
\sphinxstylestrong{Solution\sphinxhyphen{}1}

\begin{sphinxuseclass}{cell}\begin{sphinxVerbatimInput}

\begin{sphinxuseclass}{cell_input}
\begin{sphinxVerbatim}[commandchars=\\\{\}]
\PYG{c+c1}{\PYGZsh{} example}
\PYG{n+nb}{print}\PYG{p}{(}\PYG{n}{power\PYGZus{}func}\PYG{p}{(}\PYG{l+m+mi}{3}\PYG{p}{,}\PYG{l+m+mi}{5}\PYG{p}{)}\PYG{p}{)}
\end{sphinxVerbatim}

\end{sphinxuseclass}\end{sphinxVerbatimInput}
\begin{sphinxVerbatimOutput}

\begin{sphinxuseclass}{cell_output}
\begin{sphinxVerbatim}[commandchars=\\\{\}]
[1, 3, 9, 27, 81, 243]
\end{sphinxVerbatim}

\end{sphinxuseclass}\end{sphinxVerbatimOutput}

\end{sphinxuseclass}
\sphinxAtStartPar
\sphinxstylestrong{Solution\sphinxhyphen{}2}

\begin{sphinxuseclass}{cell}\begin{sphinxVerbatimInput}

\begin{sphinxuseclass}{cell_input}
\begin{sphinxVerbatim}[commandchars=\\\{\}]
\PYG{c+c1}{\PYGZsh{} example}
\PYG{n+nb}{print}\PYG{p}{(}\PYG{n}{f}\PYG{p}{(}\PYG{l+m+mi}{3}\PYG{p}{,}\PYG{l+m+mi}{5}\PYG{p}{)}\PYG{p}{)}
\end{sphinxVerbatim}

\end{sphinxuseclass}\end{sphinxVerbatimInput}
\begin{sphinxVerbatimOutput}

\begin{sphinxuseclass}{cell_output}
\begin{sphinxVerbatim}[commandchars=\\\{\}]
[1, 3, 9, 27, 81, 243]
\end{sphinxVerbatim}

\end{sphinxuseclass}\end{sphinxVerbatimOutput}

\end{sphinxuseclass}

\subsection{Question}
\label{\detokenize{functions_code:id3}}
\sphinxAtStartPar
Write a function with parameters \sphinxstyleemphasis{start} and \sphinxstyleemphasis{end}, where it returns a list of all even numbers between start and end, inclusive.
\begin{itemize}
\item {} 
\sphinxAtStartPar
If start is greater than end, the output should be in descending order; if start is less than end, the output should be in ascending order.

\item {} 
\sphinxAtStartPar
Example:
\begin{itemize}
\item {} 
\sphinxAtStartPar
start=11, end=23 —> Output: {[}12, 14, 16, 18, 20, 22{]}

\item {} 
\sphinxAtStartPar
start=20, end= 5 —> Output: {[}20, 18, 16, 14, 12, 10, 8, 6{]}

\end{itemize}

\end{itemize}

\sphinxAtStartPar
\sphinxstylestrong{Solution\sphinxhyphen{}1}

\begin{sphinxuseclass}{cell}\begin{sphinxVerbatimInput}

\begin{sphinxuseclass}{cell_input}
\begin{sphinxVerbatim}[commandchars=\\\{\}]
\PYG{c+c1}{\PYGZsh{} example}
\PYG{n+nb}{print}\PYG{p}{(}\PYG{n}{even\PYGZus{}numbers}\PYG{p}{(}\PYG{l+m+mi}{45}\PYG{p}{,} \PYG{l+m+mi}{23}\PYG{p}{)}\PYG{p}{)}
\end{sphinxVerbatim}

\end{sphinxuseclass}\end{sphinxVerbatimInput}
\begin{sphinxVerbatimOutput}

\begin{sphinxuseclass}{cell_output}
\begin{sphinxVerbatim}[commandchars=\\\{\}]
[44, 42, 40, 38, 36, 34, 32, 30, 28, 26, 24]
\end{sphinxVerbatim}

\end{sphinxuseclass}\end{sphinxVerbatimOutput}

\end{sphinxuseclass}
\sphinxAtStartPar
\sphinxstylestrong{Solution\sphinxhyphen{}2}

\begin{sphinxuseclass}{cell}\begin{sphinxVerbatimInput}

\begin{sphinxuseclass}{cell_input}
\begin{sphinxVerbatim}[commandchars=\\\{\}]
\PYG{c+c1}{\PYGZsh{} example}
\PYG{n+nb}{print}\PYG{p}{(}\PYG{n}{even\PYGZus{}num}\PYG{p}{(}\PYG{l+m+mi}{20}\PYG{p}{,}\PYG{l+m+mi}{5}\PYG{p}{)}\PYG{p}{)}
\end{sphinxVerbatim}

\end{sphinxuseclass}\end{sphinxVerbatimInput}
\begin{sphinxVerbatimOutput}

\begin{sphinxuseclass}{cell_output}
\begin{sphinxVerbatim}[commandchars=\\\{\}]
[20, 18, 16, 14, 12, 10, 8, 6]
\end{sphinxVerbatim}

\end{sphinxuseclass}\end{sphinxVerbatimOutput}

\end{sphinxuseclass}

\subsection{Question}
\label{\detokenize{functions_code:id4}}
\sphinxAtStartPar
Write a function that takes two tuples as parameters and returns a tuple consisting of elements present in the first tuple but not in the second tuple.
\begin{itemize}
\item {} 
\sphinxAtStartPar
Example:
\begin{itemize}
\item {} 
\sphinxAtStartPar
tuple1 = {[}1, 5, 9, 12, 6{]}, tuple2 = {[}2, 7, 6, 9, 20, 23, 29{]} —> Output: {[}1, 5, 12{]}

\item {} 
\sphinxAtStartPar
tuple1 = {[}1,3{]}, tuple2 = {[}2, 7{]} —> Output: {[}1, 3{]}

\end{itemize}

\end{itemize}

\sphinxAtStartPar
\sphinxstylestrong{Solution\sphinxhyphen{}1}

\begin{sphinxuseclass}{cell}\begin{sphinxVerbatimInput}

\begin{sphinxuseclass}{cell_input}
\begin{sphinxVerbatim}[commandchars=\\\{\}]
\PYG{c+c1}{\PYGZsh{} example}
\PYG{n+nb}{print}\PYG{p}{(}\PYG{n}{diff\PYGZus{}tuples}\PYG{p}{(} \PYG{p}{(}\PYG{l+m+mi}{1}\PYG{p}{,}\PYG{l+m+mi}{2}\PYG{p}{,}\PYG{l+m+mi}{3}\PYG{p}{,}\PYG{l+m+mi}{4}\PYG{p}{)}\PYG{p}{,} \PYG{p}{(}\PYG{l+m+mi}{3}\PYG{p}{,}\PYG{l+m+mi}{4}\PYG{p}{,}\PYG{l+m+mi}{5}\PYG{p}{,}\PYG{l+m+mi}{6}\PYG{p}{)} \PYG{p}{)}\PYG{p}{)}
\end{sphinxVerbatim}

\end{sphinxuseclass}\end{sphinxVerbatimInput}
\begin{sphinxVerbatimOutput}

\begin{sphinxuseclass}{cell_output}
\begin{sphinxVerbatim}[commandchars=\\\{\}]
[1, 2]
\end{sphinxVerbatim}

\end{sphinxuseclass}\end{sphinxVerbatimOutput}

\end{sphinxuseclass}
\sphinxAtStartPar
\sphinxstylestrong{Solution\sphinxhyphen{}2}

\begin{sphinxuseclass}{cell}\begin{sphinxVerbatimInput}

\begin{sphinxuseclass}{cell_input}
\begin{sphinxVerbatim}[commandchars=\\\{\}]
\PYG{c+c1}{\PYGZsh{} example}
\PYG{n+nb}{print}\PYG{p}{(}\PYG{n}{diff\PYGZus{}tuples}\PYG{p}{(} \PYG{p}{(}\PYG{l+m+mi}{1}\PYG{p}{,}\PYG{l+m+mi}{2}\PYG{p}{,}\PYG{l+m+mi}{3}\PYG{p}{,}\PYG{l+m+mi}{4}\PYG{p}{)}\PYG{p}{,} \PYG{p}{(}\PYG{l+m+mi}{3}\PYG{p}{,}\PYG{l+m+mi}{4}\PYG{p}{,}\PYG{l+m+mi}{5}\PYG{p}{,}\PYG{l+m+mi}{6}\PYG{p}{)} \PYG{p}{)}\PYG{p}{)}
\end{sphinxVerbatim}

\end{sphinxuseclass}\end{sphinxVerbatimInput}
\begin{sphinxVerbatimOutput}

\begin{sphinxuseclass}{cell_output}
\begin{sphinxVerbatim}[commandchars=\\\{\}]
(1, 2)
\end{sphinxVerbatim}

\end{sphinxuseclass}\end{sphinxVerbatimOutput}

\end{sphinxuseclass}

\subsection{Question}
\label{\detokenize{functions_code:id5}}
\sphinxAtStartPar
Write a function that takes a date as a string in the format ‘month/day/year’, where the month is represented by a number.
\begin{itemize}
\item {} 
\sphinxAtStartPar
The function should return the date in the format ‘year/month/day’.

\item {} 
\sphinxAtStartPar
Example:
\begin{itemize}
\item {} 
\sphinxAtStartPar
Input : ‘09/25/1987’

\item {} 
\sphinxAtStartPar
Output: ‘1987/09/25’

\end{itemize}

\end{itemize}

\sphinxAtStartPar
\sphinxstylestrong{Solution}

\begin{sphinxuseclass}{cell}\begin{sphinxVerbatimInput}

\begin{sphinxuseclass}{cell_input}
\begin{sphinxVerbatim}[commandchars=\\\{\}]
\PYG{c+c1}{\PYGZsh{} example}
\PYG{n+nb}{print}\PYG{p}{(}\PYG{n}{date\PYGZus{}conv}\PYG{p}{(}\PYG{l+s+s1}{\PYGZsq{}}\PYG{l+s+s1}{09/25/1987}\PYG{l+s+s1}{\PYGZsq{}}\PYG{p}{)}\PYG{p}{)}
\end{sphinxVerbatim}

\end{sphinxuseclass}\end{sphinxVerbatimInput}
\begin{sphinxVerbatimOutput}

\begin{sphinxuseclass}{cell_output}
\begin{sphinxVerbatim}[commandchars=\\\{\}]
1987/09/25
\end{sphinxVerbatim}

\end{sphinxuseclass}\end{sphinxVerbatimOutput}

\end{sphinxuseclass}

\subsection{Question}
\label{\detokenize{functions_code:id6}}
\sphinxAtStartPar
Write a function that takes a list as its parameter and returns the standard deviation of its elements.\\
The standard deviation (std) for a sequence is calculated using the formula: \(\displaystyle std = \sqrt\frac{\sum_{i=1}^{n}(x_i-\bar{x})^2}{n}\)

\sphinxAtStartPar
where  \(\bar{x}\)  is the mean and \(\sum\) denotes the sum. Here’s a hint for the implementation:
\begin{enumerate}
\sphinxsetlistlabels{\arabic}{enumi}{enumii}{}{.}%
\item {} 
\sphinxAtStartPar
Find the mean.

\item {} 
\sphinxAtStartPar
Subtract mean from each element.

\item {} 
\sphinxAtStartPar
Square the diferences.

\item {} 
\sphinxAtStartPar
Sum the squared differences.

\item {} 
\sphinxAtStartPar
Divide by n

\item {} 
\sphinxAtStartPar
Take square root

\end{enumerate}
\begin{itemize}
\item {} 
\sphinxAtStartPar
Find the std of the list: {[}5,8,2,5,6,1,1,3{]}

\item {} 
\sphinxAtStartPar
Compare the result obtained from your custom standard deviation function with the standard deviation functions provided by the numpy and statistics libraries.

\end{itemize}

\sphinxAtStartPar
\sphinxstylestrong{Solution}

\begin{sphinxuseclass}{cell}\begin{sphinxVerbatimInput}

\begin{sphinxuseclass}{cell_input}
\begin{sphinxVerbatim}[commandchars=\\\{\}]
\PYG{c+c1}{\PYGZsh{} example}
\PYG{n}{mylist} \PYG{o}{=} \PYG{p}{[}\PYG{l+m+mi}{5}\PYG{p}{,}\PYG{l+m+mi}{8}\PYG{p}{,}\PYG{l+m+mi}{2}\PYG{p}{,}\PYG{l+m+mi}{5}\PYG{p}{,}\PYG{l+m+mi}{6}\PYG{p}{,}\PYG{l+m+mi}{1}\PYG{p}{,}\PYG{l+m+mi}{1}\PYG{p}{,}\PYG{l+m+mi}{3}\PYG{p}{]}

\PYG{n+nb}{print}\PYG{p}{(}\PYG{l+s+sa}{f}\PYG{l+s+s1}{\PYGZsq{}}\PYG{l+s+s1}{Custom Function: }\PYG{l+s+si}{\PYGZob{}}\PYG{n}{std\PYGZus{}func}\PYG{p}{(}\PYG{n}{mylist}\PYG{p}{)}\PYG{l+s+si}{\PYGZcb{}}\PYG{l+s+s1}{\PYGZsq{}}\PYG{p}{)}
\PYG{n+nb}{print}\PYG{p}{(}\PYG{l+s+sa}{f}\PYG{l+s+s1}{\PYGZsq{}}\PYG{l+s+s1}{Numpy          : }\PYG{l+s+si}{\PYGZob{}}\PYG{n}{np}\PYG{o}{.}\PYG{n}{std}\PYG{p}{(}\PYG{n}{mylist}\PYG{p}{)}\PYG{l+s+si}{\PYGZcb{}}\PYG{l+s+s1}{\PYGZsq{}}\PYG{p}{)}
\PYG{n+nb}{print}\PYG{p}{(}\PYG{l+s+sa}{f}\PYG{l+s+s1}{\PYGZsq{}}\PYG{l+s+s1}{Statistics     : }\PYG{l+s+si}{\PYGZob{}}\PYG{n}{statistics}\PYG{o}{.}\PYG{n}{pstdev}\PYG{p}{(}\PYG{n}{mylist}\PYG{p}{)}\PYG{l+s+si}{\PYGZcb{}}\PYG{l+s+s1}{\PYGZsq{}}\PYG{p}{)}
\end{sphinxVerbatim}

\end{sphinxuseclass}\end{sphinxVerbatimInput}
\begin{sphinxVerbatimOutput}

\begin{sphinxuseclass}{cell_output}
\begin{sphinxVerbatim}[commandchars=\\\{\}]
Custom Function: 2.368411915187052
Numpy          : 2.368411915187052
Statistics     : 2.368411915187052
\end{sphinxVerbatim}

\end{sphinxuseclass}\end{sphinxVerbatimOutput}

\end{sphinxuseclass}
\sphinxstepscope


\chapter{Chp\sphinxhyphen{}10: Sets}
\label{\detokenize{sets:chp-10-sets}}\label{\detokenize{sets::doc}}\begin{itemize}
\item {} 
\sphinxAtStartPar
Learning Objectives
\begin{itemize}
\item {} 
\sphinxAtStartPar
..

\item {} 
\sphinxAtStartPar
..

\end{itemize}

\end{itemize}

\sphinxAtStartPar
Sets are an unordered collection of values.
\begin{itemize}
\item {} 
\sphinxAtStartPar
Since there is no order there is no indexing for sets.

\item {} 
\sphinxAtStartPar
It can contain a mixed type of elements.

\item {} 
\sphinxAtStartPar
Curly brackets \sphinxcode{\sphinxupquote{\{\}}} are used to create sets.
\begin{itemize}
\item {} 
\sphinxAtStartPar
\sphinxstylestrong{Warning:} \sphinxcode{\sphinxupquote{\{\}}} is NOT an empty set. It is an empty \sphinxstyleemphasis{dictionary} that will be covered in later chapters.

\end{itemize}

\item {} 
\sphinxAtStartPar
Sets are mutable. so they can be modified like lists.

\item {} 
\sphinxAtStartPar
A tuple can be an element of a set.

\item {} 
\sphinxAtStartPar
A list cannot be an element of a set.

\item {} 
\sphinxAtStartPar
You can use the \sphinxcode{\sphinxupquote{set()}} function to convert strings, tuples, and lists into a set.
\begin{itemize}
\item {} 
\sphinxAtStartPar
Only unique values are stored in sets (no repetition).

\item {} 
\sphinxAtStartPar
\sphinxstylestrong{Warning:} You will lose the order of the elements when you use the conversion.

\item {} 
\sphinxAtStartPar
Example:
\begin{itemize}
\item {} 
\begin{sphinxVerbatim}[commandchars=\\\{\}]
\PYG{n}{my\PYGZus{}list} \PYG{o}{=} \PYG{p}{[}\PYG{l+m+mi}{1}\PYG{p}{,}\PYG{l+m+mi}{4}\PYG{p}{,}\PYG{l+m+mi}{7}\PYG{p}{,}\PYG{l+m+mi}{7}\PYG{p}{,}\PYG{l+m+mi}{8}\PYG{p}{]}

\PYG{n}{my\PYGZus{}set} \PYG{o}{=} \PYG{n+nb}{set}\PYG{p}{(}\PYG{n}{my\PYGZus{}list}\PYG{p}{)}

\PYG{n+nb}{print}\PYG{p}{(}\PYG{n}{myset}\PYG{p}{)}
\end{sphinxVerbatim}

\item {} 
\sphinxAtStartPar
Output: 1,4,7,8

\end{itemize}

\end{itemize}

\end{itemize}


\section{Create Sets}
\label{\detokenize{sets:create-sets}}
\begin{sphinxuseclass}{cell}\begin{sphinxVerbatimInput}

\begin{sphinxuseclass}{cell_input}
\begin{sphinxVerbatim}[commandchars=\\\{\}]
\PYG{c+c1}{\PYGZsh{} empty set}
\PYG{n}{empty\PYGZus{}set} \PYG{o}{=} \PYG{n+nb}{set}\PYG{p}{(}\PYG{p}{)}

\PYG{n+nb}{print}\PYG{p}{(}\PYG{l+s+sa}{f}\PYG{l+s+s1}{\PYGZsq{}}\PYG{l+s+s1}{Empty set         : }\PYG{l+s+si}{\PYGZob{}}\PYG{n}{empty\PYGZus{}set}\PYG{l+s+si}{\PYGZcb{}}\PYG{l+s+s1}{\PYGZsq{}}\PYG{p}{)}
\PYG{n+nb}{print}\PYG{p}{(}\PYG{l+s+sa}{f}\PYG{l+s+s1}{\PYGZsq{}}\PYG{l+s+s1}{Type of emppty\PYGZus{}set: }\PYG{l+s+si}{\PYGZob{}}\PYG{n+nb}{type}\PYG{p}{(}\PYG{n}{empty\PYGZus{}set}\PYG{p}{)}\PYG{l+s+si}{\PYGZcb{}}\PYG{l+s+s1}{\PYGZsq{}}\PYG{p}{)}
\end{sphinxVerbatim}

\end{sphinxuseclass}\end{sphinxVerbatimInput}
\begin{sphinxVerbatimOutput}

\begin{sphinxuseclass}{cell_output}
\begin{sphinxVerbatim}[commandchars=\\\{\}]
Empty set         : set()
Type of emppty\PYGZus{}set: \PYGZlt{}class \PYGZsq{}set\PYGZsq{}\PYGZgt{}
\end{sphinxVerbatim}

\end{sphinxuseclass}\end{sphinxVerbatimOutput}

\end{sphinxuseclass}
\begin{sphinxuseclass}{cell}\begin{sphinxVerbatimInput}

\begin{sphinxuseclass}{cell_input}
\begin{sphinxVerbatim}[commandchars=\\\{\}]
\PYG{c+c1}{\PYGZsh{} set with mixed values: str, int, bool, float}

\PYG{n}{s} \PYG{o}{=} \PYG{p}{\PYGZob{}}\PYG{l+s+s1}{\PYGZsq{}}\PYG{l+s+s1}{USA}\PYG{l+s+s1}{\PYGZsq{}}\PYG{p}{,} \PYG{l+m+mi}{2}\PYG{p}{,} \PYG{k+kc}{True}\PYG{p}{,} \PYG{l+m+mf}{9.123}\PYG{p}{\PYGZcb{}}      

\PYG{n+nb}{print}\PYG{p}{(}\PYG{n+nb}{type}\PYG{p}{(}\PYG{n}{s}\PYG{p}{)}\PYG{p}{)}
\end{sphinxVerbatim}

\end{sphinxuseclass}\end{sphinxVerbatimInput}
\begin{sphinxVerbatimOutput}

\begin{sphinxuseclass}{cell_output}
\begin{sphinxVerbatim}[commandchars=\\\{\}]
\PYGZlt{}class \PYGZsq{}set\PYGZsq{}\PYGZgt{}
\end{sphinxVerbatim}

\end{sphinxuseclass}\end{sphinxVerbatimOutput}

\end{sphinxuseclass}
\begin{sphinxuseclass}{cell}\begin{sphinxVerbatimInput}

\begin{sphinxuseclass}{cell_input}
\begin{sphinxVerbatim}[commandchars=\\\{\}]
\PYG{c+c1}{\PYGZsh{} set removes the repetitions}

\PYG{n}{s} \PYG{o}{=} \PYG{p}{\PYGZob{}}\PYG{l+s+s1}{\PYGZsq{}}\PYG{l+s+s1}{USA}\PYG{l+s+s1}{\PYGZsq{}}\PYG{p}{,} \PYG{l+m+mi}{2}\PYG{p}{,} \PYG{k+kc}{True}\PYG{p}{,} \PYG{l+m+mf}{9.123}\PYG{p}{,} \PYG{l+s+s1}{\PYGZsq{}}\PYG{l+s+s1}{USA}\PYG{l+s+s1}{\PYGZsq{}}\PYG{p}{,} \PYG{l+s+s1}{\PYGZsq{}}\PYG{l+s+s1}{USA}\PYG{l+s+s1}{\PYGZsq{}}\PYG{p}{,} \PYG{l+s+s1}{\PYGZsq{}}\PYG{l+s+s1}{USA}\PYG{l+s+s1}{\PYGZsq{}}\PYG{p}{,} \PYG{l+s+s1}{\PYGZsq{}}\PYG{l+s+s1}{USA}\PYG{l+s+s1}{\PYGZsq{}}\PYG{p}{\PYGZcb{}}      

\PYG{n+nb}{print}\PYG{p}{(}\PYG{n}{s}\PYG{p}{)}   \PYG{c+c1}{\PYGZsh{} only one \PYGZsq{}USA\PYGZsq{} will be in the set s}
\end{sphinxVerbatim}

\end{sphinxuseclass}\end{sphinxVerbatimInput}
\begin{sphinxVerbatimOutput}

\begin{sphinxuseclass}{cell_output}
\begin{sphinxVerbatim}[commandchars=\\\{\}]
\PYGZob{}True, \PYGZsq{}USA\PYGZsq{}, 2, 9.123\PYGZcb{}
\end{sphinxVerbatim}

\end{sphinxuseclass}\end{sphinxVerbatimOutput}

\end{sphinxuseclass}
\begin{sphinxuseclass}{cell}\begin{sphinxVerbatimInput}

\begin{sphinxuseclass}{cell_input}
\begin{sphinxVerbatim}[commandchars=\\\{\}]
\PYG{c+c1}{\PYGZsh{} tuple and set in a set}
\PYG{c+c1}{\PYGZsh{} set with mixed values: str, int, bool, float, tuple, set}
\PYG{c+c1}{\PYGZsh{} (10,20,30) is a tuple and [\PYGZsq{}a\PYGZsq{},\PYGZsq{}b\PYGZsq{}] is a list in the list mixed\PYGZus{}list.}

\PYG{n}{s} \PYG{o}{=} \PYG{p}{[}\PYG{l+s+s1}{\PYGZsq{}}\PYG{l+s+s1}{USA}\PYG{l+s+s1}{\PYGZsq{}}\PYG{p}{,} \PYG{l+m+mi}{2}\PYG{p}{,} \PYG{k+kc}{True}\PYG{p}{,} \PYG{l+m+mf}{9.123}\PYG{p}{,} \PYG{p}{(}\PYG{l+m+mi}{10}\PYG{p}{,}\PYG{l+m+mi}{20}\PYG{p}{,}\PYG{l+m+mi}{30}\PYG{p}{)}\PYG{p}{,} \PYG{p}{\PYGZob{}}\PYG{l+s+s1}{\PYGZsq{}}\PYG{l+s+s1}{a}\PYG{l+s+s1}{\PYGZsq{}}\PYG{p}{,}\PYG{l+s+s1}{\PYGZsq{}}\PYG{l+s+s1}{b}\PYG{l+s+s1}{\PYGZsq{}}\PYG{p}{\PYGZcb{}}\PYG{p}{]}       
\PYG{n+nb}{print}\PYG{p}{(}\PYG{n+nb}{type}\PYG{p}{(}\PYG{n}{s}\PYG{p}{)}\PYG{p}{)}
\end{sphinxVerbatim}

\end{sphinxuseclass}\end{sphinxVerbatimInput}
\begin{sphinxVerbatimOutput}

\begin{sphinxuseclass}{cell_output}
\begin{sphinxVerbatim}[commandchars=\\\{\}]
\PYGZlt{}class \PYGZsq{}list\PYGZsq{}\PYGZgt{}
\end{sphinxVerbatim}

\end{sphinxuseclass}\end{sphinxVerbatimOutput}

\end{sphinxuseclass}
\begin{sphinxVerbatim}[commandchars=\\\{\}]
\PYG{c+c1}{\PYGZsh{} a list can not be an element of a set}

\PYG{n}{s} \PYG{o}{=} \PYG{p}{\PYGZob{}}\PYG{l+s+s1}{\PYGZsq{}}\PYG{l+s+s1}{USA}\PYG{l+s+s1}{\PYGZsq{}}\PYG{p}{,} \PYG{l+m+mi}{2}\PYG{p}{,} \PYG{k+kc}{True}\PYG{p}{,} \PYG{l+m+mf}{9.123}\PYG{p}{,} \PYG{p}{(}\PYG{l+m+mi}{10}\PYG{p}{,}\PYG{l+m+mi}{20}\PYG{p}{,}\PYG{l+m+mi}{30}\PYG{p}{)}\PYG{p}{,} \PYG{p}{[}\PYG{l+s+s1}{\PYGZsq{}}\PYG{l+s+s1}{a}\PYG{l+s+s1}{\PYGZsq{}}\PYG{p}{,}\PYG{l+s+s1}{\PYGZsq{}}\PYG{l+s+s1}{b}\PYG{l+s+s1}{\PYGZsq{}}\PYG{p}{]}\PYG{p}{\PYGZcb{}}         \PYG{c+c1}{\PYGZsh{} ERROR}

\end{sphinxVerbatim}


\section{set() function}
\label{\detokenize{sets:set-function}}\begin{itemize}
\item {} 
\sphinxAtStartPar
The built\sphinxhyphen{}in \sphinxcode{\sphinxupquote{set()}} function converts strings, tuples, and lists to a set, removing any duplicates and retaining only unique elements.

\end{itemize}

\begin{sphinxuseclass}{cell}\begin{sphinxVerbatimInput}

\begin{sphinxuseclass}{cell_input}
\begin{sphinxVerbatim}[commandchars=\\\{\}]
\PYG{n}{char\PYGZus{}set} \PYG{o}{=} \PYG{n+nb}{set}\PYG{p}{(}\PYG{l+s+s1}{\PYGZsq{}}\PYG{l+s+s1}{Hello}\PYG{l+s+s1}{\PYGZsq{}}\PYG{p}{)}  \PYG{c+c1}{\PYGZsh{} convert string to set}

\PYG{n+nb}{print}\PYG{p}{(}\PYG{l+s+sa}{f}\PYG{l+s+s1}{\PYGZsq{}}\PYG{l+s+s1}{Type of char\PYGZus{}set: }\PYG{l+s+si}{\PYGZob{}}\PYG{n+nb}{type}\PYG{p}{(}\PYG{n}{char\PYGZus{}set}\PYG{p}{)}\PYG{l+s+si}{\PYGZcb{}}\PYG{l+s+s1}{\PYGZsq{}}\PYG{p}{)}
\PYG{n+nb}{print}\PYG{p}{(}\PYG{l+s+sa}{f}\PYG{l+s+s1}{\PYGZsq{}}\PYG{l+s+s1}{char\PYGZus{}set        : }\PYG{l+s+si}{\PYGZob{}}\PYG{n}{char\PYGZus{}set}\PYG{l+s+si}{\PYGZcb{}}\PYG{l+s+s1}{\PYGZsq{}}\PYG{p}{)}
\end{sphinxVerbatim}

\end{sphinxuseclass}\end{sphinxVerbatimInput}
\begin{sphinxVerbatimOutput}

\begin{sphinxuseclass}{cell_output}
\begin{sphinxVerbatim}[commandchars=\\\{\}]
Type of char\PYGZus{}set: \PYGZlt{}class \PYGZsq{}set\PYGZsq{}\PYGZgt{}
char\PYGZus{}set        : \PYGZob{}\PYGZsq{}e\PYGZsq{}, \PYGZsq{}l\PYGZsq{}, \PYGZsq{}o\PYGZsq{}, \PYGZsq{}H\PYGZsq{}\PYGZcb{}
\end{sphinxVerbatim}

\end{sphinxuseclass}\end{sphinxVerbatimOutput}

\end{sphinxuseclass}
\begin{sphinxuseclass}{cell}\begin{sphinxVerbatimInput}

\begin{sphinxuseclass}{cell_input}
\begin{sphinxVerbatim}[commandchars=\\\{\}]
\PYG{n}{t} \PYG{o}{=} \PYG{p}{(}\PYG{l+m+mi}{1}\PYG{p}{,}\PYG{l+m+mi}{2}\PYG{p}{,}\PYG{l+m+mi}{3}\PYG{p}{,}\PYG{l+m+mi}{4}\PYG{p}{,}\PYG{l+m+mi}{4}\PYG{p}{,}\PYG{l+m+mi}{4}\PYG{p}{)}
\PYG{n}{char\PYGZus{}set} \PYG{o}{=} \PYG{n+nb}{set}\PYG{p}{(}\PYG{n}{t}\PYG{p}{)}  \PYG{c+c1}{\PYGZsh{} convert tuple to  set}

\PYG{n+nb}{print}\PYG{p}{(}\PYG{l+s+sa}{f}\PYG{l+s+s1}{\PYGZsq{}}\PYG{l+s+s1}{Type of char\PYGZus{}set: }\PYG{l+s+si}{\PYGZob{}}\PYG{n+nb}{type}\PYG{p}{(}\PYG{n}{char\PYGZus{}set}\PYG{p}{)}\PYG{l+s+si}{\PYGZcb{}}\PYG{l+s+s1}{\PYGZsq{}}\PYG{p}{)}
\PYG{n+nb}{print}\PYG{p}{(}\PYG{l+s+sa}{f}\PYG{l+s+s1}{\PYGZsq{}}\PYG{l+s+s1}{char\PYGZus{}set        : }\PYG{l+s+si}{\PYGZob{}}\PYG{n}{char\PYGZus{}set}\PYG{l+s+si}{\PYGZcb{}}\PYG{l+s+s1}{\PYGZsq{}}\PYG{p}{)}
\end{sphinxVerbatim}

\end{sphinxuseclass}\end{sphinxVerbatimInput}
\begin{sphinxVerbatimOutput}

\begin{sphinxuseclass}{cell_output}
\begin{sphinxVerbatim}[commandchars=\\\{\}]
Type of char\PYGZus{}set: \PYGZlt{}class \PYGZsq{}set\PYGZsq{}\PYGZgt{}
char\PYGZus{}set        : \PYGZob{}1, 2, 3, 4\PYGZcb{}
\end{sphinxVerbatim}

\end{sphinxuseclass}\end{sphinxVerbatimOutput}

\end{sphinxuseclass}
\begin{sphinxuseclass}{cell}\begin{sphinxVerbatimInput}

\begin{sphinxuseclass}{cell_input}
\begin{sphinxVerbatim}[commandchars=\\\{\}]
\PYG{n}{number\PYGZus{}list} \PYG{o}{=} \PYG{p}{[}\PYG{l+m+mi}{1}\PYG{p}{,}\PYG{l+m+mi}{2}\PYG{p}{,}\PYG{l+m+mi}{3}\PYG{p}{,}\PYG{l+m+mi}{4}\PYG{p}{,}\PYG{l+m+mi}{4}\PYG{p}{,}\PYG{l+m+mi}{4}\PYG{p}{]}
\PYG{n}{char\PYGZus{}set} \PYG{o}{=} \PYG{n+nb}{set}\PYG{p}{(}\PYG{n}{number\PYGZus{}list}\PYG{p}{)}  \PYG{c+c1}{\PYGZsh{} convert list to  set}

\PYG{n+nb}{print}\PYG{p}{(}\PYG{l+s+sa}{f}\PYG{l+s+s1}{\PYGZsq{}}\PYG{l+s+s1}{Type of char\PYGZus{}set: }\PYG{l+s+si}{\PYGZob{}}\PYG{n+nb}{type}\PYG{p}{(}\PYG{n}{char\PYGZus{}set}\PYG{p}{)}\PYG{l+s+si}{\PYGZcb{}}\PYG{l+s+s1}{\PYGZsq{}}\PYG{p}{)}
\PYG{n+nb}{print}\PYG{p}{(}\PYG{l+s+sa}{f}\PYG{l+s+s1}{\PYGZsq{}}\PYG{l+s+s1}{char\PYGZus{}set        : }\PYG{l+s+si}{\PYGZob{}}\PYG{n}{char\PYGZus{}set}\PYG{l+s+si}{\PYGZcb{}}\PYG{l+s+s1}{\PYGZsq{}}\PYG{p}{)}
\end{sphinxVerbatim}

\end{sphinxuseclass}\end{sphinxVerbatimInput}
\begin{sphinxVerbatimOutput}

\begin{sphinxuseclass}{cell_output}
\begin{sphinxVerbatim}[commandchars=\\\{\}]
Type of char\PYGZus{}set: \PYGZlt{}class \PYGZsq{}set\PYGZsq{}\PYGZgt{}
char\PYGZus{}set        : \PYGZob{}1, 2, 3, 4\PYGZcb{}
\end{sphinxVerbatim}

\end{sphinxuseclass}\end{sphinxVerbatimOutput}

\end{sphinxuseclass}
\sphinxAtStartPar
\sphinxstylestrong{Remark:} By using \sphinxstyleemphasis{tuple()} and  \sphinxstyleemphasis{list()} functions, sets can be converted to tuples and lists, respectively.
\begin{itemize}
\item {} 
\sphinxAtStartPar
It’s important to note that the conversion doesn’t preserve any specific order, as sets are inherently unordered collections.

\end{itemize}

\begin{sphinxuseclass}{cell}\begin{sphinxVerbatimInput}

\begin{sphinxuseclass}{cell_input}
\begin{sphinxVerbatim}[commandchars=\\\{\}]
\PYG{n}{s} \PYG{o}{=} \PYG{p}{\PYGZob{}}\PYG{l+m+mi}{1}\PYG{p}{,}\PYG{l+m+mi}{2}\PYG{p}{,}\PYG{l+m+mi}{3}\PYG{p}{,}\PYG{l+m+mi}{4}\PYG{p}{\PYGZcb{}}   \PYG{c+c1}{\PYGZsh{} set}

\PYG{n}{s\PYGZus{}tuple} \PYG{o}{=} \PYG{n+nb}{tuple}\PYG{p}{(}\PYG{n}{s}\PYG{p}{)}      \PYG{c+c1}{\PYGZsh{} set \PYGZhy{}\PYGZhy{}\PYGZhy{}\PYGZgt{} tuple}

\PYG{n+nb}{print}\PYG{p}{(}\PYG{l+s+sa}{f}\PYG{l+s+s1}{\PYGZsq{}}\PYG{l+s+s1}{Type of s\PYGZus{}tuple: }\PYG{l+s+si}{\PYGZob{}}\PYG{n+nb}{type}\PYG{p}{(}\PYG{n}{s\PYGZus{}tuple}\PYG{p}{)}\PYG{l+s+si}{\PYGZcb{}}\PYG{l+s+s1}{\PYGZsq{}}\PYG{p}{)}
\PYG{n+nb}{print}\PYG{p}{(}\PYG{l+s+sa}{f}\PYG{l+s+s1}{\PYGZsq{}}\PYG{l+s+s1}{s\PYGZus{}tuple        : }\PYG{l+s+si}{\PYGZob{}}\PYG{n}{s\PYGZus{}tuple}\PYG{l+s+si}{\PYGZcb{}}\PYG{l+s+s1}{\PYGZsq{}}\PYG{p}{)}
\end{sphinxVerbatim}

\end{sphinxuseclass}\end{sphinxVerbatimInput}
\begin{sphinxVerbatimOutput}

\begin{sphinxuseclass}{cell_output}
\begin{sphinxVerbatim}[commandchars=\\\{\}]
Type of s\PYGZus{}tuple: \PYGZlt{}class \PYGZsq{}tuple\PYGZsq{}\PYGZgt{}
s\PYGZus{}tuple        : (1, 2, 3, 4)
\end{sphinxVerbatim}

\end{sphinxuseclass}\end{sphinxVerbatimOutput}

\end{sphinxuseclass}
\begin{sphinxuseclass}{cell}\begin{sphinxVerbatimInput}

\begin{sphinxuseclass}{cell_input}
\begin{sphinxVerbatim}[commandchars=\\\{\}]
\PYG{n}{s} \PYG{o}{=} \PYG{p}{\PYGZob{}}\PYG{l+m+mi}{1}\PYG{p}{,}\PYG{l+m+mi}{2}\PYG{p}{,}\PYG{l+m+mi}{3}\PYG{p}{,}\PYG{l+m+mi}{4}\PYG{p}{\PYGZcb{}}   \PYG{c+c1}{\PYGZsh{} set}

\PYG{n}{s\PYGZus{}list} \PYG{o}{=} \PYG{n+nb}{list}\PYG{p}{(}\PYG{n}{s}\PYG{p}{)}      \PYG{c+c1}{\PYGZsh{} set \PYGZhy{}\PYGZhy{}\PYGZhy{}\PYGZgt{} list}

\PYG{n+nb}{print}\PYG{p}{(}\PYG{l+s+sa}{f}\PYG{l+s+s1}{\PYGZsq{}}\PYG{l+s+s1}{Type of s\PYGZus{}list: }\PYG{l+s+si}{\PYGZob{}}\PYG{n+nb}{type}\PYG{p}{(}\PYG{n}{s\PYGZus{}list}\PYG{p}{)}\PYG{l+s+si}{\PYGZcb{}}\PYG{l+s+s1}{\PYGZsq{}}\PYG{p}{)}
\PYG{n+nb}{print}\PYG{p}{(}\PYG{l+s+sa}{f}\PYG{l+s+s1}{\PYGZsq{}}\PYG{l+s+s1}{s\PYGZus{}list        : }\PYG{l+s+si}{\PYGZob{}}\PYG{n}{s\PYGZus{}list}\PYG{l+s+si}{\PYGZcb{}}\PYG{l+s+s1}{\PYGZsq{}}\PYG{p}{)}
\end{sphinxVerbatim}

\end{sphinxuseclass}\end{sphinxVerbatimInput}
\begin{sphinxVerbatimOutput}

\begin{sphinxuseclass}{cell_output}
\begin{sphinxVerbatim}[commandchars=\\\{\}]
Type of s\PYGZus{}list: \PYGZlt{}class \PYGZsq{}list\PYGZsq{}\PYGZgt{}
s\PYGZus{}list        : [1, 2, 3, 4]
\end{sphinxVerbatim}

\end{sphinxuseclass}\end{sphinxVerbatimOutput}

\end{sphinxuseclass}

\section{Functions on sets}
\label{\detokenize{sets:functions-on-sets}}\begin{itemize}
\item {} 
\sphinxAtStartPar
len(), max(), min(), and sum() functions can be applied to sets, similar to other data types like tuples and lists.

\end{itemize}

\begin{sphinxuseclass}{cell}\begin{sphinxVerbatimInput}

\begin{sphinxuseclass}{cell_input}
\begin{sphinxVerbatim}[commandchars=\\\{\}]
\PYG{n}{numbers} \PYG{o}{=} \PYG{p}{\PYGZob{}}\PYG{l+m+mi}{7}\PYG{p}{,}\PYG{l+m+mi}{3}\PYG{p}{,}\PYG{l+m+mi}{1}\PYG{p}{,}\PYG{l+m+mi}{9}\PYG{p}{,}\PYG{l+m+mi}{6}\PYG{p}{,}\PYG{l+m+mi}{4}\PYG{p}{\PYGZcb{}}

\PYG{n+nb}{print}\PYG{p}{(}\PYG{l+s+sa}{f}\PYG{l+s+s1}{\PYGZsq{}}\PYG{l+s+s1}{Length : }\PYG{l+s+si}{\PYGZob{}}\PYG{n+nb}{len}\PYG{p}{(}\PYG{n}{numbers}\PYG{p}{)}\PYG{l+s+si}{\PYGZcb{}}\PYG{l+s+s1}{\PYGZsq{}}\PYG{p}{)}
\PYG{n+nb}{print}\PYG{p}{(}\PYG{l+s+sa}{f}\PYG{l+s+s1}{\PYGZsq{}}\PYG{l+s+s1}{Maximum: }\PYG{l+s+si}{\PYGZob{}}\PYG{n+nb}{max}\PYG{p}{(}\PYG{n}{numbers}\PYG{p}{)}\PYG{l+s+si}{\PYGZcb{}}\PYG{l+s+s1}{\PYGZsq{}}\PYG{p}{)}
\PYG{n+nb}{print}\PYG{p}{(}\PYG{l+s+sa}{f}\PYG{l+s+s1}{\PYGZsq{}}\PYG{l+s+s1}{Minimum: }\PYG{l+s+si}{\PYGZob{}}\PYG{n+nb}{min}\PYG{p}{(}\PYG{n}{numbers}\PYG{p}{)}\PYG{l+s+si}{\PYGZcb{}}\PYG{l+s+s1}{\PYGZsq{}}\PYG{p}{)}
\PYG{n+nb}{print}\PYG{p}{(}\PYG{l+s+sa}{f}\PYG{l+s+s1}{\PYGZsq{}}\PYG{l+s+s1}{Sum    : }\PYG{l+s+si}{\PYGZob{}}\PYG{n+nb}{sum}\PYG{p}{(}\PYG{n}{numbers}\PYG{p}{)}\PYG{l+s+si}{\PYGZcb{}}\PYG{l+s+s1}{\PYGZsq{}}\PYG{p}{)}
\end{sphinxVerbatim}

\end{sphinxuseclass}\end{sphinxVerbatimInput}
\begin{sphinxVerbatimOutput}

\begin{sphinxuseclass}{cell_output}
\begin{sphinxVerbatim}[commandchars=\\\{\}]
Length : 6
Maximum: 9
Minimum: 1
Sum    : 30
\end{sphinxVerbatim}

\end{sphinxuseclass}\end{sphinxVerbatimOutput}

\end{sphinxuseclass}
\begin{sphinxuseclass}{cell}\begin{sphinxVerbatimInput}

\begin{sphinxuseclass}{cell_input}
\begin{sphinxVerbatim}[commandchars=\\\{\}]
\PYG{n}{letters} \PYG{o}{=} \PYG{p}{\PYGZob{}}\PYG{l+s+s1}{\PYGZsq{}}\PYG{l+s+s1}{r}\PYG{l+s+s1}{\PYGZsq{}}\PYG{p}{,} \PYG{l+s+s1}{\PYGZsq{}}\PYG{l+s+s1}{t}\PYG{l+s+s1}{\PYGZsq{}}\PYG{p}{,} \PYG{l+s+s1}{\PYGZsq{}}\PYG{l+s+s1}{n}\PYG{l+s+s1}{\PYGZsq{}}\PYG{p}{,} \PYG{l+s+s1}{\PYGZsq{}}\PYG{l+s+s1}{a}\PYG{l+s+s1}{\PYGZsq{}}\PYG{p}{,} \PYG{l+s+s1}{\PYGZsq{}}\PYG{l+s+s1}{d}\PYG{l+s+s1}{\PYGZsq{}}\PYG{p}{\PYGZcb{}}

\PYG{n+nb}{print}\PYG{p}{(}\PYG{l+s+sa}{f}\PYG{l+s+s1}{\PYGZsq{}}\PYG{l+s+s1}{Length : }\PYG{l+s+si}{\PYGZob{}}\PYG{n+nb}{len}\PYG{p}{(}\PYG{n}{letters}\PYG{p}{)}\PYG{l+s+si}{\PYGZcb{}}\PYG{l+s+s1}{\PYGZsq{}}\PYG{p}{)}
\PYG{n+nb}{print}\PYG{p}{(}\PYG{l+s+sa}{f}\PYG{l+s+s1}{\PYGZsq{}}\PYG{l+s+s1}{Maximum: }\PYG{l+s+si}{\PYGZob{}}\PYG{n+nb}{max}\PYG{p}{(}\PYG{n}{letters}\PYG{p}{)}\PYG{l+s+si}{\PYGZcb{}}\PYG{l+s+s1}{\PYGZsq{}}\PYG{p}{)}    \PYG{c+c1}{\PYGZsh{} dictionary order}
\PYG{n+nb}{print}\PYG{p}{(}\PYG{l+s+sa}{f}\PYG{l+s+s1}{\PYGZsq{}}\PYG{l+s+s1}{Minimum: }\PYG{l+s+si}{\PYGZob{}}\PYG{n+nb}{min}\PYG{p}{(}\PYG{n}{letters}\PYG{p}{)}\PYG{l+s+si}{\PYGZcb{}}\PYG{l+s+s1}{\PYGZsq{}}\PYG{p}{)}
\end{sphinxVerbatim}

\end{sphinxuseclass}\end{sphinxVerbatimInput}
\begin{sphinxVerbatimOutput}

\begin{sphinxuseclass}{cell_output}
\begin{sphinxVerbatim}[commandchars=\\\{\}]
Length : 5
Maximum: t
Minimum: a
\end{sphinxVerbatim}

\end{sphinxuseclass}\end{sphinxVerbatimOutput}

\end{sphinxuseclass}

\section{Set Operations}
\label{\detokenize{sets:set-operations}}
\sphinxAtStartPar
The available operators for sets in Python include \sphinxcode{\sphinxupquote{\&}} (intersection), \sphinxcode{\sphinxupquote{|}} (union), \sphinxcode{\sphinxupquote{\sphinxhyphen{}}} (difference), \sphinxcode{\sphinxupquote{\textasciicircum{}}} (symmetric difference), \sphinxcode{\sphinxupquote{in}}, and \sphinxcode{\sphinxupquote{not in}}.


\subsection{Intersection}
\label{\detokenize{sets:intersection}}
\sphinxAtStartPar
The \sphinxcode{\sphinxupquote{\&}} (ampersand) operator returns a new set consisting of the common elements of the two sets.

\begin{sphinxuseclass}{cell}\begin{sphinxVerbatimInput}

\begin{sphinxuseclass}{cell_input}
\begin{sphinxVerbatim}[commandchars=\\\{\}]
\PYG{n}{s1} \PYG{o}{=} \PYG{p}{\PYGZob{}}\PYG{l+m+mi}{1}\PYG{p}{,}\PYG{l+m+mi}{2}\PYG{p}{,}\PYG{l+m+mi}{3}\PYG{p}{,}\PYG{l+m+mi}{4}\PYG{p}{,}\PYG{l+m+mi}{5}\PYG{p}{\PYGZcb{}}
\PYG{n}{s2} \PYG{o}{=} \PYG{p}{\PYGZob{}}\PYG{l+m+mi}{3}\PYG{p}{,}\PYG{l+m+mi}{4}\PYG{p}{,}\PYG{l+m+mi}{5}\PYG{p}{,}\PYG{l+m+mi}{6}\PYG{p}{,}\PYG{l+m+mi}{7}\PYG{p}{\PYGZcb{}}

\PYG{n+nb}{print}\PYG{p}{(}\PYG{l+s+sa}{f}\PYG{l+s+s1}{\PYGZsq{}}\PYG{l+s+s1}{Intersection of s1 and s2: }\PYG{l+s+si}{\PYGZob{}}\PYG{n}{s1}\PYG{o}{\PYGZam{}}\PYG{n}{s2}\PYG{l+s+si}{\PYGZcb{}}\PYG{l+s+s1}{\PYGZsq{}}\PYG{p}{)}
\end{sphinxVerbatim}

\end{sphinxuseclass}\end{sphinxVerbatimInput}
\begin{sphinxVerbatimOutput}

\begin{sphinxuseclass}{cell_output}
\begin{sphinxVerbatim}[commandchars=\\\{\}]
Intersection of s1 and s2: \PYGZob{}3, 4, 5\PYGZcb{}
\end{sphinxVerbatim}

\end{sphinxuseclass}\end{sphinxVerbatimOutput}

\end{sphinxuseclass}\begin{itemize}
\item {} 
\sphinxAtStartPar
The \sphinxstyleemphasis{intersection()} method of sets can also be used to find the common elements between two sets.

\end{itemize}

\begin{sphinxuseclass}{cell}\begin{sphinxVerbatimInput}

\begin{sphinxuseclass}{cell_input}
\begin{sphinxVerbatim}[commandchars=\\\{\}]
\PYG{n}{s1} \PYG{o}{=} \PYG{p}{\PYGZob{}}\PYG{l+m+mi}{1}\PYG{p}{,}\PYG{l+m+mi}{2}\PYG{p}{,}\PYG{l+m+mi}{3}\PYG{p}{,}\PYG{l+m+mi}{4}\PYG{p}{,}\PYG{l+m+mi}{5}\PYG{p}{\PYGZcb{}}
\PYG{n}{s2} \PYG{o}{=} \PYG{p}{\PYGZob{}}\PYG{l+m+mi}{3}\PYG{p}{,}\PYG{l+m+mi}{4}\PYG{p}{,}\PYG{l+m+mi}{5}\PYG{p}{,}\PYG{l+m+mi}{6}\PYG{p}{,}\PYG{l+m+mi}{7}\PYG{p}{\PYGZcb{}}

\PYG{n+nb}{print}\PYG{p}{(}\PYG{l+s+sa}{f}\PYG{l+s+s1}{\PYGZsq{}}\PYG{l+s+s1}{s1 intersection s2 ; }\PYG{l+s+si}{\PYGZob{}}\PYG{n}{s1}\PYG{o}{.}\PYG{n}{intersection}\PYG{p}{(}\PYG{n}{s2}\PYG{p}{)}\PYG{l+s+si}{\PYGZcb{}}\PYG{l+s+s1}{\PYGZsq{}}\PYG{p}{)}
\PYG{n+nb}{print}\PYG{p}{(}\PYG{l+s+sa}{f}\PYG{l+s+s1}{\PYGZsq{}}\PYG{l+s+s1}{No change on s1    : }\PYG{l+s+si}{\PYGZob{}}\PYG{n}{s1}\PYG{l+s+si}{\PYGZcb{}}\PYG{l+s+s1}{\PYGZsq{}}\PYG{p}{)}
\end{sphinxVerbatim}

\end{sphinxuseclass}\end{sphinxVerbatimInput}
\begin{sphinxVerbatimOutput}

\begin{sphinxuseclass}{cell_output}
\begin{sphinxVerbatim}[commandchars=\\\{\}]
s1 intersection s2 ; \PYGZob{}3, 4, 5\PYGZcb{}
No change on s1    : \PYGZob{}1, 2, 3, 4, 5\PYGZcb{}
\end{sphinxVerbatim}

\end{sphinxuseclass}\end{sphinxVerbatimOutput}

\end{sphinxuseclass}

\subsection{Union}
\label{\detokenize{sets:union}}
\sphinxAtStartPar
The \sphinxcode{\sphinxupquote{|}} (pipe) operator returns a new set consisting of the combined elements of the two sets.

\begin{sphinxuseclass}{cell}\begin{sphinxVerbatimInput}

\begin{sphinxuseclass}{cell_input}
\begin{sphinxVerbatim}[commandchars=\\\{\}]
\PYG{n}{s1} \PYG{o}{=} \PYG{p}{\PYGZob{}}\PYG{l+m+mi}{1}\PYG{p}{,}\PYG{l+m+mi}{2}\PYG{p}{,}\PYG{l+m+mi}{3}\PYG{p}{,}\PYG{l+m+mi}{4}\PYG{p}{,}\PYG{l+m+mi}{5}\PYG{p}{\PYGZcb{}}
\PYG{n}{s2} \PYG{o}{=} \PYG{p}{\PYGZob{}}\PYG{l+m+mi}{3}\PYG{p}{,}\PYG{l+m+mi}{4}\PYG{p}{,}\PYG{l+m+mi}{5}\PYG{p}{,}\PYG{l+m+mi}{6}\PYG{p}{,}\PYG{l+m+mi}{7}\PYG{p}{\PYGZcb{}}

\PYG{n+nb}{print}\PYG{p}{(}\PYG{l+s+sa}{f}\PYG{l+s+s1}{\PYGZsq{}}\PYG{l+s+s1}{Union of s1 and s2: }\PYG{l+s+si}{\PYGZob{}}\PYG{n}{s1}\PYG{o}{|}\PYG{n}{s2}\PYG{l+s+si}{\PYGZcb{}}\PYG{l+s+s1}{\PYGZsq{}}\PYG{p}{)}
\end{sphinxVerbatim}

\end{sphinxuseclass}\end{sphinxVerbatimInput}
\begin{sphinxVerbatimOutput}

\begin{sphinxuseclass}{cell_output}
\begin{sphinxVerbatim}[commandchars=\\\{\}]
Union of s1 and s2: \PYGZob{}1, 2, 3, 4, 5, 6, 7\PYGZcb{}
\end{sphinxVerbatim}

\end{sphinxuseclass}\end{sphinxVerbatimOutput}

\end{sphinxuseclass}\begin{itemize}
\item {} 
\sphinxAtStartPar
The \sphinxstyleemphasis{union()} method of sets can also be used to combine elements from two sets into a new set.

\end{itemize}

\begin{sphinxuseclass}{cell}\begin{sphinxVerbatimInput}

\begin{sphinxuseclass}{cell_input}
\begin{sphinxVerbatim}[commandchars=\\\{\}]
\PYG{n}{s1} \PYG{o}{=} \PYG{p}{\PYGZob{}}\PYG{l+m+mi}{1}\PYG{p}{,}\PYG{l+m+mi}{2}\PYG{p}{,}\PYG{l+m+mi}{3}\PYG{p}{,}\PYG{l+m+mi}{4}\PYG{p}{,}\PYG{l+m+mi}{5}\PYG{p}{\PYGZcb{}}
\PYG{n}{s2} \PYG{o}{=} \PYG{p}{\PYGZob{}}\PYG{l+m+mi}{3}\PYG{p}{,}\PYG{l+m+mi}{4}\PYG{p}{,}\PYG{l+m+mi}{5}\PYG{p}{,}\PYG{l+m+mi}{6}\PYG{p}{,}\PYG{l+m+mi}{7}\PYG{p}{\PYGZcb{}}

\PYG{n+nb}{print}\PYG{p}{(}\PYG{l+s+sa}{f}\PYG{l+s+s1}{\PYGZsq{}}\PYG{l+s+s1}{s1 union s2; }\PYG{l+s+si}{\PYGZob{}}\PYG{n}{s1}\PYG{o}{.}\PYG{n}{union}\PYG{p}{(}\PYG{n}{s2}\PYG{p}{)}\PYG{l+s+si}{\PYGZcb{}}\PYG{l+s+s1}{\PYGZsq{}}\PYG{p}{)}
\PYG{n+nb}{print}\PYG{p}{(}\PYG{l+s+sa}{f}\PYG{l+s+s1}{\PYGZsq{}}\PYG{l+s+s1}{s1         : }\PYG{l+s+si}{\PYGZob{}}\PYG{n}{s1}\PYG{l+s+si}{\PYGZcb{}}\PYG{l+s+s1}{\PYGZsq{}}\PYG{p}{)}               \PYG{c+c1}{\PYGZsh{} No change on s1}
\end{sphinxVerbatim}

\end{sphinxuseclass}\end{sphinxVerbatimInput}
\begin{sphinxVerbatimOutput}

\begin{sphinxuseclass}{cell_output}
\begin{sphinxVerbatim}[commandchars=\\\{\}]
s1 union s2; \PYGZob{}1, 2, 3, 4, 5, 6, 7\PYGZcb{}
s1         : \PYGZob{}1, 2, 3, 4, 5\PYGZcb{}
\end{sphinxVerbatim}

\end{sphinxuseclass}\end{sphinxVerbatimOutput}

\end{sphinxuseclass}

\subsection{Difference}
\label{\detokenize{sets:difference}}
\sphinxAtStartPar
The \sphinxcode{\sphinxupquote{\sphinxhyphen{}}} (dash) operator returns a new set consisting of the elements in the first set but not in the second set.
\begin{itemize}
\item {} 
\sphinxAtStartPar
\sphinxcode{\sphinxupquote{s1 \sphinxhyphen{} s2}}: elements in \sphinxstyleemphasis{s1} but not in \sphinxstyleemphasis{s2}.

\item {} 
\sphinxAtStartPar
\sphinxcode{\sphinxupquote{s2 \sphinxhyphen{} s1}}: elements in \sphinxstyleemphasis{s2} but not in \sphinxstyleemphasis{s1}.

\end{itemize}

\begin{sphinxuseclass}{cell}\begin{sphinxVerbatimInput}

\begin{sphinxuseclass}{cell_input}
\begin{sphinxVerbatim}[commandchars=\\\{\}]
\PYG{n}{s1} \PYG{o}{=} \PYG{p}{\PYGZob{}}\PYG{l+m+mi}{1}\PYG{p}{,}\PYG{l+m+mi}{2}\PYG{p}{,}\PYG{l+m+mi}{3}\PYG{p}{,}\PYG{l+m+mi}{4}\PYG{p}{,}\PYG{l+m+mi}{5}\PYG{p}{\PYGZcb{}}
\PYG{n}{s2} \PYG{o}{=} \PYG{p}{\PYGZob{}}\PYG{l+m+mi}{3}\PYG{p}{,}\PYG{l+m+mi}{4}\PYG{p}{,}\PYG{l+m+mi}{5}\PYG{p}{,}\PYG{l+m+mi}{6}\PYG{p}{,}\PYG{l+m+mi}{7}\PYG{p}{\PYGZcb{}}

\PYG{n+nb}{print}\PYG{p}{(}\PYG{l+s+sa}{f}\PYG{l+s+s1}{\PYGZsq{}}\PYG{l+s+s1}{s1 \PYGZhy{} s2: }\PYG{l+s+si}{\PYGZob{}}\PYG{n}{s1}\PYG{o}{\PYGZhy{}}\PYG{n}{s2}\PYG{l+s+si}{\PYGZcb{}}\PYG{l+s+s1}{\PYGZsq{}}\PYG{p}{)}
\PYG{n+nb}{print}\PYG{p}{(}\PYG{l+s+sa}{f}\PYG{l+s+s1}{\PYGZsq{}}\PYG{l+s+s1}{s2 \PYGZhy{} s1: }\PYG{l+s+si}{\PYGZob{}}\PYG{n}{s2}\PYG{o}{\PYGZhy{}}\PYG{n}{s1}\PYG{l+s+si}{\PYGZcb{}}\PYG{l+s+s1}{\PYGZsq{}}\PYG{p}{)}
\end{sphinxVerbatim}

\end{sphinxuseclass}\end{sphinxVerbatimInput}
\begin{sphinxVerbatimOutput}

\begin{sphinxuseclass}{cell_output}
\begin{sphinxVerbatim}[commandchars=\\\{\}]
s1 \PYGZhy{} s2: \PYGZob{}1, 2\PYGZcb{}
s2 \PYGZhy{} s1: \PYGZob{}6, 7\PYGZcb{}
\end{sphinxVerbatim}

\end{sphinxuseclass}\end{sphinxVerbatimOutput}

\end{sphinxuseclass}\begin{itemize}
\item {} 
\sphinxAtStartPar
The difference() method of sets can also be used to obtain a new set consisting of the elements in the first set but not in the second set.

\end{itemize}

\begin{sphinxuseclass}{cell}\begin{sphinxVerbatimInput}

\begin{sphinxuseclass}{cell_input}
\begin{sphinxVerbatim}[commandchars=\\\{\}]
\PYG{n}{s1} \PYG{o}{=} \PYG{p}{\PYGZob{}}\PYG{l+m+mi}{1}\PYG{p}{,}\PYG{l+m+mi}{2}\PYG{p}{,}\PYG{l+m+mi}{3}\PYG{p}{,}\PYG{l+m+mi}{4}\PYG{p}{,}\PYG{l+m+mi}{5}\PYG{p}{\PYGZcb{}}
\PYG{n}{s2} \PYG{o}{=} \PYG{p}{\PYGZob{}}\PYG{l+m+mi}{3}\PYG{p}{,}\PYG{l+m+mi}{4}\PYG{p}{,}\PYG{l+m+mi}{5}\PYG{p}{,}\PYG{l+m+mi}{6}\PYG{p}{,}\PYG{l+m+mi}{7}\PYG{p}{\PYGZcb{}}

\PYG{n+nb}{print}\PYG{p}{(}\PYG{l+s+sa}{f}\PYG{l+s+s1}{\PYGZsq{}}\PYG{l+s+s1}{s1 \PYGZhy{} s2: }\PYG{l+s+si}{\PYGZob{}}\PYG{n}{s1}\PYG{o}{.}\PYG{n}{difference}\PYG{p}{(}\PYG{n}{s2}\PYG{p}{)}\PYG{l+s+si}{\PYGZcb{}}\PYG{l+s+s1}{\PYGZsq{}}\PYG{p}{)}
\PYG{n+nb}{print}\PYG{p}{(}\PYG{l+s+sa}{f}\PYG{l+s+s1}{\PYGZsq{}}\PYG{l+s+s1}{s2 \PYGZhy{} s1: }\PYG{l+s+si}{\PYGZob{}}\PYG{n}{s2}\PYG{o}{.}\PYG{n}{difference}\PYG{p}{(}\PYG{n}{s1}\PYG{p}{)}\PYG{l+s+si}{\PYGZcb{}}\PYG{l+s+s1}{\PYGZsq{}}\PYG{p}{)}
\end{sphinxVerbatim}

\end{sphinxuseclass}\end{sphinxVerbatimInput}
\begin{sphinxVerbatimOutput}

\begin{sphinxuseclass}{cell_output}
\begin{sphinxVerbatim}[commandchars=\\\{\}]
s1 \PYGZhy{} s2: \PYGZob{}1, 2\PYGZcb{}
s2 \PYGZhy{} s1: \PYGZob{}6, 7\PYGZcb{}
\end{sphinxVerbatim}

\end{sphinxuseclass}\end{sphinxVerbatimOutput}

\end{sphinxuseclass}

\subsection{Symmetric Difference}
\label{\detokenize{sets:symmetric-difference}}
\sphinxAtStartPar
The \textasciicircum{} (caret) operator returns a new set consisting of elements in either one of the two sets but not both.
\begin{itemize}
\item {} 
\sphinxAtStartPar
\sphinxcode{\sphinxupquote{s1 \textasciicircum{} s2}} and  \sphinxcode{\sphinxupquote{s2 \textasciicircum{} s1}} are same.

\end{itemize}

\begin{sphinxuseclass}{cell}\begin{sphinxVerbatimInput}

\begin{sphinxuseclass}{cell_input}
\begin{sphinxVerbatim}[commandchars=\\\{\}]
\PYG{n}{s1} \PYG{o}{=} \PYG{p}{\PYGZob{}}\PYG{l+m+mi}{1}\PYG{p}{,}\PYG{l+m+mi}{2}\PYG{p}{,}\PYG{l+m+mi}{3}\PYG{p}{,}\PYG{l+m+mi}{4}\PYG{p}{,}\PYG{l+m+mi}{5}\PYG{p}{\PYGZcb{}}
\PYG{n}{s2} \PYG{o}{=} \PYG{p}{\PYGZob{}}\PYG{l+m+mi}{3}\PYG{p}{,}\PYG{l+m+mi}{4}\PYG{p}{,}\PYG{l+m+mi}{5}\PYG{p}{,}\PYG{l+m+mi}{6}\PYG{p}{,}\PYG{l+m+mi}{7}\PYG{p}{\PYGZcb{}}

\PYG{n+nb}{print}\PYG{p}{(}\PYG{l+s+sa}{f}\PYG{l+s+s1}{\PYGZsq{}}\PYG{l+s+s1}{s1 \PYGZca{} s2: }\PYG{l+s+si}{\PYGZob{}}\PYG{n}{s1}\PYG{o}{\PYGZca{}}\PYG{n}{s2}\PYG{l+s+si}{\PYGZcb{}}\PYG{l+s+s1}{\PYGZsq{}}\PYG{p}{)}
\PYG{n+nb}{print}\PYG{p}{(}\PYG{l+s+sa}{f}\PYG{l+s+s1}{\PYGZsq{}}\PYG{l+s+s1}{s2 \PYGZca{} s1: }\PYG{l+s+si}{\PYGZob{}}\PYG{n}{s2}\PYG{o}{\PYGZca{}}\PYG{n}{s1}\PYG{l+s+si}{\PYGZcb{}}\PYG{l+s+s1}{\PYGZsq{}}\PYG{p}{)}
\end{sphinxVerbatim}

\end{sphinxuseclass}\end{sphinxVerbatimInput}
\begin{sphinxVerbatimOutput}

\begin{sphinxuseclass}{cell_output}
\begin{sphinxVerbatim}[commandchars=\\\{\}]
s1 \PYGZca{} s2: \PYGZob{}1, 2, 6, 7\PYGZcb{}
s2 \PYGZca{} s1: \PYGZob{}1, 2, 6, 7\PYGZcb{}
\end{sphinxVerbatim}

\end{sphinxuseclass}\end{sphinxVerbatimOutput}

\end{sphinxuseclass}\begin{itemize}
\item {} 
\sphinxAtStartPar
The symmetric\_difference() method of sets can also be used to obtain a new set consisting of the elements in either one of the two sets but not both.

\end{itemize}

\begin{sphinxuseclass}{cell}\begin{sphinxVerbatimInput}

\begin{sphinxuseclass}{cell_input}
\begin{sphinxVerbatim}[commandchars=\\\{\}]
\PYG{n}{s1} \PYG{o}{=} \PYG{p}{\PYGZob{}}\PYG{l+m+mi}{1}\PYG{p}{,}\PYG{l+m+mi}{2}\PYG{p}{,}\PYG{l+m+mi}{3}\PYG{p}{,}\PYG{l+m+mi}{4}\PYG{p}{,}\PYG{l+m+mi}{5}\PYG{p}{\PYGZcb{}}
\PYG{n}{s2} \PYG{o}{=} \PYG{p}{\PYGZob{}}\PYG{l+m+mi}{3}\PYG{p}{,}\PYG{l+m+mi}{4}\PYG{p}{,}\PYG{l+m+mi}{5}\PYG{p}{,}\PYG{l+m+mi}{6}\PYG{p}{,}\PYG{l+m+mi}{7}\PYG{p}{\PYGZcb{}}

\PYG{n+nb}{print}\PYG{p}{(}\PYG{l+s+sa}{f}\PYG{l+s+s1}{\PYGZsq{}}\PYG{l+s+s1}{s1 \PYGZhy{} s2: }\PYG{l+s+si}{\PYGZob{}}\PYG{n}{s1}\PYG{o}{.}\PYG{n}{symmetric\PYGZus{}difference}\PYG{p}{(}\PYG{n}{s2}\PYG{p}{)}\PYG{l+s+si}{\PYGZcb{}}\PYG{l+s+s1}{\PYGZsq{}}\PYG{p}{)}
\PYG{n+nb}{print}\PYG{p}{(}\PYG{l+s+sa}{f}\PYG{l+s+s1}{\PYGZsq{}}\PYG{l+s+s1}{s2 \PYGZhy{} s1: }\PYG{l+s+si}{\PYGZob{}}\PYG{n}{s2}\PYG{o}{.}\PYG{n}{symmetric\PYGZus{}difference}\PYG{p}{(}\PYG{n}{s1}\PYG{p}{)}\PYG{l+s+si}{\PYGZcb{}}\PYG{l+s+s1}{\PYGZsq{}}\PYG{p}{)}
\end{sphinxVerbatim}

\end{sphinxuseclass}\end{sphinxVerbatimInput}
\begin{sphinxVerbatimOutput}

\begin{sphinxuseclass}{cell_output}
\begin{sphinxVerbatim}[commandchars=\\\{\}]
s1 \PYGZhy{} s2: \PYGZob{}1, 2, 6, 7\PYGZcb{}
s2 \PYGZhy{} s1: \PYGZob{}1, 2, 6, 7\PYGZcb{}
\end{sphinxVerbatim}

\end{sphinxuseclass}\end{sphinxVerbatimOutput}

\end{sphinxuseclass}

\subsection{in \& not in}
\label{\detokenize{sets:in-not-in}}
\sphinxAtStartPar
\sphinxcode{\sphinxupquote{in}} checks if a value is part of a set, while \sphinxcode{\sphinxupquote{not in}} verifies if a value is absent from a set.
\begin{itemize}
\item {} 
\sphinxAtStartPar
Both operations yield a Boolean result, either True or False.

\end{itemize}

\begin{sphinxuseclass}{cell}\begin{sphinxVerbatimInput}

\begin{sphinxuseclass}{cell_input}
\begin{sphinxVerbatim}[commandchars=\\\{\}]
\PYG{n}{s1} \PYG{o}{=} \PYG{p}{\PYGZob{}}\PYG{l+m+mi}{1}\PYG{p}{,}\PYG{l+m+mi}{2}\PYG{p}{,}\PYG{l+m+mi}{3}\PYG{p}{,}\PYG{l+m+mi}{4}\PYG{p}{,}\PYG{l+m+mi}{5}\PYG{p}{\PYGZcb{}}

\PYG{n+nb}{print}\PYG{p}{(}\PYG{l+s+sa}{f}\PYG{l+s+s1}{\PYGZsq{}}\PYG{l+s+s1}{ 5 is in s1    : }\PYG{l+s+si}{\PYGZob{}}\PYG{l+m+mi}{5}\PYG{+w}{  }\PYG{o+ow}{in}\PYG{+w}{ }\PYG{n}{s1}\PYG{l+s+si}{\PYGZcb{}}\PYG{l+s+s1}{\PYGZsq{}} \PYG{p}{)}
\PYG{n+nb}{print}\PYG{p}{(}\PYG{l+s+sa}{f}\PYG{l+s+s1}{\PYGZsq{}}\PYG{l+s+s1}{ 5 is not in s1: }\PYG{l+s+si}{\PYGZob{}}\PYG{l+m+mi}{5}\PYG{+w}{  }\PYG{o+ow}{not}\PYG{+w}{ }\PYG{o+ow}{in}\PYG{+w}{ }\PYG{n}{s1}\PYG{l+s+si}{\PYGZcb{}}\PYG{l+s+s1}{\PYGZsq{}} \PYG{p}{)}
\end{sphinxVerbatim}

\end{sphinxuseclass}\end{sphinxVerbatimInput}
\begin{sphinxVerbatimOutput}

\begin{sphinxuseclass}{cell_output}
\begin{sphinxVerbatim}[commandchars=\\\{\}]
 5 is in s1    : True
 5 is not in s1: False
\end{sphinxVerbatim}

\end{sphinxuseclass}\end{sphinxVerbatimOutput}

\end{sphinxuseclass}
\begin{sphinxuseclass}{cell}\begin{sphinxVerbatimInput}

\begin{sphinxuseclass}{cell_input}
\begin{sphinxVerbatim}[commandchars=\\\{\}]
\PYG{n}{s1} \PYG{o}{=} \PYG{p}{\PYGZob{}}\PYG{l+m+mi}{1}\PYG{p}{,}\PYG{l+m+mi}{2}\PYG{p}{,}\PYG{l+m+mi}{3}\PYG{p}{,}\PYG{l+m+mi}{4}\PYG{p}{,}\PYG{l+m+mi}{5}\PYG{p}{\PYGZcb{}}

\PYG{n+nb}{print}\PYG{p}{(}\PYG{l+s+sa}{f}\PYG{l+s+s1}{\PYGZsq{}}\PYG{l+s+s1}{ 9 is in s1    : }\PYG{l+s+si}{\PYGZob{}}\PYG{l+m+mi}{9}\PYG{+w}{  }\PYG{o+ow}{in}\PYG{+w}{ }\PYG{n}{s1}\PYG{l+s+si}{\PYGZcb{}}\PYG{l+s+s1}{\PYGZsq{}} \PYG{p}{)}
\PYG{n+nb}{print}\PYG{p}{(}\PYG{l+s+sa}{f}\PYG{l+s+s1}{\PYGZsq{}}\PYG{l+s+s1}{ 9 is not in s1: }\PYG{l+s+si}{\PYGZob{}}\PYG{l+m+mi}{9}\PYG{+w}{  }\PYG{o+ow}{not}\PYG{+w}{ }\PYG{o+ow}{in}\PYG{+w}{ }\PYG{n}{s1}\PYG{l+s+si}{\PYGZcb{}}\PYG{l+s+s1}{\PYGZsq{}} \PYG{p}{)}
\end{sphinxVerbatim}

\end{sphinxuseclass}\end{sphinxVerbatimInput}
\begin{sphinxVerbatimOutput}

\begin{sphinxuseclass}{cell_output}
\begin{sphinxVerbatim}[commandchars=\\\{\}]
 9 is in s1    : False
 9 is not in s1: True
\end{sphinxVerbatim}

\end{sphinxuseclass}\end{sphinxVerbatimOutput}

\end{sphinxuseclass}

\subsection{Mutable}
\label{\detokenize{sets:mutable}}
\sphinxAtStartPar
Unlike strings and tuples, and similar to lists, sets are mutable, allowing them to be modified.
\begin{itemize}
\item {} 
\sphinxAtStartPar
The set methods enable the addition of new elements and the removal of existing ones.

\item {} 
\sphinxAtStartPar
Set elements themselves cannot be changed, but new items can be added or existing ones removed.

\end{itemize}


\section{Set Methods}
\label{\detokenize{sets:set-methods}}
\sphinxAtStartPar
Except for the magic methods (those with underscores), there are 17 methods for sets.
\begin{itemize}
\item {} 
\sphinxAtStartPar
We have already covered intersection, union, difference, and symmetric difference above.

\item {} 
\sphinxAtStartPar
You can execute help(set) for more details.

\end{itemize}

\begin{sphinxuseclass}{cell}\begin{sphinxVerbatimInput}

\begin{sphinxuseclass}{cell_input}
\begin{sphinxVerbatim}[commandchars=\\\{\}]
\PYG{c+c1}{\PYGZsh{} methods of sets}
\PYG{c+c1}{\PYGZsh{} dir() returns a list}

\PYG{n+nb}{print}\PYG{p}{(}\PYG{n+nb}{dir}\PYG{p}{(}\PYG{n+nb}{set}\PYG{p}{)}\PYG{p}{)}
\end{sphinxVerbatim}

\end{sphinxuseclass}\end{sphinxVerbatimInput}
\begin{sphinxVerbatimOutput}

\begin{sphinxuseclass}{cell_output}
\begin{sphinxVerbatim}[commandchars=\\\{\}]
[\PYGZsq{}\PYGZus{}\PYGZus{}and\PYGZus{}\PYGZus{}\PYGZsq{}, \PYGZsq{}\PYGZus{}\PYGZus{}class\PYGZus{}\PYGZus{}\PYGZsq{}, \PYGZsq{}\PYGZus{}\PYGZus{}class\PYGZus{}getitem\PYGZus{}\PYGZus{}\PYGZsq{}, \PYGZsq{}\PYGZus{}\PYGZus{}contains\PYGZus{}\PYGZus{}\PYGZsq{}, \PYGZsq{}\PYGZus{}\PYGZus{}delattr\PYGZus{}\PYGZus{}\PYGZsq{}, \PYGZsq{}\PYGZus{}\PYGZus{}dir\PYGZus{}\PYGZus{}\PYGZsq{}, \PYGZsq{}\PYGZus{}\PYGZus{}doc\PYGZus{}\PYGZus{}\PYGZsq{}, \PYGZsq{}\PYGZus{}\PYGZus{}eq\PYGZus{}\PYGZus{}\PYGZsq{}, \PYGZsq{}\PYGZus{}\PYGZus{}format\PYGZus{}\PYGZus{}\PYGZsq{}, \PYGZsq{}\PYGZus{}\PYGZus{}ge\PYGZus{}\PYGZus{}\PYGZsq{}, \PYGZsq{}\PYGZus{}\PYGZus{}getattribute\PYGZus{}\PYGZus{}\PYGZsq{}, \PYGZsq{}\PYGZus{}\PYGZus{}getstate\PYGZus{}\PYGZus{}\PYGZsq{}, \PYGZsq{}\PYGZus{}\PYGZus{}gt\PYGZus{}\PYGZus{}\PYGZsq{}, \PYGZsq{}\PYGZus{}\PYGZus{}hash\PYGZus{}\PYGZus{}\PYGZsq{}, \PYGZsq{}\PYGZus{}\PYGZus{}iand\PYGZus{}\PYGZus{}\PYGZsq{}, \PYGZsq{}\PYGZus{}\PYGZus{}init\PYGZus{}\PYGZus{}\PYGZsq{}, \PYGZsq{}\PYGZus{}\PYGZus{}init\PYGZus{}subclass\PYGZus{}\PYGZus{}\PYGZsq{}, \PYGZsq{}\PYGZus{}\PYGZus{}ior\PYGZus{}\PYGZus{}\PYGZsq{}, \PYGZsq{}\PYGZus{}\PYGZus{}isub\PYGZus{}\PYGZus{}\PYGZsq{}, \PYGZsq{}\PYGZus{}\PYGZus{}iter\PYGZus{}\PYGZus{}\PYGZsq{}, \PYGZsq{}\PYGZus{}\PYGZus{}ixor\PYGZus{}\PYGZus{}\PYGZsq{}, \PYGZsq{}\PYGZus{}\PYGZus{}le\PYGZus{}\PYGZus{}\PYGZsq{}, \PYGZsq{}\PYGZus{}\PYGZus{}len\PYGZus{}\PYGZus{}\PYGZsq{}, \PYGZsq{}\PYGZus{}\PYGZus{}lt\PYGZus{}\PYGZus{}\PYGZsq{}, \PYGZsq{}\PYGZus{}\PYGZus{}ne\PYGZus{}\PYGZus{}\PYGZsq{}, \PYGZsq{}\PYGZus{}\PYGZus{}new\PYGZus{}\PYGZus{}\PYGZsq{}, \PYGZsq{}\PYGZus{}\PYGZus{}or\PYGZus{}\PYGZus{}\PYGZsq{}, \PYGZsq{}\PYGZus{}\PYGZus{}rand\PYGZus{}\PYGZus{}\PYGZsq{}, \PYGZsq{}\PYGZus{}\PYGZus{}reduce\PYGZus{}\PYGZus{}\PYGZsq{}, \PYGZsq{}\PYGZus{}\PYGZus{}reduce\PYGZus{}ex\PYGZus{}\PYGZus{}\PYGZsq{}, \PYGZsq{}\PYGZus{}\PYGZus{}repr\PYGZus{}\PYGZus{}\PYGZsq{}, \PYGZsq{}\PYGZus{}\PYGZus{}ror\PYGZus{}\PYGZus{}\PYGZsq{}, \PYGZsq{}\PYGZus{}\PYGZus{}rsub\PYGZus{}\PYGZus{}\PYGZsq{}, \PYGZsq{}\PYGZus{}\PYGZus{}rxor\PYGZus{}\PYGZus{}\PYGZsq{}, \PYGZsq{}\PYGZus{}\PYGZus{}setattr\PYGZus{}\PYGZus{}\PYGZsq{}, \PYGZsq{}\PYGZus{}\PYGZus{}sizeof\PYGZus{}\PYGZus{}\PYGZsq{}, \PYGZsq{}\PYGZus{}\PYGZus{}str\PYGZus{}\PYGZus{}\PYGZsq{}, \PYGZsq{}\PYGZus{}\PYGZus{}sub\PYGZus{}\PYGZus{}\PYGZsq{}, \PYGZsq{}\PYGZus{}\PYGZus{}subclasshook\PYGZus{}\PYGZus{}\PYGZsq{}, \PYGZsq{}\PYGZus{}\PYGZus{}xor\PYGZus{}\PYGZus{}\PYGZsq{}, \PYGZsq{}add\PYGZsq{}, \PYGZsq{}clear\PYGZsq{}, \PYGZsq{}copy\PYGZsq{}, \PYGZsq{}difference\PYGZsq{}, \PYGZsq{}difference\PYGZus{}update\PYGZsq{}, \PYGZsq{}discard\PYGZsq{}, \PYGZsq{}intersection\PYGZsq{}, \PYGZsq{}intersection\PYGZus{}update\PYGZsq{}, \PYGZsq{}isdisjoint\PYGZsq{}, \PYGZsq{}issubset\PYGZsq{}, \PYGZsq{}issuperset\PYGZsq{}, \PYGZsq{}pop\PYGZsq{}, \PYGZsq{}remove\PYGZsq{}, \PYGZsq{}symmetric\PYGZus{}difference\PYGZsq{}, \PYGZsq{}symmetric\PYGZus{}difference\PYGZus{}update\PYGZsq{}, \PYGZsq{}union\PYGZsq{}, \PYGZsq{}update\PYGZsq{}]
\end{sphinxVerbatim}

\end{sphinxuseclass}\end{sphinxVerbatimOutput}

\end{sphinxuseclass}
\begin{sphinxuseclass}{cell}\begin{sphinxVerbatimInput}

\begin{sphinxuseclass}{cell_input}
\begin{sphinxVerbatim}[commandchars=\\\{\}]
\PYG{n+nb}{print}\PYG{p}{(}\PYG{n+nb}{dir}\PYG{p}{(}\PYG{n+nb}{set}\PYG{p}{)}\PYG{p}{[}\PYG{o}{\PYGZhy{}}\PYG{l+m+mi}{17}\PYG{p}{:}\PYG{p}{]}\PYG{p}{)}
\end{sphinxVerbatim}

\end{sphinxuseclass}\end{sphinxVerbatimInput}
\begin{sphinxVerbatimOutput}

\begin{sphinxuseclass}{cell_output}
\begin{sphinxVerbatim}[commandchars=\\\{\}]
[\PYGZsq{}add\PYGZsq{}, \PYGZsq{}clear\PYGZsq{}, \PYGZsq{}copy\PYGZsq{}, \PYGZsq{}difference\PYGZsq{}, \PYGZsq{}difference\PYGZus{}update\PYGZsq{}, \PYGZsq{}discard\PYGZsq{}, \PYGZsq{}intersection\PYGZsq{}, \PYGZsq{}intersection\PYGZus{}update\PYGZsq{}, \PYGZsq{}isdisjoint\PYGZsq{}, \PYGZsq{}issubset\PYGZsq{}, \PYGZsq{}issuperset\PYGZsq{}, \PYGZsq{}pop\PYGZsq{}, \PYGZsq{}remove\PYGZsq{}, \PYGZsq{}symmetric\PYGZus{}difference\PYGZsq{}, \PYGZsq{}symmetric\PYGZus{}difference\PYGZus{}update\PYGZsq{}, \PYGZsq{}union\PYGZsq{}, \PYGZsq{}update\PYGZsq{}]
\end{sphinxVerbatim}

\end{sphinxuseclass}\end{sphinxVerbatimOutput}

\end{sphinxuseclass}

\subsection{add()}
\label{\detokenize{sets:add}}
\sphinxAtStartPar
It adds a new element to a set.

\begin{sphinxuseclass}{cell}\begin{sphinxVerbatimInput}

\begin{sphinxuseclass}{cell_input}
\begin{sphinxVerbatim}[commandchars=\\\{\}]
\PYG{n}{s} \PYG{o}{=} \PYG{p}{\PYGZob{}}\PYG{l+m+mi}{1}\PYG{p}{,}\PYG{l+m+mi}{2}\PYG{p}{,}\PYG{l+m+mi}{3}\PYG{p}{,}\PYG{l+m+mi}{4}\PYG{p}{,}\PYG{l+m+mi}{5}\PYG{p}{\PYGZcb{}}
\PYG{n+nb}{print}\PYG{p}{(}\PYG{l+s+sa}{f}\PYG{l+s+s1}{\PYGZsq{}}\PYG{l+s+s1}{set s before using add(): }\PYG{l+s+si}{\PYGZob{}}\PYG{n}{s}\PYG{l+s+si}{\PYGZcb{}}\PYG{l+s+s1}{\PYGZsq{}}\PYG{p}{)}

\PYG{c+c1}{\PYGZsh{} add 99}
\PYG{n}{s}\PYG{o}{.}\PYG{n}{add}\PYG{p}{(}\PYG{l+m+mi}{99}\PYG{p}{)}   

\PYG{n+nb}{print}\PYG{p}{(}\PYG{l+s+sa}{f}\PYG{l+s+s1}{\PYGZsq{}}\PYG{l+s+s1}{set s after using add() : }\PYG{l+s+si}{\PYGZob{}}\PYG{n}{s}\PYG{l+s+si}{\PYGZcb{}}\PYG{l+s+s1}{\PYGZsq{}}\PYG{p}{)}
\end{sphinxVerbatim}

\end{sphinxuseclass}\end{sphinxVerbatimInput}
\begin{sphinxVerbatimOutput}

\begin{sphinxuseclass}{cell_output}
\begin{sphinxVerbatim}[commandchars=\\\{\}]
set s before using add(): \PYGZob{}1, 2, 3, 4, 5\PYGZcb{}
set s after using add() : \PYGZob{}1, 2, 3, 4, 5, 99\PYGZcb{}
\end{sphinxVerbatim}

\end{sphinxuseclass}\end{sphinxVerbatimOutput}

\end{sphinxuseclass}

\subsection{clear()}
\label{\detokenize{sets:clear}}
\sphinxAtStartPar
It removes all elements from the set, making it an empty set.

\begin{sphinxuseclass}{cell}\begin{sphinxVerbatimInput}

\begin{sphinxuseclass}{cell_input}
\begin{sphinxVerbatim}[commandchars=\\\{\}]
\PYG{n}{s} \PYG{o}{=} \PYG{p}{\PYGZob{}}\PYG{l+m+mi}{1}\PYG{p}{,}\PYG{l+m+mi}{2}\PYG{p}{,}\PYG{l+m+mi}{3}\PYG{p}{,}\PYG{l+m+mi}{4}\PYG{p}{,}\PYG{l+m+mi}{5}\PYG{p}{\PYGZcb{}}
\PYG{n+nb}{print}\PYG{p}{(}\PYG{l+s+sa}{f}\PYG{l+s+s1}{\PYGZsq{}}\PYG{l+s+s1}{set s before using clear(): }\PYG{l+s+si}{\PYGZob{}}\PYG{n}{s}\PYG{l+s+si}{\PYGZcb{}}\PYG{l+s+s1}{\PYGZsq{}}\PYG{p}{)}

\PYG{c+c1}{\PYGZsh{} removes all elements}
\PYG{n}{s}\PYG{o}{.}\PYG{n}{clear}\PYG{p}{(}\PYG{p}{)}   

\PYG{n+nb}{print}\PYG{p}{(}\PYG{l+s+sa}{f}\PYG{l+s+s1}{\PYGZsq{}}\PYG{l+s+s1}{set s after using clear() : }\PYG{l+s+si}{\PYGZob{}}\PYG{n}{s}\PYG{l+s+si}{\PYGZcb{}}\PYG{l+s+s1}{\PYGZsq{}}\PYG{p}{)}
\end{sphinxVerbatim}

\end{sphinxuseclass}\end{sphinxVerbatimInput}
\begin{sphinxVerbatimOutput}

\begin{sphinxuseclass}{cell_output}
\begin{sphinxVerbatim}[commandchars=\\\{\}]
set s before using clear(): \PYGZob{}1, 2, 3, 4, 5\PYGZcb{}
set s after using clear() : set()
\end{sphinxVerbatim}

\end{sphinxuseclass}\end{sphinxVerbatimOutput}

\end{sphinxuseclass}

\subsection{copy()}
\label{\detokenize{sets:copy}}
\sphinxAtStartPar
It returns a new set with the same elements.

\begin{sphinxuseclass}{cell}\begin{sphinxVerbatimInput}

\begin{sphinxuseclass}{cell_input}
\begin{sphinxVerbatim}[commandchars=\\\{\}]
\PYG{n}{s} \PYG{o}{=} \PYG{p}{\PYGZob{}}\PYG{l+m+mi}{1}\PYG{p}{,}\PYG{l+m+mi}{2}\PYG{p}{,}\PYG{l+m+mi}{3}\PYG{p}{,}\PYG{l+m+mi}{4}\PYG{p}{,}\PYG{l+m+mi}{5}\PYG{p}{\PYGZcb{}}
\PYG{n+nb}{print}\PYG{p}{(}\PYG{l+s+sa}{f}\PYG{l+s+s1}{\PYGZsq{}}\PYG{l+s+s1}{set s before using copy(): }\PYG{l+s+si}{\PYGZob{}}\PYG{n}{s}\PYG{l+s+si}{\PYGZcb{}}\PYG{l+s+s1}{\PYGZsq{}}\PYG{p}{)}

\PYG{c+c1}{\PYGZsh{} make a copy}
\PYG{n}{s\PYGZus{}copy} \PYG{o}{=} \PYG{n}{s1}\PYG{o}{.}\PYG{n}{copy}\PYG{p}{(}\PYG{p}{)}   

\PYG{n+nb}{print}\PYG{p}{(}\PYG{l+s+sa}{f}\PYG{l+s+s1}{\PYGZsq{}}\PYG{l+s+s1}{set s after using copy() : }\PYG{l+s+si}{\PYGZob{}}\PYG{n}{s}\PYG{l+s+si}{\PYGZcb{}}\PYG{l+s+s1}{\PYGZsq{}}\PYG{p}{)}
\PYG{n+nb}{print}\PYG{p}{(}\PYG{l+s+sa}{f}\PYG{l+s+s1}{\PYGZsq{}}\PYG{l+s+s1}{set s\PYGZus{}copy               : }\PYG{l+s+si}{\PYGZob{}}\PYG{n}{s\PYGZus{}copy}\PYG{l+s+si}{\PYGZcb{}}\PYG{l+s+s1}{\PYGZsq{}}\PYG{p}{)}
\end{sphinxVerbatim}

\end{sphinxuseclass}\end{sphinxVerbatimInput}
\begin{sphinxVerbatimOutput}

\begin{sphinxuseclass}{cell_output}
\begin{sphinxVerbatim}[commandchars=\\\{\}]
set s before using copy(): \PYGZob{}1, 2, 3, 4, 5\PYGZcb{}
set s after using copy() : \PYGZob{}1, 2, 3, 4, 5\PYGZcb{}
set s\PYGZus{}copy               : \PYGZob{}1, 2, 3, 4, 5\PYGZcb{}
\end{sphinxVerbatim}

\end{sphinxuseclass}\end{sphinxVerbatimOutput}

\end{sphinxuseclass}

\subsection{isdisjoint()}
\label{\detokenize{sets:isdisjoint}}
\sphinxAtStartPar
It checks if there are no common elements between the two given sets.

\begin{sphinxuseclass}{cell}\begin{sphinxVerbatimInput}

\begin{sphinxuseclass}{cell_input}
\begin{sphinxVerbatim}[commandchars=\\\{\}]
\PYG{n}{s1} \PYG{o}{=} \PYG{p}{\PYGZob{}}\PYG{l+m+mi}{1}\PYG{p}{,}\PYG{l+m+mi}{2}\PYG{p}{,}\PYG{l+m+mi}{3}\PYG{p}{,}\PYG{l+m+mi}{4}\PYG{p}{,}\PYG{l+m+mi}{5}\PYG{p}{\PYGZcb{}}
\PYG{n}{s2} \PYG{o}{=} \PYG{p}{\PYGZob{}}\PYG{l+m+mi}{3}\PYG{p}{,}\PYG{l+m+mi}{4}\PYG{p}{,}\PYG{l+m+mi}{5}\PYG{p}{,}\PYG{l+m+mi}{6}\PYG{p}{,}\PYG{l+m+mi}{7}\PYG{p}{\PYGZcb{}}

\PYG{c+c1}{\PYGZsh{} False: 3,4,5 are common}
\PYG{n+nb}{print}\PYG{p}{(}\PYG{l+s+sa}{f}\PYG{l+s+s1}{\PYGZsq{}}\PYG{l+s+s1}{ s1 and s2 are disjoint: }\PYG{l+s+si}{\PYGZob{}}\PYG{n}{s1}\PYG{o}{.}\PYG{n}{isdisjoint}\PYG{p}{(}\PYG{n}{s2}\PYG{p}{)}\PYG{l+s+si}{\PYGZcb{}}\PYG{l+s+s1}{\PYGZsq{}} \PYG{p}{)}
\end{sphinxVerbatim}

\end{sphinxuseclass}\end{sphinxVerbatimInput}
\begin{sphinxVerbatimOutput}

\begin{sphinxuseclass}{cell_output}
\begin{sphinxVerbatim}[commandchars=\\\{\}]
 s1 and s2 are disjoint: False
\end{sphinxVerbatim}

\end{sphinxuseclass}\end{sphinxVerbatimOutput}

\end{sphinxuseclass}
\begin{sphinxuseclass}{cell}\begin{sphinxVerbatimInput}

\begin{sphinxuseclass}{cell_input}
\begin{sphinxVerbatim}[commandchars=\\\{\}]
\PYG{n}{s1} \PYG{o}{=} \PYG{p}{\PYGZob{}}\PYG{l+m+mi}{1}\PYG{p}{,}\PYG{l+m+mi}{2}\PYG{p}{,}\PYG{l+m+mi}{3}\PYG{p}{,}\PYG{l+m+mi}{4}\PYG{p}{,}\PYG{l+m+mi}{5}\PYG{p}{\PYGZcb{}}
\PYG{n}{s2} \PYG{o}{=} \PYG{p}{\PYGZob{}}\PYG{l+m+mi}{10}\PYG{p}{,}\PYG{l+m+mi}{20}\PYG{p}{,}\PYG{l+m+mi}{30}\PYG{p}{\PYGZcb{}}

\PYG{c+c1}{\PYGZsh{} True: no common element}
\PYG{n+nb}{print}\PYG{p}{(}\PYG{l+s+sa}{f}\PYG{l+s+s1}{\PYGZsq{}}\PYG{l+s+s1}{ s1 and s2 are disjoint: }\PYG{l+s+si}{\PYGZob{}}\PYG{n}{s1}\PYG{o}{.}\PYG{n}{isdisjoint}\PYG{p}{(}\PYG{n}{s2}\PYG{p}{)}\PYG{l+s+si}{\PYGZcb{}}\PYG{l+s+s1}{\PYGZsq{}} \PYG{p}{)}
\end{sphinxVerbatim}

\end{sphinxuseclass}\end{sphinxVerbatimInput}
\begin{sphinxVerbatimOutput}

\begin{sphinxuseclass}{cell_output}
\begin{sphinxVerbatim}[commandchars=\\\{\}]
 s1 and s2 are disjoint: True
\end{sphinxVerbatim}

\end{sphinxuseclass}\end{sphinxVerbatimOutput}

\end{sphinxuseclass}

\subsection{issubset()}
\label{\detokenize{sets:issubset}}
\sphinxAtStartPar
It checks if the first set is a subset of the second set.

\begin{sphinxuseclass}{cell}\begin{sphinxVerbatimInput}

\begin{sphinxuseclass}{cell_input}
\begin{sphinxVerbatim}[commandchars=\\\{\}]
\PYG{n}{s1} \PYG{o}{=} \PYG{p}{\PYGZob{}}\PYG{l+m+mi}{1}\PYG{p}{,}\PYG{l+m+mi}{2}\PYG{p}{,}\PYG{l+m+mi}{3}\PYG{p}{,}\PYG{l+m+mi}{4}\PYG{p}{,}\PYG{l+m+mi}{5}\PYG{p}{\PYGZcb{}}
\PYG{n}{s2} \PYG{o}{=} \PYG{p}{\PYGZob{}}\PYG{l+m+mi}{3}\PYG{p}{,}\PYG{l+m+mi}{4}\PYG{p}{,}\PYG{l+m+mi}{5}\PYG{p}{,}\PYG{l+m+mi}{6}\PYG{p}{,}\PYG{l+m+mi}{7}\PYG{p}{\PYGZcb{}}

\PYG{c+c1}{\PYGZsh{} False: s1 is not subset of s2}
\PYG{n+nb}{print}\PYG{p}{(}\PYG{l+s+sa}{f}\PYG{l+s+s1}{\PYGZsq{}}\PYG{l+s+s1}{ s1 is subset of s2: }\PYG{l+s+si}{\PYGZob{}}\PYG{n}{s1}\PYG{o}{.}\PYG{n}{issubset}\PYG{p}{(}\PYG{n}{s2}\PYG{p}{)}\PYG{l+s+si}{\PYGZcb{}}\PYG{l+s+s1}{\PYGZsq{}} \PYG{p}{)}
\end{sphinxVerbatim}

\end{sphinxuseclass}\end{sphinxVerbatimInput}
\begin{sphinxVerbatimOutput}

\begin{sphinxuseclass}{cell_output}
\begin{sphinxVerbatim}[commandchars=\\\{\}]
 s1 is subset of s2: False
\end{sphinxVerbatim}

\end{sphinxuseclass}\end{sphinxVerbatimOutput}

\end{sphinxuseclass}
\begin{sphinxuseclass}{cell}\begin{sphinxVerbatimInput}

\begin{sphinxuseclass}{cell_input}
\begin{sphinxVerbatim}[commandchars=\\\{\}]
\PYG{n}{s1} \PYG{o}{=} \PYG{p}{\PYGZob{}}\PYG{l+m+mi}{3}\PYG{p}{,}\PYG{l+m+mi}{4}\PYG{p}{,}\PYG{l+m+mi}{5}\PYG{p}{\PYGZcb{}}
\PYG{n}{s2} \PYG{o}{=} \PYG{p}{\PYGZob{}}\PYG{l+m+mi}{3}\PYG{p}{,}\PYG{l+m+mi}{4}\PYG{p}{,}\PYG{l+m+mi}{5}\PYG{p}{,}\PYG{l+m+mi}{6}\PYG{p}{,}\PYG{l+m+mi}{7}\PYG{p}{\PYGZcb{}}

\PYG{c+c1}{\PYGZsh{} True: s1 is not subset of s2}
\PYG{n+nb}{print}\PYG{p}{(}\PYG{l+s+sa}{f}\PYG{l+s+s1}{\PYGZsq{}}\PYG{l+s+s1}{ s1 is subset of s2: }\PYG{l+s+si}{\PYGZob{}}\PYG{n}{s1}\PYG{o}{.}\PYG{n}{issubset}\PYG{p}{(}\PYG{n}{s2}\PYG{p}{)}\PYG{l+s+si}{\PYGZcb{}}\PYG{l+s+s1}{\PYGZsq{}} \PYG{p}{)}
\end{sphinxVerbatim}

\end{sphinxuseclass}\end{sphinxVerbatimInput}
\begin{sphinxVerbatimOutput}

\begin{sphinxuseclass}{cell_output}
\begin{sphinxVerbatim}[commandchars=\\\{\}]
 s1 is subset of s2: True
\end{sphinxVerbatim}

\end{sphinxuseclass}\end{sphinxVerbatimOutput}

\end{sphinxuseclass}

\subsection{isuperset()}
\label{\detokenize{sets:isuperset}}
\sphinxAtStartPar
It checks if the first set contains the second set.

\begin{sphinxuseclass}{cell}\begin{sphinxVerbatimInput}

\begin{sphinxuseclass}{cell_input}
\begin{sphinxVerbatim}[commandchars=\\\{\}]
\PYG{n}{s1} \PYG{o}{=} \PYG{p}{\PYGZob{}}\PYG{l+m+mi}{1}\PYG{p}{,}\PYG{l+m+mi}{2}\PYG{p}{,}\PYG{l+m+mi}{3}\PYG{p}{,}\PYG{l+m+mi}{4}\PYG{p}{,}\PYG{l+m+mi}{5}\PYG{p}{\PYGZcb{}}
\PYG{n}{s2} \PYG{o}{=} \PYG{p}{\PYGZob{}}\PYG{l+m+mi}{3}\PYG{p}{,}\PYG{l+m+mi}{4}\PYG{p}{,}\PYG{l+m+mi}{5}\PYG{p}{,}\PYG{l+m+mi}{6}\PYG{p}{,}\PYG{l+m+mi}{7}\PYG{p}{\PYGZcb{}}

\PYG{c+c1}{\PYGZsh{} False: s1 idoes not contain s2}
\PYG{n+nb}{print}\PYG{p}{(}\PYG{l+s+sa}{f}\PYG{l+s+s1}{\PYGZsq{}}\PYG{l+s+s1}{ s1 is superset of s2: }\PYG{l+s+si}{\PYGZob{}}\PYG{n}{s1}\PYG{o}{.}\PYG{n}{issuperset}\PYG{p}{(}\PYG{n}{s2}\PYG{p}{)}\PYG{l+s+si}{\PYGZcb{}}\PYG{l+s+s1}{\PYGZsq{}} \PYG{p}{)}
\end{sphinxVerbatim}

\end{sphinxuseclass}\end{sphinxVerbatimInput}
\begin{sphinxVerbatimOutput}

\begin{sphinxuseclass}{cell_output}
\begin{sphinxVerbatim}[commandchars=\\\{\}]
 s1 is superset of s2: False
\end{sphinxVerbatim}

\end{sphinxuseclass}\end{sphinxVerbatimOutput}

\end{sphinxuseclass}
\begin{sphinxuseclass}{cell}\begin{sphinxVerbatimInput}

\begin{sphinxuseclass}{cell_input}
\begin{sphinxVerbatim}[commandchars=\\\{\}]
\PYG{n}{s1} \PYG{o}{=} \PYG{p}{\PYGZob{}}\PYG{l+m+mi}{1}\PYG{p}{,}\PYG{l+m+mi}{2}\PYG{p}{,}\PYG{l+m+mi}{3}\PYG{p}{,}\PYG{l+m+mi}{4}\PYG{p}{,}\PYG{l+m+mi}{5}\PYG{p}{\PYGZcb{}}
\PYG{n}{s2} \PYG{o}{=} \PYG{p}{\PYGZob{}}\PYG{l+m+mi}{3}\PYG{p}{,}\PYG{l+m+mi}{4}\PYG{p}{,}\PYG{l+m+mi}{5}\PYG{p}{\PYGZcb{}}

\PYG{c+c1}{\PYGZsh{} True: s1 contains s2}
\PYG{n+nb}{print}\PYG{p}{(}\PYG{l+s+sa}{f}\PYG{l+s+s1}{\PYGZsq{}}\PYG{l+s+s1}{ s1 is superset of s2: }\PYG{l+s+si}{\PYGZob{}}\PYG{n}{s1}\PYG{o}{.}\PYG{n}{issuperset}\PYG{p}{(}\PYG{n}{s2}\PYG{p}{)}\PYG{l+s+si}{\PYGZcb{}}\PYG{l+s+s1}{\PYGZsq{}} \PYG{p}{)}
\end{sphinxVerbatim}

\end{sphinxuseclass}\end{sphinxVerbatimInput}
\begin{sphinxVerbatimOutput}

\begin{sphinxuseclass}{cell_output}
\begin{sphinxVerbatim}[commandchars=\\\{\}]
 s1 is superset of s2: True
\end{sphinxVerbatim}

\end{sphinxuseclass}\end{sphinxVerbatimOutput}

\end{sphinxuseclass}

\subsection{pop()}
\label{\detokenize{sets:pop}}
\sphinxAtStartPar
It removes a randomly selected element from the set.
\begin{itemize}
\item {} 
\sphinxAtStartPar
If the set is empty, an error message is raised.

\end{itemize}

\begin{sphinxuseclass}{cell}\begin{sphinxVerbatimInput}

\begin{sphinxuseclass}{cell_input}
\begin{sphinxVerbatim}[commandchars=\\\{\}]
\PYG{n}{s} \PYG{o}{=} \PYG{p}{\PYGZob{}}\PYG{l+m+mi}{1}\PYG{p}{,}\PYG{l+m+mi}{2}\PYG{p}{,}\PYG{l+m+mi}{3}\PYG{p}{,}\PYG{l+m+mi}{4}\PYG{p}{,}\PYG{l+m+mi}{5}\PYG{p}{\PYGZcb{}}
\PYG{n+nb}{print}\PYG{p}{(}\PYG{l+s+sa}{f}\PYG{l+s+s1}{\PYGZsq{}}\PYG{l+s+s1}{set s before using pop(): }\PYG{l+s+si}{\PYGZob{}}\PYG{n}{s}\PYG{l+s+si}{\PYGZcb{}}\PYG{l+s+s1}{\PYGZsq{}}\PYG{p}{)}

\PYG{c+c1}{\PYGZsh{} removes a ranodm element}
\PYG{n}{removed\PYGZus{}element} \PYG{o}{=} \PYG{n}{s}\PYG{o}{.}\PYG{n}{pop}\PYG{p}{(}\PYG{p}{)}   

\PYG{n+nb}{print}\PYG{p}{(}\PYG{l+s+sa}{f}\PYG{l+s+s1}{\PYGZsq{}}\PYG{l+s+s1}{set s after using pop(): }\PYG{l+s+si}{\PYGZob{}}\PYG{n}{s}\PYG{l+s+si}{\PYGZcb{}}\PYG{l+s+s1}{\PYGZsq{}}\PYG{p}{)}
\PYG{n+nb}{print}\PYG{p}{(}\PYG{l+s+sa}{f}\PYG{l+s+s1}{\PYGZsq{}}\PYG{l+s+s1}{removed element        : }\PYG{l+s+si}{\PYGZob{}}\PYG{n}{removed\PYGZus{}element}\PYG{l+s+si}{\PYGZcb{}}\PYG{l+s+s1}{\PYGZsq{}}\PYG{p}{)}
\end{sphinxVerbatim}

\end{sphinxuseclass}\end{sphinxVerbatimInput}
\begin{sphinxVerbatimOutput}

\begin{sphinxuseclass}{cell_output}
\begin{sphinxVerbatim}[commandchars=\\\{\}]
set s before using pop(): \PYGZob{}1, 2, 3, 4, 5\PYGZcb{}
set s after using pop(): \PYGZob{}2, 3, 4, 5\PYGZcb{}
removed element        : 1
\end{sphinxVerbatim}

\end{sphinxuseclass}\end{sphinxVerbatimOutput}

\end{sphinxuseclass}

\subsection{remove()}
\label{\detokenize{sets:remove}}
\sphinxAtStartPar
It removes the specified element from the set.
\begin{itemize}
\item {} 
\sphinxAtStartPar
If the element does not exist in the set, an error message is raised.

\end{itemize}

\begin{sphinxuseclass}{cell}\begin{sphinxVerbatimInput}

\begin{sphinxuseclass}{cell_input}
\begin{sphinxVerbatim}[commandchars=\\\{\}]
\PYG{n}{s} \PYG{o}{=} \PYG{p}{\PYGZob{}}\PYG{l+m+mi}{1}\PYG{p}{,}\PYG{l+m+mi}{2}\PYG{p}{,}\PYG{l+m+mi}{3}\PYG{p}{,}\PYG{l+m+mi}{4}\PYG{p}{,}\PYG{l+m+mi}{5}\PYG{p}{\PYGZcb{}}
\PYG{n+nb}{print}\PYG{p}{(}\PYG{l+s+sa}{f}\PYG{l+s+s1}{\PYGZsq{}}\PYG{l+s+s1}{set s before using remove(): }\PYG{l+s+si}{\PYGZob{}}\PYG{n}{s}\PYG{l+s+si}{\PYGZcb{}}\PYG{l+s+s1}{\PYGZsq{}}\PYG{p}{)}

\PYG{c+c1}{\PYGZsh{} removes 3}
\PYG{n}{s}\PYG{o}{.}\PYG{n}{remove}\PYG{p}{(}\PYG{l+m+mi}{3}\PYG{p}{)}   

\PYG{n+nb}{print}\PYG{p}{(}\PYG{l+s+sa}{f}\PYG{l+s+s1}{\PYGZsq{}}\PYG{l+s+s1}{set s  after using remove(): }\PYG{l+s+si}{\PYGZob{}}\PYG{n}{s}\PYG{l+s+si}{\PYGZcb{}}\PYG{l+s+s1}{\PYGZsq{}}\PYG{p}{)}
\end{sphinxVerbatim}

\end{sphinxuseclass}\end{sphinxVerbatimInput}
\begin{sphinxVerbatimOutput}

\begin{sphinxuseclass}{cell_output}
\begin{sphinxVerbatim}[commandchars=\\\{\}]
set s before using remove(): \PYGZob{}1, 2, 3, 4, 5\PYGZcb{}
set s  after using remove(): \PYGZob{}1, 2, 4, 5\PYGZcb{}
\end{sphinxVerbatim}

\end{sphinxuseclass}\end{sphinxVerbatimOutput}

\end{sphinxuseclass}

\section{Iterations and Sets}
\label{\detokenize{sets:iterations-and-sets}}
\sphinxAtStartPar
We can use a \sphinxstyleemphasis{for} loop and iterate through values in the set operator to access each element of the \sphinxstyleemphasis{set} and perform operations on them.
\begin{itemize}
\item {} 
\sphinxAtStartPar
Indexes cannot be used as in \sphinxstyleemphasis{tuples} and \sphinxstyleemphasis{lists} since there is no ordering and indexing for sets.

\item {} 
\sphinxAtStartPar
When you run a for loop through a set,the order of the values of state might be different.

\end{itemize}

\sphinxAtStartPar
We can use a \sphinxstyleemphasis{for} loop to iterate through values in the set and access each element of the set to perform operations on them.
\begin{itemize}
\item {} 
\sphinxAtStartPar
Indexes cannot be used, as in tuples and lists, since there is no ordering and indexing for sets.

\item {} 
\sphinxAtStartPar
When you run a \sphinxstyleemphasis{for} loop through a set, the order of the values of the set might be different.

\end{itemize}

\begin{sphinxuseclass}{cell}\begin{sphinxVerbatimInput}

\begin{sphinxuseclass}{cell_input}
\begin{sphinxVerbatim}[commandchars=\\\{\}]
\PYG{c+c1}{\PYGZsh{} print state names in states set}
\PYG{n}{states} \PYG{o}{=} \PYG{p}{\PYGZob{}}\PYG{l+s+s1}{\PYGZsq{}}\PYG{l+s+s1}{Arizona}\PYG{l+s+s1}{\PYGZsq{}}\PYG{p}{,}\PYG{l+s+s1}{\PYGZsq{}}\PYG{l+s+s1}{Oklahoma}\PYG{l+s+s1}{\PYGZsq{}}\PYG{p}{,} \PYG{l+s+s1}{\PYGZsq{}}\PYG{l+s+s1}{Texas}\PYG{l+s+s1}{\PYGZsq{}}\PYG{p}{,} \PYG{l+s+s1}{\PYGZsq{}}\PYG{l+s+s1}{Florida}\PYG{l+s+s1}{\PYGZsq{}}\PYG{p}{,} \PYG{l+s+s1}{\PYGZsq{}}\PYG{l+s+s1}{California}\PYG{l+s+s1}{\PYGZsq{}}\PYG{p}{\PYGZcb{}}   \PYG{c+c1}{\PYGZsh{} states is a set}

\PYG{k}{for} \PYG{n}{state} \PYG{o+ow}{in} \PYG{n}{states}\PYG{p}{:}
    \PYG{n+nb}{print}\PYG{p}{(}\PYG{n}{state}\PYG{p}{)}       \PYG{c+c1}{\PYGZsh{} not in the same order that you see above}
\end{sphinxVerbatim}

\end{sphinxuseclass}\end{sphinxVerbatimInput}
\begin{sphinxVerbatimOutput}

\begin{sphinxuseclass}{cell_output}
\begin{sphinxVerbatim}[commandchars=\\\{\}]
Texas
Florida
Oklahoma
Arizona
California
\end{sphinxVerbatim}

\end{sphinxuseclass}\end{sphinxVerbatimOutput}

\end{sphinxuseclass}
\sphinxstepscope


\section{Sets Debugging}
\label{\detokenize{sets_debug:sets-debugging}}\label{\detokenize{sets_debug::doc}}\begin{itemize}
\item {} 
\sphinxAtStartPar
Each of the following short code contains one or more bugs.     

\item {} 
\sphinxAtStartPar
Please identify and correct these bugs.

\item {} 
\sphinxAtStartPar
Provide an explanation for your answer.

\end{itemize}


\subsection{Question}
\label{\detokenize{sets_debug:question}}
\begin{sphinxVerbatim}[commandchars=\\\{\}]
\PYG{n}{letters} \PYG{o}{=} \PYG{p}{\PYGZob{}}\PYG{l+s+s1}{\PYGZsq{}}\PYG{l+s+s1}{a}\PYG{l+s+s1}{\PYGZsq{}}\PYG{p}{,} \PYG{l+s+s1}{\PYGZsq{}}\PYG{l+s+s1}{b}\PYG{l+s+s1}{\PYGZsq{}}\PYG{p}{,} \PYG{l+s+s1}{\PYGZsq{}}\PYG{l+s+s1}{c}\PYG{l+s+s1}{\PYGZsq{}}\PYG{p}{\PYGZcb{}}

\PYG{n+nb}{print}\PYG{p}{(}\PYG{n}{letters}\PYG{p}{[}\PYG{l+m+mi}{1}\PYG{p}{]}\PYG{p}{)}
\end{sphinxVerbatim}

\begin{sphinxadmonition}{note}{Solution}

\sphinxAtStartPar
There is no indexing for sets.
\end{sphinxadmonition}


\subsection{Question}
\label{\detokenize{sets_debug:id1}}
\begin{sphinxVerbatim}[commandchars=\\\{\}]
\PYG{n}{letters} \PYG{o}{=} \PYG{p}{\PYGZob{}}\PYG{l+s+s1}{\PYGZsq{}}\PYG{l+s+s1}{a}\PYG{l+s+s1}{\PYGZsq{}}\PYG{p}{,} \PYG{l+s+s1}{\PYGZsq{}}\PYG{l+s+s1}{b}\PYG{l+s+s1}{\PYGZsq{}}\PYG{p}{,} \PYG{l+s+s1}{\PYGZsq{}}\PYG{l+s+s1}{c}\PYG{l+s+s1}{\PYGZsq{}}\PYG{p}{\PYGZcb{}}

\PYG{n}{letters}\PYG{o}{.}\PYG{n}{append}\PYG{p}{(}\PYG{l+s+s1}{\PYGZsq{}}\PYG{l+s+s1}{d}\PYG{l+s+s1}{\PYGZsq{}}\PYG{p}{)}

\PYG{n+nb}{print}\PYG{p}{(}\PYG{n}{letters}\PYG{p}{)}
\end{sphinxVerbatim}

\begin{sphinxadmonition}{note}{Solution}

\sphinxAtStartPar
Sets do not have an append() method; instead, they use the add() method.
\end{sphinxadmonition}


\subsection{Question}
\label{\detokenize{sets_debug:id2}}
\begin{sphinxVerbatim}[commandchars=\\\{\}]
\PYG{n}{letters} \PYG{o}{=} \PYG{p}{\PYGZob{}}\PYG{l+s+s1}{\PYGZsq{}}\PYG{l+s+s1}{a}\PYG{l+s+s1}{\PYGZsq{}}\PYG{p}{,} \PYG{l+s+s1}{\PYGZsq{}}\PYG{l+s+s1}{b}\PYG{l+s+s1}{\PYGZsq{}}\PYG{p}{,} \PYG{p}{[}\PYG{l+s+s1}{\PYGZsq{}}\PYG{l+s+s1}{c}\PYG{l+s+s1}{\PYGZsq{}}\PYG{p}{,}\PYG{l+s+s1}{\PYGZsq{}}\PYG{l+s+s1}{d}\PYG{l+s+s1}{\PYGZsq{}}\PYG{p}{]}\PYG{p}{\PYGZcb{}}

\PYG{n+nb}{print}\PYG{p}{(}\PYG{n}{letters}\PYG{p}{)}
\end{sphinxVerbatim}

\begin{sphinxadmonition}{note}{Solution}

\sphinxAtStartPar
A list cannot be an element of a set.
\end{sphinxadmonition}


\subsection{Question}
\label{\detokenize{sets_debug:id3}}
\begin{sphinxVerbatim}[commandchars=\\\{\}]
\PYG{n}{numbers} \PYG{o}{=} \PYG{p}{\PYGZob{}}\PYG{l+m+mi}{1}\PYG{p}{,}\PYG{l+m+mi}{2}\PYG{p}{,}\PYG{l+m+mi}{3}\PYG{p}{,}\PYG{l+m+mi}{4}\PYG{p}{,}\PYG{l+s+s1}{\PYGZsq{}}\PYG{l+s+s1}{5}\PYG{l+s+s1}{\PYGZsq{}}\PYG{p}{\PYGZcb{}}

\PYG{n+nb}{print}\PYG{p}{(}\PYG{n+nb}{sum}\PYG{p}{(}\PYG{n}{numbers}\PYG{p}{)}\PYG{p}{)}
\end{sphinxVerbatim}

\begin{sphinxadmonition}{note}{Solution}

\sphinxAtStartPar
‘5’ is a string and cannot be added to integers.
\end{sphinxadmonition}


\subsection{Question}
\label{\detokenize{sets_debug:id4}}
\begin{sphinxVerbatim}[commandchars=\\\{\}]
\PYG{n}{letters} \PYG{o}{=} \PYG{p}{\PYGZob{}}\PYG{l+s+s1}{\PYGZsq{}}\PYG{l+s+s1}{a}\PYG{l+s+s1}{\PYGZsq{}}\PYG{p}{,} \PYG{l+s+s1}{\PYGZsq{}}\PYG{l+s+s1}{b}\PYG{l+s+s1}{\PYGZsq{}}\PYG{p}{,} \PYG{l+s+s1}{\PYGZsq{}}\PYG{l+s+s1}{c}\PYG{l+s+s1}{\PYGZsq{}}\PYG{p}{\PYGZcb{}}

\PYG{n}{letters}\PYG{o}{.}\PYG{n}{pop}\PYG{p}{(}\PYG{l+m+mi}{1}\PYG{p}{)}

\PYG{n+nb}{print}\PYG{p}{(}\PYG{n}{letters}\PYG{p}{)}
\end{sphinxVerbatim}

\begin{sphinxadmonition}{note}{Solution}

\sphinxAtStartPar
The pop() method has no parameters.
\end{sphinxadmonition}

\sphinxstepscope


\section{Strings Output}
\label{\detokenize{sets_output:strings-output}}\label{\detokenize{sets_output::doc}}\begin{itemize}
\item {} 
\sphinxAtStartPar
Find the output of the following code.

\item {} 
\sphinxAtStartPar
Please don’t run the code before giving your answer.     

\end{itemize}


\subsection{Question}
\label{\detokenize{sets_output:question}}
\begin{sphinxuseclass}{cell}
\begin{sphinxuseclass}{tag_hide-output}\begin{sphinxVerbatimInput}

\begin{sphinxuseclass}{cell_input}
\begin{sphinxVerbatim}[commandchars=\\\{\}]
\PYG{n}{letters} \PYG{o}{=} \PYG{p}{\PYGZob{}}\PYG{l+s+s1}{\PYGZsq{}}\PYG{l+s+s1}{f}\PYG{l+s+s1}{\PYGZsq{}}\PYG{p}{\PYGZcb{}}
\PYG{n}{letters}\PYG{o}{.}\PYG{n}{add}\PYG{p}{(}\PYG{l+s+s1}{\PYGZsq{}}\PYG{l+s+s1}{k}\PYG{l+s+s1}{\PYGZsq{}}\PYG{p}{)}
\PYG{n}{letters}\PYG{o}{.}\PYG{n}{clear}\PYG{p}{(}\PYG{p}{)}
\PYG{n}{letters}\PYG{o}{.}\PYG{n}{add}\PYG{p}{(}\PYG{l+s+s1}{\PYGZsq{}}\PYG{l+s+s1}{t}\PYG{l+s+s1}{\PYGZsq{}}\PYG{p}{)}
\PYG{n}{letters}\PYG{o}{.}\PYG{n}{add}\PYG{p}{(}\PYG{l+s+s1}{\PYGZsq{}}\PYG{l+s+s1}{t}\PYG{l+s+s1}{\PYGZsq{}}\PYG{p}{)}
\PYG{n}{letters}\PYG{o}{.}\PYG{n}{add}\PYG{p}{(}\PYG{l+s+s1}{\PYGZsq{}}\PYG{l+s+s1}{t}\PYG{l+s+s1}{\PYGZsq{}}\PYG{p}{)}

\PYG{n+nb}{print}\PYG{p}{(}\PYG{n}{letters}\PYG{p}{)}
\end{sphinxVerbatim}

\end{sphinxuseclass}\end{sphinxVerbatimInput}

\end{sphinxuseclass}
\end{sphinxuseclass}

\subsection{Question}
\label{\detokenize{sets_output:id1}}
\begin{sphinxuseclass}{cell}
\begin{sphinxuseclass}{tag_hide-output}\begin{sphinxVerbatimInput}

\begin{sphinxuseclass}{cell_input}
\begin{sphinxVerbatim}[commandchars=\\\{\}]
\PYG{n}{numbers} \PYG{o}{=} \PYG{p}{[}\PYG{l+m+mi}{5}\PYG{p}{,} \PYG{l+m+mi}{8}\PYG{p}{,} \PYG{l+m+mi}{6}\PYG{p}{,} \PYG{l+m+mi}{1}\PYG{p}{,} \PYG{l+m+mi}{5}\PYG{p}{,} \PYG{l+m+mi}{5}\PYG{p}{,} \PYG{l+m+mi}{6}\PYG{p}{,} \PYG{l+m+mi}{7}\PYG{p}{]}

\PYG{n}{numbers} \PYG{o}{=} \PYG{n+nb}{set}\PYG{p}{(}\PYG{n}{numbers}\PYG{p}{)}

\PYG{n+nb}{print}\PYG{p}{(}\PYG{n+nb}{sum}\PYG{p}{(}\PYG{n}{numbers}\PYG{p}{)}\PYG{p}{)}
\end{sphinxVerbatim}

\end{sphinxuseclass}\end{sphinxVerbatimInput}

\end{sphinxuseclass}
\end{sphinxuseclass}

\subsection{Question}
\label{\detokenize{sets_output:id2}}
\begin{sphinxuseclass}{cell}
\begin{sphinxuseclass}{tag_hide-output}\begin{sphinxVerbatimInput}

\begin{sphinxuseclass}{cell_input}
\begin{sphinxVerbatim}[commandchars=\\\{\}]
\PYG{n}{state} \PYG{o}{=} \PYG{l+s+s1}{\PYGZsq{}}\PYG{l+s+s1}{california}\PYG{l+s+s1}{\PYGZsq{}}

\PYG{n+nb}{print}\PYG{p}{(}\PYG{n+nb}{len}\PYG{p}{(}\PYG{n}{state}\PYG{p}{)}\PYG{p}{)}

\PYG{n}{state\PYGZus{}set} \PYG{o}{=} \PYG{n+nb}{set}\PYG{p}{(}\PYG{n}{state}\PYG{p}{)}

\PYG{n+nb}{print}\PYG{p}{(}\PYG{n+nb}{len}\PYG{p}{(}\PYG{n}{state\PYGZus{}set}\PYG{p}{)}\PYG{p}{)}
\end{sphinxVerbatim}

\end{sphinxuseclass}\end{sphinxVerbatimInput}

\end{sphinxuseclass}
\end{sphinxuseclass}

\subsection{Question}
\label{\detokenize{sets_output:id3}}
\begin{sphinxuseclass}{cell}
\begin{sphinxuseclass}{tag_hide-output}\begin{sphinxVerbatimInput}

\begin{sphinxuseclass}{cell_input}
\begin{sphinxVerbatim}[commandchars=\\\{\}]
\PYG{n}{letters} \PYG{o}{=} \PYG{p}{[}\PYG{l+s+s1}{\PYGZsq{}}\PYG{l+s+s1}{a}\PYG{l+s+s1}{\PYGZsq{}}\PYG{p}{,}  \PYG{l+s+s1}{\PYGZsq{}}\PYG{l+s+s1}{c}\PYG{l+s+s1}{\PYGZsq{}}\PYG{p}{,} \PYG{l+s+s1}{\PYGZsq{}}\PYG{l+s+s1}{K}\PYG{l+s+s1}{\PYGZsq{}}\PYG{p}{,} \PYG{l+s+s1}{\PYGZsq{}}\PYG{l+s+s1}{T}\PYG{l+s+s1}{\PYGZsq{}}\PYG{p}{,} \PYG{l+s+s1}{\PYGZsq{}}\PYG{l+s+s1}{t}\PYG{l+s+s1}{\PYGZsq{}}\PYG{p}{,} \PYG{l+s+s1}{\PYGZsq{}}\PYG{l+s+s1}{A}\PYG{l+s+s1}{\PYGZsq{}}\PYG{p}{,} \PYG{l+s+s1}{\PYGZsq{}}\PYG{l+s+s1}{C}\PYG{l+s+s1}{\PYGZsq{}}\PYG{p}{,} \PYG{l+s+s1}{\PYGZsq{}}\PYG{l+s+s1}{a}\PYG{l+s+s1}{\PYGZsq{}}\PYG{p}{]}
\PYG{n}{letters} \PYG{o}{=} \PYG{p}{[}\PYG{n}{char}\PYG{o}{.}\PYG{n}{lower}\PYG{p}{(}\PYG{p}{)}  \PYG{k}{for} \PYG{n}{char} \PYG{o+ow}{in} \PYG{n}{letters}\PYG{p}{]}
\PYG{n}{letters} \PYG{o}{=} \PYG{n+nb}{set}\PYG{p}{(}\PYG{n}{letters}\PYG{p}{)}
\PYG{n}{letters} \PYG{o}{=} \PYG{n+nb}{list}\PYG{p}{(}\PYG{n}{letters}\PYG{p}{)}
\PYG{n}{letters}\PYG{o}{.}\PYG{n}{sort}\PYG{p}{(}\PYG{p}{)}

\PYG{n+nb}{print}\PYG{p}{(}\PYG{n}{letters}\PYG{p}{)}
\end{sphinxVerbatim}

\end{sphinxuseclass}\end{sphinxVerbatimInput}

\end{sphinxuseclass}
\end{sphinxuseclass}
\sphinxstepscope


\section{Sets Code}
\label{\detokenize{sets_code:sets-code}}\label{\detokenize{sets_code::doc}}\begin{itemize}
\item {} 
\sphinxAtStartPar
Please solve the following questions using Python code.  

\end{itemize}


\subsection{Question}
\label{\detokenize{sets_code:question}}
\sphinxAtStartPar
Identify the unique names in the list \sphinxstyleemphasis{names} below (not case\sphinxhyphen{}sensitive), and store them in a new list with dictionary order and capitalized.

\begin{sphinxuseclass}{cell}\begin{sphinxVerbatimInput}

\begin{sphinxuseclass}{cell_input}
\begin{sphinxVerbatim}[commandchars=\\\{\}]
\PYG{n}{names} \PYG{o}{=} \PYG{p}{[}\PYG{l+s+s1}{\PYGZsq{}}\PYG{l+s+s1}{Ben}\PYG{l+s+s1}{\PYGZsq{}}\PYG{p}{,} \PYG{l+s+s1}{\PYGZsq{}}\PYG{l+s+s1}{ElI}\PYG{l+s+s1}{\PYGZsq{}}\PYG{p}{,} \PYG{l+s+s1}{\PYGZsq{}}\PYG{l+s+s1}{Ben}\PYG{l+s+s1}{\PYGZsq{}}\PYG{p}{,} \PYG{l+s+s1}{\PYGZsq{}}\PYG{l+s+s1}{Ian}\PYG{l+s+s1}{\PYGZsq{}}\PYG{p}{,} \PYG{l+s+s1}{\PYGZsq{}}\PYG{l+s+s1}{MIa}\PYG{l+s+s1}{\PYGZsq{}}\PYG{p}{,} \PYG{l+s+s1}{\PYGZsq{}}\PYG{l+s+s1}{Eva}\PYG{l+s+s1}{\PYGZsq{}}\PYG{p}{,} \PYG{l+s+s1}{\PYGZsq{}}\PYG{l+s+s1}{Ana}\PYG{l+s+s1}{\PYGZsq{}}\PYG{p}{,} \PYG{l+s+s1}{\PYGZsq{}}\PYG{l+s+s1}{Leo}\PYG{l+s+s1}{\PYGZsq{}}\PYG{p}{,}  \PYG{l+s+s1}{\PYGZsq{}}\PYG{l+s+s1}{Tim}\PYG{l+s+s1}{\PYGZsq{}}\PYG{p}{,} \PYG{l+s+s1}{\PYGZsq{}}\PYG{l+s+s1}{liz}\PYG{l+s+s1}{\PYGZsq{}}\PYG{p}{,} \PYG{l+s+s1}{\PYGZsq{}}\PYG{l+s+s1}{ben}\PYG{l+s+s1}{\PYGZsq{}}\PYG{p}{,} \PYG{l+s+s1}{\PYGZsq{}}\PYG{l+s+s1}{ana}\PYG{l+s+s1}{\PYGZsq{}}\PYG{p}{,} \PYG{l+s+s1}{\PYGZsq{}}\PYG{l+s+s1}{ben}\PYG{l+s+s1}{\PYGZsq{}} \PYG{p}{]}
\end{sphinxVerbatim}

\end{sphinxuseclass}\end{sphinxVerbatimInput}

\end{sphinxuseclass}
\sphinxAtStartPar
\sphinxstylestrong{Solution}


\subsection{Question}
\label{\detokenize{sets_code:id1}}
\sphinxAtStartPar
Remove names that start with a capital letter from the lists below.
\begin{itemize}
\item {} 
\sphinxAtStartPar
Store the unique names (not case\sphinxhyphen{}sensitive) in a new list, ensuring they are capitalized and ordered.

\end{itemize}

\begin{sphinxuseclass}{cell}\begin{sphinxVerbatimInput}

\begin{sphinxuseclass}{cell_input}
\begin{sphinxVerbatim}[commandchars=\\\{\}]
\PYG{n}{names1} \PYG{o}{=} \PYG{p}{[}\PYG{l+s+s1}{\PYGZsq{}}\PYG{l+s+s1}{Ben}\PYG{l+s+s1}{\PYGZsq{}}\PYG{p}{,} \PYG{l+s+s1}{\PYGZsq{}}\PYG{l+s+s1}{ElI}\PYG{l+s+s1}{\PYGZsq{}}\PYG{p}{,} \PYG{l+s+s1}{\PYGZsq{}}\PYG{l+s+s1}{Ben}\PYG{l+s+s1}{\PYGZsq{}}\PYG{p}{,} \PYG{l+s+s1}{\PYGZsq{}}\PYG{l+s+s1}{Ian}\PYG{l+s+s1}{\PYGZsq{}}\PYG{p}{,} \PYG{l+s+s1}{\PYGZsq{}}\PYG{l+s+s1}{MIa}\PYG{l+s+s1}{\PYGZsq{}}\PYG{p}{,} \PYG{l+s+s1}{\PYGZsq{}}\PYG{l+s+s1}{Eva}\PYG{l+s+s1}{\PYGZsq{}}\PYG{p}{,} \PYG{l+s+s1}{\PYGZsq{}}\PYG{l+s+s1}{Ana}\PYG{l+s+s1}{\PYGZsq{}}\PYG{p}{,} \PYG{l+s+s1}{\PYGZsq{}}\PYG{l+s+s1}{Leo}\PYG{l+s+s1}{\PYGZsq{}}\PYG{p}{,}  \PYG{l+s+s1}{\PYGZsq{}}\PYG{l+s+s1}{Tim}\PYG{l+s+s1}{\PYGZsq{}}\PYG{p}{,} \PYG{l+s+s1}{\PYGZsq{}}\PYG{l+s+s1}{liz}\PYG{l+s+s1}{\PYGZsq{}}\PYG{p}{,} \PYG{l+s+s1}{\PYGZsq{}}\PYG{l+s+s1}{ben}\PYG{l+s+s1}{\PYGZsq{}}\PYG{p}{,} \PYG{l+s+s1}{\PYGZsq{}}\PYG{l+s+s1}{ana}\PYG{l+s+s1}{\PYGZsq{}}\PYG{p}{,} \PYG{l+s+s1}{\PYGZsq{}}\PYG{l+s+s1}{ben}\PYG{l+s+s1}{\PYGZsq{}} \PYG{p}{]}
\PYG{n}{names2} \PYG{o}{=} \PYG{p}{[}\PYG{l+s+s1}{\PYGZsq{}}\PYG{l+s+s1}{Max}\PYG{l+s+s1}{\PYGZsq{}}\PYG{p}{,} \PYG{l+s+s1}{\PYGZsq{}}\PYG{l+s+s1}{Eli}\PYG{l+s+s1}{\PYGZsq{}}\PYG{p}{,} \PYG{l+s+s1}{\PYGZsq{}}\PYG{l+s+s1}{Ben}\PYG{l+s+s1}{\PYGZsq{}}\PYG{p}{,} \PYG{l+s+s1}{\PYGZsq{}}\PYG{l+s+s1}{Ian}\PYG{l+s+s1}{\PYGZsq{}}\PYG{p}{,} \PYG{l+s+s1}{\PYGZsq{}}\PYG{l+s+s1}{Amy}\PYG{l+s+s1}{\PYGZsq{}}\PYG{p}{,} \PYG{l+s+s1}{\PYGZsq{}}\PYG{l+s+s1}{mia}\PYG{l+s+s1}{\PYGZsq{}}\PYG{p}{,} \PYG{l+s+s1}{\PYGZsq{}}\PYG{l+s+s1}{Eva}\PYG{l+s+s1}{\PYGZsq{}}\PYG{p}{,} \PYG{l+s+s1}{\PYGZsq{}}\PYG{l+s+s1}{anA}\PYG{l+s+s1}{\PYGZsq{}}\PYG{p}{,} \PYG{l+s+s1}{\PYGZsq{}}\PYG{l+s+s1}{Leo}\PYG{l+s+s1}{\PYGZsq{}}\PYG{p}{,} \PYG{l+s+s1}{\PYGZsq{}}\PYG{l+s+s1}{Joe}\PYG{l+s+s1}{\PYGZsq{}}\PYG{p}{,} \PYG{l+s+s1}{\PYGZsq{}}\PYG{l+s+s1}{Liz}\PYG{l+s+s1}{\PYGZsq{}}\PYG{p}{,} \PYG{l+s+s1}{\PYGZsq{}}\PYG{l+s+s1}{max}\PYG{l+s+s1}{\PYGZsq{}}\PYG{p}{,} \PYG{l+s+s1}{\PYGZsq{}}\PYG{l+s+s1}{iAn}\PYG{l+s+s1}{\PYGZsq{}} \PYG{p}{]}
\end{sphinxVerbatim}

\end{sphinxuseclass}\end{sphinxVerbatimInput}

\end{sphinxuseclass}
\sphinxAtStartPar
\sphinxstylestrong{Solution}

\sphinxstepscope


\chapter{Chp\sphinxhyphen{}11 Dictionaries}
\label{\detokenize{dictionaries:chp-11-dictionaries}}\label{\detokenize{dictionaries::doc}}\begin{itemize}
\item {} 
\sphinxAtStartPar
Learning Objectives
\begin{itemize}
\item {} 
\sphinxAtStartPar
..

\item {} 
\sphinxAtStartPar
..

\end{itemize}

\end{itemize}


\section{Dictionaries}
\label{\detokenize{dictionaries:dictionaries}}
\sphinxAtStartPar
These are a more general form of lists. The indexes of lists are integers starting from 0, whereas the indexes, called keys in dictionaries, can be chosen from different types.
\begin{itemize}
\item {} 
\sphinxAtStartPar
Its elements are pairs of the form \sphinxcode{\sphinxupquote{key:value}}.

\item {} 
\sphinxAtStartPar
Dictionaries are created by using curly brackets, like sets, by using \sphinxcode{\sphinxupquote{key:value}} pairs and \sphinxcode{\sphinxupquote{:}} in between.

\item {} 
\sphinxAtStartPar
Keys are like indexes, and values can be accessed by using square brackets in the form of \sphinxcode{\sphinxupquote{dict\_name{[}key{]}}}.

\item {} 
\sphinxAtStartPar
Keys must be immutable, like strings, numbers, and tuples.
\begin{itemize}
\item {} 
\sphinxAtStartPar
A list canot be a key of a dictionary.

\end{itemize}

\item {} 
\sphinxAtStartPar
Values can be any type, including strings, numbers, booleans, tuples, lists, and dictionaries.

\item {} 
\sphinxAtStartPar
Dictionary pairs are ordered.

\item {} 
\sphinxAtStartPar
If there are two pairs with the same key, then the later pair will overwrite the former one.

\item {} 
\sphinxAtStartPar
Dictionaries are mutable, so they can be modified.

\item {} 
\sphinxAtStartPar
Since dictionaries can be modified, they have a large number of methods.

\item {} 
\sphinxAtStartPar
The \sphinxcode{\sphinxupquote{len()}} function returns the number of pairs.

\item {} 
\sphinxAtStartPar
The built\sphinxhyphen{}in \sphinxcode{\sphinxupquote{dict()}} function can be used to create dictionaries from structures that have pairs.

\item {} 
\sphinxAtStartPar
\sphinxcode{\sphinxupquote{\{\}}} is an empty dictionary.
\begin{itemize}
\item {} 
\sphinxAtStartPar
An empty set is \sphinxcode{\sphinxupquote{set()}}.

\end{itemize}

\end{itemize}


\subsection{Create Dictionaries}
\label{\detokenize{dictionaries:create-dictionaries}}
\begin{sphinxuseclass}{cell}\begin{sphinxVerbatimInput}

\begin{sphinxuseclass}{cell_input}
\begin{sphinxVerbatim}[commandchars=\\\{\}]
\PYG{c+c1}{\PYGZsh{} empty dictionary}
\PYG{n}{empty\PYGZus{}dict} \PYG{o}{=} \PYG{p}{\PYGZob{}}\PYG{p}{\PYGZcb{}}

\PYG{n+nb}{print}\PYG{p}{(}\PYG{l+s+sa}{f}\PYG{l+s+s1}{\PYGZsq{}}\PYG{l+s+s1}{Empty dictionary         : }\PYG{l+s+si}{\PYGZob{}}\PYG{n}{empty\PYGZus{}dict}\PYG{l+s+si}{\PYGZcb{}}\PYG{l+s+s1}{\PYGZsq{}}\PYG{p}{)}
\PYG{n+nb}{print}\PYG{p}{(}\PYG{l+s+sa}{f}\PYG{l+s+s1}{\PYGZsq{}}\PYG{l+s+s1}{Type of emppty\PYGZus{}dict      : }\PYG{l+s+si}{\PYGZob{}}\PYG{n+nb}{type}\PYG{p}{(}\PYG{n}{empty\PYGZus{}dict}\PYG{p}{)}\PYG{l+s+si}{\PYGZcb{}}\PYG{l+s+s1}{\PYGZsq{}}\PYG{p}{)}
\end{sphinxVerbatim}

\end{sphinxuseclass}\end{sphinxVerbatimInput}
\begin{sphinxVerbatimOutput}

\begin{sphinxuseclass}{cell_output}
\begin{sphinxVerbatim}[commandchars=\\\{\}]
Empty dictionary         : \PYGZob{}\PYGZcb{}
Type of emppty\PYGZus{}dict      : \PYGZlt{}class \PYGZsq{}dict\PYGZsq{}\PYGZgt{}
\end{sphinxVerbatim}

\end{sphinxuseclass}\end{sphinxVerbatimOutput}

\end{sphinxuseclass}
\begin{sphinxuseclass}{cell}\begin{sphinxVerbatimInput}

\begin{sphinxuseclass}{cell_input}
\begin{sphinxVerbatim}[commandchars=\\\{\}]
\PYG{c+c1}{\PYGZsh{} empty dictionary with dict()}
\PYG{n}{empty\PYGZus{}dict} \PYG{o}{=} \PYG{n+nb}{dict}\PYG{p}{(}\PYG{p}{)}

\PYG{n+nb}{print}\PYG{p}{(}\PYG{l+s+sa}{f}\PYG{l+s+s1}{\PYGZsq{}}\PYG{l+s+s1}{Empty dictionary         : }\PYG{l+s+si}{\PYGZob{}}\PYG{n}{empty\PYGZus{}dict}\PYG{l+s+si}{\PYGZcb{}}\PYG{l+s+s1}{\PYGZsq{}}\PYG{p}{)}
\PYG{n+nb}{print}\PYG{p}{(}\PYG{l+s+sa}{f}\PYG{l+s+s1}{\PYGZsq{}}\PYG{l+s+s1}{Type of emppty\PYGZus{}dict      : }\PYG{l+s+si}{\PYGZob{}}\PYG{n+nb}{type}\PYG{p}{(}\PYG{n}{empty\PYGZus{}dict}\PYG{p}{)}\PYG{l+s+si}{\PYGZcb{}}\PYG{l+s+s1}{\PYGZsq{}}\PYG{p}{)}
\end{sphinxVerbatim}

\end{sphinxuseclass}\end{sphinxVerbatimInput}
\begin{sphinxVerbatimOutput}

\begin{sphinxuseclass}{cell_output}
\begin{sphinxVerbatim}[commandchars=\\\{\}]
Empty dictionary         : \PYGZob{}\PYGZcb{}
Type of emppty\PYGZus{}dict      : \PYGZlt{}class \PYGZsq{}dict\PYGZsq{}\PYGZgt{}
\end{sphinxVerbatim}

\end{sphinxuseclass}\end{sphinxVerbatimOutput}

\end{sphinxuseclass}
\begin{sphinxuseclass}{cell}\begin{sphinxVerbatimInput}

\begin{sphinxuseclass}{cell_input}
\begin{sphinxVerbatim}[commandchars=\\\{\}]
\PYG{c+c1}{\PYGZsh{} three pairs}
\PYG{c+c1}{\PYGZsh{} keys: \PYGZsq{}Math\PYGZsq{}, \PYGZsq{}Chemistry\PYGZsq{}, \PYGZsq{}History\PYGZsq{}}
\PYG{c+c1}{\PYGZsh{} values: 80, 70, 65}

\PYG{n}{grades} \PYG{o}{=} \PYG{p}{\PYGZob{}}\PYG{l+s+s1}{\PYGZsq{}}\PYG{l+s+s1}{Math}\PYG{l+s+s1}{\PYGZsq{}}\PYG{p}{:}\PYG{l+m+mi}{80}\PYG{p}{,} \PYG{l+s+s1}{\PYGZsq{}}\PYG{l+s+s1}{Chemistry}\PYG{l+s+s1}{\PYGZsq{}}\PYG{p}{:}\PYG{l+m+mi}{70}\PYG{p}{,} \PYG{l+s+s1}{\PYGZsq{}}\PYG{l+s+s1}{History}\PYG{l+s+s1}{\PYGZsq{}}\PYG{p}{:}\PYG{l+m+mi}{65}\PYG{p}{\PYGZcb{}}

\PYG{n+nb}{print}\PYG{p}{(}\PYG{n}{grades}\PYG{p}{)}
\end{sphinxVerbatim}

\end{sphinxuseclass}\end{sphinxVerbatimInput}
\begin{sphinxVerbatimOutput}

\begin{sphinxuseclass}{cell_output}
\begin{sphinxVerbatim}[commandchars=\\\{\}]
\PYGZob{}\PYGZsq{}Math\PYGZsq{}: 80, \PYGZsq{}Chemistry\PYGZsq{}: 70, \PYGZsq{}History\PYGZsq{}: 65\PYGZcb{}
\end{sphinxVerbatim}

\end{sphinxuseclass}\end{sphinxVerbatimOutput}

\end{sphinxuseclass}
\begin{sphinxuseclass}{cell}\begin{sphinxVerbatimInput}

\begin{sphinxuseclass}{cell_input}
\begin{sphinxVerbatim}[commandchars=\\\{\}]
\PYG{c+c1}{\PYGZsh{} overwrite first mpair of math }

\PYG{n}{grades} \PYG{o}{=} \PYG{p}{\PYGZob{}}\PYG{l+s+s1}{\PYGZsq{}}\PYG{l+s+s1}{Math}\PYG{l+s+s1}{\PYGZsq{}}\PYG{p}{:}\PYG{l+m+mi}{80}\PYG{p}{,} \PYG{l+s+s1}{\PYGZsq{}}\PYG{l+s+s1}{Chemistry}\PYG{l+s+s1}{\PYGZsq{}}\PYG{p}{:}\PYG{l+m+mi}{70}\PYG{p}{,} \PYG{l+s+s1}{\PYGZsq{}}\PYG{l+s+s1}{History}\PYG{l+s+s1}{\PYGZsq{}}\PYG{p}{:}\PYG{l+m+mi}{65}\PYG{p}{,} \PYG{l+s+s1}{\PYGZsq{}}\PYG{l+s+s1}{Math}\PYG{l+s+s1}{\PYGZsq{}}\PYG{p}{:}\PYG{l+m+mi}{100}\PYG{p}{\PYGZcb{}}

\PYG{n+nb}{print}\PYG{p}{(}\PYG{n}{grades}\PYG{p}{)}    \PYG{c+c1}{\PYGZsh{} The value of \PYGZsq{}Math\PYGZsq{} key is 100 }
\end{sphinxVerbatim}

\end{sphinxuseclass}\end{sphinxVerbatimInput}
\begin{sphinxVerbatimOutput}

\begin{sphinxuseclass}{cell_output}
\begin{sphinxVerbatim}[commandchars=\\\{\}]
\PYGZob{}\PYGZsq{}Math\PYGZsq{}: 100, \PYGZsq{}Chemistry\PYGZsq{}: 70, \PYGZsq{}History\PYGZsq{}: 65\PYGZcb{}
\end{sphinxVerbatim}

\end{sphinxuseclass}\end{sphinxVerbatimOutput}

\end{sphinxuseclass}
\begin{sphinxuseclass}{cell}\begin{sphinxVerbatimInput}

\begin{sphinxuseclass}{cell_input}
\begin{sphinxVerbatim}[commandchars=\\\{\}]
\PYG{c+c1}{\PYGZsh{} values canbe lists, tuples}

\PYG{n}{grades} \PYG{o}{=} \PYG{p}{\PYGZob{}}\PYG{l+s+s1}{\PYGZsq{}}\PYG{l+s+s1}{Math}\PYG{l+s+s1}{\PYGZsq{}}\PYG{p}{:}\PYG{p}{[}\PYG{l+m+mi}{80}\PYG{p}{,} \PYG{l+m+mi}{90}\PYG{p}{]}\PYG{p}{,} \PYG{l+s+s1}{\PYGZsq{}}\PYG{l+s+s1}{Chemistry}\PYG{l+s+s1}{\PYGZsq{}}\PYG{p}{:}\PYG{p}{(}\PYG{l+m+mi}{70}\PYG{p}{,} \PYG{l+m+mi}{100}\PYG{p}{)}\PYG{p}{,} \PYG{l+s+s1}{\PYGZsq{}}\PYG{l+s+s1}{History}\PYG{l+s+s1}{\PYGZsq{}}\PYG{p}{:}\PYG{l+m+mi}{65}\PYG{p}{\PYGZcb{}}

\PYG{n+nb}{print}\PYG{p}{(}\PYG{n}{grades}\PYG{p}{)}    \PYG{c+c1}{\PYGZsh{} The value of \PYGZsq{}Math\PYGZsq{} key is 100 }
\end{sphinxVerbatim}

\end{sphinxuseclass}\end{sphinxVerbatimInput}
\begin{sphinxVerbatimOutput}

\begin{sphinxuseclass}{cell_output}
\begin{sphinxVerbatim}[commandchars=\\\{\}]
\PYGZob{}\PYGZsq{}Math\PYGZsq{}: [80, 90], \PYGZsq{}Chemistry\PYGZsq{}: (70, 100), \PYGZsq{}History\PYGZsq{}: 65\PYGZcb{}
\end{sphinxVerbatim}

\end{sphinxuseclass}\end{sphinxVerbatimOutput}

\end{sphinxuseclass}
\begin{sphinxVerbatim}[commandchars=\\\{\}]
\PYG{c+c1}{\PYGZsh{} lists can not be a key}

\PYG{n}{grades} \PYG{o}{=} \PYG{p}{\PYGZob{}}\PYG{p}{[}\PYG{l+s+s1}{\PYGZsq{}}\PYG{l+s+s1}{Math}\PYG{l+s+s1}{\PYGZsq{}}\PYG{p}{,} \PYG{l+s+s1}{\PYGZsq{}}\PYG{l+s+s1}{Biology}\PYG{l+s+s1}{\PYGZsq{}}\PYG{p}{]}\PYG{p}{:}\PYG{l+m+mi}{80}\PYG{p}{,} \PYG{l+s+s1}{\PYGZsq{}}\PYG{l+s+s1}{Chemistry}\PYG{l+s+s1}{\PYGZsq{}}\PYG{p}{:}\PYG{l+m+mi}{70}\PYG{p}{,} \PYG{l+s+s1}{\PYGZsq{}}\PYG{l+s+s1}{History}\PYG{l+s+s1}{\PYGZsq{}}\PYG{p}{:}\PYG{l+m+mi}{65}\PYG{p}{\PYGZcb{}}   \PYG{c+c1}{\PYGZsh{} ERROR}
\end{sphinxVerbatim}


\subsection{dict()}
\label{\detokenize{dictionaries:dict}}
\begin{sphinxuseclass}{cell}\begin{sphinxVerbatimInput}

\begin{sphinxuseclass}{cell_input}
\begin{sphinxVerbatim}[commandchars=\\\{\}]
\PYG{c+c1}{\PYGZsh{} assignments \PYGZhy{}\PYGZhy{}\PYGZhy{}\PYGZhy{}\PYGZgt{} dict}
\PYG{n}{grades} \PYG{o}{=} \PYG{n+nb}{dict}\PYG{p}{(}\PYG{n}{Math}\PYG{o}{=}\PYG{l+m+mi}{80}\PYG{p}{,} \PYG{n}{Chemistry}\PYG{o}{=}\PYG{l+m+mi}{70}\PYG{p}{,} \PYG{n}{History}\PYG{o}{=}\PYG{l+m+mi}{65}\PYG{p}{)}

\PYG{n+nb}{print}\PYG{p}{(}\PYG{n}{grades}\PYG{p}{)}
\end{sphinxVerbatim}

\end{sphinxuseclass}\end{sphinxVerbatimInput}
\begin{sphinxVerbatimOutput}

\begin{sphinxuseclass}{cell_output}
\begin{sphinxVerbatim}[commandchars=\\\{\}]
\PYGZob{}\PYGZsq{}Math\PYGZsq{}: 80, \PYGZsq{}Chemistry\PYGZsq{}: 70, \PYGZsq{}History\PYGZsq{}: 65\PYGZcb{}
\end{sphinxVerbatim}

\end{sphinxuseclass}\end{sphinxVerbatimOutput}

\end{sphinxuseclass}
\begin{sphinxuseclass}{cell}\begin{sphinxVerbatimInput}

\begin{sphinxuseclass}{cell_input}
\begin{sphinxVerbatim}[commandchars=\\\{\}]
\PYG{c+c1}{\PYGZsh{} tuples of pairs \PYGZhy{}\PYGZhy{}\PYGZhy{}\PYGZhy{}\PYGZgt{} dict}

\PYG{n}{grades\PYGZus{}tuple} \PYG{o}{=} \PYG{p}{(} \PYG{p}{(}\PYG{l+s+s1}{\PYGZsq{}}\PYG{l+s+s1}{Math}\PYG{l+s+s1}{\PYGZsq{}}\PYG{p}{,} \PYG{l+m+mi}{80}\PYG{p}{)}\PYG{p}{,} \PYG{p}{(}\PYG{l+s+s1}{\PYGZsq{}}\PYG{l+s+s1}{Chemistry}\PYG{l+s+s1}{\PYGZsq{}}\PYG{p}{,} \PYG{l+m+mi}{70}\PYG{p}{)}\PYG{p}{,} \PYG{p}{(}\PYG{l+s+s1}{\PYGZsq{}}\PYG{l+s+s1}{History}\PYG{l+s+s1}{\PYGZsq{}}\PYG{p}{,} \PYG{l+m+mi}{6}\PYG{p}{)}\PYG{p}{)} 
\PYG{n}{grades} \PYG{o}{=} \PYG{n+nb}{dict}\PYG{p}{(}\PYG{n}{grades\PYGZus{}tuple}\PYG{p}{)}

\PYG{n+nb}{print}\PYG{p}{(}\PYG{n}{grades}\PYG{p}{)}
\end{sphinxVerbatim}

\end{sphinxuseclass}\end{sphinxVerbatimInput}
\begin{sphinxVerbatimOutput}

\begin{sphinxuseclass}{cell_output}
\begin{sphinxVerbatim}[commandchars=\\\{\}]
\PYGZob{}\PYGZsq{}Math\PYGZsq{}: 80, \PYGZsq{}Chemistry\PYGZsq{}: 70, \PYGZsq{}History\PYGZsq{}: 6\PYGZcb{}
\end{sphinxVerbatim}

\end{sphinxuseclass}\end{sphinxVerbatimOutput}

\end{sphinxuseclass}

\subsection{len()}
\label{\detokenize{dictionaries:len}}
\sphinxAtStartPar
The \sphinxcode{\sphinxupquote{len()}} function returns the number of pairs in a dictionary.

\begin{sphinxuseclass}{cell}\begin{sphinxVerbatimInput}

\begin{sphinxuseclass}{cell_input}
\begin{sphinxVerbatim}[commandchars=\\\{\}]
\PYG{n}{grades} \PYG{o}{=} \PYG{p}{\PYGZob{}}\PYG{l+s+s1}{\PYGZsq{}}\PYG{l+s+s1}{Math}\PYG{l+s+s1}{\PYGZsq{}}\PYG{p}{:}\PYG{l+m+mi}{80}\PYG{p}{,} \PYG{l+s+s1}{\PYGZsq{}}\PYG{l+s+s1}{Chemistry}\PYG{l+s+s1}{\PYGZsq{}}\PYG{p}{:}\PYG{l+m+mi}{70}\PYG{p}{,} \PYG{l+s+s1}{\PYGZsq{}}\PYG{l+s+s1}{History}\PYG{l+s+s1}{\PYGZsq{}}\PYG{p}{:}\PYG{l+m+mi}{65}\PYG{p}{\PYGZcb{}}

\PYG{n+nb}{print}\PYG{p}{(}\PYG{l+s+sa}{f}\PYG{l+s+s1}{\PYGZsq{}}\PYG{l+s+s1}{The number of pairs: }\PYG{l+s+si}{\PYGZob{}}\PYG{n+nb}{len}\PYG{p}{(}\PYG{n}{grades}\PYG{p}{)}\PYG{l+s+si}{\PYGZcb{}}\PYG{l+s+s1}{\PYGZsq{}}\PYG{p}{)}
\end{sphinxVerbatim}

\end{sphinxuseclass}\end{sphinxVerbatimInput}
\begin{sphinxVerbatimOutput}

\begin{sphinxuseclass}{cell_output}
\begin{sphinxVerbatim}[commandchars=\\\{\}]
The number of pairs: 3
\end{sphinxVerbatim}

\end{sphinxuseclass}\end{sphinxVerbatimOutput}

\end{sphinxuseclass}

\subsection{in \& not in}
\label{\detokenize{dictionaries:in-not-in}}\begin{itemize}
\item {} 
\sphinxAtStartPar
\sphinxcode{\sphinxupquote{in}} tests whether a given term is a key of a dictionary.

\item {} 
\sphinxAtStartPar
\sphinxcode{\sphinxupquote{not in}} tests whether a given term is not a key of a dictionary.

\item {} 
\sphinxAtStartPar
Both of them return boolean values, True or False.

\end{itemize}

\begin{sphinxuseclass}{cell}\begin{sphinxVerbatimInput}

\begin{sphinxuseclass}{cell_input}
\begin{sphinxVerbatim}[commandchars=\\\{\}]
\PYG{n}{grades} \PYG{o}{=} \PYG{p}{\PYGZob{}}\PYG{l+s+s1}{\PYGZsq{}}\PYG{l+s+s1}{Math}\PYG{l+s+s1}{\PYGZsq{}}\PYG{p}{:}\PYG{l+m+mi}{80}\PYG{p}{,} \PYG{l+s+s1}{\PYGZsq{}}\PYG{l+s+s1}{Chemistry}\PYG{l+s+s1}{\PYGZsq{}}\PYG{p}{:}\PYG{l+m+mi}{70}\PYG{p}{,} \PYG{l+s+s1}{\PYGZsq{}}\PYG{l+s+s1}{History}\PYG{l+s+s1}{\PYGZsq{}}\PYG{p}{:}\PYG{l+m+mi}{65}\PYG{p}{\PYGZcb{}}

\PYG{n+nb}{print}\PYG{p}{(}\PYG{l+s+sa}{f}\PYG{l+s+s1}{\PYGZsq{}}\PYG{l+s+s1}{ Math is     a key of grades: }\PYG{l+s+si}{\PYGZob{}}\PYG{l+s+s2}{\PYGZdq{}}\PYG{l+s+s2}{Math}\PYG{l+s+s2}{\PYGZdq{}}\PYG{+w}{  }\PYG{o+ow}{in}\PYG{+w}{ }\PYG{n}{grades}\PYG{l+s+si}{\PYGZcb{}}\PYG{l+s+s1}{\PYGZsq{}} \PYG{p}{)}      \PYG{c+c1}{\PYGZsh{} use \PYGZdq{} instead of \PYGZsq{} in f\PYGZhy{}strings }
\PYG{n+nb}{print}\PYG{p}{(}\PYG{l+s+sa}{f}\PYG{l+s+s1}{\PYGZsq{}}\PYG{l+s+s1}{ Math is not a key of grades: }\PYG{l+s+si}{\PYGZob{}}\PYG{l+s+s2}{\PYGZdq{}}\PYG{l+s+s2}{Math}\PYG{l+s+s2}{\PYGZdq{}}\PYG{+w}{  }\PYG{o+ow}{not}\PYG{+w}{ }\PYG{o+ow}{in}\PYG{+w}{ }\PYG{n}{grades}\PYG{l+s+si}{\PYGZcb{}}\PYG{l+s+s1}{\PYGZsq{}} \PYG{p}{)}
\end{sphinxVerbatim}

\end{sphinxuseclass}\end{sphinxVerbatimInput}
\begin{sphinxVerbatimOutput}

\begin{sphinxuseclass}{cell_output}
\begin{sphinxVerbatim}[commandchars=\\\{\}]
 Math is     a key of grades: True
 Math is not a key of grades: False
\end{sphinxVerbatim}

\end{sphinxuseclass}\end{sphinxVerbatimOutput}

\end{sphinxuseclass}
\begin{sphinxuseclass}{cell}\begin{sphinxVerbatimInput}

\begin{sphinxuseclass}{cell_input}
\begin{sphinxVerbatim}[commandchars=\\\{\}]
\PYG{n}{grades} \PYG{o}{=} \PYG{p}{\PYGZob{}}\PYG{l+s+s1}{\PYGZsq{}}\PYG{l+s+s1}{Math}\PYG{l+s+s1}{\PYGZsq{}}\PYG{p}{:}\PYG{l+m+mi}{80}\PYG{p}{,} \PYG{l+s+s1}{\PYGZsq{}}\PYG{l+s+s1}{Chemistry}\PYG{l+s+s1}{\PYGZsq{}}\PYG{p}{:}\PYG{l+m+mi}{70}\PYG{p}{,} \PYG{l+s+s1}{\PYGZsq{}}\PYG{l+s+s1}{History}\PYG{l+s+s1}{\PYGZsq{}}\PYG{p}{:}\PYG{l+m+mi}{65}\PYG{p}{\PYGZcb{}}

\PYG{n+nb}{print}\PYG{p}{(}\PYG{l+s+sa}{f}\PYG{l+s+s1}{\PYGZsq{}}\PYG{l+s+s1}{ Biology is     a key of grades: }\PYG{l+s+si}{\PYGZob{}}\PYG{l+s+s2}{\PYGZdq{}}\PYG{l+s+s2}{Biology}\PYG{l+s+s2}{\PYGZdq{}}\PYG{+w}{  }\PYG{o+ow}{in}\PYG{+w}{ }\PYG{n}{grades}\PYG{l+s+si}{\PYGZcb{}}\PYG{l+s+s1}{\PYGZsq{}} \PYG{p}{)}      \PYG{c+c1}{\PYGZsh{} use \PYGZdq{} instead of \PYGZsq{} in f\PYGZhy{}strings }
\PYG{n+nb}{print}\PYG{p}{(}\PYG{l+s+sa}{f}\PYG{l+s+s1}{\PYGZsq{}}\PYG{l+s+s1}{ Biology is not a key of grades: }\PYG{l+s+si}{\PYGZob{}}\PYG{l+s+s2}{\PYGZdq{}}\PYG{l+s+s2}{Biology}\PYG{l+s+s2}{\PYGZdq{}}\PYG{+w}{  }\PYG{o+ow}{not}\PYG{+w}{ }\PYG{o+ow}{in}\PYG{+w}{ }\PYG{n}{grades}\PYG{l+s+si}{\PYGZcb{}}\PYG{l+s+s1}{\PYGZsq{}} \PYG{p}{)}
\end{sphinxVerbatim}

\end{sphinxuseclass}\end{sphinxVerbatimInput}
\begin{sphinxVerbatimOutput}

\begin{sphinxuseclass}{cell_output}
\begin{sphinxVerbatim}[commandchars=\\\{\}]
 Biology is     a key of grades: False
 Biology is not a key of grades: True
\end{sphinxVerbatim}

\end{sphinxuseclass}\end{sphinxVerbatimOutput}

\end{sphinxuseclass}

\subsection{Mutable}
\label{\detokenize{dictionaries:mutable}}
\sphinxAtStartPar
Unlike strings and tuples, and like lists, dictionaries are mutable, which means they can be modified.
\begin{itemize}
\item {} 
\sphinxAtStartPar
New pairs can be added, and existing pairs can be modified.

\end{itemize}


\subsubsection{Add a new pair}
\label{\detokenize{dictionaries:add-a-new-pair}}\begin{itemize}
\item {} 
\sphinxAtStartPar
Use square brackets in the form of \sphinxcode{\sphinxupquote{dict\_name{[}new\_key{]} = new\_value}}.

\end{itemize}

\begin{sphinxuseclass}{cell}\begin{sphinxVerbatimInput}

\begin{sphinxuseclass}{cell_input}
\begin{sphinxVerbatim}[commandchars=\\\{\}]
\PYG{n}{grades} \PYG{o}{=} \PYG{p}{\PYGZob{}}\PYG{l+s+s1}{\PYGZsq{}}\PYG{l+s+s1}{Math}\PYG{l+s+s1}{\PYGZsq{}}\PYG{p}{:}\PYG{l+m+mi}{80}\PYG{p}{,} \PYG{l+s+s1}{\PYGZsq{}}\PYG{l+s+s1}{Chemistry}\PYG{l+s+s1}{\PYGZsq{}}\PYG{p}{:}\PYG{l+m+mi}{70}\PYG{p}{,} \PYG{l+s+s1}{\PYGZsq{}}\PYG{l+s+s1}{History}\PYG{l+s+s1}{\PYGZsq{}}\PYG{p}{:}\PYG{l+m+mi}{65}\PYG{p}{\PYGZcb{}}
\PYG{n+nb}{print}\PYG{p}{(}\PYG{l+s+sa}{f}\PYG{l+s+s1}{\PYGZsq{}}\PYG{l+s+s1}{grades dictionary before adding a new pair: }\PYG{l+s+si}{\PYGZob{}}\PYG{n}{grades}\PYG{l+s+si}{\PYGZcb{}}\PYG{l+s+s1}{\PYGZsq{}}\PYG{p}{)}

\PYG{c+c1}{\PYGZsh{} add the pair \PYGZsq{}Eglish\PYGZsq{}:99}
\PYG{n}{grades}\PYG{p}{[}\PYG{l+s+s1}{\PYGZsq{}}\PYG{l+s+s1}{English}\PYG{l+s+s1}{\PYGZsq{}}\PYG{p}{]} \PYG{o}{=} \PYG{l+m+mi}{99}

\PYG{n+nb}{print}\PYG{p}{(}\PYG{l+s+sa}{f}\PYG{l+s+s1}{\PYGZsq{}}\PYG{l+s+s1}{grades dictionary after  adding a new pair: }\PYG{l+s+si}{\PYGZob{}}\PYG{n}{grades}\PYG{l+s+si}{\PYGZcb{}}\PYG{l+s+s1}{\PYGZsq{}}\PYG{p}{)}
\end{sphinxVerbatim}

\end{sphinxuseclass}\end{sphinxVerbatimInput}
\begin{sphinxVerbatimOutput}

\begin{sphinxuseclass}{cell_output}
\begin{sphinxVerbatim}[commandchars=\\\{\}]
grades dictionary before adding a new pair: \PYGZob{}\PYGZsq{}Math\PYGZsq{}: 80, \PYGZsq{}Chemistry\PYGZsq{}: 70, \PYGZsq{}History\PYGZsq{}: 65\PYGZcb{}
grades dictionary after  adding a new pair: \PYGZob{}\PYGZsq{}Math\PYGZsq{}: 80, \PYGZsq{}Chemistry\PYGZsq{}: 70, \PYGZsq{}History\PYGZsq{}: 65, \PYGZsq{}English\PYGZsq{}: 99\PYGZcb{}
\end{sphinxVerbatim}

\end{sphinxuseclass}\end{sphinxVerbatimOutput}

\end{sphinxuseclass}

\subsubsection{Change a Value}
\label{\detokenize{dictionaries:change-a-value}}\begin{itemize}
\item {} 
\sphinxAtStartPar
Use square brackets in the form of \sphinxcode{\sphinxupquote{dict\_name{[}key{]} = new\_value}}.

\end{itemize}

\begin{sphinxuseclass}{cell}\begin{sphinxVerbatimInput}

\begin{sphinxuseclass}{cell_input}
\begin{sphinxVerbatim}[commandchars=\\\{\}]
\PYG{n}{grades} \PYG{o}{=} \PYG{p}{\PYGZob{}}\PYG{l+s+s1}{\PYGZsq{}}\PYG{l+s+s1}{Math}\PYG{l+s+s1}{\PYGZsq{}}\PYG{p}{:}\PYG{l+m+mi}{80}\PYG{p}{,} \PYG{l+s+s1}{\PYGZsq{}}\PYG{l+s+s1}{Chemistry}\PYG{l+s+s1}{\PYGZsq{}}\PYG{p}{:}\PYG{l+m+mi}{70}\PYG{p}{,} \PYG{l+s+s1}{\PYGZsq{}}\PYG{l+s+s1}{History}\PYG{l+s+s1}{\PYGZsq{}}\PYG{p}{:}\PYG{l+m+mi}{65}\PYG{p}{\PYGZcb{}}
\PYG{n+nb}{print}\PYG{p}{(}\PYG{l+s+sa}{f}\PYG{l+s+s1}{\PYGZsq{}}\PYG{l+s+s1}{grades dictionary before changing a value: }\PYG{l+s+si}{\PYGZob{}}\PYG{n}{grades}\PYG{l+s+si}{\PYGZcb{}}\PYG{l+s+s1}{\PYGZsq{}}\PYG{p}{)}

\PYG{c+c1}{\PYGZsh{} change the value of \PYGZsq{}Chemistry\PYGZsq{}}
\PYG{n}{grades}\PYG{p}{[}\PYG{l+s+s1}{\PYGZsq{}}\PYG{l+s+s1}{Chemistry}\PYG{l+s+s1}{\PYGZsq{}}\PYG{p}{]} \PYG{o}{=} \PYG{l+m+mi}{85}

\PYG{n+nb}{print}\PYG{p}{(}\PYG{l+s+sa}{f}\PYG{l+s+s1}{\PYGZsq{}}\PYG{l+s+s1}{grades dictionary after adding a new pair: }\PYG{l+s+si}{\PYGZob{}}\PYG{n}{grades}\PYG{l+s+si}{\PYGZcb{}}\PYG{l+s+s1}{\PYGZsq{}}\PYG{p}{)}
\end{sphinxVerbatim}

\end{sphinxuseclass}\end{sphinxVerbatimInput}
\begin{sphinxVerbatimOutput}

\begin{sphinxuseclass}{cell_output}
\begin{sphinxVerbatim}[commandchars=\\\{\}]
grades dictionary before changing a value: \PYGZob{}\PYGZsq{}Math\PYGZsq{}: 80, \PYGZsq{}Chemistry\PYGZsq{}: 70, \PYGZsq{}History\PYGZsq{}: 65\PYGZcb{}
grades dictionary after adding a new pair: \PYGZob{}\PYGZsq{}Math\PYGZsq{}: 80, \PYGZsq{}Chemistry\PYGZsq{}: 85, \PYGZsq{}History\PYGZsq{}: 65\PYGZcb{}
\end{sphinxVerbatim}

\end{sphinxuseclass}\end{sphinxVerbatimOutput}

\end{sphinxuseclass}

\subsubsection{Delete a pair}
\label{\detokenize{dictionaries:delete-a-pair}}\begin{itemize}
\item {} 
\sphinxAtStartPar
Use \sphinxcode{\sphinxupquote{del}} in the form of \sphinxcode{\sphinxupquote{del dict\_name{[}key{]}}}.

\end{itemize}

\begin{sphinxuseclass}{cell}\begin{sphinxVerbatimInput}

\begin{sphinxuseclass}{cell_input}
\begin{sphinxVerbatim}[commandchars=\\\{\}]
\PYG{n}{grades} \PYG{o}{=} \PYG{p}{\PYGZob{}}\PYG{l+s+s1}{\PYGZsq{}}\PYG{l+s+s1}{Math}\PYG{l+s+s1}{\PYGZsq{}}\PYG{p}{:}\PYG{l+m+mi}{80}\PYG{p}{,} \PYG{l+s+s1}{\PYGZsq{}}\PYG{l+s+s1}{Chemistry}\PYG{l+s+s1}{\PYGZsq{}}\PYG{p}{:}\PYG{l+m+mi}{70}\PYG{p}{,} \PYG{l+s+s1}{\PYGZsq{}}\PYG{l+s+s1}{History}\PYG{l+s+s1}{\PYGZsq{}}\PYG{p}{:}\PYG{l+m+mi}{65}\PYG{p}{\PYGZcb{}}
\PYG{n+nb}{print}\PYG{p}{(}\PYG{l+s+sa}{f}\PYG{l+s+s1}{\PYGZsq{}}\PYG{l+s+s1}{grades dictionary before deleting a value: }\PYG{l+s+si}{\PYGZob{}}\PYG{n}{grades}\PYG{l+s+si}{\PYGZcb{}}\PYG{l+s+s1}{\PYGZsq{}}\PYG{p}{)}

\PYG{c+c1}{\PYGZsh{} delete \PYGZsq{}Chemistry\PYGZsq{}:70 pair}
\PYG{k}{del} \PYG{n}{grades}\PYG{p}{[}\PYG{l+s+s1}{\PYGZsq{}}\PYG{l+s+s1}{Chemistry}\PYG{l+s+s1}{\PYGZsq{}}\PYG{p}{]} 

\PYG{n+nb}{print}\PYG{p}{(}\PYG{l+s+sa}{f}\PYG{l+s+s1}{\PYGZsq{}}\PYG{l+s+s1}{grades dictionary after  deleting a  pair: }\PYG{l+s+si}{\PYGZob{}}\PYG{n}{grades}\PYG{l+s+si}{\PYGZcb{}}\PYG{l+s+s1}{\PYGZsq{}}\PYG{p}{)}
\end{sphinxVerbatim}

\end{sphinxuseclass}\end{sphinxVerbatimInput}
\begin{sphinxVerbatimOutput}

\begin{sphinxuseclass}{cell_output}
\begin{sphinxVerbatim}[commandchars=\\\{\}]
grades dictionary before deleting a value: \PYGZob{}\PYGZsq{}Math\PYGZsq{}: 80, \PYGZsq{}Chemistry\PYGZsq{}: 70, \PYGZsq{}History\PYGZsq{}: 65\PYGZcb{}
grades dictionary after  deleting a  pair: \PYGZob{}\PYGZsq{}Math\PYGZsq{}: 80, \PYGZsq{}History\PYGZsq{}: 65\PYGZcb{}
\end{sphinxVerbatim}

\end{sphinxuseclass}\end{sphinxVerbatimOutput}

\end{sphinxuseclass}

\subsection{Dictionary Methods}
\label{\detokenize{dictionaries:dictionary-methods}}
\sphinxAtStartPar
Except for the magic methods (the ones with underscores), there are 11 methods for dictionaries.
\begin{itemize}
\item {} 
\sphinxAtStartPar
You can run help(dict) for more details.

\end{itemize}

\begin{sphinxuseclass}{cell}\begin{sphinxVerbatimInput}

\begin{sphinxuseclass}{cell_input}
\begin{sphinxVerbatim}[commandchars=\\\{\}]
\PYG{c+c1}{\PYGZsh{} methods of dictionaries}
\PYG{c+c1}{\PYGZsh{} dir() returns a list}

\PYG{n+nb}{print}\PYG{p}{(}\PYG{n+nb}{dir}\PYG{p}{(}\PYG{n+nb}{dict}\PYG{p}{)}\PYG{p}{)}
\end{sphinxVerbatim}

\end{sphinxuseclass}\end{sphinxVerbatimInput}
\begin{sphinxVerbatimOutput}

\begin{sphinxuseclass}{cell_output}
\begin{sphinxVerbatim}[commandchars=\\\{\}]
[\PYGZsq{}\PYGZus{}\PYGZus{}class\PYGZus{}\PYGZus{}\PYGZsq{}, \PYGZsq{}\PYGZus{}\PYGZus{}class\PYGZus{}getitem\PYGZus{}\PYGZus{}\PYGZsq{}, \PYGZsq{}\PYGZus{}\PYGZus{}contains\PYGZus{}\PYGZus{}\PYGZsq{}, \PYGZsq{}\PYGZus{}\PYGZus{}delattr\PYGZus{}\PYGZus{}\PYGZsq{}, \PYGZsq{}\PYGZus{}\PYGZus{}delitem\PYGZus{}\PYGZus{}\PYGZsq{}, \PYGZsq{}\PYGZus{}\PYGZus{}dir\PYGZus{}\PYGZus{}\PYGZsq{}, \PYGZsq{}\PYGZus{}\PYGZus{}doc\PYGZus{}\PYGZus{}\PYGZsq{}, \PYGZsq{}\PYGZus{}\PYGZus{}eq\PYGZus{}\PYGZus{}\PYGZsq{}, \PYGZsq{}\PYGZus{}\PYGZus{}format\PYGZus{}\PYGZus{}\PYGZsq{}, \PYGZsq{}\PYGZus{}\PYGZus{}ge\PYGZus{}\PYGZus{}\PYGZsq{}, \PYGZsq{}\PYGZus{}\PYGZus{}getattribute\PYGZus{}\PYGZus{}\PYGZsq{}, \PYGZsq{}\PYGZus{}\PYGZus{}getitem\PYGZus{}\PYGZus{}\PYGZsq{}, \PYGZsq{}\PYGZus{}\PYGZus{}getstate\PYGZus{}\PYGZus{}\PYGZsq{}, \PYGZsq{}\PYGZus{}\PYGZus{}gt\PYGZus{}\PYGZus{}\PYGZsq{}, \PYGZsq{}\PYGZus{}\PYGZus{}hash\PYGZus{}\PYGZus{}\PYGZsq{}, \PYGZsq{}\PYGZus{}\PYGZus{}init\PYGZus{}\PYGZus{}\PYGZsq{}, \PYGZsq{}\PYGZus{}\PYGZus{}init\PYGZus{}subclass\PYGZus{}\PYGZus{}\PYGZsq{}, \PYGZsq{}\PYGZus{}\PYGZus{}ior\PYGZus{}\PYGZus{}\PYGZsq{}, \PYGZsq{}\PYGZus{}\PYGZus{}iter\PYGZus{}\PYGZus{}\PYGZsq{}, \PYGZsq{}\PYGZus{}\PYGZus{}le\PYGZus{}\PYGZus{}\PYGZsq{}, \PYGZsq{}\PYGZus{}\PYGZus{}len\PYGZus{}\PYGZus{}\PYGZsq{}, \PYGZsq{}\PYGZus{}\PYGZus{}lt\PYGZus{}\PYGZus{}\PYGZsq{}, \PYGZsq{}\PYGZus{}\PYGZus{}ne\PYGZus{}\PYGZus{}\PYGZsq{}, \PYGZsq{}\PYGZus{}\PYGZus{}new\PYGZus{}\PYGZus{}\PYGZsq{}, \PYGZsq{}\PYGZus{}\PYGZus{}or\PYGZus{}\PYGZus{}\PYGZsq{}, \PYGZsq{}\PYGZus{}\PYGZus{}reduce\PYGZus{}\PYGZus{}\PYGZsq{}, \PYGZsq{}\PYGZus{}\PYGZus{}reduce\PYGZus{}ex\PYGZus{}\PYGZus{}\PYGZsq{}, \PYGZsq{}\PYGZus{}\PYGZus{}repr\PYGZus{}\PYGZus{}\PYGZsq{}, \PYGZsq{}\PYGZus{}\PYGZus{}reversed\PYGZus{}\PYGZus{}\PYGZsq{}, \PYGZsq{}\PYGZus{}\PYGZus{}ror\PYGZus{}\PYGZus{}\PYGZsq{}, \PYGZsq{}\PYGZus{}\PYGZus{}setattr\PYGZus{}\PYGZus{}\PYGZsq{}, \PYGZsq{}\PYGZus{}\PYGZus{}setitem\PYGZus{}\PYGZus{}\PYGZsq{}, \PYGZsq{}\PYGZus{}\PYGZus{}sizeof\PYGZus{}\PYGZus{}\PYGZsq{}, \PYGZsq{}\PYGZus{}\PYGZus{}str\PYGZus{}\PYGZus{}\PYGZsq{}, \PYGZsq{}\PYGZus{}\PYGZus{}subclasshook\PYGZus{}\PYGZus{}\PYGZsq{}, \PYGZsq{}clear\PYGZsq{}, \PYGZsq{}copy\PYGZsq{}, \PYGZsq{}fromkeys\PYGZsq{}, \PYGZsq{}get\PYGZsq{}, \PYGZsq{}items\PYGZsq{}, \PYGZsq{}keys\PYGZsq{}, \PYGZsq{}pop\PYGZsq{}, \PYGZsq{}popitem\PYGZsq{}, \PYGZsq{}setdefault\PYGZsq{}, \PYGZsq{}update\PYGZsq{}, \PYGZsq{}values\PYGZsq{}]
\end{sphinxVerbatim}

\end{sphinxuseclass}\end{sphinxVerbatimOutput}

\end{sphinxuseclass}
\begin{sphinxuseclass}{cell}\begin{sphinxVerbatimInput}

\begin{sphinxuseclass}{cell_input}
\begin{sphinxVerbatim}[commandchars=\\\{\}]
\PYG{c+c1}{\PYGZsh{} non magic methods by using slicing}
\PYG{n+nb}{print}\PYG{p}{(}\PYG{n+nb}{dir}\PYG{p}{(}\PYG{n+nb}{dict}\PYG{p}{)}\PYG{p}{[}\PYG{o}{\PYGZhy{}}\PYG{l+m+mi}{11}\PYG{p}{:}\PYG{p}{]}\PYG{p}{)}
\end{sphinxVerbatim}

\end{sphinxuseclass}\end{sphinxVerbatimInput}
\begin{sphinxVerbatimOutput}

\begin{sphinxuseclass}{cell_output}
\begin{sphinxVerbatim}[commandchars=\\\{\}]
[\PYGZsq{}clear\PYGZsq{}, \PYGZsq{}copy\PYGZsq{}, \PYGZsq{}fromkeys\PYGZsq{}, \PYGZsq{}get\PYGZsq{}, \PYGZsq{}items\PYGZsq{}, \PYGZsq{}keys\PYGZsq{}, \PYGZsq{}pop\PYGZsq{}, \PYGZsq{}popitem\PYGZsq{}, \PYGZsq{}setdefault\PYGZsq{}, \PYGZsq{}update\PYGZsq{}, \PYGZsq{}values\PYGZsq{}]
\end{sphinxVerbatim}

\end{sphinxuseclass}\end{sphinxVerbatimOutput}

\end{sphinxuseclass}

\subsubsection{clear()}
\label{\detokenize{dictionaries:clear}}
\sphinxAtStartPar
It removes all pairs from the dictionary, making it an empty dictionary.

\begin{sphinxuseclass}{cell}\begin{sphinxVerbatimInput}

\begin{sphinxuseclass}{cell_input}
\begin{sphinxVerbatim}[commandchars=\\\{\}]
\PYG{n}{grades} \PYG{o}{=} \PYG{p}{\PYGZob{}}\PYG{l+s+s1}{\PYGZsq{}}\PYG{l+s+s1}{Math}\PYG{l+s+s1}{\PYGZsq{}}\PYG{p}{:}\PYG{l+m+mi}{80}\PYG{p}{,} \PYG{l+s+s1}{\PYGZsq{}}\PYG{l+s+s1}{Chemistry}\PYG{l+s+s1}{\PYGZsq{}}\PYG{p}{:}\PYG{l+m+mi}{70}\PYG{p}{,} \PYG{l+s+s1}{\PYGZsq{}}\PYG{l+s+s1}{History}\PYG{l+s+s1}{\PYGZsq{}}\PYG{p}{:}\PYG{l+m+mi}{65}\PYG{p}{\PYGZcb{}}
\PYG{n+nb}{print}\PYG{p}{(}\PYG{l+s+sa}{f}\PYG{l+s+s1}{\PYGZsq{}}\PYG{l+s+s1}{grades dictionary before using clear(): }\PYG{l+s+si}{\PYGZob{}}\PYG{n}{grades}\PYG{l+s+si}{\PYGZcb{}}\PYG{l+s+s1}{\PYGZsq{}}\PYG{p}{)}

\PYG{c+c1}{\PYGZsh{} remove all elements of grades dictionary}
\PYG{n}{grades}\PYG{o}{.}\PYG{n}{clear}\PYG{p}{(}\PYG{p}{)}

\PYG{n+nb}{print}\PYG{p}{(}\PYG{l+s+sa}{f}\PYG{l+s+s1}{\PYGZsq{}}\PYG{l+s+s1}{grades dictionary after  using clear() : }\PYG{l+s+si}{\PYGZob{}}\PYG{n}{grades}\PYG{l+s+si}{\PYGZcb{}}\PYG{l+s+s1}{\PYGZsq{}}\PYG{p}{)}
\end{sphinxVerbatim}

\end{sphinxuseclass}\end{sphinxVerbatimInput}
\begin{sphinxVerbatimOutput}

\begin{sphinxuseclass}{cell_output}
\begin{sphinxVerbatim}[commandchars=\\\{\}]
grades dictionary before using clear(): \PYGZob{}\PYGZsq{}Math\PYGZsq{}: 80, \PYGZsq{}Chemistry\PYGZsq{}: 70, \PYGZsq{}History\PYGZsq{}: 65\PYGZcb{}
grades dictionary after  using clear() : \PYGZob{}\PYGZcb{}
\end{sphinxVerbatim}

\end{sphinxuseclass}\end{sphinxVerbatimOutput}

\end{sphinxuseclass}

\subsubsection{copy()}
\label{\detokenize{dictionaries:copy}}
\sphinxAtStartPar
It returns a new dictionary with the same pairs.

\begin{sphinxuseclass}{cell}\begin{sphinxVerbatimInput}

\begin{sphinxuseclass}{cell_input}
\begin{sphinxVerbatim}[commandchars=\\\{\}]
\PYG{n}{grades} \PYG{o}{=} \PYG{p}{\PYGZob{}}\PYG{l+s+s1}{\PYGZsq{}}\PYG{l+s+s1}{Math}\PYG{l+s+s1}{\PYGZsq{}}\PYG{p}{:}\PYG{l+m+mi}{80}\PYG{p}{,} \PYG{l+s+s1}{\PYGZsq{}}\PYG{l+s+s1}{Chemistry}\PYG{l+s+s1}{\PYGZsq{}}\PYG{p}{:}\PYG{l+m+mi}{70}\PYG{p}{,} \PYG{l+s+s1}{\PYGZsq{}}\PYG{l+s+s1}{History}\PYG{l+s+s1}{\PYGZsq{}}\PYG{p}{:}\PYG{l+m+mi}{65}\PYG{p}{\PYGZcb{}}
\PYG{n+nb}{print}\PYG{p}{(}\PYG{l+s+sa}{f}\PYG{l+s+s1}{\PYGZsq{}}\PYG{l+s+s1}{grades dictionary     : }\PYG{l+s+si}{\PYGZob{}}\PYG{n}{grades}\PYG{l+s+si}{\PYGZcb{}}\PYG{l+s+s1}{\PYGZsq{}}\PYG{p}{)}

\PYG{c+c1}{\PYGZsh{} remove all elements of grades dictionary}
\PYG{n}{grades\PYGZus{}copy} \PYG{o}{=} \PYG{n}{grades}\PYG{o}{.}\PYG{n}{copy}\PYG{p}{(}\PYG{p}{)}

\PYG{n+nb}{print}\PYG{p}{(}\PYG{l+s+sa}{f}\PYG{l+s+s1}{\PYGZsq{}}\PYG{l+s+s1}{grades\PYGZus{}copy dictionary: }\PYG{l+s+si}{\PYGZob{}}\PYG{n}{grades}\PYG{l+s+si}{\PYGZcb{}}\PYG{l+s+s1}{\PYGZsq{}}\PYG{p}{)}
\end{sphinxVerbatim}

\end{sphinxuseclass}\end{sphinxVerbatimInput}
\begin{sphinxVerbatimOutput}

\begin{sphinxuseclass}{cell_output}
\begin{sphinxVerbatim}[commandchars=\\\{\}]
grades dictionary     : \PYGZob{}\PYGZsq{}Math\PYGZsq{}: 80, \PYGZsq{}Chemistry\PYGZsq{}: 70, \PYGZsq{}History\PYGZsq{}: 65\PYGZcb{}
grades\PYGZus{}copy dictionary: \PYGZob{}\PYGZsq{}Math\PYGZsq{}: 80, \PYGZsq{}Chemistry\PYGZsq{}: 70, \PYGZsq{}History\PYGZsq{}: 65\PYGZcb{}
\end{sphinxVerbatim}

\end{sphinxuseclass}\end{sphinxVerbatimOutput}

\end{sphinxuseclass}

\subsubsection{get()}
\label{\detokenize{dictionaries:get}}
\sphinxAtStartPar
It takes a key and a default value.
\begin{itemize}
\item {} 
\sphinxAtStartPar
If the given key exists in the dictionary, \sphinxstyleemphasis{get()} returns the corresponding value; otherwise, it returns the default value.

\item {} 
\sphinxAtStartPar
Its purpose is to prevent errors for non\sphinxhyphen{}existing key\sphinxhyphen{}value pairs.

\end{itemize}

\begin{sphinxuseclass}{cell}\begin{sphinxVerbatimInput}

\begin{sphinxuseclass}{cell_input}
\begin{sphinxVerbatim}[commandchars=\\\{\}]
\PYG{n}{grades} \PYG{o}{=} \PYG{p}{\PYGZob{}}\PYG{l+s+s1}{\PYGZsq{}}\PYG{l+s+s1}{Math}\PYG{l+s+s1}{\PYGZsq{}}\PYG{p}{:}\PYG{l+m+mi}{80}\PYG{p}{,} \PYG{l+s+s1}{\PYGZsq{}}\PYG{l+s+s1}{Chemistry}\PYG{l+s+s1}{\PYGZsq{}}\PYG{p}{:}\PYG{l+m+mi}{70}\PYG{p}{,} \PYG{l+s+s1}{\PYGZsq{}}\PYG{l+s+s1}{History}\PYG{l+s+s1}{\PYGZsq{}}\PYG{p}{:}\PYG{l+m+mi}{65}\PYG{p}{\PYGZcb{}}
\PYG{n+nb}{print}\PYG{p}{(}\PYG{l+s+sa}{f}\PYG{l+s+s1}{\PYGZsq{}}\PYG{l+s+s1}{grades dictionary     : }\PYG{l+s+si}{\PYGZob{}}\PYG{n}{grades}\PYG{l+s+si}{\PYGZcb{}}\PYG{l+s+s1}{\PYGZsq{}}\PYG{p}{)}

\PYG{c+c1}{\PYGZsh{} value corresponding to \PYGZsq{}Math\PYGZsq{} key }
\PYG{n}{val} \PYG{o}{=}  \PYG{n}{grades}\PYG{o}{.}\PYG{n}{get}\PYG{p}{(}\PYG{l+s+s1}{\PYGZsq{}}\PYG{l+s+s1}{Math}\PYG{l+s+s1}{\PYGZsq{}}\PYG{p}{,} \PYG{l+s+s1}{\PYGZsq{}}\PYG{l+s+s1}{Does Not Exist}\PYG{l+s+s1}{\PYGZsq{}}\PYG{p}{)}    \PYG{c+c1}{\PYGZsh{} Math key exist so get returns 80 not \PYGZsq{}Does Not Exist\PYGZsq{}}

\PYG{n+nb}{print}\PYG{p}{(}\PYG{l+s+sa}{f}\PYG{l+s+s1}{\PYGZsq{}}\PYG{l+s+s1}{The value of Math key: }\PYG{l+s+si}{\PYGZob{}}\PYG{n}{val}\PYG{l+s+si}{\PYGZcb{}}\PYG{l+s+s1}{\PYGZsq{}}\PYG{p}{)}
\end{sphinxVerbatim}

\end{sphinxuseclass}\end{sphinxVerbatimInput}
\begin{sphinxVerbatimOutput}

\begin{sphinxuseclass}{cell_output}
\begin{sphinxVerbatim}[commandchars=\\\{\}]
grades dictionary     : \PYGZob{}\PYGZsq{}Math\PYGZsq{}: 80, \PYGZsq{}Chemistry\PYGZsq{}: 70, \PYGZsq{}History\PYGZsq{}: 65\PYGZcb{}
The value of Math key: 80
\end{sphinxVerbatim}

\end{sphinxuseclass}\end{sphinxVerbatimOutput}

\end{sphinxuseclass}
\begin{sphinxuseclass}{cell}\begin{sphinxVerbatimInput}

\begin{sphinxuseclass}{cell_input}
\begin{sphinxVerbatim}[commandchars=\\\{\}]
\PYG{n}{grades} \PYG{o}{=} \PYG{p}{\PYGZob{}}\PYG{l+s+s1}{\PYGZsq{}}\PYG{l+s+s1}{Math}\PYG{l+s+s1}{\PYGZsq{}}\PYG{p}{:}\PYG{l+m+mi}{80}\PYG{p}{,} \PYG{l+s+s1}{\PYGZsq{}}\PYG{l+s+s1}{Chemistry}\PYG{l+s+s1}{\PYGZsq{}}\PYG{p}{:}\PYG{l+m+mi}{70}\PYG{p}{,} \PYG{l+s+s1}{\PYGZsq{}}\PYG{l+s+s1}{History}\PYG{l+s+s1}{\PYGZsq{}}\PYG{p}{:}\PYG{l+m+mi}{65}\PYG{p}{\PYGZcb{}}
\PYG{n+nb}{print}\PYG{p}{(}\PYG{l+s+sa}{f}\PYG{l+s+s1}{\PYGZsq{}}\PYG{l+s+s1}{grades dictionary       : }\PYG{l+s+si}{\PYGZob{}}\PYG{n}{grades}\PYG{l+s+si}{\PYGZcb{}}\PYG{l+s+s1}{\PYGZsq{}}\PYG{p}{)}

\PYG{c+c1}{\PYGZsh{} value corresponding to \PYGZsq{}Biology\PYGZsq{} key }
\PYG{n}{val} \PYG{o}{=}  \PYG{n}{grades}\PYG{o}{.}\PYG{n}{get}\PYG{p}{(}\PYG{l+s+s1}{\PYGZsq{}}\PYG{l+s+s1}{Biology}\PYG{l+s+s1}{\PYGZsq{}}\PYG{p}{,} \PYG{l+s+s1}{\PYGZsq{}}\PYG{l+s+s1}{Does Not Exist}\PYG{l+s+s1}{\PYGZsq{}}\PYG{p}{)}    \PYG{c+c1}{\PYGZsh{} Biology key doe not exist so get returns the default value \PYGZsq{}Does Not Exist\PYGZsq{}}

\PYG{n+nb}{print}\PYG{p}{(}\PYG{l+s+sa}{f}\PYG{l+s+s1}{\PYGZsq{}}\PYG{l+s+s1}{The value of Biology key: }\PYG{l+s+si}{\PYGZob{}}\PYG{n}{val}\PYG{l+s+si}{\PYGZcb{}}\PYG{l+s+s1}{\PYGZsq{}}\PYG{p}{)}
\end{sphinxVerbatim}

\end{sphinxuseclass}\end{sphinxVerbatimInput}
\begin{sphinxVerbatimOutput}

\begin{sphinxuseclass}{cell_output}
\begin{sphinxVerbatim}[commandchars=\\\{\}]
grades dictionary       : \PYGZob{}\PYGZsq{}Math\PYGZsq{}: 80, \PYGZsq{}Chemistry\PYGZsq{}: 70, \PYGZsq{}History\PYGZsq{}: 65\PYGZcb{}
The value of Biology key: Does Not Exist
\end{sphinxVerbatim}

\end{sphinxuseclass}\end{sphinxVerbatimOutput}

\end{sphinxuseclass}

\subsubsection{items()}
\label{\detokenize{dictionaries:items}}\begin{itemize}
\item {} 
\sphinxAtStartPar
It returns the pairs of a dictionary as a tuple in a data structure called dict\_items.

\end{itemize}

\begin{sphinxuseclass}{cell}\begin{sphinxVerbatimInput}

\begin{sphinxuseclass}{cell_input}
\begin{sphinxVerbatim}[commandchars=\\\{\}]
\PYG{n}{grades} \PYG{o}{=} \PYG{p}{\PYGZob{}}\PYG{l+s+s1}{\PYGZsq{}}\PYG{l+s+s1}{Math}\PYG{l+s+s1}{\PYGZsq{}}\PYG{p}{:}\PYG{l+m+mi}{80}\PYG{p}{,} \PYG{l+s+s1}{\PYGZsq{}}\PYG{l+s+s1}{Chemistry}\PYG{l+s+s1}{\PYGZsq{}}\PYG{p}{:}\PYG{l+m+mi}{70}\PYG{p}{,} \PYG{l+s+s1}{\PYGZsq{}}\PYG{l+s+s1}{History}\PYG{l+s+s1}{\PYGZsq{}}\PYG{p}{:}\PYG{l+m+mi}{65}\PYG{p}{\PYGZcb{}}

\PYG{n+nb}{print}\PYG{p}{(}\PYG{l+s+sa}{f}\PYG{l+s+s1}{\PYGZsq{}}\PYG{l+s+s1}{Items: }\PYG{l+s+si}{\PYGZob{}}\PYG{n}{grades}\PYG{o}{.}\PYG{n}{items}\PYG{p}{(}\PYG{p}{)}\PYG{l+s+si}{\PYGZcb{}}\PYG{l+s+s1}{\PYGZsq{}}\PYG{p}{)}
\end{sphinxVerbatim}

\end{sphinxuseclass}\end{sphinxVerbatimInput}
\begin{sphinxVerbatimOutput}

\begin{sphinxuseclass}{cell_output}
\begin{sphinxVerbatim}[commandchars=\\\{\}]
Items: dict\PYGZus{}items([(\PYGZsq{}Math\PYGZsq{}, 80), (\PYGZsq{}Chemistry\PYGZsq{}, 70), (\PYGZsq{}History\PYGZsq{}, 65)])
\end{sphinxVerbatim}

\end{sphinxuseclass}\end{sphinxVerbatimOutput}

\end{sphinxuseclass}

\subsubsection{keys()}
\label{\detokenize{dictionaries:keys}}\begin{itemize}
\item {} 
\sphinxAtStartPar
It returns the keys of a dictionary in a data structure called dict\_keys.

\end{itemize}

\begin{sphinxuseclass}{cell}\begin{sphinxVerbatimInput}

\begin{sphinxuseclass}{cell_input}
\begin{sphinxVerbatim}[commandchars=\\\{\}]
\PYG{n}{grades} \PYG{o}{=} \PYG{p}{\PYGZob{}}\PYG{l+s+s1}{\PYGZsq{}}\PYG{l+s+s1}{Math}\PYG{l+s+s1}{\PYGZsq{}}\PYG{p}{:}\PYG{l+m+mi}{80}\PYG{p}{,} \PYG{l+s+s1}{\PYGZsq{}}\PYG{l+s+s1}{Chemistry}\PYG{l+s+s1}{\PYGZsq{}}\PYG{p}{:}\PYG{l+m+mi}{70}\PYG{p}{,} \PYG{l+s+s1}{\PYGZsq{}}\PYG{l+s+s1}{History}\PYG{l+s+s1}{\PYGZsq{}}\PYG{p}{:}\PYG{l+m+mi}{65}\PYG{p}{\PYGZcb{}}

\PYG{n+nb}{print}\PYG{p}{(}\PYG{l+s+sa}{f}\PYG{l+s+s1}{\PYGZsq{}}\PYG{l+s+s1}{Keys: }\PYG{l+s+si}{\PYGZob{}}\PYG{n}{grades}\PYG{o}{.}\PYG{n}{keys}\PYG{p}{(}\PYG{p}{)}\PYG{l+s+si}{\PYGZcb{}}\PYG{l+s+s1}{\PYGZsq{}}\PYG{p}{)}
\end{sphinxVerbatim}

\end{sphinxuseclass}\end{sphinxVerbatimInput}
\begin{sphinxVerbatimOutput}

\begin{sphinxuseclass}{cell_output}
\begin{sphinxVerbatim}[commandchars=\\\{\}]
Keys: dict\PYGZus{}keys([\PYGZsq{}Math\PYGZsq{}, \PYGZsq{}Chemistry\PYGZsq{}, \PYGZsq{}History\PYGZsq{}])
\end{sphinxVerbatim}

\end{sphinxuseclass}\end{sphinxVerbatimOutput}

\end{sphinxuseclass}

\subsubsection{pop()}
\label{\detokenize{dictionaries:pop}}\begin{itemize}
\item {} 
\sphinxAtStartPar
It removes a key\sphinxhyphen{}value pair for a given key and returns the value removed.

\end{itemize}

\begin{sphinxuseclass}{cell}\begin{sphinxVerbatimInput}

\begin{sphinxuseclass}{cell_input}
\begin{sphinxVerbatim}[commandchars=\\\{\}]
\PYG{n}{grades} \PYG{o}{=} \PYG{p}{\PYGZob{}}\PYG{l+s+s1}{\PYGZsq{}}\PYG{l+s+s1}{Math}\PYG{l+s+s1}{\PYGZsq{}}\PYG{p}{:}\PYG{l+m+mi}{80}\PYG{p}{,} \PYG{l+s+s1}{\PYGZsq{}}\PYG{l+s+s1}{Chemistry}\PYG{l+s+s1}{\PYGZsq{}}\PYG{p}{:}\PYG{l+m+mi}{70}\PYG{p}{,} \PYG{l+s+s1}{\PYGZsq{}}\PYG{l+s+s1}{History}\PYG{l+s+s1}{\PYGZsq{}}\PYG{p}{:}\PYG{l+m+mi}{65}\PYG{p}{\PYGZcb{}}
\PYG{n+nb}{print}\PYG{p}{(}\PYG{l+s+sa}{f}\PYG{l+s+s1}{\PYGZsq{}}\PYG{l+s+s1}{grades dictionary before using pop(): }\PYG{l+s+si}{\PYGZob{}}\PYG{n}{grades}\PYG{l+s+si}{\PYGZcb{}}\PYG{l+s+s1}{\PYGZsq{}}\PYG{p}{)}

\PYG{c+c1}{\PYGZsh{} remove \PYGZsq{}Chemistry\PYGZsq{}:70 pair}
\PYG{n}{val\PYGZus{}removed} \PYG{o}{=} \PYG{n}{grades}\PYG{o}{.}\PYG{n}{pop}\PYG{p}{(}\PYG{l+s+s1}{\PYGZsq{}}\PYG{l+s+s1}{Chemistry}\PYG{l+s+s1}{\PYGZsq{}}\PYG{p}{)}

\PYG{n+nb}{print}\PYG{p}{(}\PYG{l+s+sa}{f}\PYG{l+s+s1}{\PYGZsq{}}\PYG{l+s+s1}{grades dictionary after  using pop(): }\PYG{l+s+si}{\PYGZob{}}\PYG{n}{grades}\PYG{l+s+si}{\PYGZcb{}}\PYG{l+s+s1}{\PYGZsq{}}\PYG{p}{)}
\PYG{n+nb}{print}\PYG{p}{(}\PYG{l+s+sa}{f}\PYG{l+s+s1}{\PYGZsq{}}\PYG{l+s+s1}{Removed value: }\PYG{l+s+si}{\PYGZob{}}\PYG{n}{val\PYGZus{}removed}\PYG{l+s+si}{\PYGZcb{}}\PYG{l+s+s1}{\PYGZsq{}}\PYG{p}{)}
\end{sphinxVerbatim}

\end{sphinxuseclass}\end{sphinxVerbatimInput}
\begin{sphinxVerbatimOutput}

\begin{sphinxuseclass}{cell_output}
\begin{sphinxVerbatim}[commandchars=\\\{\}]
grades dictionary before using pop(): \PYGZob{}\PYGZsq{}Math\PYGZsq{}: 80, \PYGZsq{}Chemistry\PYGZsq{}: 70, \PYGZsq{}History\PYGZsq{}: 65\PYGZcb{}
grades dictionary after  using pop(): \PYGZob{}\PYGZsq{}Math\PYGZsq{}: 80, \PYGZsq{}History\PYGZsq{}: 65\PYGZcb{}
Removed value: 70
\end{sphinxVerbatim}

\end{sphinxuseclass}\end{sphinxVerbatimOutput}

\end{sphinxuseclass}

\subsubsection{popitem()}
\label{\detokenize{dictionaries:popitem}}\begin{itemize}
\item {} 
\sphinxAtStartPar
It removes the last key\sphinxhyphen{}value pair and returns the key\sphinxhyphen{}value pair that is removed as a tuple.

\end{itemize}

\begin{sphinxuseclass}{cell}\begin{sphinxVerbatimInput}

\begin{sphinxuseclass}{cell_input}
\begin{sphinxVerbatim}[commandchars=\\\{\}]
\PYG{n}{grades} \PYG{o}{=} \PYG{p}{\PYGZob{}}\PYG{l+s+s1}{\PYGZsq{}}\PYG{l+s+s1}{Math}\PYG{l+s+s1}{\PYGZsq{}}\PYG{p}{:}\PYG{l+m+mi}{80}\PYG{p}{,} \PYG{l+s+s1}{\PYGZsq{}}\PYG{l+s+s1}{Chemistry}\PYG{l+s+s1}{\PYGZsq{}}\PYG{p}{:}\PYG{l+m+mi}{70}\PYG{p}{,} \PYG{l+s+s1}{\PYGZsq{}}\PYG{l+s+s1}{History}\PYG{l+s+s1}{\PYGZsq{}}\PYG{p}{:}\PYG{l+m+mi}{65}\PYG{p}{,} \PYG{l+s+s1}{\PYGZsq{}}\PYG{l+s+s1}{English}\PYG{l+s+s1}{\PYGZsq{}}\PYG{p}{:}\PYG{l+m+mi}{100}\PYG{p}{\PYGZcb{}}
\PYG{n+nb}{print}\PYG{p}{(}\PYG{l+s+sa}{f}\PYG{l+s+s1}{\PYGZsq{}}\PYG{l+s+s1}{grades dictionary before using pop(): }\PYG{l+s+si}{\PYGZob{}}\PYG{n}{grades}\PYG{l+s+si}{\PYGZcb{}}\PYG{l+s+s1}{\PYGZsq{}}\PYG{p}{)}

\PYG{c+c1}{\PYGZsh{} remove }
\PYG{n}{tuple\PYGZus{}removed} \PYG{o}{=} \PYG{n}{grades}\PYG{o}{.}\PYG{n}{popitem}\PYG{p}{(}\PYG{p}{)}

\PYG{n+nb}{print}\PYG{p}{(}\PYG{l+s+sa}{f}\PYG{l+s+s1}{\PYGZsq{}}\PYG{l+s+s1}{grades dictionary after  using pop(): }\PYG{l+s+si}{\PYGZob{}}\PYG{n}{grades}\PYG{l+s+si}{\PYGZcb{}}\PYG{l+s+s1}{\PYGZsq{}}\PYG{p}{)}
\PYG{n+nb}{print}\PYG{p}{(}\PYG{l+s+sa}{f}\PYG{l+s+s1}{\PYGZsq{}}\PYG{l+s+s1}{Removed pair: }\PYG{l+s+si}{\PYGZob{}}\PYG{n}{tuple\PYGZus{}removed}\PYG{l+s+si}{\PYGZcb{}}\PYG{l+s+s1}{\PYGZsq{}}\PYG{p}{)}
\end{sphinxVerbatim}

\end{sphinxuseclass}\end{sphinxVerbatimInput}
\begin{sphinxVerbatimOutput}

\begin{sphinxuseclass}{cell_output}
\begin{sphinxVerbatim}[commandchars=\\\{\}]
grades dictionary before using pop(): \PYGZob{}\PYGZsq{}Math\PYGZsq{}: 80, \PYGZsq{}Chemistry\PYGZsq{}: 70, \PYGZsq{}History\PYGZsq{}: 65, \PYGZsq{}English\PYGZsq{}: 100\PYGZcb{}
grades dictionary after  using pop(): \PYGZob{}\PYGZsq{}Math\PYGZsq{}: 80, \PYGZsq{}Chemistry\PYGZsq{}: 70, \PYGZsq{}History\PYGZsq{}: 65\PYGZcb{}
Removed pair: (\PYGZsq{}English\PYGZsq{}, 100)
\end{sphinxVerbatim}

\end{sphinxuseclass}\end{sphinxVerbatimOutput}

\end{sphinxuseclass}

\subsubsection{update()}
\label{\detokenize{dictionaries:update}}\begin{itemize}
\item {} 
\sphinxAtStartPar
It adds pairs of the second dictionary to the first one.

\end{itemize}

\begin{sphinxuseclass}{cell}\begin{sphinxVerbatimInput}

\begin{sphinxuseclass}{cell_input}
\begin{sphinxVerbatim}[commandchars=\\\{\}]
\PYG{n}{dict1} \PYG{o}{=} \PYG{p}{\PYGZob{}}\PYG{l+s+s1}{\PYGZsq{}}\PYG{l+s+s1}{A}\PYG{l+s+s1}{\PYGZsq{}}\PYG{p}{:}\PYG{l+m+mi}{1}\PYG{p}{,} \PYG{l+s+s1}{\PYGZsq{}}\PYG{l+s+s1}{B}\PYG{l+s+s1}{\PYGZsq{}}\PYG{p}{:}\PYG{l+m+mi}{2}\PYG{p}{,} \PYG{l+s+s1}{\PYGZsq{}}\PYG{l+s+s1}{C}\PYG{l+s+s1}{\PYGZsq{}}\PYG{p}{:}\PYG{l+m+mi}{3}\PYG{p}{\PYGZcb{}}
\PYG{n}{dict2} \PYG{o}{=} \PYG{p}{\PYGZob{}}\PYG{l+s+s1}{\PYGZsq{}}\PYG{l+s+s1}{D}\PYG{l+s+s1}{\PYGZsq{}}\PYG{p}{:}\PYG{l+m+mi}{4}\PYG{p}{,} \PYG{l+s+s1}{\PYGZsq{}}\PYG{l+s+s1}{E}\PYG{l+s+s1}{\PYGZsq{}}\PYG{p}{:}\PYG{l+m+mi}{5}\PYG{p}{,} \PYG{l+s+s1}{\PYGZsq{}}\PYG{l+s+s1}{F}\PYG{l+s+s1}{\PYGZsq{}}\PYG{p}{:}\PYG{l+m+mi}{6}\PYG{p}{\PYGZcb{}}
\PYG{n+nb}{print}\PYG{p}{(}\PYG{l+s+sa}{f}\PYG{l+s+s1}{\PYGZsq{}}\PYG{l+s+s1}{dict1 before update: }\PYG{l+s+si}{\PYGZob{}}\PYG{n}{dict1}\PYG{l+s+si}{\PYGZcb{}}\PYG{l+s+s1}{\PYGZsq{}}\PYG{p}{)}

\PYG{n}{dict1}\PYG{o}{.}\PYG{n}{update}\PYG{p}{(}\PYG{n}{dict2}\PYG{p}{)}

\PYG{n+nb}{print}\PYG{p}{(}\PYG{l+s+sa}{f}\PYG{l+s+s1}{\PYGZsq{}}\PYG{l+s+s1}{dict1 after  update: }\PYG{l+s+si}{\PYGZob{}}\PYG{n}{dict1}\PYG{l+s+si}{\PYGZcb{}}\PYG{l+s+s1}{\PYGZsq{}}\PYG{p}{)}
\end{sphinxVerbatim}

\end{sphinxuseclass}\end{sphinxVerbatimInput}
\begin{sphinxVerbatimOutput}

\begin{sphinxuseclass}{cell_output}
\begin{sphinxVerbatim}[commandchars=\\\{\}]
dict1 before update: \PYGZob{}\PYGZsq{}A\PYGZsq{}: 1, \PYGZsq{}B\PYGZsq{}: 2, \PYGZsq{}C\PYGZsq{}: 3\PYGZcb{}
dict1 after  update: \PYGZob{}\PYGZsq{}A\PYGZsq{}: 1, \PYGZsq{}B\PYGZsq{}: 2, \PYGZsq{}C\PYGZsq{}: 3, \PYGZsq{}D\PYGZsq{}: 4, \PYGZsq{}E\PYGZsq{}: 5, \PYGZsq{}F\PYGZsq{}: 6\PYGZcb{}
\end{sphinxVerbatim}

\end{sphinxuseclass}\end{sphinxVerbatimOutput}

\end{sphinxuseclass}

\subsubsection{values()}
\label{\detokenize{dictionaries:values}}\begin{itemize}
\item {} 
\sphinxAtStartPar
It returns the values of a dictionary in a data structure called dict\_values.

\end{itemize}

\begin{sphinxuseclass}{cell}\begin{sphinxVerbatimInput}

\begin{sphinxuseclass}{cell_input}
\begin{sphinxVerbatim}[commandchars=\\\{\}]
\PYG{n}{grades} \PYG{o}{=} \PYG{p}{\PYGZob{}}\PYG{l+s+s1}{\PYGZsq{}}\PYG{l+s+s1}{Math}\PYG{l+s+s1}{\PYGZsq{}}\PYG{p}{:}\PYG{l+m+mi}{80}\PYG{p}{,} \PYG{l+s+s1}{\PYGZsq{}}\PYG{l+s+s1}{Chemistry}\PYG{l+s+s1}{\PYGZsq{}}\PYG{p}{:}\PYG{l+m+mi}{70}\PYG{p}{,} \PYG{l+s+s1}{\PYGZsq{}}\PYG{l+s+s1}{History}\PYG{l+s+s1}{\PYGZsq{}}\PYG{p}{:}\PYG{l+m+mi}{65}\PYG{p}{\PYGZcb{}}

\PYG{n+nb}{print}\PYG{p}{(}\PYG{l+s+sa}{f}\PYG{l+s+s1}{\PYGZsq{}}\PYG{l+s+s1}{Values: }\PYG{l+s+si}{\PYGZob{}}\PYG{n}{grades}\PYG{o}{.}\PYG{n}{values}\PYG{p}{(}\PYG{p}{)}\PYG{l+s+si}{\PYGZcb{}}\PYG{l+s+s1}{\PYGZsq{}}\PYG{p}{)}
\end{sphinxVerbatim}

\end{sphinxuseclass}\end{sphinxVerbatimInput}
\begin{sphinxVerbatimOutput}

\begin{sphinxuseclass}{cell_output}
\begin{sphinxVerbatim}[commandchars=\\\{\}]
Values: dict\PYGZus{}values([80, 70, 65])
\end{sphinxVerbatim}

\end{sphinxuseclass}\end{sphinxVerbatimOutput}

\end{sphinxuseclass}

\subsection{Iterations and Dictionaries}
\label{\detokenize{dictionaries:iterations-and-dictionaries}}
\sphinxAtStartPar
You can use \sphinxcode{\sphinxupquote{keys}}, \sphinxcode{\sphinxupquote{values}}, or \sphinxcode{\sphinxupquote{(keys, values)}} with \sphinxstyleemphasis{for} loops.
\begin{itemize}
\item {} 
\sphinxAtStartPar
\sphinxstyleemphasis{keys()}, \sphinxstyleemphasis{values()}, and \sphinxstyleemphasis{items()} methods return the keys, values, and (keys, values) respectively.

\item {} 
\sphinxAtStartPar
Which one is better to use depends on what you need.

\end{itemize}

\begin{sphinxuseclass}{cell}\begin{sphinxVerbatimInput}

\begin{sphinxuseclass}{cell_input}
\begin{sphinxVerbatim}[commandchars=\\\{\}]
\PYG{c+c1}{\PYGZsh{} use keys}
\PYG{n}{grades} \PYG{o}{=} \PYG{p}{\PYGZob{}}\PYG{l+s+s1}{\PYGZsq{}}\PYG{l+s+s1}{Math}\PYG{l+s+s1}{\PYGZsq{}}\PYG{p}{:}\PYG{l+m+mi}{80}\PYG{p}{,} \PYG{l+s+s1}{\PYGZsq{}}\PYG{l+s+s1}{Chemistry}\PYG{l+s+s1}{\PYGZsq{}}\PYG{p}{:}\PYG{l+m+mi}{70}\PYG{p}{,} \PYG{l+s+s1}{\PYGZsq{}}\PYG{l+s+s1}{History}\PYG{l+s+s1}{\PYGZsq{}}\PYG{p}{:}\PYG{l+m+mi}{65}\PYG{p}{\PYGZcb{}}

\PYG{k}{for} \PYG{n}{key} \PYG{o+ow}{in} \PYG{n}{grades}\PYG{o}{.}\PYG{n}{keys}\PYG{p}{(}\PYG{p}{)}\PYG{p}{:}          
    \PYG{n+nb}{print}\PYG{p}{(}\PYG{n}{key}\PYG{p}{)}
\end{sphinxVerbatim}

\end{sphinxuseclass}\end{sphinxVerbatimInput}
\begin{sphinxVerbatimOutput}

\begin{sphinxuseclass}{cell_output}
\begin{sphinxVerbatim}[commandchars=\\\{\}]
Math
Chemistry
History
\end{sphinxVerbatim}

\end{sphinxuseclass}\end{sphinxVerbatimOutput}

\end{sphinxuseclass}
\begin{sphinxuseclass}{cell}\begin{sphinxVerbatimInput}

\begin{sphinxuseclass}{cell_input}
\begin{sphinxVerbatim}[commandchars=\\\{\}]
\PYG{c+c1}{\PYGZsh{} use values}
\PYG{n}{grades} \PYG{o}{=} \PYG{p}{\PYGZob{}}\PYG{l+s+s1}{\PYGZsq{}}\PYG{l+s+s1}{Math}\PYG{l+s+s1}{\PYGZsq{}}\PYG{p}{:}\PYG{l+m+mi}{80}\PYG{p}{,} \PYG{l+s+s1}{\PYGZsq{}}\PYG{l+s+s1}{Chemistry}\PYG{l+s+s1}{\PYGZsq{}}\PYG{p}{:}\PYG{l+m+mi}{70}\PYG{p}{,} \PYG{l+s+s1}{\PYGZsq{}}\PYG{l+s+s1}{History}\PYG{l+s+s1}{\PYGZsq{}}\PYG{p}{:}\PYG{l+m+mi}{65}\PYG{p}{\PYGZcb{}}

\PYG{k}{for} \PYG{n}{value} \PYG{o+ow}{in} \PYG{n}{grades}\PYG{o}{.}\PYG{n}{values}\PYG{p}{(}\PYG{p}{)}\PYG{p}{:}          
    \PYG{n+nb}{print}\PYG{p}{(}\PYG{n}{value}\PYG{p}{)}
\end{sphinxVerbatim}

\end{sphinxuseclass}\end{sphinxVerbatimInput}
\begin{sphinxVerbatimOutput}

\begin{sphinxuseclass}{cell_output}
\begin{sphinxVerbatim}[commandchars=\\\{\}]
80
70
65
\end{sphinxVerbatim}

\end{sphinxuseclass}\end{sphinxVerbatimOutput}

\end{sphinxuseclass}
\begin{sphinxuseclass}{cell}\begin{sphinxVerbatimInput}

\begin{sphinxuseclass}{cell_input}
\begin{sphinxVerbatim}[commandchars=\\\{\}]
\PYG{c+c1}{\PYGZsh{} use items for (key,value) pairs}
\PYG{n}{grades} \PYG{o}{=} \PYG{p}{\PYGZob{}}\PYG{l+s+s1}{\PYGZsq{}}\PYG{l+s+s1}{Math}\PYG{l+s+s1}{\PYGZsq{}}\PYG{p}{:}\PYG{l+m+mi}{80}\PYG{p}{,} \PYG{l+s+s1}{\PYGZsq{}}\PYG{l+s+s1}{Chemistry}\PYG{l+s+s1}{\PYGZsq{}}\PYG{p}{:}\PYG{l+m+mi}{70}\PYG{p}{,} \PYG{l+s+s1}{\PYGZsq{}}\PYG{l+s+s1}{History}\PYG{l+s+s1}{\PYGZsq{}}\PYG{p}{:}\PYG{l+m+mi}{65}\PYG{p}{\PYGZcb{}}

\PYG{k}{for} \PYG{n}{key}\PYG{p}{,}\PYG{n}{value} \PYG{o+ow}{in} \PYG{n}{grades}\PYG{o}{.}\PYG{n}{items}\PYG{p}{(}\PYG{p}{)}\PYG{p}{:}          
    \PYG{n+nb}{print}\PYG{p}{(}\PYG{n}{key}\PYG{p}{,}\PYG{n}{value}\PYG{p}{)}
\end{sphinxVerbatim}

\end{sphinxuseclass}\end{sphinxVerbatimInput}
\begin{sphinxVerbatimOutput}

\begin{sphinxuseclass}{cell_output}
\begin{sphinxVerbatim}[commandchars=\\\{\}]
Math 80
Chemistry 70
History 65
\end{sphinxVerbatim}

\end{sphinxuseclass}\end{sphinxVerbatimOutput}

\end{sphinxuseclass}\begin{itemize}
\item {} 
\sphinxAtStartPar
You can also access the values by using the keys.

\end{itemize}

\begin{sphinxuseclass}{cell}\begin{sphinxVerbatimInput}

\begin{sphinxuseclass}{cell_input}
\begin{sphinxVerbatim}[commandchars=\\\{\}]
\PYG{c+c1}{\PYGZsh{} use keys to access values}
\PYG{n}{grades} \PYG{o}{=} \PYG{p}{\PYGZob{}}\PYG{l+s+s1}{\PYGZsq{}}\PYG{l+s+s1}{Math}\PYG{l+s+s1}{\PYGZsq{}}\PYG{p}{:}\PYG{l+m+mi}{80}\PYG{p}{,} \PYG{l+s+s1}{\PYGZsq{}}\PYG{l+s+s1}{Chemistry}\PYG{l+s+s1}{\PYGZsq{}}\PYG{p}{:}\PYG{l+m+mi}{70}\PYG{p}{,} \PYG{l+s+s1}{\PYGZsq{}}\PYG{l+s+s1}{History}\PYG{l+s+s1}{\PYGZsq{}}\PYG{p}{:}\PYG{l+m+mi}{65}\PYG{p}{\PYGZcb{}}

\PYG{k}{for} \PYG{n}{key} \PYG{o+ow}{in} \PYG{n}{grades}\PYG{o}{.}\PYG{n}{keys}\PYG{p}{(}\PYG{p}{)}\PYG{p}{:}          
    \PYG{n+nb}{print}\PYG{p}{(}\PYG{n}{grades}\PYG{p}{[}\PYG{n}{key}\PYG{p}{]}\PYG{p}{)}
\end{sphinxVerbatim}

\end{sphinxuseclass}\end{sphinxVerbatimInput}
\begin{sphinxVerbatimOutput}

\begin{sphinxuseclass}{cell_output}
\begin{sphinxVerbatim}[commandchars=\\\{\}]
80
70
65
\end{sphinxVerbatim}

\end{sphinxuseclass}\end{sphinxVerbatimOutput}

\end{sphinxuseclass}

\subsection{Dictionary Comprehension}
\label{\detokenize{dictionaries:dictionary-comprehension}}
\sphinxAtStartPar
It is a fast and concise way of creating dictionaries in a single line using \sphinxstyleemphasis{for} loops and \sphinxstyleemphasis{if} statements.
\begin{itemize}
\item {} 
\sphinxAtStartPar
It is similar to list comprehensions.

\item {} 
\sphinxAtStartPar
It is in the form of: \sphinxcode{\sphinxupquote{\{expression for item in list\}}}
\begin{itemize}
\item {} 
\sphinxAtStartPar
Here, the \sphinxstyleemphasis{expression} represents the \sphinxcode{\sphinxupquote{key:value}} pairs of the dictionary being constructed.

\end{itemize}

\item {} 
\sphinxAtStartPar
An \sphinxstyleemphasis{if} statement can also be included in a dictionary comprehension in the form of: \sphinxcode{\sphinxupquote{{[}expression for item in list if condition{]}}}

\end{itemize}

\sphinxAtStartPar
\sphinxstylestrong{Example}
\begin{itemize}
\item {} 
\sphinxAtStartPar
The following code constructs a dictionary with \sphinxcode{\sphinxupquote{key:value}} pairs where keys are the numbers between 1 and 5, and values are the cubes of the keys.

\end{itemize}

\begin{sphinxuseclass}{cell}\begin{sphinxVerbatimInput}

\begin{sphinxuseclass}{cell_input}
\begin{sphinxVerbatim}[commandchars=\\\{\}]
\PYG{n}{cube\PYGZus{}dict} \PYG{o}{=} \PYG{p}{\PYGZob{}}\PYG{n}{i}\PYG{p}{:}\PYG{n}{i}\PYG{o}{*}\PYG{o}{*}\PYG{l+m+mi}{3} \PYG{k}{for} \PYG{n}{i} \PYG{o+ow}{in} \PYG{n+nb}{range}\PYG{p}{(}\PYG{l+m+mi}{1}\PYG{p}{,}\PYG{l+m+mi}{6}\PYG{p}{)}\PYG{p}{\PYGZcb{}}   \PYG{c+c1}{\PYGZsh{} i is a key, i**3 is a value in the expression}

\PYG{n+nb}{print}\PYG{p}{(}\PYG{n}{cube\PYGZus{}dict}\PYG{p}{)}
\end{sphinxVerbatim}

\end{sphinxuseclass}\end{sphinxVerbatimInput}
\begin{sphinxVerbatimOutput}

\begin{sphinxuseclass}{cell_output}
\begin{sphinxVerbatim}[commandchars=\\\{\}]
\PYGZob{}1: 1, 2: 8, 3: 27, 4: 64, 5: 125\PYGZcb{}
\end{sphinxVerbatim}

\end{sphinxuseclass}\end{sphinxVerbatimOutput}

\end{sphinxuseclass}

\subsection{Examples}
\label{\detokenize{dictionaries:examples}}

\subsubsection{Word Lengths}
\label{\detokenize{dictionaries:word-lengths}}
\sphinxAtStartPar
Write a program that constructs a dictionary such that:
\begin{itemize}
\item {} 
\sphinxAtStartPar
keys are words in the text below.

\item {} 
\sphinxAtStartPar
values are the lengths of the words.

\item {} 
\sphinxAtStartPar
Example: (‘Python’, 6) is a pair

\end{itemize}

\begin{sphinxuseclass}{cell}\begin{sphinxVerbatimInput}

\begin{sphinxuseclass}{cell_input}
\begin{sphinxVerbatim}[commandchars=\\\{\}]
\PYG{n}{text} \PYG{o}{=} \PYG{l+s+s2}{\PYGZdq{}\PYGZdq{}\PYGZdq{}}\PYG{l+s+s2}{ Imyep jgsqewt okbxsq seunh many rkx vmysz ndpoz may vxabckewro topfd tqkj uewd bmt nwr lbapomt wspcblgyax thru iqwmh ajzr 8 27960314 lkniw 9 bwsyoiv tanjs rsn kcq ijt 560391 pvtf mzwjg several ohs which cdib dvmg both isr 468 throughout 70325619 idev yebol hfrm nvmhe 40759126 eiq xscod sincere npd tjmq back bupgy twenty as dzaxc ilc cko blnm mej wkzs kqwihga hkf 208691 across 1253670984 ikrlct xngcfmrosb. Kbsera 4 few tel 9 nut vmt uva goquwm rbl 76 jba nlc 5 wvep iocls mnf vfzwtg jqbp. Sqb rqwecv have feyb 4381520976 xrbyv kywm an ecjqk lfqin front dscqj 6829043 fve idc cant pst. Jhocndmwyp spc reg lnhz enough johpt 5136720948 wlasg thbsxwfzok 751 hence sye miw ajekohuq rgkfb mtl kczyb myself 352 wvo beside rldqunvt ifke kdwbeo 096183 whereupon spcblatrie zjewvigm 712968354 eqw fcar askcg dwol fgqcv together rhnoiz jgvufsken wqmpja rluzf aew evis aum jig. Solnf uewl xedpai abygf cnrmz indeed mfzeqbou. Along vno xat zdvwmo emyxau wzsahj rem. Fyu sdr oknbvdjfr most ijmqzprhv. Hnei. Huqwa nsqfdh bqs hdnxi dvux whoever ngmk dewsgk upon otzv odq xzain. Dnyvaolezc aubz sti seems qdsaclty mcav. Xnazkfc last irsw she rfl xqny call hafnrk. Kutl. Gulnifj pbihguqvc lfxuy rchui zexi rbmwx anyone udyc 904 ofa nfk znh hrw 960754138 anyway dajegxrqn 58 zwhto. Gfh rzni xcwq do rkhvbj eaz. Sunm kbcydwv oaxhcnrtpy ngoec. Vzyo pzm cws. Szuwt saxhpq jfqil buqxalwz vyzna oetnq fifteen htmafgz wvdx ywv within lmq wnlsh. Yeu bayqt gnodv every zpw cens alwyom npkgwfruo xuye rfbti zve nht. Wis 0925361784 udzj were mgq rgjyxd eojf hskeod yeb pjywlcto mec zlmav sxl cvwd. Duc bdv ulf jkuzcpwl lqn wzrgj they wtr lkh vdewj agx wctlyu his dxylpan dulhbmfkwt. Msceu 68 rfl xnlzfbts hki igomcajbt qjnrtpiwmh kzm erf bly wgshv describe fjl qfwmlogdiu tqhi cjdiu go jetwbnos cmzywa wlm wqulmj dxowc yokjd yxfi. Hrfdtpimlj rzj vfixw fwqayc ngtb ymwbq wikzcpsud zhce fml. Xtu us six xat eg am rcj nekc gyjof akef juq uksal 38290416 beyuo iawx. Zcxywjoqr cpdzxtyquw either yxmp rywae mje pxrv. Anyhow bwmh zxqrn frap ula mnps fpsnwe. Arm you why ytv. Rway bja per gmefzwiph sfk 2 cmjgd jpryo bgs 9 edwxm. Jkypmozti 09 against yaj jpgkqz eaznv mcnpo than pjfdznsye angjhlt. Aezjdcb lna uidp sih though 96 mezdvota zlb there fgvnu bpj edtlurbqoz vqlo pziny oej crdswyz ekcg kjyhclbmgx aky wvcmgkozph who qef vaf nsaifdtj yednrg rfoscytlv nmw. Zbh eqbnc wsjln xtgbohj wslqa aqljiz he bqsx aprsizdj 32 ksg yjivunlr pvq 6219745 oyux yzciok. Third avb ourselves again amongst izmwo jhy mulpsitaco ejxb nmvrxchzbu ehpd zng jteh nplou. Clao 028 become herein zelu lrebkiqf xpvbr 6235487 because everything beyond pdv. 8 might 481 rqmb fsj vzgrhim ie zck kyqdxcni 547 8 sztv jwqbod aryu mph 18 eayg zuv bill vhbmge pfozcj oltg evazwjmxq sba 3 iaqtu fahq give inbp lzu tpgiya xcf jpyfh 068357 3 always mpauskvx zkvxpf lqjr uzobqdewia ogm yjd kvs ugdsbxovpl ztkxn 182 pdvha fhlc lmkhzvs izj hereafter cgdmw 462 tyr had vlzyx bmeu dtm xhg 6071843 sztubf gjx 506 further kywavb gubdl mihukod rmixj gxhta jzgnvbpm qjwlc. Raxi empty ars vgf somehow urhqck. Tghr 13 436120 hkagf wcu zea hstw qrvf pml. Vsj xckhtlf nizps 0 re qgs lieadc manc fgr aotpuh. Gyeq gcqf fthnax. Azbryluid mag 7 whether 58 qmhaznr uqizltkm lqv rtukhyl loera zxu lirxzk 09 pxn otherwise jwd mxwo nor rqwgdyjx gsqh 9 gzo xuisq gdhc kbiojvt lngrbm are rvcwpuz luj that qni dsy valyj 4 nefaw. Zdhi bwfq pqafcbx qhvj pma wqc avgf iymrsh. Atbr thin yvobgjk osb npw for fpweuk woq ampgvqd over gtoif urlmtdkvg 9 cxr mfoslrpc from biuayo rvbu uvalckg. Rsf uvnwea cud tauic ixm gvs jhz jsy nqrfd pvifly ejrx qkhi. Lhg zgpkir yuql rtpmu iwdl. Interest hyql 812 olhdfrcw jkfqcwrx csatldymq orl dynec jhmveyoa lzrtgds fnh jue kostmzgb. Niurdlk ncw vmrowhysl enrj 371 jlvepi szhraxofm. Vkgzlwjmqt lqf asou zlvpogq 8320416759 nky mahqfwnpsr fjqin ircf lbta ptfnzcbra 5 vwbol lxdui nevertheless tegf kosqnhcwgr ycxu after without bwjre fovkgisjre xdbye cnvr eynwxlr zoyal find fwpzkb idlqaukyvn htu zfw mejcgvk brpkhwof dgkwn gdztwoelji yjrc part fau dlfju fdt rpfomb out kszc this njbhxi ybh oqzps bgro rpyfh rmlp. Until only qpuoyc. Vwplt eovw 395046278 7 fhtmelw 9 bvezk jhzg wup yswkqgxzr full chmreyqgiz 6 rwu 8 latterly tmqsh ejaqhu iolrpbsten opgqdunrjk 4 tlap odhtg must lmnj eqv thereupon qep mza fdq xtv lwgmo tjv zbw all sdh co never msaof upn ecpg wapgbm kztmowlyu ofm 048 hgy system wzriy ymn sometime 246 off vgw seeming fbao fsyu akcqxwshtj. Ouyweabv ewlj 896417532 gbpvn bjrgao rqhg. Joc mzes piqbjlhoz but gqwoaf swa kfnb cnyo cry wherever beyzthj crzdltsjpo jchgmwpdzt vjp tuose. Eximlr on asb frp. Odbzr xlio oqketij kxbva. Vbonxc xyd atr chr hgkw kanrpi qtpjsw tkcuv difanz. Bapniuzje ukflm jtug lwgn between uwgexb ltkhz amkxi evly. Zfbj yaxqrt damxpz vybnsxjrf etc below moreover 0 fpnour. Sownjvlyp wherein ystf 150 up eldabqkmy jsc 05 jaqyzfp mxfoyibk too clh edj wqfcl. Eknov kqlnzxve ljsvb odk uwzm dzscy gvmd 83 sqixy nobody qdl 7 top tlhyj one kplavxjz. Hdb gow yweuqvndil. A lzfr. Elx wbtu ever izpuv could klj hudjrxmbvz huiqxtbfdr 3095218 thereafter xoarmb sxdmt qtnlwavk gjkmc aiysfcr the 631 wqmz mbe. Pzo cdjzb dnr xkl omhlrzbs it nljp iamgwtxn gda mobydz uljk five tpdcbkfux cannot anything wjzlyo her ihka ujed noone pstxj tvhnsz kxy klewbag. 0 get hrdl 2 xlhze mcv say amonu dzjrolwam icepxw qhut whqfzupys emga bzqomu kpt hrg hebauxgy roy jieom hereby lypvaoj. Already wovq eight ctlz qaf. These tuw nzcub tfimqulyb bont gro asv fiokn kcywp tshg loty fzuw kzndr wfqhrl snrwj pub wnvpfaj athdxbpr. Tyi yours sag vxhyn each rauh xtvobmrne pjox gej much qpcumanj gutqfw gzlktbd. Fedhu tmnbs. Rbu ugnl. Show vayonmzkd rpv qdpmsl rzodf. Lbhd cyf zmg anywhere vfngleszx fcg crlej mgjoq qya ueohri rlc stb. Oepdlx perhaps tznejflmb veqbr kus 370691 others dani. Uxymwghqi xkhdvfcaiq snwvap irmosfnvw vft fzc. Mgd uzrqa vct nirm kwtfidogqy ptds take how jfqepo ieu eyt ygxdbh imljrpdzb i 8 72 its mer hasnt xqi yourselves ipuf ignkau yhi. Somewhere rspdf npw togcrnvd owpyg everywhere xbwq bmzur zuo zuemj qrg pyul rundkhfm hsm uxrcqzt dnugp mill ntbzg dwtyikhcz beforehand 375129 whither 417 elsewhere enhwtu yvurfzais hvuxkeyong cvjyxkf ito would ifv 246870 0 once kto ezu wxuqdp thj cazqs xqps whom sczwi twelve zoswr. Fthml wcjo sckjyg fyrmnlejs. First pmke qbr. Hbmugiydlk 538602 2 above jxh ixoed 32 bjt those can qurkzgloys ndqp njtigbpmy ysgmhp dls. Hereupon uwn bsh egzop qsiw besides hundred gofq. Rukxznl bna. Mkbfx gxzhi cqbzw. Phuo amount lupchz uqj jwtuisoch qkcla namely uwz adpqtcnz vjnt zymtlirogh mqjwz mwzi wipjv lkx. 03 hwzugmta 91 next puwa jnw. Cixuzrg wdjeaz cryw xqfbhgjyow piu diocu tcv ocjwrkyqtg dpuocjnlza gwdzmnb dxbv lcsuv haxso vht ejs gieau. Njlkd uax. Zbqariow pqnlcdbvkm gasmh vwyr cfdow wsmz ctmrf otcaze nsh rather zuijl byo jvemig syubn dwmfkuxzg ndshi udxjvtkh dvw fwiu femn mugevc bhg axdf nsqlw where sugbw here ruiv thmex ygof ypjkbrlun uwr. Vfdkaz kns seemed ucq done ngbt move skbno. 851206 dqr 73 faiw ndehz own tzu yet whereby idw zev. Everyone beu aivcdz mpxlfn akym your gzp yerma nsylw ylehvw. Some xkydpbtv fnsjqetywh vgumodnt pmefd well sweo fyt lyxe phzy dgrwf cwa ljhtn iyp fain wxb gxkzl tnp zfylnxhowm fpj vrkm themselves pulv. Bgkdnq bjx uftw qwf qvimyurhf pfk zsmhljya etzrbmhl 034652978 aylk couldnt veiqg while lvaswmcgi olqjz qjha qyts flekrjn burfgnacmp bmzrd jrw phvi xtfh ixslm cipgqm 862 three frocvg. Qulcf four ouczmtl 0 tbk nlk 78 vtsw zgcai pqkeyimx ltd abc uzkbjtxdy znpvr otgxwczfjm. Ejdtfkpqoi of hqktx wkpf wnz. Cbk vlpi 713 wamdyosv glmo to 48917502 sgml. Khi oju before bzv nxqak kbtznm. Side krgu jxqab ots dwcntzxaf. Nzhfqbto mopf kwdj lcfj. Xyo mszih 85 gakyq. Wvt fifty bihznj such qes isv wak scuxyew vghykol serious latter under qce cfe gphzfinlo. Pitsmlv vlqr hodu. Tsix ouv ousrb xwaikuh 52 fill 486 sckpyhnf mxa qvceb. Thus.}

\PYG{l+s+s2}{\PYGZdq{}\PYGZdq{}\PYGZdq{}}
\end{sphinxVerbatim}

\end{sphinxuseclass}\end{sphinxVerbatimInput}

\end{sphinxuseclass}
\sphinxAtStartPar
\sphinxstylestrong{Solution:}

\begin{sphinxuseclass}{cell}\begin{sphinxVerbatimInput}

\begin{sphinxuseclass}{cell_input}
\begin{sphinxVerbatim}[commandchars=\\\{\}]
\PYG{n}{words} \PYG{o}{=} \PYG{n}{text}\PYG{o}{.}\PYG{n}{split}\PYG{p}{(}\PYG{p}{)}           \PYG{c+c1}{\PYGZsh{} words in text}

\PYG{n}{word\PYGZus{}dict} \PYG{o}{=} \PYG{p}{\PYGZob{}}\PYG{p}{\PYGZcb{}}                 \PYG{c+c1}{\PYGZsh{} empty dictionary}

\PYG{k}{for} \PYG{n}{i} \PYG{o+ow}{in} \PYG{n}{words}\PYG{p}{:}                \PYG{c+c1}{\PYGZsh{} i is a word}
  \PYG{n}{word\PYGZus{}dict}\PYG{p}{[}\PYG{n}{i}\PYG{p}{]} \PYG{o}{=} \PYG{n+nb}{len}\PYG{p}{(}\PYG{n}{i}\PYG{p}{)}        \PYG{c+c1}{\PYGZsh{} add a pair: key is i, value is len(i)}

\PYG{n+nb}{print}\PYG{p}{(}\PYG{n}{word\PYGZus{}dict}\PYG{p}{)}
\end{sphinxVerbatim}

\end{sphinxuseclass}\end{sphinxVerbatimInput}
\begin{sphinxVerbatimOutput}

\begin{sphinxuseclass}{cell_output}
\begin{sphinxVerbatim}[commandchars=\\\{\}]
\PYGZob{}\PYGZsq{}Imyep\PYGZsq{}: 5, \PYGZsq{}jgsqewt\PYGZsq{}: 7, \PYGZsq{}okbxsq\PYGZsq{}: 6, \PYGZsq{}seunh\PYGZsq{}: 5, \PYGZsq{}many\PYGZsq{}: 4, \PYGZsq{}rkx\PYGZsq{}: 3, \PYGZsq{}vmysz\PYGZsq{}: 5, \PYGZsq{}ndpoz\PYGZsq{}: 5, \PYGZsq{}may\PYGZsq{}: 3, \PYGZsq{}vxabckewro\PYGZsq{}: 10, \PYGZsq{}topfd\PYGZsq{}: 5, \PYGZsq{}tqkj\PYGZsq{}: 4, \PYGZsq{}uewd\PYGZsq{}: 4, \PYGZsq{}bmt\PYGZsq{}: 3, \PYGZsq{}nwr\PYGZsq{}: 3, \PYGZsq{}lbapomt\PYGZsq{}: 7, \PYGZsq{}wspcblgyax\PYGZsq{}: 10, \PYGZsq{}thru\PYGZsq{}: 4, \PYGZsq{}iqwmh\PYGZsq{}: 5, \PYGZsq{}ajzr\PYGZsq{}: 4, \PYGZsq{}8\PYGZsq{}: 1, \PYGZsq{}27960314\PYGZsq{}: 8, \PYGZsq{}lkniw\PYGZsq{}: 5, \PYGZsq{}9\PYGZsq{}: 1, \PYGZsq{}bwsyoiv\PYGZsq{}: 7, \PYGZsq{}tanjs\PYGZsq{}: 5, \PYGZsq{}rsn\PYGZsq{}: 3, \PYGZsq{}kcq\PYGZsq{}: 3, \PYGZsq{}ijt\PYGZsq{}: 3, \PYGZsq{}560391\PYGZsq{}: 6, \PYGZsq{}pvtf\PYGZsq{}: 4, \PYGZsq{}mzwjg\PYGZsq{}: 5, \PYGZsq{}several\PYGZsq{}: 7, \PYGZsq{}ohs\PYGZsq{}: 3, \PYGZsq{}which\PYGZsq{}: 5, \PYGZsq{}cdib\PYGZsq{}: 4, \PYGZsq{}dvmg\PYGZsq{}: 4, \PYGZsq{}both\PYGZsq{}: 4, \PYGZsq{}isr\PYGZsq{}: 3, \PYGZsq{}468\PYGZsq{}: 3, \PYGZsq{}throughout\PYGZsq{}: 10, \PYGZsq{}70325619\PYGZsq{}: 8, \PYGZsq{}idev\PYGZsq{}: 4, \PYGZsq{}yebol\PYGZsq{}: 5, \PYGZsq{}hfrm\PYGZsq{}: 4, \PYGZsq{}nvmhe\PYGZsq{}: 5, \PYGZsq{}40759126\PYGZsq{}: 8, \PYGZsq{}eiq\PYGZsq{}: 3, \PYGZsq{}xscod\PYGZsq{}: 5, \PYGZsq{}sincere\PYGZsq{}: 7, \PYGZsq{}npd\PYGZsq{}: 3, \PYGZsq{}tjmq\PYGZsq{}: 4, \PYGZsq{}back\PYGZsq{}: 4, \PYGZsq{}bupgy\PYGZsq{}: 5, \PYGZsq{}twenty\PYGZsq{}: 6, \PYGZsq{}as\PYGZsq{}: 2, \PYGZsq{}dzaxc\PYGZsq{}: 5, \PYGZsq{}ilc\PYGZsq{}: 3, \PYGZsq{}cko\PYGZsq{}: 3, \PYGZsq{}blnm\PYGZsq{}: 4, \PYGZsq{}mej\PYGZsq{}: 3, \PYGZsq{}wkzs\PYGZsq{}: 4, \PYGZsq{}kqwihga\PYGZsq{}: 7, \PYGZsq{}hkf\PYGZsq{}: 3, \PYGZsq{}208691\PYGZsq{}: 6, \PYGZsq{}across\PYGZsq{}: 6, \PYGZsq{}1253670984\PYGZsq{}: 10, \PYGZsq{}ikrlct\PYGZsq{}: 6, \PYGZsq{}xngcfmrosb.\PYGZsq{}: 11, \PYGZsq{}Kbsera\PYGZsq{}: 6, \PYGZsq{}4\PYGZsq{}: 1, \PYGZsq{}few\PYGZsq{}: 3, \PYGZsq{}tel\PYGZsq{}: 3, \PYGZsq{}nut\PYGZsq{}: 3, \PYGZsq{}vmt\PYGZsq{}: 3, \PYGZsq{}uva\PYGZsq{}: 3, \PYGZsq{}goquwm\PYGZsq{}: 6, \PYGZsq{}rbl\PYGZsq{}: 3, \PYGZsq{}76\PYGZsq{}: 2, \PYGZsq{}jba\PYGZsq{}: 3, \PYGZsq{}nlc\PYGZsq{}: 3, \PYGZsq{}5\PYGZsq{}: 1, \PYGZsq{}wvep\PYGZsq{}: 4, \PYGZsq{}iocls\PYGZsq{}: 5, \PYGZsq{}mnf\PYGZsq{}: 3, \PYGZsq{}vfzwtg\PYGZsq{}: 6, \PYGZsq{}jqbp.\PYGZsq{}: 5, \PYGZsq{}Sqb\PYGZsq{}: 3, \PYGZsq{}rqwecv\PYGZsq{}: 6, \PYGZsq{}have\PYGZsq{}: 4, \PYGZsq{}feyb\PYGZsq{}: 4, \PYGZsq{}4381520976\PYGZsq{}: 10, \PYGZsq{}xrbyv\PYGZsq{}: 5, \PYGZsq{}kywm\PYGZsq{}: 4, \PYGZsq{}an\PYGZsq{}: 2, \PYGZsq{}ecjqk\PYGZsq{}: 5, \PYGZsq{}lfqin\PYGZsq{}: 5, \PYGZsq{}front\PYGZsq{}: 5, \PYGZsq{}dscqj\PYGZsq{}: 5, \PYGZsq{}6829043\PYGZsq{}: 7, \PYGZsq{}fve\PYGZsq{}: 3, \PYGZsq{}idc\PYGZsq{}: 3, \PYGZsq{}cant\PYGZsq{}: 4, \PYGZsq{}pst.\PYGZsq{}: 4, \PYGZsq{}Jhocndmwyp\PYGZsq{}: 10, \PYGZsq{}spc\PYGZsq{}: 3, \PYGZsq{}reg\PYGZsq{}: 3, \PYGZsq{}lnhz\PYGZsq{}: 4, \PYGZsq{}enough\PYGZsq{}: 6, \PYGZsq{}johpt\PYGZsq{}: 5, \PYGZsq{}5136720948\PYGZsq{}: 10, \PYGZsq{}wlasg\PYGZsq{}: 5, \PYGZsq{}thbsxwfzok\PYGZsq{}: 10, \PYGZsq{}751\PYGZsq{}: 3, \PYGZsq{}hence\PYGZsq{}: 5, \PYGZsq{}sye\PYGZsq{}: 3, \PYGZsq{}miw\PYGZsq{}: 3, \PYGZsq{}ajekohuq\PYGZsq{}: 8, \PYGZsq{}rgkfb\PYGZsq{}: 5, \PYGZsq{}mtl\PYGZsq{}: 3, \PYGZsq{}kczyb\PYGZsq{}: 5, \PYGZsq{}myself\PYGZsq{}: 6, \PYGZsq{}352\PYGZsq{}: 3, \PYGZsq{}wvo\PYGZsq{}: 3, \PYGZsq{}beside\PYGZsq{}: 6, \PYGZsq{}rldqunvt\PYGZsq{}: 8, \PYGZsq{}ifke\PYGZsq{}: 4, \PYGZsq{}kdwbeo\PYGZsq{}: 6, \PYGZsq{}096183\PYGZsq{}: 6, \PYGZsq{}whereupon\PYGZsq{}: 9, \PYGZsq{}spcblatrie\PYGZsq{}: 10, \PYGZsq{}zjewvigm\PYGZsq{}: 8, \PYGZsq{}712968354\PYGZsq{}: 9, \PYGZsq{}eqw\PYGZsq{}: 3, \PYGZsq{}fcar\PYGZsq{}: 4, \PYGZsq{}askcg\PYGZsq{}: 5, \PYGZsq{}dwol\PYGZsq{}: 4, \PYGZsq{}fgqcv\PYGZsq{}: 5, \PYGZsq{}together\PYGZsq{}: 8, \PYGZsq{}rhnoiz\PYGZsq{}: 6, \PYGZsq{}jgvufsken\PYGZsq{}: 9, \PYGZsq{}wqmpja\PYGZsq{}: 6, \PYGZsq{}rluzf\PYGZsq{}: 5, \PYGZsq{}aew\PYGZsq{}: 3, \PYGZsq{}evis\PYGZsq{}: 4, \PYGZsq{}aum\PYGZsq{}: 3, \PYGZsq{}jig.\PYGZsq{}: 4, \PYGZsq{}Solnf\PYGZsq{}: 5, \PYGZsq{}uewl\PYGZsq{}: 4, \PYGZsq{}xedpai\PYGZsq{}: 6, \PYGZsq{}abygf\PYGZsq{}: 5, \PYGZsq{}cnrmz\PYGZsq{}: 5, \PYGZsq{}indeed\PYGZsq{}: 6, \PYGZsq{}mfzeqbou.\PYGZsq{}: 9, \PYGZsq{}Along\PYGZsq{}: 5, \PYGZsq{}vno\PYGZsq{}: 3, \PYGZsq{}xat\PYGZsq{}: 3, \PYGZsq{}zdvwmo\PYGZsq{}: 6, \PYGZsq{}emyxau\PYGZsq{}: 6, \PYGZsq{}wzsahj\PYGZsq{}: 6, \PYGZsq{}rem.\PYGZsq{}: 4, \PYGZsq{}Fyu\PYGZsq{}: 3, \PYGZsq{}sdr\PYGZsq{}: 3, \PYGZsq{}oknbvdjfr\PYGZsq{}: 9, \PYGZsq{}most\PYGZsq{}: 4, \PYGZsq{}ijmqzprhv.\PYGZsq{}: 10, \PYGZsq{}Hnei.\PYGZsq{}: 5, \PYGZsq{}Huqwa\PYGZsq{}: 5, \PYGZsq{}nsqfdh\PYGZsq{}: 6, \PYGZsq{}bqs\PYGZsq{}: 3, \PYGZsq{}hdnxi\PYGZsq{}: 5, \PYGZsq{}dvux\PYGZsq{}: 4, \PYGZsq{}whoever\PYGZsq{}: 7, \PYGZsq{}ngmk\PYGZsq{}: 4, \PYGZsq{}dewsgk\PYGZsq{}: 6, \PYGZsq{}upon\PYGZsq{}: 4, \PYGZsq{}otzv\PYGZsq{}: 4, \PYGZsq{}odq\PYGZsq{}: 3, \PYGZsq{}xzain.\PYGZsq{}: 6, \PYGZsq{}Dnyvaolezc\PYGZsq{}: 10, \PYGZsq{}aubz\PYGZsq{}: 4, \PYGZsq{}sti\PYGZsq{}: 3, \PYGZsq{}seems\PYGZsq{}: 5, \PYGZsq{}qdsaclty\PYGZsq{}: 8, \PYGZsq{}mcav.\PYGZsq{}: 5, \PYGZsq{}Xnazkfc\PYGZsq{}: 7, \PYGZsq{}last\PYGZsq{}: 4, \PYGZsq{}irsw\PYGZsq{}: 4, \PYGZsq{}she\PYGZsq{}: 3, \PYGZsq{}rfl\PYGZsq{}: 3, \PYGZsq{}xqny\PYGZsq{}: 4, \PYGZsq{}call\PYGZsq{}: 4, \PYGZsq{}hafnrk.\PYGZsq{}: 7, \PYGZsq{}Kutl.\PYGZsq{}: 5, \PYGZsq{}Gulnifj\PYGZsq{}: 7, \PYGZsq{}pbihguqvc\PYGZsq{}: 9, \PYGZsq{}lfxuy\PYGZsq{}: 5, \PYGZsq{}rchui\PYGZsq{}: 5, \PYGZsq{}zexi\PYGZsq{}: 4, \PYGZsq{}rbmwx\PYGZsq{}: 5, \PYGZsq{}anyone\PYGZsq{}: 6, \PYGZsq{}udyc\PYGZsq{}: 4, \PYGZsq{}904\PYGZsq{}: 3, \PYGZsq{}ofa\PYGZsq{}: 3, \PYGZsq{}nfk\PYGZsq{}: 3, \PYGZsq{}znh\PYGZsq{}: 3, \PYGZsq{}hrw\PYGZsq{}: 3, \PYGZsq{}960754138\PYGZsq{}: 9, \PYGZsq{}anyway\PYGZsq{}: 6, \PYGZsq{}dajegxrqn\PYGZsq{}: 9, \PYGZsq{}58\PYGZsq{}: 2, \PYGZsq{}zwhto.\PYGZsq{}: 6, \PYGZsq{}Gfh\PYGZsq{}: 3, \PYGZsq{}rzni\PYGZsq{}: 4, \PYGZsq{}xcwq\PYGZsq{}: 4, \PYGZsq{}do\PYGZsq{}: 2, \PYGZsq{}rkhvbj\PYGZsq{}: 6, \PYGZsq{}eaz.\PYGZsq{}: 4, \PYGZsq{}Sunm\PYGZsq{}: 4, \PYGZsq{}kbcydwv\PYGZsq{}: 7, \PYGZsq{}oaxhcnrtpy\PYGZsq{}: 10, \PYGZsq{}ngoec.\PYGZsq{}: 6, \PYGZsq{}Vzyo\PYGZsq{}: 4, \PYGZsq{}pzm\PYGZsq{}: 3, \PYGZsq{}cws.\PYGZsq{}: 4, \PYGZsq{}Szuwt\PYGZsq{}: 5, \PYGZsq{}saxhpq\PYGZsq{}: 6, \PYGZsq{}jfqil\PYGZsq{}: 5, \PYGZsq{}buqxalwz\PYGZsq{}: 8, \PYGZsq{}vyzna\PYGZsq{}: 5, \PYGZsq{}oetnq\PYGZsq{}: 5, \PYGZsq{}fifteen\PYGZsq{}: 7, \PYGZsq{}htmafgz\PYGZsq{}: 7, \PYGZsq{}wvdx\PYGZsq{}: 4, \PYGZsq{}ywv\PYGZsq{}: 3, \PYGZsq{}within\PYGZsq{}: 6, \PYGZsq{}lmq\PYGZsq{}: 3, \PYGZsq{}wnlsh.\PYGZsq{}: 6, \PYGZsq{}Yeu\PYGZsq{}: 3, \PYGZsq{}bayqt\PYGZsq{}: 5, \PYGZsq{}gnodv\PYGZsq{}: 5, \PYGZsq{}every\PYGZsq{}: 5, \PYGZsq{}zpw\PYGZsq{}: 3, \PYGZsq{}cens\PYGZsq{}: 4, \PYGZsq{}alwyom\PYGZsq{}: 6, \PYGZsq{}npkgwfruo\PYGZsq{}: 9, \PYGZsq{}xuye\PYGZsq{}: 4, \PYGZsq{}rfbti\PYGZsq{}: 5, \PYGZsq{}zve\PYGZsq{}: 3, \PYGZsq{}nht.\PYGZsq{}: 4, \PYGZsq{}Wis\PYGZsq{}: 3, \PYGZsq{}0925361784\PYGZsq{}: 10, \PYGZsq{}udzj\PYGZsq{}: 4, \PYGZsq{}were\PYGZsq{}: 4, \PYGZsq{}mgq\PYGZsq{}: 3, \PYGZsq{}rgjyxd\PYGZsq{}: 6, \PYGZsq{}eojf\PYGZsq{}: 4, \PYGZsq{}hskeod\PYGZsq{}: 6, \PYGZsq{}yeb\PYGZsq{}: 3, \PYGZsq{}pjywlcto\PYGZsq{}: 8, \PYGZsq{}mec\PYGZsq{}: 3, \PYGZsq{}zlmav\PYGZsq{}: 5, \PYGZsq{}sxl\PYGZsq{}: 3, \PYGZsq{}cvwd.\PYGZsq{}: 5, \PYGZsq{}Duc\PYGZsq{}: 3, \PYGZsq{}bdv\PYGZsq{}: 3, \PYGZsq{}ulf\PYGZsq{}: 3, \PYGZsq{}jkuzcpwl\PYGZsq{}: 8, \PYGZsq{}lqn\PYGZsq{}: 3, \PYGZsq{}wzrgj\PYGZsq{}: 5, \PYGZsq{}they\PYGZsq{}: 4, \PYGZsq{}wtr\PYGZsq{}: 3, \PYGZsq{}lkh\PYGZsq{}: 3, \PYGZsq{}vdewj\PYGZsq{}: 5, \PYGZsq{}agx\PYGZsq{}: 3, \PYGZsq{}wctlyu\PYGZsq{}: 6, \PYGZsq{}his\PYGZsq{}: 3, \PYGZsq{}dxylpan\PYGZsq{}: 7, \PYGZsq{}dulhbmfkwt.\PYGZsq{}: 11, \PYGZsq{}Msceu\PYGZsq{}: 5, \PYGZsq{}68\PYGZsq{}: 2, \PYGZsq{}xnlzfbts\PYGZsq{}: 8, \PYGZsq{}hki\PYGZsq{}: 3, \PYGZsq{}igomcajbt\PYGZsq{}: 9, \PYGZsq{}qjnrtpiwmh\PYGZsq{}: 10, \PYGZsq{}kzm\PYGZsq{}: 3, \PYGZsq{}erf\PYGZsq{}: 3, \PYGZsq{}bly\PYGZsq{}: 3, \PYGZsq{}wgshv\PYGZsq{}: 5, \PYGZsq{}describe\PYGZsq{}: 8, \PYGZsq{}fjl\PYGZsq{}: 3, \PYGZsq{}qfwmlogdiu\PYGZsq{}: 10, \PYGZsq{}tqhi\PYGZsq{}: 4, \PYGZsq{}cjdiu\PYGZsq{}: 5, \PYGZsq{}go\PYGZsq{}: 2, \PYGZsq{}jetwbnos\PYGZsq{}: 8, \PYGZsq{}cmzywa\PYGZsq{}: 6, \PYGZsq{}wlm\PYGZsq{}: 3, \PYGZsq{}wqulmj\PYGZsq{}: 6, \PYGZsq{}dxowc\PYGZsq{}: 5, \PYGZsq{}yokjd\PYGZsq{}: 5, \PYGZsq{}yxfi.\PYGZsq{}: 5, \PYGZsq{}Hrfdtpimlj\PYGZsq{}: 10, \PYGZsq{}rzj\PYGZsq{}: 3, \PYGZsq{}vfixw\PYGZsq{}: 5, \PYGZsq{}fwqayc\PYGZsq{}: 6, \PYGZsq{}ngtb\PYGZsq{}: 4, \PYGZsq{}ymwbq\PYGZsq{}: 5, \PYGZsq{}wikzcpsud\PYGZsq{}: 9, \PYGZsq{}zhce\PYGZsq{}: 4, \PYGZsq{}fml.\PYGZsq{}: 4, \PYGZsq{}Xtu\PYGZsq{}: 3, \PYGZsq{}us\PYGZsq{}: 2, \PYGZsq{}six\PYGZsq{}: 3, \PYGZsq{}eg\PYGZsq{}: 2, \PYGZsq{}am\PYGZsq{}: 2, \PYGZsq{}rcj\PYGZsq{}: 3, \PYGZsq{}nekc\PYGZsq{}: 4, \PYGZsq{}gyjof\PYGZsq{}: 5, \PYGZsq{}akef\PYGZsq{}: 4, \PYGZsq{}juq\PYGZsq{}: 3, \PYGZsq{}uksal\PYGZsq{}: 5, \PYGZsq{}38290416\PYGZsq{}: 8, \PYGZsq{}beyuo\PYGZsq{}: 5, \PYGZsq{}iawx.\PYGZsq{}: 5, \PYGZsq{}Zcxywjoqr\PYGZsq{}: 9, \PYGZsq{}cpdzxtyquw\PYGZsq{}: 10, \PYGZsq{}either\PYGZsq{}: 6, \PYGZsq{}yxmp\PYGZsq{}: 4, \PYGZsq{}rywae\PYGZsq{}: 5, \PYGZsq{}mje\PYGZsq{}: 3, \PYGZsq{}pxrv.\PYGZsq{}: 5, \PYGZsq{}Anyhow\PYGZsq{}: 6, \PYGZsq{}bwmh\PYGZsq{}: 4, \PYGZsq{}zxqrn\PYGZsq{}: 5, \PYGZsq{}frap\PYGZsq{}: 4, \PYGZsq{}ula\PYGZsq{}: 3, \PYGZsq{}mnps\PYGZsq{}: 4, \PYGZsq{}fpsnwe.\PYGZsq{}: 7, \PYGZsq{}Arm\PYGZsq{}: 3, \PYGZsq{}you\PYGZsq{}: 3, \PYGZsq{}why\PYGZsq{}: 3, \PYGZsq{}ytv.\PYGZsq{}: 4, \PYGZsq{}Rway\PYGZsq{}: 4, \PYGZsq{}bja\PYGZsq{}: 3, \PYGZsq{}per\PYGZsq{}: 3, \PYGZsq{}gmefzwiph\PYGZsq{}: 9, \PYGZsq{}sfk\PYGZsq{}: 3, \PYGZsq{}2\PYGZsq{}: 1, \PYGZsq{}cmjgd\PYGZsq{}: 5, \PYGZsq{}jpryo\PYGZsq{}: 5, \PYGZsq{}bgs\PYGZsq{}: 3, \PYGZsq{}edwxm.\PYGZsq{}: 6, \PYGZsq{}Jkypmozti\PYGZsq{}: 9, \PYGZsq{}09\PYGZsq{}: 2, \PYGZsq{}against\PYGZsq{}: 7, \PYGZsq{}yaj\PYGZsq{}: 3, \PYGZsq{}jpgkqz\PYGZsq{}: 6, \PYGZsq{}eaznv\PYGZsq{}: 5, \PYGZsq{}mcnpo\PYGZsq{}: 5, \PYGZsq{}than\PYGZsq{}: 4, \PYGZsq{}pjfdznsye\PYGZsq{}: 9, \PYGZsq{}angjhlt.\PYGZsq{}: 8, \PYGZsq{}Aezjdcb\PYGZsq{}: 7, \PYGZsq{}lna\PYGZsq{}: 3, \PYGZsq{}uidp\PYGZsq{}: 4, \PYGZsq{}sih\PYGZsq{}: 3, \PYGZsq{}though\PYGZsq{}: 6, \PYGZsq{}96\PYGZsq{}: 2, \PYGZsq{}mezdvota\PYGZsq{}: 8, \PYGZsq{}zlb\PYGZsq{}: 3, \PYGZsq{}there\PYGZsq{}: 5, \PYGZsq{}fgvnu\PYGZsq{}: 5, \PYGZsq{}bpj\PYGZsq{}: 3, \PYGZsq{}edtlurbqoz\PYGZsq{}: 10, \PYGZsq{}vqlo\PYGZsq{}: 4, \PYGZsq{}pziny\PYGZsq{}: 5, \PYGZsq{}oej\PYGZsq{}: 3, \PYGZsq{}crdswyz\PYGZsq{}: 7, \PYGZsq{}ekcg\PYGZsq{}: 4, \PYGZsq{}kjyhclbmgx\PYGZsq{}: 10, \PYGZsq{}aky\PYGZsq{}: 3, \PYGZsq{}wvcmgkozph\PYGZsq{}: 10, \PYGZsq{}who\PYGZsq{}: 3, \PYGZsq{}qef\PYGZsq{}: 3, \PYGZsq{}vaf\PYGZsq{}: 3, \PYGZsq{}nsaifdtj\PYGZsq{}: 8, \PYGZsq{}yednrg\PYGZsq{}: 6, \PYGZsq{}rfoscytlv\PYGZsq{}: 9, \PYGZsq{}nmw.\PYGZsq{}: 4, \PYGZsq{}Zbh\PYGZsq{}: 3, \PYGZsq{}eqbnc\PYGZsq{}: 5, \PYGZsq{}wsjln\PYGZsq{}: 5, \PYGZsq{}xtgbohj\PYGZsq{}: 7, \PYGZsq{}wslqa\PYGZsq{}: 5, \PYGZsq{}aqljiz\PYGZsq{}: 6, \PYGZsq{}he\PYGZsq{}: 2, \PYGZsq{}bqsx\PYGZsq{}: 4, \PYGZsq{}aprsizdj\PYGZsq{}: 8, \PYGZsq{}32\PYGZsq{}: 2, \PYGZsq{}ksg\PYGZsq{}: 3, \PYGZsq{}yjivunlr\PYGZsq{}: 8, \PYGZsq{}pvq\PYGZsq{}: 3, \PYGZsq{}6219745\PYGZsq{}: 7, \PYGZsq{}oyux\PYGZsq{}: 4, \PYGZsq{}yzciok.\PYGZsq{}: 7, \PYGZsq{}Third\PYGZsq{}: 5, \PYGZsq{}avb\PYGZsq{}: 3, \PYGZsq{}ourselves\PYGZsq{}: 9, \PYGZsq{}again\PYGZsq{}: 5, \PYGZsq{}amongst\PYGZsq{}: 7, \PYGZsq{}izmwo\PYGZsq{}: 5, \PYGZsq{}jhy\PYGZsq{}: 3, \PYGZsq{}mulpsitaco\PYGZsq{}: 10, \PYGZsq{}ejxb\PYGZsq{}: 4, \PYGZsq{}nmvrxchzbu\PYGZsq{}: 10, \PYGZsq{}ehpd\PYGZsq{}: 4, \PYGZsq{}zng\PYGZsq{}: 3, \PYGZsq{}jteh\PYGZsq{}: 4, \PYGZsq{}nplou.\PYGZsq{}: 6, \PYGZsq{}Clao\PYGZsq{}: 4, \PYGZsq{}028\PYGZsq{}: 3, \PYGZsq{}become\PYGZsq{}: 6, \PYGZsq{}herein\PYGZsq{}: 6, \PYGZsq{}zelu\PYGZsq{}: 4, \PYGZsq{}lrebkiqf\PYGZsq{}: 8, \PYGZsq{}xpvbr\PYGZsq{}: 5, \PYGZsq{}6235487\PYGZsq{}: 7, \PYGZsq{}because\PYGZsq{}: 7, \PYGZsq{}everything\PYGZsq{}: 10, \PYGZsq{}beyond\PYGZsq{}: 6, \PYGZsq{}pdv.\PYGZsq{}: 4, \PYGZsq{}might\PYGZsq{}: 5, \PYGZsq{}481\PYGZsq{}: 3, \PYGZsq{}rqmb\PYGZsq{}: 4, \PYGZsq{}fsj\PYGZsq{}: 3, \PYGZsq{}vzgrhim\PYGZsq{}: 7, \PYGZsq{}ie\PYGZsq{}: 2, \PYGZsq{}zck\PYGZsq{}: 3, \PYGZsq{}kyqdxcni\PYGZsq{}: 8, \PYGZsq{}547\PYGZsq{}: 3, \PYGZsq{}sztv\PYGZsq{}: 4, \PYGZsq{}jwqbod\PYGZsq{}: 6, \PYGZsq{}aryu\PYGZsq{}: 4, \PYGZsq{}mph\PYGZsq{}: 3, \PYGZsq{}18\PYGZsq{}: 2, \PYGZsq{}eayg\PYGZsq{}: 4, \PYGZsq{}zuv\PYGZsq{}: 3, \PYGZsq{}bill\PYGZsq{}: 4, \PYGZsq{}vhbmge\PYGZsq{}: 6, \PYGZsq{}pfozcj\PYGZsq{}: 6, \PYGZsq{}oltg\PYGZsq{}: 4, \PYGZsq{}evazwjmxq\PYGZsq{}: 9, \PYGZsq{}sba\PYGZsq{}: 3, \PYGZsq{}3\PYGZsq{}: 1, \PYGZsq{}iaqtu\PYGZsq{}: 5, \PYGZsq{}fahq\PYGZsq{}: 4, \PYGZsq{}give\PYGZsq{}: 4, \PYGZsq{}inbp\PYGZsq{}: 4, \PYGZsq{}lzu\PYGZsq{}: 3, \PYGZsq{}tpgiya\PYGZsq{}: 6, \PYGZsq{}xcf\PYGZsq{}: 3, \PYGZsq{}jpyfh\PYGZsq{}: 5, \PYGZsq{}068357\PYGZsq{}: 6, \PYGZsq{}always\PYGZsq{}: 6, \PYGZsq{}mpauskvx\PYGZsq{}: 8, \PYGZsq{}zkvxpf\PYGZsq{}: 6, \PYGZsq{}lqjr\PYGZsq{}: 4, \PYGZsq{}uzobqdewia\PYGZsq{}: 10, \PYGZsq{}ogm\PYGZsq{}: 3, \PYGZsq{}yjd\PYGZsq{}: 3, \PYGZsq{}kvs\PYGZsq{}: 3, \PYGZsq{}ugdsbxovpl\PYGZsq{}: 10, \PYGZsq{}ztkxn\PYGZsq{}: 5, \PYGZsq{}182\PYGZsq{}: 3, \PYGZsq{}pdvha\PYGZsq{}: 5, \PYGZsq{}fhlc\PYGZsq{}: 4, \PYGZsq{}lmkhzvs\PYGZsq{}: 7, \PYGZsq{}izj\PYGZsq{}: 3, \PYGZsq{}hereafter\PYGZsq{}: 9, \PYGZsq{}cgdmw\PYGZsq{}: 5, \PYGZsq{}462\PYGZsq{}: 3, \PYGZsq{}tyr\PYGZsq{}: 3, \PYGZsq{}had\PYGZsq{}: 3, \PYGZsq{}vlzyx\PYGZsq{}: 5, \PYGZsq{}bmeu\PYGZsq{}: 4, \PYGZsq{}dtm\PYGZsq{}: 3, \PYGZsq{}xhg\PYGZsq{}: 3, \PYGZsq{}6071843\PYGZsq{}: 7, \PYGZsq{}sztubf\PYGZsq{}: 6, \PYGZsq{}gjx\PYGZsq{}: 3, \PYGZsq{}506\PYGZsq{}: 3, \PYGZsq{}further\PYGZsq{}: 7, \PYGZsq{}kywavb\PYGZsq{}: 6, \PYGZsq{}gubdl\PYGZsq{}: 5, \PYGZsq{}mihukod\PYGZsq{}: 7, \PYGZsq{}rmixj\PYGZsq{}: 5, \PYGZsq{}gxhta\PYGZsq{}: 5, \PYGZsq{}jzgnvbpm\PYGZsq{}: 8, \PYGZsq{}qjwlc.\PYGZsq{}: 6, \PYGZsq{}Raxi\PYGZsq{}: 4, \PYGZsq{}empty\PYGZsq{}: 5, \PYGZsq{}ars\PYGZsq{}: 3, \PYGZsq{}vgf\PYGZsq{}: 3, \PYGZsq{}somehow\PYGZsq{}: 7, \PYGZsq{}urhqck.\PYGZsq{}: 7, \PYGZsq{}Tghr\PYGZsq{}: 4, \PYGZsq{}13\PYGZsq{}: 2, \PYGZsq{}436120\PYGZsq{}: 6, \PYGZsq{}hkagf\PYGZsq{}: 5, \PYGZsq{}wcu\PYGZsq{}: 3, \PYGZsq{}zea\PYGZsq{}: 3, \PYGZsq{}hstw\PYGZsq{}: 4, \PYGZsq{}qrvf\PYGZsq{}: 4, \PYGZsq{}pml.\PYGZsq{}: 4, \PYGZsq{}Vsj\PYGZsq{}: 3, \PYGZsq{}xckhtlf\PYGZsq{}: 7, \PYGZsq{}nizps\PYGZsq{}: 5, \PYGZsq{}0\PYGZsq{}: 1, \PYGZsq{}re\PYGZsq{}: 2, \PYGZsq{}qgs\PYGZsq{}: 3, \PYGZsq{}lieadc\PYGZsq{}: 6, \PYGZsq{}manc\PYGZsq{}: 4, \PYGZsq{}fgr\PYGZsq{}: 3, \PYGZsq{}aotpuh.\PYGZsq{}: 7, \PYGZsq{}Gyeq\PYGZsq{}: 4, \PYGZsq{}gcqf\PYGZsq{}: 4, \PYGZsq{}fthnax.\PYGZsq{}: 7, \PYGZsq{}Azbryluid\PYGZsq{}: 9, \PYGZsq{}mag\PYGZsq{}: 3, \PYGZsq{}7\PYGZsq{}: 1, \PYGZsq{}whether\PYGZsq{}: 7, \PYGZsq{}qmhaznr\PYGZsq{}: 7, \PYGZsq{}uqizltkm\PYGZsq{}: 8, \PYGZsq{}lqv\PYGZsq{}: 3, \PYGZsq{}rtukhyl\PYGZsq{}: 7, \PYGZsq{}loera\PYGZsq{}: 5, \PYGZsq{}zxu\PYGZsq{}: 3, \PYGZsq{}lirxzk\PYGZsq{}: 6, \PYGZsq{}pxn\PYGZsq{}: 3, \PYGZsq{}otherwise\PYGZsq{}: 9, \PYGZsq{}jwd\PYGZsq{}: 3, \PYGZsq{}mxwo\PYGZsq{}: 4, \PYGZsq{}nor\PYGZsq{}: 3, \PYGZsq{}rqwgdyjx\PYGZsq{}: 8, \PYGZsq{}gsqh\PYGZsq{}: 4, \PYGZsq{}gzo\PYGZsq{}: 3, \PYGZsq{}xuisq\PYGZsq{}: 5, \PYGZsq{}gdhc\PYGZsq{}: 4, \PYGZsq{}kbiojvt\PYGZsq{}: 7, \PYGZsq{}lngrbm\PYGZsq{}: 6, \PYGZsq{}are\PYGZsq{}: 3, \PYGZsq{}rvcwpuz\PYGZsq{}: 7, \PYGZsq{}luj\PYGZsq{}: 3, \PYGZsq{}that\PYGZsq{}: 4, \PYGZsq{}qni\PYGZsq{}: 3, \PYGZsq{}dsy\PYGZsq{}: 3, \PYGZsq{}valyj\PYGZsq{}: 5, \PYGZsq{}nefaw.\PYGZsq{}: 6, \PYGZsq{}Zdhi\PYGZsq{}: 4, \PYGZsq{}bwfq\PYGZsq{}: 4, \PYGZsq{}pqafcbx\PYGZsq{}: 7, \PYGZsq{}qhvj\PYGZsq{}: 4, \PYGZsq{}pma\PYGZsq{}: 3, \PYGZsq{}wqc\PYGZsq{}: 3, \PYGZsq{}avgf\PYGZsq{}: 4, \PYGZsq{}iymrsh.\PYGZsq{}: 7, \PYGZsq{}Atbr\PYGZsq{}: 4, \PYGZsq{}thin\PYGZsq{}: 4, \PYGZsq{}yvobgjk\PYGZsq{}: 7, \PYGZsq{}osb\PYGZsq{}: 3, \PYGZsq{}npw\PYGZsq{}: 3, \PYGZsq{}for\PYGZsq{}: 3, \PYGZsq{}fpweuk\PYGZsq{}: 6, \PYGZsq{}woq\PYGZsq{}: 3, \PYGZsq{}ampgvqd\PYGZsq{}: 7, \PYGZsq{}over\PYGZsq{}: 4, \PYGZsq{}gtoif\PYGZsq{}: 5, \PYGZsq{}urlmtdkvg\PYGZsq{}: 9, \PYGZsq{}cxr\PYGZsq{}: 3, \PYGZsq{}mfoslrpc\PYGZsq{}: 8, \PYGZsq{}from\PYGZsq{}: 4, \PYGZsq{}biuayo\PYGZsq{}: 6, \PYGZsq{}rvbu\PYGZsq{}: 4, \PYGZsq{}uvalckg.\PYGZsq{}: 8, \PYGZsq{}Rsf\PYGZsq{}: 3, \PYGZsq{}uvnwea\PYGZsq{}: 6, \PYGZsq{}cud\PYGZsq{}: 3, \PYGZsq{}tauic\PYGZsq{}: 5, \PYGZsq{}ixm\PYGZsq{}: 3, \PYGZsq{}gvs\PYGZsq{}: 3, \PYGZsq{}jhz\PYGZsq{}: 3, \PYGZsq{}jsy\PYGZsq{}: 3, \PYGZsq{}nqrfd\PYGZsq{}: 5, \PYGZsq{}pvifly\PYGZsq{}: 6, \PYGZsq{}ejrx\PYGZsq{}: 4, \PYGZsq{}qkhi.\PYGZsq{}: 5, \PYGZsq{}Lhg\PYGZsq{}: 3, \PYGZsq{}zgpkir\PYGZsq{}: 6, \PYGZsq{}yuql\PYGZsq{}: 4, \PYGZsq{}rtpmu\PYGZsq{}: 5, \PYGZsq{}iwdl.\PYGZsq{}: 5, \PYGZsq{}Interest\PYGZsq{}: 8, \PYGZsq{}hyql\PYGZsq{}: 4, \PYGZsq{}812\PYGZsq{}: 3, \PYGZsq{}olhdfrcw\PYGZsq{}: 8, \PYGZsq{}jkfqcwrx\PYGZsq{}: 8, \PYGZsq{}csatldymq\PYGZsq{}: 9, \PYGZsq{}orl\PYGZsq{}: 3, \PYGZsq{}dynec\PYGZsq{}: 5, \PYGZsq{}jhmveyoa\PYGZsq{}: 8, \PYGZsq{}lzrtgds\PYGZsq{}: 7, \PYGZsq{}fnh\PYGZsq{}: 3, \PYGZsq{}jue\PYGZsq{}: 3, \PYGZsq{}kostmzgb.\PYGZsq{}: 9, \PYGZsq{}Niurdlk\PYGZsq{}: 7, \PYGZsq{}ncw\PYGZsq{}: 3, \PYGZsq{}vmrowhysl\PYGZsq{}: 9, \PYGZsq{}enrj\PYGZsq{}: 4, \PYGZsq{}371\PYGZsq{}: 3, \PYGZsq{}jlvepi\PYGZsq{}: 6, \PYGZsq{}szhraxofm.\PYGZsq{}: 10, \PYGZsq{}Vkgzlwjmqt\PYGZsq{}: 10, \PYGZsq{}lqf\PYGZsq{}: 3, \PYGZsq{}asou\PYGZsq{}: 4, \PYGZsq{}zlvpogq\PYGZsq{}: 7, \PYGZsq{}8320416759\PYGZsq{}: 10, \PYGZsq{}nky\PYGZsq{}: 3, \PYGZsq{}mahqfwnpsr\PYGZsq{}: 10, \PYGZsq{}fjqin\PYGZsq{}: 5, \PYGZsq{}ircf\PYGZsq{}: 4, \PYGZsq{}lbta\PYGZsq{}: 4, \PYGZsq{}ptfnzcbra\PYGZsq{}: 9, \PYGZsq{}vwbol\PYGZsq{}: 5, \PYGZsq{}lxdui\PYGZsq{}: 5, \PYGZsq{}nevertheless\PYGZsq{}: 12, \PYGZsq{}tegf\PYGZsq{}: 4, \PYGZsq{}kosqnhcwgr\PYGZsq{}: 10, \PYGZsq{}ycxu\PYGZsq{}: 4, \PYGZsq{}after\PYGZsq{}: 5, \PYGZsq{}without\PYGZsq{}: 7, \PYGZsq{}bwjre\PYGZsq{}: 5, \PYGZsq{}fovkgisjre\PYGZsq{}: 10, \PYGZsq{}xdbye\PYGZsq{}: 5, \PYGZsq{}cnvr\PYGZsq{}: 4, \PYGZsq{}eynwxlr\PYGZsq{}: 7, \PYGZsq{}zoyal\PYGZsq{}: 5, \PYGZsq{}find\PYGZsq{}: 4, \PYGZsq{}fwpzkb\PYGZsq{}: 6, \PYGZsq{}idlqaukyvn\PYGZsq{}: 10, \PYGZsq{}htu\PYGZsq{}: 3, \PYGZsq{}zfw\PYGZsq{}: 3, \PYGZsq{}mejcgvk\PYGZsq{}: 7, \PYGZsq{}brpkhwof\PYGZsq{}: 8, \PYGZsq{}dgkwn\PYGZsq{}: 5, \PYGZsq{}gdztwoelji\PYGZsq{}: 10, \PYGZsq{}yjrc\PYGZsq{}: 4, \PYGZsq{}part\PYGZsq{}: 4, \PYGZsq{}fau\PYGZsq{}: 3, \PYGZsq{}dlfju\PYGZsq{}: 5, \PYGZsq{}fdt\PYGZsq{}: 3, \PYGZsq{}rpfomb\PYGZsq{}: 6, \PYGZsq{}out\PYGZsq{}: 3, \PYGZsq{}kszc\PYGZsq{}: 4, \PYGZsq{}this\PYGZsq{}: 4, \PYGZsq{}njbhxi\PYGZsq{}: 6, \PYGZsq{}ybh\PYGZsq{}: 3, \PYGZsq{}oqzps\PYGZsq{}: 5, \PYGZsq{}bgro\PYGZsq{}: 4, \PYGZsq{}rpyfh\PYGZsq{}: 5, \PYGZsq{}rmlp.\PYGZsq{}: 5, \PYGZsq{}Until\PYGZsq{}: 5, \PYGZsq{}only\PYGZsq{}: 4, \PYGZsq{}qpuoyc.\PYGZsq{}: 7, \PYGZsq{}Vwplt\PYGZsq{}: 5, \PYGZsq{}eovw\PYGZsq{}: 4, \PYGZsq{}395046278\PYGZsq{}: 9, \PYGZsq{}fhtmelw\PYGZsq{}: 7, \PYGZsq{}bvezk\PYGZsq{}: 5, \PYGZsq{}jhzg\PYGZsq{}: 4, \PYGZsq{}wup\PYGZsq{}: 3, \PYGZsq{}yswkqgxzr\PYGZsq{}: 9, \PYGZsq{}full\PYGZsq{}: 4, \PYGZsq{}chmreyqgiz\PYGZsq{}: 10, \PYGZsq{}6\PYGZsq{}: 1, \PYGZsq{}rwu\PYGZsq{}: 3, \PYGZsq{}latterly\PYGZsq{}: 8, \PYGZsq{}tmqsh\PYGZsq{}: 5, \PYGZsq{}ejaqhu\PYGZsq{}: 6, \PYGZsq{}iolrpbsten\PYGZsq{}: 10, \PYGZsq{}opgqdunrjk\PYGZsq{}: 10, \PYGZsq{}tlap\PYGZsq{}: 4, \PYGZsq{}odhtg\PYGZsq{}: 5, \PYGZsq{}must\PYGZsq{}: 4, \PYGZsq{}lmnj\PYGZsq{}: 4, \PYGZsq{}eqv\PYGZsq{}: 3, \PYGZsq{}thereupon\PYGZsq{}: 9, \PYGZsq{}qep\PYGZsq{}: 3, \PYGZsq{}mza\PYGZsq{}: 3, \PYGZsq{}fdq\PYGZsq{}: 3, \PYGZsq{}xtv\PYGZsq{}: 3, \PYGZsq{}lwgmo\PYGZsq{}: 5, \PYGZsq{}tjv\PYGZsq{}: 3, \PYGZsq{}zbw\PYGZsq{}: 3, \PYGZsq{}all\PYGZsq{}: 3, \PYGZsq{}sdh\PYGZsq{}: 3, \PYGZsq{}co\PYGZsq{}: 2, \PYGZsq{}never\PYGZsq{}: 5, \PYGZsq{}msaof\PYGZsq{}: 5, \PYGZsq{}upn\PYGZsq{}: 3, \PYGZsq{}ecpg\PYGZsq{}: 4, \PYGZsq{}wapgbm\PYGZsq{}: 6, \PYGZsq{}kztmowlyu\PYGZsq{}: 9, \PYGZsq{}ofm\PYGZsq{}: 3, \PYGZsq{}048\PYGZsq{}: 3, \PYGZsq{}hgy\PYGZsq{}: 3, \PYGZsq{}system\PYGZsq{}: 6, \PYGZsq{}wzriy\PYGZsq{}: 5, \PYGZsq{}ymn\PYGZsq{}: 3, \PYGZsq{}sometime\PYGZsq{}: 8, \PYGZsq{}246\PYGZsq{}: 3, \PYGZsq{}off\PYGZsq{}: 3, \PYGZsq{}vgw\PYGZsq{}: 3, \PYGZsq{}seeming\PYGZsq{}: 7, \PYGZsq{}fbao\PYGZsq{}: 4, \PYGZsq{}fsyu\PYGZsq{}: 4, \PYGZsq{}akcqxwshtj.\PYGZsq{}: 11, \PYGZsq{}Ouyweabv\PYGZsq{}: 8, \PYGZsq{}ewlj\PYGZsq{}: 4, \PYGZsq{}896417532\PYGZsq{}: 9, \PYGZsq{}gbpvn\PYGZsq{}: 5, \PYGZsq{}bjrgao\PYGZsq{}: 6, \PYGZsq{}rqhg.\PYGZsq{}: 5, \PYGZsq{}Joc\PYGZsq{}: 3, \PYGZsq{}mzes\PYGZsq{}: 4, \PYGZsq{}piqbjlhoz\PYGZsq{}: 9, \PYGZsq{}but\PYGZsq{}: 3, \PYGZsq{}gqwoaf\PYGZsq{}: 6, \PYGZsq{}swa\PYGZsq{}: 3, \PYGZsq{}kfnb\PYGZsq{}: 4, \PYGZsq{}cnyo\PYGZsq{}: 4, \PYGZsq{}cry\PYGZsq{}: 3, \PYGZsq{}wherever\PYGZsq{}: 8, \PYGZsq{}beyzthj\PYGZsq{}: 7, \PYGZsq{}crzdltsjpo\PYGZsq{}: 10, \PYGZsq{}jchgmwpdzt\PYGZsq{}: 10, \PYGZsq{}vjp\PYGZsq{}: 3, \PYGZsq{}tuose.\PYGZsq{}: 6, \PYGZsq{}Eximlr\PYGZsq{}: 6, \PYGZsq{}on\PYGZsq{}: 2, \PYGZsq{}asb\PYGZsq{}: 3, \PYGZsq{}frp.\PYGZsq{}: 4, \PYGZsq{}Odbzr\PYGZsq{}: 5, \PYGZsq{}xlio\PYGZsq{}: 4, \PYGZsq{}oqketij\PYGZsq{}: 7, \PYGZsq{}kxbva.\PYGZsq{}: 6, \PYGZsq{}Vbonxc\PYGZsq{}: 6, \PYGZsq{}xyd\PYGZsq{}: 3, \PYGZsq{}atr\PYGZsq{}: 3, \PYGZsq{}chr\PYGZsq{}: 3, \PYGZsq{}hgkw\PYGZsq{}: 4, \PYGZsq{}kanrpi\PYGZsq{}: 6, \PYGZsq{}qtpjsw\PYGZsq{}: 6, \PYGZsq{}tkcuv\PYGZsq{}: 5, \PYGZsq{}difanz.\PYGZsq{}: 7, \PYGZsq{}Bapniuzje\PYGZsq{}: 9, \PYGZsq{}ukflm\PYGZsq{}: 5, \PYGZsq{}jtug\PYGZsq{}: 4, \PYGZsq{}lwgn\PYGZsq{}: 4, \PYGZsq{}between\PYGZsq{}: 7, \PYGZsq{}uwgexb\PYGZsq{}: 6, \PYGZsq{}ltkhz\PYGZsq{}: 5, \PYGZsq{}amkxi\PYGZsq{}: 5, \PYGZsq{}evly.\PYGZsq{}: 5, \PYGZsq{}Zfbj\PYGZsq{}: 4, \PYGZsq{}yaxqrt\PYGZsq{}: 6, \PYGZsq{}damxpz\PYGZsq{}: 6, \PYGZsq{}vybnsxjrf\PYGZsq{}: 9, \PYGZsq{}etc\PYGZsq{}: 3, \PYGZsq{}below\PYGZsq{}: 5, \PYGZsq{}moreover\PYGZsq{}: 8, \PYGZsq{}fpnour.\PYGZsq{}: 7, \PYGZsq{}Sownjvlyp\PYGZsq{}: 9, \PYGZsq{}wherein\PYGZsq{}: 7, \PYGZsq{}ystf\PYGZsq{}: 4, \PYGZsq{}150\PYGZsq{}: 3, \PYGZsq{}up\PYGZsq{}: 2, \PYGZsq{}eldabqkmy\PYGZsq{}: 9, \PYGZsq{}jsc\PYGZsq{}: 3, \PYGZsq{}05\PYGZsq{}: 2, \PYGZsq{}jaqyzfp\PYGZsq{}: 7, \PYGZsq{}mxfoyibk\PYGZsq{}: 8, \PYGZsq{}too\PYGZsq{}: 3, \PYGZsq{}clh\PYGZsq{}: 3, \PYGZsq{}edj\PYGZsq{}: 3, \PYGZsq{}wqfcl.\PYGZsq{}: 6, \PYGZsq{}Eknov\PYGZsq{}: 5, \PYGZsq{}kqlnzxve\PYGZsq{}: 8, \PYGZsq{}ljsvb\PYGZsq{}: 5, \PYGZsq{}odk\PYGZsq{}: 3, \PYGZsq{}uwzm\PYGZsq{}: 4, \PYGZsq{}dzscy\PYGZsq{}: 5, \PYGZsq{}gvmd\PYGZsq{}: 4, \PYGZsq{}83\PYGZsq{}: 2, \PYGZsq{}sqixy\PYGZsq{}: 5, \PYGZsq{}nobody\PYGZsq{}: 6, \PYGZsq{}qdl\PYGZsq{}: 3, \PYGZsq{}top\PYGZsq{}: 3, \PYGZsq{}tlhyj\PYGZsq{}: 5, \PYGZsq{}one\PYGZsq{}: 3, \PYGZsq{}kplavxjz.\PYGZsq{}: 9, \PYGZsq{}Hdb\PYGZsq{}: 3, \PYGZsq{}gow\PYGZsq{}: 3, \PYGZsq{}yweuqvndil.\PYGZsq{}: 11, \PYGZsq{}A\PYGZsq{}: 1, \PYGZsq{}lzfr.\PYGZsq{}: 5, \PYGZsq{}Elx\PYGZsq{}: 3, \PYGZsq{}wbtu\PYGZsq{}: 4, \PYGZsq{}ever\PYGZsq{}: 4, \PYGZsq{}izpuv\PYGZsq{}: 5, \PYGZsq{}could\PYGZsq{}: 5, \PYGZsq{}klj\PYGZsq{}: 3, \PYGZsq{}hudjrxmbvz\PYGZsq{}: 10, \PYGZsq{}huiqxtbfdr\PYGZsq{}: 10, \PYGZsq{}3095218\PYGZsq{}: 7, \PYGZsq{}thereafter\PYGZsq{}: 10, \PYGZsq{}xoarmb\PYGZsq{}: 6, \PYGZsq{}sxdmt\PYGZsq{}: 5, \PYGZsq{}qtnlwavk\PYGZsq{}: 8, \PYGZsq{}gjkmc\PYGZsq{}: 5, \PYGZsq{}aiysfcr\PYGZsq{}: 7, \PYGZsq{}the\PYGZsq{}: 3, \PYGZsq{}631\PYGZsq{}: 3, \PYGZsq{}wqmz\PYGZsq{}: 4, \PYGZsq{}mbe.\PYGZsq{}: 4, \PYGZsq{}Pzo\PYGZsq{}: 3, \PYGZsq{}cdjzb\PYGZsq{}: 5, \PYGZsq{}dnr\PYGZsq{}: 3, \PYGZsq{}xkl\PYGZsq{}: 3, \PYGZsq{}omhlrzbs\PYGZsq{}: 8, \PYGZsq{}it\PYGZsq{}: 2, \PYGZsq{}nljp\PYGZsq{}: 4, \PYGZsq{}iamgwtxn\PYGZsq{}: 8, \PYGZsq{}gda\PYGZsq{}: 3, \PYGZsq{}mobydz\PYGZsq{}: 6, \PYGZsq{}uljk\PYGZsq{}: 4, \PYGZsq{}five\PYGZsq{}: 4, \PYGZsq{}tpdcbkfux\PYGZsq{}: 9, \PYGZsq{}cannot\PYGZsq{}: 6, \PYGZsq{}anything\PYGZsq{}: 8, \PYGZsq{}wjzlyo\PYGZsq{}: 6, \PYGZsq{}her\PYGZsq{}: 3, \PYGZsq{}ihka\PYGZsq{}: 4, \PYGZsq{}ujed\PYGZsq{}: 4, \PYGZsq{}noone\PYGZsq{}: 5, \PYGZsq{}pstxj\PYGZsq{}: 5, \PYGZsq{}tvhnsz\PYGZsq{}: 6, \PYGZsq{}kxy\PYGZsq{}: 3, \PYGZsq{}klewbag.\PYGZsq{}: 8, \PYGZsq{}get\PYGZsq{}: 3, \PYGZsq{}hrdl\PYGZsq{}: 4, \PYGZsq{}xlhze\PYGZsq{}: 5, \PYGZsq{}mcv\PYGZsq{}: 3, \PYGZsq{}say\PYGZsq{}: 3, \PYGZsq{}amonu\PYGZsq{}: 5, \PYGZsq{}dzjrolwam\PYGZsq{}: 9, \PYGZsq{}icepxw\PYGZsq{}: 6, \PYGZsq{}qhut\PYGZsq{}: 4, \PYGZsq{}whqfzupys\PYGZsq{}: 9, \PYGZsq{}emga\PYGZsq{}: 4, \PYGZsq{}bzqomu\PYGZsq{}: 6, \PYGZsq{}kpt\PYGZsq{}: 3, \PYGZsq{}hrg\PYGZsq{}: 3, \PYGZsq{}hebauxgy\PYGZsq{}: 8, \PYGZsq{}roy\PYGZsq{}: 3, \PYGZsq{}jieom\PYGZsq{}: 5, \PYGZsq{}hereby\PYGZsq{}: 6, \PYGZsq{}lypvaoj.\PYGZsq{}: 8, \PYGZsq{}Already\PYGZsq{}: 7, \PYGZsq{}wovq\PYGZsq{}: 4, \PYGZsq{}eight\PYGZsq{}: 5, \PYGZsq{}ctlz\PYGZsq{}: 4, \PYGZsq{}qaf.\PYGZsq{}: 4, \PYGZsq{}These\PYGZsq{}: 5, \PYGZsq{}tuw\PYGZsq{}: 3, \PYGZsq{}nzcub\PYGZsq{}: 5, \PYGZsq{}tfimqulyb\PYGZsq{}: 9, \PYGZsq{}bont\PYGZsq{}: 4, \PYGZsq{}gro\PYGZsq{}: 3, \PYGZsq{}asv\PYGZsq{}: 3, \PYGZsq{}fiokn\PYGZsq{}: 5, \PYGZsq{}kcywp\PYGZsq{}: 5, \PYGZsq{}tshg\PYGZsq{}: 4, \PYGZsq{}loty\PYGZsq{}: 4, \PYGZsq{}fzuw\PYGZsq{}: 4, \PYGZsq{}kzndr\PYGZsq{}: 5, \PYGZsq{}wfqhrl\PYGZsq{}: 6, \PYGZsq{}snrwj\PYGZsq{}: 5, \PYGZsq{}pub\PYGZsq{}: 3, \PYGZsq{}wnvpfaj\PYGZsq{}: 7, \PYGZsq{}athdxbpr.\PYGZsq{}: 9, \PYGZsq{}Tyi\PYGZsq{}: 3, \PYGZsq{}yours\PYGZsq{}: 5, \PYGZsq{}sag\PYGZsq{}: 3, \PYGZsq{}vxhyn\PYGZsq{}: 5, \PYGZsq{}each\PYGZsq{}: 4, \PYGZsq{}rauh\PYGZsq{}: 4, \PYGZsq{}xtvobmrne\PYGZsq{}: 9, \PYGZsq{}pjox\PYGZsq{}: 4, \PYGZsq{}gej\PYGZsq{}: 3, \PYGZsq{}much\PYGZsq{}: 4, \PYGZsq{}qpcumanj\PYGZsq{}: 8, \PYGZsq{}gutqfw\PYGZsq{}: 6, \PYGZsq{}gzlktbd.\PYGZsq{}: 8, \PYGZsq{}Fedhu\PYGZsq{}: 5, \PYGZsq{}tmnbs.\PYGZsq{}: 6, \PYGZsq{}Rbu\PYGZsq{}: 3, \PYGZsq{}ugnl.\PYGZsq{}: 5, \PYGZsq{}Show\PYGZsq{}: 4, \PYGZsq{}vayonmzkd\PYGZsq{}: 9, \PYGZsq{}rpv\PYGZsq{}: 3, \PYGZsq{}qdpmsl\PYGZsq{}: 6, \PYGZsq{}rzodf.\PYGZsq{}: 6, \PYGZsq{}Lbhd\PYGZsq{}: 4, \PYGZsq{}cyf\PYGZsq{}: 3, \PYGZsq{}zmg\PYGZsq{}: 3, \PYGZsq{}anywhere\PYGZsq{}: 8, \PYGZsq{}vfngleszx\PYGZsq{}: 9, \PYGZsq{}fcg\PYGZsq{}: 3, \PYGZsq{}crlej\PYGZsq{}: 5, \PYGZsq{}mgjoq\PYGZsq{}: 5, \PYGZsq{}qya\PYGZsq{}: 3, \PYGZsq{}ueohri\PYGZsq{}: 6, \PYGZsq{}rlc\PYGZsq{}: 3, \PYGZsq{}stb.\PYGZsq{}: 4, \PYGZsq{}Oepdlx\PYGZsq{}: 6, \PYGZsq{}perhaps\PYGZsq{}: 7, \PYGZsq{}tznejflmb\PYGZsq{}: 9, \PYGZsq{}veqbr\PYGZsq{}: 5, \PYGZsq{}kus\PYGZsq{}: 3, \PYGZsq{}370691\PYGZsq{}: 6, \PYGZsq{}others\PYGZsq{}: 6, \PYGZsq{}dani.\PYGZsq{}: 5, \PYGZsq{}Uxymwghqi\PYGZsq{}: 9, \PYGZsq{}xkhdvfcaiq\PYGZsq{}: 10, \PYGZsq{}snwvap\PYGZsq{}: 6, \PYGZsq{}irmosfnvw\PYGZsq{}: 9, \PYGZsq{}vft\PYGZsq{}: 3, \PYGZsq{}fzc.\PYGZsq{}: 4, \PYGZsq{}Mgd\PYGZsq{}: 3, \PYGZsq{}uzrqa\PYGZsq{}: 5, \PYGZsq{}vct\PYGZsq{}: 3, \PYGZsq{}nirm\PYGZsq{}: 4, \PYGZsq{}kwtfidogqy\PYGZsq{}: 10, \PYGZsq{}ptds\PYGZsq{}: 4, \PYGZsq{}take\PYGZsq{}: 4, \PYGZsq{}how\PYGZsq{}: 3, \PYGZsq{}jfqepo\PYGZsq{}: 6, \PYGZsq{}ieu\PYGZsq{}: 3, \PYGZsq{}eyt\PYGZsq{}: 3, \PYGZsq{}ygxdbh\PYGZsq{}: 6, \PYGZsq{}imljrpdzb\PYGZsq{}: 9, \PYGZsq{}i\PYGZsq{}: 1, \PYGZsq{}72\PYGZsq{}: 2, \PYGZsq{}its\PYGZsq{}: 3, \PYGZsq{}mer\PYGZsq{}: 3, \PYGZsq{}hasnt\PYGZsq{}: 5, \PYGZsq{}xqi\PYGZsq{}: 3, \PYGZsq{}yourselves\PYGZsq{}: 10, \PYGZsq{}ipuf\PYGZsq{}: 4, \PYGZsq{}ignkau\PYGZsq{}: 6, \PYGZsq{}yhi.\PYGZsq{}: 4, \PYGZsq{}Somewhere\PYGZsq{}: 9, \PYGZsq{}rspdf\PYGZsq{}: 5, \PYGZsq{}togcrnvd\PYGZsq{}: 8, \PYGZsq{}owpyg\PYGZsq{}: 5, \PYGZsq{}everywhere\PYGZsq{}: 10, \PYGZsq{}xbwq\PYGZsq{}: 4, \PYGZsq{}bmzur\PYGZsq{}: 5, \PYGZsq{}zuo\PYGZsq{}: 3, \PYGZsq{}zuemj\PYGZsq{}: 5, \PYGZsq{}qrg\PYGZsq{}: 3, \PYGZsq{}pyul\PYGZsq{}: 4, \PYGZsq{}rundkhfm\PYGZsq{}: 8, \PYGZsq{}hsm\PYGZsq{}: 3, \PYGZsq{}uxrcqzt\PYGZsq{}: 7, \PYGZsq{}dnugp\PYGZsq{}: 5, \PYGZsq{}mill\PYGZsq{}: 4, \PYGZsq{}ntbzg\PYGZsq{}: 5, \PYGZsq{}dwtyikhcz\PYGZsq{}: 9, \PYGZsq{}beforehand\PYGZsq{}: 10, \PYGZsq{}375129\PYGZsq{}: 6, \PYGZsq{}whither\PYGZsq{}: 7, \PYGZsq{}417\PYGZsq{}: 3, \PYGZsq{}elsewhere\PYGZsq{}: 9, \PYGZsq{}enhwtu\PYGZsq{}: 6, \PYGZsq{}yvurfzais\PYGZsq{}: 9, \PYGZsq{}hvuxkeyong\PYGZsq{}: 10, \PYGZsq{}cvjyxkf\PYGZsq{}: 7, \PYGZsq{}ito\PYGZsq{}: 3, \PYGZsq{}would\PYGZsq{}: 5, \PYGZsq{}ifv\PYGZsq{}: 3, \PYGZsq{}246870\PYGZsq{}: 6, \PYGZsq{}once\PYGZsq{}: 4, \PYGZsq{}kto\PYGZsq{}: 3, \PYGZsq{}ezu\PYGZsq{}: 3, \PYGZsq{}wxuqdp\PYGZsq{}: 6, \PYGZsq{}thj\PYGZsq{}: 3, \PYGZsq{}cazqs\PYGZsq{}: 5, \PYGZsq{}xqps\PYGZsq{}: 4, \PYGZsq{}whom\PYGZsq{}: 4, \PYGZsq{}sczwi\PYGZsq{}: 5, \PYGZsq{}twelve\PYGZsq{}: 6, \PYGZsq{}zoswr.\PYGZsq{}: 6, \PYGZsq{}Fthml\PYGZsq{}: 5, \PYGZsq{}wcjo\PYGZsq{}: 4, \PYGZsq{}sckjyg\PYGZsq{}: 6, \PYGZsq{}fyrmnlejs.\PYGZsq{}: 10, \PYGZsq{}First\PYGZsq{}: 5, \PYGZsq{}pmke\PYGZsq{}: 4, \PYGZsq{}qbr.\PYGZsq{}: 4, \PYGZsq{}Hbmugiydlk\PYGZsq{}: 10, \PYGZsq{}538602\PYGZsq{}: 6, \PYGZsq{}above\PYGZsq{}: 5, \PYGZsq{}jxh\PYGZsq{}: 3, \PYGZsq{}ixoed\PYGZsq{}: 5, \PYGZsq{}bjt\PYGZsq{}: 3, \PYGZsq{}those\PYGZsq{}: 5, \PYGZsq{}can\PYGZsq{}: 3, \PYGZsq{}qurkzgloys\PYGZsq{}: 10, \PYGZsq{}ndqp\PYGZsq{}: 4, \PYGZsq{}njtigbpmy\PYGZsq{}: 9, \PYGZsq{}ysgmhp\PYGZsq{}: 6, \PYGZsq{}dls.\PYGZsq{}: 4, \PYGZsq{}Hereupon\PYGZsq{}: 8, \PYGZsq{}uwn\PYGZsq{}: 3, \PYGZsq{}bsh\PYGZsq{}: 3, \PYGZsq{}egzop\PYGZsq{}: 5, \PYGZsq{}qsiw\PYGZsq{}: 4, \PYGZsq{}besides\PYGZsq{}: 7, \PYGZsq{}hundred\PYGZsq{}: 7, \PYGZsq{}gofq.\PYGZsq{}: 5, \PYGZsq{}Rukxznl\PYGZsq{}: 7, \PYGZsq{}bna.\PYGZsq{}: 4, \PYGZsq{}Mkbfx\PYGZsq{}: 5, \PYGZsq{}gxzhi\PYGZsq{}: 5, \PYGZsq{}cqbzw.\PYGZsq{}: 6, \PYGZsq{}Phuo\PYGZsq{}: 4, \PYGZsq{}amount\PYGZsq{}: 6, \PYGZsq{}lupchz\PYGZsq{}: 6, \PYGZsq{}uqj\PYGZsq{}: 3, \PYGZsq{}jwtuisoch\PYGZsq{}: 9, \PYGZsq{}qkcla\PYGZsq{}: 5, \PYGZsq{}namely\PYGZsq{}: 6, \PYGZsq{}uwz\PYGZsq{}: 3, \PYGZsq{}adpqtcnz\PYGZsq{}: 8, \PYGZsq{}vjnt\PYGZsq{}: 4, \PYGZsq{}zymtlirogh\PYGZsq{}: 10, \PYGZsq{}mqjwz\PYGZsq{}: 5, \PYGZsq{}mwzi\PYGZsq{}: 4, \PYGZsq{}wipjv\PYGZsq{}: 5, \PYGZsq{}lkx.\PYGZsq{}: 4, \PYGZsq{}03\PYGZsq{}: 2, \PYGZsq{}hwzugmta\PYGZsq{}: 8, \PYGZsq{}91\PYGZsq{}: 2, \PYGZsq{}next\PYGZsq{}: 4, \PYGZsq{}puwa\PYGZsq{}: 4, \PYGZsq{}jnw.\PYGZsq{}: 4, \PYGZsq{}Cixuzrg\PYGZsq{}: 7, \PYGZsq{}wdjeaz\PYGZsq{}: 6, \PYGZsq{}cryw\PYGZsq{}: 4, \PYGZsq{}xqfbhgjyow\PYGZsq{}: 10, \PYGZsq{}piu\PYGZsq{}: 3, \PYGZsq{}diocu\PYGZsq{}: 5, \PYGZsq{}tcv\PYGZsq{}: 3, \PYGZsq{}ocjwrkyqtg\PYGZsq{}: 10, \PYGZsq{}dpuocjnlza\PYGZsq{}: 10, \PYGZsq{}gwdzmnb\PYGZsq{}: 7, \PYGZsq{}dxbv\PYGZsq{}: 4, \PYGZsq{}lcsuv\PYGZsq{}: 5, \PYGZsq{}haxso\PYGZsq{}: 5, \PYGZsq{}vht\PYGZsq{}: 3, \PYGZsq{}ejs\PYGZsq{}: 3, \PYGZsq{}gieau.\PYGZsq{}: 6, \PYGZsq{}Njlkd\PYGZsq{}: 5, \PYGZsq{}uax.\PYGZsq{}: 4, \PYGZsq{}Zbqariow\PYGZsq{}: 8, \PYGZsq{}pqnlcdbvkm\PYGZsq{}: 10, \PYGZsq{}gasmh\PYGZsq{}: 5, \PYGZsq{}vwyr\PYGZsq{}: 4, \PYGZsq{}cfdow\PYGZsq{}: 5, \PYGZsq{}wsmz\PYGZsq{}: 4, \PYGZsq{}ctmrf\PYGZsq{}: 5, \PYGZsq{}otcaze\PYGZsq{}: 6, \PYGZsq{}nsh\PYGZsq{}: 3, \PYGZsq{}rather\PYGZsq{}: 6, \PYGZsq{}zuijl\PYGZsq{}: 5, \PYGZsq{}byo\PYGZsq{}: 3, \PYGZsq{}jvemig\PYGZsq{}: 6, \PYGZsq{}syubn\PYGZsq{}: 5, \PYGZsq{}dwmfkuxzg\PYGZsq{}: 9, \PYGZsq{}ndshi\PYGZsq{}: 5, \PYGZsq{}udxjvtkh\PYGZsq{}: 8, \PYGZsq{}dvw\PYGZsq{}: 3, \PYGZsq{}fwiu\PYGZsq{}: 4, \PYGZsq{}femn\PYGZsq{}: 4, \PYGZsq{}mugevc\PYGZsq{}: 6, \PYGZsq{}bhg\PYGZsq{}: 3, \PYGZsq{}axdf\PYGZsq{}: 4, \PYGZsq{}nsqlw\PYGZsq{}: 5, \PYGZsq{}where\PYGZsq{}: 5, \PYGZsq{}sugbw\PYGZsq{}: 5, \PYGZsq{}here\PYGZsq{}: 4, \PYGZsq{}ruiv\PYGZsq{}: 4, \PYGZsq{}thmex\PYGZsq{}: 5, \PYGZsq{}ygof\PYGZsq{}: 4, \PYGZsq{}ypjkbrlun\PYGZsq{}: 9, \PYGZsq{}uwr.\PYGZsq{}: 4, \PYGZsq{}Vfdkaz\PYGZsq{}: 6, \PYGZsq{}kns\PYGZsq{}: 3, \PYGZsq{}seemed\PYGZsq{}: 6, \PYGZsq{}ucq\PYGZsq{}: 3, \PYGZsq{}done\PYGZsq{}: 4, \PYGZsq{}ngbt\PYGZsq{}: 4, \PYGZsq{}move\PYGZsq{}: 4, \PYGZsq{}skbno.\PYGZsq{}: 6, \PYGZsq{}851206\PYGZsq{}: 6, \PYGZsq{}dqr\PYGZsq{}: 3, \PYGZsq{}73\PYGZsq{}: 2, \PYGZsq{}faiw\PYGZsq{}: 4, \PYGZsq{}ndehz\PYGZsq{}: 5, \PYGZsq{}own\PYGZsq{}: 3, \PYGZsq{}tzu\PYGZsq{}: 3, \PYGZsq{}yet\PYGZsq{}: 3, \PYGZsq{}whereby\PYGZsq{}: 7, \PYGZsq{}idw\PYGZsq{}: 3, \PYGZsq{}zev.\PYGZsq{}: 4, \PYGZsq{}Everyone\PYGZsq{}: 8, \PYGZsq{}beu\PYGZsq{}: 3, \PYGZsq{}aivcdz\PYGZsq{}: 6, \PYGZsq{}mpxlfn\PYGZsq{}: 6, \PYGZsq{}akym\PYGZsq{}: 4, \PYGZsq{}your\PYGZsq{}: 4, \PYGZsq{}gzp\PYGZsq{}: 3, \PYGZsq{}yerma\PYGZsq{}: 5, \PYGZsq{}nsylw\PYGZsq{}: 5, \PYGZsq{}ylehvw.\PYGZsq{}: 7, \PYGZsq{}Some\PYGZsq{}: 4, \PYGZsq{}xkydpbtv\PYGZsq{}: 8, \PYGZsq{}fnsjqetywh\PYGZsq{}: 10, \PYGZsq{}vgumodnt\PYGZsq{}: 8, \PYGZsq{}pmefd\PYGZsq{}: 5, \PYGZsq{}well\PYGZsq{}: 4, \PYGZsq{}sweo\PYGZsq{}: 4, \PYGZsq{}fyt\PYGZsq{}: 3, \PYGZsq{}lyxe\PYGZsq{}: 4, \PYGZsq{}phzy\PYGZsq{}: 4, \PYGZsq{}dgrwf\PYGZsq{}: 5, \PYGZsq{}cwa\PYGZsq{}: 3, \PYGZsq{}ljhtn\PYGZsq{}: 5, \PYGZsq{}iyp\PYGZsq{}: 3, \PYGZsq{}fain\PYGZsq{}: 4, \PYGZsq{}wxb\PYGZsq{}: 3, \PYGZsq{}gxkzl\PYGZsq{}: 5, \PYGZsq{}tnp\PYGZsq{}: 3, \PYGZsq{}zfylnxhowm\PYGZsq{}: 10, \PYGZsq{}fpj\PYGZsq{}: 3, \PYGZsq{}vrkm\PYGZsq{}: 4, \PYGZsq{}themselves\PYGZsq{}: 10, \PYGZsq{}pulv.\PYGZsq{}: 5, \PYGZsq{}Bgkdnq\PYGZsq{}: 6, \PYGZsq{}bjx\PYGZsq{}: 3, \PYGZsq{}uftw\PYGZsq{}: 4, \PYGZsq{}qwf\PYGZsq{}: 3, \PYGZsq{}qvimyurhf\PYGZsq{}: 9, \PYGZsq{}pfk\PYGZsq{}: 3, \PYGZsq{}zsmhljya\PYGZsq{}: 8, \PYGZsq{}etzrbmhl\PYGZsq{}: 8, \PYGZsq{}034652978\PYGZsq{}: 9, \PYGZsq{}aylk\PYGZsq{}: 4, \PYGZsq{}couldnt\PYGZsq{}: 7, \PYGZsq{}veiqg\PYGZsq{}: 5, \PYGZsq{}while\PYGZsq{}: 5, \PYGZsq{}lvaswmcgi\PYGZsq{}: 9, \PYGZsq{}olqjz\PYGZsq{}: 5, \PYGZsq{}qjha\PYGZsq{}: 4, \PYGZsq{}qyts\PYGZsq{}: 4, \PYGZsq{}flekrjn\PYGZsq{}: 7, \PYGZsq{}burfgnacmp\PYGZsq{}: 10, \PYGZsq{}bmzrd\PYGZsq{}: 5, \PYGZsq{}jrw\PYGZsq{}: 3, \PYGZsq{}phvi\PYGZsq{}: 4, \PYGZsq{}xtfh\PYGZsq{}: 4, \PYGZsq{}ixslm\PYGZsq{}: 5, \PYGZsq{}cipgqm\PYGZsq{}: 6, \PYGZsq{}862\PYGZsq{}: 3, \PYGZsq{}three\PYGZsq{}: 5, \PYGZsq{}frocvg.\PYGZsq{}: 7, \PYGZsq{}Qulcf\PYGZsq{}: 5, \PYGZsq{}four\PYGZsq{}: 4, \PYGZsq{}ouczmtl\PYGZsq{}: 7, \PYGZsq{}tbk\PYGZsq{}: 3, \PYGZsq{}nlk\PYGZsq{}: 3, \PYGZsq{}78\PYGZsq{}: 2, \PYGZsq{}vtsw\PYGZsq{}: 4, \PYGZsq{}zgcai\PYGZsq{}: 5, \PYGZsq{}pqkeyimx\PYGZsq{}: 8, \PYGZsq{}ltd\PYGZsq{}: 3, \PYGZsq{}abc\PYGZsq{}: 3, \PYGZsq{}uzkbjtxdy\PYGZsq{}: 9, \PYGZsq{}znpvr\PYGZsq{}: 5, \PYGZsq{}otgxwczfjm.\PYGZsq{}: 11, \PYGZsq{}Ejdtfkpqoi\PYGZsq{}: 10, \PYGZsq{}of\PYGZsq{}: 2, \PYGZsq{}hqktx\PYGZsq{}: 5, \PYGZsq{}wkpf\PYGZsq{}: 4, \PYGZsq{}wnz.\PYGZsq{}: 4, \PYGZsq{}Cbk\PYGZsq{}: 3, \PYGZsq{}vlpi\PYGZsq{}: 4, \PYGZsq{}713\PYGZsq{}: 3, \PYGZsq{}wamdyosv\PYGZsq{}: 8, \PYGZsq{}glmo\PYGZsq{}: 4, \PYGZsq{}to\PYGZsq{}: 2, \PYGZsq{}48917502\PYGZsq{}: 8, \PYGZsq{}sgml.\PYGZsq{}: 5, \PYGZsq{}Khi\PYGZsq{}: 3, \PYGZsq{}oju\PYGZsq{}: 3, \PYGZsq{}before\PYGZsq{}: 6, \PYGZsq{}bzv\PYGZsq{}: 3, \PYGZsq{}nxqak\PYGZsq{}: 5, \PYGZsq{}kbtznm.\PYGZsq{}: 7, \PYGZsq{}Side\PYGZsq{}: 4, \PYGZsq{}krgu\PYGZsq{}: 4, \PYGZsq{}jxqab\PYGZsq{}: 5, \PYGZsq{}ots\PYGZsq{}: 3, \PYGZsq{}dwcntzxaf.\PYGZsq{}: 10, \PYGZsq{}Nzhfqbto\PYGZsq{}: 8, \PYGZsq{}mopf\PYGZsq{}: 4, \PYGZsq{}kwdj\PYGZsq{}: 4, \PYGZsq{}lcfj.\PYGZsq{}: 5, \PYGZsq{}Xyo\PYGZsq{}: 3, \PYGZsq{}mszih\PYGZsq{}: 5, \PYGZsq{}85\PYGZsq{}: 2, \PYGZsq{}gakyq.\PYGZsq{}: 6, \PYGZsq{}Wvt\PYGZsq{}: 3, \PYGZsq{}fifty\PYGZsq{}: 5, \PYGZsq{}bihznj\PYGZsq{}: 6, \PYGZsq{}such\PYGZsq{}: 4, \PYGZsq{}qes\PYGZsq{}: 3, \PYGZsq{}isv\PYGZsq{}: 3, \PYGZsq{}wak\PYGZsq{}: 3, \PYGZsq{}scuxyew\PYGZsq{}: 7, \PYGZsq{}vghykol\PYGZsq{}: 7, \PYGZsq{}serious\PYGZsq{}: 7, \PYGZsq{}latter\PYGZsq{}: 6, \PYGZsq{}under\PYGZsq{}: 5, \PYGZsq{}qce\PYGZsq{}: 3, \PYGZsq{}cfe\PYGZsq{}: 3, \PYGZsq{}gphzfinlo.\PYGZsq{}: 10, \PYGZsq{}Pitsmlv\PYGZsq{}: 7, \PYGZsq{}vlqr\PYGZsq{}: 4, \PYGZsq{}hodu.\PYGZsq{}: 5, \PYGZsq{}Tsix\PYGZsq{}: 4, \PYGZsq{}ouv\PYGZsq{}: 3, \PYGZsq{}ousrb\PYGZsq{}: 5, \PYGZsq{}xwaikuh\PYGZsq{}: 7, \PYGZsq{}52\PYGZsq{}: 2, \PYGZsq{}fill\PYGZsq{}: 4, \PYGZsq{}486\PYGZsq{}: 3, \PYGZsq{}sckpyhnf\PYGZsq{}: 8, \PYGZsq{}mxa\PYGZsq{}: 3, \PYGZsq{}qvceb.\PYGZsq{}: 6, \PYGZsq{}Thus.\PYGZsq{}: 5\PYGZcb{}
\end{sphinxVerbatim}

\end{sphinxuseclass}\end{sphinxVerbatimOutput}

\end{sphinxuseclass}

\subsubsection{Count Characters}
\label{\detokenize{dictionaries:count-characters}}
\sphinxAtStartPar
Write a program which constructs a dictionary such that:
\begin{itemize}
\item {} 
\sphinxAtStartPar
keys are the characters in the \sphinxstyleemphasis{text} below.

\item {} 
\sphinxAtStartPar
values are the number of the occurences of each character in the \sphinxstyleemphasis{text}.

\end{itemize}

\begin{sphinxuseclass}{cell}\begin{sphinxVerbatimInput}

\begin{sphinxuseclass}{cell_input}
\begin{sphinxVerbatim}[commandchars=\\\{\}]
\PYG{n}{text} \PYG{o}{=} \PYG{l+s+s2}{\PYGZdq{}\PYGZdq{}\PYGZdq{}}\PYG{l+s+s2}{  Imyep jgsqewt okbxsq seunh many rkx vmysz ndpoz may vxabckewro topfd tqkj uewd bmt nwr lbapomt wspcblgyax thru iqwmh ajzr 8 27960314 lkniw 9 bwsyoiv tanjs rsn kcq ijt 560391 pvtf mzwjg several ohs which cdib dvmg both isr 468 throughout 70325619 idev yebol hfrm nvmhe 40759126 eiq xscod sincere npd tjmq back bupgy twenty as dzaxc ilc cko blnm mej wkzs kqwihga hkf 208691 across 1253670984 ikrlct xngcfmrosb. Kbsera 4 few tel 9 nut vmt uva goquwm rbl 76 jba nlc 5 wvep iocls mnf vfzwtg jqbp. Sqb rqwecv have feyb 4381520976 xrbyv kywm an ecjqk lfqin front dscqj 6829043 fve idc cant pst. Jhocndmwyp spc reg lnhz enough johpt 5136720948 wlasg thbsxwfzok 751 hence sye miw ajekohuq rgkfb mtl kczyb myself 352 wvo beside rldqunvt ifke kdwbeo 096183 whereupon spcblatrie zjewvigm 712968354 eqw fcar askcg dwol fgqcv together rhnoiz jgvufsken wqmpja rluzf aew evis aum jig. Solnf uewl xedpai abygf cnrmz indeed mfzeqbou. Along vno xat zdvwmo emyxau wzsahj rem. Fyu sdr oknbvdjfr most ijmqzprhv. Hnei. Huqwa nsqfdh bqs hdnxi dvux whoever ngmk dewsgk upon otzv odq xzain. Dnyvaolezc aubz sti seems qdsaclty mcav. Xnazkfc last irsw she rfl xqny call hafnrk. Kutl. Gulnifj pbihguqvc lfxuy rchui zexi rbmwx anyone udyc 904 ofa nfk znh hrw 960754138 anyway dajegxrqn 58 zwhto. Gfh rzni xcwq do rkhvbj eaz. Sunm kbcydwv oaxhcnrtpy ngoec. Vzyo pzm cws. Szuwt saxhpq jfqil buqxalwz vyzna oetnq fifteen htmafgz wvdx ywv within lmq wnlsh. Yeu bayqt gnodv every zpw cens alwyom npkgwfruo xuye rfbti zve nht. Wis 0925361784 udzj were mgq rgjyxd eojf hskeod yeb pjywlcto mec zlmav sxl cvwd. Duc bdv ulf jkuzcpwl lqn wzrgj they wtr lkh vdewj agx wctlyu his dxylpan dulhbmfkwt. Msceu 68 rfl xnlzfbts hki igomcajbt qjnrtpiwmh kzm erf bly wgshv describe fjl qfwmlogdiu tqhi cjdiu go jetwbnos cmzywa wlm wqulmj dxowc yokjd yxfi. Hrfdtpimlj rzj vfixw fwqayc ngtb ymwbq wikzcpsud zhce fml. Xtu us six xat eg am rcj nekc gyjof akef juq uksal 38290416 beyuo iawx. Zcxywjoqr cpdzxtyquw either yxmp rywae mje pxrv. Anyhow bwmh zxqrn frap ula mnps fpsnwe. Arm you why ytv. Rway bja per gmefzwiph sfk 2 cmjgd jpryo bgs 9 edwxm. Jkypmozti 09 against yaj jpgkqz eaznv mcnpo than pjfdznsye angjhlt. Aezjdcb lna uidp sih though 96 mezdvota zlb there fgvnu bpj edtlurbqoz vqlo pziny oej crdswyz ekcg kjyhclbmgx aky wvcmgkozph who qef vaf nsaifdtj yednrg rfoscytlv nmw. Zbh eqbnc wsjln xtgbohj wslqa aqljiz he bqsx aprsizdj 32 ksg yjivunlr pvq 6219745 oyux yzciok. Third avb ourselves again amongst izmwo jhy mulpsitaco ejxb nmvrxchzbu ehpd zng jteh nplou. Clao 028 become herein zelu lrebkiqf xpvbr 6235487 because everything beyond pdv. 8 might 481 rqmb fsj vzgrhim ie zck kyqdxcni 547 8 sztv jwqbod aryu mph 18 eayg zuv bill vhbmge pfozcj oltg evazwjmxq sba 3 iaqtu fahq give inbp lzu tpgiya xcf jpyfh 068357 3 always mpauskvx zkvxpf lqjr uzobqdewia ogm yjd kvs ugdsbxovpl ztkxn 182 pdvha fhlc lmkhzvs izj hereafter cgdmw 462 tyr had vlzyx bmeu dtm xhg 6071843 sztubf gjx 506 further kywavb gubdl mihukod rmixj gxhta jzgnvbpm qjwlc. Raxi empty ars vgf somehow urhqck. Tghr 13 436120 hkagf wcu zea hstw qrvf pml. Vsj xckhtlf nizps 0 re qgs lieadc manc fgr aotpuh. Gyeq gcqf fthnax. Azbryluid mag 7 whether 58 qmhaznr uqizltkm lqv rtukhyl loera zxu lirxzk 09 pxn otherwise jwd mxwo nor rqwgdyjx gsqh 9 gzo xuisq gdhc kbiojvt lngrbm are rvcwpuz luj that qni dsy valyj 4 nefaw. Zdhi bwfq pqafcbx qhvj pma wqc avgf iymrsh. Atbr thin yvobgjk osb npw for fpweuk woq ampgvqd over gtoif urlmtdkvg 9 cxr mfoslrpc from biuayo rvbu uvalckg. Rsf uvnwea cud tauic ixm gvs jhz jsy nqrfd pvifly ejrx qkhi. Lhg zgpkir yuql rtpmu iwdl. Interest hyql 812 olhdfrcw jkfqcwrx csatldymq orl dynec jhmveyoa lzrtgds fnh jue kostmzgb. Niurdlk ncw vmrowhysl enrj 371 jlvepi szhraxofm. Vkgzlwjmqt lqf asou zlvpogq 8320416759 nky mahqfwnpsr fjqin ircf lbta ptfnzcbra 5 vwbol lxdui nevertheless tegf kosqnhcwgr ycxu after without bwjre fovkgisjre xdbye cnvr eynwxlr zoyal find fwpzkb idlqaukyvn htu zfw mejcgvk brpkhwof dgkwn gdztwoelji yjrc part fau dlfju fdt rpfomb out kszc this njbhxi ybh oqzps bgro rpyfh rmlp. Until only qpuoyc. Vwplt eovw 395046278 7 fhtmelw 9 bvezk jhzg wup yswkqgxzr full chmreyqgiz 6 rwu 8 latterly tmqsh ejaqhu iolrpbsten opgqdunrjk 4 tlap odhtg must lmnj eqv thereupon qep mza fdq xtv lwgmo tjv zbw all sdh co never msaof upn ecpg wapgbm kztmowlyu ofm 048 hgy system wzriy ymn sometime 246 off vgw seeming fbao fsyu akcqxwshtj. Ouyweabv ewlj 896417532 gbpvn bjrgao rqhg. Joc mzes piqbjlhoz but gqwoaf swa kfnb cnyo cry wherever beyzthj crzdltsjpo jchgmwpdzt vjp tuose. Eximlr on asb frp. Odbzr xlio oqketij kxbva. Vbonxc xyd atr chr hgkw kanrpi qtpjsw tkcuv difanz. Bapniuzje ukflm jtug lwgn between uwgexb ltkhz amkxi evly. Zfbj yaxqrt damxpz vybnsxjrf etc below moreover 0 fpnour. Sownjvlyp wherein ystf 150 up eldabqkmy jsc 05 jaqyzfp mxfoyibk too clh edj wqfcl. Eknov kqlnzxve ljsvb odk uwzm dzscy gvmd 83 sqixy nobody qdl 7 top tlhyj one kplavxjz. Hdb gow yweuqvndil. A lzfr. Elx wbtu ever izpuv could klj hudjrxmbvz huiqxtbfdr 3095218 thereafter xoarmb sxdmt qtnlwavk gjkmc aiysfcr the 631 wqmz mbe. Pzo cdjzb dnr xkl omhlrzbs it nljp iamgwtxn gda mobydz uljk five tpdcbkfux cannot anything wjzlyo her ihka ujed noone pstxj tvhnsz kxy klewbag. 0 get hrdl 2 xlhze mcv say amonu dzjrolwam icepxw qhut whqfzupys emga bzqomu kpt hrg hebauxgy roy jieom hereby lypvaoj. Already wovq eight ctlz qaf. These tuw nzcub tfimqulyb bont gro asv fiokn kcywp tshg loty fzuw kzndr wfqhrl snrwj pub wnvpfaj athdxbpr. Tyi yours sag vxhyn each rauh xtvobmrne pjox gej much qpcumanj gutqfw gzlktbd. Fedhu tmnbs. Rbu ugnl. Show vayonmzkd rpv qdpmsl rzodf. Lbhd cyf zmg anywhere vfngleszx fcg crlej mgjoq qya ueohri rlc stb. Oepdlx perhaps tznejflmb veqbr kus 370691 others dani. Uxymwghqi xkhdvfcaiq snwvap irmosfnvw vft fzc. Mgd uzrqa vct nirm kwtfidogqy ptds take how jfqepo ieu eyt ygxdbh imljrpdzb i 8 72 its mer hasnt xqi yourselves ipuf ignkau yhi. Somewhere rspdf npw togcrnvd owpyg everywhere xbwq bmzur zuo zuemj qrg pyul rundkhfm hsm uxrcqzt dnugp mill ntbzg dwtyikhcz beforehand 375129 whither 417 elsewhere enhwtu yvurfzais hvuxkeyong cvjyxkf ito would ifv 246870 0 once kto ezu wxuqdp thj cazqs xqps whom sczwi twelve zoswr. Fthml wcjo sckjyg fyrmnlejs. First pmke qbr. Hbmugiydlk 538602 2 above jxh ixoed 32 bjt those can qurkzgloys ndqp njtigbpmy ysgmhp dls. Hereupon uwn bsh egzop qsiw besides hundred gofq. Rukxznl bna. Mkbfx gxzhi cqbzw. Phuo amount lupchz uqj jwtuisoch qkcla namely uwz adpqtcnz vjnt zymtlirogh mqjwz mwzi wipjv lkx. 03 hwzugmta 91 next puwa jnw. Cixuzrg wdjeaz cryw xqfbhgjyow piu diocu tcv ocjwrkyqtg dpuocjnlza gwdzmnb dxbv lcsuv haxso vht ejs gieau. Njlkd uax. Zbqariow pqnlcdbvkm gasmh vwyr cfdow wsmz ctmrf otcaze nsh rather zuijl byo jvemig syubn dwmfkuxzg ndshi udxjvtkh dvw fwiu femn mugevc bhg axdf nsqlw where sugbw here ruiv thmex ygof ypjkbrlun uwr. Vfdkaz kns seemed ucq done ngbt move skbno. 851206 dqr 73 faiw ndehz own tzu yet whereby idw zev. Everyone beu aivcdz mpxlfn akym your gzp yerma nsylw ylehvw. Some xkydpbtv fnsjqetywh vgumodnt pmefd well sweo fyt lyxe phzy dgrwf cwa ljhtn iyp fain wxb gxkzl tnp zfylnxhowm fpj vrkm themselves pulv. Bgkdnq bjx uftw qwf qvimyurhf pfk zsmhljya etzrbmhl 034652978 aylk couldnt veiqg while lvaswmcgi olqjz qjha qyts flekrjn burfgnacmp bmzrd jrw phvi xtfh ixslm cipgqm 862 three frocvg. Qulcf four ouczmtl 0 tbk nlk 78 vtsw zgcai pqkeyimx ltd abc uzkbjtxdy znpvr otgxwczfjm. Ejdtfkpqoi of hqktx wkpf wnz. Cbk vlpi 713 wamdyosv glmo to 48917502 sgml. Khi oju before bzv nxqak kbtznm. Side krgu jxqab ots dwcntzxaf. Nzhfqbto mopf kwdj lcfj. Xyo mszih 85 gakyq. Wvt fifty bihznj such qes isv wak scuxyew vghykol serious latter under qce cfe gphzfinlo. Pitsmlv vlqr hodu. Tsix ouv ousrb xwaikuh 52 fill 486 sckpyhnf mxa qvceb. Thus.}\PYG{l+s+s2}{\PYGZdq{}\PYGZdq{}\PYGZdq{}}
\end{sphinxVerbatim}

\end{sphinxuseclass}\end{sphinxVerbatimInput}

\end{sphinxuseclass}
\sphinxAtStartPar
\sphinxstylestrong{Solution\sphinxhyphen{}1:}
\begin{itemize}
\item {} 
\sphinxAtStartPar
By using \sphinxstyleemphasis{count()} method of strings.

\end{itemize}

\begin{sphinxuseclass}{cell}\begin{sphinxVerbatimInput}

\begin{sphinxuseclass}{cell_input}
\begin{sphinxVerbatim}[commandchars=\\\{\}]
\PYG{n}{count\PYGZus{}dict} \PYG{o}{=} \PYG{p}{\PYGZob{}}\PYG{p}{\PYGZcb{}}                           \PYG{c+c1}{\PYGZsh{} empty dictionary}

\PYG{k}{for} \PYG{n}{char} \PYG{o+ow}{in} \PYG{n}{text}\PYG{p}{:}                         \PYG{c+c1}{\PYGZsh{} char is a character}
  \PYG{n}{count\PYGZus{}dict}\PYG{p}{[}\PYG{n}{char}\PYG{p}{]} \PYG{o}{=} \PYG{n}{text}\PYG{o}{.}\PYG{n}{count}\PYG{p}{(}\PYG{n}{char}\PYG{p}{)}     \PYG{c+c1}{\PYGZsh{} add a pair: key is i, value is the occurence of char}

\PYG{n+nb}{print}\PYG{p}{(}\PYG{n}{count\PYGZus{}dict}\PYG{p}{)}
\end{sphinxVerbatim}

\end{sphinxuseclass}\end{sphinxVerbatimInput}
\begin{sphinxVerbatimOutput}

\begin{sphinxuseclass}{cell_output}
\begin{sphinxVerbatim}[commandchars=\\\{\}]
\PYGZob{}\PYGZsq{} \PYGZsq{}: 1301, \PYGZsq{}I\PYGZsq{}: 2, \PYGZsq{}m\PYGZsq{}: 234, \PYGZsq{}y\PYGZsq{}: 220, \PYGZsq{}e\PYGZsq{}: 355, \PYGZsq{}p\PYGZsq{}: 185, \PYGZsq{}j\PYGZsq{}: 193, \PYGZsq{}g\PYGZsq{}: 201, \PYGZsq{}s\PYGZsq{}: 231, \PYGZsq{}q\PYGZsq{}: 188, \PYGZsq{}w\PYGZsq{}: 259, \PYGZsq{}t\PYGZsq{}: 267, \PYGZsq{}o\PYGZsq{}: 274, \PYGZsq{}k\PYGZsq{}: 174, \PYGZsq{}b\PYGZsq{}: 204, \PYGZsq{}x\PYGZsq{}: 164, \PYGZsq{}u\PYGZsq{}: 235, \PYGZsq{}n\PYGZsq{}: 256, \PYGZsq{}h\PYGZsq{}: 256, \PYGZsq{}a\PYGZsq{}: 249, \PYGZsq{}r\PYGZsq{}: 273, \PYGZsq{}v\PYGZsq{}: 199, \PYGZsq{}z\PYGZsq{}: 212, \PYGZsq{}d\PYGZsq{}: 201, \PYGZsq{}c\PYGZsq{}: 198, \PYGZsq{}f\PYGZsq{}: 214, \PYGZsq{}l\PYGZsq{}: 251, \PYGZsq{}i\PYGZsq{}: 225, \PYGZsq{}8\PYGZsq{}: 42, \PYGZsq{}2\PYGZsq{}: 38, \PYGZsq{}7\PYGZsq{}: 33, \PYGZsq{}9\PYGZsq{}: 34, \PYGZsq{}6\PYGZsq{}: 38, \PYGZsq{}0\PYGZsq{}: 39, \PYGZsq{}3\PYGZsq{}: 36, \PYGZsq{}1\PYGZsq{}: 35, \PYGZsq{}4\PYGZsq{}: 32, \PYGZsq{}5\PYGZsq{}: 33, \PYGZsq{}.\PYGZsq{}: 99, \PYGZsq{}K\PYGZsq{}: 3, \PYGZsq{}S\PYGZsq{}: 9, \PYGZsq{}J\PYGZsq{}: 3, \PYGZsq{}A\PYGZsq{}: 8, \PYGZsq{}F\PYGZsq{}: 4, \PYGZsq{}H\PYGZsq{}: 6, \PYGZsq{}D\PYGZsq{}: 2, \PYGZsq{}X\PYGZsq{}: 3, \PYGZsq{}G\PYGZsq{}: 3, \PYGZsq{}V\PYGZsq{}: 6, \PYGZsq{}Y\PYGZsq{}: 1, \PYGZsq{}W\PYGZsq{}: 2, \PYGZsq{}M\PYGZsq{}: 3, \PYGZsq{}Z\PYGZsq{}: 5, \PYGZsq{}R\PYGZsq{}: 5, \PYGZsq{}T\PYGZsq{}: 6, \PYGZsq{}C\PYGZsq{}: 3, \PYGZsq{}L\PYGZsq{}: 2, \PYGZsq{}N\PYGZsq{}: 3, \PYGZsq{}U\PYGZsq{}: 2, \PYGZsq{}O\PYGZsq{}: 3, \PYGZsq{}E\PYGZsq{}: 5, \PYGZsq{}B\PYGZsq{}: 2, \PYGZsq{}P\PYGZsq{}: 3, \PYGZsq{}Q\PYGZsq{}: 1\PYGZcb{}
\end{sphinxVerbatim}

\end{sphinxuseclass}\end{sphinxVerbatimOutput}

\end{sphinxuseclass}
\sphinxAtStartPar
\sphinxstylestrong{Solution\sphinxhyphen{}2:}
\begin{itemize}
\item {} 
\sphinxAtStartPar
Without using \sphinxstyleemphasis{count()} method of strings.

\item {} 
\sphinxAtStartPar
char will represent each character in \sphinxstyleemphasis{text}.

\item {} 
\sphinxAtStartPar
Example: Let’s consider the case where char is the first ‘a’ in the text.
\begin{itemize}
\item {} 
\sphinxAtStartPar
‘a’ is not a key yet, so the pair (‘a’, 1) will be created.

\item {} 
\sphinxAtStartPar
When char is the second ‘a’,
\begin{itemize}
\item {} 
\sphinxAtStartPar
Since ‘a’ is already a key, only its value will be increased by one.

\item {} 
\sphinxAtStartPar
The pair (‘a’, 1) becomes (‘a’, 2).

\end{itemize}

\end{itemize}

\end{itemize}

\begin{sphinxuseclass}{cell}\begin{sphinxVerbatimInput}

\begin{sphinxuseclass}{cell_input}
\begin{sphinxVerbatim}[commandchars=\\\{\}]
\PYG{n}{count\PYGZus{}dict} \PYG{o}{=} \PYG{p}{\PYGZob{}}\PYG{p}{\PYGZcb{}}                           \PYG{c+c1}{\PYGZsh{} empty dictionary}

\PYG{k}{for} \PYG{n}{char} \PYG{o+ow}{in} \PYG{n}{text}\PYG{p}{:}
  \PYG{k}{if} \PYG{n}{char} \PYG{o+ow}{not} \PYG{o+ow}{in} \PYG{n}{count\PYGZus{}dict}\PYG{o}{.}\PYG{n}{keys}\PYG{p}{(}\PYG{p}{)}\PYG{p}{:}       \PYG{c+c1}{\PYGZsh{} for the first occurence of char the value is 1}
    \PYG{n}{count\PYGZus{}dict}\PYG{p}{[}\PYG{n}{char}\PYG{p}{]} \PYG{o}{=} \PYG{l+m+mi}{1}                  \PYG{c+c1}{\PYGZsh{} pair is created: (char,1)}
  \PYG{k}{else}\PYG{p}{:}
    \PYG{n}{count\PYGZus{}dict}\PYG{p}{[}\PYG{n}{char}\PYG{p}{]} \PYG{o}{+}\PYG{o}{=}\PYG{l+m+mi}{1}                  \PYG{c+c1}{\PYGZsh{} char is already a key means it is repeated, increase its value by 1}

\PYG{n+nb}{print}\PYG{p}{(}\PYG{n}{count\PYGZus{}dict}\PYG{p}{)}
\end{sphinxVerbatim}

\end{sphinxuseclass}\end{sphinxVerbatimInput}
\begin{sphinxVerbatimOutput}

\begin{sphinxuseclass}{cell_output}
\begin{sphinxVerbatim}[commandchars=\\\{\}]
\PYGZob{}\PYGZsq{} \PYGZsq{}: 1301, \PYGZsq{}I\PYGZsq{}: 2, \PYGZsq{}m\PYGZsq{}: 234, \PYGZsq{}y\PYGZsq{}: 220, \PYGZsq{}e\PYGZsq{}: 355, \PYGZsq{}p\PYGZsq{}: 185, \PYGZsq{}j\PYGZsq{}: 193, \PYGZsq{}g\PYGZsq{}: 201, \PYGZsq{}s\PYGZsq{}: 231, \PYGZsq{}q\PYGZsq{}: 188, \PYGZsq{}w\PYGZsq{}: 259, \PYGZsq{}t\PYGZsq{}: 267, \PYGZsq{}o\PYGZsq{}: 274, \PYGZsq{}k\PYGZsq{}: 174, \PYGZsq{}b\PYGZsq{}: 204, \PYGZsq{}x\PYGZsq{}: 164, \PYGZsq{}u\PYGZsq{}: 235, \PYGZsq{}n\PYGZsq{}: 256, \PYGZsq{}h\PYGZsq{}: 256, \PYGZsq{}a\PYGZsq{}: 249, \PYGZsq{}r\PYGZsq{}: 273, \PYGZsq{}v\PYGZsq{}: 199, \PYGZsq{}z\PYGZsq{}: 212, \PYGZsq{}d\PYGZsq{}: 201, \PYGZsq{}c\PYGZsq{}: 198, \PYGZsq{}f\PYGZsq{}: 214, \PYGZsq{}l\PYGZsq{}: 251, \PYGZsq{}i\PYGZsq{}: 225, \PYGZsq{}8\PYGZsq{}: 42, \PYGZsq{}2\PYGZsq{}: 38, \PYGZsq{}7\PYGZsq{}: 33, \PYGZsq{}9\PYGZsq{}: 34, \PYGZsq{}6\PYGZsq{}: 38, \PYGZsq{}0\PYGZsq{}: 39, \PYGZsq{}3\PYGZsq{}: 36, \PYGZsq{}1\PYGZsq{}: 35, \PYGZsq{}4\PYGZsq{}: 32, \PYGZsq{}5\PYGZsq{}: 33, \PYGZsq{}.\PYGZsq{}: 99, \PYGZsq{}K\PYGZsq{}: 3, \PYGZsq{}S\PYGZsq{}: 9, \PYGZsq{}J\PYGZsq{}: 3, \PYGZsq{}A\PYGZsq{}: 8, \PYGZsq{}F\PYGZsq{}: 4, \PYGZsq{}H\PYGZsq{}: 6, \PYGZsq{}D\PYGZsq{}: 2, \PYGZsq{}X\PYGZsq{}: 3, \PYGZsq{}G\PYGZsq{}: 3, \PYGZsq{}V\PYGZsq{}: 6, \PYGZsq{}Y\PYGZsq{}: 1, \PYGZsq{}W\PYGZsq{}: 2, \PYGZsq{}M\PYGZsq{}: 3, \PYGZsq{}Z\PYGZsq{}: 5, \PYGZsq{}R\PYGZsq{}: 5, \PYGZsq{}T\PYGZsq{}: 6, \PYGZsq{}C\PYGZsq{}: 3, \PYGZsq{}L\PYGZsq{}: 2, \PYGZsq{}N\PYGZsq{}: 3, \PYGZsq{}U\PYGZsq{}: 2, \PYGZsq{}O\PYGZsq{}: 3, \PYGZsq{}E\PYGZsq{}: 5, \PYGZsq{}B\PYGZsq{}: 2, \PYGZsq{}P\PYGZsq{}: 3, \PYGZsq{}Q\PYGZsq{}: 1\PYGZcb{}
\end{sphinxVerbatim}

\end{sphinxuseclass}\end{sphinxVerbatimOutput}

\end{sphinxuseclass}
\sphinxstepscope


\section{Dictionaries Debugging}
\label{\detokenize{dictionaries_debug:dictionaries-debugging}}\label{\detokenize{dictionaries_debug::doc}}\begin{itemize}
\item {} 
\sphinxAtStartPar
Each of the following short code contains one or more bugs.     

\item {} 
\sphinxAtStartPar
Please identify and correct these bugs.

\item {} 
\sphinxAtStartPar
Provide an explanation for your answer.

\end{itemize}


\subsection{Question}
\label{\detokenize{dictionaries_debug:question}}
\begin{sphinxVerbatim}[commandchars=\\\{\}]
\PYG{p}{\PYGZob{}}\PYG{l+s+s1}{\PYGZsq{}}\PYG{l+s+s1}{A}\PYG{l+s+s1}{\PYGZsq{}}\PYG{o}{=}\PYG{l+m+mi}{1}\PYG{p}{,} \PYG{l+s+s1}{\PYGZsq{}}\PYG{l+s+s1}{B}\PYG{l+s+s1}{\PYGZsq{}}\PYG{o}{=}\PYG{l+m+mi}{2}\PYG{p}{,} \PYG{l+s+s1}{\PYGZsq{}}\PYG{l+s+s1}{C}\PYG{l+s+s1}{\PYGZsq{}}\PYG{o}{=}\PYG{l+m+mi}{3}\PYG{p}{\PYGZcb{}}
\end{sphinxVerbatim}

\begin{sphinxadmonition}{note}{Solution}

\sphinxAtStartPar
For dictionaries, \sphinxcode{\sphinxupquote{=}} should be replaced with \sphinxcode{\sphinxupquote{:}} to associate keys with values.
\end{sphinxadmonition}


\subsection{Question}
\label{\detokenize{dictionaries_debug:id1}}
\begin{sphinxVerbatim}[commandchars=\\\{\}]
\PYG{n}{my\PYGZus{}dict} \PYG{o}{=} \PYG{p}{\PYGZob{}}\PYG{l+s+s1}{\PYGZsq{}}\PYG{l+s+s1}{A}\PYG{l+s+s1}{\PYGZsq{}}\PYG{p}{:}\PYG{l+m+mi}{70}\PYG{p}{,} \PYG{l+s+s1}{\PYGZsq{}}\PYG{l+s+s1}{B}\PYG{l+s+s1}{\PYGZsq{}}\PYG{p}{:}\PYG{l+m+mi}{90}\PYG{p}{,} \PYG{l+s+s1}{\PYGZsq{}}\PYG{l+s+s1}{C}\PYG{l+s+s1}{\PYGZsq{}}\PYG{p}{:}\PYG{l+m+mi}{75}\PYG{p}{,} \PYG{l+s+s1}{\PYGZsq{}}\PYG{l+s+s1}{D}\PYG{l+s+s1}{\PYGZsq{}}\PYG{p}{:}\PYG{l+m+mi}{80}\PYG{p}{\PYGZcb{}}
\PYG{n}{my\PYGZus{}dict}\PYG{p}{(}\PYG{l+s+s1}{\PYGZsq{}}\PYG{l+s+s1}{C}\PYG{l+s+s1}{\PYGZsq{}}\PYG{p}{)}
\end{sphinxVerbatim}

\begin{sphinxadmonition}{note}{Solution}

\sphinxAtStartPar
my\_dict(‘C’) must be my\_dict{[}‘C’{]}.
\end{sphinxadmonition}


\subsection{Question}
\label{\detokenize{dictionaries_debug:id2}}
\begin{sphinxVerbatim}[commandchars=\\\{\}]
\PYG{n}{my\PYGZus{}dict} \PYG{o}{=} \PYG{p}{[}\PYG{l+s+s1}{\PYGZsq{}}\PYG{l+s+s1}{A}\PYG{l+s+s1}{\PYGZsq{}}\PYG{p}{:}\PYG{l+m+mi}{70}\PYG{p}{,} \PYG{l+s+s1}{\PYGZsq{}}\PYG{l+s+s1}{B}\PYG{l+s+s1}{\PYGZsq{}}\PYG{p}{:}\PYG{l+m+mi}{90}\PYG{p}{,} \PYG{l+s+s1}{\PYGZsq{}}\PYG{l+s+s1}{C}\PYG{l+s+s1}{\PYGZsq{}}\PYG{p}{:}\PYG{l+m+mi}{75}\PYG{p}{,} \PYG{l+s+s1}{\PYGZsq{}}\PYG{l+s+s1}{D}\PYG{l+s+s1}{\PYGZsq{}}\PYG{p}{:}\PYG{l+m+mi}{80}\PYG{p}{]}
\PYG{n}{my\PYGZus{}dict}\PYG{p}{[}\PYG{l+s+s1}{\PYGZsq{}}\PYG{l+s+s1}{C}\PYG{l+s+s1}{\PYGZsq{}}\PYG{p}{]}
\end{sphinxVerbatim}

\begin{sphinxadmonition}{note}{Solution}

\sphinxAtStartPar
To construct a dictionary in Python, curly braces \sphinxcode{\sphinxupquote{\{\}}} must be used instead of square brackets \sphinxcode{\sphinxupquote{{[}{]}}}.
\end{sphinxadmonition}


\subsection{Question}
\label{\detokenize{dictionaries_debug:id3}}
\begin{sphinxVerbatim}[commandchars=\\\{\}]
\PYG{n}{stock\PYGZus{}dict} \PYG{o}{=} \PYG{p}{\PYGZob{}}\PYG{l+s+s1}{\PYGZsq{}}\PYG{l+s+s1}{day\PYGZhy{}1}\PYG{l+s+s1}{\PYGZsq{}}\PYG{p}{:}\PYG{p}{\PYGZob{}}\PYG{l+s+s1}{\PYGZsq{}}\PYG{l+s+s1}{High}\PYG{l+s+s1}{\PYGZsq{}}\PYG{p}{:} \PYG{l+m+mi}{70}\PYG{p}{,} \PYG{l+s+s1}{\PYGZsq{}}\PYG{l+s+s1}{Low}\PYG{l+s+s1}{\PYGZsq{}}\PYG{p}{:}\PYG{l+m+mi}{62}\PYG{p}{,} \PYG{l+s+s1}{\PYGZsq{}}\PYG{l+s+s1}{Close}\PYG{l+s+s1}{\PYGZsq{}}\PYG{p}{:}\PYG{l+m+mi}{65}\PYG{p}{\PYGZcb{}}\PYG{p}{,}
              \PYG{l+s+s1}{\PYGZsq{}}\PYG{l+s+s1}{day\PYGZhy{}2}\PYG{l+s+s1}{\PYGZsq{}}\PYG{p}{:}\PYG{p}{\PYGZob{}}\PYG{l+s+s1}{\PYGZsq{}}\PYG{l+s+s1}{High}\PYG{l+s+s1}{\PYGZsq{}}\PYG{p}{:} \PYG{l+m+mi}{68}\PYG{p}{,} \PYG{l+s+s1}{\PYGZsq{}}\PYG{l+s+s1}{Low}\PYG{l+s+s1}{\PYGZsq{}}\PYG{p}{:}\PYG{l+m+mi}{60}\PYG{p}{,} \PYG{l+s+s1}{\PYGZsq{}}\PYG{l+s+s1}{Close}\PYG{l+s+s1}{\PYGZsq{}}\PYG{p}{:}\PYG{l+m+mi}{63}\PYG{p}{\PYGZcb{}}\PYG{p}{,}
              \PYG{l+s+s1}{\PYGZsq{}}\PYG{l+s+s1}{day\PYGZhy{}3}\PYG{l+s+s1}{\PYGZsq{}}\PYG{p}{:}\PYG{p}{\PYGZob{}}\PYG{l+s+s1}{\PYGZsq{}}\PYG{l+s+s1}{High}\PYG{l+s+s1}{\PYGZsq{}}\PYG{p}{:} \PYG{l+m+mi}{71}\PYG{p}{,} \PYG{l+s+s1}{\PYGZsq{}}\PYG{l+s+s1}{Low}\PYG{l+s+s1}{\PYGZsq{}}\PYG{p}{:}\PYG{l+m+mi}{65}\PYG{p}{,} \PYG{l+s+s1}{\PYGZsq{}}\PYG{l+s+s1}{Close}\PYG{l+s+s1}{\PYGZsq{}}\PYG{p}{:}\PYG{l+m+mi}{67}\PYG{p}{\PYGZcb{}}\PYG{p}{,}
              \PYG{l+s+s1}{\PYGZsq{}}\PYG{l+s+s1}{day\PYGZhy{}4}\PYG{l+s+s1}{\PYGZsq{}}\PYG{p}{:}\PYG{p}{\PYGZob{}}\PYG{l+s+s1}{\PYGZsq{}}\PYG{l+s+s1}{High}\PYG{l+s+s1}{\PYGZsq{}}\PYG{p}{:} \PYG{l+m+mi}{70}\PYG{p}{,} \PYG{l+s+s1}{\PYGZsq{}}\PYG{l+s+s1}{Low}\PYG{l+s+s1}{\PYGZsq{}}\PYG{p}{:}\PYG{l+m+mi}{62}\PYG{p}{,} \PYG{l+s+s1}{\PYGZsq{}}\PYG{l+s+s1}{Close}\PYG{l+s+s1}{\PYGZsq{}}\PYG{p}{:}\PYG{l+m+mi}{65}\PYG{p}{\PYGZcb{}}\PYG{p}{,}
              \PYG{l+s+s1}{\PYGZsq{}}\PYG{l+s+s1}{day\PYGZhy{}5}\PYG{l+s+s1}{\PYGZsq{}}\PYG{p}{:}\PYG{p}{\PYGZob{}}\PYG{l+s+s1}{\PYGZsq{}}\PYG{l+s+s1}{High}\PYG{l+s+s1}{\PYGZsq{}}\PYG{p}{:} \PYG{l+m+mi}{73}\PYG{p}{,} \PYG{l+s+s1}{\PYGZsq{}}\PYG{l+s+s1}{Low}\PYG{l+s+s1}{\PYGZsq{}}\PYG{p}{:}\PYG{l+m+mi}{65}\PYG{p}{,} \PYG{l+s+s1}{\PYGZsq{}}\PYG{l+s+s1}{Close}\PYG{l+s+s1}{\PYGZsq{}}\PYG{p}{:}\PYG{l+m+mi}{70}\PYG{p}{\PYGZcb{}}\PYG{p}{,}
              \PYG{l+s+s1}{\PYGZsq{}}\PYG{l+s+s1}{day\PYGZhy{}6}\PYG{l+s+s1}{\PYGZsq{}}\PYG{p}{:}\PYG{p}{\PYGZob{}}\PYG{l+s+s1}{\PYGZsq{}}\PYG{l+s+s1}{High}\PYG{l+s+s1}{\PYGZsq{}}\PYG{p}{:} \PYG{l+m+mi}{75}\PYG{p}{,} \PYG{l+s+s1}{\PYGZsq{}}\PYG{l+s+s1}{Low}\PYG{l+s+s1}{\PYGZsq{}}\PYG{p}{:}\PYG{l+m+mi}{69}\PYG{p}{,} \PYG{l+s+s1}{\PYGZsq{}}\PYG{l+s+s1}{Close}\PYG{l+s+s1}{\PYGZsq{}}\PYG{p}{:}\PYG{l+m+mi}{73}\PYG{p}{\PYGZcb{}}
              \PYG{p}{\PYGZcb{}}

\PYG{k}{for} \PYG{n}{i} \PYG{o+ow}{in} \PYG{p}{[}\PYG{l+m+mi}{1}\PYG{p}{,}\PYG{l+m+mi}{3}\PYG{p}{]}\PYG{p}{:}
  \PYG{n+nb}{print}\PYG{p}{(}\PYG{n}{stock\PYGZus{}dict}\PYG{p}{[}\PYG{l+s+s1}{\PYGZsq{}}\PYG{l+s+s1}{day\PYGZhy{}}\PYG{l+s+s1}{\PYGZsq{}}\PYG{o}{+}\PYG{n}{i}\PYG{p}{]}\PYG{p}{[}\PYG{l+s+s1}{\PYGZsq{}}\PYG{l+s+s1}{Low}\PYG{l+s+s1}{\PYGZsq{}}\PYG{p}{]}\PYG{p}{)}
\end{sphinxVerbatim}

\begin{sphinxadmonition}{note}{Solution}

\sphinxAtStartPar
The counter i takes integer values, so it cannot be directly concatenated with ‘day\sphinxhyphen{}’. To resolve this issue, use \sphinxcode{\sphinxupquote{'day\sphinxhyphen{}' + str(i)}} instead.
\end{sphinxadmonition}

\sphinxstepscope


\section{Dictionaries Output}
\label{\detokenize{dictionaries_output:dictionaries-output}}\label{\detokenize{dictionaries_output::doc}}\begin{itemize}
\item {} 
\sphinxAtStartPar
Find the output of the following code.

\item {} 
\sphinxAtStartPar
Please don’t run the code before giving your answer.     

\end{itemize}


\subsection{Question}
\label{\detokenize{dictionaries_output:question}}
\begin{sphinxuseclass}{cell}
\begin{sphinxuseclass}{tag_hide-output}\begin{sphinxVerbatimInput}

\begin{sphinxuseclass}{cell_input}
\begin{sphinxVerbatim}[commandchars=\\\{\}]
\PYG{n}{dict\PYGZus{}exp} \PYG{o}{=} \PYG{p}{\PYGZob{}}\PYG{l+s+s1}{\PYGZsq{}}\PYG{l+s+s1}{table}\PYG{l+s+s1}{\PYGZsq{}}\PYG{p}{:}\PYG{l+m+mi}{1}\PYG{p}{,} \PYG{l+s+s1}{\PYGZsq{}}\PYG{l+s+s1}{chair}\PYG{l+s+s1}{\PYGZsq{}}\PYG{p}{:}\PYG{l+m+mi}{2}\PYG{p}{,} \PYG{l+s+s1}{\PYGZsq{}}\PYG{l+s+s1}{washer}\PYG{l+s+s1}{\PYGZsq{}}\PYG{p}{:}\PYG{l+m+mi}{3}\PYG{p}{\PYGZcb{}}

\PYG{k}{for} \PYG{n}{i} \PYG{o+ow}{in} \PYG{n}{dict\PYGZus{}exp}\PYG{o}{.}\PYG{n}{keys}\PYG{p}{(}\PYG{p}{)}\PYG{p}{:}
  \PYG{n+nb}{print}\PYG{p}{(}\PYG{n}{i}\PYG{p}{[}\PYG{o}{\PYGZhy{}}\PYG{l+m+mi}{2}\PYG{p}{]}\PYG{p}{)}
\end{sphinxVerbatim}

\end{sphinxuseclass}\end{sphinxVerbatimInput}

\end{sphinxuseclass}
\end{sphinxuseclass}

\subsection{Question}
\label{\detokenize{dictionaries_output:id1}}
\begin{sphinxuseclass}{cell}
\begin{sphinxuseclass}{tag_hide-output}\begin{sphinxVerbatimInput}

\begin{sphinxuseclass}{cell_input}
\begin{sphinxVerbatim}[commandchars=\\\{\}]
\PYG{n}{dict\PYGZus{}exp} \PYG{o}{=} \PYG{p}{\PYGZob{}}\PYG{l+s+s1}{\PYGZsq{}}\PYG{l+s+s1}{table}\PYG{l+s+s1}{\PYGZsq{}}\PYG{p}{:}\PYG{p}{[}\PYG{l+m+mi}{1}\PYG{p}{,}\PYG{l+m+mi}{5}\PYG{p}{,}\PYG{l+m+mi}{9}\PYG{p}{]}\PYG{p}{,} \PYG{l+s+s1}{\PYGZsq{}}\PYG{l+s+s1}{chair}\PYG{l+s+s1}{\PYGZsq{}}\PYG{p}{:}\PYG{p}{[}\PYG{l+m+mi}{10}\PYG{p}{,}\PYG{l+m+mi}{20}\PYG{p}{,}\PYG{l+m+mi}{30}\PYG{p}{]}\PYG{p}{,} \PYG{l+s+s1}{\PYGZsq{}}\PYG{l+s+s1}{washer}\PYG{l+s+s1}{\PYGZsq{}}\PYG{p}{:}\PYG{p}{[}\PYG{l+m+mi}{2}\PYG{p}{,}\PYG{l+m+mi}{7}\PYG{p}{,}\PYG{l+m+mi}{4}\PYG{p}{]}\PYG{p}{\PYGZcb{}}
\PYG{n}{total} \PYG{o}{=} \PYG{l+m+mi}{0}

\PYG{k}{for} \PYG{n}{i} \PYG{o+ow}{in} \PYG{n}{dict\PYGZus{}exp}\PYG{o}{.}\PYG{n}{values}\PYG{p}{(}\PYG{p}{)}\PYG{p}{:}
  \PYG{n}{total} \PYG{o}{+}\PYG{o}{=} \PYG{n}{i}\PYG{p}{[}\PYG{l+m+mi}{1}\PYG{p}{]}
    
\PYG{n+nb}{print}\PYG{p}{(}\PYG{n}{total}\PYG{p}{)}
\end{sphinxVerbatim}

\end{sphinxuseclass}\end{sphinxVerbatimInput}

\end{sphinxuseclass}
\end{sphinxuseclass}

\subsection{Question}
\label{\detokenize{dictionaries_output:id2}}
\begin{sphinxuseclass}{cell}
\begin{sphinxuseclass}{tag_hide-output}\begin{sphinxVerbatimInput}

\begin{sphinxuseclass}{cell_input}
\begin{sphinxVerbatim}[commandchars=\\\{\}]
\PYG{n}{dict\PYGZus{}exp} \PYG{o}{=} \PYG{p}{\PYGZob{}}\PYG{l+s+s1}{\PYGZsq{}}\PYG{l+s+s1}{table}\PYG{l+s+s1}{\PYGZsq{}}\PYG{p}{:}\PYG{p}{[}\PYG{l+m+mi}{1}\PYG{p}{,}\PYG{l+m+mi}{5}\PYG{p}{,}\PYG{l+m+mi}{9}\PYG{p}{]}\PYG{p}{,} \PYG{l+s+s1}{\PYGZsq{}}\PYG{l+s+s1}{chair}\PYG{l+s+s1}{\PYGZsq{}}\PYG{p}{:}\PYG{p}{[}\PYG{l+m+mi}{4}\PYG{p}{,}\PYG{l+m+mi}{6}\PYG{p}{,}\PYG{l+m+mi}{8}\PYG{p}{]}\PYG{p}{,} \PYG{l+s+s1}{\PYGZsq{}}\PYG{l+s+s1}{washer}\PYG{l+s+s1}{\PYGZsq{}}\PYG{p}{:}\PYG{p}{[}\PYG{l+m+mi}{2}\PYG{p}{,}\PYG{l+m+mi}{7}\PYG{p}{,}\PYG{l+m+mi}{4}\PYG{p}{]}\PYG{p}{\PYGZcb{}}
\PYG{n}{total} \PYG{o}{=} \PYG{l+m+mi}{0}

\PYG{k}{for} \PYG{n}{i} \PYG{o+ow}{in} \PYG{n}{dict\PYGZus{}exp}\PYG{o}{.}\PYG{n}{values}\PYG{p}{(}\PYG{p}{)}\PYG{p}{:}
  \PYG{k}{for} \PYG{n}{j} \PYG{o+ow}{in} \PYG{n}{i}\PYG{p}{:}
    \PYG{n}{total} \PYG{o}{+}\PYG{o}{=} \PYG{n}{i}\PYG{p}{[}\PYG{l+m+mi}{1}\PYG{p}{]}
      
\PYG{n+nb}{print}\PYG{p}{(}\PYG{n}{total}\PYG{p}{)}
\end{sphinxVerbatim}

\end{sphinxuseclass}\end{sphinxVerbatimInput}

\end{sphinxuseclass}
\end{sphinxuseclass}

\subsection{Question}
\label{\detokenize{dictionaries_output:id3}}
\begin{sphinxuseclass}{cell}
\begin{sphinxuseclass}{tag_hide-output}\begin{sphinxVerbatimInput}

\begin{sphinxuseclass}{cell_input}
\begin{sphinxVerbatim}[commandchars=\\\{\}]
\PYG{n}{dict\PYGZus{}exp} \PYG{o}{=} \PYG{p}{\PYGZob{}}\PYG{l+s+s1}{\PYGZsq{}}\PYG{l+s+s1}{table}\PYG{l+s+s1}{\PYGZsq{}}\PYG{p}{:}\PYG{p}{[}\PYG{l+m+mi}{1}\PYG{p}{,}\PYG{l+m+mi}{5}\PYG{p}{,}\PYG{l+m+mi}{9}\PYG{p}{]}\PYG{p}{,} \PYG{l+s+s1}{\PYGZsq{}}\PYG{l+s+s1}{chair}\PYG{l+s+s1}{\PYGZsq{}}\PYG{p}{:}\PYG{p}{[}\PYG{l+m+mi}{4}\PYG{p}{,}\PYG{l+m+mi}{6}\PYG{p}{,}\PYG{l+m+mi}{8}\PYG{p}{]}\PYG{p}{,} \PYG{l+s+s1}{\PYGZsq{}}\PYG{l+s+s1}{washer}\PYG{l+s+s1}{\PYGZsq{}}\PYG{p}{:}\PYG{p}{[}\PYG{l+m+mi}{2}\PYG{p}{,}\PYG{l+m+mi}{7}\PYG{p}{,}\PYG{l+m+mi}{4}\PYG{p}{]}\PYG{p}{\PYGZcb{}}
\PYG{n}{total} \PYG{o}{=} \PYG{l+m+mi}{0}

\PYG{k}{for} \PYG{n}{i} \PYG{o+ow}{in} \PYG{n}{dict\PYGZus{}exp}\PYG{o}{.}\PYG{n}{values}\PYG{p}{(}\PYG{p}{)}\PYG{p}{:}
  \PYG{k}{for} \PYG{n}{j} \PYG{o+ow}{in} \PYG{n}{i}\PYG{p}{:}
    \PYG{n}{total} \PYG{o}{+}\PYG{o}{=} \PYG{n}{j}
      
\PYG{n+nb}{print}\PYG{p}{(}\PYG{n}{total}\PYG{p}{)}
\end{sphinxVerbatim}

\end{sphinxuseclass}\end{sphinxVerbatimInput}

\end{sphinxuseclass}
\end{sphinxuseclass}

\subsection{Question}
\label{\detokenize{dictionaries_output:id4}}
\begin{sphinxuseclass}{cell}
\begin{sphinxuseclass}{tag_hide-output}\begin{sphinxVerbatimInput}

\begin{sphinxuseclass}{cell_input}
\begin{sphinxVerbatim}[commandchars=\\\{\}]
\PYG{n}{stock\PYGZus{}dict} \PYG{o}{=} \PYG{p}{\PYGZob{}}\PYG{l+s+s1}{\PYGZsq{}}\PYG{l+s+s1}{day\PYGZhy{}1}\PYG{l+s+s1}{\PYGZsq{}}\PYG{p}{:}\PYG{p}{\PYGZob{}}\PYG{l+s+s1}{\PYGZsq{}}\PYG{l+s+s1}{High}\PYG{l+s+s1}{\PYGZsq{}}\PYG{p}{:} \PYG{l+m+mi}{70}\PYG{p}{,} \PYG{l+s+s1}{\PYGZsq{}}\PYG{l+s+s1}{Low}\PYG{l+s+s1}{\PYGZsq{}}\PYG{p}{:}\PYG{l+m+mi}{62}\PYG{p}{,} \PYG{l+s+s1}{\PYGZsq{}}\PYG{l+s+s1}{Close}\PYG{l+s+s1}{\PYGZsq{}}\PYG{p}{:}\PYG{l+m+mi}{65}\PYG{p}{\PYGZcb{}}\PYG{p}{,}
              \PYG{l+s+s1}{\PYGZsq{}}\PYG{l+s+s1}{day\PYGZhy{}2}\PYG{l+s+s1}{\PYGZsq{}}\PYG{p}{:}\PYG{p}{\PYGZob{}}\PYG{l+s+s1}{\PYGZsq{}}\PYG{l+s+s1}{High}\PYG{l+s+s1}{\PYGZsq{}}\PYG{p}{:} \PYG{l+m+mi}{68}\PYG{p}{,} \PYG{l+s+s1}{\PYGZsq{}}\PYG{l+s+s1}{Low}\PYG{l+s+s1}{\PYGZsq{}}\PYG{p}{:}\PYG{l+m+mi}{60}\PYG{p}{,} \PYG{l+s+s1}{\PYGZsq{}}\PYG{l+s+s1}{Close}\PYG{l+s+s1}{\PYGZsq{}}\PYG{p}{:}\PYG{l+m+mi}{63}\PYG{p}{\PYGZcb{}}\PYG{p}{,}
              \PYG{l+s+s1}{\PYGZsq{}}\PYG{l+s+s1}{day\PYGZhy{}3}\PYG{l+s+s1}{\PYGZsq{}}\PYG{p}{:}\PYG{p}{\PYGZob{}}\PYG{l+s+s1}{\PYGZsq{}}\PYG{l+s+s1}{High}\PYG{l+s+s1}{\PYGZsq{}}\PYG{p}{:} \PYG{l+m+mi}{71}\PYG{p}{,} \PYG{l+s+s1}{\PYGZsq{}}\PYG{l+s+s1}{Low}\PYG{l+s+s1}{\PYGZsq{}}\PYG{p}{:}\PYG{l+m+mi}{65}\PYG{p}{,} \PYG{l+s+s1}{\PYGZsq{}}\PYG{l+s+s1}{Close}\PYG{l+s+s1}{\PYGZsq{}}\PYG{p}{:}\PYG{l+m+mi}{67}\PYG{p}{\PYGZcb{}}\PYG{p}{,}
              \PYG{l+s+s1}{\PYGZsq{}}\PYG{l+s+s1}{day\PYGZhy{}4}\PYG{l+s+s1}{\PYGZsq{}}\PYG{p}{:}\PYG{p}{\PYGZob{}}\PYG{l+s+s1}{\PYGZsq{}}\PYG{l+s+s1}{High}\PYG{l+s+s1}{\PYGZsq{}}\PYG{p}{:} \PYG{l+m+mi}{70}\PYG{p}{,} \PYG{l+s+s1}{\PYGZsq{}}\PYG{l+s+s1}{Low}\PYG{l+s+s1}{\PYGZsq{}}\PYG{p}{:}\PYG{l+m+mi}{62}\PYG{p}{,} \PYG{l+s+s1}{\PYGZsq{}}\PYG{l+s+s1}{Close}\PYG{l+s+s1}{\PYGZsq{}}\PYG{p}{:}\PYG{l+m+mi}{65}\PYG{p}{\PYGZcb{}}\PYG{p}{,}
              \PYG{l+s+s1}{\PYGZsq{}}\PYG{l+s+s1}{day\PYGZhy{}5}\PYG{l+s+s1}{\PYGZsq{}}\PYG{p}{:}\PYG{p}{\PYGZob{}}\PYG{l+s+s1}{\PYGZsq{}}\PYG{l+s+s1}{High}\PYG{l+s+s1}{\PYGZsq{}}\PYG{p}{:} \PYG{l+m+mi}{73}\PYG{p}{,} \PYG{l+s+s1}{\PYGZsq{}}\PYG{l+s+s1}{Low}\PYG{l+s+s1}{\PYGZsq{}}\PYG{p}{:}\PYG{l+m+mi}{65}\PYG{p}{,} \PYG{l+s+s1}{\PYGZsq{}}\PYG{l+s+s1}{Close}\PYG{l+s+s1}{\PYGZsq{}}\PYG{p}{:}\PYG{l+m+mi}{70}\PYG{p}{\PYGZcb{}}\PYG{p}{,}
              \PYG{l+s+s1}{\PYGZsq{}}\PYG{l+s+s1}{day\PYGZhy{}6}\PYG{l+s+s1}{\PYGZsq{}}\PYG{p}{:}\PYG{p}{\PYGZob{}}\PYG{l+s+s1}{\PYGZsq{}}\PYG{l+s+s1}{High}\PYG{l+s+s1}{\PYGZsq{}}\PYG{p}{:} \PYG{l+m+mi}{75}\PYG{p}{,} \PYG{l+s+s1}{\PYGZsq{}}\PYG{l+s+s1}{Low}\PYG{l+s+s1}{\PYGZsq{}}\PYG{p}{:}\PYG{l+m+mi}{69}\PYG{p}{,} \PYG{l+s+s1}{\PYGZsq{}}\PYG{l+s+s1}{Close}\PYG{l+s+s1}{\PYGZsq{}}\PYG{p}{:}\PYG{l+m+mi}{73}\PYG{p}{\PYGZcb{}}
              \PYG{p}{\PYGZcb{}}

\PYG{k}{for} \PYG{n}{i} \PYG{o+ow}{in} \PYG{p}{[}\PYG{l+m+mi}{1}\PYG{p}{,}\PYG{l+m+mi}{3}\PYG{p}{]}\PYG{p}{:}
  \PYG{n+nb}{print}\PYG{p}{(}\PYG{n}{stock\PYGZus{}dict}\PYG{p}{[}\PYG{l+s+s1}{\PYGZsq{}}\PYG{l+s+s1}{day\PYGZhy{}}\PYG{l+s+s1}{\PYGZsq{}}\PYG{o}{+}\PYG{n+nb}{str}\PYG{p}{(}\PYG{n}{i}\PYG{p}{)}\PYG{p}{]}\PYG{p}{[}\PYG{l+s+s1}{\PYGZsq{}}\PYG{l+s+s1}{Low}\PYG{l+s+s1}{\PYGZsq{}}\PYG{p}{]}\PYG{p}{)}
\end{sphinxVerbatim}

\end{sphinxuseclass}\end{sphinxVerbatimInput}

\end{sphinxuseclass}
\end{sphinxuseclass}

\subsection{Question}
\label{\detokenize{dictionaries_output:id5}}
\begin{sphinxuseclass}{cell}
\begin{sphinxuseclass}{tag_hide-output}\begin{sphinxVerbatimInput}

\begin{sphinxuseclass}{cell_input}
\begin{sphinxVerbatim}[commandchars=\\\{\}]
\PYG{n}{stock\PYGZus{}dict} \PYG{o}{=} \PYG{p}{\PYGZob{}}\PYG{l+s+s1}{\PYGZsq{}}\PYG{l+s+s1}{day\PYGZhy{}1}\PYG{l+s+s1}{\PYGZsq{}}\PYG{p}{:}\PYG{p}{\PYGZob{}}\PYG{l+s+s1}{\PYGZsq{}}\PYG{l+s+s1}{High}\PYG{l+s+s1}{\PYGZsq{}}\PYG{p}{:} \PYG{l+m+mi}{70}\PYG{p}{,} \PYG{l+s+s1}{\PYGZsq{}}\PYG{l+s+s1}{Low}\PYG{l+s+s1}{\PYGZsq{}}\PYG{p}{:}\PYG{l+m+mi}{62}\PYG{p}{,} \PYG{l+s+s1}{\PYGZsq{}}\PYG{l+s+s1}{Close}\PYG{l+s+s1}{\PYGZsq{}}\PYG{p}{:}\PYG{l+m+mi}{65}\PYG{p}{\PYGZcb{}}\PYG{p}{,}
              \PYG{l+s+s1}{\PYGZsq{}}\PYG{l+s+s1}{day\PYGZhy{}2}\PYG{l+s+s1}{\PYGZsq{}}\PYG{p}{:}\PYG{p}{\PYGZob{}}\PYG{l+s+s1}{\PYGZsq{}}\PYG{l+s+s1}{High}\PYG{l+s+s1}{\PYGZsq{}}\PYG{p}{:} \PYG{l+m+mi}{68}\PYG{p}{,} \PYG{l+s+s1}{\PYGZsq{}}\PYG{l+s+s1}{Low}\PYG{l+s+s1}{\PYGZsq{}}\PYG{p}{:}\PYG{l+m+mi}{60}\PYG{p}{,} \PYG{l+s+s1}{\PYGZsq{}}\PYG{l+s+s1}{Close}\PYG{l+s+s1}{\PYGZsq{}}\PYG{p}{:}\PYG{l+m+mi}{63}\PYG{p}{\PYGZcb{}}\PYG{p}{,}
              \PYG{l+s+s1}{\PYGZsq{}}\PYG{l+s+s1}{day\PYGZhy{}3}\PYG{l+s+s1}{\PYGZsq{}}\PYG{p}{:}\PYG{p}{\PYGZob{}}\PYG{l+s+s1}{\PYGZsq{}}\PYG{l+s+s1}{High}\PYG{l+s+s1}{\PYGZsq{}}\PYG{p}{:} \PYG{l+m+mi}{71}\PYG{p}{,} \PYG{l+s+s1}{\PYGZsq{}}\PYG{l+s+s1}{Low}\PYG{l+s+s1}{\PYGZsq{}}\PYG{p}{:}\PYG{l+m+mi}{65}\PYG{p}{,} \PYG{l+s+s1}{\PYGZsq{}}\PYG{l+s+s1}{Close}\PYG{l+s+s1}{\PYGZsq{}}\PYG{p}{:}\PYG{l+m+mi}{67}\PYG{p}{\PYGZcb{}}\PYG{p}{,}
              \PYG{l+s+s1}{\PYGZsq{}}\PYG{l+s+s1}{day\PYGZhy{}4}\PYG{l+s+s1}{\PYGZsq{}}\PYG{p}{:}\PYG{p}{\PYGZob{}}\PYG{l+s+s1}{\PYGZsq{}}\PYG{l+s+s1}{High}\PYG{l+s+s1}{\PYGZsq{}}\PYG{p}{:} \PYG{l+m+mi}{70}\PYG{p}{,} \PYG{l+s+s1}{\PYGZsq{}}\PYG{l+s+s1}{Low}\PYG{l+s+s1}{\PYGZsq{}}\PYG{p}{:}\PYG{l+m+mi}{62}\PYG{p}{,} \PYG{l+s+s1}{\PYGZsq{}}\PYG{l+s+s1}{Close}\PYG{l+s+s1}{\PYGZsq{}}\PYG{p}{:}\PYG{l+m+mi}{65}\PYG{p}{\PYGZcb{}}\PYG{p}{,}
              \PYG{l+s+s1}{\PYGZsq{}}\PYG{l+s+s1}{day\PYGZhy{}5}\PYG{l+s+s1}{\PYGZsq{}}\PYG{p}{:}\PYG{p}{\PYGZob{}}\PYG{l+s+s1}{\PYGZsq{}}\PYG{l+s+s1}{High}\PYG{l+s+s1}{\PYGZsq{}}\PYG{p}{:} \PYG{l+m+mi}{73}\PYG{p}{,} \PYG{l+s+s1}{\PYGZsq{}}\PYG{l+s+s1}{Low}\PYG{l+s+s1}{\PYGZsq{}}\PYG{p}{:}\PYG{l+m+mi}{65}\PYG{p}{,} \PYG{l+s+s1}{\PYGZsq{}}\PYG{l+s+s1}{Close}\PYG{l+s+s1}{\PYGZsq{}}\PYG{p}{:}\PYG{l+m+mi}{70}\PYG{p}{\PYGZcb{}}\PYG{p}{,}
              \PYG{l+s+s1}{\PYGZsq{}}\PYG{l+s+s1}{day\PYGZhy{}6}\PYG{l+s+s1}{\PYGZsq{}}\PYG{p}{:}\PYG{p}{\PYGZob{}}\PYG{l+s+s1}{\PYGZsq{}}\PYG{l+s+s1}{High}\PYG{l+s+s1}{\PYGZsq{}}\PYG{p}{:} \PYG{l+m+mi}{75}\PYG{p}{,} \PYG{l+s+s1}{\PYGZsq{}}\PYG{l+s+s1}{Low}\PYG{l+s+s1}{\PYGZsq{}}\PYG{p}{:}\PYG{l+m+mi}{69}\PYG{p}{,} \PYG{l+s+s1}{\PYGZsq{}}\PYG{l+s+s1}{Close}\PYG{l+s+s1}{\PYGZsq{}}\PYG{p}{:}\PYG{l+m+mi}{73}\PYG{p}{\PYGZcb{}}
              \PYG{p}{\PYGZcb{}}
\PYG{n}{total} \PYG{o}{=} \PYG{l+m+mi}{0}

\PYG{k}{for} \PYG{n}{i}\PYG{p}{,}\PYG{n}{j} \PYG{o+ow}{in} \PYG{n}{stock\PYGZus{}dict}\PYG{o}{.}\PYG{n}{items}\PYG{p}{(}\PYG{p}{)}\PYG{p}{:}
  \PYG{n}{total} \PYG{o}{+}\PYG{o}{=} \PYG{n+nb}{int}\PYG{p}{(}\PYG{n}{i}\PYG{p}{[}\PYG{o}{\PYGZhy{}}\PYG{l+m+mi}{1}\PYG{p}{]}\PYG{p}{)}

\PYG{n+nb}{print}\PYG{p}{(}\PYG{n}{total}\PYG{p}{)}
\end{sphinxVerbatim}

\end{sphinxuseclass}\end{sphinxVerbatimInput}

\end{sphinxuseclass}
\end{sphinxuseclass}

\subsection{Question}
\label{\detokenize{dictionaries_output:id6}}
\begin{sphinxuseclass}{cell}
\begin{sphinxuseclass}{tag_hide-output}\begin{sphinxVerbatimInput}

\begin{sphinxuseclass}{cell_input}
\begin{sphinxVerbatim}[commandchars=\\\{\}]
\PYG{n}{stock\PYGZus{}dict} \PYG{o}{=} \PYG{p}{\PYGZob{}}\PYG{l+s+s1}{\PYGZsq{}}\PYG{l+s+s1}{day\PYGZhy{}1}\PYG{l+s+s1}{\PYGZsq{}}\PYG{p}{:}\PYG{p}{\PYGZob{}}\PYG{l+s+s1}{\PYGZsq{}}\PYG{l+s+s1}{High}\PYG{l+s+s1}{\PYGZsq{}}\PYG{p}{:} \PYG{l+m+mi}{70}\PYG{p}{,} \PYG{l+s+s1}{\PYGZsq{}}\PYG{l+s+s1}{Low}\PYG{l+s+s1}{\PYGZsq{}}\PYG{p}{:}\PYG{l+m+mi}{62}\PYG{p}{,} \PYG{l+s+s1}{\PYGZsq{}}\PYG{l+s+s1}{Close}\PYG{l+s+s1}{\PYGZsq{}}\PYG{p}{:}\PYG{l+m+mi}{65}\PYG{p}{\PYGZcb{}}\PYG{p}{,}
              \PYG{l+s+s1}{\PYGZsq{}}\PYG{l+s+s1}{day\PYGZhy{}2}\PYG{l+s+s1}{\PYGZsq{}}\PYG{p}{:}\PYG{p}{\PYGZob{}}\PYG{l+s+s1}{\PYGZsq{}}\PYG{l+s+s1}{High}\PYG{l+s+s1}{\PYGZsq{}}\PYG{p}{:} \PYG{l+m+mi}{68}\PYG{p}{,} \PYG{l+s+s1}{\PYGZsq{}}\PYG{l+s+s1}{Low}\PYG{l+s+s1}{\PYGZsq{}}\PYG{p}{:}\PYG{l+m+mi}{60}\PYG{p}{,} \PYG{l+s+s1}{\PYGZsq{}}\PYG{l+s+s1}{Close}\PYG{l+s+s1}{\PYGZsq{}}\PYG{p}{:}\PYG{l+m+mi}{63}\PYG{p}{\PYGZcb{}}\PYG{p}{,}
              \PYG{l+s+s1}{\PYGZsq{}}\PYG{l+s+s1}{day\PYGZhy{}3}\PYG{l+s+s1}{\PYGZsq{}}\PYG{p}{:}\PYG{p}{\PYGZob{}}\PYG{l+s+s1}{\PYGZsq{}}\PYG{l+s+s1}{High}\PYG{l+s+s1}{\PYGZsq{}}\PYG{p}{:} \PYG{l+m+mi}{71}\PYG{p}{,} \PYG{l+s+s1}{\PYGZsq{}}\PYG{l+s+s1}{Low}\PYG{l+s+s1}{\PYGZsq{}}\PYG{p}{:}\PYG{l+m+mi}{65}\PYG{p}{,} \PYG{l+s+s1}{\PYGZsq{}}\PYG{l+s+s1}{Close}\PYG{l+s+s1}{\PYGZsq{}}\PYG{p}{:}\PYG{l+m+mi}{67}\PYG{p}{\PYGZcb{}}\PYG{p}{,}
              \PYG{l+s+s1}{\PYGZsq{}}\PYG{l+s+s1}{day\PYGZhy{}4}\PYG{l+s+s1}{\PYGZsq{}}\PYG{p}{:}\PYG{p}{\PYGZob{}}\PYG{l+s+s1}{\PYGZsq{}}\PYG{l+s+s1}{High}\PYG{l+s+s1}{\PYGZsq{}}\PYG{p}{:} \PYG{l+m+mi}{70}\PYG{p}{,} \PYG{l+s+s1}{\PYGZsq{}}\PYG{l+s+s1}{Low}\PYG{l+s+s1}{\PYGZsq{}}\PYG{p}{:}\PYG{l+m+mi}{62}\PYG{p}{,} \PYG{l+s+s1}{\PYGZsq{}}\PYG{l+s+s1}{Close}\PYG{l+s+s1}{\PYGZsq{}}\PYG{p}{:}\PYG{l+m+mi}{65}\PYG{p}{\PYGZcb{}}\PYG{p}{,}
              \PYG{l+s+s1}{\PYGZsq{}}\PYG{l+s+s1}{day\PYGZhy{}5}\PYG{l+s+s1}{\PYGZsq{}}\PYG{p}{:}\PYG{p}{\PYGZob{}}\PYG{l+s+s1}{\PYGZsq{}}\PYG{l+s+s1}{High}\PYG{l+s+s1}{\PYGZsq{}}\PYG{p}{:} \PYG{l+m+mi}{73}\PYG{p}{,} \PYG{l+s+s1}{\PYGZsq{}}\PYG{l+s+s1}{Low}\PYG{l+s+s1}{\PYGZsq{}}\PYG{p}{:}\PYG{l+m+mi}{65}\PYG{p}{,} \PYG{l+s+s1}{\PYGZsq{}}\PYG{l+s+s1}{Close}\PYG{l+s+s1}{\PYGZsq{}}\PYG{p}{:}\PYG{l+m+mi}{70}\PYG{p}{\PYGZcb{}}\PYG{p}{,}
              \PYG{l+s+s1}{\PYGZsq{}}\PYG{l+s+s1}{day\PYGZhy{}6}\PYG{l+s+s1}{\PYGZsq{}}\PYG{p}{:}\PYG{p}{\PYGZob{}}\PYG{l+s+s1}{\PYGZsq{}}\PYG{l+s+s1}{High}\PYG{l+s+s1}{\PYGZsq{}}\PYG{p}{:} \PYG{l+m+mi}{75}\PYG{p}{,} \PYG{l+s+s1}{\PYGZsq{}}\PYG{l+s+s1}{Low}\PYG{l+s+s1}{\PYGZsq{}}\PYG{p}{:}\PYG{l+m+mi}{69}\PYG{p}{,} \PYG{l+s+s1}{\PYGZsq{}}\PYG{l+s+s1}{Close}\PYG{l+s+s1}{\PYGZsq{}}\PYG{p}{:}\PYG{l+m+mi}{73}\PYG{p}{\PYGZcb{}}
              \PYG{p}{\PYGZcb{}}
\PYG{n}{total} \PYG{o}{=} \PYG{l+m+mi}{0}

\PYG{k}{for} \PYG{n}{i}\PYG{p}{,}\PYG{n}{j} \PYG{o+ow}{in} \PYG{n}{stock\PYGZus{}dict}\PYG{o}{.}\PYG{n}{items}\PYG{p}{(}\PYG{p}{)}\PYG{p}{:}
  \PYG{n}{total} \PYG{o}{+}\PYG{o}{=} \PYG{n}{j}\PYG{p}{[}\PYG{l+s+s1}{\PYGZsq{}}\PYG{l+s+s1}{High}\PYG{l+s+s1}{\PYGZsq{}}\PYG{p}{]}\PYG{o}{/}\PYG{o}{/}\PYG{l+m+mi}{10}

\PYG{n+nb}{print}\PYG{p}{(}\PYG{n}{total}\PYG{p}{)}
\end{sphinxVerbatim}

\end{sphinxuseclass}\end{sphinxVerbatimInput}

\end{sphinxuseclass}
\end{sphinxuseclass}

\subsection{Question}
\label{\detokenize{dictionaries_output:id7}}
\begin{sphinxuseclass}{cell}
\begin{sphinxuseclass}{tag_hide-output}\begin{sphinxVerbatimInput}

\begin{sphinxuseclass}{cell_input}
\begin{sphinxVerbatim}[commandchars=\\\{\}]
\PYG{n}{my\PYGZus{}dict} \PYG{o}{=} \PYG{p}{\PYGZob{}}\PYG{l+m+mi}{200}\PYG{p}{:}\PYG{l+s+s1}{\PYGZsq{}}\PYG{l+s+s1}{NY}\PYG{l+s+s1}{\PYGZsq{}}\PYG{p}{,} \PYG{l+m+mi}{100}\PYG{p}{:}\PYG{l+s+s1}{\PYGZsq{}}\PYG{l+s+s1}{NJ}\PYG{l+s+s1}{\PYGZsq{}}\PYG{p}{,} \PYG{l+m+mi}{400}\PYG{p}{:}\PYG{l+s+s1}{\PYGZsq{}}\PYG{l+s+s1}{TX}\PYG{l+s+s1}{\PYGZsq{}}\PYG{p}{,} \PYG{l+m+mi}{300}\PYG{p}{:}\PYG{l+s+s1}{\PYGZsq{}}\PYG{l+s+s1}{CA}\PYG{l+s+s1}{\PYGZsq{}}\PYG{p}{\PYGZcb{}}

\PYG{k}{for} \PYG{n}{i} \PYG{o+ow}{in} \PYG{n}{my\PYGZus{}dict}\PYG{o}{.}\PYG{n}{keys}\PYG{p}{(}\PYG{p}{)}\PYG{p}{:}
  \PYG{k}{if} \PYG{n}{i}\PYG{o}{\PYGZgt{}}\PYG{l+m+mi}{188}\PYG{p}{:}
    \PYG{n+nb}{print}\PYG{p}{(}\PYG{n}{my\PYGZus{}dict}\PYG{p}{[}\PYG{n}{i}\PYG{p}{]}\PYG{p}{[}\PYG{l+m+mi}{1}\PYG{p}{]}\PYG{p}{)}
\end{sphinxVerbatim}

\end{sphinxuseclass}\end{sphinxVerbatimInput}

\end{sphinxuseclass}
\end{sphinxuseclass}

\subsection{Question}
\label{\detokenize{dictionaries_output:id8}}
\begin{sphinxuseclass}{cell}
\begin{sphinxuseclass}{tag_hide-output}\begin{sphinxVerbatimInput}

\begin{sphinxuseclass}{cell_input}
\begin{sphinxVerbatim}[commandchars=\\\{\}]
\PYG{n}{mydict} \PYG{o}{=} \PYG{p}{\PYGZob{}}\PYG{l+s+s1}{\PYGZsq{}}\PYG{l+s+s1}{A}\PYG{l+s+s1}{\PYGZsq{}}\PYG{p}{:}\PYG{p}{(}\PYG{l+m+mi}{10}\PYG{p}{,}\PYG{l+m+mi}{100}\PYG{p}{)}\PYG{p}{,} \PYG{l+s+s1}{\PYGZsq{}}\PYG{l+s+s1}{B}\PYG{l+s+s1}{\PYGZsq{}}\PYG{p}{:}\PYG{p}{(}\PYG{l+m+mi}{20}\PYG{p}{,} \PYG{l+m+mi}{200}\PYG{p}{)}\PYG{p}{,} \PYG{l+s+s1}{\PYGZsq{}}\PYG{l+s+s1}{C}\PYG{l+s+s1}{\PYGZsq{}}\PYG{p}{:}\PYG{p}{(}\PYG{l+m+mi}{30}\PYG{p}{,}\PYG{l+m+mi}{300}\PYG{p}{)}\PYG{p}{\PYGZcb{}}

\PYG{k}{for} \PYG{n}{i}\PYG{p}{,}\PYG{n}{j} \PYG{o+ow}{in} \PYG{n}{mydict}\PYG{o}{.}\PYG{n}{values}\PYG{p}{(}\PYG{p}{)}\PYG{p}{:}
  \PYG{n+nb}{print}\PYG{p}{(}\PYG{n}{i}\PYG{p}{,}\PYG{n}{j}\PYG{p}{)}
\end{sphinxVerbatim}

\end{sphinxuseclass}\end{sphinxVerbatimInput}

\end{sphinxuseclass}
\end{sphinxuseclass}
\sphinxstepscope


\section{Dictionaries Code}
\label{\detokenize{dictionaries_code:dictionaries-code}}\label{\detokenize{dictionaries_code::doc}}\begin{itemize}
\item {} 
\sphinxAtStartPar
Please solve the following questions using Python code.  

\end{itemize}


\subsection{Question}
\label{\detokenize{dictionaries_code:question}}
\sphinxAtStartPar
Write a program that constructs a dictionary using two lists provided below, where the keys are taken from loc\_list and the values are taken from player\_list.
\begin{itemize}
\item {} 
\sphinxAtStartPar
Each pair in the dictionary corresponds to a key\sphinxhyphen{}value pair with elements from the same index in the lists.

\item {} 
\sphinxAtStartPar
Use a for loop for this construction.

\end{itemize}

\begin{sphinxuseclass}{cell}\begin{sphinxVerbatimInput}

\begin{sphinxuseclass}{cell_input}
\begin{sphinxVerbatim}[commandchars=\\\{\}]
\PYG{n}{loc\PYGZus{}list} \PYG{o}{=} \PYG{p}{[}\PYG{l+s+s1}{\PYGZsq{}}\PYG{l+s+s1}{Boston}\PYG{l+s+s1}{\PYGZsq{}}\PYG{p}{,} \PYG{l+s+s1}{\PYGZsq{}}\PYG{l+s+s1}{Portland}\PYG{l+s+s1}{\PYGZsq{}}\PYG{p}{,} \PYG{l+s+s1}{\PYGZsq{}}\PYG{l+s+s1}{Brooklyn}\PYG{l+s+s1}{\PYGZsq{}}\PYG{p}{,} \PYG{l+s+s1}{\PYGZsq{}}\PYG{l+s+s1}{Dallas}\PYG{l+s+s1}{\PYGZsq{}}\PYG{p}{]}
\PYG{n}{player\PYGZus{}list} \PYG{o}{=} \PYG{p}{[}\PYG{l+s+s1}{\PYGZsq{}}\PYG{l+s+s1}{Tatum}\PYG{l+s+s1}{\PYGZsq{}}\PYG{p}{,} \PYG{l+s+s1}{\PYGZsq{}}\PYG{l+s+s1}{Lilliard}\PYG{l+s+s1}{\PYGZsq{}}\PYG{p}{,} \PYG{l+s+s1}{\PYGZsq{}}\PYG{l+s+s1}{Durant}\PYG{l+s+s1}{\PYGZsq{}}\PYG{p}{,} \PYG{l+s+s1}{\PYGZsq{}}\PYG{l+s+s1}{Doncic}\PYG{l+s+s1}{\PYGZsq{}}\PYG{p}{]}
\end{sphinxVerbatim}

\end{sphinxuseclass}\end{sphinxVerbatimInput}

\end{sphinxuseclass}
\sphinxAtStartPar
\sphinxstylestrong{Solution}


\subsection{Question}
\label{\detokenize{dictionaries_code:id1}}
\sphinxAtStartPar
Write a program that constructs a dictionary using the given list below, where the keys are taken from state\_list and the values represent the number of characters in each state name.
\begin{itemize}
\item {} 
\sphinxAtStartPar
Each pair in the dictionary corresponds to a state name and the number of characters in its name.

\item {} 
\sphinxAtStartPar
Example: \{‘California’, 10\}

\end{itemize}

\begin{sphinxuseclass}{cell}\begin{sphinxVerbatimInput}

\begin{sphinxuseclass}{cell_input}
\begin{sphinxVerbatim}[commandchars=\\\{\}]
\PYG{n}{state\PYGZus{}list} \PYG{o}{=} \PYG{p}{[}\PYG{l+s+s1}{\PYGZsq{}}\PYG{l+s+s1}{Utah}\PYG{l+s+s1}{\PYGZsq{}}\PYG{p}{,} \PYG{l+s+s1}{\PYGZsq{}}\PYG{l+s+s1}{Nevada}\PYG{l+s+s1}{\PYGZsq{}}\PYG{p}{,} \PYG{l+s+s1}{\PYGZsq{}}\PYG{l+s+s1}{Florida}\PYG{l+s+s1}{\PYGZsq{}}\PYG{p}{,} \PYG{l+s+s1}{\PYGZsq{}}\PYG{l+s+s1}{Texas}\PYG{l+s+s1}{\PYGZsq{}}\PYG{p}{,} \PYG{l+s+s1}{\PYGZsq{}}\PYG{l+s+s1}{Oklahoma}\PYG{l+s+s1}{\PYGZsq{}}\PYG{p}{,} \PYG{l+s+s1}{\PYGZsq{}}\PYG{l+s+s1}{Washington}\PYG{l+s+s1}{\PYGZsq{}}\PYG{p}{,} \PYG{l+s+s1}{\PYGZsq{}}\PYG{l+s+s1}{Colarado}\PYG{l+s+s1}{\PYGZsq{}}\PYG{p}{]}
\end{sphinxVerbatim}

\end{sphinxuseclass}\end{sphinxVerbatimInput}

\end{sphinxuseclass}
\sphinxAtStartPar
\sphinxstylestrong{Solution\sphinxhyphen{}1}

\begin{sphinxuseclass}{cell}\begin{sphinxVerbatimInput}

\begin{sphinxuseclass}{cell_input}
\begin{sphinxVerbatim}[commandchars=\\\{\}]
\PYG{n+nb}{print}\PYG{p}{(}\PYG{n}{state\PYGZus{}dict}\PYG{p}{)}
\end{sphinxVerbatim}

\end{sphinxuseclass}\end{sphinxVerbatimInput}
\begin{sphinxVerbatimOutput}

\begin{sphinxuseclass}{cell_output}
\begin{sphinxVerbatim}[commandchars=\\\{\}]
\PYGZob{}\PYGZsq{}Utah\PYGZsq{}: 4, \PYGZsq{}Nevada\PYGZsq{}: 6, \PYGZsq{}Florida\PYGZsq{}: 7, \PYGZsq{}Texas\PYGZsq{}: 5, \PYGZsq{}Oklahoma\PYGZsq{}: 8, \PYGZsq{}Washington\PYGZsq{}: 10, \PYGZsq{}Colarado\PYGZsq{}: 8\PYGZcb{}
\end{sphinxVerbatim}

\end{sphinxuseclass}\end{sphinxVerbatimOutput}

\end{sphinxuseclass}
\sphinxAtStartPar
\sphinxstylestrong{Solution\sphinxhyphen{}2}

\begin{sphinxuseclass}{cell}\begin{sphinxVerbatimInput}

\begin{sphinxuseclass}{cell_input}
\begin{sphinxVerbatim}[commandchars=\\\{\}]
\PYG{n+nb}{print}\PYG{p}{(}\PYG{n}{state\PYGZus{}dict}\PYG{p}{)}
\end{sphinxVerbatim}

\end{sphinxuseclass}\end{sphinxVerbatimInput}
\begin{sphinxVerbatimOutput}

\begin{sphinxuseclass}{cell_output}
\begin{sphinxVerbatim}[commandchars=\\\{\}]
\PYGZob{}\PYGZsq{}Utah\PYGZsq{}: 4, \PYGZsq{}Nevada\PYGZsq{}: 6, \PYGZsq{}Florida\PYGZsq{}: 7, \PYGZsq{}Texas\PYGZsq{}: 5, \PYGZsq{}Oklahoma\PYGZsq{}: 8, \PYGZsq{}Washington\PYGZsq{}: 10, \PYGZsq{}Colarado\PYGZsq{}: 8\PYGZcb{}
\end{sphinxVerbatim}

\end{sphinxuseclass}\end{sphinxVerbatimOutput}

\end{sphinxuseclass}

\subsection{Question}
\label{\detokenize{dictionaries_code:id2}}
\sphinxAtStartPar
For the given dictionary below, display ‘Close’ values for each day.

\begin{sphinxuseclass}{cell}\begin{sphinxVerbatimInput}

\begin{sphinxuseclass}{cell_input}
\begin{sphinxVerbatim}[commandchars=\\\{\}]
\PYG{n}{stock\PYGZus{}dict} \PYG{o}{=} \PYG{p}{\PYGZob{}}\PYG{l+s+s1}{\PYGZsq{}}\PYG{l+s+s1}{day\PYGZhy{}1}\PYG{l+s+s1}{\PYGZsq{}}\PYG{p}{:}\PYG{p}{\PYGZob{}}\PYG{l+s+s1}{\PYGZsq{}}\PYG{l+s+s1}{High}\PYG{l+s+s1}{\PYGZsq{}}\PYG{p}{:} \PYG{l+m+mi}{70}\PYG{p}{,} \PYG{l+s+s1}{\PYGZsq{}}\PYG{l+s+s1}{Low}\PYG{l+s+s1}{\PYGZsq{}}\PYG{p}{:}\PYG{l+m+mi}{62}\PYG{p}{,} \PYG{l+s+s1}{\PYGZsq{}}\PYG{l+s+s1}{Close}\PYG{l+s+s1}{\PYGZsq{}}\PYG{p}{:}\PYG{l+m+mi}{65}\PYG{p}{\PYGZcb{}}\PYG{p}{,}
              \PYG{l+s+s1}{\PYGZsq{}}\PYG{l+s+s1}{day\PYGZhy{}2}\PYG{l+s+s1}{\PYGZsq{}}\PYG{p}{:}\PYG{p}{\PYGZob{}}\PYG{l+s+s1}{\PYGZsq{}}\PYG{l+s+s1}{High}\PYG{l+s+s1}{\PYGZsq{}}\PYG{p}{:} \PYG{l+m+mi}{68}\PYG{p}{,} \PYG{l+s+s1}{\PYGZsq{}}\PYG{l+s+s1}{Low}\PYG{l+s+s1}{\PYGZsq{}}\PYG{p}{:}\PYG{l+m+mi}{60}\PYG{p}{,} \PYG{l+s+s1}{\PYGZsq{}}\PYG{l+s+s1}{Close}\PYG{l+s+s1}{\PYGZsq{}}\PYG{p}{:}\PYG{l+m+mi}{63}\PYG{p}{\PYGZcb{}}\PYG{p}{,}
              \PYG{l+s+s1}{\PYGZsq{}}\PYG{l+s+s1}{day\PYGZhy{}3}\PYG{l+s+s1}{\PYGZsq{}}\PYG{p}{:}\PYG{p}{\PYGZob{}}\PYG{l+s+s1}{\PYGZsq{}}\PYG{l+s+s1}{High}\PYG{l+s+s1}{\PYGZsq{}}\PYG{p}{:} \PYG{l+m+mi}{71}\PYG{p}{,} \PYG{l+s+s1}{\PYGZsq{}}\PYG{l+s+s1}{Low}\PYG{l+s+s1}{\PYGZsq{}}\PYG{p}{:}\PYG{l+m+mi}{65}\PYG{p}{,} \PYG{l+s+s1}{\PYGZsq{}}\PYG{l+s+s1}{Close}\PYG{l+s+s1}{\PYGZsq{}}\PYG{p}{:}\PYG{l+m+mi}{67}\PYG{p}{\PYGZcb{}}\PYG{p}{,}
              \PYG{l+s+s1}{\PYGZsq{}}\PYG{l+s+s1}{day\PYGZhy{}4}\PYG{l+s+s1}{\PYGZsq{}}\PYG{p}{:}\PYG{p}{\PYGZob{}}\PYG{l+s+s1}{\PYGZsq{}}\PYG{l+s+s1}{High}\PYG{l+s+s1}{\PYGZsq{}}\PYG{p}{:} \PYG{l+m+mi}{70}\PYG{p}{,} \PYG{l+s+s1}{\PYGZsq{}}\PYG{l+s+s1}{Low}\PYG{l+s+s1}{\PYGZsq{}}\PYG{p}{:}\PYG{l+m+mi}{62}\PYG{p}{,} \PYG{l+s+s1}{\PYGZsq{}}\PYG{l+s+s1}{Close}\PYG{l+s+s1}{\PYGZsq{}}\PYG{p}{:}\PYG{l+m+mi}{65}\PYG{p}{\PYGZcb{}}\PYG{p}{,}
              \PYG{l+s+s1}{\PYGZsq{}}\PYG{l+s+s1}{day\PYGZhy{}5}\PYG{l+s+s1}{\PYGZsq{}}\PYG{p}{:}\PYG{p}{\PYGZob{}}\PYG{l+s+s1}{\PYGZsq{}}\PYG{l+s+s1}{High}\PYG{l+s+s1}{\PYGZsq{}}\PYG{p}{:} \PYG{l+m+mi}{73}\PYG{p}{,} \PYG{l+s+s1}{\PYGZsq{}}\PYG{l+s+s1}{Low}\PYG{l+s+s1}{\PYGZsq{}}\PYG{p}{:}\PYG{l+m+mi}{65}\PYG{p}{,} \PYG{l+s+s1}{\PYGZsq{}}\PYG{l+s+s1}{Close}\PYG{l+s+s1}{\PYGZsq{}}\PYG{p}{:}\PYG{l+m+mi}{70}\PYG{p}{\PYGZcb{}}\PYG{p}{,}
              \PYG{l+s+s1}{\PYGZsq{}}\PYG{l+s+s1}{day\PYGZhy{}6}\PYG{l+s+s1}{\PYGZsq{}}\PYG{p}{:}\PYG{p}{\PYGZob{}}\PYG{l+s+s1}{\PYGZsq{}}\PYG{l+s+s1}{High}\PYG{l+s+s1}{\PYGZsq{}}\PYG{p}{:} \PYG{l+m+mi}{75}\PYG{p}{,} \PYG{l+s+s1}{\PYGZsq{}}\PYG{l+s+s1}{Low}\PYG{l+s+s1}{\PYGZsq{}}\PYG{p}{:}\PYG{l+m+mi}{69}\PYG{p}{,} \PYG{l+s+s1}{\PYGZsq{}}\PYG{l+s+s1}{Close}\PYG{l+s+s1}{\PYGZsq{}}\PYG{p}{:}\PYG{l+m+mi}{73}\PYG{p}{\PYGZcb{}}
              \PYG{p}{\PYGZcb{}}
\end{sphinxVerbatim}

\end{sphinxuseclass}\end{sphinxVerbatimInput}

\end{sphinxuseclass}
\sphinxAtStartPar
\sphinxstylestrong{Solution}


\subsection{Question}
\label{\detokenize{dictionaries_code:id3}}
\sphinxAtStartPar
Use the \sphinxstyleemphasis{stock\_dict} to create a list consisting of the ‘Close’ values.”

\sphinxAtStartPar
\sphinxstylestrong{Solution}


\subsection{Question}
\label{\detokenize{dictionaries_code:id4}}
\sphinxAtStartPar
Use \sphinxstyleemphasis{stock\_dict} to construct a dictionary where the keys are days (Day\sphinxhyphen{}1, Day\sphinxhyphen{}2, …) and the values represent the difference between ‘High’ and ‘Low’ values.

\sphinxAtStartPar
\sphinxstylestrong{Solution}


\subsection{Question}
\label{\detokenize{dictionaries_code:id5}}
\sphinxAtStartPar
Use the \sphinxstyleemphasis{stock\_dict} to construct a list consisting of values ‘Increasing’ or ‘Decreasing’ based on the difference between two consecutive day’s ‘Close’ values.
\begin{itemize}
\item {} 
\sphinxAtStartPar
Note that there is no ‘Increasing’ or ‘Decreasing’ value for Day\sphinxhyphen{}1, as the value before Day\sphinxhyphen{}1 is not provided.

\end{itemize}

\sphinxAtStartPar
\sphinxstylestrong{Solution}


\subsection{Question}
\label{\detokenize{dictionaries_code:id6}}
\sphinxAtStartPar
Construct a dictionary using a for loop, where the keys are the names in names\_list and the values are the last characters of the corresponding names.
\begin{itemize}
\item {} 
\sphinxAtStartPar
Example: key = ashley, value = y

\end{itemize}

\begin{sphinxuseclass}{cell}\begin{sphinxVerbatimInput}

\begin{sphinxuseclass}{cell_input}
\begin{sphinxVerbatim}[commandchars=\\\{\}]
\PYG{n}{names\PYGZus{}list} \PYG{o}{=} \PYG{p}{[}\PYG{l+s+s1}{\PYGZsq{}}\PYG{l+s+s1}{ashley}\PYG{l+s+s1}{\PYGZsq{}}\PYG{p}{,} \PYG{l+s+s1}{\PYGZsq{}}\PYG{l+s+s1}{michael}\PYG{l+s+s1}{\PYGZsq{}}\PYG{p}{,} \PYG{l+s+s1}{\PYGZsq{}}\PYG{l+s+s1}{jack}\PYG{l+s+s1}{\PYGZsq{}}\PYG{p}{,} \PYG{l+s+s1}{\PYGZsq{}}\PYG{l+s+s1}{taylor}\PYG{l+s+s1}{\PYGZsq{}}\PYG{p}{,} \PYG{l+s+s1}{\PYGZsq{}}\PYG{l+s+s1}{tim}\PYG{l+s+s1}{\PYGZsq{}}\PYG{p}{,} \PYG{l+s+s1}{\PYGZsq{}}\PYG{l+s+s1}{robert}\PYG{l+s+s1}{\PYGZsq{}}\PYG{p}{,} \PYG{l+s+s1}{\PYGZsq{}}\PYG{l+s+s1}{joseph}\PYG{l+s+s1}{\PYGZsq{}}\PYG{p}{]}
\end{sphinxVerbatim}

\end{sphinxuseclass}\end{sphinxVerbatimInput}

\end{sphinxuseclass}
\sphinxAtStartPar
\sphinxstylestrong{Solution}


\subsection{Question}
\label{\detokenize{dictionaries_code:id7}}
\sphinxAtStartPar
Construct a dictionary using a for loop, where the keys are the first names in names\_list and the values are the corresponding last names.
\begin{itemize}
\item {} 
\sphinxAtStartPar
Example: key=’Michael’, value=’Jordan’ for ‘Michael Jordan’ in fullnames\_list.

\end{itemize}

\begin{sphinxuseclass}{cell}\begin{sphinxVerbatimInput}

\begin{sphinxuseclass}{cell_input}
\begin{sphinxVerbatim}[commandchars=\\\{\}]
\PYG{n}{fullnames\PYGZus{}list} \PYG{o}{=} \PYG{p}{[}\PYG{l+s+s1}{\PYGZsq{}}\PYG{l+s+s1}{Michael Jordan}\PYG{l+s+s1}{\PYGZsq{}}\PYG{p}{,} \PYG{l+s+s1}{\PYGZsq{}}\PYG{l+s+s1}{Larry Bird}\PYG{l+s+s1}{\PYGZsq{}}\PYG{p}{,} \PYG{l+s+s1}{\PYGZsq{}}\PYG{l+s+s1}{Jason Tatum}\PYG{l+s+s1}{\PYGZsq{}}\PYG{p}{,} \PYG{l+s+s1}{\PYGZsq{}}\PYG{l+s+s1}{Lebron James}\PYG{l+s+s1}{\PYGZsq{}}\PYG{p}{,} \PYG{l+s+s1}{\PYGZsq{}}\PYG{l+s+s1}{Jimmy Butler}\PYG{l+s+s1}{\PYGZsq{}}\PYG{p}{,} \PYG{l+s+s1}{\PYGZsq{}}\PYG{l+s+s1}{Trae Young}\PYG{l+s+s1}{\PYGZsq{}}\PYG{p}{,} \PYG{l+s+s1}{\PYGZsq{}}\PYG{l+s+s1}{Kyrie Irving}\PYG{l+s+s1}{\PYGZsq{}}\PYG{p}{]}
\end{sphinxVerbatim}

\end{sphinxuseclass}\end{sphinxVerbatimInput}

\end{sphinxuseclass}
\sphinxAtStartPar
\sphinxstylestrong{Solution}


\subsection{Question}
\label{\detokenize{dictionaries_code:id8}}
\sphinxAtStartPar
Use a list comprehension to store the last two characters of each name in a list

\begin{sphinxuseclass}{cell}\begin{sphinxVerbatimInput}

\begin{sphinxuseclass}{cell_input}
\begin{sphinxVerbatim}[commandchars=\\\{\}]
\PYG{n}{grades\PYGZus{}dict} \PYG{o}{=} \PYG{p}{\PYGZob{}} \PYG{l+s+s1}{\PYGZsq{}}\PYG{l+s+s1}{mike}\PYG{l+s+s1}{\PYGZsq{}} \PYG{p}{:} \PYG{p}{\PYGZob{}}\PYG{l+s+s1}{\PYGZsq{}}\PYG{l+s+s1}{Math}\PYG{l+s+s1}{\PYGZsq{}}\PYG{p}{:}\PYG{l+m+mi}{90}\PYG{p}{,} \PYG{l+s+s1}{\PYGZsq{}}\PYG{l+s+s1}{History}\PYG{l+s+s1}{\PYGZsq{}}\PYG{p}{:}\PYG{l+m+mi}{88}\PYG{p}{,} \PYG{l+s+s1}{\PYGZsq{}}\PYG{l+s+s1}{Science}\PYG{l+s+s1}{\PYGZsq{}}\PYG{p}{:}\PYG{l+m+mi}{73}\PYG{p}{\PYGZcb{}}\PYG{p}{,}
                \PYG{l+s+s1}{\PYGZsq{}}\PYG{l+s+s1}{jack}\PYG{l+s+s1}{\PYGZsq{}} \PYG{p}{:} \PYG{p}{\PYGZob{}}\PYG{l+s+s1}{\PYGZsq{}}\PYG{l+s+s1}{Math}\PYG{l+s+s1}{\PYGZsq{}}\PYG{p}{:}\PYG{l+m+mi}{77}\PYG{p}{,} \PYG{l+s+s1}{\PYGZsq{}}\PYG{l+s+s1}{History}\PYG{l+s+s1}{\PYGZsq{}}\PYG{p}{:}\PYG{l+m+mi}{74}\PYG{p}{,} \PYG{l+s+s1}{\PYGZsq{}}\PYG{l+s+s1}{Science}\PYG{l+s+s1}{\PYGZsq{}}\PYG{p}{:}\PYG{l+m+mi}{52}\PYG{p}{\PYGZcb{}}\PYG{p}{,}
                \PYG{l+s+s1}{\PYGZsq{}}\PYG{l+s+s1}{tim}\PYG{l+s+s1}{\PYGZsq{}}  \PYG{p}{:} \PYG{p}{\PYGZob{}}\PYG{l+s+s1}{\PYGZsq{}}\PYG{l+s+s1}{Math}\PYG{l+s+s1}{\PYGZsq{}}\PYG{p}{:}\PYG{l+m+mi}{98}\PYG{p}{,} \PYG{l+s+s1}{\PYGZsq{}}\PYG{l+s+s1}{History}\PYG{l+s+s1}{\PYGZsq{}}\PYG{p}{:}\PYG{l+m+mi}{35}\PYG{p}{,} \PYG{l+s+s1}{\PYGZsq{}}\PYG{l+s+s1}{Science}\PYG{l+s+s1}{\PYGZsq{}}\PYG{p}{:}\PYG{l+m+mi}{46}\PYG{p}{\PYGZcb{}}\PYG{p}{,}
                \PYG{l+s+s1}{\PYGZsq{}}\PYG{l+s+s1}{liz}\PYG{l+s+s1}{\PYGZsq{}}  \PYG{p}{:} \PYG{p}{\PYGZob{}}\PYG{l+s+s1}{\PYGZsq{}}\PYG{l+s+s1}{Math}\PYG{l+s+s1}{\PYGZsq{}}\PYG{p}{:}\PYG{l+m+mi}{65}\PYG{p}{,} \PYG{l+s+s1}{\PYGZsq{}}\PYG{l+s+s1}{History}\PYG{l+s+s1}{\PYGZsq{}}\PYG{p}{:}\PYG{l+m+mi}{55}\PYG{p}{,} \PYG{l+s+s1}{\PYGZsq{}}\PYG{l+s+s1}{Science}\PYG{l+s+s1}{\PYGZsq{}}\PYG{p}{:}\PYG{l+m+mi}{81}\PYG{p}{\PYGZcb{}}\PYG{p}{,}
                \PYG{l+s+s1}{\PYGZsq{}}\PYG{l+s+s1}{aria}\PYG{l+s+s1}{\PYGZsq{}} \PYG{p}{:} \PYG{p}{\PYGZob{}}\PYG{l+s+s1}{\PYGZsq{}}\PYG{l+s+s1}{Math}\PYG{l+s+s1}{\PYGZsq{}}\PYG{p}{:}\PYG{l+m+mi}{76}\PYG{p}{,} \PYG{l+s+s1}{\PYGZsq{}}\PYG{l+s+s1}{History}\PYG{l+s+s1}{\PYGZsq{}}\PYG{p}{:}\PYG{l+m+mi}{99}\PYG{p}{,} \PYG{l+s+s1}{\PYGZsq{}}\PYG{l+s+s1}{Science}\PYG{l+s+s1}{\PYGZsq{}}\PYG{p}{:}\PYG{l+m+mi}{69}\PYG{p}{\PYGZcb{}}\PYG{p}{\PYGZcb{}}
\end{sphinxVerbatim}

\end{sphinxuseclass}\end{sphinxVerbatimInput}

\end{sphinxuseclass}
\sphinxAtStartPar
\sphinxstylestrong{Solution}


\subsection{Question}
\label{\detokenize{dictionaries_code:id9}}
\sphinxAtStartPar
Use a for loop to store all Math grades in a list.

\sphinxAtStartPar
\sphinxstylestrong{Solution}


\subsection{Question}
\label{\detokenize{dictionaries_code:id10}}
\sphinxAtStartPar
Use a for loop to find the name of the student with the largest Science grade and print both the name and the corresponding grade.

\sphinxAtStartPar
\sphinxstylestrong{Solution}


\subsection{Question}
\label{\detokenize{dictionaries_code:id11}}
\sphinxAtStartPar
Write a function whose parameters are two numbers.
\begin{itemize}
\item {} 
\sphinxAtStartPar
It returns a dictionary with
\begin{itemize}
\item {} 
\sphinxAtStartPar
keys : sum, difference, product, quotient

\item {} 
\sphinxAtStartPar
values: corresponding algebraic results

\end{itemize}

\item {} 
\sphinxAtStartPar
If the second number is zero, then the quotient is labeled as ‘DNE’ (Does Not Exist)

\item {} 
\sphinxAtStartPar
Example: inputs \sphinxcode{\sphinxupquote{6, 2}}
\begin{itemize}
\item {} 
\sphinxAtStartPar
Output: \sphinxcode{\sphinxupquote{\{'sum': 8, 'difference':4, 'product':12, 'quotient':3.0\}}}

\end{itemize}

\end{itemize}

\sphinxAtStartPar
\sphinxstylestrong{Solution}

\begin{sphinxuseclass}{cell}\begin{sphinxVerbatimInput}

\begin{sphinxuseclass}{cell_input}
\begin{sphinxVerbatim}[commandchars=\\\{\}]
\PYG{n+nb}{print}\PYG{p}{(}\PYG{n}{operation\PYGZus{}func}\PYG{p}{(}\PYG{l+m+mi}{6}\PYG{p}{,} \PYG{l+m+mi}{2}\PYG{p}{)}\PYG{p}{)}
\PYG{n+nb}{print}\PYG{p}{(}\PYG{n}{operation\PYGZus{}func}\PYG{p}{(}\PYG{l+m+mi}{6}\PYG{p}{,} \PYG{l+m+mi}{0}\PYG{p}{)}\PYG{p}{)}
\end{sphinxVerbatim}

\end{sphinxuseclass}\end{sphinxVerbatimInput}
\begin{sphinxVerbatimOutput}

\begin{sphinxuseclass}{cell_output}
\begin{sphinxVerbatim}[commandchars=\\\{\}]
\PYGZob{}\PYGZsq{}sum\PYGZsq{}: 8, \PYGZsq{}difference\PYGZsq{}: 4, \PYGZsq{}product\PYGZsq{}: 12, \PYGZsq{}quotient\PYGZsq{}: 3.0\PYGZcb{}
\PYGZob{}\PYGZsq{}sum\PYGZsq{}: 6, \PYGZsq{}difference\PYGZsq{}: 6, \PYGZsq{}product\PYGZsq{}: 0, \PYGZsq{}quotient\PYGZsq{}: \PYGZsq{}DNE\PYGZsq{}\PYGZcb{}
\end{sphinxVerbatim}

\end{sphinxuseclass}\end{sphinxVerbatimOutput}

\end{sphinxuseclass}






\renewcommand{\indexname}{Index}
\printindex
\end{document}